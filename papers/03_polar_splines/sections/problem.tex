\section{Preliminaries}

\subsection{Polar mappings}

In what follows we distinguish between the logical domain $\hat \Omega = \hat\Omega_s \times \hat\Omega_\theta \times \hat\Omega_\vphi$ and the "mapped domain" or "physical domain" $\Omega \subset \RR^3$. The logical domain is assumed a rectangular cuboid defined as the product of the three coordinate spaces $\hat\Omega_s, \hat\Omega_\theta,\hat\Omega_\vphi \subset \RR$. The mapped domain is the image of $\hat\Omega \subset \RR^3$ under the map $\Fb:\hat\Omega \to \RR^3$, that is $\Omega = \Fb(\hat\Omega)$. We focus on the three-dimensional case because of its practical importance. Specifically, 
\be
 \Fb(s,\theta,\vphi) = \Fb(\etab) = \begin{bmatrix}
                                 F_1(\etab)
                                 \\
                                 F_2(\etab)
                                 \\
                                 F_3(\etab)
                                \end{bmatrix} =
                                \begin{bmatrix}
                                 x \\ y \\ z
                                \end{bmatrix} = \xb \in \Omega\,,                        
\ee
where $\etab = (s,\theta,\vphi) \in  \hat\Omega$ denote the logical coordinates and $\xb = (x,y,z) \in \Omega$ are the mapped (Cartesian) coordinates. $\Fb$ is assumed to have a pole in the plane spanned by the first two logical coordinates. This means that for any $\vphi \in \hat\Omega_\vphi$, the logical edge $s=0$ collapses to a single point $(x_0(\vphi), y_0(\vphi), z_0(\vphi)) $ in the physical domain. The planar pole is thus located along the line $\vphi \mapsto (x_0(\vphi), y_0(\vphi), z_0(\vphi))$, characterized by
\be
 \Fb(0,\theta,\vphi) =     \begin{bmatrix}
                        x_0(\vphi) \\[1mm] y_0(\vphi) \\[1mm] z_0(\vphi)
                       \end{bmatrix} \qquad \forall \,\theta.
\ee
This means in particular that
\be
 \pder{^n\Fb}{\theta^n} \Big |_{s=0} = \begin{bmatrix}
                        0 \\ 0 \\ 0
                       \end{bmatrix} \qquad \tn{for}\quad n\geq 1 \,.
\ee
The Jacobian matrix at the pole reads
\be \label{DF0}
 D\Fb \big|_{s=0} = \begin{bmatrix}
         F_{1,s} \big|_{s=0} & 0 & x_0'
         \\[1mm]
         F_{2,s} \big|_{s=0} & 0 & y_0'
         \\[1mm]
         F_{3,s} \big|_{s=0} & 0 & z_0'
        \end{bmatrix}\,,
\ee
which shows the singularity at $s = 0$. In practical applications the mapping $\Fb$ can be quite complicated, sometimes given as the result of a computer simulation. W.l.o.g. we set $\hat\Omega_s = [0,1]$, $\hat\Omega_\theta = [0,2\pi), \hat\Omega_\vphi=[0,L)$ and decompose $\Fb = \Pb \circ \Gb$, where $\Gb: \hat\Omega \to \Omega_\Pb$ is a simple cylindrical mapping given by
\be \label{Gmap}
 \Gb(\etab) = 
 \begin{bmatrix} f(s) \cos\theta \\[1mm] f(s)\sin\theta \\[1mm] \vphi \end{bmatrix}
  = \begin{bmatrix} X \\[1mm] Y \\[1mm] Z \end{bmatrix} = \Xb \in \Omega_\Pb\,,
\ee
In \cite{YamanEdo2019} the coordinates $\Xb$ were termed \emph{pseudo-Cartesian} coordinates, a notion we shall adopt in this work. The function $f$ is such that
\be
 f: [0,1] \to \RR\,,\qquad f(0) = 0\,,\qquad 0 < f'(s) < \infty \quad \tn{for}\ s > 0 \,.
\ee
Moreover, $f$ should be chosen such that $\Pb = \Fb \circ \Gb^{-1}: \Omega_\Pb \to \Omega$ is continuously differentiable and invertible on $\Omega_\Pb$. In that case the problem of the singularity at the pole $s=0$ occurs entirely in the cylindrical mapping $\Gb$. The relation between the three coordinates spaces $\hat\Omega \to \Omega_\Pb \to \Omega$ is illustrated in Figure~\ref{fig:map}.
The Jacobian $D\Gb$ and its inverse are given by
\be \label{DG}
 D\Gb  =
 \begin{bmatrix}
  f'\cos\theta & -f\sin\theta & 0
  \\
  f'\sin\theta & f \cos\theta & 0
  \\
  0 & 0 & 1 
 \end{bmatrix}\,, \qquad
 (D\Gb)^{-1} = \begin{bmatrix}
  \frac{1}{f'}\cos\theta & \frac{1}{f'}\sin\theta & 0
  \\
  - \frac{1}{f}\sin\theta & \frac{1}{f}\cos\theta & 0
  \\
  0 & 0 & 1
 \end{bmatrix} \,.
\ee
The Jacobian is $J_\Gb = \det D \Gb = ff'$. Choosing $f(s) = s$ for example leads to the standard cylindrical coordinates with $J_\Gb(0) = 0$, whereas choosing $f(s) = \sqrt s$ leads to $J_\Gb(0) = 1/2$. Even though $\Gb$ is singular at $s=0$, an inverse of $\Gb$ can be created via
\be \label{Gmap:inv}
 \Gb^{-1}(\Xb) = 
 \begin{bmatrix} f^{-1}(\sqrt{X^2+Y^2}) \\[1mm] \tn{atan}2(Y,X) \\[1mm] Z \end{bmatrix}
  = \begin{bmatrix} s \\[1mm] \theta \\[1mm] \vphi \end{bmatrix} = \etab \in \hat\Omega\,,
\ee
where $\tn{atan}2$ returns the phase angle of the complex number $X + iY$ in the range $(-\pi,\pi]$, which we then shift to $[0,2\pi)$. This assigns a unique value to the angular variable $\theta$ at the pole $ (X=0, Y=0)$.



% \begin{figure}[htb]
% \includegraphics[width=\textwidth]{pics/mapping2.png}
% \caption{Cylindrical coordinates.} \label{fig:map}
% \end{figure}


\subsection{Particle equations of motion}


In the physical domain $\Omega$, the trajectory $\xb(t)$ of a charged particle is obtained from the equations of motion
\be
\left\{
\begin{aligned}
 \dt{\xb} &= \vb\,,
 \\[1mm]
 \dt{\vb} &= \Eb(\xb) + \vb \times \Bb(\xb)\,,
\end{aligned}
\right.
\ee
where $\vb(t) \in \RR^3$ stands for the particle's velocity and $\Eb,\Bb: \Omega\to \RR^3$ are the electromagnetic fields. With respect to the pseudo-Cartesian position coordinate $\xb = \Pb(\Xb)$, these equations read
\be \label{eqmos:X}
\left\{
\begin{aligned}
 \dt{\Xb} &=  (D\Pb)^{-1}(\Xb)\,\vb\,,
 \\[1mm]
 \dt{\vb} &= \wt \Eb(\Xb) + \vb \times \wt \Bb(\Xb)\,,
\end{aligned}
\right.
\ee
where $\wt \Eb = \Eb \circ \Pb$ and $\wt \Bb = \Bb \circ \Pb$. Remark that these equations are well-defined for all $\Xb\in\Omega_\Pb$ because $\Pb \in C^1(\Omega_\Pb)$. From our finite element solver we compute however not $\wt \Eb$ and $\wt \Bb$ on $\Omega_\Pb$ but rather the 1-form $\hat \Eb^1$ resp. the 2-form $\hat \Bb^2$ on in the logical domain $\hat\Omega$. The relation between these representations is given by
\be \label{star_to_form}
 \wt \Eb(\Xb) = [(D\Gb^{-\top} \hat \Eb^1) \circ \Gb^{-1} ](\Xb)\,,\qquad \wt \Bb = \left[ \left(\frac{1}{J_\Gb} D\Gb\, \hat \Bb^2 \right) \circ \Gb^{-1} \right](\Xb)\,.
\ee
Particle pushing in the polar GEMPIC scheme is achieved by solving the equations \eqref{eqmos:X}-\eqref{star_to_form}. The difficulty arises in the coupling to the $p$-forms via \eqref{star_to_form}, which involves singular metric coefficients. The functions
\be
 \wt \Eb = D\Gb^{-\top} \hat \Eb^1 \,,\qquad \wt \Bb = \frac{1}{J_\Gb} D\Gb\, \hat \Bb^2\,,
\ee
are bounded on $\Omega_\Pb$; by substituting \eqref{DG} this means in terms of the components 
\begin{subequations}
\begin{align}
 \begin{bmatrix}
  \wt E_1 
  \\[1mm]
  \wt E_2
  \\[1mm]
  \wt E_3
  \end{bmatrix} =
  \begin{bmatrix}
   \hat E^1_1 \frac{1}{f'} \cos \theta  - \hat E^1_2 \frac{1}{f} \sin \theta 
   \\[1mm]
   \hat E^1_1 \frac{1}{f'} \sin \theta  + \hat E^1_2 \frac{1}{f} \cos \theta 
   \\[1mm]
   \hat E^1_3
  \end{bmatrix} \,,
  \\[2mm]
   \begin{bmatrix}
  \wt B_1 
  \\[1mm]
  \wt B_2
  \\[1mm]
  \wt B_3
  \end{bmatrix} =
  \begin{bmatrix}
   \hat B^2_1 \frac{1}{f} \cos \theta  - \hat B^2_2 \frac{1}{f'} \sin \theta 
   \\[1mm]
   \hat B^2_1 \frac{1}{f} \sin \theta  + \hat B^2_2 \frac{1}{f'} \cos \theta 
   \\[1mm]
   \frac{1}{ff'} \hat B^2_3
  \end{bmatrix} \,,
\end{align}
\end{subequations}
which implies
\begin{subequations}
\begin{alignat}{2}
 \hat E^1_1 &= O(f')\,,\qquad \hat E^1_2 = O(f)\,,\qquad \hat E^1_3 = O(1)\,,\qquad  &\tn {as} \quad s \to 0\,,
 \\[2mm]
 \hat B^2_1 &= O(f)\,,\qquad \hat B^2_2 = O(f')\,,\qquad \hat B^2_3 = O(ff')\qquad &\tn{as} \quad s \to 0\,.  \label{conds:B} 
\end{alignat}
\end{subequations}
When discretizing the 1-form $\hat \Eb^1$ and the 2-form $\hat \Bb^2$ on the logical domain these properties must be guaranteed by the FE basis functions. We shall see that a mere tensor-product basis violates these conditions, whereas polar splines can be constructed to satisfy them.



\subsection{Extraction operator} 

We aim to create the polar de Rham complex starting from the tensor-product basis in $V_0$, created from univariate B-splines $\hat N^p_i$. Hence we start from 
\be
V_0 = \tn{span}\big[\, \hat N_i^{p_1}(s)\hat N_j^{p_2}(\theta) \hat N_k^{p_3}(\vphi)\, \big]_{i=0,j=0,k=0}^{n_s-1,n_\theta-1,n_\vphi-1}\,,
\ee
where the B-splines $\hat N_i^{p_1}: \hat\Omega_s \to \RR$ are clamped and the B-splines in $\theta$ and $\vphi$ are periodic. The total number of basis functions is thus $n_{tot} = n_s n_\theta n_\vphi$. Let us moreover introduce the $M$-splines
\be
 \hat D_i^{p} := \frac{p}{\eta_{i+p} - \eta_i} \hat N_i^{p-1}\,,
\ee
where $(\eta_i)_i$ is the knot vector defining the B-splines $\hat N_i^p$. We have for the derivative of B-splines the formula
\be
 (\hat N_i^p)' = \hat D_{i-1}^p - \hat D_i^p\,.
\ee
For clamped B-splines we have $\hat D_{-1}^p = 0$ and, due to partition of unity in $\hat\Omega_s = [0,1]$, at the left boundary we have
\begin{gather}
 \hat N_0^{p_1}(0) = 1\,,\qquad \hat N_i^{p_1}(0) = 0 \quad \tn{for} \quad i\geq 1\,, \label{leftborder}
 \\[3mm]
 (\hat N_0^{p_1})'(0) = - \hat D_0^{p_1}(0) = - (\hat N_1^{p_1})'(0)\,,\qquad (\hat N_i^{p_1})'(0) = 0 \quad \tn{for} \quad i \geq 2\,. \label{only2}
\end{gather}
From \eqref{only2} we see that only the B-splines $i=0$ and $i=1$ have non-vanishing derivative at the left boundary. We thus take only these B-splines to form $n$ new polar basis functions $\hat\chi_n(s,\theta,\vphi)$ via linear combinations of $2 n_\theta n_\vphi$ tensor-product basis functions:
\be \label{Efull}
 \hat \chi_n(s,\theta,\vphi) = \sum_{j=0}^{n_\theta-1} \sum_{k=0}^{n_\vphi-1} \left[ E_{n,0,j,k}\,\hat N_0^{p_1}(s) + E_{n,1,j,k}\,\hat N_1^{p_1}(s) \right] \hat N_j^{p_2}(\theta) \hat N_k^{p_3}(\vphi)\,.
\ee
Introducing the index $m= k + j n_\vphi + i n_\theta n_\vphi$, we write this as
\be \label{Enm}
 \hat \chi_n(s,\theta,\vphi) = \sum_{m=0}^{2n_\theta n_\vphi-1} E_{n,m} \hat N_m(s,\theta,\vphi)\,,
\ee
with $\hat N_m(s,\theta,\vphi) = \hat N_i^{p_1}(s) \hat N_j^{p_2}(\theta) \hat N_k^{p_3}(\vphi)$ appropriately. The other tensor-product basis with $m\geq 2n_\theta n_\vphi$ remain unchanged and are taken as basis functions of the polar spline space. The $ n^* = n_{tot} - 2 n_\theta n_\vphi + 2$ new polar basis functions will be denoted by $\hat \Qb = (\hat Q_l)_{l=0}^{n^*-1}$. We have
\be
 \hat \Qb = \EE_{tot}\, \hat \Nb\,,
\ee
where $\hat \Nb = (\hat N_m)_{m=0}^{n_{tot} -1}$ and $\EE_{tot} \in \RR^{n^* \times n_{tot}}$ is the \emph{extraction operator} composed of the following blocks:
\be
 \EE_{tot} = \begin{bmatrix}
        \EE & \mathbf 0
        \\[1mm]
        \mathbf 0 & \mathbb I_{n^*-2}
       \end{bmatrix}\,.
\ee
Here, $\EE \in \RR^{2\times 2 n_s n_\theta}$ and $\mathbb I_{n^*-2} \in \RR^{(n^*-2) \times (n^*-2)}$ is the identity matrix. The entries of $\EE$ are the $E_{n,m}$ from \eqref{Enm} or \eqref{Efull}, respectively. The new polar spline space is thus 
\be
 V_0^* := \tn{span}(\hat \Qb)\,.
\ee
It remains to construct the other polar spaces $V^*_1$, $V^*_2$ and $V^*_3$ such that the exact sequence property is preserved:
\be
 Im(\grad) = Ker(\curl)\,,\qquad Im(\curl) = Ker(\div)\,.
\ee
Moreover, we demand the following from the polar basis $\hat \Qb$:
\begin{enumerate}
 \item $\hat \Qb \in C^0(\hat\Omega)$, including the pole $s = 0$,
 \item The mixed derivative w.r.t $(s,\theta)$ must vanish at the pole:
 \be
  \frac{\partial^2 \hat\Qb}{\partial s \partial \theta}\bigg|_{s=0} = 0\,.
 \ee
 This will be encessary to fulfill the third condition in \eqref{conds:B}.
 \item The push-forward $\hat\Omega \to \Omega^*$ of $\hat\nabla \hat\Qb$ must be single-valued at the pole:
 \be
  \nabla \Qb \big|_{s=0} := D\Gb^{-\top} \hat \nabla \hat \Qb \big|_{s=0} = a \in \RR \qquad \forall\, \theta\,.
 \ee
 \item Partition of unity: 
 \be \label{partition}
 \sum_n E_{n,m} = 1\,.
 \ee
\end{enumerate}
Let us address these points one-by-one.

\subsection{Continuity at the pole}

We need only to consider the new basis functions $\hat\chi_n$, as the others remain the usual tensor product basis which are zero at the pole. Setting $s = 0$ in \eqref{Efull} and substituting \eqref{leftborder} leads to 
\be 
 \hat \chi_n(0,\theta,\vphi) = \sum_{j=0}^{n_\theta-1} \sum_{k=0}^{n_\vphi-1}  E_{n,0,j,k}\, \hat N_j^{p_2}(\theta) \hat N_k^{p_3}(\vphi)\,.
\ee
Form the partition of unity of $N_j^{p_2}$ and $N_k^{p_3}$ we conclude that setting \be \label{E0}
E_{n,0,j,k} = 1/n\,,
\ee
leads to $\hat \chi_n(0,\theta,\vphi) = 1/n$ for all $(\theta,\vphi)$ and moreover respects the partition of unity property \eqref{partition}. The new basis functions are thus written as
\be \label{Efull:1}
 \hat \chi_n(s,\theta,\vphi) = \frac 1n \,\hat N_0^{p_1}(s) + \sum_{j=0}^{n_\theta-1} \sum_{k=0}^{n_\vphi-1} E_{n,1,j,k}\,\hat N_1^{p_1}(s) \hat N_j^{p_2}(\theta) \hat N_k^{p_3}(\vphi)\,.
\ee

\subsection{Mixed derivative}

Let us take the gradient of \eqref{Efull:1}:
\begin{align}
 \pder{\hat\chi_n}{s}(s,\theta,\vphi) &= -\frac 1n \,\hat \hat D^{p_1}_0(s) + \sum_{j=0}^{n_\theta-1} \sum_{k=0}^{n_\vphi-1} E_{n,1,j,k}\,\big[\hat D_0^{p_1}(s) - \hat D_1^{p_1}(s) \big] \hat N_j^{p_2}(\theta) \hat N_k^{p_3}(\vphi)\,,
 \\[1mm]
 \pder{\hat\chi_n}{\theta}(s,\theta,\vphi) &= \sum_{j=0}^{n_\theta-1} \sum_{k=0}^{n_\vphi-1} E_{n,1,j,k}\,\hat N_1^{p_1}(s) \big[\hat D_{j-1}^{p_2}(\theta) - \hat D_j^{p_2}(\theta) \big]  \hat N_k^{p_3}(\vphi)\,,
 \\[1mm]
 \pder{\hat\chi_n}{\vphi}(s,\theta,\vphi) &= \sum_{j=0}^{n_\theta-1} \sum_{k=0}^{n_\vphi-1} E_{n,1,j,k}\,\hat N_1^{p_1}(s) \hat N_j^{p_2}(\theta) \big[\hat D_{k-1}^{p_3}(\vphi) - \hat D_k^{p_3}(\vphi) \big] \,.
\end{align}
For the mixed derivative in $(s,\theta)$ we obtain
\be
 \frac{\partial^2 \hat\chi_n}{\pa s \pa \theta}(s,\theta,\vphi) = \sum_{j=0}^{n_\theta-1} \sum_{k=0}^{n_\vphi-1} E_{n,1,j,k}\,\big[\hat D_0^{p_1}(s) - \hat D_1^{p_1}(s) \big]  \big[\hat D_{j-1}^{p_2}(\theta) - \hat D_j^{p_2}(\theta) \big]  \hat N_k^{p_3}(\vphi)\,.
\ee
At the pole $s=0$ this becomes
\be
 \frac{\partial^2 \hat\chi_n}{\pa s \pa \theta}(0,\theta,\vphi) = \hat D_0^{p_1}(0) \sum_{j=0}^{n_\theta-1} \sum_{k=0}^{n_\vphi-1} E_{n,1,j,k}\,  \big[\hat D_{j-1}^{p_2}(\theta) - \hat D_j^{p_2}(\theta) \big]  \hat N_k^{p_3}(\vphi)\,.
\ee
This is zero if and only if $E_{n,1,j,k}$ does not depend on $(j,k)$ and can be taken out of the sum. Hence, we set
\be
 E_{n,1,j,k} = \frac 1n \,,
\ee
which also respects the partition of unity property \eqref{partition}. The new basis functions now read
\be \label{Efull:2}
 \hat \chi_n(s,\theta,\vphi) = \frac 1n \,\hat N_0^{p_1}(s) + \frac 1n \,\hat N_1^{p_1}(s)\,.
\ee
Since this is independent of $n$ up to a constant we may set $n=1$. This means we only need one new polar basis functions $\hat\chi$, given by \eqref{Efull:2}. The $s$-derivative of this basis function is given by
\be
 \pder{\hat \chi}{s}(s,\theta,\vphi) = - D^{p_1}_1(s)\,.
\ee
This means that at the pole,
\be
 \pder{\hat \chi}{s}(0,\theta,\vphi) = 0 \qquad \forall \,\theta,\vphi\,.
\ee


\subsection{Push-forward}


%%%%%%%%%%%%%%%%%%%
\subsection{Hilbert spaces of differential forms}

\begin{figure}[htb]
\includegraphics[width=\textwidth]{pics/deRham3D.png}
\caption{Commuting diagram for the logical domain $\hat \Omega$.} \label{fig:diag}
\end{figure}
Conforming FE methods in three dimensions can be built upon the commuting diagram depicted in Figure \ref{fig:diag}. All spaces in this diagram refer to functions on the logical domain $\hat\Omega$. The upper line contains the continuous spaces well-known in FE analysis. In the framework of FEEC, these spaces refer to the components of differentiable $n$-forms, with $0\leq n \leq 3$. We use the symbol
\begin{alignat}{2}
 H^1(\hat\Omega) &= \left\{ a :\hat\Omega \to \RR\ s.t.\ |a|_0 + |\grad\, a|_1 < \infty \right\} \qquad &&\tn{(0-forms)}\,,
 \\[1mm]
 H(\curl,\hat\Omega) &= \left\{ \ab :\hat\Omega \to \RR^3\ s.t.\ |\ab|_1 + |\curl\, \ab|_2 < \infty \right\}\qquad &&\tn{(1-forms)}\,,
 \\[1mm]
 H(\div,\hat\Omega) &= \left\{ \ab :\hat\Omega \to \RR^3\ s.t.\ |\ab|_2 + |\div\, \ab|_3 < \infty \right\}\qquad &&\tn{(2-forms)}\,,
 \\[1mm]
 L^2(\hat\Omega) &= \left\{ a :\hat\Omega \to \RR\ s.t.\ |a|_3 < \infty \right\}\qquad &&\tn{(3-forms)}\,,
\end{alignat}
where the seminorms $|\cdot|_{0\leq n\leq 3}$ are given by
\begin{align}
 |a|_0^2 &:= \int_{\hat\Omega} a^2\,\sqrt g\,\tn d\etab\,,
 \\[1mm]
 |\ab|_1^2 &:= \int_{\hat\Omega} \ab\, G^{-1} \ab\,\sqrt g\,\tn d\etab\,,
 \\[1mm]
 |\ab|_2^2 &:= \int_{\hat\Omega} \ab\, G\, \ab\,\frac{1}{\sqrt g}\,\tn d\etab\,,
 \\[1mm]
 |a|_3^2 &:= \int_{\hat\Omega} a^2\,\frac{1}{\sqrt g}\,\tn d\etab\,.
\end{align}
Denoting $\hat\nabla=(\pa_\eta,\pa_\theta,\pa_{z'})$ in logical coordinates, the differential operators can be written as
\be
 \grad = \hat \nabla\,,\qquad \curl = (\hat\nabla \times)\,,\qquad \div = (\hat\nabla \cdot)\,.
\ee
% In cylindrical coordinates, the above Hilbert spaces are defined as follows:
% \begin{align}
%  a \in H^1(\hat\Omega): &\ \int_{\hat\Omega} a^2\,ff'\,\tn d \etab + \int_{\hat\Omega} \left[ (\pa_\eta a)^2 \frac{f}{f'} + (\pa_\theta a)^2 \frac{f'}{f} + (\pa_{z'} a)^2 \,ff' \right]\tn d \etab < \infty\,,
%  \\[3mm]
%  \ab \in H(\curl,\hat\Omega) : &\ \int_{\hat\Omega} \left[ a_\eta^2 \frac{f}{f'} + a_\theta^2 \frac{f'}{f} + a_{z'}^2 \,ff' \right]\tn d \etab 
%  \\[1mm]
%   +&\ \int_{\hat\Omega} \left[ (\pa_\theta a_{z'} - \pa_{z'} a_\theta)^2 \frac{f'}{f} + (\pa_{z'} a_\eta - \pa_\eta a_{z'})^2 \frac{f}{f'} + (\pa_\eta a_\theta - \pa_\theta a_\eta)^2 \,\frac{1}{ff'}\right]\tn d \etab < \infty\,,  \nonumber
%  \\[3mm]
%  \ab \in H(\div,\hat\Omega) : &\ \int_{\hat\Omega} \left[ a_\eta^2 \frac{f'}{f} + a_\theta^2 \frac{f}{f'} + a_{z'}^2 \,\frac{1}{ff'} \right]\tn d \etab 
%  \\[1mm]
%  +&\ \int_{\hat\Omega} \left[ (\pa_\eta a_\eta)^2 + (\pa_\theta a_\theta)^2  + (\pa_{z'}a_{z'})^2 \right]\frac{1}{ff'}\, \tn d \etab < \infty\,, \nonumber
%  \\[3mm]
%  a \in L^2(\hat\Omega): &\ \int_{\hat\Omega} a^2\,\frac{1}{ff'}\,\tn d \etab< \infty\,.
% \end{align}
% For $f(\eta) = \eta^q$ with $q>0$ we have $f/f' = \eta/q$ and $ff' = q \eta^{2q-1}$. Then $q=1/2$ yields $f/f' = 2\eta$ and $ff' = 1/2$ such that integrals featuring the factor $f'/f$ must be handled with care on the discrete level. The Hilbert spaces form an exact sequence, meaning that
% \be \label{sequence}
%  \grad\,H^1 = \ker(\curl\,H(\curl))\,,\qquad \curl\,H(\curl) = \ker(\div\, H(\div))\,.
% \ee

The operators $\Pi_j$, $0\leq j\leq 3$ project onto the finite-dimensional subspaces $V_j$, $0\leq j \leq 3$, which will be spanned by tensor product basis functions, constructed from univariate B-splines of degree $p$, denoted by $\hat N^p_i(\eta)$, $0\leq i\leq \hat n_N-1$.  The sequence of $\hat n_N$ splines $(\hat N^p_i)_i$ is constructed from the knot vector $\cT_p = \{\eta_i\}_{0\leq i\leq n+2p}$, composed of $n+2p+1$ non-decreasing points $\eta_i$ in a logical interval $\hat I\subset \RR$. Here, $n$ is the number of cells partitioning the interval $\hat I$ to define the 1D space grid. Each spline $\hat N^p_i$ is defined by $p+2$ neighbouring knots, such that we can fit $n+p$ spline functions into the knot vector $\cT_p$. The ensuing spline basis $(\hat N^p_i)_i$ can be either periodic or "clamped". In the periodic case we relate the first $p$ and the last $p$ splines to obtain $\hat n_N = n$ basis functions. In the clamped case we have $\hat n_N = n+p$ basis functions. Moreover, for clamped splines $\hat N^p_0(\eta_0) = \hat N^p_{\hat n_N-1}(\eta_{n+2p}) = 1$, where $\eta_0$ is the left and $\eta_{n+2p}$ is the right boundary of $\hat I$.  Because of partition of unity we have
\be \label{Nto0}
 \tn{clamped:}\qquad \hat N^p_i(\eta_0) = \hat N^p_i(\eta_{n+2p}) = 0\,,\qquad 1\leq i\leq \hat n_N -2\,.
\ee
The derivative of $\hat N^p_i(\eta)$ can be written as
\be \label{N'}
 ({\hat N_i^p})'(\eta) = \hat D_{i-1}^{p-1}(\eta)-\hat D_{i}^{p-1}(\eta)\,,
\ee
where we introduced the "D-splines" of degree $p-1$ as
\be \label{def:D}
\begin{aligned}
 \hat D_{i}^{p-1}(\eta) &= \frac{p}{\eta_{i+p+1}-\eta_{i+1}} \hat N_{i+1}^{p-1}(\eta)\,,\qquad -1\leq i \leq \hat n_N-1\,,
% \\[1mm]
% D_{\hat n_N-1}^{p-1}(\eta) &= \begin{cases}
% D_{0}^{p-1}(\eta) & \tn{for periodic}
% \\
% 0 & \tn{for clamped}
% \end{cases}
% \,.
 \end{aligned}
\ee
It is convenient to view D-splines as usual B-splines of degree $p-1$ created from the same knot vector $\cT_p$ as the $\hat N^p_i$, and multiplied by the factor $p/(\eta_{i+p+1}-\eta_{i+1})$. We can fit $n+p+1$ basis splines of degree $p-1$ into the knot vector $\cT_p$. In the periodic case we relate the first $p+1$ D-splines with the last $p+1$ D-splines. In the clamped case we have $\hat D_{-1}^{p-1}(\eta) = \hat D_{\hat n_N-1}^{p-1}(\eta) = 0$. Thus, we finally end up with the D-spline sequence $(\hat D^{p-1}_i)_i$, $0\leq i \leq \hat n_D-1$, where $\hat n_D = \hat n_N$ for periodic and $\hat n_D = \hat n_N-1$ for clamped splines.






%%%%%%%%%%%%%%%%%%%
\subsection{Construction of polar basis functions}

We start from the tensor product space $V_0$ defined by
\be
 V_0 = \tn{span} (\hat \Lambda^0_i)\,,\qquad \hat \Lambda^0_i = \hat N^{p_1}_{i_1}(\eta)\,\hat N^{p_2}_{i_2}(\theta)\,\hat N^{p_3}_{i_3}(z')\,, \quad i = i_1(\hat n_{N}^2 \hat n_{N}^3) + i_2\,\hat n_{N}^3 + i_3\,,
\ee
for $0\leq i_j \leq \hat n_{N}^j-1$, $j=1,2,3$. We assume $\hat N^{p_1}_{i_1}(\eta)$ to be clamped splines, whereas the other two directions are periodic.
%Moreover, we have to remove the interpolator spline $N^{p_1}_0$ from the basis, imposing homogeneous Dirichlet conditions at $\eta=0$, in order to assure $\Lambda^0_i$ to be single-valued a the pole:
%\be
%\lim_{\eta\to 0} \Lambda^0_i = \lim_{\eta\to 0} N^{p_1}_{i_1}(\eta)\,N^{p_2}_{i_2}(\theta)\,N^{p_3}_{i_3}(z') = 0 \quad \forall \ (\theta,z')\,,\qquad 1 \leq i_1 \leq \hat n_N^1\,.
%\ee
%This holds because of \eqref{Nto0}.
 In order to maintain the exact sequence property \eqref{sequence} we construct the other spaces $V_{1\leq j \leq 3}$ as follows:
\begin{align}
 V_1 &:= \tn{span}\left(
 \begin{pmatrix}
 \pa_\eta \hat\Lambda^0_i \\ 0 \\ 0
 \end{pmatrix},
  \begin{pmatrix}
 0 \\ \pa_\theta \hat\Lambda^0_i \\ 0
 \end{pmatrix},
  \begin{pmatrix}
 0 \\ 0 \\ \pa_{z'} \hat\Lambda^0_i 
 \end{pmatrix}
  \right)\,,  \label{spanV1}
  \\[3mm]
  V_2 &:= \tn{span}\left(
 \begin{pmatrix}
 \pa_\theta\pa_{z'} \hat\Lambda^0_i \\ 0 \\ 0
 \end{pmatrix},
  \begin{pmatrix}
 0 \\ \pa_\eta\pa_{z'} \hat\Lambda^0_i \\ 0
 \end{pmatrix},
  \begin{pmatrix}
 0 \\ 0 \\ \pa_\eta\pa_\theta \hat\Lambda^0_i 
 \end{pmatrix}
  \right)\,,  \label{spanV2}
  \\[3mm]
  V_3 &:= \tn{span}(\pa_\eta\pa_\theta\pa_{z'} \hat\Lambda^0_i)\,.
\end{align}
We shall hold on to this construction even when the basis $\hat\Lambda^0$ is not a tensor product basis anymore. One problem of the tensor product basis in the case of cylindrical coordinates is immediately obvious, namely that $\hat\Lambda^0_i$ is not single-valued as $\eta\to 0$, hence at the pole. This means: 

\begin{itemize}
\item Tensor product $V_0$-basis functions $\Lambda^0_i(\xb)$ are not $C^0$ at the pole in the physical domain. 
\item If we construct $\Lambda^0_i(\xb)$ to be $C^0$ somehow, $V_1$-basis functions are not single-valued at the pole.
\item If we construct $\Lambda^0_i(\xb)$ to be $C^1$ somehow, the third $V_2$-basis functions and the $V_3$-basis functions (mixed derivatives $\pa_\eta\pa_\theta$) are not single-valued at the pole.
\item Our goal is thus as follows: {\bf $\Lambda^0_i(\xb)$ must be $C^1$ at the pole and $\pa_\eta\pa_\theta \hat\Lambda^0_i$ must be single-valued at the pole. We also want an IGA-compatible basis.}
\end{itemize}

