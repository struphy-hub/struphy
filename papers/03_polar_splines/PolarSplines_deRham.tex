\documentclass[a4paper,12pt]{article}
%\documentclass[12pt,reqno]{amsart}

\usepackage[utf8]{inputenc}
\usepackage{a4wide}
\usepackage{amssymb,amsmath,amsthm,newlfont,enumerate}
\usepackage{appendix}
\usepackage{dsfont}
\usepackage{amsfonts}
\usepackage{amssymb}
\usepackage{amsbsy}
\usepackage[dvips]{graphicx}
%\usepackage{epstopdf}
\usepackage{psfrag}
%\usepackage[hang,center]{caption}
\usepackage{verbatim} 
\usepackage{float}
%\usepackage{cite}
\usepackage{upgreek}
\usepackage[colorlinks=true,linkcolor=blue,urlcolor=blue]{hyperref}

% \hypersetup{
%     bookmarks=true,         % show bookmarks bar?
%     unicode=false,          % non-Latin characters in Acrobat’s bookmarks
%     pdftoolbar=true,        % show Acrobat’s toolbar?
%     pdfmenubar=true,        % show Acrobat’s menu?
%     pdffitwindow=false,     % window fit to page when opened
% %    pdfstartview={FitW},    % fits the width of the page to the window
%     pdftitle={Certificate},    % title
%     pdfauthor={Dr. Harish Kumar},     % author
%     pdfsubject={TEQIP certificates},   % subject of the document
%     pdfcreator={Dr. Harish Kumar},   % creator of the document
%     pdfproducer={},  % producer of the document
%     pdfkeywords={Certificates,} {TEQIP} {Participation}, % list of keywords
%     pdfnewwindow=true,      % links in new window
%     colorlinks=false,       % false: boxed links; true: colored links
%     linkcolor=red,          % color of internal links
%     citecolor=green,        % color of links to bibliography
%     filecolor=magenta,      % color of file links
%     urlcolor=cyan           % color of external links
% }

\pdfinfo{ /Creator ()  /Producer () /ModDate ()  /CreationDate () }

%\usepackage{hyperref}
\usepackage{color}
\usepackage[normalem]{ulem}
\usepackage{fancyhdr}
\usepackage{bm}
\usepackage{xcolor}
\usepackage{empheq}
\usepackage{placeins}
%\usepackage[all,cmtip]{xy}
%\usepackage[]{showkeys}
\usepackage{nccmath}
%\usepackage{mathtools}
%\mathtoolsset{showonlyrefs}
\usepackage{cases}
\usepackage{authblk}
\usepackage{tikz-cd}
\usepackage{pdflscape}
\usepackage{bbold}
\usepackage{natbib}

\def\NN{\mathbb{N}}
\def\QQ{\mathbb{Q}}
\def\RR{\mathbb{R}}
\def\CC{\mathbb{C}}
\def\FF{\mathbb F}
\def\PP{\mathbb P}
\def\EE{\mathbb E}
\def\MM{{\mathbb M}}
\def\DD{{\mathbb D}}
\def\GG{{\mathbb G}}
\def\SS{{\mathbb S}}
\def\TT{{\mathbb T}}
\def\JJ{{\mathbb J}}
\def\WW{{\mathbb W}}

\def\cA{\mathcal A}
\def\cB{\mathcal B}
\def\cC{\mathcal C}
\def\cD{\mathcal D}
\def\cE{\mathcal E}
\def\cF{\mathcal F}
\def\cG{\mathcal G}
\def\cH{\mathcal H}
\def\cO{\mathcal O}
\def\cP{\mathcal P}
\def\cR{\mathcal R}
\def\cQ{\mathcal Q}
\def\cN{\mathcal N}
\def\cX{\mathcal X}
\def\cY{\mathcal Y}
\def\cz{\mathcal z}
\def\cT{\mathcal T}
\def\cM{\mathcal M}
\def\cS{\mathcal S}
\def\cV{\mathcal V}
\def\cL{\mathcal L}
\def\cZ{\mathcal Z}
\def\cK{\mathcal K}

\def\bfu{\mathbf{u}}
\def\bfh{\mathbf{h}}
\def\bfv{\mathbf{v}}
\def\bfx{\mathbf{x}}
\def\bfk{\mathbf{k}}
\def\bfq{\mathbf{q}}
\def\bfE{\mathbf{E}}
\def\bfF{\mathbf{F}}
\def\bfB{\mathbf{B}}
\def\bfb{\mathbf{b}}
\def\bfe{\mathbf{e}}
\def\bfW{\mathbf{W}}
\def\bfA{\mathbf{A}}
\def\bfG{\mathbf{\Gamma}}
\def\bfU{\mathbf{U}}
\def\bfom{\bm\omega}
\def\bfrho{\bm\rho}
\def\bfs{\mathbf{s}}
\def\bfX{\mathbf{X}}
\def\bfJ{\mathbf{J}}
\def\bfa{\mathbf{a}}
\def\bfc{\mathbf{c}}
\def\bfR{\mathbf{R}}
\def\bfkap{\bm\kappa}
\def\bfz{\mathbf{z}}
\def\bfz{\mathbf{z}}
\def\bfX{\mathbf{X}}
\def\bfV{\mathbf{V}}

\newcommand{\xb}{\mathbf{x}}
\newcommand{\rb}{\mathbf{r}}
\newcommand{\vb}{\mathbf{v}}
\newcommand{\zb}{\mathbf{z}}
\newcommand{\ub}{\mathbf{u}}
\newcommand{\eb}{\mathbf{e}}
\newcommand{\jb}{\mathbf{j}}
\newcommand{\ib}{\mathbf{i}}
\newcommand{\Eb}{\mathbf{E}}
\newcommand{\Nb}{\mathbf{N}}
\newcommand{\Bb}{\mathbf{B}}
\newcommand{\ab}{\mathbf{a}}
\newcommand{\bb}{\mathbf{b}}
\newcommand{\cb}{\mathbf{c}}
\newcommand{\Ab}{\mathbf{A}}
\newcommand{\db}{\mathbf{d}}
\newcommand{\kb}{\mathbf{k}}
\newcommand{\Ub}{\mathbf{U}}
\newcommand{\Cb}{\mathbf{C}}
\newcommand{\Db}{\mathbf{D}}
\newcommand{\Ib}{\mathbf{I}}
\newcommand{\Jb}{\mathbf{J}}
\newcommand{\Zb}{\mathbf{Z}}
\newcommand{\Fb}{\mathbf{F}}
\newcommand{\Tb}{\mathbf{T}}
\newcommand{\pb}{\mathbf{p}}
\newcommand{\qb}{\mathbf{q}}
\newcommand{\Mb}{\mathbf{M}}
\newcommand{\Pb}{\mathbf{P}}
\newcommand{\Qb}{\mathbf{Q}}
\newcommand{\wb}{\mathbf{w}}
\newcommand{\Gb}{\mathbf{G}}
\newcommand{\Sb}{\mathbf{S}}
\newcommand{\Xb}{\mathbf{X}}
\newcommand{\Vb}{\mathbf{V}}
\newcommand{\Wb}{\mathbf{W}}
\newcommand{\Rb}{\mathbf{R}}
\newcommand{\lb}{\mathbf{\ell}}
\newcommand{\vs}{v_{\perp}}
\newcommand{\rhob}{\boldsymbol{\rho}}
\newcommand{\gamb}{\boldsymbol{\gamma}}
\newcommand{\psib}{\boldsymbol{\psi}}
\newcommand{\etab}{\boldsymbol{\eta}}
\newcommand{\omb}{\boldsymbol{\omega}}
\newcommand{\sigb}{\boldsymbol{\sigma}}
\newcommand{\vg}{\vb_\tn{g}}
\newcommand{\uE}{\ub_{E}}
\newcommand{\bs}[1]{{\boldsymbol #1}}

\let\eps\varepsilon
\let\wt\widetilde
\let\tn\textnormal
\let\vphi\varphi
\let\pa\partial
\let\para\parallel
\let\wh\widehat
\let\lra\leftrightarrow
\let\ov\overline

\newcommand\Tstrut{\rule{0pt}{2.6ex}}         % = `top' strut
\newcommand\Bstrut{\rule[-0.9ex]{0pt}{0pt}}

\def\ts{\tn s}
\def\tA{\tn A}
\def\tc{\tn c}
\def\ti{\tn i}
\def\te{\tn e}
\def\th{\tn h}
\def\tt{\tn t}
\def\tg{\tn g}
\def\tE{\tn E}
\def\tB{\tn B}
\def\td{\tn d}
\def\tb{\tn b}

\def\vs{\mathsf v}
\def\ws{\mathsf w}
\def\dd{\mathrm{d}}
\def\pab{\bold \pa}
\def\grad{\tn{grad}}
\def\curl{\tn{curl}}
\def\div{\tn{div}}
\def\vol{\tn{vol}}
\def\unit{\mathbb 1}
\def\pab{\boldsymbol \pa}
\def\Lamb{\boldsymbol \Lambda}

\def\be{\begin{equation}}
\def\ee{\end{equation}}
\def\bes{\begin{equation*}}
\def\ees{\end{equation*}}

\DeclareMathOperator{\arctantwo}{arctan2}

\newcommand{\fder}[2]{\frac{\delta #1}{\delta #2}}
\newcommand{\pder}[2]{\frac{\partial #1}{\partial #2}}
\newcommand{\intv}[1]{\int_{\RR^3} #1\,d\bfv }
\newcommand{\parfra}[2]{\frac{\partial #1}{\partial #2}}
\newcommand{\paz}[2]{\frac{\partial #1}{\partial z_{#2}}}
\newcommand{\pay}[2]{\frac{\partial #1}{\partial y_{#2}}}
\newcommand{\dt}[1]{\frac{\mathrm d #1}{\mathrm dt}}
\newcommand{\dtpr}[1]{\frac{\mathrm d #1}{\mathrm dt'}}
\newcommand{\ds}[1]{\frac{\mathrm d #1}{\mathrm ds}}
\newcommand{\dalph}[1]{\frac{d #1}{d\theta}}
\newcommand{\dlam}[1]{\frac{d #1}{d\lambda}}
\newcommand{\gavg}[1]{\langle #1 \rangle}
%\newcommand{\gavg}[1]{\overline{ #1 }}

\newcommand{\stef}[1]{\color{blue}#1 \color{black}}
\newcommand{\alert}[1]{\color{red}#1 \color{black}}
\newcommand{\delet}[1]{{\tiny \sout{#1}}}

\newcommand{\bigzero}{\mbox{\normalfont\Large\bfseries 0}}
\newcommand{\rvline}{\hspace*{-\arraycolsep}\vline\hspace*{-\arraycolsep}}

\renewcommand{\div}{\tn{div}}

\theoremstyle{definition}
\newtheorem{definition}{Definition}
\newtheorem{assumption}{Assumption}
\newtheorem{remark}{Remark}
\newtheorem*{example*}{Example}

\theoremstyle{plain}
\newtheorem{lemma}{Lemma}
\newtheorem{prop}{Proposition}
\newtheorem{theorem}{Theorem}
\newtheorem*{theorem*}{Theorem}
\newtheorem{corollary}[theorem]{Corollary}
\newtheorem{propT}{Proposition}
\newtheorem{thmT}{Theorem}






% \pagestyle{fancy}
% \fancyhf{}
% \rhead{S. Possanner}
% \lhead{Notes on plasma transport models}
% \rfoot{\thepage}




\title{GEMPIC with smooth polar splines}

\author[2]{Florian Holderied}
\author[2]{Katharina Kormann}
\author[1,2]{Benedikt Perse}
\author[1,2]{Stefan Possanner}
%\email[]{stefan.possanner@ma.tum.de}
%\homepage[]{http://www-m16.ma.tum.de/Allgemeines/StefanPossanner}
%\thanks{}
%\affiliation{Max Planck Institute for Plasma Physics, Boltzmannstraße 2, 85748 Garching, 
%Germany}
%\altaffiliation{Technical University of Munich, Department of Mathematics, Boltzmannstraße 3, 
%85748 
%Garching, Germany}
\affil[1]{Technical University of Munich, Department of Mathematics, Boltzmannstraße 3, 
85748 Garching, Germany}
\affil[2]{Max Planck Institute for Plasma Physics, Boltzmannstraße 2, 85748 Garching, Germany}

\date{\today}


\begin{document}
\maketitle

\begin{abstract}
We construct smooth polar splines within the geometric electromagentic particle-in-cell (GEMPIC) framework for domains with a polar singularity. In GEMPIC the electromagntic fields are represented as 1- resp. 2-forms and discretized with tensor product B-spline spaces satisfying a commuting diagram conform to the exact de Rham sequence. We build the finite dimensional sequence using smooth polar splines such that these properties stay intact. The resulting basis is not a tensor product anymore, but single-valued at the pole when pushed forward to the mapped domain. With these new bases we design a particle pusher that allows particles to cross the pole without loss of precision. The polar GEMPIC framework allows for structure-preserving integration in mapped domains with a pole, as for instance in toroidal geometries. 
\end{abstract}

\tableofcontents


\section{Introduction}

\newpage
\section{Problem statement}

\subsection{Polar mapping} 

\begin{figure}[htb]
\includegraphics[width=\textwidth]{pics/mapping2.png}
\caption{Cylindrical coordinates.} \label{fig:map}
\end{figure}
Let us denote the "physical domain" by $\Omega\subset \RR^3$ and its Cartesian coordinates by $\xb=(x,y,z) \in \Omega$. The "logical domain" $\hat \Omega \subset \RR^3$ is assumed to be box-shaped, suitable for tensor product construction, and with logical (or patch) coordinates $\etab=(\eta,\xi,z') \in \hat \Omega$. The two domains are related by the mapping
\be
 F: \hat \Omega\to\Omega\,,\qquad  (\eta,\xi,z') \mapsto (x,y,z)\,, \qquad F^{-1}\in C^{p}(\Omega\setminus(x_0,y_0))\,.
\ee
The mapping $F$ is $C^{p}$, $p\geq 1$ (later the spline degree), and invertible everywhere except at the pole $(x_0,y_0)$. As a generic example, let us consider cylindrical coordinates defined on $\hat \Omega=[0,1]\times[0,2\pi)\times[0,L]$ via
\be \label{Fmap}
 F: \etab \mapsto \xb\, \qquad \begin{pmatrix} x \\ y \\ z \end{pmatrix} = 
 \begin{pmatrix} f(\eta) \cos\xi \\ f(\eta)\sin\xi \\ z' \end{pmatrix}\,,
\ee
where we assume $f$ to be some function with the properties
\be
 f:[0,1] \to \RR\,,\quad f(0) = 0\,,\quad 0 < f' < \infty\,.
\ee
Hence, in the following $\eta$ denotes the "radial coordinate" while $\xi$ plays the role of the angular coordinate. The pole is attained for $\eta\to 0$. The Jacobian $DF$ of $F$ and its inverse are given by
\be
 DF  =
 \begin{pmatrix}
  f'\cos\xi & -f\sin\xi & 0
  \\
  f'\sin\xi & f \cos\xi & 0
  \\
  0 & 0 & 1 
 \end{pmatrix}\,, \qquad
 DF^{-1} = \begin{pmatrix}
  1/f'\cos\xi & 1/f'\sin\xi & 0
  \\
  - 1/f\sin\xi & 1/f\cos\xi & 0
  \\
  0 & 0 & 1
 \end{pmatrix} \,,
\ee
from which follow the metric tensor $G$ and its inverse,
\begin{align}
 G = DF^\top DF = \begin{pmatrix}
 (f')^2 & 0 & 0 \\
 0 & f^2 & 0 \\
 0 & 0 & 1
 \end{pmatrix}\,,\qquad 
 G^{-1} = \begin{pmatrix}
 1/(f')^2 & 0 & 0 \\
 0 & 1/f^2 & 0 \\
 0 & 0 & 1
 \end{pmatrix}\,,
\end{align}
with the determinant $g = \det G = (ff')^2$. The cylindrical mapping is illustrated in Figure~\ref{fig:map}.



%%%%%%%%%%%%%%%%%%%
\subsection{Hilbert spaces of differential forms}

\begin{figure}[htb]
\includegraphics[width=\textwidth]{pics/deRham3D.png}
\caption{Commuting diagram for the logical domain $\hat \Omega$.} \label{fig:diag}
\end{figure}
Conforming FE methods in three dimensions can be built upon the commuting diagram depicted in Figure \ref{fig:diag}. All spaces in this diagram refer to functions on the logical domain $\hat\Omega$. The upper line contains the continuous spaces well-known in FE analysis. In the framework of FEEC, these spaces refer to the components of differentiable $n$-forms, with $0\leq n \leq 3$. We use the symbol
\begin{alignat}{2}
 H^1(\hat\Omega) &= \left\{ a :\hat\Omega \to \RR\ s.t.\ |a|_0 + |\grad\, a|_1 < \infty \right\} \qquad &&\tn{(0-forms)}\,,
 \\[1mm]
 H(\curl,\hat\Omega) &= \left\{ \ab :\hat\Omega \to \RR^3\ s.t.\ |\ab|_1 + |\curl\, \ab|_2 < \infty \right\}\qquad &&\tn{(1-forms)}\,,
 \\[1mm]
 H(\div,\hat\Omega) &= \left\{ \ab :\hat\Omega \to \RR^3\ s.t.\ |\ab|_2 + |\div\, \ab|_3 < \infty \right\}\qquad &&\tn{(2-forms)}\,,
 \\[1mm]
 L^2(\hat\Omega) &= \left\{ a :\hat\Omega \to \RR\ s.t.\ |a|_3 < \infty \right\}\qquad &&\tn{(3-forms)}\,,
\end{alignat}
where the seminorms $|\cdot|_{0\leq n\leq 3}$ are given by
\begin{align}
 |a|_0^2 &:= \int_{\hat\Omega} a^2\,\sqrt g\,\tn d\etab\,,
 \\[1mm]
 |\ab|_1^2 &:= \int_{\hat\Omega} \ab\, G^{-1} \ab\,\sqrt g\,\tn d\etab\,,
 \\[1mm]
 |\ab|_2^2 &:= \int_{\hat\Omega} \ab\, G\, \ab\,\frac{1}{\sqrt g}\,\tn d\etab\,,
 \\[1mm]
 |a|_3^2 &:= \int_{\hat\Omega} a^2\,\frac{1}{\sqrt g}\,\tn d\etab\,.
\end{align}
Denoting $\hat\nabla=(\pa_\eta,\pa_\xi,\pa_{z'})$ in logical coordinates, the differential operators can be written as
\be
 \grad = \hat \nabla\,,\qquad \curl = (\hat\nabla \times)\,,\qquad \div = (\hat\nabla \cdot)\,.
\ee
In cylindrical coordinates, the above Hilbert spaces are defined as follows:
\begin{align}
 a \in H^1(\hat\Omega): &\ \int_{\hat\Omega} a^2\,ff'\,\tn d \etab + \int_{\hat\Omega} \left[ (\pa_\eta a)^2 \frac{f}{f'} + (\pa_\xi a)^2 \frac{f'}{f} + (\pa_{z'} a)^2 \,ff' \right]\tn d \etab < \infty\,,
 \\[3mm]
 \ab \in H(\curl,\hat\Omega) : &\ \int_{\hat\Omega} \left[ a_\eta^2 \frac{f}{f'} + a_\xi^2 \frac{f'}{f} + a_{z'}^2 \,ff' \right]\tn d \etab 
 \\[1mm]
  +&\ \int_{\hat\Omega} \left[ (\pa_\xi a_{z'} - \pa_{z'} a_\xi)^2 \frac{f'}{f} + (\pa_{z'} a_\eta - \pa_\eta a_{z'})^2 \frac{f}{f'} + (\pa_\eta a_\xi - \pa_\xi a_\eta)^2 \,\frac{1}{ff'}\right]\tn d \etab < \infty\,,  \nonumber
 \\[3mm]
 \ab \in H(\div,\hat\Omega) : &\ \int_{\hat\Omega} \left[ a_\eta^2 \frac{f'}{f} + a_\xi^2 \frac{f}{f'} + a_{z'}^2 \,\frac{1}{ff'} \right]\tn d \etab 
 \\[1mm]
 +&\ \int_{\hat\Omega} \left[ (\pa_\eta a_\eta)^2 + (\pa_\xi a_\xi)^2  + (\pa_{z'}a_{z'})^2 \right]\frac{1}{ff'}\, \tn d \etab < \infty\,, \nonumber
 \\[3mm]
 a \in L^2(\hat\Omega): &\ \int_{\hat\Omega} a^2\,\frac{1}{ff'}\,\tn d \etab< \infty\,.
\end{align}
For $f(\eta) = \eta^q$ with $q>0$ we have $f/f' = \eta/q$ and $ff' = q \eta^{2q-1}$. Then $q=1/2$ yields $f/f' = 2\eta$ and $ff' = 1/2$ such that integrals featuring the factor $f'/f$ must be handled with care on the discrete level. The Hilbert spaces form an exact sequence, meaning that
\be \label{sequence}
 \grad\,H^1 = \ker(\curl\,H(\curl))\,,\qquad \curl\,H(\curl) = \ker(\div\, H(\div))\,.
\ee

The operators $\Pi_j$, $0\leq j\leq 3$ project onto the finite-dimensional subspaces $V_j$, $0\leq j \leq 3$, which will be spanned by tensor product basis functions, constructed from univariate B-splines of degree $p$, denoted by $\hat N^p_i(\eta)$, $0\leq i\leq \hat n_N-1$.  The sequence of $\hat n_N$ splines $(\hat N^p_i)_i$ is constructed from the knot vector $\cT_p = \{\eta_i\}_{0\leq i\leq n+2p}$, composed of $n+2p+1$ non-decreasing points $\eta_i$ in a logical interval $\hat I\subset \RR$. Here, $n$ is the number of cells partitioning the interval $\hat I$ to define the 1D space grid. Each spline $\hat N^p_i$ is defined by $p+2$ neighbouring knots, such that we can fit $n+p$ spline functions into the knot vector $\cT_p$. The ensuing spline basis $(\hat N^p_i)_i$ can be either periodic or "clamped". In the periodic case we relate the first $p$ and the last $p$ splines to obtain $\hat n_N = n$ basis functions. In the clamped case we have $\hat n_N = n+p$ basis functions. Moreover, for clamped splines $\hat N^p_0(\eta_0) = \hat N^p_{\hat n_N-1}(\eta_{n+2p}) = 1$, where $\eta_0$ is the left and $\eta_{n+2p}$ is the right boundary of $\hat I$.  Because of partition of unity we have
\be \label{Nto0}
 \tn{clamped:}\qquad \hat N^p_i(\eta_0) = \hat N^p_i(\eta_{n+2p}) = 0\,,\qquad 1\leq i\leq \hat n_N -2\,.
\ee
The derivative of $\hat N^p_i(\eta)$ can be written as
\be \label{N'}
 ({\hat N_i^p})'(\eta) = \hat D_{i-1}^{p-1}(\eta)-\hat D_{i}^{p-1}(\eta)\,,
\ee
where we introduced the "D-splines" of degree $p-1$ as
\be \label{def:D}
\begin{aligned}
 \hat D_{i}^{p-1}(\eta) &= \frac{p}{\eta_{i+p+1}-\eta_{i+1}} \hat N_{i+1}^{p-1}(\eta)\,,\qquad -1\leq i \leq \hat n_N-1\,,
% \\[1mm]
% D_{\hat n_N-1}^{p-1}(\eta) &= \begin{cases}
% D_{0}^{p-1}(\eta) & \tn{for periodic}
% \\
% 0 & \tn{for clamped}
% \end{cases}
% \,.
 \end{aligned}
\ee
It is convenient to view D-splines as usual B-splines of degree $p-1$ created from the same knot vector $\cT_p$ as the $\hat N^p_i$, and multiplied by the factor $p/(\eta_{i+p+1}-\eta_{i+1})$. We can fit $n+p+1$ basis splines of degree $p-1$ into the knot vector $\cT_p$. In the periodic case we relate the first $p+1$ D-splines with the last $p+1$ D-splines. In the clamped case we have $\hat D_{-1}^{p-1}(\eta) = \hat D_{\hat n_N-1}^{p-1}(\eta) = 0$. Thus, we finally end up with the D-spline sequence $(\hat D^{p-1}_i)_i$, $0\leq i \leq \hat n_D-1$, where $\hat n_D = \hat n_N$ for periodic and $\hat n_D = \hat n_N-1$ for clamped splines.






%%%%%%%%%%%%%%%%%%%
\subsection{Construction of polar basis functions}

We start from the tensor product space $V_0$ defined by
\be
 V_0 = \tn{span} (\hat \Lambda^0_i)\,,\qquad \hat \Lambda^0_i = \hat N^{p_1}_{i_1}(\eta)\,\hat N^{p_2}_{i_2}(\xi)\,\hat N^{p_3}_{i_3}(z')\,, \quad i = i_1(\hat n_{N}^2 \hat n_{N}^3) + i_2\,\hat n_{N}^3 + i_3\,,
\ee
for $0\leq i_j \leq \hat n_{N}^j-1$, $j=1,2,3$. We assume $\hat N^{p_1}_{i_1}(\eta)$ to be clamped splines, whereas the other two directions are periodic.
%Moreover, we have to remove the interpolator spline $N^{p_1}_0$ from the basis, imposing homogeneous Dirichlet conditions at $\eta=0$, in order to assure $\Lambda^0_i$ to be single-valued a the pole:
%\be
%\lim_{\eta\to 0} \Lambda^0_i = \lim_{\eta\to 0} N^{p_1}_{i_1}(\eta)\,N^{p_2}_{i_2}(\xi)\,N^{p_3}_{i_3}(z') = 0 \quad \forall \ (\xi,z')\,,\qquad 1 \leq i_1 \leq \hat n_N^1\,.
%\ee
%This holds because of \eqref{Nto0}.
 In order to maintain the exact sequence property \eqref{sequence} we construct the other spaces $V_{1\leq j \leq 3}$ as follows:
\begin{align}
 V_1 &:= \tn{span}\left(
 \begin{pmatrix}
 \pa_\eta \hat\Lambda^0_i \\ 0 \\ 0
 \end{pmatrix},
  \begin{pmatrix}
 0 \\ \pa_\xi \hat\Lambda^0_i \\ 0
 \end{pmatrix},
  \begin{pmatrix}
 0 \\ 0 \\ \pa_{z'} \hat\Lambda^0_i 
 \end{pmatrix}
  \right)\,,  \label{spanV1}
  \\[3mm]
  V_2 &:= \tn{span}\left(
 \begin{pmatrix}
 \pa_\xi\pa_{z'} \hat\Lambda^0_i \\ 0 \\ 0
 \end{pmatrix},
  \begin{pmatrix}
 0 \\ \pa_\eta\pa_{z'} \hat\Lambda^0_i \\ 0
 \end{pmatrix},
  \begin{pmatrix}
 0 \\ 0 \\ \pa_\eta\pa_\xi \hat\Lambda^0_i 
 \end{pmatrix}
  \right)\,,  \label{spanV2}
  \\[3mm]
  V_3 &:= \tn{span}(\pa_\eta\pa_\xi\pa_{z'} \hat\Lambda^0_i)\,.
\end{align}
We shall hold on to this construction even when the basis $\hat\Lambda^0$ is not a tensor product basis anymore. One problem of the tensor product basis in the case of cylindrical coordinates is immediately obvious, namely that $\hat\Lambda^0_i$ is not single-valued as $\eta\to 0$, hence at the pole. This means: 

\begin{itemize}
\item Tensor product $V_0$-basis functions $\Lambda^0_i(\xb)$ are not $C^0$ at the pole in the physical domain. 
\item If we construct $\Lambda^0_i(\xb)$ to be $C^0$ somehow, $V_1$-basis functions are not single-valued at the pole.
\item If we construct $\Lambda^0_i(\xb)$ to be $C^1$ somehow, the third $V_2$-basis functions and the $V_3$-basis functions (mixed derivatives $\pa_\eta\pa_\xi$) are not single-valued at the pole.
\item Our goal is thus as follows: {\bf $\Lambda^0_i(\xb)$ must be $C^1$ at the pole and $\pa_\eta\pa_\xi \hat\Lambda^0_i$ must be single-valued at the pole. We also want an IGA-compatible basis.}
\end{itemize}








\bibliography{\string~/Desktop/PLASMA/WORK/mybib.bib}
\bibliographystyle{plain}

\end{document}          
