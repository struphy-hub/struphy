 \documentclass[11pt,oneside,a4paper,fleqn]{article}


% Usepackages
\usepackage[left=2cm,right=2cm,top=2cm,bottom=2cm]{geometry}
\usepackage[intoc]{nomencl}
\usepackage{float}
\usepackage{floatflt}
\restylefloat{figure}
\usepackage{color}
\usepackage{siunitx}
\usepackage{hyperref}
\usepackage[utf8]{inputenc}
\sisetup{separate-uncertainty}
\usepackage[T1]{fontenc}
\usepackage[english]{babel}
\usepackage{ae}
\usepackage{chngcntr}
\usepackage[round]{natbib}
\usepackage[overload]{empheq}
\usepackage{amsthm}
\usepackage{amssymb}
\usepackage{latexsym}
\usepackage[title]{appendix}
\usepackage{multicol}
\usepackage{amsmath}
\usepackage{amsfonts}
\usepackage{tabularx}
\usepackage{caption}
\usepackage[multiple]{footmisc}
\usepackage{color}
\definecolor{grau}{rgb}{0.95,0.95,0.95}
\definecolor{dunkelgrau}{rgb}{0.8,0.8,0.8}
\usepackage{colortbl}
\usepackage{authblk}
\usepackage{url}
\usepackage{xcolor}
\usepackage{pgf}
\usepackage{wrapfig}
\usepackage[printwatermark]{xwatermark}
\usepackage{xcolor}
\usepackage{graphicx}
\usepackage{lipsum}
\usepackage{tikz}
\usepackage{abstract}
\numberwithin{equation}{section}
\usepackage{mathbbol}
\usepackage{amssymb}             % AMS Math

\DeclareSymbolFontAlphabet{\amsmathbb}{AMSb}%






% Commands
\newcommand{\pa}{\partial}
\newcommand{\mr}[1]{\mathrm{#1}}
\newcommand{\mb}[1]{\mathbf{#1}}
\newcommand{\bo}[1]{\boldsymbol{#1}}
\newcommand{\fh}{f_\mr{h}^\mr{D}}
\newcommand{\fho}{f_\mr{h0}^\mr{D}}
\newcommand{\fhi}{f_\mr{h1}^\mr{D}}
\newcommand{\Bpa}{B_\parallel^\ast}


\renewcommand\Affilfont{\fontsize{10}{20}\itshape}

% Title
\title{Hybrid kinetic-shear Alfv\'{e}n physics using discrete differential forms}

% Date
\date{}

% Authors and affiliations
\author[1]{Florian Holderied}


\affil[1]{\textit{Max Planck Institute for Plasma Physics, Boltzmannstrasse 2, 85748 Garching, Germany}}
\affil[2]{\textit{Technical University of Munich, Department of Physics, Boltzmannstrasse 2, 85748 Garching, Germany}}





\begin{document} 

\maketitle

\begin{abstract}
\noindent This work documents the progress regarding the development of a new code which simulates the equations of linear ideal magnetohydrodynamics (MHD) which are nonlinearly coupled to a kinetic equation (either full-orbit or drift-kinetic). Such so-called hybrid models are a suitable way to describe the self-consistent interaction of a hot plasma (governed by a kinetic theory) with a fluid bulk (governed by MHD). Typical examples are those of nuclear fusion devices, in which energetic ions, either fusion born or coming from external heating devices, interact with the ambient plasma, and those of spaces plasmas, involving the interaction of fast electron in the solar wind with Earth's magnetosphere. The goal of the present work is to explore the usage of numerical methods which are related to \textit{finite element exterior calculus} (FEEC) with the aim to exactly preserve as many properties of the continuous model as possible, e.g. conservation of energy, the divergence-free constraint for the magnetic field or ideally the full Hamiltonian structure, leading to good long-time stability properties. The FEEC approach is based on considerations from algebraic geometry allowing for preserving cohomological invariants at the discrete level. In practice, it ensures the reproduction of the well-known identities curl(grad)=0 and div(curl)=0 at the discrete level.
\end{abstract}

\section{Full model and model reduction}
We consider a hybrid kinetic-MHD model where the coupling of the fluid and kinetic species is done via a current coupling scheme. In classical vector calculus notation the model reads
\begin{align}
&\frac{\pa\rho}{\pa t}+\nabla\cdot(\rho\mb{U})=0,\label{eq_continuity}\\
&\frac{\pa\mb{U}}{\pa t}+(\mb{U}\cdot\nabla)\mb{U}=\frac{1}{\rho}(\nabla\times\mb{B}+\rho_\text{h}\mb{U}-\mb{j}_\text{h})\times\mb{B}-\frac{\nabla p}{\rho},\\
&\frac{\pa\mb{B}}{\pa t}=\nabla\times(\mb{U}\times\mb{B}),\\
&\frac{\pa p}{\pa t}+\nabla\cdot(p\mb{U})+(\gamma-1)p\nabla\cdot\mb{U}=0,
\label{eq_pressure}\\
&\frac{\pa f_\text{h}}{\pa t}+\mb{v}\cdot\nabla f_\text{h}+(\mb{B}\times\mb{U}+\mb{v}\times\mb{B})\cdot\nabla_\mb{v}f_\text{h}=0,\\
&\rho_\text{h}=\int f_\text{h}\text{d}^3v,\qquad\mb{j}_\text{h}=\int\mb{v}f_\text{h}\text{d}^3v.\label{eq_moments}
\end{align}
where we set all physical constants equal to one\footnote{We assume hot ions with a positive charge. Therefore, one does not have to be careful with the signs in (\ref{eq_moments})}. This set of equations forms a closed system of nonlinear partial differential equations for the the bulk mass density $\rho$, the bulk velocity $\mb{U}$, the magnetic induction $\mb{B}$ (which we will simply refer to as magnetic field), the bulk pressure $p$ and the distribution function of the hot ions $f_\text{h}$. Furthermore, $\gamma=5/3$ is the adiabatic exponent. This system possesses a Hamiltonian structure with the following conserved energy:
\begin{align}
\mathcal{H}_0(t)=\frac{1}{2}\int\rho\mb{U}^2\text{d}^3x+\frac{1}{\gamma-1}\int p\text{d}^3x+\frac{1}{2}\int\mb{B}^2\text{d}^3x+\frac{1}{2}\int\int\mb{v}^2f_\text{h}\mr{d}^3v\mr{d}^3x.\label{eq_H0}
\end{align}
In order to start from a simpler model we shall restrict ourselves for the moment on linearized MHD equations by assuming that MHD waves are small perturbations about an equilibrium state. However, we keep all nonlinearities which are related to the coupling to the kinetic species. Making the ansatzes $\mb{B}=\mb{B}_\text{eq}+\tilde{\mb{B}}$, $\mb{U}=\tilde{\mb{U}}$ (zero-flow equilibrium), $\rho=\rho_\text{eq} + \tilde{\rho}$ and $p=p_\text{eq}+\tilde{p}$ and neglecting nonlinear terms in the MHD part yields
\begin{align}
&\frac{\pa\tilde{\rho}}{\pa t}+\nabla\cdot(\rho_\text{eq}\tilde{\mb{U}})=0,\label{eq_continuity_linear}\\
\rho_\text{eq}&\frac{\pa\tilde{\mb{U}}}{\pa t}=(\nabla\times\tilde{\mb{B}})\times\mb{B}_\text{eq}+(\nabla\times\mb{B}_\text{eq})\times\tilde{\mb{B}}+(\rho_\text{h}\tilde{\mb{U}}-\mb{j}_\text{h})\times\mb{B}-\nabla\tilde{p},\label{eq_momentum}\\
&\frac{\pa\tilde{\mb{B}}}{\pa t}=\nabla\times(\tilde{\mb{U}}\times\mb{B}_\text{eq}),\label{eq_induction}\\
&\frac{\pa\tilde{p}}{\pa t}+\nabla\cdot(p_\text{eq}\tilde{\mb{U}})+(\gamma-1)p_\text{eq}\nabla\cdot\tilde{\mb{U}}=0,\label{eq_pressure_linear}\\
&\frac{\pa f_\text{h}}{\pa t}+\mb{v}\cdot\nabla f_\text{h}+(\mb{B}\times\tilde{\mb{U}}+\mb{v}\times\mb{B})\cdot\nabla_\mb{v}f_\text{h}=0.\label{eq_Vlasov}
\end{align}
The linearization of the MHD part has the consequence that the Hamiltonian (\ref{eq_H0}) is no longer conserved. However, if we define the new Hamiltonian
\begin{align}
\mathcal{H}_1(t)=\frac{1}{2}\int\rho_\text{eq}\tilde{\mb{U}}^2\,\text{d}^3x+\frac{1}{2}\int\tilde{\mb{B}}^2\,\text{d}^3x+\frac{1}{2}\int\int\mb{v}^2f_\text{h}\,\mr{d}^3v\,\mr{d}^3x+\frac{1}{\gamma-1}\int\tilde{p}\,\text{d}^3x,\label{eq_H1}
\end{align} 
we get the following energy balance:
\begin{align}
\frac{\text{d}\mathcal{H}_1}{\text{d}t}&=\int\tilde{\mb{U}}\cdot\left[(\nabla\times\mb{B}_\text{eq})\times\tilde{\mb{B}}\right]\,\text{d}^3x-\int\tilde{\mb{U}}\cdot\nabla\tilde{p}\,\text{d}^3x-\int p_\text{eq}\nabla\cdot\tilde{\mb{U}}\,\text{d}^3x\\
&=\int\tilde{\mb{U}}\cdot\left[(\nabla\times\mb{B}_\text{eq})\times\tilde{\mb{B}}\right]\,\text{d}^3x-\int\tilde{\mb{U}}\cdot\nabla\tilde{p}\,\text{d}^3x+\int\tilde{\mb{U}}\cdot\nabla p_\text{eq}\,\text{d}^3x.
\end{align}
\begin{figure}
\centering
\includegraphics[scale=0.55]{deRham3d_forms.pdf}
\caption{Commuting diagram for function spaces in 3d. The upper line represents the de Rham sequence for the continuous spaces, while the lower line represents the discrete counterpart.\label{fig_diagram}}
\end{figure}
We shall use classical particle-in-cell techniques for solving the kinetic equation (\ref{eq_Vlasov}) and the framework of \textit{finite element exterior calculus} (FEEC) for solving field equations. In the latter, one works with differential forms rather than vector and scalar field. This allows us to treat arbitrary geometries in a natural fashion. For physical reasons we assume the bulk mass density and the hot charge density to be 3-forms ($\rho,\rho_\text{h}\rightarrow\rho_\text{eq}^3,\rho_\text{h}^3$), the magnetic field and the hot ion current density to be 2-forms ($\mb{B},\mb{j}_\text{h}\rightarrow B^2,j^2_\text{h}$) and the pressure to be a 0-form ($p\rightarrow p^0$). The bulk velocity field, which is naturally a vector field, we choose to be a 2-form ($\tilde{\mb{U}}\rightarrow \tilde{U}^2$). Eq. (\ref{eq_continuity_linear})-(\ref{eq_pressure_linear}) can then be written as
\begin{align}
&\frac{\pa\tilde{\rho}^3}{\pa t}+\text{d}(i_{\lozenge U^2}\rho_\text{eq}^3)=0,\\
(\ast\rho_\text{eq}^3)\wedge&\frac{\pa\tilde{U}^2}{\pa t}=\ast i_{\lozenge B^2_\text{eq}}\mr{d}\ast \tilde{B}^2+\ast i_{\lozenge\tilde{B}^2}\mr{d}\ast B^2_\text{eq}-(\ast\rho^3_\text{h})\wedge \ast i_{\lozenge\tilde{U}^2}B^2+\ast i_{\lozenge j_\text{h}^2}B^2-\ast d\tilde{p}^0,\\
&\frac{\pa \tilde{B}^2}{\pa t}+\mr{d}(i_{\lozenge U^2}B^2_\text{eq})=0,\\
&\frac{\pa \tilde{p}^0}{\pa t}+\ast\text{d}(p_\text{eq}^0\wedge U^2)+(\gamma-1)\ast(p_\text{eq}^0\wedge\text{d}U^2)=0,
\end{align} 
where $\ast$ is the Hodge-star operator, $\wedge$ the wedge product, $i$ the interior product and $\lozenge$ the operator which transforms a 2-form to a vector field.



\section{Semi-discretization in space}
As a next step, we introduce finite element basis functions which satisfy a discrete deRham sequence and which form a commuting diagram with the continuous functions via the interpolation-histopolation projectors $\Pi_0$, $\Pi_1$, $\Pi_2$ and $\Pi_3$. This is depicted in Fig. \ref{fig_diagram}. 
\begin{figure}
\centering
\includegraphics[scale=0.5]{mapping.pdf}
\caption{Commuting projectors. Unlike classical finite element methods, the degrees of freedom do not only represent point values, but also edge integrals, face integrals and volume integrals. \label{fig_projectors}}
\end{figure}
Assuming that we know the basis functions in each space (how this can be done with e.g. tensor-product B-splines, see Sec. ...), we express the forms in their respective bases as
\begin{alignat}{2}
&\tilde{p}^0(t,\boldsymbol{\xi})\approx \tilde{p}_h^0(t,\boldsymbol{\xi})=\sum_{\mathbf{i}}p_\mb{i}(t)\Lambda^0_\mb{i}(\boldsymbol{\xi}), &&\mb{p}^\top:=(p_0(t),\ldots,p_{N-1}(t))\in\mathbb{R}^N,\\
&\tilde{U}^2(t,\boldsymbol{\xi})\approx \tilde{U}_h^2(t,\boldsymbol{\xi})=\sum_{\mathbf{i}}\sum_{\mu = 1}^3u_{\mu,\mathbf{i}}(t)\Lambda^2_{\mu,\mathbf{i}}(\boldsymbol{\xi})(\mathrm{d}\xi^\alpha\wedge\mr{d}\xi^\beta)_\mu, \qquad &&\mb{u}^\top:=(\mb{u}_1^\top(t),\mb{u}_2^\top(t),\mb{u}_3^\top(t))\in\mathbb{R}^{3N},\\
&\tilde{B}^2(t,\boldsymbol{\xi})\approx \tilde{B}_h^2(t,\boldsymbol{\xi})=\sum_{\mathbf{i}}\sum_{\mu = 1}^3b_{\mu,\mathbf{i}}(t)\Lambda^2_{\mu,\mathbf{i}}(\boldsymbol{\xi})(\mathrm{d}\xi^\alpha\wedge\mr{d}\xi^\beta)_\mu, \qquad &&\mb{b}^\top:=(\mb{b}_1^\top(t),\mb{b}_2^\top(t),\mb{b}_3^\top(t))\in\mathbb{R}^{3N},\\
&\tilde{\rho}^3(t,\boldsymbol{\xi})\approx \tilde{\rho}_h^3(t,\boldsymbol{\xi})=\sum_{\mathbf{i}}\rho_\mb{i}(t)\Lambda^3_\mb{i}(\boldsymbol{\xi})\,\text{d}\xi^1\wedge\text{d}\xi^2\wedge\text{d}\xi^3, &&\boldsymbol{\rho}^\top:=(\rho_0(t),\ldots,\rho_{N-1}(t))\in\mathbb{R}^N,
\end{alignat}
where $\mb{i}=(i_1,i_2,i_3)$ is a multi-index representing the three coordinate directions on the logical domain and $N=N_1N_2N_3$ is the total number of basis functions. To simplify the notation, we write for the components of the differential forms
\begin{alignat}{2}
&\tilde{p}_h^0\leftrightarrow \tilde{p}_h=(p_0,\ldots,p_{N-1})\begin{pmatrix}
\Lambda_0^0\\ \vdots \\ \Lambda^0_{N-1}
\end{pmatrix}=\mb{p}^\top\boldsymbol{\Lambda}^0, && \boldsymbol{\Lambda}^0\in\mathbb{R}^N\\
&\tilde{U}^2_h\leftrightarrow \tilde{\mb{U}}_h^\top=(\mb{u}_1^\top,\mb{u}_2^\top,\mb{u}_3^\top)\begin{pmatrix}
\bo{\Lambda}^2_1 &0 &0 \\ 0 &\bo{\Lambda}^2_2 &0 \\ 0 &0 &\bo{\Lambda}^2_3
\end{pmatrix}=\mb{u}^\top\mathbb{\Lambda}^2,\qquad&&\mathbb{\Lambda}^2\in\mathbb{R}^{3N\times 3},\\
&\tilde{B}^2_h\leftrightarrow \tilde{\mb{B}}_h^\top=(\mb{b}_1^\top,\mb{b}_2^\top,\mb{b}_3^\top)\begin{pmatrix}
\bo{\Lambda}^2_1 &0 &0 \\ 0 &\bo{\Lambda}^2_2 &0 \\ 0 &0 &\bo{\Lambda}^2_3
\end{pmatrix}=\mb{b}^\top\mathbb{\Lambda}^2,\qquad&&\mathbb{\Lambda}^2\in\mathbb{R}^{3N\times 3},\\
&\tilde{\rho}_h^3\leftrightarrow \tilde{\rho}_{123,h}=(\rho_0,\ldots,\rho_{N-1})\begin{pmatrix}
\Lambda_0^3\\ \vdots \\ \Lambda^3_{N-1}
\end{pmatrix}=\boldsymbol{\rho}^\top\boldsymbol{\Lambda}^3, && \boldsymbol{\Lambda}^3\in\mathbb{R}^N
\end{alignat}
As already stated, we solve the kinetic equation with particle-in-cell techniques. Hence we assume a particle-like distribution function which, in physical space, takes the form
\begin{align}
f_\mr{h}=f_\mr{h}(t,\mb{x},\mb{v})\approx \sum_k w_k(t)\delta(\mb{x}-\mb{x}_k(t))\delta(\mb{v}-\mb{v}_k(t)).
\end{align}
From this, the hot ion charge density, current density and energy density can easily be obtained by taking the first three moments in velocity space:
\begin{align}
&\mathring{\rho}_\mr{h,123}(t,\mb{x})=\sum_k w_k(t)\delta(\mb{x}-\mb{x}_k(t)),\\
&\mathring{\mb{j}}_\mr{h}(t,\mb{x})=\sum_k w_k(t)\delta(\mb{x}-\mb{x}_k(t))\mb{v}_k(t),\\
&\mathring{\epsilon}_\mr{h,123}(t,\mb{x})=\frac{1}{2}\sum_{k}w_k(t)\delta(\mb{x}-\mb{x}_k(t))\mb{v}_k^2(t).
\end{align}
To avoid confusions, we use the notation $\mathring{(\cdot)}$, where necessary, for quantities which are defined on the physical space. Since there is no difference between vectors/scalars and forms in physical space, these expressions are as well the components of the 3-form number density, the 2-form current density and the 3-form energy density. To get the components on the logical domain we apply the transformation formulas for 3-forms, respectively 1-forms to obtain
\begin{align}
&\rho_\mr{h,123}(t,\boldsymbol{\xi})=\sqrt{g}\mathring{\rho}_\mr{h}(t,F(\boldsymbol{\xi}))=\sum_k w_k(t)\delta(\boldsymbol{\xi}-\boldsymbol{\xi}(t)),\\
&\mb{j}_\mr{h}(t,\boldsymbol{\xi})=\sqrt{g}DF^{-1}\mathring{\mb{j}}_\mr{h}(t, F(\boldsymbol{\xi}))=DF^{-1}\sum_k w_k(t)\delta(\boldsymbol{\xi}-\boldsymbol{\xi}_k(t))\mb{v}_k(t),\\
&\epsilon_\mr{h,123}(t,\boldsymbol{\xi})=\sqrt{g}\mathring{\epsilon}_\mr{h}(t,F(\boldsymbol{\xi}))=\frac{1}{2}\sum_k w_k(t)\delta(\boldsymbol{\xi}-\boldsymbol{\xi}_k(t))\mb{v}_k^2(t),
\end{align}
where we made use of the transformation formula
\begin{align}
\delta(\mb{x}-\mb{x}_k(t))=\frac{1}{\sqrt{g}}\delta(\boldsymbol{\xi}-\boldsymbol{\xi}_k(t)).
\end{align}
Let us use these results to derive an energy conserving semi-discrete system for the finite element coefficients of $\tilde{p}_h^0$, $\tilde{U}_h^2$, $\tilde{B}_h^2$ and $\tilde{\rho}^3_h$ and the particle's positions $(\boldsymbol{\xi}_k)_{k=1,\ldots,N_\text{p}}$ and velocities $(\mb{v}_k)_{k=1,\ldots,N_\text{p}}$.


\subsection{Continuity equation}
We start with the discretization of the mass continuity equation which we shall keep in strong form. Following the diagram in Fig. \ref{fig_diagram}, we apply the projector $\Pi_3$ and use the diagram's commutativity to exchange projectors and exterior derivatives:
\begin{align}
&\frac{\pa(\Pi_3\tilde{\rho}^3)}{\pa t}+\Pi_3\text{d}(i_{\lozenge\tilde{U}^2}\rho^3_\text{eq})=\frac{\pa\tilde{\rho}^3_h}{\pa t}+\text{d}\Pi_2(i_{\lozenge\tilde{U}^2}\rho^3_\text{eq})=0.
\end{align}
In terms of the components of the forms\footnote{We use the same symbol for the projectors if it returns the entire form or just its components.}, this amounts to
\begin{align}
&\frac{\pa\tilde{\rho}_h}{\pa t}+\hat{\nabla}\cdot\Pi_2\left[\frac{\rho_\text{eq,123}}{\sqrt{g}}\tilde{\mb{U}}_h\right]=0,\\
\Leftrightarrow\quad&\frac{\pa\tilde{\rho}_h}{\pa t}+\hat{\nabla}\cdot\Pi_2\left[\frac{\rho_\text{eq,123}}{\sqrt{g}}(\mathbb{\Lambda}^2)^\top\right]\mb{u}=0,\\
\Leftrightarrow\quad&\frac{\text{d}\bo{\rho}}{\text{d} t}+\mathbb{D}\mathcal{Q}\mb{u}=0,
\end{align}
where we have introduced the discrete divergence matrix $\mathbb{D}\in\mathbb{R}^{N\times 3N}$ and the projection matrix $\mathcal{Q}\in\mathbb{R}^{3N\times3N}$. We shall use calligraphic symbols for tensors which are related to projections. Explicitly, we have
\begin{align}
\mathcal{Q}_{ij}:=\hat{\Pi}_{2,\mu}^{i_\mu}\left[\frac{\rho_\text{eq,123}}{\sqrt{g}}\Lambda^2_{j\mu}\right],\qquad i=\begin{cases}i_\mu, \quad &\mu=1\\ N+i_\mu &\mu=2\\2N+i_\mu &\mu=3\end{cases}
\end{align}
for $\mu=\{1,2,3\}$ and $\hat{\Pi}_{2,\mu}^{i_\mu}$ selects the $i_\mu$-th coefficient of the projection on the space $V_2$ ($0\leq i_\mu\leq N-1$). If we stack these coefficients in a column vector, $\mathcal{Q}$ can be written as
\begin{align}
\mathcal{Q}&=\begin{pmatrix}
\hat{\Pi}_{2,1}\left[\frac{\rho_\text{eq,123}}{\sqrt{g}}(\bo{\Lambda}_1^2)^\top\right] &0 &0 \\
0 &\hat{\Pi}_{2,2}\left[\frac{\rho_\text{eq,123}}{\sqrt{g}}(\bo{\Lambda}_2^2)^\top\right] &0 \\
0 &0 &\hat{\Pi}_{2,3}\left[\frac{\rho_\text{eq,123}}{\sqrt{g}}(\bo{\Lambda}_3^2)^\top\right]
\end{pmatrix}\label{proj_matrix_p12}\\[1mm]
&=\begin{pmatrix}
\mb{c}_{1,0} &\cdots &\mb{c}_{1,N-1} &0 &\cdots &0 &0 &\cdots &0\\
0 &\cdots &0 &\mb{c}_{2,0} &\cdots &\mb{c}_{2,N-1} &0 &\cdots &0\\
0 &\cdots &0 &0 &\cdots &0 &\mb{c}_{3,0} &\cdots &\mb{c}_{3,N-1}\\
\end{pmatrix}.
\end{align}
Here, e.g. $\mb{c}_{2,0}$ are the coefficients resulting from the projection of the basis function with the index 0 in the block 12 in the matrix (\ref{proj_matrix_p12}). Unfortunately, this is a dense matrix, which is problematic from a memory consumption point of view. Therefore, we just save the right-hand sides, which define a sparse matrix, and perform the final projection in every time step again. Denoting by $(\mathcal{I}_{2,1},\mathcal{I}_{2,2},\mathcal{I}_{2,3})$ the mixed interpolation-histopolation matrices and by $(\mr{vec}_{2,1}(f^2),\mr{vec}_{2,2}(f^2),\mr{vec}_{2,3}(f^2))$ the right-hand side vectors for a 2-form $f^2$, we can write
\begin{align}
Q&=\begin{pmatrix}
\mathcal{I}_{2,1}^{-1} &0 &0 \\ 0 &\mathcal{I}_{2,2}^{-1} &0 \\ 0 &0 &\mathcal{I}_{2,3}^{-1} 
\end{pmatrix}
\begin{pmatrix}
\mr{vec}_{2,1}\left[\frac{\rho_\text{eq,123}}{\sqrt{g}}(\bo{\Lambda}_1^2)^\top\right] &0 &0 \\
0 &\mr{vec}_{2,2}\left[\frac{\rho_\text{eq,123}}{\sqrt{g}}(\bo{\Lambda}_2^2)^\top\right] &0 \\
0 &0 &\mr{vec}_{2,3}\left[\frac{\rho_\text{eq,123}}{\sqrt{g}}(\bo{\Lambda}_3^2)^\top\right]
\end{pmatrix}\\ &=:\bo{\mathcal{I}}_2^{-1}\tilde{\mathcal{Q}}.
\end{align}
Thus, we only precompute the matrices $\tilde{\mathcal{Q}}$ and $\mathcal{I}_2$, whose sparsity immediately from the local support of the basis functions.


\subsection{Momentum equation}
We choose a weak formulation for the momentum equation. Consequently, we take the inner product with a test function $C^2\in H\Lambda^2(\Omega)$ to obtain the variational formulation: Find $\tilde{U}^2\in H\Lambda^2(\Omega)$ such that
\begin{align}
\left((\ast\rho_\text{eq}^3)\wedge\frac{\pa\tilde{U}^2}{\pa t},C^2\right)=&\left(\ast i_{\lozenge B^2_\text{eq}}\mr{d}\ast \tilde{B}^2,C^2\right)+\left(\ast i_{\lozenge\tilde{B}^2}\mr{d}\ast B^2_\text{eq},C^2\right)-\left(\ast\text{d}\tilde{p}^0,C^2\right)\\-&\left((\ast\rho^3_\text{h})\wedge\ast i_{\lozenge\tilde{U}^2}B^2,C^2\right)+\left(\ast i_{\lozenge j_\text{h}^2}B^2,C^2\right)\qquad\forall\;C^2\in H\Lambda^2(\Omega).
\end{align}
We apply the Galerkin approximation to each term and project back into the right spaces where necessary. Let us start with the first term on the left-hand side involving the equilibrium bulk density. To achieve conservation of energy at the discrete level we make use of the fact that the wedge product with a 0-form is just a multiplication with a scalar. Hence the wedge product can as well be applied to the test function and the following equality holds:
\begin{align}
\left((\ast\rho_\text{eq}^3)\wedge\frac{\pa\tilde{U}^2}{\pa t},C^2\right)=\frac{1}{2}\left((\ast\rho_\text{eq}^3)\wedge\frac{\pa\tilde{U}^2}{\pa t},C^2\right)+\frac{1}{2}\left(\frac{\pa\tilde{U}^2}{\pa t},(\ast\rho_\text{eq}^3)\wedge C^2\right).
\end{align}
Using the definition of the inner product of 2-forms involving the metric tensor $G$ yields for the first term
\begin{align}
\left((\ast\rho_\text{eq}^3)\wedge\frac{\pa\tilde{U}^2}{\pa t},C^2\right)&=\int_{\hat{\Omega}}\frac{1}{\sqrt{g}}\rho_\text{eq,123}\left(\frac{\pa\tilde{\mb{U}}}{\pa t}\right)^\top G\mb{C}\frac{1}{\sqrt{g}}\,\mr{d}^3\xi\\
&\approx\int_{\hat{\Omega}}\Pi_2\left[\frac{1}{\sqrt{g}}\rho_{\text{eq},123}\left(\frac{\pa\tilde{\mb{U}}_h}{\pa t}\right)^\top\right]G \mb{C}_h\frac{1}{\sqrt{g}}\,\mr{d}^3\xi\\&=\dot{\mb{u}}^\top\mathcal{Q}^\top\underbrace{\int_{\hat{\Omega}}\frac{1}{\sqrt{g}}\mathbb{\Lambda}^2G(\mathbb{\Lambda}^2)^\top\,\mr{d}^3\xi}_{=:\mathbb{M}^2}\,\mb{c}=\dot{\mb{u}}^\top\mathcal{Q}^\top\mathbb{M}^2\mb{c}\quad\forall\;\mb{c}\in\mathbb{R}^{3N},
\end{align}
where $\mathbb{M}^2\in\mathbb{R}^{3N\times 3N}$ is the mass matrix in the space $V_2$ and the projection matrix is the same one as in the discretized continuity equation. Finally, we obtain
\begin{align}
\left((\ast\rho_\text{eq}^3)\wedge\frac{\pa\tilde{U}^2}{\pa t},C^2\right)\approx\frac{1}{2}\mb{c}^\top\left(\mathbb{M}^2\mathcal{Q}+\mathcal{Q}^\top\mathbb{M}^2\right)\dot{\mb{u}}=:\mb{c}^\top\mathcal{A}\dot{\mb{u}},
\end{align}
with $\mathcal{A}\in\mathbb{R}^{3N\times 3N}$ being a symmetric matrix.

For the first term on the right-hand side, we make use of the identities $\langle\ast i_{\lozenge\gamma^2}\alpha^2,\beta^2\rangle=\langle\alpha^2,\ast\gamma^2\wedge\ast\beta^2\rangle$ and $\ast(\ast \alpha^2\wedge\ast\beta^2)=i_{\lozenge\beta^2}\alpha^2$, such that we obtain
\begin{align}
\left(\ast i_{\lozenge B^2_\text{eq}}\mr{d}\ast \tilde{B}^2,C^2\right)=\left(\mr{d}\ast\tilde{B}^2,\ast B^2_\text{eq}\wedge \ast C^2\right)=\left(\ast\mr{d}\ast\tilde{B}^2,\ast(\ast B_\text{eq}^2\wedge \ast C^2)\right)=\left(\mr{d}^\ast\tilde{B}^2,i_{\lozenge C^2}B_\text{eq}^2\right),
\end{align}
where we introduced the co-differential operator $\mr{d}^\ast\alpha^p=(-1)^p\ast\mr{d}\ast\alpha^p$. Applying the Green formula for differential forms and assuming that the resulting boundary term vanishes yields
\begin{align}
\left(\ast i_{\lozenge B^2_\text{eq}}\mr{d}\ast \tilde{B}^2,C^2\right)&=\left(\tilde{B}^2,\mr{d}i_{\lozenge C^2}B_\text{eq}^2\right)=\int_{\hat{\Omega}}\frac{1}{g}\tilde{\mb{B}}^\top G\left[\hat{\nabla}\times\left(\mb{B}_\text{eq}\times \frac{1}{\sqrt{g}}\mb{C}\right)\right]\sqrt{g}\,\mr{d}^3\xi\\
&\approx\mb{b}^\top\int_{\hat{\Omega}}\frac{1}{\sqrt{g}}\mathbb{\Lambda}^2G(\mathbb{\Lambda}^2)^\top\mr{d}^3\xi\,\mathbb{C}\hat{\Pi}_1\left[\mathbb{B}_\text{eq}\frac{1}{\sqrt{g}}(\mathbb{\Lambda}^2)^\top\right]\mb{c}=\mb{b}^\top\mathbb{M}^2\mathbb{C}\mathcal{T}\mb{c}\quad\forall\,\mb{c}\in\mathbb{R}^{3N},
\end{align}
where we introduced the discrete curl matrix $\mathbb{C}\in\mathbb{R}^{3N\times 3N}$ and we wrote the vector product of the background magnetic field with the test function in terms of a matrix-vector product by using the anti-symmetric matrix
\begin{align}
\mathbb{B}_\text{eq}:=\begin{pmatrix}
0 &-B_{\text{eq},3} &B_{\text{eq},2} \\ B_{\text{eq},3} &0 &-B_{\text{eq},1} \\ -B_{\text{eq},2} &B_{\text{eq},1} &0
\end{pmatrix}\in\mathbb{R}^{3\times 3}.
\end{align}
The projection matrix $\mathcal{T}$ is given by
\begin{align}
\mathcal{T}_{ij}:=\hat{\Pi}_{1,\mu}^{i_\mu}\left[\frac{\mathbb{B}_{\text{eq},\mu k}}{\sqrt{g}}\Lambda^2_{jk}\right]=\Pi_{1,\mu}^{i_\mu}\left[\frac{\epsilon_{\mu mn}B_{\text{eq},m}}{\sqrt{g}}\Lambda^2_{jn}\right],\qquad\mu=\{1,2,3\},
\end{align}
which has the right-hand sides
\begin{align}
&\tilde{\mathcal{T}}=\begin{pmatrix}
0 &-\mr{vec}_{1,1}\left[\frac{B_{\text{eq},3}}{\sqrt{g}}(\bo{\Lambda}^2_2)^\top\right] &\mr{vec}_{1,1}\left[\frac{B_{\text{eq},2}}{\sqrt{g}}(\bo{\Lambda}^2_3)^\top\right] \\[2mm]
\mr{vec}_{1,2}\left[\frac{B_{\text{eq},3}}{\sqrt{g}}(\bo{\Lambda}^2_1)^\top\right] &0 &-\mr{vec}_{1,2}\left[\frac{B_{\text{eq},1}}{\sqrt{g}}(\bo{\Lambda}^2_3)^\top\right] \\[2mm]
-\mr{vec}_{1,3}\left[\frac{B_{\text{eq},2}}{\sqrt{g}}(\bo{\Lambda}^2_1)^\top\right] &\mr{vec}_{1,3}\left[\frac{B_{\text{eq},1}}{\sqrt{g}}(\bo{\Lambda}^2_2)^\top\right] &0
\end{pmatrix}.
\end{align}
\normalsize
Performing the same steps for the second term without integration by parts yields
\begin{align}
\left(\ast i_{\lozenge\tilde{B}^2}\mr{d}\ast B^2_\text{eq},C^2\right)&=\left(\mr{d}\ast\tilde{B}^2,\ast i_{\lozenge C^2}B_\text{eq}^2\right)\\
&=\int_{\hat{\Omega}}\frac{1}{g}\left[\hat{\nabla}\times\left(\frac{1}{\sqrt{g}}G\tilde{\mb{B}}\right)\right]^\top G\sqrt{g}G^{-1}\left(\mb{B}_\text{eq}\times\frac{1}{\sqrt{g}}\mb{C}\right)\sqrt{g}\,\mr{d}^3q\\
&\approx\mb{b}^\top\hat{\Pi}_1\left[\frac{1}{\sqrt{g}}\mathbb{\Lambda}^2G\right]\mathbb{C}^\top\int_{\hat{\Omega}}\frac{1}{\sqrt{g}}\mathbb{\Lambda}^2G(\mathbb{\Lambda}^2)^\top\mr{d}^3\xi\,\hat{\Pi}_2\left[G^{-1}\mathbb{B}_\text{eq}(\mathbb{\Lambda}^2)^\top\right]\mb{c}\\
&=\mb{b}^\top\mathcal{P}_1^\top\mathbb{C}^\top\mathbb{M}^2\mathcal{P}_2\,\mb{c}\quad\forall \,\mb{c}\in\mathbb{R}^{3N}.
\end{align}
The projection matrices $\mathcal{P}_1\in\mathbb{R}^{3N\times3N}$ and $\mathcal{P}_2\in\mathbb{R}^{3N\times3N}$ are given by
\begin{align}
\mathcal{P}_{1,ij}:=\hat{\Pi}_{1,\mu}^{i_\mu}\left[\frac{G_{\mu k}\Lambda_{jk}^2}{\sqrt{g}}\right],\qquad\mathcal{P}_{2,ij}:=\hat{\Pi}_{1,\mu}^{i_\mu}\left[G_{\mu l}\epsilon_{lmn}B_{\text{eq}, m}\Lambda_{jn}^2\right].
\end{align}
As a next step, we consider the term involving the pressure gradient. Using the fact that the Hodge operator is self-adjoint in the inner product, we obtain
\begin{align}
(\ast\text{d}\tilde{p}^0,C^2)=(\text{d}\tilde{p}^0,\ast C^2)&=\int_{\hat{\Omega}}\hat{\nabla}\tilde{p}^\top\,G^{-1}\frac{1}{\sqrt{g}}G\mb{C}\sqrt{g}\,\text{d}^3\xi\\
&\hspace{-2cm}\approx \mb{p}^\top\mathbb{G}^\top\int_{\hat{\Omega}}\mathbb{\Lambda}^1G^{-1}(\mathbb{\Lambda}^1)^\top\,\text{d}^3\xi\,\hat{\Pi}_1\left[\frac{1}{\sqrt{g}}G(\mathbb{\Lambda}^2)^\top\right]\mb{c}=\mb{p}^\top\mathbb{G}^\top\mathbb{M}^1\mathcal{W}\,\mb{c}\quad\forall \,\mb{c}\in\mathbb{R}^{3N},
\end{align}
where $\mathbb{G}\in\mathbb{R}^{3N\times N}$ being the discrete gradient matrix, $\mathbb{M}^1\in\mathbb{R}^{3N\times 3N}$ the mass matrix in the space $V_1$ and $\mathcal{W}\in\mathbb{R}^{3N\times3N}$ is defined as
\begin{align}
\mathcal{W}_{ij}:=\hat{\Pi}_{1,\mu}^{i_\mu}\left[\frac{1}{\sqrt{g}}G_{\mu k}\Lambda_{jk}^2\right].
\end{align}
The term involving the hot ion charge density must be computed from the particle-in-cell approximation of the hot ion distribution function. Hence we can generally make use of a control variate $M_0$ for variance reduction:
\begin{align}
&\left((\ast\rho^3_\text{h})\wedge(\ast i_{\lozenge\tilde{U}^2}B^2),C^2\right)=\int_{\hat{\Omega}}\frac{1}{g}\mb{C}^\top G\sqrt{g}G^{-1}\left(\mathring{\rho_\mr{h}}\mb{B}\times \frac{1}{\sqrt{g}}\tilde{\mb{U}}\right)\sqrt{g}\,\mr{d}^3\xi\\ 
=&\int_{\hat{\Omega}}\frac{1}{g}\mb{C}^\top \left[\int(\mathring{f}_\mr{h}^0-\mathring{M}_0)\,\mr{d}^3v\,\mb{B}\times\tilde{\mb{U}}\right]\sqrt{g}\,\mr{d}^3\xi+\int_{\hat{\Omega}}\frac{1}{g}\mb{C}^\top\left(\int \mathring{M}_0\,\mr{d}^3v\,\mb{B}\times\tilde{\mb{U}}\right)\sqrt{g}\,\mr{d}^3\xi\\
=&\int_{\hat{\Omega}}\int\left\lbrace\frac{1}{g}\mb{C}^\top\left[\left(\frac{f_\mr{h}^0-M_0}{g_\text{h}}\right)\,\mb{B}\times\tilde{\mb{U}}\right]\right\rbrace g_\text{h}\,\mr{d}^3v\,\mr{d}^3\xi+\int_{\hat{\Omega}}\frac{1}{\sqrt{g}}\mb{C}^\top\left(\mathring{\rho}_{\text{h0}}\,\mb{B}\times\tilde{\mb{U}}\right)\,\mr{d}^3\xi.
\end{align}
In the last line we introduced the probability density $g_\text{h}$ which satisfies the Vlasov equation and which is normalized to one. This allows us interpret the expression in the curly brackets of the first term as a random variable and we can write down the Monte Carlo estimate 
\begin{align}
&\left((\ast\rho^3_\text{h})\wedge(\ast i_{\lozenge\tilde{U}^2}B^2),C^2\right)\\
\approx &\sum_k\underbrace{\frac{1}{N_k}\left(\frac{f_\mr{h}^0(\boldsymbol{\xi}_k^0,\mb{v}_k^0)}{g_\mr{h}^0(\boldsymbol{\xi}_k^0,\mb{v}_k^0)}-\frac{M_0(\boldsymbol{\xi_k},\mb{v}_k)}{g_\mr{h}^0(\boldsymbol{\xi}_k^0,\mb{v}_k^0)}\right)}_{=:w_k(t)}\frac{1}{g(\boldsymbol{\xi}_k)}\mb{C}_h^\top(\boldsymbol{\xi}_k)(\mb{B}_h(\boldsymbol{\xi}_k)\times\tilde{\mb{U}}_h(\boldsymbol{\xi}_k))\\
&\hspace{8cm}+\mb{c}^\top\int_{\hat{\Omega}}\frac{1}{\sqrt{g}}\mathbb{\Lambda}^2\mathring{\rho}_{\text{h0}}\,\mathbb{B}(\mathbb{\Lambda}^2)^\top\,\mr{d}^3\xi\,\mb{u}\\
= &\sum_k \frac{1}{g(\boldsymbol{\xi}_k)} w_k\mb{C}_h^\top(\boldsymbol{\xi}_k)(\mb{B}_h(\boldsymbol{\xi}_k)\times\tilde{\mb{U}}_h(\boldsymbol{\xi}_k))+\mb{c}^\top \mathbb{X}\mb{u}\\
=&\,\,\mb{c}^\top\mathbb{P}^2\bar{\mathbb{g}}^{-1}\mathbb{W}\bar{\mathbb{B}}\mathbb{P}^{2\top}\mb{u}+\mb{c}^\top \mathbb{X}(\mb{b})\mb{u},
\end{align}
where both terms are clearly anti-symmetric because $\mathbb{B}$ is anti-symmetric. In the last line we introduced the anti-symmetric block matrix $\bar{\mathbb{B}}\in\mathbb{R}^{3N_\text{p}\times3N_\text{p}}$ given by
\begin{align}
\bar{\mathbb{B}}=\bar{\mathbb{B}}(\mb{b},\boldsymbol{\Xi})=\begin{pmatrix}
0 &-\text{diag}[\mathbb{P}^{2\top}_3(\boldsymbol{\Xi})\mb{b}_3] &\text{diag}[\mathbb{P}^{2\top}_2(\boldsymbol{\Xi})\mb{b}_2] \\
\text{diag}[\mathbb{P}^{2\top}_3(\boldsymbol{\Xi})\mb{b}_3] &0 &-\text{diag}[\mathbb{P}^{2\top}_1(\boldsymbol{\Xi})\mb{b}_1] \\
-\text{diag}[\mathbb{P}^{2\top}_2(\boldsymbol{\Xi})\mb{b}_2] &\text{diag}[\mathbb{P}^{2\top}_1(\boldsymbol{\Xi})\mb{b}_1] &0
\end{pmatrix}+\bar{\mathbb{B}}_\text{eq}(\mb{\Xi}),
\end{align}
where $N_\text{p}$ is the number of particles and $\boldsymbol{\Xi}\in\mathbb{R}^{3N_\text{p}}$ is a vector holding all particle positions. Moreover,
\begin{itemize}
\item $\mathbb{P}^2_{\mu,ik}=\Lambda^2_{\mu,i}(\boldsymbol{\xi}_k)\in\mathbb{R}^{N\times N_p}$,
\item $\mathbb{P}^2=\text{diag}(\mathbb{P}^2_1,\,\mathbb{P}^2_2,\,\mathbb{P}^2_3)\in\mathbb{R}^{3N\times3N_\text{p}}$,
\item $\mathbb{g}^{-1}=\mathbb{I}_3\otimes\text{diag}(1/g(\boldsymbol{\xi}_1),\ldots,1/g(\boldsymbol{\xi}_{N_\text{p}}))\in\mathbb{R}^{3N_\text{p}\times3N_\text{p}}$,
\item $\mathbb{W}=\mathbb{I}_3\otimes\text{diag}(w_1,\ldots,w_{N_\text{p}})\in\mathbb{R}^{3N_\text{p}\times3N_\text{p}}$.
\end{itemize}
A similar procedure for the term involving the hot ion current density yields
\begin{align}
&\left(\ast i_{\lozenge j_\mr{h}^2}B^2,C^2\right)=\int_{\hat{\Omega}}\frac{1}{g}\mb{C}^\top G\sqrt{g}G^{-1}\left(\mb{B}\times \frac{1}{\sqrt{g}}\mb{j}_\mr{h}\right)\sqrt{g}\,\mr{d}^3\xi\\
=&\int_{\hat{\Omega}}\frac{1}{g}\mb{C}^\top\sqrt{g}\left[\mb{B}\times DF^{-1}\int (\mathring{f}_\mr{h}-\mathring{M}_0)\mb{v}\,\mr{d}^3v\right]\sqrt{g}\,\mr{d}^3\xi\\
&\qquad\qquad+\int_{\hat{\Omega}}\frac{1}{g}\mb{C}^\top \sqrt{g}\left[\mb{B}\times DF^{-1}\int \mathring{M}_0\mb{v}\,\mr{d}^3v\right]\sqrt{g}\,\mr{d}^3\xi\\
=&\int_{\hat{\Omega}}\int\frac{1}{\sqrt{g}}\mb{C}^\top\left[\mb{B}\times DF^{-1}\left(\frac{f_\mr{h}-M_0}{g_\mr{h}}\right)\mb{v}\right]g_\mr{h}\,\mr{d}^3v\,\mr{d}^3\xi+\int_{\hat{\Omega}}\mb{C}^\top\left[\mb{B}\times DF^{-1}\mathring{\mb{j}}_{\text{h0}}\right]\,\mr{d}^3\xi\\
\approx&\sum_k\frac{1}{\sqrt{g(\boldsymbol{\xi}_k)}}w_k\mb{C}^\top_h(\boldsymbol{\xi}_k)(\mb{B}_h(\boldsymbol{\xi}_k)\times DF^{-1}(\boldsymbol{\xi}_k)\mb{v}_k)+\mb{c}^\top\int_{\hat{\Omega}}\mathbb{\Lambda}^2\left(\mb{B}_h\times DF^{-1}\mathring{\mb{j}}_\mr{h0}\right)\,\mr{d}^3\xi\\
=&\,\,\mb{c}^\top\mathbb{P}^2\mathbb{g}^{-1/2}\mathbb{W}\bar{\mathbb{B}}\bar{DF}^{-1}\mb{V}+\mb{c}^\top\mb{x}(\mb{b})\quad\forall\,\mb{c}\in\mathbb{R}^{3N}.
\end{align}
Here, $\mb{V}=(v_{1x},\ldots,v_{N_\text{p}x},v_{1y},\ldots,v_{N_\text{p}y},v_{1z},\ldots,v_{N_\text{p}z})\in\mathbb{R}^{3N_\text{p}}$ is the vector holding all particle velocities and the block matrix $(\bar{DF}^{-1})_{ab}=\text{diag}\left[(DF^{-1})_{ab}(\boldsymbol{\Xi})\right]$ ($1\leq a,b\leq3$). The second term can be computed numerically for some given magnetic field. In total we get the following semi-discrete momentum balance equation:
\begin{align}
\mathcal{A}\dot{\mb{u}}=\mathcal{T}^\top\mathbb{C}^\top\mathbb{M}^2\mb{b}+\mathcal{P}_2^\top\mathbb{M}^2\mathbb{C}\mathcal{P}_1\mb{b}-\mathcal{W}^\top\mathbb{M}^1\mathbb{G}\mb{p}-\mathbb{P}^2\bar{\mathbb{g}}^{-1}\mathbb{W}\bar{\mathbb{B}}\mathbb{P}^{2\top}\mb{u}-\mathbb{X}\mb{u}+\mathbb{P}^2\bar{\mathbb{g}}^{-1/2}\mathbb{W}\bar{\mathbb{B}}\bar{DF}^{-1}\mb{V}+\mb{x}.
\end{align}
Not that recovering the full-$f$ case just means choosing the control variate $\mathcal{M}_0=0$ which results in $\mathbb{X}=\mb{x}=0$.
 
\subsection{Induction equation}
Like the mass continuity equation, we keep the induction equation in strong form. Hence we apply the projector $\Pi_2$ and once more use the diagram's commutativity to exchange projectors and exterior derivatives:
\begin{align}
\frac{\pa(\Pi_2\tilde{B}^2)}{\pa t}+\Pi_2\text{d}(i_{\lozenge U^2}B_\text{eq}^2)=\frac{\pa(\Pi_2\tilde{B}^2)}{\pa t}+\text{d}\Pi_1(i_{\lozenge U^2}B_\text{eq}^2).
\end{align}
In terms of the components of the forms, this amounts to
\begin{align}
&\frac{\pa \tilde{\mb{B}}_h}{\pa t}+\hat{\nabla}\times\Pi_1\left[\mb{B}_\text{eq}\times\frac{1}{\sqrt{g}}\tilde{\mb{U}}_h\right]=0\\
\Leftrightarrow\quad&\frac{\pa \tilde{\mb{B}}_h}{\pa t}+\hat{\nabla}\times\Pi_1\left[\mathbb{B}_\text{eq}\frac{1}{\sqrt{g}}(\mathbb{\Lambda}^2)^\top\right]\mb{u}=0\\
\Leftrightarrow\quad &\frac{\text{d}\mb{b}}{\text{d} t}+\mathbb{C}\mathcal{T}\mb{u}=0.
\end{align}
We immediately see that we obtain the same projection matrix $\mathcal{T}$ as for the Hall term in the momentum equation.

\subsection{Pressure equation}
We once more choose a weak formulation for the pressure equation. Hence we take the inner product with a test function $r^0\in H\Lambda^0(\Omega)$ which yields the variational formulation: find $\tilde{p}^0\in H\Lambda^0(\Omega)$ such that
\begin{align}
&\left(\frac{\pa\tilde{p}^0}{\pa t},r^0\right)-(\text{d}^\ast(p_\text{eq}^0\wedge\ast\tilde{U}^2),r^0)-(\gamma-1)(\text{d}^\ast(\ast\tilde{U}^2),p^0_\text{eq}\wedge r^0)=0\quad\forall\,r^0\in H\Lambda^0(\Omega)\\
\Leftrightarrow\quad&\left(\frac{\pa\tilde{p}^0}{\pa t},r^0\right)-(p_\text{eq}^0\wedge\ast\tilde{U}^2,\text{d}r^0)-(\gamma-1)(\ast\tilde{U}^2,\text{d}(p^0_\text{eq}\wedge r^0))=0\quad\forall\,r^0\in H\Lambda^0(\Omega),
\end{align}
if we again assume all boundary terms to vanish. For the first term we simply get
\begin{align}
\left(\frac{\pa\tilde{p}^0}{\pa t},r^0\right)=\int_{\hat{\Omega}}\frac{\pa\tilde{p}}{\pa t}r\sqrt{g}\text{d}^3\xi\approx\dot{\mb{p}}^\top\underbrace{\int_{\hat{\Omega}}\boldsymbol{\Lambda}^0(\boldsymbol{\Lambda}^0)^\top\sqrt{g}\,\text{d}^3\xi}_{=:\mathbb{M}^0}\mb{r}=\dot{\mb{p}}^\top\mathbb{M}^0\mb{r}\quad\forall\,\mb{r}\in\mathbb{R}^N,
\end{align}
where $\mathbb{M}^0\in\mathbb{R}^{N\times N}$ is the mass matrix in the space $V_0$. The second term amount to
\begin{align}
(p_\text{eq}^0\wedge\ast\tilde{U}^2,\text{d}r^0)&=\int_{\hat{\Omega}}p_\text{eq}\frac{1}{\sqrt{g}}(G\tilde{\mb{U}})^\top G^{-1}\hat{\nabla}r\sqrt{g}\,\text{d}^3\xi\\
&=\mb{u}^\top\hat{\Pi}_1\left[\frac{p_\text{eq}}{\sqrt{g}}\mathbb{\Lambda}^2G\right]\underbrace{\int_{\hat{\Omega}}\mathbb{\Lambda}^1G^{-1}(\mathbb{\Lambda}^1)^\top\sqrt{g}\,\text{d}^3\xi}_{=:\mathbb{M}^1}\,\mathbb{G}\mb{r}=\mb{u}^\top\mathcal{S}^\top\mathbb{M}^1\mathbb{G}\mb{r}\quad\forall\,\mb{r}\in\mathbb{R}^N,
\end{align} 
where $\mathbb{M}^1\in\mathbb{R}^{3N\times 3N}$ is the mass matrix in the space $V_1$, $\mathbb{G}\in\mathbb{R}^{3N\times N}$ is the discrete gradient matrix and the projection matrix $\mathcal{S}\in\mathbb{R}^{3N\times 3N}$ is defined as
\begin{align}
\mathcal{S}_{ij}:=\hat{\Pi}_{1,\mu}^{i_\mu}\left[p_\text{eq}\frac{G_{\mu k}\Lambda_{jk}^2}{\sqrt{g}}\right].
\end{align}
Finally, the third term yields
\begin{align}
(\ast\tilde{U}^2,\text{d}(p^0_\text{eq}\wedge r^0))&=\int_{\hat{\Omega}}\frac{1}{\sqrt{g}}(G\tilde{\mb{U}})^\top G^{-1}\hat{\nabla}(p_\text{eq}r)\sqrt{g}\,\text{d}^3\xi\\
&=\mb{u}^\top\hat{\Pi}_1\left[\frac{1}{\sqrt{g}}\mathbb{\Lambda}^2G\right]\int_{\hat{\Omega}}\mathbb{\Lambda}^1G^{-1}(\mathbb{\Lambda}^1)^\top\sqrt{g}\,\text{d}^3\xi\,\mathbb{G}\,\hat{\Pi}_0\left[p_\text{eq}(\boldsymbol{\Lambda}^0)^\top\right]\mb{r}\\
&=\mb{u}^\top\mathcal{W}^\top\mathbb{M}^1\mathbb{G}\mathcal{K}\mb{r}\quad\forall\,\mb{r}\in\mathbb{R}^N.
\end{align} 
The projection matrix $\mathcal{W}$ already appeared in the momentum equation, while the new projection matrix $\mathcal{K}\in\mathbb{R}^{N\times N}$ is given by
\begin{align}
\mathcal{K}_{ij}:=\hat{\Pi}_0^i\left[p_\text{eq}\Lambda^0_j\right].
\end{align}
Summing up all terms yields the following semi-discrete pressure equation:
\begin{align}
&\dot{\mb{p}}^\top\mathbb{M}^0\mb{r}-\mb{u}^\top\mathcal{S}^\top\mathbb{M}^1\mathbb{G}\mb{r}-(\gamma - 1)\mb{u}^\top\mathcal{W}^\top\mathbb{M}^1\mathbb{G}\mathcal{K}\mb{r}=0\quad\forall\,\mb{r}\in\mathbb{R}^N\\
\Leftrightarrow\quad &\mathbb{M}^0\dot{\mb{p}}=\mathbb{G}^\top\mathbb{M}^1\mathcal{S}\mb{u}+(\gamma-1)\mathcal{K}^\top\mathbb{G}^\top\mathbb{M}^1\mathcal{W}\mb{u}.
\end{align}
\subsection{Particles' equation of motion}
As a last step, we derive the equations of motion for a single particle with logical coordinate $\mb{\xi}_k$ and physical velocity $\mb{v}_k$. For the latter, we have to transform the forms $B^2$ and $\tilde{U}^2$ to physical coordinates:
\begin{align}
\frac{\mr{d}\boldsymbol{\xi}_k}{\mr{d}t}&=DF^{-1}(\boldsymbol{\xi}_k)\mb{v}_k,\\
\frac{\mr{d}\mb{v}_k}{\mr{d}t}&=\frac{1}{\sqrt{g(\boldsymbol{\xi}_k)}}DF(\boldsymbol{\xi}_k)\mb{B}_h(\boldsymbol{\xi}_k)\times\frac{1}{\sqrt{g(\boldsymbol{\xi}_k)}}DF(\boldsymbol{\xi}_k)\tilde{\mb{U}}_h(\boldsymbol{\xi}_k)-\frac{1}{\sqrt{g(\boldsymbol{\xi}_k)}}DF(\boldsymbol{\xi}_k)\mb{B}_h(\boldsymbol{\xi}_k)\times\mb{v}_k\\
&=\frac{1}{\sqrt{g(\boldsymbol{\xi}_k)}}DF^{-\top}(\boldsymbol{\xi}_k)(\mb{B}_h(\boldsymbol{\xi}_k)\times \tilde{\mb{U}}_h(\boldsymbol{\xi}_k))-DF^{-\top}(\boldsymbol{\xi}_k)(\mb{B}_h(\boldsymbol{\xi}_k)\times DF^{-1}(\boldsymbol{\xi}_k)\mb{v}_k),
\end{align}
where we used the identity $A\mb{b}\times A\mb{c}=\det(A)A^{-\top}(\mb{b}\times\mb{c})$. Writing the above equation of motion in matrix-vector form for all particles yields
\begin{align}
&\frac{\mr{d}\boldsymbol{\Xi}}{\mr{d}t}=\tilde{DF}^{-1}\mb{V},\\
&\frac{\mr{d}\mb{V}}{\mr{d}t}=\mathbb{g}^{-1/2}\tilde{DF}^{-\top}\mathbb{B}\mathbb{P}^{2\top}\mb{u}-\tilde{DF}^{-\top}\mathbb{B}\tilde{DF}^{-1}\mb{V}.
\end{align}

\subsection{Hamiltonian system}
Let us try to write the semi-discrete system of equation in Hamiltonian form. For this, we define the discrete Hamiltonian
\begin{align}
\mathcal{H}_h:=&\frac{1}{2}\left((\ast\rho_\text{eq}^3)\wedge\tilde{U}_h^2,\tilde{U}_h^2\right)+\frac{1}{2}(\tilde{B}_h^2,\tilde{B}_h^2)+\frac{1}{\gamma-1}(\tilde{p}^0,1)+\frac{1}{2}\int_{\hat{\Omega}}\epsilon_{\text{h},123}\,\text{d}^3\xi\\
=&\frac{1}{2}\mb{u}^\top\mathcal{A}\mb{u}+\frac{1}{2}\mb{b}^\top\mathbb{M}^2\mb{b}+\frac{1}{\gamma-1}\mb{p}^\top\mb{n}+\frac{1}{2}\mb{V}^\top\mathbb{W}\mb{V}+\epsilon_{\text{h}0}:=\mathcal{H}_{h1}+\frac{1}{\gamma-1}\mb{p}^\top\mb{n}+\epsilon_{\text{h}0}.
\end{align}
With this we can write the semi-discrete system in the following form:
\begin{align}
\frac{\mr{d}}{\mr{d}t}\begin{pmatrix}
\boldsymbol{\rho}  \\ \mb{u} \\ \mb{b} \\ \mb{p} \\ \boldsymbol{\Xi} \\ \mb{V}
\end{pmatrix}&=
\begin{pmatrix}
0 &0 &0 &0 &0 &0 \\
0 &\textcolor{red}{\mathbb{J}_{11}(\mb{b},\boldsymbol{\Xi})} &\textcolor{blue}{\mathbb{J}_{12}} &0 &0 &\textcolor{green}{\mathbb{J}_{14}(\mb{b},\boldsymbol{\Xi})}\\
0 &\textcolor{blue}{-\mathbb{J}_{12}^\top} &0 &0 &0 &0 \\
0 &0 &0 &0 &0 &0 \\
0 &0 &0 &0 &0 &\textcolor{magenta}{\mathbb{J}_{34}(\boldsymbol{\Xi})} \\
0 &\textcolor{green}{-\mathbb{J}_{14}^\top(\mb{b},\boldsymbol{\Xi})} &0 &0 &\textcolor{magenta}{-\mathbb{J}_{34}^\top(\boldsymbol{\Xi})} &\mathbb{J}_{44}(\mb{b},\boldsymbol{\Xi})
\end{pmatrix}\begin{pmatrix}
0\\
\mathcal{A}\mb{u} \\
\mathbb{M}^2\mb{b}\\
0 \\
0 \\
\mathbb{W}\mb{V}
\end{pmatrix}\\[2mm]
&+\begin{pmatrix}
-\mathbb{D}\mathcal{Q}\mb{u} \\
\mathcal{A}^{-1}\mathcal{P}_2^\top\mathbb{M}^2\mathbb{C}\mathbb{P}_1\mb{b}-\mathcal{A}^{-1}\mathcal{W}^\top\mathbb{M}^1\mathbb{G}\mb{p}-\mathcal{A}^{-1}\mathbb{X}\mb{u}+\mathcal{A}^{-1}\mb{x}\\
0\\
(\mathbb{M}^0)^{-1}\mathbb{G}\top\mathbb{M}^1\mathcal{S}\mb{u}+(\gamma-1)\mathcal{K}^\top\mathbb{G}^\top\mathbb{M}^1\mathcal{W}\mb{u}\\
0\\
0
\end{pmatrix}.
\end{align}
We see that we can write the system as the sum of a non-canonical Hamiltonian part with the Hamiltonian $\mathcal{H}_{h1}$ and an additional non-Hamiltonian part. The latter vanishes if we you use the full-$f$ description, if $\nabla\times\mb{B}_\text{eq}=0$ and if there are no pressure and density perturbations. The blocks in the Poisson matrix are given by 
\begin{align}
&\textcolor{red}{\mathbb{J}_{11}(\mb{b},\mb{Q})}=-\mathcal{A}^{-1}\mathbb{P}_1(\mb{Q})\mathbb{W}\tilde{G}^{-1}(\mb{Q})\hat{\mathbb{B}}(\mb{b},\mb{Q})\tilde{G}^{-1}(\mb{Q})\mathbb{P}_1^\top(\mb{Q})\mathcal{A}^{-1},\\
&\textcolor{blue}{\mathbb{J}_{12}}=\mathcal{A}^{-1}\mathcal{T}^\top\mathbb{C}^\top,\\
&\textcolor{green}{\mathbb{J}_{14}(\mb{b},\mb{Q})}=\mathcal{A}^{-1}\mathbb{P}_1(\mb{Q})\tilde{G}^{-1}(\mb{Q})\hat{\mathbb{B}}(\mb{b},\mb{Q})\tilde{DF}^{-1}(\mb{Q}),\\
&\textcolor{magenta}{\mathbb{J}_{34}(\mb{Q})}=\tilde{DF}^{-1}(\mb{Q})\mathbb{W}^{-1},\\
&\mathbb{J}_{44}(\mb{Q})=-\tilde{DF}^{-\top}(\mb{Q})\hat{\mathbb{B}}(\mb{b},\mb{Q})\tilde{DF}^{-1}(\mb{Q})\mathbb{W}^{-1}.
\end{align}

\section{Time integration}
\subsection{Poisson splitting}
We split the Poisson matrix into antisymmetric sub-systems such that each sub-system again defines a Hamiltonian system
\begin{align}
&\frac{\mr{d}}{\mr{d}t}\begin{pmatrix}
\mb{u} \\ \mb{b} \\ \mb{Q} \\ \mb{V}
\end{pmatrix}=
\begin{pmatrix}
\textcolor{red}{\mathbb{J}_{11}(\mb{b},\mb{Q})} &\textcolor{blue}{\mathbb{J}_{12}} &0 &\textcolor{green}{\mathbb{J}_{14}(\mb{b},\mb{Q})}\\
\textcolor{blue}{-\mathbb{J}_{12}^\top} &0 &0 &0 \\
0 &0 &0 &\textcolor{magenta}{\mathbb{J}_{34}(\mb{Q})} \\
\textcolor{green}{-\mathbb{J}_{14}^\top(\mb{b},\mb{Q})} &0 &\textcolor{magenta}{-\mathbb{J}_{34}^\top(\mb{Q})} &\mathbb{J}_{44}(\mb{b},\mb{Q})
\end{pmatrix}\begin{pmatrix}
\mathcal{A}\mb{u} \\
\mathbb{M}^2\mb{b}\\
0 \\
\mathbb{W}\mb{V}
\end{pmatrix},\\[1cm]
&\textcolor{red}{\mathbb{J}_{11}(\mb{b},\mb{Q})}=-\mathcal{A}^{-1}\mathbb{P}_1(\mb{Q})\mathbb{W}\tilde{G}^{-1}(\mb{Q})\hat{\mathbb{B}}(\mb{b},\mb{Q})\tilde{G}^{-1}(\mb{Q})\mathbb{P}_1^\top(\mb{Q})\mathcal{A}^{-1},\\
&\textcolor{blue}{\mathbb{J}_{12}}=\mathcal{A}^{-1}\mathcal{T}^\top\mathbb{C}^\top,\\
&\textcolor{green}{\mathbb{J}_{14}(\mb{b},\mb{Q})}=\mathcal{A}^{-1}\mathbb{P}_1(\mb{Q})\tilde{G}^{-1}(\mb{Q})\hat{\mathbb{B}}(\mb{b},\mb{Q})\tilde{DF}^{-1}(\mb{Q}),\\
&\textcolor{magenta}{\mathbb{J}_{34}(\mb{Q})}=\tilde{DF}^{-1}(\mb{Q})\mathbb{W}^{-1},\\
&\mathbb{J}_{44}(\mb{Q})=-\tilde{DF}^{-\top}(\mb{Q})\hat{\mathbb{B}}(\mb{b},\mb{Q})\tilde{DF}^{-1}(\mb{Q})\mathbb{W}^{-1}.
\end{align}
\paragraph{Sub-step 1}
The first sub-system reads
\begin{align}
&\dot{\mb{u}}=\mathbb{J}_{11}(\mb{b},\mb{Q})\mathcal{A}\mb{u},\\
&\dot{\mb{b}}=0,\\
&\dot{\mb{Q}}=0,\\
&\dot{\mb{V}}=0.
\end{align}
We solve the this equation with the energy-preserving Crank-Nicolson method:
\begin{align}
&\frac{\mb{u}^{n+1}-\mb{u}^n}{\Delta t}=\mathbb{J}_{11}(\mb{b}^n,\mb{Q}^n)\mathcal{A}\frac{\mb{u}^n+\mb{u}^{n+1}}{2}.\\
\Leftrightarrow \quad &\left(\mathbb{I}-\frac{\Delta t}{2}\mathbb{J}_{11}(\mb{b}^n,\mb{Q}^n)\mathcal{A}\right)\mb{u}^{n+1}=\left(\mathbb{I}+\frac{\Delta t}{2}\mathbb{J}_{11}(\mb{b}^n,\mb{Q}^n)\mathcal{A}\right)\mb{u}^n.
\end{align}
To avoid multiple matrix inversions, we multiply everything with $\mathcal{A}$ from the left to obtain 
\begin{align}
\left(\mathcal{A}-\frac{\Delta t}{2}\mathcal{A}\mathbb{J}_{11}(\mb{b}^n,\mb{Q}^n)\mathcal{A}\right)\mb{u}^{n+1}=\left(\mathcal{A}+\frac{\Delta t}{2}\mathcal{A}\mathbb{J}_{11}(\mb{b}^n,\mb{Q}^n)\mathcal{A}\right)\mb{u}^n.
\end{align}
We denote the corresponding integrator by $\Phi_{\Delta t}^1$.
\paragraph{Sub-step 2}
The second sub-system reads
\begin{align}
&\dot{\mb{u}}=\mathbb{J}_{12}\mathbb{M}^2\mb{b},\\
&\dot{\mb{b}}=-\mathbb{J}_{12}^\top\mathcal{A}\mb{u},\\
&\dot{\mb{Q}}=0,\\
&\dot{\mb{V}}=0.
\end{align}
We again solve this system with an energy-preserving Crank-Nicolson method:
\begin{align}
\begin{pmatrix}
\mathcal{A} &-\frac{\Delta t}{2}\mathcal{T}^\top\mathbb{C}^\top\mathbb{M}^2 \\
\frac{\Delta t}{2}\mathbb{C}\mathcal{T} &\mathbb{I}
\end{pmatrix}
\begin{pmatrix}
\mb{u}^{n+1}\\ \mb{b}^{n+1}
\end{pmatrix}=
\begin{pmatrix}
\mathcal{A} &\frac{\Delta t}{2}\mathcal{T}^\top\mathbb{C}^\top\mathbb{M}^2 \\
-\frac{\Delta t}{2}\mathbb{C}\mathcal{T} &\mathbb{I}
\end{pmatrix}\begin{pmatrix}
\mb{u}^{n}\\ \mb{b}^{n}
\end{pmatrix}.
\end{align}
Using the Schur complement $S:=\mathcal{A}+\frac{\Delta t^2}{4}\mathcal{T}^\top\mathbb{C}^\top\mathbb{M}^2\mathbb{C}\mathcal{T}$, we can calculate the in inverse of the matrix on the left-hand side which we can then multiply with the matrix on the right-hand side:
\begin{align}
&\begin{pmatrix} \mathcal{S}^{-1} &\frac{\Delta t}{2}\mathcal{S}^{-1}\mathcal{T}^\top\mathbb{C}^\top\mathbb{M}^2\\
-\frac{\Delta t}{2}\mathbb{C}\mathcal{T}S^{-1} &\mathbb{I}-\frac{\Delta t^2}{4}\mathbb{C}\mathcal{T}\mathcal{S}^{-1}\mathcal{T}^\top
\mathbb{C}^\top\mathbb{M}^2 
\end{pmatrix}\begin{pmatrix}
\mathcal{A} &\frac{\Delta t}{2}\mathcal{T}^\top\mathbb{C}^\top\mathbb{M}^2 \\
-\frac{\Delta t}{2}\mathbb{C}\mathcal{T} &\mathbb{I}
\end{pmatrix}=\\[2mm]
=&\begin{pmatrix}
\mathcal{S}^{-1}\mathcal{A}-\frac{\Delta t^2}{4}\mathcal{S}^{-1}\mathcal{T}^\top\mathbb{C}^\top\mathbb{M}^2\mathbb{C}\mathcal{T} &\Delta t\mathcal{S}^{-1}\mathcal{T}^\top\mathbb{C}^\top\mathbb{M}^2\\
-\frac{\Delta t}{2}\mathbb{C}\mathcal{T}S^{-1}\mathcal{A}-\frac{\Delta t}{2}\mathbb{C}\mathcal{T}+\frac{\Delta t^3}{8}\mathbb{C}\mathcal{T}\mathcal{S}^{-1}\mathcal{T}^\top
\mathbb{C}^\top\mathbb{M}^2\mathbb{C}\mathcal{T} &\mathbb{I}-\frac{\Delta t^2}{2}\mathbb{C}\mathcal{T}\mathcal{S}^{-1}\mathcal{T}^\top\mathbb{C}^\top\mathbb{M}^2
\end{pmatrix}\\
&\Rightarrow\quad\mb{u}^{n+1}=\mathcal{S}^{-1}\left[\left(\mathcal{A}-\frac{\Delta t^2}{4}\mathcal{T}^\top\mathbb{C}^\top\mathbb{M}^2\mathbb{C}\mathcal{T}\right)\mb{u}^{n}+\Delta t\mathcal{T}^\top\mathbb{C}^\top\mathbb{M}^2\mb{b}^n\right]\\
&\Rightarrow\quad\mb{b}^{n+1}=\mb{b}^n-\frac{\Delta t}{2}\mathbb{C}\mathcal{T}\left(\mb{u}^n+\mathcal{S}^{-1}\mathcal{A}\mb{u}^n-\frac{\Delta t^2}{2}\mathcal{S}^{-1}\mathcal{T}^\top\mathbb{C}^\top\mathbb{M}^2\mathbb{C}\mathcal{T}\mb{u}^n+\Delta t\mathcal{S}^{-1}\mathcal{T}^\top\mathbb{C}^\top\mathbb{M}^2\mb{b}^n\right)\\
&=\mb{b}^n-\frac{\Delta t}{2}\mathbb{C}\mathcal{T}(\mb{u}^n+\mb{u}^{n+1})
\end{align}
We immediately see that the update for $\mb{b}$ preserves the divergence-free constraint. We denote the corresponding integrator by $\Phi_{\Delta t}^2$.
\paragraph{Sub-step 3}
The third sub-system reads
\begin{align}
&\dot{\mb{u}}=\mathbb{J}_{14}(\mb{b},\mb{Q})\mathbb{W}\mb{V},\\
&\dot{\mb{b}}=0,\\
&\dot{\mb{Q}}=0,\\
&\dot{\mb{V}}=-\mathbb{J}_{14}^\top(\mb{b},\mb{Q})\mathcal{A}\mb{u}.
\end{align}
We solve this system in the same way as before. Since $\mb{b}$ and $\mb{Q}$ do not change in this step, the same is true for the matrix $\mathbb{J}_{14}$. Hence $\mathbb{J}_{14}=\mathbb{J}_{14}(\mb{b}^n,\mathbb{Q}^n)$ and we have
\begin{align}
\begin{pmatrix}
\mathcal{A} &-\frac{\Delta t}{2}\mathcal{A}\mathbb{J}_{14}\mathbb{W} \\
\frac{\Delta t}{2}\mathbb{J}_{14}^\top\mathcal{A} &\mathbb{I}
\end{pmatrix}
\begin{pmatrix}
\mb{u}^{n+1}\\ \mb{V}^{n+1}
\end{pmatrix}=
\begin{pmatrix}
\mathcal{A} &\frac{\Delta t}{2}\mathcal{A}\mathbb{J}_{14}\mathbb{W} \\
-\frac{\Delta t}{2}\mathbb{J}_{14}^\top\mathcal{A} &\mathbb{I}
\end{pmatrix}\begin{pmatrix}
\mb{u}^{n}\\ \mb{V}^{n}
\end{pmatrix}.
\end{align}
Using once more the Schur complement $\mathcal{S}^{-1}:=\mathcal{A}+\frac{\Delta t^2}{4}\mathcal{A}\mathbb{J}_{14}\mathbb{W}\mathbb{J}_{14}^\top\mathcal{A}$ yields
\begin{align}
&\mb{u}^{n+1}=\mathcal{S}^{-1}\left[\left(\mathcal{A}-\frac{\Delta t^2}{4}\mathcal{A}\mathbb{J}_{14}\mathbb{W}\mathbb{J}_{14}^\top\mathcal{A}\right)\mb{u}^{n}+\Delta t\mathcal{A}\mathbb{J}_{14}\mathbb{W}\mb{V}^n\right],\\
&\mb{V}^{n+1}=\mb{V}^n-\frac{\Delta t}{2}\mathbb{J}_{14}^\top\mathcal{A}(\mb{u}^n+\mb{u}^{n+1})
\end{align}
We denote the corresponding integrator by $\Phi_{\Delta t}^3$.
\paragraph{Sub-step 4}
The fourth sub-system reads
\begin{align}
&\dot{\mb{u}}=0,\\
&\dot{\mb{b}}=0,\\
&\dot{\mb{Q}}=\mathbb{J}_{34}(\mb{Q})\mathbb{W}\mb{V},\\
&\dot{\mb{V}}=0.
\end{align}
Using again a Crank-Nicolson approximation for particle $k$ yields
\begin{align}
\mb{q}_k^{n+1}=\mb{q}_k^n+\frac{\Delta t}{2}(DF^{-1}(\mb{q}_k^n)+DF^{-1}(\mb{q}_k^{n+1}))\mb{v}_k^n,
\end{align}
which can be solved for $\mb{q}^{n+1}$ using a fix point iteration. We denote the corresponding integrator by $\Phi_{\Delta t}^4$.
\paragraph{Sub-step 5}
The fifth sub-system reads
\begin{align}
&\dot{\mb{u}}=0,\\
&\dot{\mb{b}}=0,\\
&\dot{\mb{Q}}=0,\\
&\dot{\mb{V}}=\mathbb{J}_{44}(\mb{b},\mb{Q})\mathbb{W}\mb{V}.
\end{align}
Using again a Crank-Nicolson approximation for particle $k$ yields
\begin{align}
\left(\mathbb{I}+\frac{\Delta t}{2}DF^{-\top}(\mb{q}_k^n)\hat{\mathbb{B}}_h(\mb{q}_k^n)DF^{-1}(\mb{q}_k^n)\right)\mb{v}_k^{n+1}=\left(\mathbb{I}-\frac{\Delta t}{2}DF^{-\top}(\mb{q}_k^n)\hat{\mathbb{B}}_h(\mb{q}_k^n)DF^{-1}(\mb{q}_k^n)\right)\mb{v}_k^{n}.
\end{align}
We denote the corresponding integrator by $\Phi_{\Delta t}^5$.

\section{Dispersion relation}
In this section we derive the dispersion relation for wave propagation parallel to an external, uniform magnetic field $\mb{B}=B_0\mb{e}_z$. Although we have $\mb{E}=-\mb{U}\times\mb{B}$ in ideal MHD, we assume for the moment a general electric field. With this, linearization of the Vlasov equation about an homogeneous (in space) equilibrium $f_\mr{h}^0=f_\mr{h}^0(v_\parallel, v_\perp)$ yields
\begin{align}
\frac{\pa f_\mr{h}}{\pa t}+\mb{v}\cdot\nabla f_\mr{h}+\Omega_\mr{ch}(\mb{v}\times\mb{e}_z)\cdot\nabla_\mb{v}f_\mr{h}=-\frac{q_\mr{h}}{m_\mr{h}}(\mb{E}+\mb{v}\times\mb{B})\cdot\nabla_\mb{v}f_\mr{h}^0.
\end{align}
The solution of this equation in Fourier space leads to an Ohm's law of the form
\begin{align}
\hat{\mb{j}}_\mr{h}=\begin{pmatrix}
-i\frac{q_\mr{h}^2}{m_\mr{h}}\int\frac{v_\perp(\omega-kv_\parallel)\hat{G}f_\mr{h}^0}{2\Omega_+\Omega_-}\mr{d}^3v &\frac{q_\mr{h}^2}{m_\mr{h}}\Omega_\mr{ch}\int\frac{v_\perp\hat{G}f_\mr{h}^0}{2\Omega_+\Omega_-}\mr{d}^3v &0 \\
-\frac{q_\mr{h}^2}{m_\mr{h}}\Omega_\mr{ch}\int\frac{v_\perp\hat{G}f_\mr{h}^0}{2\Omega_+\Omega_-}\mr{d}^3v &-i\frac{q_\mr{h}^2}{m_\mr{h}}\int\frac{v_\perp(\omega-kv_\parallel)\hat{G}f_\mr{h}^0}{2\Omega_+\Omega_-}\mr{d}^3v &0 \\
0 &0 &-i\frac{q_\mr{h}^2}{m_\mr{h}}\int\frac{v_\parallel\pa_\parallel f_\mr{h}^0}{\omega-kv_\parallel}\mr{d}^3v
\end{pmatrix}\hat{\mb{E}}=\sigma_\mr{h}(k,\omega)\hat{\mb{E}},
\end{align}
where $\hat{G}=\pa/\pa v_\perp+k(v_\perp\pa/\pa v_\parallel-v_\parallel\pa/\pa v_\perp)/\omega$ is a differential operator measuring the anisotropy of the distribution function and velocity space and $\Omega_\pm=\omega-kv_\parallel\pm\Omega_\mr{ce}$ denote the frequencies of resonant particles. 
We make now use of the fact that $\hat{\mb{E}}=-B_0\hat{\mb{U}}\times\mb{e}_z$ in ideal MHD. This yields a pure transverse current with components
\begin{align}
&\hat{j}_{\mr{h}x}=-\sigma_{\mr{h}xx}B_0U_y+\sigma_{\mr{h}xy}B_0U_x \\
&\hat{j}_{\mr{h}y}=-\sigma_{\mr{h}yx}B_0U_y+\sigma_{\mr{h}yy}B_0U_x.
\end{align}
The momentum balance equation then reads
\begin{align}
&-i\omega\hat{\mb{U}}+i\frac{v_\mr{A}^2k^2}{\omega}\hat{\mb{U}}_\perp+i\frac{c_\mr{S}^2k^2}{\omega}\mb{e}_z\hat{U}_\parallel=\frac{Z_\mr{h}\nu_\mr{h}\Omega_\mr{ch}}{A_\mr{MHD}}\hat{\mb{U}}\times\mb{e}_z+\frac{B_0}{\rho_\mr{eq}}(\mb{e}_z\times\hat{\mb{j}}_\mr{h}),
\end{align}
where we approximated the bulk mass density by the ion contribution $\rho_\mr{eq}=A_\mr{MHD}m_\mr{i}n_0$, $Z_\mr{h}$ is the hot ion charge number and $\nu_\mr{h}=n_\mr{h0}/n_0$ is the the ratio of the equilibrium bulk and energetic ion number densities. Writing everything in terms of the bulk velocity yields the linear system
\begin{align}
\begin{pmatrix}
\omega^2-v_\mr{A}^2k^2+iv_\mr{A}^2\sigma_{\mr{h}yy}\omega &i\left(-\frac{Z_\mr{h}\nu_\mr{h}\Omega_\mr{ch}\omega}{A_\mr{MHD}}-v_\mr{A}^2\sigma_{\mr{h}yx}\omega\right) &0 \\
-i\left(-\frac{Z_\mr{h}\nu_\mr{h}\Omega_\mr{ch}\omega}{A_\mr{MHD}}+v_\mr{A}^2\sigma_{\mr{h}xy}\omega\right) &\omega^2-v_\mr{A}^2k^2+iv_\mr{A}^2\sigma_{\mr{h}xx}\omega &0 \\
0 &0 &\omega^2-c_\mr{S}^2k^2
\end{pmatrix}\begin{pmatrix}
\hat{U}_x \\ \hat{U}_y \\ \hat{U}_\parallel
\end{pmatrix}=0.
\end{align}
The structure of this system immediately reveals that we deal with circularly polarized waves in perpendicular direction characterized by $iU_x/U_y=\pm 1$ for R/L-waves, respectively. This leads to the dispersion relation
\begin{align}
D_\mr{R/L}(k,\omega)&=1-\frac{v_\mr{A}^2k^2}{\omega^2}+\frac{iv_\mr{A}^2\sigma_{\mr{h}xx}}{\omega}\pm\frac{Z_\mr{h}\nu_\mr{h}\Omega_\mr{ch}}{A_\mr{MHD}\omega}\mp\frac{v_\mr{A}^2\sigma_\mr{hxy}}{\omega}\\
&=1-\frac{v_\mr{A}^2k^2}{\omega^2}\pm\frac{Z_\mr{h}\nu_\mr{h}\Omega_\mr{ch}}{A_\mr{MHD}\omega}+\frac{\nu_\mr{h}\Omega_\mr{ch}^2Z_\mr{h}^2}{A_\mr{h}A_\mr{MHD}\omega}\int\frac{v_\perp}{2}\frac{\hat{G}F_\mr{h}^0}{\omega-kv_\parallel\pm\Omega_\mr{ch}}\mr{d}^3v=0.
\end{align}
Assuming a shifted, isotropic Maxwellian of the form
\begin{align}
&F_\mr{h}^0=\frac{1}{\pi^{3/2}v_\mr{th}^3}\exp\left(-\frac{(v_\parallel-v_0)^2+v_\perp^2}{v_\mr{th}^2}\right)\\
&\Rightarrow \frac{\pa F_\mr{h}^0}{\pa v_\parallel}=-F_\mr{h}^0\frac{2}{v_\mr{th}^2}(v_\parallel-v_0)\\
&\Rightarrow \frac{\pa F_\mr{h}^0}{\pa v_\perp}=-F_\mr{h}^0\frac{2}{v_\mr{th}^2}v_\perp\\
&\Rightarrow \hat{G}F_\mr{h}^0=-F_\mr{h}^0\frac{2}{v_\mr{th}^2}v_\perp+\frac{k}{\omega}F_\mr{h}^0\frac{2}{v_\mr{th}^2}v_\perp v_0=F_\mr{h}^0\frac{2v_\perp}{v_\mr{th}^2}\left(\frac{kv_0}{\omega}-1\right)\\
&\Rightarrow \pi\int_0^\infty\frac{1}{\pi^{3/2}v_\mr{th}^3}\begin{pmatrix}
1 \\ v_\perp \\ v_\perp^2 \\ v_\perp^3
\end{pmatrix}\exp\left(-\frac{v_\perp^2}{v_\mr{th}^2}\right)\mr{d}v_\perp=\begin{pmatrix}
1/(2v_\mr{th}^2)\\
1/(2v_\mr{th}\sqrt{\pi}) \\
1/4\\
v_\mr{th}/(2\sqrt{\pi})
\end{pmatrix},
\end{align}
leads to
\begin{align}
D_\mr{R/L}(k,\omega)=1-\frac{v_\mr{A}^2k^2}{\omega^2}\pm\frac{Z_\mr{h}\nu_\mr{h}\Omega_\mr{ch}}{A_\mr{MHD}\omega}+\frac{\nu_\mr{h}\Omega_\mr{ch}^2Z_\mr{h}^2}{A_\mr{h}A_\mr{MHD}\omega^2}\frac{\omega-kv_0}{k v_\mr{th}}Z(\xi^\pm),
\end{align}
where $Z$ is the plasma dispersion function and $\xi\pm=(\omega-kv_0\pm\Omega_\mr{ch})/(kv_\mr{th})$.
\begin{figure}
\centering
\includegraphics[scale=0.5]{R-wave.pdf}
\includegraphics[scale=0.5]{L-wave.pdf}
\caption{Growth rates for different shifts of the Maxwellian}
\end{figure}

\end{document}