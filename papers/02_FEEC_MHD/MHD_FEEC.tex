\documentclass[11pt,oneside,a4paper,fleqn]{article}


% Usepackages
\usepackage[left=1.5cm,right=1.5cm,top=2cm,bottom=2cm]{geometry}
\usepackage[intoc]{nomencl}
\usepackage{float}
\usepackage{floatflt}
\restylefloat{figure}
\usepackage{color}
\usepackage{siunitx}
\usepackage{hyperref}
\usepackage[utf8]{inputenc}
\sisetup{separate-uncertainty}
\usepackage[T1]{fontenc}
\usepackage[english]{babel}
\usepackage{ae}
\usepackage{chngcntr}
\usepackage[round]{natbib}
\usepackage[overload]{empheq}
\usepackage{amsthm}
\usepackage{amssymb}
\usepackage{latexsym}
\usepackage[title]{appendix}
\usepackage{multicol}
\usepackage{amsmath}
\usepackage{amsfonts}
\usepackage{tabularx}
\usepackage{caption}
\usepackage[multiple]{footmisc}
\usepackage{color}
\definecolor{grau}{rgb}{0.95,0.95,0.95}
\definecolor{dunkelgrau}{rgb}{0.8,0.8,0.8}
\usepackage{colortbl}
\usepackage{authblk}
\usepackage{url}
\usepackage{xcolor}
\usepackage{pgf}
\usepackage{wrapfig}
\usepackage[printwatermark]{xwatermark}
\usepackage{xcolor}
\usepackage{graphicx}
\usepackage{lipsum}
\usepackage{tikz}
\usepackage{abstract}
\numberwithin{equation}{section}







% Commands
\newcommand{\pa}{\partial}
\newcommand{\mr}[1]{\mathrm{#1}}
\newcommand{\mb}[1]{\mathbf{#1}}
\newcommand{\bo}[1]{\boldsymbol{#1}}
\newcommand{\fh}{f_\mr{h}^\mr{D}}
\newcommand{\fho}{f_\mr{h0}^\mr{D}}
\newcommand{\fhi}{f_\mr{h1}^\mr{D}}
\newcommand{\Bpa}{B_\parallel^\ast}




% Title
\title{Linear MHD using discrete differential forms}

% Date
\date{}

% Authors and affiliations
\author[1]{Florian Holderied}


\affil[1]{\textit{Max Planck Institute for Plasma Physics, Boltzmannstrasse 2, 85748 Garching, Germany}}





\begin{document} 

\maketitle


\section{The model}
Let us recall the standard form of the ideal magnetohydrodynamic (MHD) equations, which is a system of nonlinear partial differential equations for the mass density $\rho$, the fluid velocity $\mb{U}$, the magnetic induction $\mb{B}$ and the pressure $p$:
\begin{align}
&\frac{\pa\rho}{\pa t}+\nabla\cdot(\rho\mb{U})=0,\\
&\frac{\pa\mb{U}}{\pa t}+(\mb{U}\cdot\nabla)\mb{U}=\frac{1}{\mu_0}\frac{\nabla\times\mb{B}}{\rho}\times\mb{B}-\frac{\nabla p}{\rho},\\
&\frac{\pa\mb{B}}{\pa t}=\nabla\times(\mb{U}\times\mb{B}),\\
&\frac{\pa p}{\pa t}+\nabla\cdot(p\mb{U})+(\gamma-1)p\nabla\cdot\mb{U}=0,
\end{align}
where $\gamma=5/3$ is the adiabatic exponent. The corresponding linearized system about a time-independent equilibrium with small perturbations ($\rho=\rho_0+\rho_1$, $\mb{U}=\mb{U}_1$, $\mb{B}=\mb{B}_0+\mb{B}_1$, $p=p_0+p_1$) reads
\begin{align}
&\frac{\pa\rho}{\pa t}+\nabla\cdot(\rho_0\mb{U})=0,\\
&\frac{\pa\mb{U}}{\pa t}=\frac{1}{\mu_0}\frac{\nabla\times\mb{B}_0}{\rho_0}\times\mb{B}+\frac{1}{\mu_0}\frac{\nabla\times\mb{B}}{\rho_0}\times\mb{B}_0-\frac{\nabla p}{\rho_0},\\
&\frac{\pa\mb{B}}{\pa t}=\nabla\times(\mb{U}\times\mb{B}_0),\\
&\frac{\pa p}{\pa t}+\nabla\cdot(p_0\mb{U})+(\gamma-1)p_0\nabla\cdot\mb{U}=0,
\end{align}
where we performed the relabeling $\rho_1\rightarrow \rho$, $\mb{U}_1\rightarrow\mb{U}$, etc. Both the full and linearized model can equivalently be written in terms of coordinate independent differential forms and for physical reasons we assume the mass density to be a 3-form ($\rho\rightarrow\rho^3$), the pressure to be a 0-form ($p\rightarrow p^0$), the magnetic field to be a 2-form ($\mb{B}\rightarrow B^2$) and the velocity to be a 1-form ($\mb{U}\rightarrow U^1$). The full model then takes the form
\begin{align}
&\frac{\pa\rho^3}{\pa t}+\mr{d}(i_{\#U^1}\rho^3)=0,\\
&\ast\rho^3\left[\frac{\pa U^1}{\pa t}+\frac{1}{2}\mr{d}(i_{\#U^1}U^1)+i_{\#U^1}\mr{d}U^1\right]+\mr{d}p^0=\frac{1}{\mu_0}i_{\#\ast B^2}\mr{d}\ast B^2,\\
&\frac{\pa B^2}{\pa t}+\mr{d}(i_{\#U^1}B^2)=0,\\
&\frac{\pa p^0}{\pa t}+\ast\mr{d}\ast(p^0U^1)+(\gamma-1)p^0\ast\mr{d}\ast U^1=0.
\end{align} 
Note that the wedge product with a 0-form is just a multiplication with a scalar. The corresponding linearized system is given by
\begin{align}
&\frac{\pa\rho^3}{\pa t}+\mr{d}(i_{\#U^1}\rho_0)=0,\\
&(\ast\rho_0)\frac{\pa U^1}{\pa t}+\mr{d}p^0=\frac{1}{\mu_0}i_{\#\ast B_0}\mr{d}\ast B^2+\frac{1}{\mu_0}i_{\#\ast B^2}\mr{d}\ast B_0,\\
&\frac{\pa B^2}{\pa t}+\mr{d}(i_{\#U^1}B_0)=0,\\
&\frac{\pa p^0}{\pa t}+\ast\mr{d}\ast(p_0U^1)+(\gamma-1)p_0\ast\mr{d}\ast U^1=0,
\end{align} 
where one needs to keep in mind that the background quantities are still differential forms and not vector and scalar fields.

\section{Discretization}
As a next step, we introduce finite element basis functions which satisfy a discrete deRham sequence and which form a commuting diagram with the continuous functions via the interpolation-histopolation projectors $\Pi_0$, $\Pi_1$, $\Pi_2$ and $\Pi_3$. This is depicted in Fig. \ref{test} 
\begin{figure}
\centering
\includegraphics[scale=0.55]{01_Figures/deRham3d_forms.pdf}
\caption{Commuting diagram in 3d. The upper line represents the sequence for the continuous spaces, while the lower line represents the discrete counterpart.\label{test}}
\end{figure}
Assuming that we know the basis functions in each space (how this can be done with e.g. tensor-product B-splines, see Sec. ), we express the forms in their resepective bases as
\begin{alignat}{2}
&p^0(\mb{q})\approx p^0_h(\mb{q})=\sum_{\mathbf{i}}p_\mathbf{i}\Lambda_\mathbf{i}^0(\mb{q}),  &&\mb{p}^\top:=(p_0,\ldots,p_{N-1})\in\mathbb{R}^N,\\
&U^1(\mb{q})\approx U_h^1(\mb{q})=\sum_{\mathbf{i}}\sum_{\mu = 1}^3u_{\mu,\mathbf{i}}\Lambda^1_{\mu,\mathbf{i}}(\mb{q})\mathrm{d}q^\mu, \qquad &&\mb{u}^\top:=(\mb{u}_1^\top,\mb{u}_2^\top,\mb{u}_3^\top)\in\mathbb{R}^{3N},\\
&B^2(\mb{q})\approx B_h^2(\mb{q})=\sum_{\mathbf{i}}\sum_{\mu = 1}^3b_{\mu,\mathbf{i}}\Lambda^2_{\mu,\mathbf{i}}(\mb{q})(\mathrm{d}q^\alpha\wedge\mr{d}q^\beta)_\mu, \qquad &&\mb{b}^\top:=(\mb{b}_1^\top,\mb{b}_2^\top,\mb{b}_3^\top)\in\mathbb{R}^{3N},\\
&\rho^3(\mb{q})\approx \rho^3_h(\mb{q})=\sum_{\mathbf{i}}\rho_{123,\mathbf{i}}\Lambda_\mathbf{i}^3(\mb{q})\mr{d}q^1\wedge\mr{d}q^2\wedge\mr{d}q^3,  &&\bo{\rho}^\top:=(\rho_{123,0},\ldots,\rho_{123,N-1})\in\mathbb{R}^N,
\end{alignat}
where $\mb{i}=(i_1,i_2,i_3)$ is a multi-index and $N$ the total number of basis functions. To simplify the notation, we write for the components of the differential forms
\begin{alignat}{2}
&p^0_h\leftrightarrow p_h=(p_0,\ldots,p_{N-1})\begin{pmatrix}
\Lambda^0_0 \\ \vdots \\ \Lambda^0_{N-1}
\end{pmatrix}=\mb{p}^\top\bo{\Lambda}^0,&&\bo{\Lambda}^0\in\mathbb{R}^N, \\
&U^1_h\leftrightarrow \mb{U}_h^\top=(\mb{u}_1^\top,\mb{u}_2^\top,\mb{u}_3^\top)\begin{pmatrix}
\bo{\Lambda}^1_1 &0 &0 \\ 0 &\bo{\Lambda}^1_2 &0 \\ 0 &0 &\bo{\Lambda}^1_3
\end{pmatrix}=\mb{u}^\top\bo{\Lambda}^1,\qquad&&\bo{\Lambda}^1\in\mathbb{R}^{3N\times 3},\\
&B^2_h\leftrightarrow \hat{\mb{B}}_h^\top=(\mb{b}_1^\top,\mb{b}_2^\top,\mb{b}_3^\top)\begin{pmatrix}
\bo{\Lambda}^2_1 &0 &0 \\ 0 &\bo{\Lambda}^2_2 &0 \\ 0 &0 &\bo{\Lambda}^2_3
\end{pmatrix}=\mb{b}^\top\bo{\Lambda}^2,\qquad&&\bo{\Lambda}^2\in\mathbb{R}^{3N\times 3},\\
&\rho^3_h\leftrightarrow \rho_{123,h}=(\rho_{123,0},\ldots,\rho_{123,N-1})\begin{pmatrix}
\Lambda^3_0 \\ \vdots \\ \Lambda^3_{N-1}
\end{pmatrix}=\bo{\rho}^\top\bo{\Lambda}^3,&&\bo{\Lambda}^3\in\mathbb{R}^N,
\end{alignat}

\subsection{Continuity equation}
We start with the discretization of the mass continuity equation which we shall keep in strong form. In order to stay in the correct polynomial space, we need to project the second term back into the subspace of 3-forms by applying the projector $\Pi_3$. We use the same symbol for actions on forms and the components of a form. In the latter case the projector returns a vector of coefficients in the new basis.
\begin{align}
&\frac{\pa\rho_{123,h}}{\pa t}+\Pi_3\left[\nabla\cdot(\rho_0G^{-1}\mb{U}_h)\right]=0\\
\Leftrightarrow\quad&\frac{\pa\bo{\rho}}{\pa t}+\mathbb{D}\Pi_2\left[\rho_0G^{-1}(\bo{\Lambda}^1)^\top\right]\mb{u}=0\\
\Leftrightarrow\quad&\frac{\pa\bo{\rho}}{\pa t}+\mathbb{D}\mathcal{Q}\mb{u}=0
\end{align}
Note that we have used the commuting diagram property for exchanging projectors and differential operators. Furthermore, we have introduced the discrete divergence matrix $\mathbb{D}\in\mathbb{R}^{N\times 3N}$ and the projection matrix $\mathcal{Q}\in\mathbb{R}^{3N\times3N}$. We shall use calligraphic symbols for tensors which are related to projections. Explicitly, we have
\begin{align}
\mathcal{Q}_{ij}:=\Pi_{2,\mu}^{i_\mu}\left[\rho_0 G^{\mu k}\bo{\Lambda}^1_{jk}\right],\qquad i=\begin{cases}i_\mu, \quad &\mu=1\\ N+i_\mu &\mu=2\\2N+i_\mu &\mu=3\end{cases}
\end{align}
for $\mu=\{1,2,3\}$. Summing over repeated indices inside the squared bracket is assumed and $\Pi_2^{i_\mu}$ selects the $i_\mu$-th coefficient of the projection ($0\leq i_\mu\leq N-1$). If we stack these coefficients in a column vector, $\mathcal{Q}$ can be written as
\begin{align}
\mathcal{Q}&=\begin{pmatrix}
\Pi_{2,1}\left[\rho_0G^{11}(\bo{\Lambda}_1^1)^\top\right] &\Pi_{2,1}\left[\rho_0G^{12}(\bo{\Lambda}_2^1)^\top\right] &\Pi_{2,1}\left[\rho_0G^{13}(\bo{\Lambda}_3^1)^\top\right] \\[2mm]
\Pi_{2,2}\left[\rho_0G^{21}(\bo{\Lambda}_1^1)^\top\right] &\Pi_{2,2}\left[\rho_0G^{22}(\bo{\Lambda}_2^1)^\top\right] &\Pi_{2,2}\left[\rho_0G^{23}(\bo{\Lambda}_3^1)^\top\right] \\[2mm]
\Pi_{2,3}\left[\rho_0G^{31}(\bo{\Lambda}_1^1)^\top\right] &\Pi_{2,3}\left[\rho_0G^{32}(\bo{\Lambda}_2^1)^\top\right] &\Pi_{2,3}\left[\rho_0G^{33}(\bo{\Lambda}_3^1)^\top\right]
\end{pmatrix}\label{proj_matrix_p12}\\[1mm]
&=\begin{pmatrix}
\mb{c}_{11,0} &\mb{c}_{11,1} &\cdots &\mb{c}_{12,0} &\mb{c}_{12,1} &\cdots &\mb{c}_{13,0} &\mb{c}_{13,1} &\cdots \\
\mb{c}_{21,0} &\mb{c}_{21,1} &\cdots &\mb{c}_{22,0} &\mb{c}_{22,1} &\cdots &\mb{c}_{23,0} &\mb{c}_{23,1} &\cdots \\
\mb{c}_{31,0} &\mb{c}^1_{31,1} &\cdots &\mb{c}_{32,0} &\mb{c}_{32,1} &\cdots &\mb{c}_{33,0} &\mb{c}_{33,1} &\cdots
\end{pmatrix}.
\end{align}
Here, e.g. $\mb{c}_{23,1}$ are the coefficients resulting from the projection of the basis function with the index 1 in the block 23 in the matrix (\ref{proj_matrix_p12}). Unfortunately, this is a dense matrix, which is problematic from a memory consumption point of view. Therefore, we just save the right-hand sides, which define a sparse matrix, and perform the final projection in every time step again. Denoting by $(\mathcal{I}_{2,1},\mathcal{I}_{2,2},\mathcal{I}_{2,3})$ the mixed interpolation-histopolation matrices and by $(\mr{vec}_{2,1}(f),\mr{vec}_{2,2}(f),\mr{vec}_{2,3}(f))$ the right-hand side vectors for a 2-form with components $(f_1,f_2,f_3)$, we can write
\begin{align}
Q&=\begin{pmatrix}
\mathcal{I}_{2,1}^{-1} &0 &0 \\ 0 &\mathcal{I}_{2,2}^{-1} &0 \\ 0 &0 &\mathcal{I}_{2,3}^{-1} 
\end{pmatrix}
\begin{pmatrix}
\mr{vec}_{2,1}\left[\rho_0G^{11}(\bo{\Lambda}_1^1)^\top\right] &\mr{vec}_{2,1}\left[\rho_0G^{12}(\bo{\Lambda}_2^1)^\top\right] &\mr{vec}_{2,1}\left[\rho_0G^{13}(\bo{\Lambda}_3^1)^\top\right] \\[2mm]
\mr{vec}_{2,2}\left[\rho_0G^{21}(\bo{\Lambda}_1^1)^\top\right] &\mr{vec}_{2,2}\left[\rho_0G^{22}(\bo{\Lambda}_2^1)^\top\right] &\mr{vec}_{2,2}\left[\rho_0G^{23}(\bo{\Lambda}_3^1)^\top\right] \\[2mm]
\mr{vec}_{2,3}\left[\rho_0G^{31}(\bo{\Lambda}_1^1)^\top\right] &\mr{vec}_{2,3}\left[\rho_0G^{32}(\bo{\Lambda}_2^1)^\top\right] &\mr{vec}_{2,3}\left[\rho_0G^{33}(\bo{\Lambda}_3^1)^\top\right]
\end{pmatrix}\\ &=:\bo{\mathcal{I}}_2^{-1}\tilde{\mathcal{Q}}.
\end{align}
Thus, we only precompute the sparse matrices $\tilde{\mathcal{Q}}$ and $\mathcal{I}_2$. The entries of the former are defined by
\begin{align}
&(\mr{vec}_{2,1})_i(f):=\int_{\xi_{i_2}}^{\xi_{i_2+1}}\int_{\xi_{i_3}}^{\xi_{i_3+1}}f_1(\xi_{i_2},q_2,q_3)\mr{d}q^2\mr{d}q^3,  \\
&(\mr{vec}_{2,2})_i(f)=\int_{\xi_{i_1}}^{\xi_{i_1+1}}\int_{\xi_{i_3}}^{\xi_{i_3+1}}f_2(q_1,\xi_{i_2},q_3)\mr{d}q^1\mr{d}q^3, \\
&(\mr{vec}_{2,3})_i(f)=\int_{\xi_{i_1}}^{\xi_{i_1+1}}\int_{\xi_{i_2}}^{\xi_{i_2+1}}f_3(q_1, q_2, \xi_{i_3})\mr{d}q^1\mr{d}q^2, 
\end{align}
where the $\xi_{i_\mu}$ are some well chosen interpolation points. The mixed interpolation-histopolation matrices are given by
\begin{align}
(\mathcal{I}_{2,1})_{ij}:=\int_{\xi_{i_2}}^{\xi_{i_2+1}}\int_{\xi_{i_3}}^{\xi_{i_3+1}}\bo{\Lambda}^2_{1,j}(\xi_{i_2},q_2,q_3)\mr{d}q^2\mr{d}q^3 \\
(\mathcal{I}_{2,2})_{ij}:=\int_{\xi_{i_1}}^{\xi_{i_1+1}}\int_{\xi_{i_3}}^{\xi_{i_3+1}}\bo{\Lambda}_{2,j}^2(q_1,\xi_{i_2},q_3)\mr{d}q^1\mr{d}q^3 \\
(\mathcal{I}_{2,3})_{ij}:=\int_{\xi_{i_1}}^{\xi_{i_1+1}}\int_{\xi_{i_2}}^{\xi_{i_2+1}}\bo{\Lambda}_{3,j}^2(q_1, q_2, \xi_{i_3})\mr{d}q^1\mr{d}q^2
\end{align}
The sparsity of $\tilde{\mathcal{Q}}$ and $\mathcal{I}_2$ follows immediately from the local support of the basis functions.

\subsection{Momentum equation}
Unlike the continuity equation we choose a weak formulation for the momentum equation and consequently take the inner product with a test function $V^1\in H\Lambda^1(\Omega)$ to obtain the formulation: Find $U^1\in H\Lambda^1(\Omega)$ such that
\begin{align}
\left(\ast\rho_0\frac{\pa U^1}{\pa t},V^1\right)+\left(\mr{d}p^0,V^1\right)=\frac{1}{\mu_0}\left(i_{\#\ast B_0}\mr{d}\ast B^2,V^1\right)+\frac{1}{\mu_0}\left(i_{\#\ast B^2}\mr{d}\ast B_0,V^1\right) \quad\forall V^1\in H\Lambda^1(\Omega).
\end{align}
We apply the Galerkin approximation to each term and project back into the right spaces where necessary. Let us start with the first term:
\begin{align}
\left(\ast\rho_0\frac{\pa U^1}{\pa t},V^1\right)&=\int_{\hat{\Omega}}\ast\rho_0\dot{\mb{U}}^\top G^{-1}\mb{V}\sqrt{g}\,\mr{d}^3q\approx \int_{\hat{\Omega}}\Pi_1\left(\ast\rho_0\dot{\mb{U}}_h^\top\right)G^{-1}\mb{V}_h\sqrt{g}\,\mr{d}^3q\\&=\dot{\mb{u}}^\top\mathcal{W}^\top\underbrace{\int_{\hat{\Omega}}\bo{\Lambda}^1G^{-1}(\bo{\Lambda}^1)^\top\sqrt{g}\,\mr{d}^3q}_{=:\mathbb{M}^1}\,\mb{v}=\dot{\mb{u}}^\top\mathcal{W}^\top\mathbb{M}^1\mb{v}\quad\forall \mb{v}\in\mathbb{R}^{3N},
\end{align}
where $\mathbb{M}^1\in\mathbb{R}^{3N\times 3N}$ is the mass matrix in the supspace $V_1$. The projection matrix is given by
\begin{align}
\mathcal{W}_{ij}=\Pi_{1,\mu}^{i_\mu}\left[\ast\rho_0\bo{\Lambda}^1_{j\mu}\right]
\end{align}
where the right-hand side explicitly reads
\begin{align}
&\tilde{\mathcal{W}}:=\begin{pmatrix}
\mr{vec}_{1,1}\left[\rho_0/\sqrt{g}(\bo{\Lambda}_1^1)^\top\right] &0 &0 \\
0 &\mr{vec}_{1,2}\left[\rho_0/\sqrt{g}(\bo{\Lambda}_2^1)^\top\right] &0 \\
0 &0 &\mr{vec}_{1,3}\left[\rho_0/\sqrt{g}(\bo{\Lambda}_3^1)^\top\right]
\end{pmatrix}\\
\Rightarrow\quad&\mathcal{W}=\bo{\mathcal{I}}_1^{-1}\tilde{\mathcal{W}}.
\end{align}
The projections are once more mixed interpolation-histopolation problems and defined by
\begin{align}
&(\mr{vec}_{1,1})_i(f)=\int_{\xi_{i_1}}^{\xi_{i_1+1}}f_1(q_1,\xi_{i_2},\xi_{i_3})\mr{d}q^1,\qquad  (\mathcal{I}_{1,1})_{ij}=\int_{\xi_{i_1}}^{\xi_{i_1+1}}\bo{\Lambda}^1_{1,j}(q_1,\xi_{i_2},\xi_{i_3})\mr{d}q^1 \\
&(\mr{vec}_{1,2})_i(f)=\int_{\xi_{i_2}}^{\xi_{i_2+1}}f_2(\xi_{i_1},q_2,\xi_{i_3})\mr{d}q^2,\qquad  (\mathcal{I}_{1,2})_{ij}=\int_{\xi_{i_2}}^{\xi_{i_2+1}}\bo{\Lambda}^1_{2,j}(\xi_{i_1},q_2,\xi_{i_3})\mr{d}q^2 \\
&(\mr{vec}_{1,3})_i(f)=\int_{\xi_{i_3}}^{\xi_{i_3+1}}f_3(\xi_{i_1},\xi_{i_2},q_3)\mr{d}q^3,\qquad (\mathcal{I}_{1,3})_{ij}=\int_{\xi_{i_3}}^{\xi_{i_3+1}}\bo{\Lambda}^1_{3,j}(\xi_{i_1},\xi_{i_2},q_3)\mr{d}q^3.
\end{align}
For the second term including the pressure we get
\begin{align}
\left(\mr{d}p^0,V^1\right)=\int_{\hat{\Omega}}(\nabla p)^\top G^{-1}\mb{V}\sqrt{g}\,\mr{d}^3q\,\approx(\mathbb{G}\mb{p})^\top\int_{\hat{\Omega}}\bo{\Lambda}^1G^{-1}(\bo{\Lambda}^1)^\top\sqrt{g}\mr{d}^3q\mb{v}=\mb{p}^\top\mathbb{G}^\top\mathbb{M}^1\mb{v}\quad\forall \mb{v}\in\mathbb{R}^{3N},
\end{align}
with $\mathbb{G}\in\mathbb{R}^{3N\times N}$ being the discrete gradient matrix. Using the identites $\langle i_{\#\gamma^1}\alpha^2,\beta^1\rangle=\langle\alpha^2,\gamma^1\wedge\beta^1\rangle$ and $\ast(\ast B_0\wedge V^1)=i_{\#V^1}B_0$, the third term yields (omitting the $1/\mu_0$)
\begin{align}
\left(i_{\#\ast B_0}\mr{d}\ast B^2,V^1\right)=\left(d\ast B^2,\ast B_0\wedge V^1\right)=\left(\ast\mr{d}\ast B^2,\ast(\ast B_0\wedge V^1)\right)=\left(d^\ast B^2,i_{\#V^1}B_0\right),
\end{align}
where we introduced the co-differential operator $\mr{d}^\ast\alpha^p=(-1)^p\ast\mr{d}\ast\alpha^p$. Applying the Green formula for differential forms and assuming that the boundary term vanishes yields
\begin{align}
\left(i_{\#\ast B_0}\mr{d}\ast B^2,V^1\right)&=\left(B^2,\mr{d}i_{\#V^1}B_0\right)=\int_{\hat{\Omega}}\frac{1}{g}\hat{\mb{B}}^\top G\left(\nabla\times(\hat{\mb{B}}_0\times G^{-1}\mb{V})\right)\sqrt{g}\,\mr{d}^3q\\
&\approx\mb{b}^\top\underbrace{\int_{\hat{\Omega}}\frac{1}{\sqrt{g}}\bo{\Lambda}^2G(\bo{\Lambda}^2)^\top\mr{d}^3q}_{=:\mathbb{M}^2}\,\mathbb{C}\Pi_1\left(\mathbb{B}_0G^{-1}(\bo{\Lambda}^1)^\top\right)\mb{v}=\mb{b}^\top\mathbb{M}^2\mathbb{C}\mathcal{T}\mb{v}\quad\forall\mb{v}\in\mathbb{R}^{3N},
\end{align}
where we introduced the discrete curl matrix $\mathbb{C}\in\mathbb{R}^{3N\times 3N}$, the mass matrix $\mathbb{M}^2$ in the space $V_2$ and we wrote the vector product of the background magnetic field with the velocity field in terms of a matrix-vector product by using the matrix
\begin{align}
\mathbb{B}_0:=\begin{pmatrix}
0 &-B_{0,12} &B_{0,31} \\ B_{0,12} &0 &-B_{0,23} \\ -B_{0,31} &B_{0,23} &0
\end{pmatrix}\in\mathbb{R}^{3\times 3}.
\end{align}
The projection matrix $\mathcal{T}$ is given by
\begin{align}
\mathcal{T}_{ij}:=\Pi_{1,\mu}^{i_\mu}\left[(\mathbb{B}_0)_{\mu k}G^{kl}\bo{\Lambda}^1_{jl}\right]=\Pi_{1,\mu}^{i_\mu}\left[\epsilon_{\mu mk}B_{0,m}G^{kl}\bo{\Lambda}^1_{jl}\right],
\end{align}
which explicity amounts to
\small
\begin{align}
&\mathcal{T}=\bo{\mathcal{I}}_1^{-1}\tilde{\mathcal{T}}\\
&=\begin{pmatrix}
\mr{vec}_{1,1}\left[(B_{0,31}G^{31}-B_{0,12}G^{21})(\bo{\Lambda}^1_1)^\top, (B_{0,31}G^{32}-B_{0,12}G^{22})(\bo{\Lambda}^1_2)^\top, (B_{0,31}G^{33}-B_{0,12}G^{23})(\bo{\Lambda}^1_3)^\top\right] \\[2mm]
\mr{vec}_{1,2}\left[(B_{0,12}G^{11}-B_{0,23}G^{31})(\bo{\Lambda}^1_1)^\top, (B_{0,12}G^{12}-B_{0,23}G^{32})(\bo{\Lambda}^1_2)^\top, (B_{0,12}G^{13}-B_{0,23}G^{33})(\bo{\Lambda}^1_3)^\top\right] \\[2mm]
\mr{vec}_{1,3}\left[(B_{0,23}G^{21}-B_{0,31}G^{11})(\bo{\Lambda}^1_1)^\top, (B_{0,23}G^{22}-B_{0,31}G^{12})(\bo{\Lambda}^1_2)^\top, (B_{0,23}G^{23}-B_{0,31}G^{13})(\bo{\Lambda}^1_3)^\top\right]
\end{pmatrix}.
\end{align}
\normalsize
Note that we can also use the Levi-Civita symbol in the definition of $\mathcal{T}$. Finally, we perform the same steps for the last term:
\begin{align}
\left(i_{\#\ast B^2}\mr{d}\ast B_0,V^1\right)&=\left(B_0,\mr{d}i_{\#V^1}B^2\right)=\int_{\hat{\Omega}}\frac{1}{g}\hat{\mb{B}}_0^\top G\left(\nabla\times(\hat{\mb{B}}\times G^{-1}\mb{V})\right)\sqrt{g}\,\mr{d}^3q\\
&\approx\Pi_2\left(\hat{\mb{B}}_0^\top\right)\int_{\hat{\Omega}}\frac{1}{\sqrt{g}}\bo{\Lambda}^2G(\bo{\Lambda}^2)^\top\mr{d}^3q\,\mathbb{C}\Pi_1\left[(\bo{\Lambda}_2)^\top\mb{b}\times G^{-1}(\bo{\Lambda}_1)^\top\mb{v}\right]\\
&=\mb{b}_0^\top\mathbb{M}^2\mathbb{C}(\mb{b}^\top\mathcal{P}\mb{v})\quad\forall \mb{v}\in\mathbb{R}^{3N}.
\end{align}
The projection tensor $\mathcal{P}\in\mathbb{R}^{3N\times3N\times3N}$ is given by
\begin{align}
\mathcal{P}_{ijk}=\Pi_1^{j_\mu}\left[\epsilon_{\mu lm}\Lambda^2_{il}G^{mn}\Lambda^1_{kn}\right],
\end{align}
where it is important to note that the entries of $\mb{b}$ contract from left with the index $i$ and the entries of $\mb{v}$ from right with the index $k$. The result is then a vector defined by the index $j$. In total we get
\begin{align}
&\dot{\mb{u}}^\top\mathcal{W}^\top\mathbb{M}^1\mb{v}=\frac{1}{\mu_0}\mb{b}^\top\mathbb{M}^2\mathbb{C}\mathcal{T}\mb{v}+\frac{1}{\mu_0}\mb{b}_0^\top\mathbb{M}^2\mathbb{C}(\mb{b}^\top\mathcal{P}\mb{v})-\mb{p}^\top\mathbb{G}^\top\mathbb{M}^1\mb{v}\quad\forall\mb{v}\in\mathbb{R}^{3N}\\
\Leftrightarrow\quad &\mathbb{M}^1\mathcal{W}\dot{\mb{u}}=\frac{1}{\mu_0}\mathcal{T}^\top\mathbb{C}^\top\mathbb{M}^2\mb{b}+\frac{1}{\mu_0}\mb{b}_0^\top\mathbb{M}^2\mathbb{C}\mathcal{P}^{\top_{i,jk}}\mb{b}-\mathbb{M}^1\mathbb{G}\mb{p},
\end{align}
where $\mathcal{P}^{\top_{i,jk}}$ means that order of the indices is changed from $()_{ijk}$ to $()_{jki}$ such that $\mb{b}$ still contracts with the index $i$ from right.

\subsection{Induction equation}
Like the continuity equation we keep the induction equation in strong form. This time we have to use the projector $\Pi_2$ which commutes with the curl operator.
\begin{align}
&\frac{\pa \hat{\mb{B}}_h}{\pa t}+\Pi_2\left[\nabla\times(\hat{\mb{B}}_0\times G^{-1}\mb{U}_h)\right]=0\\
\Leftrightarrow\quad &\frac{\pa\mb{b}}{\pa t}+\mathbb{C}\Pi_1\left[\mathbb{B}_0G^{-1}(\bo{\Lambda}^1)^\top\right]\mb{u}=0 \\
\Leftrightarrow\quad &\frac{\pa\mb{b}}{\pa t}+\mathbb{C}\mathcal{T}\mb{u}=0.
\end{align}
We immediately see that we obtain the same projection matrix as for the force term in the momentum equation.

\subsection{Energy equation}
The appearance of the co-differential operator in all terms of the energy equation already indicates that it is convenient to solve this equation once more weakly. Taking the inner product with a test function $r^0\in H\Lambda^0(\Omega)$ yields the formulation: find $p^0\in H\Lambda^0(\Omega)$ such that
\begin{align}
&\left(\frac{\pa p^0}{\pa t},r^0\right)-\left(\mr{d}^\ast(p_0U^1),r^0\right)-(\gamma-1)\left(\mr{d}^\ast U^1,p_0r^0\right)=0\quad\forall r^0\in H\Lambda^0(\Omega)\\
\Leftrightarrow\quad&\left(\frac{\pa p^0}{\pa t},r^0\right)-\left(p_0U^1,\mr{d}r^0\right)-(\gamma-1)\left(U^1,\mr{d}(p_0r^0)\right)=0\quad\forall r^0\in H\Lambda^0(\Omega),
\end{align}
if we again assume all boundary terms to vanish. For the first term we simply get
\begin{align}
\left(\frac{\pa p^0}{\pa t},r^0\right)=\int_{\hat{\Omega}}\dot{p}r\sqrt{g}\mr{d}^3q\approx\dot{\mb{p}}^\top\underbrace{\int_{\hat{\Omega}}\bo{\Lambda}^0(\bo{\Lambda}^0)^\top\sqrt{g}\,\mr{d}^3q}_{=:\mathbb{M}^0}\,\mb{r}=\dot{\mb{p}}^\top\mathbb{M}^0\mb{r},
\end{align}
where $\mathbb{M}^0\in\mathbb{R}^{N\times N}$ is the mass matrix in the space $V_0$. The second term reads
\begin{align}
\left(p_0U^1,\mr{d}r^0\right)&=\int_{\hat{\Omega}}p_0\mb{U}^\top G^{-1}\nabla r\sqrt{g}\,\mr{d}^3q\approx\int_{\hat{\Omega}}\Pi_1(p_0\mb{U}_h^\top)G^{-1}\nabla r_h\sqrt{g}\,\mr{d}^3q\\
&=\mb{u}^\top\mathcal{A}^\top\int_{\hat{\Omega}}\bo{\Lambda}^1G^{-1}(\bo{\Lambda}^1)^\top\sqrt{g}\,\mr{d}^3q\,\mathbb{G}\mb{r}=\mb{u}^\top\mathcal{A}^\top\mathbb{M}^1\mathbb{G}\mb{r},
\end{align}
with $\mathcal{A}\in\mathbb{R}^{3N\times3N}$ being the projection matrix with entries
\begin{align}
&\mathcal{A}_{ij}:=\Pi_{1,\mu}^{i_\mu}\left[p_0\Lambda^1_{j\mu}\right]\\
\Rightarrow\quad&\mathcal{A}=\bo{\mathcal{I}}_1^{-1}\tilde{\mathcal{A}}=\bo{\mathcal{I}}_1^{-1}\begin{pmatrix}
\mr{vec}_{1,1}\left[p_0(\bo{\Lambda}_1^1)^\top\right] &0 &0 \\
0 &\mr{vec}_{1,2}\left[p_0(\bo{\Lambda}_2^1)^\top\right] &0 \\
0 &0 &\mr{vec}_{1,3}\left[p_0(\bo{\Lambda}_3^1)^\top\right]
\end{pmatrix}.
\end{align} 
Finally, we obtain for the last term
\begin{align}
\left(U^1,\mr{d}(p_0r^0)\right)&=\int_{\hat{\Omega}}\mb{U}^\top G^{-1}\nabla(p_0r)\sqrt{g}\,\mr{d}^3q\approx \mb{u}^\top\int_{\hat{\Omega}}\bo{\Lambda}^1G^{-1}(\bo{\Lambda}^1)^\top\mr{d}^3q\,\mathbb{G}\Pi_0\left(p_0(\bo{\Lambda}^0)^\top\right)\mb{r}\\
&=\mb{u}^\top\mathbb{M}^1\mathbb{G}\mathcal{S}\mb{r},
\end{align}
where $\mathcal{S}\in\mathbb{R}^{N\times N}$ is the projection matrix with entries
\begin{align}
&\mathcal{S}_{ij}:=\Pi_0^i\left[p_0\Lambda^0_j\right]\\
\Rightarrow\quad&\mathcal{S}=\bo{\mathcal{I}}_0^{-1}\tilde{\mathcal{S}}=\bo{\mathcal{I}}_0^{-1}\mr{vec}_0\left[p_0(\bo{\Lambda}^0)^\top\right],
\end{align}
whose right-hand side follows from a pure interpolation problem defined by
\begin{align}
(\mr{vec}_0)_i(f)=f(\xi_{i_1},\xi_{i_2},\xi_{i_3}),\qquad (\mathcal{I}_0)_{ij}=\Lambda^0_j(\xi_{i_1},\xi_{i_2},\xi_{i_3}).
\end{align}
In summary, we obtain the following semi-discrete energy equation:
\begin{align}
&\dot{\mb{p}}^\top\mathbb{M}^0\mb{r}-\mb{u}^\top\mathcal{A}^\top\mathbb{M}^1\mathbb{G}\mb{r}-(\gamma-1)\mb{u}^\top\mathbb{M}^1\mathbb{G}\mathcal{S}\mb{r}=0\quad\forall\mb{r}\in\mathbb{R}^N\\
\Leftrightarrow\quad&\mathbb{M}^0\dot{\mb{p}}-\mathbb{G}^\top\mathbb{M}^1\mathcal{A}\mb{u}-(\gamma-1)\mathcal{S}^\top\mathbb{G}^\top\mathbb{M}^1\mb{u}=0.
\end{align}
\section{Implementation}
For simplicity we shall restrict ourselves on the moment on periodic boundary conditions in all directions as well as a smooth analytical mapping $F:\hat{\Omega}\rightarrow\Omega$.
\subsection{Discrete differential operators}
We use uniform tensor product B-splines of degree $p=(p_1,p_2,p_3)$ as a basis for the space $V_0$ which are created from knot vectors $\hat{T}^{p_\mu}=\{-p_\mu\Delta q_\mu, -(p_\mu-1)\Delta q_\mu,\ldots,0,\Delta q_\mu,2\Delta q_\mu,\ldots,1,1 +\Delta q_\mu,\ldots,1 + p_\mu\Delta q_\mu\}$, one for each of the coordinates on the logical domain $\hat{\Omega}$, i.e. $\mu=\{1,2,3\}$. $ \Delta q_\mu$ is just the element size of the discretized logical domain in $\mu$-direction. A family of B-splines is then recursively defined by
\begin{align}
&\hat{N}_{i_\mu}^{p_\mu}(q_\mu)=\frac{q_\mu-\hat{T}^\mu_{i_\mu}}{\hat{T}^\mu_{i_\mu+p_\mu}-\hat{T}^\mu_{i_\mu}}\hat{N}_{i_\mu}^{p_\mu-1}(q_\mu)+\frac{\hat{T}^\mu_{i_\mu+p_\mu+1}-q_\mu}{\hat{T}^\mu_{i_\mu+p_\mu+1}-\hat{T}^\mu_{i_\mu+1}}\hat{N}_{i_\mu+1}^{p_\mu-1}(q_\mu),\\
&\hat{N}^0_{i_\mu}(q_\mu)=\begin{cases}1\quad q_\mu\in[\hat{T}^{p_\mu}_{i_\mu},\hat{T}^{p_\mu}_{i_\mu+1}], \\ 0\quad \text{else}\end{cases}.
\end{align}
This defines a spline space with $N_\mu=\text{len}(T^\mu)-2p_\mu-1$ distinct B-splines. We will also need a compatible spline space of one degree less which is created from a reduced knot vector $\hat{t}^{p_\mu-1}=\hat{T}^{p_\mu}(1:-1)$, i.e. from deleting the first and last entry of the original knot vector. We denote the resulting spline family weighted with the element size $\Delta q_\mu$ by $\hat{D}_{i_\mu}^{p_\mu-1}=\hat{N}_{i_\mu}^{p_\mu-1}/\Delta q_\mu$. Note that the reduced space has the same number of basis functions which is specific to periodic boundary conditions. With this choice for the reduced space, the derivative of a spline in the original space can simply be written as
\begin{align}
(N_{i_\mu}^{p_\mu})^\prime(q_\mu)=D_{i_\mu-1}^{p_\mu-1}-D_{i_\mu}^{p_\mu-1}.
\end{align}
This has the consequence that the derivative of a finite element field of the form
\begin{align}
f_\mu(q_\mu)=\sum_{i_\mu}f_{i_\mu}N_{i_\mu}^{p_\mu}(q_\mu)
\end{align}
is just an operation on the vector of coefficients $\mb{f}_\mu$ with a matrix that contains 1,-1 and 0 only, i.e.
\begin{align}
(f_\mu)^\prime(q_\mu)=&\sum_{i_\mu}f_{i_\mu}(D_{i_\mu-1}^{p_\mu-1}-D_{i_\mu}^{p_\mu-1})=\sum_{i_\mu}(f_{i_\mu+1}-f_{i_\mu})D_{i_\mu}^{p_\mu-1}:=\sum_{i_\mu}\hat{f}_{i_\mu}D_{i_\mu}^{p_\mu-1} \\
\Rightarrow&\quad\mb{\hat{f}}_\mu=\begin{pmatrix}
-1 &1 & & & \\  &-1 &1 & & \\  & & \ddots &\ddots & \\ & & &-1 &1 \\ 1 & & & &-1
\end{pmatrix}\mb{f}_\mu=\mathbb{G}_\mu\mb{f}_\mu,\quad\mathbb{G}_\mu\in\mathbb{R}^{N_\mu\times N_\mu}
\end{align}
If we use a tensor-product basis for functions in the space $V_0$, i.e. we write 
\begin{align}
f^0(q_1,q_2,q_3)=\sum_{i_1,i_2,i_3}f_{i_1i_2i_3}N_{i_1}^{p_1}(q_1)N_{i_2}^{p_2}(q_2)N_{i_3}^{p_3}(q_3),
\end{align}
we can construct the sequence:
\begin{align}
N_{i_1}^{p_1}N_{i_2}^{p_2}N_{i_3}^{p_3}\,\overset{\mathbb{G}\,(\nabla)}{\longrightarrow}\,\begin{pmatrix}
D_{i_1}^{p_1-1}N_{i_2}^{p_2}N_{i_3}^{p_3} \\ N_{i_1}^{p_1}D_{i_2}^{p_2-1}N_{i_3}^{p_3} \\ N_{i_1}^{p_1}N_{i_2}^{p_2}D_{i_3}^{p_3-1}
\end{pmatrix}\,\overset{\mathbb{C}\,(\nabla\times)}{\longrightarrow}\,\begin{pmatrix}
N_{i_1}^{p_1}D_{i_2}^{p_2-1}D_{i_3}^{p_3-1} \\ D_{i_1}^{p_1-1}N_{i_2}^{p_2}D_{i_3}^{p_3-1} \\ D_{i_1}^{p_1-1}D_{i_2}^{p_2-1}N_{i_3}^{p_3}
\end{pmatrix}\,\overset{\mathbb{D}\,(\nabla\cdot)}{\longrightarrow}\,
D_{i_1}^{p_1-1}D_{i_2}^{p_2-1}D_{i_3}^{p_3-1},
\end{align}
where the discrete derivatives are given by
\begin{align}
&\mathbb{G}=\begin{pmatrix}
\mathbb{G}_1\otimes\mathbb{I}_{N_2}\otimes\mathbb{I}_{N_3} \\ \mathbb{I}_{N_1}\otimes\mathbb{G}_2\otimes\mathbb{I}_{N_3} \\ \mathbb{I}_{N_1}\otimes\mathbb{I}_{N_2}\otimes\mathbb{G}_3
\end{pmatrix}\in\mathbb{R}^{3N\times N},\\
&\mathbb{C}=\begin{pmatrix}
0 &-\mathbb{I}_{N_1}\otimes\mathbb{I}_{N_2}\otimes\mathbb{G}_3 &\mathbb{I}_{N_1}\otimes\mathbb{G}_2\otimes\mathbb{I}_{N_3} \\ \mathbb{I}_{N_1}\otimes\mathbb{I}_{N_2}\otimes\mathbb{G}_3 &0 & -\mathbb{G}_1\otimes\mathbb{I}_{N_2}\otimes\mathbb{I}_{N_3}\\ -\mathbb{I}_{N_1}\otimes\mathbb{G}_2\otimes\mathbb{I}_{N_3} &\mathbb{G}_1\otimes\mathbb{I}_{N_2}\otimes\mathbb{I}_{N_3} &0
\end{pmatrix}\in\mathbb{R}^{3N\times 3N},\\
&\mathbb{D}=\begin{pmatrix}
\mathbb{G}_1\otimes\mathbb{I}_{N_2}\otimes\mathbb{I}_{N_3} &\mathbb{I}_{N_1}\otimes\mathbb{G}_2\otimes\mathbb{I}_{N_3} &\mathbb{I}_{N_1}\otimes\mathbb{I}_{N_2}\otimes\mathbb{G}_3
\end{pmatrix}\in\mathbb{R}^{N\times 3N}.
\end{align}
Here $N=N_1N_2N_3$ is the total number of basis functions in the space $V_0$. Note that we have $\mathbb{C}\mathbb{G}=0$ and $\mathbb{D}\mathbb{C}=0$ which are just the discrete counterparts of the well-known identities $\nabla\times(\nabla)=0$ and $\nabla\cdot(\nabla\times)=0$. 
\subsection{Projections}
In order to perform projections on the right finite element spaces, we have to deal with interpolation and histopolation problems. For the latter, we need to perform integration between the so-called Greville points, one associated to every basis function. In the case of periodic boundary conditions, these points are just the element vertices for odd polynomial degrees and the element centers for even degrees. We use a Gauss-Legendre quadrature rule, which means that we first define a suitable set of quadrature points. If not specified differently, we use a quadrature rule of degree $\mr{nq}=\{p_1+1,p_2+1,p_3+1\}$. We denote the set of global quadrature points and weights by $\text{pts}=\{\text{pts}_1,\text{pts}_2,\text{pts}_3\}$ and $\text{wts}=\{\text{wts}_1,\text{wts}_2,\text{wts}_3\}$, respectively.
\begin{figure}
\includegraphics[scale=0.5]{01_Figures/N_odd.pdf}
\includegraphics[scale=0.5]{01_Figures/D_odd.pdf}\\
\includegraphics[scale=0.5]{01_Figures/N_even.pdf}
\includegraphics[scale=0.5]{01_Figures/D_even.pdf}
\caption{Periodic spline spaces with corresponding reduced spaces for odd (upper) and even (lower) polynomial degree.}
\end{figure} 
\section{Test case 1: Shear Alfv\'{e}n waves}
The most simple first test case is the simulation of shear Alfv\'{e}n waves in a homogeneous equilibrium and slab geometry. The latter means that we use a mapping of the form
\begin{align}
&F:\hat{\Omega}=[0,1]\times [0,1]\times [0,1]\rightarrow\Omega=[0,L_x]\times[0,L_y]\times[0,L_z],\quad \begin{pmatrix}
q_1\\ q_2 \\ q_3\end{pmatrix}\mapsto\begin{pmatrix}
L_x q_1 \\L_y q_2 \\L_z q_3
\end{pmatrix}=\begin{pmatrix}
x\\ y\\ z
\end{pmatrix}.\\
\Rightarrow\quad&\sqrt{g}=\sqrt{\det G}=L_xL_yL_z,\quad\mathcal{W}=\rho_{0,123}/\sqrt{g}\,\mathbb{I},
\end{align}
where $\mathbb{I}$ denotes the identity and $\rho_{0,123}$ is the (constant) component of the prescribed background density volume form. Due to the fact that shear Alfv\'{e}n waves are pure transverse waves with perturbations in the magnetic field and the velocity and no perturbations in the density and the pressure, only the momentum and induction equation are needed. For this case, the semi-discrete system takes the reduced form
\begin{align}
\frac{\pa}{\pa t}\begin{pmatrix}
\mb{u} \\ \mb{b}
\end{pmatrix}=\begin{pmatrix}
0 &\sqrt{g}(\mathbb{M}^1)^{-1}\mathcal{T}^\top\mathbb{C}^\top\mathbb{M}^2/(\mu_0\rho_{0,123}) \\
-\mathbb{C}\mathcal{T} &0
\end{pmatrix}\begin{pmatrix}
\mb{u}\\ \mb{b}
\end{pmatrix}=\mathbb{A}\mb{S},
\end{align} 
which the state vector $\mb{S}=(\mb{u},\mb{b})$. If we additionally introduce the total energy
\begin{align}
\mathcal{H}=\mathcal{H}_U+\mathcal{H}_B=\frac{1}{2}\left(\ast\rho_0U^1,U^1\right)+\frac{1}{2\mu_0}\left(B^2,B^2\right)\approx\frac{1}{2}\rho_{0,123}/\sqrt{g}\,\mb{u}^\top\mathbb{M}^1\mb{u}+\frac{1}{2\mu_0}\mb{b}^\top\mathbb{M}^2\mb{b},
\end{align}
we can write the system in the canonical Hamiltonian form
\begin{align}
\frac{\pa \mb{S}}{\pa t}=\mathbb{J}\nabla_\mb{S}H\:\Leftrightarrow\:
\frac{\pa}{\pa t}\begin{pmatrix}
\mb{u} \\ \mb{b}
\end{pmatrix}=\begin{pmatrix}
0 &\sqrt{g}\,(\mathbb{M}^1)^{-1}\mathcal{T}^\top\mathbb{C}^\top/\rho_{0,123} \\
-\sqrt{g}\,\mathbb{C}\mathcal{T}(\mathbb{M}^1)^{-1}/\rho_{0,123} &0
\end{pmatrix}\begin{pmatrix}
\rho_{0,123}/\sqrt{g}\,\mathbb{M}^1\mb{u}\\ \mathbb{M}^2\mb{b}/\mu_0
\end{pmatrix}.
\end{align} 
We observe that the resulting Poisson matrix is skew-symmetric $\mathbb{J}^\top=-\mathbb{J}$ which means that the Hamiltonian is conserved by the dynamics:
\begin{align}
\frac{\mr{d}\mathcal{H}}{\mr{d}t}=\nabla_\mb{S}\mathcal{H}^\top\frac{\pa\mb{S}}{\pa t}=\nabla_\mb{S}\mathcal{H}^\top\mathbb{J}\nabla_\mb{S}\mathcal{H}=-\nabla_\mb{S}\mathcal{H}^\top\mathbb{J}\nabla_\mb{S}\mathcal{H}=0.
\end{align}
Finally, it is worth reminding ourselves that in our framework the background density and the background magnetic field must be forms. This means that if one prescribes a scalar and vector field in physical coordinates, one must first compute the componets in logical coordinates followed by relating them to the components of the form:
\begin{align}
&\tilde{\mb{B}}_0(\mb{x})=B_{0x}(\mb{x})\mb{e}_x+B_{0y}(\mb{x})\mb{e}_y+B_{0z}(\mb{x})\mb{e}_z=DF(F^{-1}(\mb{x}))\mb{B}_0(F^{-1}(\mb{x}))\\
\Rightarrow\quad &\hat{\mb{B}}_0=o\sqrt{g}DF^{-1}\tilde{\mb{B}}_0\\
&\tilde{\rho}_0=\tilde{\rho}_0(\mb{x})\\
\Rightarrow \quad&\rho_{0,123}(\mb{q})=\sqrt{g}\tilde{\rho}(\mb{x}),
\end{align}
where quantities with a tilde are on the physical domain.
\subsection{Time discretization}
Since we are dealing with a Hamiltonian system, there are some naturals choices for integrating the dynamical system in time: First, we shall look at the Hamiltonian splitting which consists of splitting the Hamiltonian in $\mathcal{H}=\mathcal{H}_U+\mathcal{H}_B$ while keeping the full Poisson matrix. Second, we shall use a discrete gradient method which should yield exact energy conservation.
\subsubsection{Hamiltonian splitting}
Splitting the Hamiltonian in the aforementioned way leads to the two sub-systems
\begin{alignat}{2}
&\mathcal{H}_U:\enspace\mathbb{J}\nabla_\mb{S}\mathcal{H}_U=\begin{pmatrix}
0 \\ -\mathbb{C}\mathcal{T}\mb{u}
\end{pmatrix}\\ &\qquad\Rightarrow \mb{b}(t)=\mb{b}(t_0)-t\mathbb{C}\mathcal{T}\mb{u},\\
&\mathcal{H}_B:\enspace\mathbb{J}\nabla_\mb{S}\mathcal{H}_U=\begin{pmatrix}
\sqrt{g}(\mathbb{M}^1)^{-1}\mathcal{T}^\top\mathbb{C}^\top\mathbb{M}^2/(\mu_0\rho_{0,123})\mb{b} \\ 0
\end{pmatrix} \\ &\qquad\Rightarrow \enspace\mb{u}(t)=\mb{u}(t_0)+t\sqrt{g}(\mathbb{M}^1)^{-1}\mathcal{T}^\top\mathbb{C}^\top\mathbb{M}^2/(\mu_0\rho_{0,123})\mb{b},
\end{alignat}
which can easily be solved analytically.
\subsubsection{Discrete gradient method}
Denoting the $n$- th time step by $t_n=n\Delta t$ and using an implicit midpoint rule gives the energy conserving method
\begin{align}
\frac{\mb{S}^{n+1}-\mb{S}^n}{\Delta t}=\frac{1}{2}\mathbb{A}(\mb{S}^{n+1}+\mb{S}^n)\quad\Leftrightarrow\quad (\mathbb{I}-\frac{\Delta t}{2}\mathbb{A})\mb{S}^{n+1}=(\mathbb{I}+\frac{\Delta t}{2}\mathbb{A})\mb{S}^n,
\end{align}
which allows larger time steps but has the drawback of solving a linear system in each time step.
\section{Test case 2: Full system}
As a next step, we can simulate the full system, however still in slab geometry with periodic boundary conditions and a homogeneous equilibrium. Without loss of generality we can align the $z$-axis of our coordinate system to the direction of wave propagation ($\mb{k}=k\mb{e}_z$). In this case one can derive the dispersion relations
\begin{align}
&\omega_{\mr{M}}^\pm(k)^2=\frac{1}{2}k^2(c_\mr{S}^2+v_\mr{A}^2)(1\pm\sqrt{1-\delta}),\quad\delta=\frac{4B_{0z}^2c_\mr{S}^2v_\mr{A}^2}{(c_\mr{S}^2+v_\mr{A}^2)^2B_0^2},\quad c_\mr{S}^2=\frac{\gamma p_0}{\rho_0},\quad v_\mr{A}^2=\frac{B_0^2}{\mu_0\rho_0}, \\
&\omega_\mr{S}^2(k)=v_\mr{A}^2k^2\cos^2\vartheta=v_\mr{A}^2k^2\frac{B_{0z}}{B_0},
\end{align}
where we have defined the two characteristic veclocities in the system, that is the sound speed $c_\mr{S}$ and the Alfv\'{e}n velocity $\mr{v}_\mr{A}$, respectively. Hence there can be three types of waves depending on the orientation of the background magnetic field $\mb{B}_0$. The two waves corresponding to $\omega_\mr{M}^\pm$ are referred to as the fast (+) and slow (-) magnetosonic wave, whereas the wave corresponding to $\omega_\mr{S}$ is the already known shear Alfv\'{e}n wave. In the special case of $\mb{B}_0=B_0\mb{e}_z$, the slow magnetosonic wave is just a "normal" sound wave with perturbations parallel to the direction of propagation (longitudinal wave) and the fast magnetosonic wave is identical to the shear Alfv\'{e}n wave which is a transverse wave. If the background magnetic field is perpendicular to the direction of propagation, i.e. lies in the $xy$-plane, there is only the fast magnetosonic wave (or compressional Alfv\'{e}n wave. The semi-discrete system which should be able to describe these waves reads
\begin{align}
\frac{\pa}{\pa t}\begin{pmatrix}
\bo{\rho} \\ \mb{u} \\ \mb{b} \\ \mb{p}
\end{pmatrix}=
\begin{pmatrix}
0 & -\mathbb{D}\mathcal{Q} & 0 & 0 \\
0 & 0 &\sqrt{g}(\mathbb{M}^1)^{-1}\mathcal{T}^\top\mathbb{C}^\top\mathbb{M}^2/(\mu_0\rho_{0,123}) &-\sqrt{g}\mathbb{G}/\rho_{0,123} \\
0 &-\mathbb{C}\mathcal{T} &0 &0 \\
0 &p_0\gamma(\mathbb{M}^0)^{-1}\mathbb{G}^\top\mathbb{M}^1 &0 &0
\end{pmatrix}\begin{pmatrix}
\bo{\rho} \\ \mb{u} \\ \mb{b} \\ \mb{p}
\end{pmatrix}.
\end{align}
We shall test different time integration schemes for this system. However, let us have a look at the energy
\begin{align}
\mathcal{H}&=\mathcal{H}_U+\mathcal{H}_B+\mathcal{H}_p=\frac{1}{2}(\ast\rho_0 U^1,U^1)+\frac{1}{2\mu_0}(B^2,B^2)+\frac{1}{\gamma-1}(p^0,1)\\
&\approx\frac{\rho_{0,123}}{2\sqrt{g}}\mb{u}^\top\mathbb{M}^1\mb{u}+\frac{1}{2\mu_0}\mb{b}^\top\mathbb{M}^2\mb{b}+\frac{1}{\gamma-1}\mb{p}^\top\mathbb{M}^0\mb{1}\\
\Rightarrow\quad\frac{\mr{d}\mathcal{H}}{\mr{d}t}&=-\mb{u}^\top\mathbb{M}^1\mathbb{G}\mb{p}+\frac{\gamma}{\gamma-1}\mb{u}^\top\mathbb{M}^1\mathbb{G}\mb{p}_0
\end{align}
\section{Adding a kinetic species}
\subsection{Full orbit current coupling scheme}
We shall first study the coupling of ideal MHD to the full Vlasov equation and compare the results later to the reduced kinetic models. Although we have $\mb{E}=-\mb{U}\times\mb{B}$ in ideal MHD, we assume for the moment a general electric field and thus describe the kinetic component by
\begin{align}
\frac{\pa f_\mr{h}}{\pa t}+\mb{v}\cdot\nabla f_\mr{h}+\frac{q_\mr{h}}{m_\mr{h}}(\mb{E}+\mb{v}\times\mb{B})\cdot\nabla_\mb{v}f_\mr{h}=0,
\end{align}
which, after linearization about an homogeneous (in space) equilibrium $f_\mr{h}^0=f_\mr{h}(v_\parallel, v_\perp)$, amounts to
\begin{align}
\frac{\pa f_\mr{h}}{\pa t}+\mb{v}\cdot\nabla f_\mr{h}+\Omega_\mr{ch}(\mb{v}\times\mb{e}_z)\cdot\nabla_\mb{v}f_\mr{h}=-\frac{q_\mr{h}}{m_\mr{h}}(\mb{E}+\mb{v}\times\mb{B})\cdot\nabla_\mb{v}f_\mr{h}^0.
\end{align}
The solution of this equation leads to an Ohm's law of the form
\begin{align}
\hat{\mb{j}}_\mr{h}=\begin{pmatrix}
-i\frac{q_\mr{h}^2}{m_\mr{h}}\int\frac{v_\perp(\omega-kv_\parallel)\hat{G}f_\mr{h}^0}{2\Omega_+\Omega_-}\mr{d}^3v &\frac{q_\mr{h}^2}{m_\mr{h}}\Omega_\mr{ch}\int\frac{v_\perp\hat{G}f_\mr{h}^0}{2\Omega_+\Omega_-}\mr{d}^3v &0 \\
-\frac{q_\mr{h}^2}{m_\mr{h}}\Omega_\mr{ch}\int\frac{v_\perp\hat{G}f_\mr{h}^0}{2\Omega_+\Omega_-}\mr{d}^3v &-i\frac{q_\mr{h}^2}{m_\mr{h}}\int\frac{v_\perp(\omega-kv_\parallel)\hat{G}f_\mr{h}^0}{2\Omega_+\Omega_-}\mr{d}^3v &0 \\
0 &0 &-i\frac{q_\mr{h}^2}{m_\mr{h}}\int\frac{v_\parallel\pa_\parallel f_\mr{h}^0}{\omega-kv_\parallel}\mr{d}^3v
\end{pmatrix}\hat{\mb{E}}=:\sigma_\mr{h}(k,\omega)\hat{\mb{E}}.
\end{align}
We make now use of the fact that $\hat{\mb{E}}=-B_0\hat{\mb{U}}\times\mb{e}_z$ in ideal MHD. This yields a pure transverse current with components
\begin{align}
&\hat{j}_{\mr{h}x}=-\sigma_{\mr{h}xx}B_0U_y+\sigma_{\mr{h}xy}B_0U_x \\
&\hat{j}_{\mr{h}y}=-\sigma_{\mr{h}yx}B_0U_y+\sigma_{\mr{h}yy}B_0U_x.
\end{align}
The momentum balance equation then reads
\begin{align}
&-i\omega\hat{\mb{U}}+i\frac{v_\mr{A}^2k^2}{\omega}\hat{\mb{U}}_\perp+i\frac{c_\mr{S}^2k^2}{\omega}\mb{e}_z\hat{U}_\parallel=\nu_\mr{h}\Omega_\mr{ch}\hat{\mb{U}}\times\mb{e}_z+\frac{B_0}{\rho_0}(\mb{e}_z\times\hat{\mb{j}}_\mr{h}) \\
\Leftrightarrow \quad &\begin{pmatrix}
\omega^2-v_\mr{A}^2k^2+i\mu_0v_\mr{A}^2\sigma_{\mr{h}yy}\omega &-i(\nu_\mr{h}\Omega_\mr{ch}\omega+\mu_0v_\mr{A}^2\sigma_{\mr{h}yx}\omega) &0 \\
i(\nu_\mr{h}\Omega_\mr{ch}\omega-\mu_0v_\mr{A}^2\sigma_{\mr{h}xy}\omega) &\omega^2-v_\mr{A}^2k^2+i\mu_0v_\mr{A}^2\sigma_{\mr{h}xx}\omega &0 \\
0 &0 &\omega^2-c_\mr{S}^2k^2
\end{pmatrix}\begin{pmatrix}
\hat{U}_x \\ \hat{U}_y \\ \hat{U}_\parallel
\end{pmatrix}=0,
\end{align}
which leads to the dispersion relations
\begin{align}
D_\mr{R/L}(k,\omega)=1-\frac{v_\mr{A}^2k^2}{\omega^2}\pm\frac{\nu_\mr{h}\Omega_\mr{ch}}{\omega}+\frac{\nu_\mr{h}\Omega_\mr{ch}^2Z_\mr{h}^2}{A_\mr{h}\omega}\int\frac{v_\perp}{2}\frac{\hat{G}F_\mr{h}^0}{\omega-kv_\parallel\pm\Omega_\mr{ch}}\mr{d}^3v=0.
\end{align}
Assuming a shifted, isotropic Maxwellian of the form
\begin{align}
&F_\mr{h}^0=\frac{1}{\pi^{3/2}v_\mr{th}^3}\exp\left(-\frac{(v_\parallel-v_0)^2+v_\perp^2}{v_\mr{th}^2}\right)\\
&\Rightarrow \frac{\pa F_\mr{h}^0}{\pa v_\parallel}=-F_\mr{h}^0\frac{2}{v_\mr{th}^2}(v_\parallel-v_0)\\
&\Rightarrow \frac{\pa F_\mr{h}^0}{\pa v_\perp}=-F_\mr{h}^0\frac{2}{v_\mr{th}^2}v_\perp\\
&\Rightarrow \hat{G}F_\mr{h}^0=-F_\mr{h}^0\frac{2}{v_\mr{th}^2}v_\perp+\frac{k}{\omega}F_\mr{h}^0\frac{2}{v_\mr{th}^2}v_\perp v_0\\
&\Rightarrow \pi\int_0^\infty\frac{1}{\pi^{3/2}v_\mr{th}^3}\begin{pmatrix}
1 \\ v_\perp \\ v_\perp^2 \\ v_\perp^3
\end{pmatrix}\exp\left(-\frac{v_\perp^2}{v_\mr{th}^2}\right)\mr{d}v_\perp=\begin{pmatrix}
1/(2v_\mr{th}^2)\\
1/(2v_\mr{th}\sqrt{\pi}) \\
1/4\\
v_\mr{th}/(2\sqrt{\pi})
\end{pmatrix},
\end{align}
finally leads to
\begin{align}
D_\mr{R/L}(k,\omega)=1-\frac{v_\mr{A}^2k^2}{\omega^2}\pm\frac{\nu_\mr{h}\Omega_\mr{ch}}{\omega}+\frac{\nu_\mr{h}\Omega_\mr{ch}^2Z_\mr{h}^2}{A_\mr{h}\omega^2}\frac{\omega-kv_0}{k v_\mr{th}}Z(\xi^\pm),
\end{align}
where $Z$ is the plasma dispersion function and $\xi\pm=(\omega-kv_0\pm\Omega_\mr{ch})/(kv_\mr{th})$. Compared to the pure MHD we now have to deal with a new linearized  momentum balance equation of the form
\begin{align}
&\rho_0\frac{\pa\mb{U}}{\pa t}=\frac{1}{\mu_0}(\nabla\times\mb{B}_0)\times\mb{B}+\frac{1}{\mu_0}(\nabla\times\mb{B})\times\mb{B}_0-\nabla p+q_\mr{h}n_\mr{h0}\mb{U}\times\mb{B}-\mb{j}_\mr{h}\times\mb{B}\\
\Leftrightarrow\quad &(\ast\rho_0)\frac{\pa U^1}{\pa t}+\mr{d}p^0=\frac{1}{\mu_0}i_{\#\ast B_0}\mr{d}\ast B^2-q_\mr{h}(\ast n_\mr{h}^3)i_{\#U^1}B^2+i_{\#j_\mr{h}^1}B^2.
\end{align}
We assume a particle-like distribution function in physical space, i.e.
\begin{align}
f_\mr{h}=f_\mr{h}(t,\mb{x},\mb{v})\approx \sum_k w_k\delta(\mb{x}-\mb{x}_k(t))\delta(\mb{v}-\mb{v}_k(t)),
\end{align}
which means that the current density and number density in physical space are given by
\begin{align}
&\tilde{\mb{j}}_\mr{h}(t,\mb{x})=q_\mr{h}\sum_k w_k\mb{v}_k(t)\delta(\mb{x}-\mb{x}_k(t)), \\
&\tilde{n}_\mr{h}(t,\mb{x})=\sum_k w_k\delta(\mb{x}-\mb{x}_k(t)).
\end{align}
Since there is no difference between vectors and forms in physical space, these expression are are as well the components of 1-form current density and the 3-form number density. To get the components on the logical domain we apply the transformation formulas for 1-forms, respectively 3-forms:
\begin{align}
&\mb{j}_\mr{h}(t,\mb{q})=DF^\top\tilde{\mb{j}}_\mr{h}(t, F(\mb{q}))=q_\mr{h}\frac{1}{\sqrt{g}}DF^\top\sum_k w_k\mb{v}_k(t)\delta(\mb{q}-\mb{q}_k(t)),\\
&n_\mr{h,123}(t,\mb{q})=\sqrt{g}\tilde{n}_\mr{h}(t,F(\mb{q}))=\sum_k w_k\delta(\mb{q}-\mb{q}_k(t)),
\end{align}
where we made use of the transformation formula
\begin{align}
\delta(\mb{x}-\mb{x}_k(t))=\frac{1}{\sqrt{g}}\delta(\mb{q}-\mb{q}_k(t)).
\end{align}
Using these results in the weak fromulation of the momentum balance equation we get
\begin{align}
&\left(i_{\#j_\mr{h}^1}B^2,V^1\right)=\int_{\hat{\Omega}}(\hat{\mb{B}}\times G^{-1}\mb{j}_\mr{h})^\top G^{-1}\mb{V}\sqrt{g}\,\mr{d}^3q\approx q_\mr{h}\sum_k w_k(\hat{\mb{B}}_h(\mb{q}_k)\times DF^{-1}\mb{v}_k)^\top G^{-1}\mb{V}_h(\mb{q}_k), \\
&q_\mr{h}\left((\ast n_\mr{h}^3)i_{\#U^1}B^2,V^1\right)=q_\mr{h}\int_{\hat{\Omega}}\left(\frac{1}{\sqrt{g}}n_\mr{h,123}\hat{\mb{B}}\times G^{-1}\mb{U}\right)^\top G^{-1}\mb{V}\sqrt{g}\,\mr{d}^3q\\ &\approx q_\mr{h}\sum_k w_k(\hat{\mb{B}}_h(\mb{q}_k)\times G^{-1}\mb{U}_h(\mb{q}_k))^\top G^{-1}\mb{V}_h(\mb{q}_k)
\end{align}
We define the antisymmetric block matrix 
\begin{align}
\hat{\mathbb{B}}=\hat{\mathbb{B}}(\mb{b},\mb{Q})=\begin{pmatrix}
0 &-\text{diag}[\mathbb{P}^{2\top}_3(\mb{Q})\mb{b}_3] &\text{diag}[\mathbb{P}^{2\top}_2(\mb{Q})\mb{b}_2] \\
\text{diag}[\mathbb{P}^{2\top}_3(\mb{Q})\mb{b}_3] &0 &-\text{diag}[\mathbb{P}^{2\top}_1(\mb{Q})\mb{b}_1] \\
-\text{diag}[\mathbb{P}^{2\top}_2(\mb{Q})\mb{b}_2] &\text{diag}[\mathbb{P}^{2\top}_1(\mb{Q})\mb{b}_1] &0
\end{pmatrix}\in\mathbb{R}^{3N_p\times 3N_p},
\end{align}
where $(\mathbb{P}^2_{1/2/3})_{ik}=\Lambda^2_{1/2/3,i}(\mb{q}_k)\in\mathbb{R}^{N\times N_p}$. With this we can write
\begin{align}
&\left(i_{\#j_\mr{h}^1}B^2,V^1\right)\approx q_\mr{h}\mb{v}^\top\mathbb{P}^1\mathbb{W}\hat{\mathbb{B}}\mb{V}\quad\forall \mb{v}\in\mathbb{R}^{3N},
\end{align}
where 
\begin{align}
&\mb{V}=(v_{1x},v_{2x},\ldots,v_{N_px},v_{1y},v_{2y},\ldots,v_{N_py},v_{1z},v_{2z},\ldots,v_{N_pz})^\top\in\mathbb{R}^{3N_p},\\
&\mathbb{W}=\begin{pmatrix}
\text{diag}(w_1,\ldots,w_{N_p}) & & \\
&\text{diag}(w_1,\ldots,w_{N_p}) &  \\
& &\text{diag}(w_1,\ldots,w_{N_p})
\end{pmatrix}\in\mathbb{R}^{3N_p\times 3N_p},\\
&\mathbb{P}^1=\text{diag}(\mathbb{P}^1_{1/2/3})\in\mathbb{R}^{3N\times 3N_p}.
\end{align}
The second term amounts to
\begin{align}
q_\mr{h}\left((\ast n_\mr{h}^3)i_{\#U^1}B^2,V^1\right)\approx q_\mr{h}\mb{v}^\top\mathbb{P}^1\mathbb{W}\hat{\mathbb{B}}\mathbb{P}^{1\top}\mb{u}\quad\forall\mb{v}\in\mathbb{R}^{3N}
\end{align}
which is clearly antisymmetric. Finally, we have to write the particles' equations of motion, which read for a single particle with logical coordinate $\mb{q}_k$ and physical velocity $\mb{v}_k$
\begin{align}
&\frac{\mr{d}\mb{q}_k}{\mr{d}t}=DF^{-1}\mb{v}_k,\\
&\frac{\mr{d}\mb{v}_k}{\mr{d}t}=\frac{q_\mr{h}}{m_\mr{h}}\left(\frac{1}{\sqrt{g}}DF\hat{\mb{B}}_h(\mb{q}_k)\times DF^{-\top}\mb{U}_h(\mb{q}_k)-\frac{1}{\sqrt{g}}DF\hat{\mb{B}}_h(\mb{q}_k)\times\mb{v}_k\right).\\
\Leftrightarrow \quad &\frac{\mr{d}\mb{Q}}{\mr{d}t}=\mb{V},\\
&\frac{\mr{d}\mb{V}}{\mr{d}t}=\frac{q_\mr{h}}{m_\mr{h}}\left(\hat{\mathbb{B}}\mathbb{P}^{1\top}\mb{u}-\hat{\mathbb{B}}\mb{V}\right).
\end{align}
Let us introduce the discrete Hamiltonian in the case of incompressible shear Alfv\'{e}n waves
\small
\begin{align}
&\mathcal{H}=\mathcal{H}_U+\mathcal{H}_B+\mathcal{H}_\mr{h}\approx\frac{1}{2}\rho_0\,\mb{u}^\top\mathbb{M}^1\mb{u}+\frac{1}{2\mu_0}\mb{b}^\top\mathbb{M}^2\mb{b}+\frac{m_\mr{h}}{2}\mb{V}^\top\mathbb{W}\mb{V},\\
\Rightarrow\quad &\nabla\mathcal{H}=\begin{pmatrix}
\rho_0\mathbb{M}^1\mb{u} \\
\mathbb{M}^2\mb{b}/\mu_0 \\
0 \\
m_\mr{h}\mathbb{W}\mb{V}
\end{pmatrix},\\
\Rightarrow &\frac{\mr{d}}{\mr{d}t}\begin{pmatrix}
\mb{u} \\ \mb{b} \\ \mb{Q} \\ \mb{V}
\end{pmatrix}=
\begin{pmatrix}
q_\mr{h}/\rho_0^2\mathbb{M}^{-1}\mathbb{P}^1\mathbb{W}\hat{\mathbb{B}}\mathbb{P}^{1\top}\mathbb{M}^{-1} &\mathbb{M}^{-1}\mathcal{T}^\top\mathbb{C}^\top/\rho_0 &0 &q_\mr{h}/(\rho_0m_\mr{h})\mathbb{M}^{-1}\mathbb{P}^1\hat{\mathbb{B}}\\
-\mathbb{C}\mathcal{T}\mathbb{M}^{-1}/\rho_0 &0 &0 &0 \\
0 &0 &0 &1/\mr{m}_\mr{h}\mathbb{W}^{-1} \\
q_\mr{h}/(\rho_0m_\mr{h})\hat{\mathbb{B}}\mathbb{P}^{1\top}\mathbb{M}^{-1} &0 &-1/m_\mr{h}\mathbb{W}^{-1} &q_\mr{h}/m_\mr{h}^2\hat{\mathbb{B}}\mathbb{W}^{-1}
\end{pmatrix}\begin{pmatrix}
\rho_0\mathbb{M}^1\mb{u} \\
\mathbb{M}^2\mb{b}/\mu_0 \\
0 \\
m_\mr{h}\mathbb{W}\mb{V}
\end{pmatrix}.
\end{align}
\normalsize
We end up with an antisymmetric Poisson matrix meaning that the discrete Hamiltonian is conserved.
\subsection{Current coupling scheme}
In this section we add a species of fast particles described by a drift-kinetic equation (in the following DK equation) to the ideal MHD equations and derive the corresponding dispersion relation for parallel wave propagation to look for possible instabilities. The DK equation can be obtained by a guiding center approach where the fast gyromotion around the magnetic field lines is systematically eliminated such that one only works on time scales governed by the slower drift motion of the guiding center. The DK equation for the drift-kinetic distribution function $f^\mr{D}_\mr{h}=\fh(\mb{x},v_\parallel,\hat{\mu},t)$ (the "h" stands for "hot") reads
\begin{align}
\frac{\pa\fh}{\pa t}+\mb{u}_\mr{g}\cdot\nabla\fh+E_\mr{g}\frac{\pa\fh}{\pa v_\parallel}=0,
\end{align} 
where
\begin{align}
\mb{u}_\mr{g}=v_\parallel\frac{\mb{B}^\ast}{B_\parallel^\ast}+\frac{\mb{E}^\ast\times\mb{b}}{B_\parallel^\ast},\quad E_\mr{g}=\frac{q_\mr{h}}{m_\mr{h}}\frac{\mb{E}^\ast\cdot \mb{B}^\ast}{B_\parallel^\ast}.
\end{align}
$q_\mr{h}$ and $m_\mr{h}$ stand for the hot particle's charge and mass, respectively, and
\begin{subequations}
\begin{align}
&\mb{B}^\ast=\mb{B}+\frac{m_\mr{h}}{q_\mr{h}}v_\parallel\nabla\times\mb{b},\quad\mb{b}=\mb{B}/B,\quad\Bpa=\mb{B}^\ast\cdot\mb{b},\\
&\mb{E}^\ast=-\mb{U}\times\mb{B}-\frac{m_\mr{h}}{q_\mr{h}}\hat{\mu}\nabla B-\frac{m_\mr{h}}{q_\mr{h}}v_\parallel\frac{\pa\mb{b}}{\pa t},\\
&n_\mr{h}^\mr{D}=\int_{\hat{\mu}}\fh B_\parallel^\ast\mr{d}v_\parallel, \\
&\mb{j}_\mr{h}^\mr{D}=q_\mr{h}\int_{\hat{\mu}}\mb{u}_\mr{g}\fh\Bpa\mr{d}v_\parallel+\nabla\times\mb{M},\\
&\mb{M}=-m_\mr{h}\int_{\hat{\mu}}\left[\hat{\mu}\mb{b}-\frac{v_\parallel}{B}\mb{u}_{\mr{g}\perp}\right]\fh\Bpa\mr{d}v_\parallel,\\
&\mb{u}_\mr{g\perp}=v_\parallel\frac{\mb{B}_\perp^\ast}{B_\parallel^\ast}+\frac{\mb{E}^\ast\times\mb{b}}{B_\parallel^\ast}=-\frac{m_\mr{h}v_\parallel^2}{q_\mr{h}B_\parallel^\ast}\mb{b}\times\mb{b}\times\nabla\times\mb{b}+\frac{\mb{E}^\ast\times\mb{b}}{B_\parallel^\ast}.
\end{align}
\end{subequations}
For integrations over velocity space we introduced the abbreveation
\begin{align}
\int_{\hat{\mu}}\ldots\mr{d}v_\parallel=2\pi\int_{-\infty}^\infty\int_0^\infty\ldots\mr{d}v_\parallel\mr{d}\hat{\mu}.
\end{align}
The coupling to the fluid bulk plasma is done via a current-coupling scheme (CCS) which leads to a modfied momentum balance equation of the form
\begin{align}
\frac{\pa\mb{U}}{\pa t}+(\mb{U}\cdot\nabla)\mb{U}=\frac{1}{\mu_0}\frac{\nabla\times\mb{B}}{\rho}\times\mb{B}-\frac{\nabla p}{\rho}+\frac{q_\mr{h}n_\mr{h}^\mr{D}\mb{U}-\mb{j}_\mr{h}^\mr{D}}{\rho}\times\mb{B}.
\end{align}
We linearize the extended model about a known homogeneous (in space) equilibrium, i.e. we write $\fh=\fho+\fhi$ and use a shifted Maxwellian for the DK equilibrium distribution function:
\begin{align}
&\fho=n_{\mr{h}0}\left(\frac{m_\mr{h}}{2\pi T_\parallel}\right)^{3/2}\exp\left(-\frac{m_\mr{h}(v_\parallel-v_0)^2+2m_\mr{h}\hat{\mu}B_0}{2T_\parallel}\right)\\
&\Rightarrow \int_{\hat{\mu}}\begin{pmatrix}
1\\v_\parallel \\v_\parallel^2 \\v_\parallel^3
\end{pmatrix}\fho B_0\mr{d}v_\parallel=\begin{pmatrix}
n_\mr{h0}\\ n_\mr{h0}v_0\\n_\mr{h0}T_\parallel/m_\mr{h}+n_\mr{h0}v_0^2\\ 3v_0n_\mr{h0}T_\parallel/m_\mr{h}+n_\mr{h0}v_0^3 
\end{pmatrix}=:\begin{pmatrix}
n_\mr{h0}\\ n_\mr{h0}v_0\\ p_\parallel/m_\mr{h}  \\ \tilde{p}_\parallel/m_\mr{h}
\end{pmatrix}\\
&\Rightarrow  \int_{\hat{\mu}}\hat{\mu}\fho B_0\mr{d}v_\parallel=\frac{n_\mr{h0}T_\parallel}{m_\mr{h}B_0}=:\frac{p_\perp}{m_\mr{h}B_0}.
\end{align}
Furthermore, we assume the magnetic field to point in $z$-direction, i.e. we have $\mb{B}=B_0\mb{e}_z+\mb{B}_1$.
\subsubsection{Linearized drift-kinetic equation}
Let us start with the calculation of the linearized DK equation. After seperating the distribution function into an equilibrium part and a small fluctuation and upon neglecting nonlinear terms in the latter, we obtain
\begin{align}
\frac{\pa \fhi}{\pa t}+\mb{u}_\mr{g0}\cdot\nabla\fhi=-E_\mr{g1}\frac{\pa\fho}{\pa v_\parallel}.
\end{align}
To find $\mb{u}_\mr{g0}$ and $E_\mr{g1}$ we have to systematically linerize all quantities related to the magnetic field. To identify orders in a Taylor expansion more easily, we use the small dimensionless parameter $\epsilon\ll 1$ to write
\begin{align}
&B=\sqrt{(\mb{B}_0+\epsilon\mb{B}_1)^2}=\sqrt{B_0^2+2\epsilon\mb{B}_0\cdot\mb{B}_1+\epsilon^2 B_1^2}\\
\Rightarrow\quad &\frac{1}{B}=\frac{1}{B_0}\frac{1}{\sqrt{1+2\epsilon\mb{B}_0\cdot\mb{B}_1/B_0^2+\epsilon^2 B_1^2/B_0^2}}=\frac{1}{B_0}(1-\epsilon\mb{B}_0\cdot\mb{B}_1/B_0^2+O(\epsilon^2)) \\
\Rightarrow \quad &\mb{b}=(\mb{B}_0+\epsilon\mb{B}_1)\frac{1}{B_0}(1-\epsilon\mb{B}_0\cdot\mb{B}_1/B_0^2+O(\epsilon^2))=\frac{\mb{B}_0}{B_0}+\epsilon\left(\frac{\mb{B}_1}{B_0}-\frac{\mb{B}_0}{B_0}\frac{\mb{B}_0\cdot\mb{B}_1}{B_0^2}\right)+O(\epsilon^2)\\
\Rightarrow\quad &\mb{b}_0=\frac{\mb{B}_0}{B_0},\quad\mb{b}_1=\frac{\mb{B}_1}{B_0}-\mb{b}_0\frac{\mb{B}_0\cdot\mb{B}_1}{B_0}.
\end{align}
This result we can use to linearize the expressions for $\mb{B}^\ast$ and $\mb{E}^\ast$:
\begin{align}
&\mb{B}^\ast=\mb{B}_0+\epsilon\mb{B}_1+\frac{m_\mr{h}}{q_\mr{h}}v_\parallel\nabla\times\mb{b}_0+\epsilon\frac{m_\mr{h}}{q_\mr{h}}v_\parallel\nabla\times\mb{b}_1+O(\epsilon^2)\\
\Rightarrow\quad&\mb{B}^\ast_0=\mb{B}_0+\frac{m_\mr{h}}{q_\mr{h}}v_\parallel\nabla\times\mb{b}_0,\quad\mb{B}_1^\ast=\mb{B}_1+\frac{m_\mr{h}}{q_\mr{h}}v_\parallel\nabla\times\mb{b}_1.
\end{align}
For the calculation of $\mb{E}^\ast$ we need to linearize the modulus of $\mb{B}$:
\begin{align}
B=B_0\sqrt{1+2\epsilon\mb{B}_0\cdot\mb{B}_1/B_0^2+\epsilon^2 B_1^2/B_0^2}=B_0+\epsilon\mb{b}_0\cdot\mb{B}_1+O(\epsilon^2).
\end{align}
Plugging this into the expression for $\mb{E}^\ast$ yields
\begin{alignat}{2}
&\mb{E}^\ast&&=-\epsilon\mb{U}_1\times(\mb{B}_0+\epsilon\mb{B}_1)-\frac{m_\mr{h}}{q_\mr{h}}\hat{\mu}\nabla(B_0+\epsilon\mb{b}_0\cdot\mb{B}_1)-\frac{m_\mr{h}}{q_\mr{h}}v_\parallel\frac{\pa}{\pa t}(\mb{b}_0+\epsilon\mb{b}_1)+O(\epsilon^2)\\
& &&=-\epsilon\mb{U}_1\times\mb{B}_0-\frac{m_\mr{h}}{q_\mr{h}}\hat{\mu}\nabla(B_0+\epsilon\mb{b}_0\cdot\mb{B}_1)-\epsilon\frac{m_\mr{h}}{q_\mr{h}}v_\parallel\frac{\pa\mb{b}_1}{\pa t}+O(\epsilon^2)\\
\Rightarrow\quad&\mb{E}^\ast_0&&=-\frac{m_\mr{h}}{q_\mr{h}}\hat{\mu}\nabla B_0,\quad\mb{E}^\ast_1=-\mb{U}_1\times\mb{B}_0-\frac{m_\mr{h}}{q_\mr{h}}\hat{\mu}\nabla(\mb{b}_0\cdot\mb{B}_1)-\frac{m_\mr{h}}{q_\mr{h}}v_\parallel\frac{\pa\mb{b}_1}{\pa t}.
\end{alignat}
For the inverse of the parallel projection of $\mb{B}^\ast$ we find
\begin{align}
\frac{1}{B_\parallel^\ast}=\frac{1}{\mb{B}_0^\ast\cdot\mb{b}_0}\left[1-\epsilon\left(\frac{\mb{B}_0^\ast\cdot\mb{b}_1}{\mb{B}_0^\ast\cdot\mb{b}_0}+\frac{\mb{B}_1^\ast\cdot\mb{b}_0}{\mb{B}_0^\ast\cdot\mb{b}_0}\right)\right]+O(\epsilon^2).
\end{align}
This finally yields
\begin{align}
\mb{u}_\mr{g0}&=v_\parallel\frac{\mb{B}_0}{\mb{B}_0^\ast\cdot\mb{b}_0}+\frac{1}{\mb{B}_0^\ast\cdot\mb{b}_0}\frac{m_\mr{h}}{q_\mr{h}}v_\parallel^2\nabla\times\mb{b}_0-\frac{1}{\mb{B}_0^\ast\cdot\mb{b}_0}\frac{m_\mr{h}}{q_\mr{h}}\hat{\mu}\nabla B_0\times \mb{b}_0,\\
\mb{u}_\mr{g1}&=-v_\parallel\frac{\mb{B}_0^\ast}{\mb{B}_0^\ast\cdot\mb{b}_0}\left(\frac{\mb{B}_0^\ast\cdot\mb{b}_1}{\mb{B}_0^\ast\cdot\mb{b}_0}+\frac{\mb{B}_1^\ast\cdot\mb{b}_0}{\mb{B}_0^\ast\cdot\mb{b}_0}\right)+v_\parallel\frac{\mb{B}_1^\ast}{\mb{B}_0^\ast\cdot\mb{b}_0}-\frac{\mb{E}_0^\ast\times\mb{b}_0}{\mb{B}_0^\ast\cdot\mb{b}_0}\left(\frac{\mb{B}_0^\ast\cdot\mb{b}_1}{\mb{B}_0^\ast\cdot\mb{b}_0}+\frac{\mb{B}_1^\ast\cdot\mb{b}_0}{\mb{B}_0^\ast\cdot\mb{b}_0}\right)\\
&\phantom{=}+\frac{\mb{E}_0^\ast\times\mb{b}_1}{\mb{B}_0^\ast\cdot\mb{b}_0}+\frac{\mb{E}_1^\ast\times\mb{b}_0}{\mb{B}_0^\ast\cdot\mb{b}_0}
\end{align}
for the guiding center velocity and for the acceleration we obtain
\begin{align}
&E_\mr{g0}=\frac{q_\mr{h}}{m_\mr{h}}\frac{\mb{E}_0^\ast\cdot\mb{B}_0^\ast}{\mb{B}_0^\ast\cdot\mb{b}_0},\\
&E_\mr{g1}=-\frac{q_\mr{h}}{m_\mr{h}}\frac{\mb{E}_0^\ast\cdot\mb{B}_0^\ast}{\mb{B}_0^\ast\cdot\mb{b}_0}\left(\frac{\mb{B}_0^\ast\cdot\mb{b}_1}{\mb{B}_0^\ast\cdot\mb{b}_0}+\frac{\mb{B}_1^\ast\cdot\mb{b}_0}{\mb{B}_0^\ast\cdot\mb{b}_0}\right)+\frac{q_\mr{h}}{m_\mr{h}}\frac{\mb{E}_0^\ast\cdot\mb{B}_1^\ast}{\mb{B}_0^\ast\cdot\mb{b}_0}+\frac{q_\mr{h}}{m_\mr{h}}\frac{\mb{E}_1^\ast\cdot\mb{B}_0^\ast}{\mb{B}_0^\ast\cdot\mb{b}_0}.
\end{align}
As a next step, we make use of our specific choice for the background magnetic field to be uniform and to point in $z$-direction, i.e, we have $\mb{b}_0=\mb{e}_z$. This implies
\begin{align}
&\mb{E}_0^\ast=0, \\
&\mb{B}_0^\ast=\mb{B}_0.
\end{align}
Moreover, we assume parallel wave propagation which is equivalent to allow for variations in $z$-direction only. Hence we have $\nabla=\mb{e}_z\pa_z$. From Faraday's law one can deduce that this results in $\pa_t B_{1z}=0$, i.e. if we choose $B_{1z}(t=0)=0$ as an initial condition, this will remain true for all later times. Thus, we only have to deal with perpendicular disturbances with respect to the background magnetic field ($\mb{B}_1\perp\mb{B}_0,\mb{b}_0$) and we find the following simpler expressions:
\begin{align}
&\mb{b}_0=\mb{e}_z,\quad\mb{b}_1=\mb{B}_1/B_0,\\
&B=B_0+O(\epsilon^2),\quad 1/B=1/B_0+O(\epsilon^2)\\
&\mb{E}_1^\ast=-\mb{U}_1\times\mb{B}_0-\frac{m_\mr{h}}{q_\mr{h}B_0}v_\parallel\frac{\pa\mb{B}_1}{\pa t},\\
&B_\parallel^\ast=B_0+O(\epsilon^2),\quad 1/B_\parallel^\ast= 1/B_0+O(\epsilon^2),\\
&\mb{u}_\mr{g0}=v_\parallel\mb{e}_z,\\
&\mb{u}_\mr{g1}=v_\parallel\frac{\mb{B}_1}{B_0}+\frac{m_\mr{h}}{q_\mr{h}B_0^2}v_\parallel^2\nabla\times\mb{B}_1+\mb{U}_{1\perp}-\frac{m_\mr{h}}{q_\mr{h}B_0^2}v_\parallel\frac{\pa\mb{B}_1}{\pa t}\times\mb{e}_z,\\
&E_\mr{g0}=E_\mr{g1}=0.
\end{align}
Consequently, we obtain for the DK equation
\begin{align}
\frac{\pa \fhi}{\pa t}+v_\parallel\pa_z\fhi=0\quad\Rightarrow\quad\fhi=f_\mr{h1}^\mr{D0}(x,y,z-v_\parallel t,v_\parallel,\hat{\mu}),
\end{align}
where $f_\mr{h1}^\mr{D0}$ is the initial condition for the perturbation of the DK distribution function.
\subsubsection{Linearized drift-kinetic density}
For the coupling to the fluid momentum equation via the current-coupling approach, we need to take velocity moments of the DK distribution function. For the zeroth moment, which is the number density $n_\mr{h}^\mr{D}$, we simply get
\begin{align}
n_\mr{h}^\mr{D}=\int_{\hat{\mu}}(\fho+\epsilon \fhi)B_0\mr{d}v_\parallel+O(\epsilon^2)=n_\mr{h0}+\epsilon\int_{\hat{\mu}}\fhi B_0\mr{d}v_\parallel+O(\epsilon^2).
\end{align}
\subsubsection{Linearized magnetization}
For the calculation of the linearized magnetization, we first need to linearize the perpendicular component of the guiding-center velocity
\begin{align}
\mb{u}_\mr{g\perp}=\epsilon\frac{m_\mr{h}}{q_\mr{h}B_0^2}v_\parallel^2\nabla\times\mb{B}_1+\epsilon\mb{U}_{1\perp}-\epsilon\frac{m_\mr{h}}{q_\mr{h}B_0}v_\parallel\pa_z\mb{U}_{1\perp}\times\mb{e}_z+O(\epsilon^2),
\end{align}
where we used $\pa_t\mb{B}_1=B_0\nabla\times(\mb{U}_1\times\mb{e}_z)=B_0\pa_z\mb{U}_{1\perp}$, which follows from Faraday's law. I.e. $\mb{u}_\mr{g\perp0}=0$. Using this for the magnetization yields
\begin{align}
\mb{M}&=-m_\mr{h}\int_{\hat{\mu}}\hat{\mu}\mb{b}_0\fho B_0\mr{d}v_\parallel-\epsilon m_\mr{h}\int_{\hat{\mu}}\hat{\mu}\mb{b}_1\fho B_0\mr{d}v_\parallel-\epsilon m_\mr{h}\int_{\hat{\mu}}\hat{\mu}\mb{b}_0\fhi B_0\mr{d}v_\parallel\\
&\phantom{=}+\epsilon m_\mr{h}\int_{\hat{\mu}}v_\parallel\mb{u}_\mr{g\perp 1}\fho\mr{d}v_\parallel+O(\epsilon^2)\\
&=-\frac{p_\perp}{B_0}\mb{e}_z-\epsilon\frac{p_\perp}{B_0^2}\mb{B}_1-\epsilon m_\mr{h}\int_{\hat{\mu}}\hat{\mu}\fhi B_0\mr{d}v_\parallel\mb{e}_z+\epsilon\frac{m_\mr{h}\tilde{p}_\parallel}{q_\mr{h}B_0^3}\nabla\times\mb{B}_1+\epsilon\frac{m_\mr{h}n_\mr{h0}v_0}{B_0}\mb{U}_{1\perp}\\
&\phantom{=}-\epsilon\frac{m_\mr{h}p_\parallel}{q_\mr{h}B_0^2}\pa_z\mb{U}_{1\perp}\times\mb{e}_z+O(\epsilon^2),
\end{align}
which, after taking the curl, amounts to
\begin{align}
\nabla\times\mb{M}_0&=0\\
\nabla\times\mb{M}_1&=-\frac{p_\perp}{B_0^2}\nabla\times\mb{B}_1+\frac{m_\mr{h}\tilde{p}_\parallel}{q_\mr{h}B_0^3}\nabla\times(\nabla\times\mb{B}_1)+\frac{m_\mr{h}n_\mr{h0}v_0}{B_0}\nabla\times\mb{U}_{1\perp}-\frac{m_\mr{h}p_\parallel}{q_\mr{h}B_0^2}\nabla\times\left(\pa_z\mb{U}_{1\perp}\times\mb{e}_z\right).
\end{align}
\subsubsection{Linearized drift-kinetic current and forces}
Performing the same steps as before, we get for the first term in the definition of the DK current involving the guiding-center velocity
\begin{align}
q_\mr{h}\int_{\hat{\mu}}\mb{u}_\mr{g}\fh\Bpa\mr{d}v_\parallel&=q_\mr{h}\int_{\hat{\mu}}\mb{u}_\mr{g0}\fho B_0\mr{d}v_\parallel+\epsilon q_\mr{h}\int_{\hat{\mu}}\mb{u}_\mr{g0}\fhi B_0\mr{d}v_\parallel+\epsilon q_\mr{h}\int_{\hat{\mu}}\mb{u}_\mr{g1}\fho B_0\mr{d}v_\parallel+O(\epsilon^2)\\
&=q_\mr{h}n_\mr{h0}v_0\mb{e}_z+\epsilon q_\mr{h}\mb{e}_z\int_{\hat{\mu}}v_\parallel\fhi B_0\mr{d}v_\parallel+\epsilon\frac{q_\mr{h}n_\mr{h0}v_0}{B_0}\mb{B}_1+\epsilon\frac{p_\parallel}{B_0^2}\nabla\times\mb{B}_1\\
&+\epsilon q_\mr{h}n_\mr{h0}\mb{U}_{1\perp}-\epsilon\frac{m_\mr{h}n_\mr{h0}v_0}{B_0}\pa_z\mb{U}_{1\perp}\times\mb{e}_z+O(\epsilon^2).
\end{align}
Hence we get for the current-coupling term in the momentum balance equation
\begin{align}
&-\frac{1}{\rho_0}(\mb{j}_\mr{h0}^\mr{D}\times\mb{B}_1+\mb{j}_\mr{h1}^\mr{D}\times\mb{B}_0)=\frac{p_\parallel-p_\perp}{\rho_0B_0}\mb{e}_z\times(\nabla\times\mb{B}_1)+\frac{q_\mr{h}n_\mr{h0}B_0}{\rho_0}\mb{e}_z\times\mb{U}_{1\perp}-\frac{m_\mr{h}n_\mr{h0}v_0}{\rho_0}\pa_z\mb{U}_\perp\\
&-\frac{m_\mr{h}\tilde{p}_\parallel}{q_\mr{h}B_0^2\rho_0}\mb{e}_z\times\Delta\mb{B}_1-\frac{m_\mr{h}n_\mr{h0}v_0}{\rho_0}\pa_z\mb{U}_\perp-\frac{m_\mr{h}p_\parallel}{q_\mr{h}B_0\rho_0}\mb{e}_z\times\pa_z^2\mb{U}_{1\perp}\\
\Leftrightarrow\quad&-\frac{1}{\rho_0}(\mb{j}_\mr{h0}^\mr{D}\times\mb{B}_1+\mb{j}_\mr{h1}^\mr{D}\times\mb{B}_0)=-\frac{\nu_\mr{h}v_0^2A_\mr{h}}{B_0}\pa_z\mb{B}_1+\nu_\mr{h}\Omega_\mr{ci}Z_\mr{h}\mb{e}_z\times\mb{U}_{1\perp}-\frac{m_\mr{h}\tilde{p}_\parallel}{q_\mr{h}B_0^2\rho_0}\mb{e}_z\times\Delta\mb{B}_1\\
&-\frac{\nu_\mr{h}A_\mr{h}^2}{2\Omega_\mr{ci}Z_\mr{h}}(v_\mr{th}^2+2v_0^2)\mb{e}_z\times\pa_z^2\mb{U}_{1\perp}-2\nu_\mr{h}v_0A_\mr{h}\pa_z\mb{U}_\perp,
\end{align}
where we introduced the ratio between the number densities of the hot and bulk species $\nu_\mr{h}=n_\mr{h0}/n_0$, the ion cyclotron frequency $\Omega_\mr{ci}=eB_0/m_\mr{i}$ and the hot particle's thermal velocity $v_\mr{th}^2=2T_\parallel/m_\mr{h}$. Moreover, we assumed hot ions with mass $m_\mr{h}=A_\mr{h}m_\mr{i}$ and charge $q_\mr{h}=Z_\mr{h}e$ ($A_\mr{h},Z_\mr{h}\in\mathbb{N}$) and a pure hydrogen background plasma, i.e. $\rho_0=m_\mr{i}n_0$.
\subsubsection{Dispersion relation}
We solve the obtained linearized set of equation with a plane wave ansatz ($\sim\exp(ikz-i\omega t)$) for all perturbed quantities. This has the consequence that we can make the substitutions $\nabla\rightarrow ik\mb{e}_z$ and $\pa_t\rightarrow i\omega$. From the mass continuity equation, induction equation and the energy equation we can then deduce the following relations for the Fourier coefficients:
\begin{align}
&\hat{\rho}=\frac{k\rho_0}{\omega}\hat{U}_\parallel,\quad\hat{\mb{B}}=-\frac{kB_0}{\omega}\hat{\mb{U}}_\perp,\quad\hat{p}=\frac{\gamma p_0k}{\omega}\hat{U}_\parallel.
\end{align}
The momentumm balance equation then amounts to
\begin{align}
&-i\omega\hat{\mb{U}}+i\frac{v_\mr{A}^2k^2}{\omega}\hat{\mb{U}}_\perp+i\frac{c_\mr{S}^2k^2}{\omega}\mb{e}_z\hat{U}_\parallel=i\frac{\nu_\mr{h}v_0^2A_\mr{h}k^2}{\omega}\hat{\mb{U}}_\perp-\frac{\nu_\mr{h}A_\mr{h}^2k^3}{2\omega\Omega_\mr{ci}Z_\mr{h}}(3v_0v_\mr{th}^2+2v_0^3)\mb{e}_z\times\hat{\mb{U}}_\perp\\
&+\frac{\nu_\mr{h}A_\mr{h}^2k^2}{2\Omega_\mr{ci}Z_\mr{h}}(v_\mr{th}^2+2v_0^2)\mb{e}_z\times\hat{\mb{U}}_\perp-i2\nu_\mr{h}v_0A_\mr{h}k\hat{\mb{U}}_\perp.
\end{align}
We observe that the sound waves with $\omega=\pm c_\mr{S}k$ are unaffacted by the presence of energetic particles. However, for perpendicular components we obtain the linearized equation of motion
\small
\begin{align}
\left\lbrace(\omega^2-v_\mr{A}^2k^2+\nu_\mr{h}v_0^2A_\mr{h}k^2-2\omega\nu_\mr{h}v_0A_\mr{h}k)\mathbb{I}_2+i\left[\frac{\omega\nu_\mr{h}A_\mr{h}^2k^2}{2\Omega_\mr{ci}Z_\mr{h}}(v_\mr{th}^2+2v_0^2)-\frac{\nu_\mr{h}A_\mr{h}^2k^3}{2\Omega_\mr{ci}Z_\mr{h}}(3v_0v_\mr{th}^2+2v_0^3)\right]\mathbb{J}_2\right\rbrace\hat{\mb{U}}_\perp=0,
\end{align} 
\normalsize
where $\mathbb{J}_2$ is the canonical symplectic form such that $\mb{e}_z\times\hat{\mb{U}}_\perp=-\mathbb{J}_2\hat{\mb{U}}_\perp$. Finally, the dispersion relation can be obtained by requiring the determinant to vanish:
\begin{align}
(\omega^2-v_\mr{A}^2k^2+\nu_\mr{h}v_0^2A_\mr{h}k^2-2\omega\nu_\mr{h}v_0A_\mr{h}k)^2-\left[\frac{\omega\nu_\mr{h}A_\mr{h}^2k^2}{2\Omega_\mr{ci}Z_\mr{h}}(v_\mr{th}^2+2v_0^2)-\frac{\nu_\mr{h}A_\mr{h}^2k^3}{2\Omega_\mr{ci}Z_\mr{h}}(3v_0v_\mr{th}^2+2v_0^3)\right]^2=0.
\end{align}
We consider the two roots
\small
\begin{align}
D_\mr{R/L}(k,\omega)=\omega^2-v_\mr{A}^2k^2+\nu_\mr{h}v_0^2A_\mr{h}k^2-2\omega\nu_\mr{h}v_0A_\mr{h}k-p_\mr{R/L}\left[\frac{\omega\nu_\mr{h}A_\mr{h}^2k^2}{2\Omega_\mr{ci}Z_\mr{h}}(v_\mr{th}^2+2v_0^2)-\frac{\nu_\mr{h}A_\mr{h}^2k^3}{2\Omega_\mr{ci}Z_\mr{h}}(3v_0v_\mr{th}^2+2v_0^3)\right]=0,
\end{align}
\normalsize
where $p_\mr{R/L}=\pm1$ means right- and left-handed polarization, respectively. We can see that we recover shear Alfv\'{e}n waves in the absence of the energetic component ($\nu_\mr{h}\rightarrow 0$). Solving for the frequency yields the four solutions
\small
\begin{align}
&\omega_{\mr{R/L},\pm}(k)=\nu_\mr{h}v_0A_\mr{h}k+p_\mr{R/L}\frac{\nu_\mr{h}A_\mr{h}^2k^2}{4\Omega_\mr{ci}Z_\mr{h}}(v_\mr{th}^2+2v_0^2)\\
&\pm k\sqrt{v_\mr{A}^2-v_0^2\nu_\mr{h}A_\mr{h}(1-\nu_\mr{h}A_\mr{h})-p_\mr{R/L}\frac{\nu_\mr{h}A_\mr{h}^2k}{2\Omega_\mr{ci}Z_\mr{h}}(3v_0v_\mr{th}^2+2v_0^3-\nu_\mr{h}v_0v_\mr{th}^2A_\mr{h}-2\nu_\mr{h}v_0^3A_\mr{h})+\frac{\nu_\mr{h}^2A_\mr{h}^4k^2}{16\Omega_\mr{ci}^2Z_\mr{h}^2}(v_\mr{th}^2+2v_0^2)^2}.
\end{align}
\normalsize
In order to have wave growth or damping we need the argument of the square root to be negative. We immediately see that this is not possible in the limit $v_0\rightarrow0$. Let us use the normalizations $\omega^\prime=\omega/\Omega_\mr{ci}$, $v_\mr{th}^\prime=v_\mr{th}/v_\mr{A}$, $v_0^\prime=v_0/v_\mr{A}$ and $k^\prime=kv_\mr{A}/ \Omega_\mr{ci}$ to get the scaled dispersion relation
\small
\begin{align}
&\omega^\prime(k^\prime)=\nu_\mr{h}v_0^\prime A_\mr{h}k^\prime+p_\mr{R/L}\frac{\nu_\mr{h}A_\mr{h}^2k^{\prime 2}}{4Z_\mr{h}}(v_\mr{th}^{\prime 2}+2v_0^{\prime 2})\\
&\pm k^\prime\sqrt{1-v_0^{\prime 2}\nu_\mr{h}A_\mr{h}(1-\nu_\mr{h}A_\mr{h})-p_\mr{R/L}\frac{\nu_\mr{h}A_\mr{h}^2k^\prime}{2Z_\mr{h}}(3v_0^\prime v_\mr{th}^{\prime 2}+2v_0^{\prime 3}-\nu_\mr{h}v_0^\prime v_\mr{th}^{\prime 2}A_\mr{h}-2\nu_\mr{h}v_0^{\prime 3}A_\mr{h})+\frac{\nu_\mr{h}^2A_\mr{h}^4k^{\prime 2}}{16Z_\mr{h}^2}(v_\mr{th}^{\prime 2}+2v_0^{\prime 2})^2}.
\end{align}
\normalsize
\subsubsection{Discretization of the coupling terms}
Let us for the moment assume that we have the case for which we derived the dispersion relation, i.e. a homogeneous equilibrium with the magnetic field pointing in $z$-direction. We saw that the perturbed distribution function did not result in an additional force term in the momentum balance equation. However, from the equilibrium distribution function we got the additional terms
\begin{align}
q_\mr{h}n_\mr{h0}^\mr{D}\mb{U}_1\times\mb{B}_0-\mb{j}_{h0}^\mr{D}\times\mb{B}_1-\mb{j}_\mr{h1}^\mr{D}\times\mb{B}_0&=-\frac{\rho_0\nu_\mr{h}v_0^2A_\mr{h}}{B_0}\nabla\times\mb{B}_1+\frac{m_\mr{h}\tilde{p}_\parallel}{q_\mr{h}B_0^3}\left[\nabla\times(\nabla\times\mb{B}_1)\right]\times\mb{B}_0\\
&+\frac{\nu_hA_\mr{h}^2\rho_0}{2\Omega_\mr{ci}Z_\mr{h}B_0}\left[\nabla\times(\nabla\times\mb{U}_1)\right]\times\mb{B}_0-2\nu_\mr{h}v_0\rho_0A_\mr{h}\nabla\times\mb{U}_1.
\end{align}
First, we shall put these terms in our geometric framework and we assume a mapping that just elongates the $z$-direction. This has the consequence that the components of the 2-form magnetic  field are equal to the components of the corresponding vector in physical space. However, this is not true for the background density. Keeping in mind that we have to end up with a 1-form, we hence relate
\begin{align}
&\nabla\times\mb{B}_1\leftrightarrow \ast\mr{d}\ast B^2, \\
&\left[\nabla\times(\nabla\times\mb{B}_1)\right]\times\mb{B}_0\leftrightarrow i_{\#\ast B_0}\mr{d}\ast\mr{d}\ast B^2,\\
&\left[\nabla\times(\nabla\times\mb{U}_1)\right]\times\mb{B}_0\leftrightarrow i_{\#\ast B_0}\mr{d}\ast\mr{d}U^1,\\
&\nabla\times\mb{U}_1\leftrightarrow \ast\mr{d}U^1.
\end{align}
In the weak formulation this results in the additional terms
\begin{align}
&\left(\ast\mr{d}\ast B^2,V^1\right)=\left(B^2,\mr{d}V^1\right)\approx\mb{b}^\top\mathbb{M}^2\mathbb{C}\mb{v},\\
&\left(i_{\#\ast B_0}\mr{d}\ast\mr{d}\ast B^2,V^1\right)=\left(\mr{d}^\ast\mr{d}\ast B^2,i_{\#V^1}B_0\right)=\left(\mr{d}\ast B^2,\mr{d}i_{\#V^1}B_0\right)=\mb{b}^\top\mathcal{L}^\top\mathbb{C}^\top\mathbb{M}^2\mathbb{C}\mathcal{T}\mb{v},\\
&\left(i_{\#\ast B_0}\mr{d}\ast\mr{d}U^1,V^1\right)=\left(\mr{d}^\ast\mr{d}U^1,i_{\#V^1}B_0\right)=\left(\mr{d}U^1,\mr{d}i_{\#V^1}B_0\right)=\mb{u}^\top\mathbb{C}^\top\mathbb{M}^2\mathbb{C}\mathcal{T}\mb{v},\\
&\left(\ast\mr{d}U^1,V^1\right)=(\mr{d}U^1,\ast V^1)=\mb{u}^\top\mathbb{C}^\top\mathbb{M}^2\mathcal{Y}\mb{v}, 
\end{align}
where
\begin{align}
&\mathcal{L}_{ij}=\Pi_{1,\mu}^{i_\mu}\left[\frac{1}{\sqrt{g}}G_{\mu k}\Lambda^2_{jk}\right],\\
&\mathcal{Y}_{ij}=\Pi_{1,\mu}^{i_\mu}\left[\sqrt{g}G^{\mu k}\Lambda^1_{jk}\right].
\end{align}
In total we obtain the modified semi-discrete momentum balance equation
\begin{align}
\mathbb{M}^1\mathcal{W}\dot{\mb{u}}&=\frac{1}{\mu_0}\mathcal{T}^\top\mathbb{C}^\top\mathbb{M}^2\mb{b}-\frac{\rho_{0,123}\nu_\mr{h}v_0^2A_\mr{h}}{B_0\sqrt{g}}\mathbb{C}^\top\mathbb{M}^2\mb{b}+\frac{m_\mr{i}A_\mr{h}^2n_\mr{h0}}{2Z_\mr{h}\Omega_\mr{ci}B_0^2}(3v_0v_\mr{th}^2+2v_0^3)\mathcal{T}^\top\mathbb{C}^\top\mathbb{M}^2\mathbb{C}\mathcal{L}\mb{b}\\
&+\frac{\nu_hA_\mr{h}^2\rho_{0,123}}{2\Omega_\mr{ci}Z_\mr{h}B_0\sqrt{g}}\mathcal{T}^\top\mathbb{C}^\top\mathbb{M}^2\mathbb{C}\mb{u}-\frac{2\nu_\mr{h}v_0\rho_{0,123}A_\mr{h}}{\sqrt{g}}\mathcal{Y}^\top\mathbb{M}^2\mathbb{C}\mb{u},
\end{align}
which in the case of our affine mapping ($\mathcal{W}=\rho_{0,123}/\sqrt{g}\mathbb{I}$) is equivalent to 
\begin{align}
\mathbb{M}^1\dot{\mb{u}}&=\frac{\sqrt{g}}{\mu_0\rho_{0,123}}\mathcal{T}^\top\mathbb{C}^\top\mathbb{M}^2\mb{b}-\frac{\nu_\mr{h}v_0^2A_\mr{h}}{B_0}\mathbb{C}^\top\mathbb{M}^2\mb{b}+\frac{A_\mr{h}^2\nu_\mr{h}}{2Z_\mr{h}\Omega_\mr{ci}B_0^2}(3v_0v_\mr{th}^2+2v_0^3)\mathcal{T}^\top\mathbb{C}^\top\mathbb{M}^2\mathbb{C}\mathcal{L}\mb{b}\\
&+\frac{\nu_hA_\mr{h}^2}{2\Omega_\mr{ci}Z_\mr{h}B_0}\mathcal{T}^\top\mathbb{C}^\top\mathbb{M}^2\mathbb{C}\mb{u}-2\nu_\mr{h}v_0A_\mr{h}\mathcal{Y}^\top\mathbb{M}^2\mathbb{C}\mb{u}.
\end{align}
\subsection{Pressure coupling scheme}
\end{document}