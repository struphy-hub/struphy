\documentclass[11pt,oneside,a4paper,fleqn]{article}


% Usepackages
\usepackage[left=1.5cm,right=1.5cm,top=2cm,bottom=2cm]{geometry}
\usepackage[intoc]{nomencl}
\usepackage{float}
\usepackage{floatflt}
\restylefloat{figure}
\usepackage{color}
\usepackage{siunitx}
\usepackage{hyperref}
\usepackage[utf8]{inputenc}
\sisetup{separate-uncertainty}
\usepackage[T1]{fontenc}
\usepackage[english]{babel}
\usepackage{ae}
\usepackage{chngcntr}
\usepackage[round]{natbib}
\usepackage[overload]{empheq}
\usepackage{amsthm}
\usepackage{amssymb}
\usepackage{latexsym}
\usepackage[title]{appendix}
\usepackage{multicol}
\usepackage{amsmath}
\usepackage{amsfonts}
\usepackage{tabularx}
\usepackage{caption}
\usepackage[multiple]{footmisc}
\usepackage{color}
\definecolor{grau}{rgb}{0.95,0.95,0.95}
\definecolor{dunkelgrau}{rgb}{0.8,0.8,0.8}
\usepackage{colortbl}
\usepackage{authblk}
\usepackage{url}
\usepackage{xcolor}
\usepackage{pgf}
\usepackage{wrapfig}
\usepackage[printwatermark]{xwatermark}
\usepackage{xcolor}
\usepackage{graphicx}
\usepackage{lipsum}
\usepackage{tikz}
\usepackage{abstract}
\numberwithin{equation}{section}







% Commands
\newcommand{\vb}{{\mathbf{v}}}

\newcommand{\xib}{{\boldsymbol{\xi}}}
\newcommand{\Xib}{{\boldsymbol{\Xi}}}

\newcommand{\tLa}{\ensuremath{\tilde{\Lambda}}}
\newcommand{\tLab}{\ensuremath{\tilde{\mathbf{\Lambda}}}}
\newcommand{\Lab}{\ensuremath{\boldsymbol{\Lambda}}}
\newcommand{\LaB}{\ensuremath{\mathbb{\Lambda}}}
\newcommand{\tLaB}{\ensuremath{\tilde{\mathbb{\Lambda}}}}
\newcommand{\curl}{\nabla\times}




% Title
\title{Linear MHD using discrete differential forms}

% Date
\date{}

% Authors and affiliations
\author[1]{Florian Holderied}


\affil[1]{\textit{Max Planck Institute for Plasma Physics, Boltzmannstrasse 2, 85748 Garching, Germany}}





\begin{document} 

\maketitle

\begin{align*}
\tilde{\mathbf{J}}_h(\xib)&=q DF^\top(\xib) \int \tilde f_h(\xib,\vb,t)\vb \ d\vb=q DF^\top(\xib) \sum_{p=1}^{N_p} \omega_p \frac{\delta(\xib-\xib_p)}{|J_F(\xib_p)|}\vb_p
\end{align*}

For Ampere's law, we insert \eqref{ef} and \eqref{bf} into \eqref{weakAmpere} and choose the basis functions $\{\tLab^1_i\}_{i=1..3}$ as testfunctions
\begin{align*}
 &\frac{d}{dt} \int_{\tilde \Omega} \left(N(\xib) \tLab^1(\xib)\right)^\top N(\xib)\tLab^1(\xib)   |J_F(\xib)| d\xib=
 \\&\int_{\tilde \Omega} \left(\frac{DF(\xib)}{J_F(\xib)} \nabla_{\xib} \times \tLab^1(\xib)\right)^\top \frac{DF(\xib)}{J_F(\xib)}\tLab^2(\xib)  |J_F(\xib)|  d\xib -\int_{\tilde \Omega}  \left(N(\xib) \tLab^1(\xib)\right)^\top N(\xib) \tilde{\mathbf{J}}_h(\xib) |J_F(\xib)|  d\xib.
\end{align*}



Next, we use the relation \eqref{diffop} for the $\curl$  and insert the transformed current \eqref{current}
\begin{align*}
\int_{\tilde \Omega} \tLab^1(\xib)^\top N(\xib)^\top N(\xib) \tLab^1(\xib)\dot{} |J_F(\xib)| d\xib  &= \int_{\tilde{\Omega}}  (\tLab^2(\xib)  )^\top DF(\xib)^\top DF(\xib)\tLab^2(\xib) \frac{1}{|J_F(\xib)|} d\xib
\\&-  \int_{\tilde \Omega}
  \tLab^1(\xib)^\top N(\xib)^\top q \sum_{p=1}^{N_p} \omega_p \frac{\delta(\xib-\xib_p)}{|J_F(\xib_p)|}\vb_p |J_F(\xib)| d\xib
 \\
 \Leftrightarrow
\int_{\tilde \Omega} \tLab^1(\xib)^\top N(\xib)^\top N(\xib) \tLab^1(\xib) |J_F(\xib)|  d\xib \ \dot{}
 &=  \int_{\tilde{\Omega}}  \tLab^2(\xib)^\top DF(\xib)^\top  DF(\xib)\tLab^2(\xib) \frac{1}{|J_F(\xib)|}  d\xib \  
\\  &- \sum_{p=1}^{N_p} q\omega_p  \tLab^1(\xib_p)^\top N(\xib_p)^\top  \vb_p.
\end{align*}





\end{document}