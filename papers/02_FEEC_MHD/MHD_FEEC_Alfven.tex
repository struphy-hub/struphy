\documentclass[11pt,oneside,a4paper,fleqn]{article}


% Usepackages
\usepackage[left=2cm,right=2cm,top=2cm,bottom=2cm]{geometry}
\usepackage[intoc]{nomencl}
\usepackage{float}
\usepackage{floatflt}
\restylefloat{figure}
\usepackage{color}
\usepackage{siunitx}
\usepackage{hyperref}
\usepackage[utf8]{inputenc}
\sisetup{separate-uncertainty}
\usepackage[T1]{fontenc}
\usepackage[english]{babel}
\usepackage{ae}
\usepackage{chngcntr}
\usepackage[round]{natbib}
\usepackage[overload]{empheq}
\usepackage{amsthm}
\usepackage{amssymb}
\usepackage{latexsym}
\usepackage[title]{appendix}
\usepackage{multicol}
\usepackage{amsmath}
\usepackage{amsfonts}
\usepackage{tabularx}
\usepackage{caption}
\usepackage[multiple]{footmisc}
\usepackage{color}
\definecolor{grau}{rgb}{0.95,0.95,0.95}
\definecolor{dunkelgrau}{rgb}{0.8,0.8,0.8}
\usepackage{colortbl}
\usepackage{authblk}
\usepackage{url}
\usepackage{xcolor}
\usepackage{pgf}
\usepackage{wrapfig}
\usepackage[printwatermark]{xwatermark}
\usepackage{xcolor}
\usepackage{graphicx}
\usepackage{lipsum}
\usepackage{tikz}
\usepackage{abstract}
\numberwithin{equation}{section}
\usepackage{mathbbol}
\usepackage{amssymb}             % AMS Math

\DeclareSymbolFontAlphabet{\amsmathbb}{AMSb}%






% Commands
\newcommand{\pa}{\partial}
\newcommand{\mr}[1]{\mathrm{#1}}
\newcommand{\mb}[1]{\mathbf{#1}}
\newcommand{\bo}[1]{\boldsymbol{#1}}
\newcommand{\fh}{f_\mr{h}^\mr{D}}
\newcommand{\fho}{f_\mr{h0}^\mr{D}}
\newcommand{\fhi}{f_\mr{h1}^\mr{D}}
\newcommand{\Bpa}{B_\parallel^\ast}


\renewcommand\Affilfont{\fontsize{10}{20}\itshape}

% Title
\title{Hybrid kinetic-shear Alfv\'{e}n physics using discrete differential forms}

% Date
\date{}

% Authors and affiliations
\author[1]{Florian Holderied}


\affil[1]{\textit{Max Planck Institute for Plasma Physics, Boltzmannstrasse 2, 85748 Garching, Germany}}
\affil[2]{\textit{Technical University of Munich, Department of Physics, Boltzmannstrasse 2, 85748 Garching, Germany}}





\begin{document} 

\maketitle


\section{Full model and model reduction}
We consider a hybrid kinetic-MHD model where the coupling of the fluid and kinetic species is done via a current coupling scheme. In classical vector calculus notation the model reads
\begin{align}
&\frac{\pa\rho}{\pa t}+\nabla\cdot(\rho\mb{U})=0,\label{eq_continuity}\\
&\frac{\pa\mb{U}}{\pa t}+(\mb{U}\cdot\nabla)\mb{U}=\frac{1}{\rho}(\nabla\times\mb{B}+\rho_\text{h}\mb{U}-\mb{j}_\text{h})\times\mb{B}-\frac{\nabla p}{\rho},\\
&\frac{\pa\mb{B}}{\pa t}=\nabla\times(\mb{U}\times\mb{B}),\\
&\frac{\pa p}{\pa t}+\nabla\cdot(p\mb{U})+(\gamma-1)p\nabla\cdot\mb{U}=0,
\label{eq_pressure}\\
&\frac{\pa f_\text{h}}{\pa t}+\mb{v}\cdot\nabla f_\text{h}+(\mb{B}\times\mb{U}+\mb{v}\times\mb{B})\cdot\nabla_\mb{v}f_\text{h}=0,\\
&\rho_\text{h}=\int f_\text{h}\text{d}^3v,\qquad\mb{j}_\text{h}=\int\mb{v}f_\text{h}\text{d}^3v.\label{eq_moments}
\end{align}
where we set all physical constants equal to one\footnote{We assume hot ions with a positive charge. Therefore, one does not have to be careful with the signs in (\ref{eq_moments})}. This set of equations forms a closed system of nonlinear partial differential equations for the the bulk mass density $\rho$, the bulk velocity $\mb{U}$, the magnetic induction $\mb{B}$ (which we will simply refer to as magnetic field), the bulk pressure $p$ and the distribution function of the hot ions $f_\text{h}$. Furthermore, $\gamma=5/3$ is the adiabatic exponent. This system possesses a Hamiltonian structure with the following conserved energy:
\begin{align}
\mathcal{H}_0(t)=\frac{1}{2}\int\rho\mb{U}^2\text{d}^3x+\frac{1}{\gamma-1}\int p\text{d}^3x+\frac{1}{2}\int\mb{B}^2\text{d}^3x+\frac{1}{2}\int\int\mb{v}^2f_\text{h}\mr{d}^3v\mr{d}^3x.\label{eq_H0}
\end{align}
In order to start from a simpler model we shall restrict ourselves for the moment on waves which are non-perturbative in density and pressure. Hence we drop equations (\ref{eq_continuity}) and (\ref{eq_pressure}). Moreover, we linearize all terms related to the MHD part by assuming that MHD waves are small perturbations about an equilibrium state. Making the Ansatzes $\mb{B}=\mb{B}_\text{eq}+\tilde{\mb{B}}$, $\mb{U}=\tilde{\mb{U}}$ and $\rho=\rho_\text{eq}$ and neglecting nonlinear terms in the MHD part yields
\begin{align}
\rho_\text{eq}&\frac{\pa\tilde{\mb{U}}}{\pa t}=(\nabla\times\tilde{\mb{B}})\times\mb{B}_\text{eq}+(\nabla\times\mb{B}_\text{eq})\times\tilde{\mb{B}}+(\rho_\text{h}\tilde{\mb{U}}-\mb{j}_\text{h})\times\mb{B},\label{eq_momentum}\\
&\frac{\pa\tilde{\mb{B}}}{\pa t}=\nabla\times(\tilde{\mb{U}}\times\mb{B}_\text{eq}),\label{eq_induction}\\
&\frac{\pa f_\text{h}}{\pa t}+\mb{v}\cdot\nabla f_\text{h}+(\mb{B}\times\mb{U}+\mb{v}\times\mb{B})\cdot\nabla_\mb{v}f_\text{h}=0.\label{eq_Vlasov}
\end{align}
The linearization of the MHD part breaks energy conservation, i.e. the Hamiltonian (\ref{eq_H0}) is no longer conserved. However, we can write down the new Hamiltonian
\begin{align}
\mathcal{H}_1(t)=\frac{1}{2}\int\rho_\text{eq}\tilde{\mb{U}}^2\text{d}^3x+\frac{1}{2}\int\tilde{\mb{B}}^2\text{d}^3x+\frac{1}{2}\int\int\mb{v}^2f_\text{h}\mr{d}^3v\mr{d}^3x,\label{eq_H1}
\end{align} 
which is conserved if $\nabla\times\mb{B}_\text{eq}=0$. 

We shall use classical particle-in-cell techniques for solving the kinetic equation (\ref{eq_Vlasov}) and the framework of \textit{finite element exterior calculus} (FEEC) for solving field equations. In the latter one works with differential forms instead of vector and scalar field. This allows us to treat arbitrary geometries in a natural fashion. For physical reasons we assume the bulk mass density and the hot charge density to be 3-forms ($\rho_\text{eq},\rho_\text{h}\rightarrow\rho_\text{eq}^3,\rho_\text{h}^3$), the magnetic field to be a 2-form ($\mb{B}_\text{eq},\tilde{\mb{B}}\rightarrow B_\text{eq}^2,\tilde{B}^2$) and the bulk velocity and the hot current density to be 1-forms ($\tilde{\mb{U}},\mb{j}_\text{h}\rightarrow \tilde{U}^1,j^1_\text{h}$). Eq. (\ref{eq_momentum}) and (\ref{eq_induction}) can then be written as
\begin{align}
&(\ast\rho_\text{eq}^3)\wedge\frac{\pa\tilde{U}^1}{\pa t}=i_{\#\ast B^2_\text{eq}}\mr{d}\ast \tilde{B}^2+i_{\#\ast\tilde{B}^2}\mr{d}\ast B^2_\text{eq}-(\ast\rho^3_\text{h})\wedge i_{\#\tilde{U}^1}B^2+i_{\#j_\text{h}^1}B^2,\\
&\frac{\pa \tilde{B}^2}{\pa t}+\mr{d}(i_{\#U^1}B^2_\text{eq})=0,
\end{align} 
where $\ast$ is the Hodge-star operator, $\wedge$ the wedge product, $i$ the interior product and $\#$ the sharp operator which transforms a 1-form to a vector field.



\section{Semi-discretization in space}
As a next step, we introduce finite element basis functions which satisfy a discrete deRham sequence and which form a commuting diagram with the continuous functions via the interpolation-histopolation projectors $\Pi_0$, $\Pi_1$, $\Pi_2$ and $\Pi_3$. This is depicted in Fig. \ref{fig_diagram}. 
\begin{figure}
\centering
\includegraphics[scale=0.55]{01_Figures/deRham3d_forms.pdf}
\caption{Commuting diagram for function spaces in 3d. The upper line represents the de Rham sequence for the continuous spaces, while the lower line represents the discrete counterpart.\label{fig_diagram}}
\end{figure}
Assuming that we know the basis functions in each space (how this can be done with e.g. tensor-product B-splines, see Sec. ...), we express the forms in their resepective bases as
\begin{alignat}{2}
&\tilde{U}^1(\mb{q})\approx \tilde{U}_h^1(\mb{q})=\sum_{\mathbf{i}}\sum_{\mu = 1}^3u_{\mu,\mathbf{i}}\Lambda^1_{\mu,\mathbf{i}}(\mb{q})\mathrm{d}q^\mu, \qquad &&\mb{u}^\top:=(\mb{u}_1^\top,\mb{u}_2^\top,\mb{u}_3^\top)\in\mathbb{R}^{3N},\\
&\tilde{B}^2(\mb{q})\approx \tilde{B}_h^2(\mb{q})=\sum_{\mathbf{i}}\sum_{\mu = 1}^3b_{\mu,\mathbf{i}}\Lambda^2_{\mu,\mathbf{i}}(\mb{q})(\mathrm{d}q^\alpha\wedge\mr{d}q^\beta)_\mu, \qquad &&\mb{b}^\top:=(\mb{b}_1^\top,\mb{b}_2^\top,\mb{b}_3^\top)\in\mathbb{R}^{3N},
\end{alignat}
where $\mb{i}=(i_1,i_2,i_3)$ is a multi-index and $N$ the total number of basis functions. To simplify the notation, we write for the components of the differential forms
\begin{alignat}{2}
&\tilde{U}^1_h\leftrightarrow \tilde{\mb{U}}_h^\top=(\mb{u}_1^\top,\mb{u}_2^\top,\mb{u}_3^\top)\begin{pmatrix}
\bo{\Lambda}^1_1 &0 &0 \\ 0 &\bo{\Lambda}^1_2 &0 \\ 0 &0 &\bo{\Lambda}^1_3
\end{pmatrix}=\mb{u}^\top\mathbb{\Lambda}^1,\qquad&&\mathbb{\Lambda}^1\in\mathbb{R}^{3N\times 3},\\
&\tilde{B}^2_h\leftrightarrow \tilde{\mb{B}}_h^\top=(\mb{b}_1^\top,\mb{b}_2^\top,\mb{b}_3^\top)\begin{pmatrix}
\bo{\Lambda}^2_1 &0 &0 \\ 0 &\bo{\Lambda}^2_2 &0 \\ 0 &0 &\bo{\Lambda}^2_3
\end{pmatrix}=\mb{b}^\top\mathbb{\Lambda}^2,\qquad&&\mathbb{\Lambda}^2\in\mathbb{R}^{3N\times 3}.
\end{alignat}
As already stated, solve the kinetic equation with particle-in-cell techniques. Hence we assume a particle-like distribution function which in physical space takes the form
\begin{align}
f_\mr{h}=f_\mr{h}(t,\mb{x},\mb{v})\approx \sum_k w_k\delta(\mb{x}-\mb{x}_k(t))\delta(\mb{v}-\mb{v}_k(t)).
\end{align}
From this, the hot ion charge density, current density and energy density can easily be obtained by taking the first three moments in velocity space:
\begin{align}
\mathring{\rho}_\mr{h}(t,\mb{x})=&\sum_k w_k\delta(\mb{x}-\mb{x}_k(t)),\\
\mathring{\mb{j}}_\mr{h}(t,\mb{x})=&\sum_k w_k\delta(\mb{x}-\mb{x}_k(t))\mb{v}_k(t),\\
\mathring{\epsilon}_\mr{h}(t,\mb{x})=&\sum_{k}w_k\delta(\mb{x}-\mb{x}_k(t))\mb{v}_k^2(t).
\end{align}
To avoid confusions, we use the notation $\mathring{(\cdot)}$ where necessary for quantities which are defined on physical space. Since there is no difference between vectors/scalars and forms in physical space, these expression are are as well the components of the 3-form number density, the 1-form current density and the 3-form energy density. To get the components on the logical domain we apply the transformation formulas for 3-forms, respectively 1-forms to obtain
\begin{align}
&\rho_\mr{h,123}(t,\mb{q})=\sqrt{g}\mathring{\rho}_\mr{h}(t,F(\mb{q}))=\sum_k w_k\delta(\mb{q}-\mb{q}_k(t)),\\
&\mb{j}_\mr{h}(t,\mb{q})=DF^\top\mathring{\mb{j}}_\mr{h}(t, F(\mb{q}))=\frac{1}{\sqrt{g}}DF^\top\sum_k w_k\delta(\mb{q}-\mb{q}_k(t))\mb{v}_k(t),\\
&\epsilon_\mr{h,123}(t,\mb{q})=\sqrt{g}\mathring{\epsilon}_\mr{h}(t,F(\mb{q}))=\sum_k w_k\delta(\mb{q}-\mb{q}_k(t))\mb{v}_k^2(t),
\end{align}
where we made use of the transformation formula
\begin{align}
\delta(\mb{x}-\mb{x}_k(t))=\frac{1}{\sqrt{g}}\delta(\mb{q}-\mb{q}_k(t)).
\end{align}
Let us use these results to derive an energy conserving semi-discrete system for the finite element coefficients of  $\tilde{U}_h^1$ and $\tilde{B}_h^2$ and the particle's positions and velocities $(\mb{q}_k,\mb{v}_k)_{k=1,\ldots,N_\text{p}}$.

\subsection{Momentum equation}
We choose a weak formulation for the momentum equation and consequently take the inner product with a test function $V^1\in H\Lambda^1(\Omega)$ to obtain the variational formulation: Find $\tilde{U}^1\in H\Lambda^1(\Omega)$ such that
\begin{align}
\left((\ast\rho_\text{eq}^3)\wedge\frac{\pa\tilde{U}^1}{\pa t},V^1\right)=&\left(i_{\#\ast B^2_\text{eq}}\mr{d}\ast \tilde{B}^2,V^1\right)+\left(i_{\#\ast\tilde{B}^2}\mr{d}\ast B^2_\text{eq},V^1\right)\\-&\left((\ast\rho^3_\text{h})\wedge i_{\#\tilde{U}^1}B^2,\tilde{V}^1\right)+\left(i_{\#j_\text{h}^1}B^2,V^1\right)\qquad\forall\;V^1\in H\Lambda^1(\Omega).
\end{align}
We apply the Galerkin approximation to each term and project back into the right spaces where necessary. Let us start with the first term on the left-hand side involving the equilibrium bulk density. To achieve conservation of energy at the discrete level we make use of the fact that the wedge with a 0-form is just a multiplication with a scalar. Hence the wedge product can as well be applied to the test function and the following equality holds:
\begin{align}
\left((\ast\rho_\text{eq}^3)\wedge\frac{\pa\tilde{U}^1}{\pa t},V^1\right)=\frac{1}{2}\left((\ast\rho_\text{eq}^3)\wedge\frac{\pa\tilde{U}^1}{\pa t},V^1\right)+\frac{1}{2}\left(\frac{\pa\tilde{U}^1}{\pa t},(\ast\rho_\text{eq}^3)\wedge V^1\right).
\end{align}
Using the definition of the inner product of 1-forms yields for the first term
\begin{align}
\left((\ast\rho_\text{eq}^3)\wedge\frac{\pa\tilde{U}^1}{\pa t},V^1\right)&=\int_{\hat{\Omega}}\frac{1}{\sqrt{g}}\rho_\text{eq,123}\dot{\tilde{\mb{U}}}^\top G^{-1}\mb{V}\sqrt{g}\,\mr{d}^3q\approx\int_{\hat{\Omega}}\Pi_1\left(\frac{1}{\sqrt{g}}\rho_{\text{eq},123}\dot{\tilde{\mb{U}}}_h^\top\right)G^{-1}\mb{V}_h\sqrt{g}\,\mr{d}^3q\\&=\dot{\mb{u}}^\top\mathcal{W}^\top\underbrace{\int_{\hat{\Omega}}\mathbb{\Lambda}^1G^{-1}(\mathbb{\Lambda}^1)^\top\sqrt{g}\,\mr{d}^3q}_{=:\mathbb{M}^1}\,\mb{v}=\dot{\mb{u}}^\top\mathcal{W}^\top\mathbb{M}^1\mb{v}\quad\forall\;\mb{v}\in\mathbb{R}^{3N},
\end{align}
where $\mathbb{M}^1\in\mathbb{R}^{3N\times 3N}$ is the mass matrix in the space $V_1$. The projection matrix, for which we use calligraphic symbols, is given by
\begin{align}
\mathcal{W}_{ij}=\Pi_{1,\mu}^{i_\mu}\left[\frac{1}{\sqrt{g}}\rho_{\text{eq},123}\mathbb{\Lambda}^1_{j\mu}\right],\qquad i=\begin{cases}i_\mu, \quad &\mu=1\\ N+i_\mu, &\mu=2\\2N+i_\mu, &\mu=3\end{cases}
\end{align}
for $i_\mu=\{1,\ldots,N\}$. Unfortunately, this matrix is dense which is problematic from a memory consumption point of view. Therefore, we just compute the right-hand sides of the projection for each basis function which defines a sparse matrix due to the compact support of B-splines. The final projection we then perform in each time step again. Explicitly, the right-hand side reads
\begin{align}
&\tilde{\mathcal{W}}:=\begin{pmatrix}
\mr{vec}_{1,1}\left[\rho_0/\sqrt{g}(\bo{\Lambda}_1^1)^\top\right] &0 &0 \\
0 &\mr{vec}_{1,2}\left[\rho_0/\sqrt{g}(\bo{\Lambda}_2^1)^\top\right] &0 \\
0 &0 &\mr{vec}_{1,3}\left[\rho_0/\sqrt{g}(\bo{\Lambda}_3^1)^\top\right]
\end{pmatrix}\\
&\bo{\mathcal{I}}_1^{-1}:=\begin{pmatrix}
\mathcal{I}_{1,1}^{-1} &0 &0 \\ 0 &\mathcal{I}_{1,2}^{-1} &0 \\ 0 &0 &\mathcal{I}_{1,3}^{-1} 
\end{pmatrix},\\
\Rightarrow\quad&\mathcal{W}=\bo{\mathcal{I}}_1^{-1}\tilde{\mathcal{W}}.
\end{align}
For some 1-form $f^1=f_1\text{d}q^1+f_2\text{d}q^2+f_3\text{d}q^3$ the projection is a mixed interpolation-histopolation problem defined by
\begin{align}
&(\mr{vec}_{1,1})_i(f^1)=\int_{\xi_{i_1}}^{\xi_{i_1+1}}f_1(q_1,\xi_{i_2},\xi_{i_3})\mr{d}q^1,\qquad  (\mathcal{I}_{1,1})_{ij}=\int_{\xi_{i_1}}^{\xi_{i_1+1}}\bo{\Lambda}^1_{1,j}(q_1,\xi_{i_2},\xi_{i_3})\mr{d}q^1 \\
&(\mr{vec}_{1,2})_i(f^1)=\int_{\xi_{i_2}}^{\xi_{i_2+1}}f_2(\xi_{i_1},q_2,\xi_{i_3})\mr{d}q^2,\qquad  (\mathcal{I}_{1,2})_{ij}=\int_{\xi_{i_2}}^{\xi_{i_2+1}}\bo{\Lambda}^1_{2,j}(\xi_{i_1},q_2,\xi_{i_3})\mr{d}q^2 \\
&(\mr{vec}_{1,3})_i(f^1)=\int_{\xi_{i_3}}^{\xi_{i_3+1}}f_3(\xi_{i_1},\xi_{i_2},q_3)\mr{d}q^3,\qquad (\mathcal{I}_{1,3})_{ij}=\int_{\xi_{i_3}}^{\xi_{i_3+1}}\bo{\Lambda}^1_{3,j}(\xi_{i_1},\xi_{i_2},q_3)\mr{d}q^3,
\end{align}
where the $\xi_\mb{i}$ are some well-chosen interpolation points on the logical domain. Finally, we obtain
\begin{align}
\left((\ast\rho_\text{eq}^3)\wedge\frac{\pa\tilde{U}^1}{\pa t},V^1\right)\approx\frac{1}{2}\mb{v}^\top\left(\mathbb{M}^1\mathcal{W}+\mathcal{W}^\top\mathbb{M}^1\right)\dot{\mb{u}}=:\mb{v}^\top\mathcal{A}\dot{\mb{u}},
\end{align}
with $\mathcal{A}\in\mathbb{R}^{3N\times 3N}$ being symmetric.

Using the identities $\langle i_{\#\gamma^1}\alpha^2,\beta^1\rangle=\langle\alpha^2,\gamma^1\wedge\beta^1\rangle$ and $\ast(\ast \alpha^2\wedge \beta^1)=i_{\#\beta^1}\alpha^2$, the first term on the right-hand side can be written as
\begin{align}
\left(i_{\#\ast B^2_\text{eq}}\mr{d}\ast \tilde{B}^2,V^1\right)=\left(d\ast\tilde{B}^2,\ast B^2_\text{eq}\wedge V^1\right)=\left(\ast\mr{d}\ast\tilde{B}^2,\ast(\ast B_\text{eq}^2\wedge V^1)\right)=\left(d^\ast\tilde{B}^2,i_{\#V^1}B_\text{eq}^2\right),
\end{align}
where we introduced the co-differential operator $\mr{d}^\ast\alpha^p=(-1)^p\ast\mr{d}\ast\alpha^p$. Applying the Green formula for differential forms and assuming that the boundary term vanishes yields
\begin{align}
\left(i_{\#\ast B^2_\text{eq}}\mr{d}\ast \tilde{B}^2,V^1\right)&=\left(\tilde{B}^2,\mr{d}i_{\#V^1}B_\text{eq}^2\right)=\int_{\hat{\Omega}}\frac{1}{g}\hat{\tilde{\mb{B}}}^\top G\left(\nabla\times(\hat{\mb{B}}_\text{eq}\times G^{-1}\mb{V})\right)\sqrt{g}\,\mr{d}^3q\\
&\approx\mb{b}^\top\underbrace{\int_{\hat{\Omega}}\frac{1}{\sqrt{g}}\mathbb{\Lambda}^2G(\mathbb{\Lambda}^2)^\top\mr{d}^3q}_{=:\mathbb{M}^2}\,\mathbb{C}\tilde{\Pi}_1\left(\mathbb{B}_\text{eq}G^{-1}(\mathbb{\Lambda}^1)^\top\right)\mb{v}=\mb{b}^\top\mathbb{M}^2\mathbb{C}\mathcal{T}\mb{v}\quad\forall\mb{v}\in\mathbb{R}^{3N},
\end{align}
where we introduced the discrete curl matrix $\mathbb{C}\in\mathbb{R}^{3N\times 3N}$, the mass matrix $\mathbb{M}^2\in\mathbb{R}^{3N\times 3N}$ in the space $V_2$ and we wrote the vector product of the background magnetic field with the velocity field in terms of a matrix-vector product by using the matrix
\begin{align}
\mathbb{B}_\text{eq}=\begin{pmatrix}
0 &-B_{\text{eq},12} &B_{\text{eq},31} \\ B_{\text{eq},12} &0 &-B_{\text{eq},23} \\ -B_{\text{eq},31} &B_{\text{eq},23} &0
\end{pmatrix}\in\mathbb{R}^{3\times 3}.
\end{align}
The projection matrix $\mathcal{T}$ is given by
\begin{align}
\mathcal{T}_{ij}:=\Pi_{1,\mu}^{i_\mu}\left[(\mathbb{B}_\text{eq})_{\mu k}G^{kl}\Lambda^1_{j,l}\right]=\Pi_{1,\mu}^{i_\mu}\left[\epsilon_{\mu mk}B_{\text{eq},m}G^{kl}\Lambda^1_{j,l}\right],\qquad\mu=\{1,2,3\},
\end{align}
which has the right-hand sides
\small
\begin{align}
&\tilde{\mathcal{T}}=\\
&=\begin{pmatrix}
\mr{vec}_{1,1}\left[(B_{0,31}G^{31}-B_{0,12}G^{21})(\bo{\Lambda}^1_1)^\top, (B_{0,31}G^{32}-B_{0,12}G^{22})(\bo{\Lambda}^1_2)^\top, (B_{0,31}G^{33}-B_{0,12}G^{23})(\bo{\Lambda}^1_3)^\top\right] \\[2mm]
\mr{vec}_{1,2}\left[(B_{0,12}G^{11}-B_{0,23}G^{31})(\bo{\Lambda}^1_1)^\top, (B_{0,12}G^{12}-B_{0,23}G^{32})(\bo{\Lambda}^1_2)^\top, (B_{0,12}G^{13}-B_{0,23}G^{33})(\bo{\Lambda}^1_3)^\top\right] \\[2mm]
\mr{vec}_{1,3}\left[(B_{0,23}G^{21}-B_{0,31}G^{11})(\bo{\Lambda}^1_1)^\top, (B_{0,23}G^{22}-B_{0,31}G^{12})(\bo{\Lambda}^1_2)^\top, (B_{0,23}G^{23}-B_{0,31}G^{13})(\bo{\Lambda}^1_3)^\top\right]
\end{pmatrix}.
\end{align}
\normalsize
Performing the same steps for the second term yields
\begin{align}
\left(i_{\#\ast\tilde{B}^2}\mr{d}\ast B^2_\text{eq},V^1\right)&=\left(B_\text{eq}^2,\mr{d}i_{\#V^1}\tilde{B}^2\right)=\int_{\hat{\Omega}}\frac{1}{g}\hat{\mb{B}}_\text{eq}^\top G\left(\nabla\times(\tilde{\mb{B}}\times G^{-1}\mb{V})\right)\sqrt{g}\,\mr{d}^3q\\
&\approx\tilde{\Pi}_2\left(\hat{\mb{B}}_\text{eq}^\top\right)\int_{\hat{\Omega}}\frac{1}{\sqrt{g}}\mathbb{\Lambda}^2G(\mathbb{\Lambda}^2)^\top\mr{d}^3q\,\mathbb{C}\tilde{\Pi}_1\left[(\mathbb{\Lambda}^2)^\top\mb{b}\times G^{-1}(\mathbb{\Lambda}^1)^\top\mb{v}\right]\\
&=\mb{b}_\text{eq}^\top\mathbb{M}^2\mathbb{C}(\mb{b}^\top\mathcal{P}\mb{v})\quad\forall \mb{v}\in\mathbb{R}^{3N}.
\end{align}
The projection tensor $\mathcal{P}\in\mathbb{R}^{3N\times3N\times3N}$ is given by
\begin{align}
\mathcal{P}_{ijk}=\Pi_1^{j_\mu}\left[\epsilon_{\mu lm}\Lambda^2_{i,l}G^{mn}\Lambda^1_{k,n}\right],
\end{align}
where it is important to note that the entries of $\mb{b}$ contract from left with the index $i$ and the entries of $\mb{v}$ from right with the index $k$. The result is then a vector defined by the index $j$. 

Next, we consider the last term involving the hot ion current density
\begin{align}
\left(i_{\#j_\mr{h}^1}B^2,V^1\right)=&\int_{\hat{\Omega}}(\hat{\mb{B}}\times G^{-1}\mb{j}_\mr{h})^\top G^{-1}\mb{V}\sqrt{g}\,\mr{d}^3q\approx \sum_k w_k\mb{V}_h^\top(\mb{q}_k)G^{-1}(\hat{\mb{B}}_h(\mb{q}_k)\times DF^{-1}\mb{v}_k), \\
=&\mb{v}^\top\mathbb{P}_1\mathbb{W}\tilde{G}^{-1}\hat{\mathbb{B}}\tilde{DF}^{-1}\mb{V},
\end{align}
where we introduced the antisymmetric block matrix 
\small
\begin{align}
&\hat{\mathbb{B}}=\hat{\mathbb{B}}(\mb{b},\mb{Q})=\\=&\begin{pmatrix}
0 &-\text{diag}[\mathbb{P}^{2\top}_3(\mb{Q})\mb{b}_3+B_{\text{eq},12}(\mb{Q})] &\text{diag}[\mathbb{P}^{2\top}_2(\mb{Q})\mb{b}_2+B_{\text{eq},31}(\mb{Q})] \\
\text{diag}[\mathbb{P}^{2\top}_3(\mb{Q})\mb{b}_3+B_{\text{eq},12}(\mb{Q})] &0 &-\text{diag}[\mathbb{P}^{2\top}_1(\mb{Q})\mb{b}_1+B_{\text{eq},23}(\mb{Q})] \\
-\text{diag}[\mathbb{P}^{2\top}_2(\mb{Q})\mb{b}_2+B_{\text{eq},31}(\mb{Q})] &\text{diag}[\mathbb{P}^{2\top}_1(\mb{Q})\mb{b}_1+B_{\text{eq},23}(\mb{Q})] &0
\end{pmatrix}.
\end{align}
\normalsize
where $(\mathbb{P}^2_{1/2/3})_{ik}=\Lambda^2_{1/2/3,i}(\mb{q}_k)\in\mathbb{R}^{N\times N_p}$ represents the evaluation of all basis functions at all particle positions and $\mb{Q}\in\mathbb{R}^{3N_p}$ is the vector holding all particle positions. Moreover,
\begin{align}
&\mb{V}=(v_{1x},v_{2x},\ldots,v_{N_px},v_{1y},v_{2y},\ldots,v_{N_py},v_{1z},v_{2z},\ldots,v_{N_pz})^\top\in\mathbb{R}^{3N_p},\\
&\mathbb{W}=\begin{pmatrix}
\text{diag}(w_1,\ldots,w_{N_p}) & & \\
&\text{diag}(w_1,\ldots,w_{N_p}) &  \\
& &\text{diag}(w_1,\ldots,w_{N_p})
\end{pmatrix}\in\mathbb{R}^{3N_p\times 3N_p},\\
&\mathbb{P}^1=\mathbb{P}^1(\mb{Q})=\begin{pmatrix}
\mathbb{P}^1_1(\mb{Q}) & & \\ &\mathbb{P}^1_2(\mb{Q}) & \\ & &\mathbb{P}^1_3(\mb{Q})
\end{pmatrix}\in\mathbb{R}^{3N\times 3N_p}.\\
&\tilde{G}^{-1}=\tilde{G}^{-1}(\mb{Q})\begin{pmatrix}
\text{diag}(G^{11}(\mb{Q})) &\text{diag}(G^{12}(\mb{Q})) &\text{diag}(G^{13}(\mb{Q})) \\
\text{diag}(G^{21}(\mb{Q})) &\text{diag}(G^{22}(\mb{Q})) &\text{diag}(G^{23}(\mb{Q})) \\
\text{diag}(G^{31}(\mb{Q})) &\text{diag}(G^{32}(\mb{Q})) &\text{diag}(G^{33}(\mb{Q}))
\end{pmatrix}\in\mathbb{R}^{3N_p\times 3N_p},\\
&\tilde{DF}^{-1}=\tilde{DF}^{-1}(\mb{Q})\begin{pmatrix}
\text{diag}(DF^{11}(\mb{Q})) &\text{diag}(DF^{12}(\mb{Q})) &\text{diag}(DF^{13}(\mb{Q})) \\
\text{diag}(DF^{21}(\mb{Q})) &\text{diag}(DF^{22}(\mb{Q})) &\text{diag}(DF^{23}(\mb{Q})) \\
\text{diag}(DF^{31}(\mb{Q})) &\text{diag}(DF^{32}(\mb{Q})) &\text{diag}(DF^{33}(\mb{Q}))
\end{pmatrix}\in\mathbb{R}^{3N_p\times 3N_p}.
\end{align}

Finally, the term involving the hot ion charge density amounts to
\begin{align}
&\left((\ast\rho^3_\text{h})\wedge i_{\#\tilde{U}^1}B^2,\tilde{V}^1\right)=\int_{\hat{\Omega}}\left(\frac{1}{\sqrt{g}}\rho_\mr{h,123}\hat{\mb{B}}\times G^{-1}\tilde{\mb{U}}\right)^\top G^{-1}\mb{V}\sqrt{g}\,\mr{d}^3q\\ 
&\approx \sum_k w_k\mb{V}_h^\top(\mb{q}_k)G^{-1}(\hat{\mb{B}}_h(\mb{q}_k)\times G^{-1}\tilde{\mb{U}}_h(\mb{q}_k))=\mb{v}^\top\mathbb{P}_1\mathbb{W}\tilde{G}^{-1}\hat{\mathbb{B}}\tilde{G}^{-1}\mathbb{P}_1^\top\mb{u},
\end{align}
which is clearly antisymmetric because $\hat{\mathbb{B}}$ is antisymmetric. In total we get the following semi-discrete momentum balance equation:
\begin{align}
\mathcal{A}\dot{\mb{u}}=\mathcal{T}^\top\mathbb{C}^\top\mathbb{M}^2\mb{b}+\mb{b}_\mb{eq}^\top\mathbb{M}^2\mathbb{C}\mathcal{P}^{jki}\mb{b}+\mathbb{P}_1\mathbb{W}\tilde{G}^{-1}\hat{\mathbb{B}}\tilde{G}^{-1}\mathbb{P}_1^\top\mb{u}+\mathbb{P}_1\mathbb{W}\tilde{G}^{-1}\hat{\mathbb{B}}\tilde{DF}^{-1}\mb{V}
\end{align}
where $\mathcal{P}^{jki}$ means that order of the indices is changed from $()_{ijk}$ to $()_{jki}$ such that $\mb{b}$ still contracts with the index $i$ from right and the $\mathbb{C}$ with the index $j$ from left.
 
\subsection{Induction equation}
In contrast to the momentum equation, we keep the induction equation in strong form. This time we have to use the projector $\Pi_2$ which commutes with the curl operator:
\begin{align}
&\frac{\pa \tilde{\mb{B}}_h}{\pa t}+\Pi_2\left[\nabla\times(\hat{\mb{B}}_\text{eq}\times G^{-1}\tilde{\mb{U}}_h)\right]=0\\
\Leftrightarrow\quad &\frac{\pa\mb{b}}{\pa t}+\mathbb{C}\tilde{\Pi}_1\left[\mathbb{B}_\text{eq}G^{-1}(\mathbb{\Lambda}^1)^\top\right]\mb{u}=0 \\
\Leftrightarrow\quad &\frac{\pa\mb{b}}{\pa t}+\mathbb{C}\mathcal{T}\mb{u}=0.
\end{align}
We immediately see that we obtain the same projection matrix as for the Hall term in the momentum equation.

\subsection{Particles' equation of motion}
As a last step, the equations of motion for a single particle with logical coordinate $\mb{q}_k$ and physical velocity $\mb{v}_k$ read
\begin{align}
\frac{\mr{d}\mb{q}_k}{\mr{d}t}&=DF^{-1}(\mb{q}_k)\mb{v}_k,\\
\frac{\mr{d}\mb{v}_k}{\mr{d}t}&=\frac{1}{\sqrt{g}}DF(\mb{q}_k)\hat{\mb{B}}_h(\mb{q}_k)\times DF^{-\top}(\mb{q}_k)\mb{U}_h(\mb{q}_k)-\frac{1}{\sqrt{g}}DF(\mb{q}_k)\hat{\mb{B}}_h(\mb{q}_k)\times\mb{v}_k.\\
&=DF^{-\top}(\mb{q}_k)(\hat{\mb{B}}_h(\mb{q}_k)\times G^{-1}(\mb{q}_k)\mb{U}_h(\mb{q}_k))-DF^{-\top}(\mb{q}_k)(\hat{\mb{B}}_h(\mb{q}_k)\times DF^{-1}\mb{v}_k),
\end{align}
where we used the identity $A\mb{b}\times A\mb{c}=\det(A)A^{-\top}(\mb{b}\times\mb{c})$ from vector calculus. Writing the above equation of motion in matrix-vector form for all particles yields
\begin{align}
&\frac{\mr{d}\mb{Q}}{\mr{d}t}=\tilde{DF}^{-1}\mb{V},\\
&\frac{\mr{d}\mb{V}}{\mr{d}t}=\tilde{DF}^{-\top}\hat{\mathbb{B}}\tilde{G}^{-1}\mathbb{P}^{1\top}\mb{u}-\tilde{DF}^{-\top}\hat{\mathbb{B}}\tilde{DF}^{-1}(\mb{q}_k)\mb{V}.
\end{align}

\subsection{Hamiltonian system}
To write down the semi-discrete system in Hamiltonian form, let us introduce the discrete Hamiltonian
\begin{align}
\mathcal{H}_h:=&\frac{1}{2}\left((\ast\rho_\text{eq}^3)\wedge\tilde{U}_h^1,\tilde{U}_h^1\right)+\frac{1}{2}(\tilde{B}_h^2,\tilde{B}_h^2)+\frac{1}{2}\int_{\hat{\Omega}}\epsilon_{\text{h},123}\text{d}^3q\\
=&\frac{1}{2}\mb{u}^\top\mathcal{A}\mb{u}+\frac{1}{2}\mb{b}^\top\mathbb{M}^2\mb{b}+\frac{1}{2}\mb{V}^\top\mathbb{W}\mb{V}.
\end{align}
With this we can formulate the semi-discrete system in a $4\times 4$ block structure:
\begin{align}
&\frac{\mr{d}}{\mr{d}t}\begin{pmatrix}
\mb{u} \\ \mb{b} \\ \mb{Q} \\ \mb{V}
\end{pmatrix}=\mathbb{J}(\mb{b},\mb{Q})\nabla\mathcal{H}=\mathbb{J}(\mb{b},\mb{Q})\begin{pmatrix}
\mathcal{A}\mb{u} \\
\mathbb{M}^2\mb{b}\\
0 \\
\mathbb{W}\mb{V}
\end{pmatrix}
\end{align}
We end up with an antisymmetric Poisson matrix meaning that the discrete Hamiltonian is conserved. 

\section{Time integration}
\subsection{Poisson splitting}
\begin{align}
&\frac{\mr{d}}{\mr{d}t}\begin{pmatrix}
\mb{u} \\ \mb{b} \\ \mb{Q} \\ \mb{V}
\end{pmatrix}=
\begin{pmatrix}
\textcolor{red}{\mathbb{J}_{11}(\mb{b},\mb{Q})} &\textcolor{blue}{\mathbb{J}_{12}} &0 &\textcolor{green}{\mathbb{J}_{14}(\mb{b},\mb{Q})}\\
\textcolor{blue}{-\mathbb{J}_{12}^\top} &0 &0 &0 \\
0 &0 &0 &\textcolor{magenta}{\mathbb{J}_{34}(\mb{Q})} \\
\textcolor{green}{-\mathbb{J}_{14}^\top(\mb{b},\mb{Q})} &0 &\textcolor{magenta}{-\mathbb{J}_{34}^\top(\mb{Q})} &\mathbb{J}_{44}(\mb{b},\mb{Q})
\end{pmatrix}\begin{pmatrix}
\mathcal{A}\mb{u} \\
\mathbb{M}^2\mb{b}\\
0 \\
\mathbb{W}\mb{V}
\end{pmatrix}\\[1cm]
&\textcolor{red}{\mathbb{J}_{11}(\mb{b},\mb{Q})}=\mathcal{A}^{-1}\mathbb{P}_1(\mb{Q})\mathbb{W}\tilde{G}^{-1}(\mb{Q})\hat{\mathbb{B}}(\mb{b},\mb{Q})\tilde{G}^{-1}(\mb{Q})\mathbb{P}_1^\top(\mb{Q})\mathcal{A}^{-1}\\
&\textcolor{blue}{\mathbb{J}_{12}}=\mathcal{A}^{-1}\mathcal{T}^\top\mathbb{C}^\top\\
&\textcolor{green}{\mathbb{J}_{14}(\mb{b},\mb{Q})}=\mathcal{A}^{-1}\mathbb{P}_1(\mb{Q})\tilde{G}^{-1}(\mb{Q})\hat{\mathbb{B}}(\mb{b},\mb{Q})\tilde{DF}^{-1}(\mb{Q})\\
&\textcolor{magenta}{\mathbb{J}_{34}(\mb{Q})}=\tilde{DF}^{-1}(\mb{Q})\mathbb{W}^{-1}\\
&\mathbb{J}_{44}(\mb{Q})=-\tilde{DF}^{-\top}(\mb{Q})\hat{\mathbb{B}}(\mb{b},\mb{Q})\tilde{DF}^{-1}(\mb{Q})\mathbb{W}^{-1}
\end{align}
\paragraph{Sub-step 1}
The first sub-system reads
\begin{align}
&\dot{\mb{u}}=\mathbb{J}_{11}(\mb{b},\mb{Q})\mathcal{A}\mb{u}\\
&\dot{\mb{b}}=0\\
&\dot{\mb{Q}}=0\\
&\dot{\mb{V}}=0
\end{align}
We solve the this equation with the Crank-Nicolson method:
\begin{align}
\frac{\mb{u}^{n+1}-\mb{u}^n}{\Delta t}=\mathbb{J}_{11}(\mb{b}^n,\mb{Q}^n)\mathcal{A}\frac{\mb{u}^n+\mb{u}^{n+1}}{2}
\end{align} 
\paragraph{Sub-step 2}
The second sub-system reads
\begin{align}
&\dot{\mb{u}}=\mathbb{J}_{12}\mathbb{M}^2\mb{b}\\
&\dot{\mb{b}}=-\mathbb{J}_{12}^\top\mathcal{A}\mb{u}\\
&\dot{\mb{Q}}=0\\
&\dot{\mb{V}}=0
\end{align}
We solve the first two equations analytically with an exponential integrator:
\begin{align}
\begin{pmatrix}
\mb{u}^{n+1}\\ \mb{b}^{n+1}
\end{pmatrix}=\exp\begin{pmatrix}
0 &\mathbb{J}_{12}\mathbb{M}^2\Delta t\\-\mathbb{J}_{12}^\top\mathcal{A}\Delta t &0
\end{pmatrix}\begin{pmatrix}
\mb{u}^{n}\\ \mb{b}^{n}
\end{pmatrix}
\end{align}
\paragraph{Sub-step 3}
The third sub-system reads
\begin{align}
&\dot{\mb{u}}=\mathbb{J}_{14}(\mb{b},\mb{Q})\mathbb{W}\mb{V}\\
&\dot{\mb{b}}=0\\
&\dot{\mb{Q}}=0\\
&\dot{\mb{V}}=-\mathbb{J}_{14}^\top(\mb{b},\mb{Q})\mathcal{A}\mb{u}
\end{align}
\paragraph{Sub-step 4}
The fourth sub-system reads
\begin{align}
&\dot{\mb{u}}=0\\
&\dot{\mb{b}}=0\\
&\dot{\mb{Q}}=\mathbb{J}_{34}^\top(\mb{b},\mb{Q})\mathbb{W}\mb{V}\\
&\dot{\mb{V}}=0
\end{align}
\paragraph{Sub-step 5}
The fifth sub-system reads
\begin{align}
&\dot{\mb{u}}=0\\
&\dot{\mb{b}}=0\\
&\dot{\mb{Q}}=0\\
&\dot{\mb{V}}=\mathbb{J}_{44}^\top(\mb{b},\mb{Q})\mathbb{W}\mb{V}
\end{align}
\appendix
\begin{align}
\mathbb{J}(\mb{b},\mb{Q})=\\
&\rotatebox{90}{$\begin{pmatrix}
\mathcal{A}^{-1}\mathbb{P}_1(\mb{Q})\mathbb{W}\tilde{G}^{-1}(\mb{Q})\hat{\mathbb{B}}(\mb{b},\mb{Q})\tilde{G}^{-1}(\mb{Q})\mathbb{P}_1^\top(\mb{Q})\mathcal{A}^{-1} &\mathcal{A}^{-1}\mathcal{T}^\top\mathbb{C}^\top &0 &\mathcal{A}^{-1}\mathbb{P}_1(\mb{Q})\tilde{G}^{-1}(\mb{Q})\hat{\mathbb{B}}(\mb{b},\mb{Q})\tilde{DF}^{-1}(\mb{Q})\\
-\mathbb{C}\mathcal{T}\mathcal{A}^{-1} &0 &0 &0 \\
0 &0 &0 &\tilde{DF}^{-1}(\mb{Q})\mathbb{W}^{-1} \\
\tilde{DF}^{-\top}(\mb{Q})\hat{\mathbb{B}}(\mb{b},\mb{Q})\tilde{G}^{-1}(\mb{Q})\mathbb{P}^{1\top}(\mb{Q})\mathcal{A}^{-1} &0 &-\mathbb{W}^{-1}\tilde{DF}^{-\top}(\mb{Q}) &-\tilde{DF}^{-\top}(\mb{Q})\hat{\mathbb{B}}(\mb{b},\mb{Q})\tilde{DF}^{-1}(\mb{Q})\mathbb{W}^{-1}
\end{pmatrix}$}
\end{align}




\end{document}