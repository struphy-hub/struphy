%% This is file `jcomp-template.tex',
%% 
%% Copyright 2017 Elsevier Ltd
%% 
%% This file is part of the 'Elsarticle Bundle'.
%% ---------------------------------------------
%% 
%% It may be distributed under the conditions of the LaTeX Project Public
%% License, either version 1.2 of this license or (at your option) any
%% later version.  The latest version of this license is in
%%    http://www.latex-project.org/lppl.txt
%% and version 1.2 or later is part of all distributions of LaTeX
%% version 1999/12/01 or later.
%% 
%% The list of all files belonging to the 'Elsarticle Bundle' is
%% given in the file `manifest.txt'.
%% 
%% Template article for Elsevier's document class `elsarticle'
%% with harvard style bibliographic references
%%
%% $Id: jcomp-template.tex 100 2017-07-14 13:15:12Z rishi $
%%
%% Use the option review to obtain double line spacing
%\documentclass[times,review,preprint,authoryear]{elsarticle}

%% Use the options `twocolumn,final' to obtain the final layout
%% Use longtitle option to break abstract to multiple pages if overfull.
%% For Review pdf (With double line spacing)
%\documentclass[times,twocolumn,review]{elsarticle}
%% For abstracts longer than one page.
%\documentclass[times,twocolumn,review,longtitle]{elsarticle}
%% For Review pdf without preprint line
%\documentclass[times,twocolumn,review,nopreprintline]{elsarticle}
%% Final pdf
\documentclass[fleqn,times,final]{elsarticle}
%%
%\documentclass[times,twocolumn,final,longtitle]{elsarticle}
%%

% Commands
\newcommand{\pa}{\partial}
\newcommand{\mr}[1]{\mathrm{#1}}
\newcommand{\mb}[1]{\mathbf{#1}}
\newcommand{\vpar}{v_{\mr{th}\parallel}}
\newcommand{\vperp}{v_{\mr{th}\perp}}


%% Stylefile to load JCOMP template
\usepackage{jcomp}
\usepackage{framed,multirow}

%% The amssymb package provides various useful mathematical symbols
\usepackage{amssymb}
\usepackage{latexsym}
\usepackage{subfigure}
\usepackage{amsmath}
\usepackage[overload]{empheq}
\usepackage{calc}
\usepackage{wrapfig}
\usepackage{enumitem}
\usepackage[toc,page]{appendix}


% Following three lines are needed for this document.
% If you are not loading colors or url, then these are
% not required.
\usepackage{url}
\usepackage{xcolor}
\usepackage{pgf}
\definecolor{newcolor}{rgb}{.8,.349,.1}

\journal{Journal of Computational Physics}

\begin{document}

\verso{\textit{F. Holderied et al.}}

\begin{frontmatter}

\title{Comparison of standard and structure-preserving finite element particle-in-cell methods for a four-dimensional electron hybrid plasma model}%

\author[1]{Florian \snm{Holderied}\corref{cor1}}
\cortext[cor1]{Corresponding author:}
\ead{florian.holderied@ipp.mpg.de}
\author[1,2]{Stefan \snm{Possanner}}
\author[1]{Xin \snm{Wang}}
\author[1]{Ahmed \snm{Ratnani}}

\address[1]{Max Planck Institute for Plasma Physics, Boltzmannstrasse 2, 85748 Garching, Germany}
\address[2]{Technical University of Munich, Department of Mathematics, Boltzmannstrasse 3, 85748 Garching, Germany}

\received{1 May 2013}
\finalform{10 May 2013}
\accepted{13 May 2013}
\availableonline{15 May 2013}
\communicated{S. Sarkar}


\begin{abstract}
%%%
Two numerical methods, which both belong to the class of finite element particle-in-cell methods, have been applied on a four-dimensional (one dimension in real space and three dimensions in velocity space) hybrid plasma model for electrons in a stationary, neutralizing background of ions. Here, the term \textit{hybrid} means that (energetic) electrons with velocities close to the phase velocities of the model's characteristic waves are treated kinetically whereas electrons that are much slower than the phase velocity are treated with fluid equations. The two developed numerical schemes, which belong on the one hand to the class of standard finite element particle-in-cell methods and on the other hand to the more modern class of structure-preserving finite element particle-in-cell methods, which use techniques from the \textit{finite element exterior calculus} (FEEC), have been implemented, tested successfully by using results from the analytical theory in the linear stage and compared in terms of long-term energy conservation, which is expected on the continuous level. Regarding the latter, we show that FEEC applied on the model leads to better results which is due to the fact the spatial discretization gives rise to a large system of ordinary differential equations in time that exhibits a non-canonical Hamiltonian structure for which special time integration schemes with good conservation properties exist.
%%%%
\end{abstract}

%\begin{keyword}
%% MSC codes here, in the form: \MSC code \sep code
%% or \MSC[2008] code \sep code (2000 is the default)
%\MSC 41A05\sep 41A10\sep 65D05\sep 65D17
%% Keywords
%\KWD Keyword1\sep Keyword2\sep Keyword3
%\end{keyword}

\end{frontmatter}

%\linenumbers

%% main text
 
\section{Introduction}
\label{sec_intro}
We present a comparison of two novel algorithms for the numerical solution of a hybrid plasma model in order to demonstrate similarities and differences of standard finite element particle-in-cell methods compared to structure-preserving finite element particle-in-cell methods. The latter use techniques from the \textit{finite element exterior calculus} \citep{Arnoldetal2006} and were first introduced by Kraus et al. for the full six-dimensional Vlasov-Maxwell model \citep{Krausetal2017}, which is a coupled system of partial differential equations in phase space describing the self-consistent dynamics of charged particles in electromagnetic fields generated by the particles themselves as well as generated externally. By taking into account the geometric structure of the system of equations in terms of differential geometry, such methods naturally preserve conservation laws like conservation of energy, for instance, as well as the two Gauss's laws of electrodynamics, $\nabla\cdot\mb{E}=\rho/\epsilon_0$ and $\nabla\cdot\mb{B}=0$ exactly on the semi-discrete level (discrete in space and continuous in time). Here, $\mb{E}=\mb{E}(t,\mb{x})$, $\mb{B}=\mb{B}(t,\mb{x})$, $\rho=\rho(t,\mb{x})$ and $\epsilon_0$ denote the electric and magnetic field, the charge density and the vacuum permittivity, respectively. As shown by Arnold, Falk \& Winther , this goes hand in hand with numerical stability \citep{Arnoldetal2010}. In this work, we shall apply these methods as well as classical finite element particle-in-cell methods on an extended model which falls into the class of so-called \textit{hybrid} plasma models, which use a combined fluid/kinetic description for different particle species in order to get a good balance between accuracy (kinetic models in the six-dimensional phase space) and computational costs (fluid models in the three-dimensional real space). The investigated model, which will be introduced in the next section, is applicable to plasma dynamics in planetary magnetospheres, for instance, and has been used successfully for the simulation of a special type of electromagnetic waves called \textit{Chorus waves}. These waves are electromagnetic emissions whose frequency-time-spectrograms show a series of discrete elements
with rising frequencies with respect to time. An important condition for its excitation
is the injection of energetic electrons with an anisotropic velocity distribution with respect to the earth's magnetic field into the magnetosphere which then interact  with waves propagating in a thermal background plasma \citep{Katohetal2007, Tao2014}.
 
\section{Theoretical background}
\label{sec_theory}

\subsection{The model}
\label{sec_model}
The plasma model, which is the scope of this work, is a high-frequency plasma model, i.e. wave frequencies $\omega$ are of the order of the electron cyclotron frequency $\Omega_\mr{ce}=q_\mr{e}|\mb{B}|/m_\mr{e}$, which is why the plasma ions (denoted by the subscript i) cannot react on the fast fluctuations of the electromagnetic fields and are therefore treated as a stationary, neutralizing background. $q_\mr{e}$ and $m_\mr{e}$ are the electron charge and mass, respectively. Furthermore, we assume that the electron population consists mainly of cold electrons (denoted by the subscript c for ``cold''), which are in local thermal equilibrium and have negligible thermal effects (temperature $T_\mr{c}\approx0$). In this case, fluid equations without thermal forces are applicable. Moreover, we assume that there is a small amount of energetic electrons (denoted by the subscript h for ``hot'') for which we shall use a kinetic description with negligible collisionality by assuming that the average collision times are much larger than the considered time scales. Using the mass continuity and momentum balance equation for the cold electrons, the Vlasov equation for the energetic electrons and Maxwell's equations for the self-consistent dynamics of the electromagnetic fields, the full set of equations in SI-units reads 
\begin{subequations}
\label{eq_model_full}
\begin{align}[left ={\text{cold fluid electrons}\hspace{2.4mm}\empheqlbrace}]
&\frac{\pa n_\mr{c}}{\pa t}+\nabla\cdot(n_\mr{c}\mb{u}_\mr{c})=0,\label{eq_model_full_cold_1}\\
&\frac{\partial \mb{u}_\mr{c}}{\partial t}+(\mb{u}_\mr{c}\cdot\nabla)\mb{u}_\mr{c}=\frac{q_\mr{e}}{m_\mr{e}}(\mb{E}+\mb{u}_\mr{c}\times\mb{B}),\label{eq_model_full_cold_2}\\
&\mb{j}_\mr{c}=q_\mr{e}n_\mr{c}\mb{u}_\mr{c},\label{eq_model_full_cold_3}
\end{align}
\begin{align}[left ={\text{hot kinetic electrons}\hspace{0.7mm}\empheqlbrace}]
&\frac{\pa f_\mr{h}}{\pa t}+\mb{v}\cdot\nabla f_\mr{h}+\frac{q_\mr{e}}{m_\mr{e}}(\mb{E}+\mb{v}\times\mb{B})\cdot\nabla_\mb{v}f_\mr{h}=0,\label{eq_model_full_hot_1}\\
&n_\mr{h}=\int f_\mr{h}\mr{d}^3\mb{v},\label{eq_model_full_hot_2}\\
&\mb{j}_\mr{h}=q_\mr{e}\int f_\mr{h}\mb{v}\mr{d}^3\mb{v},\label{eq_model_full_hot_3}
\end{align}
\begin{align}[left ={\text{Maxwell's equations}\empheqlbrace}]
&\frac{\pa \mb{B}}{\pa t}=-\nabla\times\mb{E},\label{eq_model_full_maxwell_1}\\
&\frac{1}{c^2}\frac{\pa \mb{E}}{\pa t}=\nabla\times\mb{B}-\mu_0(\mb{j}_\mr{c}+\mb{j}_\mr{h}),\label{eq_model_full_maxwell_2}\\
&\nabla\cdot\mb{E}=\frac{1}{\epsilon_0}[q_\mr{i}n_\mr{i}+q_\mr{e}(n_\mr{c}+n_\mr{h})],\label{eq_model_full_maxwell_3}\\
&\nabla\cdot\mb{B}=0,\label{eq_model_full_maxwell_4}
\end{align}
\end{subequations}
where, as stated above, the ions shall form a stationary background. This implies a constant number density $n_\mr{i}=n_\mr{i}(\mb{x})$ in time, i.e. $\pa n_\mr{i}/\pa t=0$, and a vanishing ion current $\mb{j}_\mr{i}=0$ for all times. Furthermore, $n_{\mr{c}/\mr{h}}=n_{\mr{c}/\mr{h}}(t,\mb{x})$ denotes the number density of the cold/hot electrons, $\mb{j}_{\mr{c}/\mr{h}}$ the current densities, $\mb{u}_\mr{c}=\mb{u}_\mr{c}(t,\mb{x})$ the mean velocity of the cold electrons and $f_\mr{h}=f_\mr{h}(t,\mb{v},\mb{x})$ the distribution function of the energetic electrons, respectively. Furthermore, $c$ is the speed of light and $\mu_0$ the vaccuum permeability, all satisfying $c^2=1/\mu_0\epsilon_0$. Note that, roughly speaking, the cold plasma approximation is valid as long as the thermal velocity of a particle species is much smaller than the phase velocity of the considered wave \citep{Brambilla1998}.

\subsection{Model reduction}
The full model (\ref{eq_model_full}) can be reduced to an equivalent set of equations for the evolution of the fields $\mb{u}_\mr{c}$, $\mb{E}$ and $\mb{B}$ and the distribution function $f_\mr{h}$ with the constraint that the two Gauss's laws (\ref{eq_model_full_maxwell_3}) and (\ref{eq_model_full_maxwell_4}) must be satisfied at the initial time $t=0$. The reduced model then takes the form
\begin{subequations}
\label{eq_model_reduced}
\begin{align}
&\frac{\partial \mb{u}_\mr{c}}{\partial t}+(\mb{u}_\mr{c}\cdot\nabla)\mb{u}_\mr{c}=\frac{q_\mr{e}}{m_\mr{e}}(\mb{E}+\mb{v}\times\mb{B}),\label{eq_model_reduced_1}\\
&\frac{\pa f_\mr{h}}{\pa t}+\mb{v}\cdot\nabla f_\mr{h}+\frac{q_\mr{e}}{m_\mr{e}}(\mb{E}+\mb{v}\times\mb{B})\cdot\nabla_\mb{v}f_\mr{h}=0,\label{eq_model_reduced_2}\\
&\frac{\pa \mb{B}}{\pa t}=-\nabla\times\mb{E},\label{eq_model_reduced_3}\\
&\frac{1}{c^2}\frac{\pa \mb{E}}{\pa t}=\nabla\times\mb{B}-\mu_0(\mb{j}_\mr{c}+\mb{j}_\mr{h}),\label{eq_model_reduced_4}
\end{align}
\end{subequations}
combined with the aforementioned constraints at $t=0$ and the definitions of the current densities (\ref{eq_model_full_cold_3}) and (\ref{eq_model_full_hot_3}) and the definition of the hot electron number density (\ref{eq_model_full_hot_2}), respectively. The proof that the model (\ref{eq_model_reduced}) is indeed equivalent to the full model ( \ref{eq_model_full}) follows directly from the fact that Faraday's law conserves the divergence constraint for the magnetic field,
\begin{align}
\nabla\cdot\left(\frac{\pa \mb{B}}{\pa t}+\nabla\times\mb{E}\right)=\frac{\pa}{\pa t}(\nabla\cdot\mb{B})\overset{!}{=}0,
\end{align}
i.e. the divergence constraint remains satisfied at late times $t>0$ provided that it was satisfied at the initial time $t=0$. Likewise, the mass continuity equation for the fluid electrons (\ref{eq_model_full_cold_1}) is automatically satisfied by Amp\'{e}re's law (\ref{eq_model_full_maxwell_2}) by assuming that the cold electron number density $n_\mr{c}$ can be reconstructed from (\ref{eq_model_full_maxwell_4}) at any time $t\geq0$:
\begin{align}
&\nabla\cdot\left[\frac{1}{c^2}\frac{\pa\mb{E}}{\pa t}-\nabla\times\mb{B}+\mu_0(\mb{j}_\mr{c}+\mb{j}_\mr{h})\right]=\frac{1}{c^2}\frac{\pa}{\pa t}(\nabla\cdot\mb{E})+\mu_0\nabla\cdot(\mb{j}_\mr{c}+\mb{j}_\mr{h})=\frac{q_\mr{e}}{c^2\epsilon_0	}\frac{\pa}{\pa t}(n_\mr{c}+n_\mr{h})+\mu_0\nabla\cdot(\mb{j}_\mr{c}+\mb{j}_\mr{h})\\
&=q_\mr{e}\mu_0\underbrace{\left[\frac{\pa n_\mr{c}}{\pa t}+\nabla\cdot(n_\mr{c}\mb{u}_\mr{c})\right]}_{\text{cont. eq. (\ref{eq_model_full_cold_1})}}+\mu_0\underbrace{\left(\frac{\pa n_\mr{h}}{\pa t}+\nabla\cdot\mb{j}_\mr{h}\right)}_{=0}\overset{!}{=}0.
\end{align}
The disappearance of the round bracket for the hot electrons in the second line follows from the fact that the first velocity moment of the Vlasov equation leads to the mass continuity equation \citep{Brambilla1998}. Thus, the divergence of Amp\'{e}re's law reduces to the the mass continuity equation for the fluid electrons, which is therefore satisfied automatically. In summary, we have shown that solutions of the reduced model (\ref{eq_model_reduced}) with compatible initial conditions are indeed solutions of the full model (\ref{eq_model_full}).

The model can further be simplified by only considering small perturbations (denoted by tildes) about an time-independent equilibrium state (denoted by the subscript ``0''). In this case, the fluid quantities and the electromagnetic fields are expressed as
\begin{subequations}
\label{eq_linearization}
\begin{align}
&n_\mr{c}(\mb{x},t)=n_{\mr{c}0}(\mb{x})+\tilde{n}_\mr{c}(\mb{x},t),\label{eq_linearization_1}\\
&\mb{u}_\mr{c}(\mb{x},t)=\tilde{\mb{u}}_\mr{c}(\mb{x},t),\label{eq_linearization_2}\\
&\mb{B}(\mb{x},t)=\mb{B}_0(\mb{x})+\tilde{\mb{B}}(\mb{x},t),\label{eq_linearization_3}\\
&\mb{E}(\mb{x},t)=\tilde{\mb{E}}(\mb{x},t),\label{eq_linearization_4}
\end{align}  
\end{subequations}
where we have assumed that there is no background electric field and no equilibrium plasma flow (which also means that there is no cold equilibrium current and thus $\nabla\times\mb{B}_0=-\mu_0\mb{j}_\mr{h}$ must be satisfied). In what follows, nonlinear terms in the perturbations are neglected. E.g. the cold current density transfers to
\begin{align}
\mb{j}_{c}=q_\mr{e}(n_{\mr{c}0}+\tilde{n}_{c})\tilde{\mb{u}}_\mr{c}\approx q_\mr{e}n_{\mr{c}0}\tilde{\mb{u}}_\mr{c},
\end{align}
which leads to a modified momentum balance equation by first linearizing (\ref{eq_model_reduced_1}) and subsequently expressing $\tilde{\mb{u}}_\mr{c}$ in terms of $\mb{j}_\mr{c}$ according to the above expression. Finally, this leads to the model
\begin{subequations}
\label{eq_model_linearized}
\begin{align}
&\frac{\partial\mb{j}_\mr{c}}{\pa t}=\epsilon_0\Omega_\mr{pe}^2\tilde{\mb{E}}+\mb{j}_\mr{c}\times\mb{\Omega}_\mr{ce},\label{eq_model_linearized_1}\\
&\frac{\pa f_\mr{h}}{\pa t}+\mb{v}\cdot\nabla f_\mr{h}+\frac{q_\mr{e}}{m_\mr{e}}(\mb{E}+\mb{v}\times\mb{B})\cdot\nabla_\mb{v}f_\mr{h}=0,\label{eq_model_linearized_2}\\
&\frac{\pa \tilde{\mb{B}}}{\pa t}=-\nabla\times\tilde{\mb{E}},\label{eq_model_linearized_3}\\
&\frac{1}{c^2}\frac{\pa \tilde{\mb{E}}}{\pa t}=\nabla\times\tilde{\mb{B}}-\mu_0(\mb{j}_\mr{c}+\mb{j}_\mr{h}),\label{eq_model_linearized_4}
\end{align}
\end{subequations}
where we have introduced the spatially dependent electron plasma frequency $\Omega_\mr{pe}^2(\mb{x})=e^2n_{\mr{c}0}(\mb{x})/\epsilon_0m_\mr{e}$ of the cold electrons and the oriented electron cyclotron frequency $\mb{\Omega}_\mr{ce}(\mb{x})=q_\mr{e}\mb{B}_0(\mb{x})/m_\mr{e}$. Note that we keep nonlinearities for the Vlasov equation (\ref{eq_model_linearized_2}) in order to model nonlinear wave-particle interaction as well as to enable the application of the particle-in-cell method. 

\subsection{Energy theorem}
An important property of the linearized model (\ref{eq_model_linearized}) is that its dynamics conserves the total energy 
\begin{align}
\epsilon:=\underbrace{\frac{\epsilon_0}{2}\int_\Omega\tilde{\mb{E}}^2\mr{d}^3\mb{x}}_{=:\epsilon_E}+\underbrace{\frac{2}{\mu_0}\int
_\Omega\tilde{\mb{B}}^2\mr{d}^3\mb{x}}_{=:\epsilon_B}+\underbrace{\frac{1}{2\epsilon_0}\int_\Omega\frac{1}{\Omega_\mr{pe}^2}\mb{j}_\mr{c}^2\mr{d}^3\mb{x}}_{:=\epsilon_\mr{c}}+\underbrace{\frac{m_\mr{e}}{2}\int_\Omega\int v^2f_\mr{h}\mr{d}^3\mb{v}\mr{d}^3\mb{x}}_{\epsilon_\mr{h}}\label{eq_total_energy}
\end{align}
in the domain $\Omega=\mathbb{R}^3$, which is the sum of the electric field energy $\epsilon_E$, the magnetic field energy $\epsilon_B$, the kinetic energy of the cold electrons $\epsilon_\mr{c}$ and the kinetic energy of the hot electrons $\epsilon_\mr{h}$, respectively. It is relatively straightforward to proof this property by computing $\mr{d}\epsilon/\mr{d}t$, using the dynamical equations (\ref{eq_model_linearized}) to replace the occurring partial time derivatives, noting that all quantities vanish at infinity (or assuming a periodic domain) and then summing everything up to show that $\mr{d}\epsilon/\mr{d}t=0$. We will use this energy conservation property later as a criterion for the performances of the developed numerical schemes.

\subsection{Linear dispersion relation}
A linear dispersion relation for the fully linearized model (\ref{eq_model_linearized}) can be derived for the case of wave propagation parallel to a uniform magnetic field $\mb{B}_0=B_0\mb{e}_z$, i.e. the wave vector $\mb{k}=k\mb{e}_z\parallel\mb{B}_0\parallel\mb{e}_z$ and a uniform plasma in the equilibrium state. The latter implies a constant cold electron plasma frequency $\Omega_\mr{pe}(\mb{x})=\Omega_\mr{pe}=const.$ and a uniform equilibrium distribution function $f_\mr{h}^0=f_\mr{h}^0(\mb{v})$ for the hot electrons. In analogy to (\ref{eq_linearization}), the distribution function is thus split into an equilibrium part and a fluctuating part
\begin{align}
f_\mr{h}(\mb{x},\mb{v},t)=f_\mr{h}^0(\mb{v})+\tilde{f}_\mr{h}(\mb{x},\mb{v},t),
\end{align}
with $\tilde{f}_\mr{h}\ll f_\mr{h}^0$. Plugging this in the Vlasov equation (\ref{eq_model_linearized_2}), neglecting nonlinear terms in the perturbed quantities and relabeling ($\tilde{\mb{B}}\rightarrow\mb{B}$, $\tilde{f}_\mr{h}\rightarrow f_\mr{h}$) for reasons of clarity yields the fully linearized model
\begin{subequations}
\label{eq_model_fullylinearized}
\begin{align}
&\frac{\partial\mb{j}_\mr{c}}{\pa t}=\epsilon_0\Omega_\mr{pe}^2\mb{E}+\Omega_\mr{ce}\mb{j}_\mr{c}\times\mb{e}_z,\label{eq_model_fullylinearized_1}\\
&\frac{\pa f_\mr{h}}{\pa t}+\mb{v}\cdot\nabla f_\mr{h}+\Omega_\mr{ce}(\mb{v}\times\mb{e}_z)\cdot\nabla_\mb{v}f_\mr{h}=-\frac{q_\mr{e}}{m_\mr{e}}(\mb{E}+\mb{v}\times\mb{B})\cdot\nabla_\mb{v}f_\mr{h}^0,\label{eq_model_fullylinearized_2}\\
&\frac{\pa \mb{B}}{\pa t}=-\nabla\times\mb{E},\label{eq_model_fullylinearized_3}\\
&\frac{1}{c^2}\frac{\pa \mb{E}}{\pa t}=\nabla\times\mb{B}-\mu_0(\mb{j}_\mr{c}+\mb{j}_\mr{h}).\label{eq_model_fullylinearized_4}
\end{align}
\end{subequations}
Note that $\Omega_\mr{ce}<0$ for electrons ($q_\mr{e}=-e$ with $e$ being the elementary charge). In the above stated case of parallel wave propagation, the problem becomes additionally one-dimensional in space, which is why $\nabla=\mb{e}_z\pa/\pa z$ in (\ref{eq_model_fullylinearized}). By looking for plane wave solutions for all quantities $\sim \exp[i(kz-\omega t)]$ one ends up with three types of solutions \citep{Brambilla1998, Xiaoetal1998}: One of these solutions corresponds to electrostatic waves (longitudinal waves in direction of the background magnetic field) which we do not consider further. 
\begin{figure}[!t]
\centering
\subfigure[]{%% Creator: Matplotlib, PGF backend
%%
%% To include the figure in your LaTeX document, write
%%   \input{<filename>.pgf}
%%
%% Make sure the required packages are loaded in your preamble
%%   \usepackage{pgf}
%%
%% Figures using additional raster images can only be included by \input if
%% they are in the same directory as the main LaTeX file. For loading figures
%% from other directories you can use the `import` package
%%   \usepackage{import}
%% and then include the figures with
%%   \import{<path to file>}{<filename>.pgf}
%%
%% Matplotlib used the following preamble
%%   \usepackage{fontspec}
%%   \setmainfont{DejaVu Serif}
%%   \setsansfont{DejaVu Sans}
%%   \setmonofont{DejaVu Sans Mono}
%%
\begingroup%
\makeatletter%
\begin{pgfpicture}%
\pgfpathrectangle{\pgfpointorigin}{\pgfqpoint{2.934722in}{2.188841in}}%
\pgfusepath{use as bounding box, clip}%
\begin{pgfscope}%
\pgfsetbuttcap%
\pgfsetmiterjoin%
\definecolor{currentfill}{rgb}{1.000000,1.000000,1.000000}%
\pgfsetfillcolor{currentfill}%
\pgfsetlinewidth{0.000000pt}%
\definecolor{currentstroke}{rgb}{1.000000,1.000000,1.000000}%
\pgfsetstrokecolor{currentstroke}%
\pgfsetdash{}{0pt}%
\pgfpathmoveto{\pgfqpoint{0.000000in}{0.000000in}}%
\pgfpathlineto{\pgfqpoint{2.934722in}{0.000000in}}%
\pgfpathlineto{\pgfqpoint{2.934722in}{2.188841in}}%
\pgfpathlineto{\pgfqpoint{0.000000in}{2.188841in}}%
\pgfpathclose%
\pgfusepath{fill}%
\end{pgfscope}%
\begin{pgfscope}%
\pgfsetbuttcap%
\pgfsetmiterjoin%
\definecolor{currentfill}{rgb}{1.000000,1.000000,1.000000}%
\pgfsetfillcolor{currentfill}%
\pgfsetlinewidth{0.000000pt}%
\definecolor{currentstroke}{rgb}{0.000000,0.000000,0.000000}%
\pgfsetstrokecolor{currentstroke}%
\pgfsetstrokeopacity{0.000000}%
\pgfsetdash{}{0pt}%
\pgfpathmoveto{\pgfqpoint{0.461111in}{0.526079in}}%
\pgfpathlineto{\pgfqpoint{2.786111in}{0.526079in}}%
\pgfpathlineto{\pgfqpoint{2.786111in}{2.036079in}}%
\pgfpathlineto{\pgfqpoint{0.461111in}{2.036079in}}%
\pgfpathclose%
\pgfusepath{fill}%
\end{pgfscope}%
\begin{pgfscope}%
\pgfsetbuttcap%
\pgfsetroundjoin%
\definecolor{currentfill}{rgb}{0.000000,0.000000,0.000000}%
\pgfsetfillcolor{currentfill}%
\pgfsetlinewidth{0.803000pt}%
\definecolor{currentstroke}{rgb}{0.000000,0.000000,0.000000}%
\pgfsetstrokecolor{currentstroke}%
\pgfsetdash{}{0pt}%
\pgfsys@defobject{currentmarker}{\pgfqpoint{0.000000in}{-0.048611in}}{\pgfqpoint{0.000000in}{0.000000in}}{%
\pgfpathmoveto{\pgfqpoint{0.000000in}{0.000000in}}%
\pgfpathlineto{\pgfqpoint{0.000000in}{-0.048611in}}%
\pgfusepath{stroke,fill}%
}%
\begin{pgfscope}%
\pgfsys@transformshift{0.461111in}{0.526079in}%
\pgfsys@useobject{currentmarker}{}%
\end{pgfscope}%
\end{pgfscope}%
\begin{pgfscope}%
\pgftext[x=0.461111in,y=0.428857in,,top]{\rmfamily\fontsize{10.000000}{12.000000}\selectfont \(\displaystyle 0\)}%
\end{pgfscope}%
\begin{pgfscope}%
\pgfsetbuttcap%
\pgfsetroundjoin%
\definecolor{currentfill}{rgb}{0.000000,0.000000,0.000000}%
\pgfsetfillcolor{currentfill}%
\pgfsetlinewidth{0.803000pt}%
\definecolor{currentstroke}{rgb}{0.000000,0.000000,0.000000}%
\pgfsetstrokecolor{currentstroke}%
\pgfsetdash{}{0pt}%
\pgfsys@defobject{currentmarker}{\pgfqpoint{0.000000in}{-0.048611in}}{\pgfqpoint{0.000000in}{0.000000in}}{%
\pgfpathmoveto{\pgfqpoint{0.000000in}{0.000000in}}%
\pgfpathlineto{\pgfqpoint{0.000000in}{-0.048611in}}%
\pgfusepath{stroke,fill}%
}%
\begin{pgfscope}%
\pgfsys@transformshift{1.042361in}{0.526079in}%
\pgfsys@useobject{currentmarker}{}%
\end{pgfscope}%
\end{pgfscope}%
\begin{pgfscope}%
\pgftext[x=1.042361in,y=0.428857in,,top]{\rmfamily\fontsize{10.000000}{12.000000}\selectfont \(\displaystyle 2\)}%
\end{pgfscope}%
\begin{pgfscope}%
\pgfsetbuttcap%
\pgfsetroundjoin%
\definecolor{currentfill}{rgb}{0.000000,0.000000,0.000000}%
\pgfsetfillcolor{currentfill}%
\pgfsetlinewidth{0.803000pt}%
\definecolor{currentstroke}{rgb}{0.000000,0.000000,0.000000}%
\pgfsetstrokecolor{currentstroke}%
\pgfsetdash{}{0pt}%
\pgfsys@defobject{currentmarker}{\pgfqpoint{0.000000in}{-0.048611in}}{\pgfqpoint{0.000000in}{0.000000in}}{%
\pgfpathmoveto{\pgfqpoint{0.000000in}{0.000000in}}%
\pgfpathlineto{\pgfqpoint{0.000000in}{-0.048611in}}%
\pgfusepath{stroke,fill}%
}%
\begin{pgfscope}%
\pgfsys@transformshift{1.623611in}{0.526079in}%
\pgfsys@useobject{currentmarker}{}%
\end{pgfscope}%
\end{pgfscope}%
\begin{pgfscope}%
\pgftext[x=1.623611in,y=0.428857in,,top]{\rmfamily\fontsize{10.000000}{12.000000}\selectfont \(\displaystyle 4\)}%
\end{pgfscope}%
\begin{pgfscope}%
\pgfsetbuttcap%
\pgfsetroundjoin%
\definecolor{currentfill}{rgb}{0.000000,0.000000,0.000000}%
\pgfsetfillcolor{currentfill}%
\pgfsetlinewidth{0.803000pt}%
\definecolor{currentstroke}{rgb}{0.000000,0.000000,0.000000}%
\pgfsetstrokecolor{currentstroke}%
\pgfsetdash{}{0pt}%
\pgfsys@defobject{currentmarker}{\pgfqpoint{0.000000in}{-0.048611in}}{\pgfqpoint{0.000000in}{0.000000in}}{%
\pgfpathmoveto{\pgfqpoint{0.000000in}{0.000000in}}%
\pgfpathlineto{\pgfqpoint{0.000000in}{-0.048611in}}%
\pgfusepath{stroke,fill}%
}%
\begin{pgfscope}%
\pgfsys@transformshift{2.204861in}{0.526079in}%
\pgfsys@useobject{currentmarker}{}%
\end{pgfscope}%
\end{pgfscope}%
\begin{pgfscope}%
\pgftext[x=2.204861in,y=0.428857in,,top]{\rmfamily\fontsize{10.000000}{12.000000}\selectfont \(\displaystyle 6\)}%
\end{pgfscope}%
\begin{pgfscope}%
\pgfsetbuttcap%
\pgfsetroundjoin%
\definecolor{currentfill}{rgb}{0.000000,0.000000,0.000000}%
\pgfsetfillcolor{currentfill}%
\pgfsetlinewidth{0.803000pt}%
\definecolor{currentstroke}{rgb}{0.000000,0.000000,0.000000}%
\pgfsetstrokecolor{currentstroke}%
\pgfsetdash{}{0pt}%
\pgfsys@defobject{currentmarker}{\pgfqpoint{0.000000in}{-0.048611in}}{\pgfqpoint{0.000000in}{0.000000in}}{%
\pgfpathmoveto{\pgfqpoint{0.000000in}{0.000000in}}%
\pgfpathlineto{\pgfqpoint{0.000000in}{-0.048611in}}%
\pgfusepath{stroke,fill}%
}%
\begin{pgfscope}%
\pgfsys@transformshift{2.786111in}{0.526079in}%
\pgfsys@useobject{currentmarker}{}%
\end{pgfscope}%
\end{pgfscope}%
\begin{pgfscope}%
\pgftext[x=2.786111in,y=0.428857in,,top]{\rmfamily\fontsize{10.000000}{12.000000}\selectfont \(\displaystyle 8\)}%
\end{pgfscope}%
\begin{pgfscope}%
\pgftext[x=1.623611in,y=0.238889in,,top]{\rmfamily\fontsize{10.000000}{12.000000}\selectfont \(\displaystyle kc / |\Omega_\mathrm{ce}|\)}%
\end{pgfscope}%
\begin{pgfscope}%
\pgfsetbuttcap%
\pgfsetroundjoin%
\definecolor{currentfill}{rgb}{0.000000,0.000000,0.000000}%
\pgfsetfillcolor{currentfill}%
\pgfsetlinewidth{0.803000pt}%
\definecolor{currentstroke}{rgb}{0.000000,0.000000,0.000000}%
\pgfsetstrokecolor{currentstroke}%
\pgfsetdash{}{0pt}%
\pgfsys@defobject{currentmarker}{\pgfqpoint{-0.048611in}{0.000000in}}{\pgfqpoint{0.000000in}{0.000000in}}{%
\pgfpathmoveto{\pgfqpoint{0.000000in}{0.000000in}}%
\pgfpathlineto{\pgfqpoint{-0.048611in}{0.000000in}}%
\pgfusepath{stroke,fill}%
}%
\begin{pgfscope}%
\pgfsys@transformshift{0.461111in}{0.526079in}%
\pgfsys@useobject{currentmarker}{}%
\end{pgfscope}%
\end{pgfscope}%
\begin{pgfscope}%
\pgftext[x=0.294444in,y=0.473318in,left,base]{\rmfamily\fontsize{10.000000}{12.000000}\selectfont \(\displaystyle 0\)}%
\end{pgfscope}%
\begin{pgfscope}%
\pgfsetbuttcap%
\pgfsetroundjoin%
\definecolor{currentfill}{rgb}{0.000000,0.000000,0.000000}%
\pgfsetfillcolor{currentfill}%
\pgfsetlinewidth{0.803000pt}%
\definecolor{currentstroke}{rgb}{0.000000,0.000000,0.000000}%
\pgfsetstrokecolor{currentstroke}%
\pgfsetdash{}{0pt}%
\pgfsys@defobject{currentmarker}{\pgfqpoint{-0.048611in}{0.000000in}}{\pgfqpoint{0.000000in}{0.000000in}}{%
\pgfpathmoveto{\pgfqpoint{0.000000in}{0.000000in}}%
\pgfpathlineto{\pgfqpoint{-0.048611in}{0.000000in}}%
\pgfusepath{stroke,fill}%
}%
\begin{pgfscope}%
\pgfsys@transformshift{0.461111in}{0.903579in}%
\pgfsys@useobject{currentmarker}{}%
\end{pgfscope}%
\end{pgfscope}%
\begin{pgfscope}%
\pgftext[x=0.294444in,y=0.850818in,left,base]{\rmfamily\fontsize{10.000000}{12.000000}\selectfont \(\displaystyle 1\)}%
\end{pgfscope}%
\begin{pgfscope}%
\pgfsetbuttcap%
\pgfsetroundjoin%
\definecolor{currentfill}{rgb}{0.000000,0.000000,0.000000}%
\pgfsetfillcolor{currentfill}%
\pgfsetlinewidth{0.803000pt}%
\definecolor{currentstroke}{rgb}{0.000000,0.000000,0.000000}%
\pgfsetstrokecolor{currentstroke}%
\pgfsetdash{}{0pt}%
\pgfsys@defobject{currentmarker}{\pgfqpoint{-0.048611in}{0.000000in}}{\pgfqpoint{0.000000in}{0.000000in}}{%
\pgfpathmoveto{\pgfqpoint{0.000000in}{0.000000in}}%
\pgfpathlineto{\pgfqpoint{-0.048611in}{0.000000in}}%
\pgfusepath{stroke,fill}%
}%
\begin{pgfscope}%
\pgfsys@transformshift{0.461111in}{1.281079in}%
\pgfsys@useobject{currentmarker}{}%
\end{pgfscope}%
\end{pgfscope}%
\begin{pgfscope}%
\pgftext[x=0.294444in,y=1.228318in,left,base]{\rmfamily\fontsize{10.000000}{12.000000}\selectfont \(\displaystyle 2\)}%
\end{pgfscope}%
\begin{pgfscope}%
\pgfsetbuttcap%
\pgfsetroundjoin%
\definecolor{currentfill}{rgb}{0.000000,0.000000,0.000000}%
\pgfsetfillcolor{currentfill}%
\pgfsetlinewidth{0.803000pt}%
\definecolor{currentstroke}{rgb}{0.000000,0.000000,0.000000}%
\pgfsetstrokecolor{currentstroke}%
\pgfsetdash{}{0pt}%
\pgfsys@defobject{currentmarker}{\pgfqpoint{-0.048611in}{0.000000in}}{\pgfqpoint{0.000000in}{0.000000in}}{%
\pgfpathmoveto{\pgfqpoint{0.000000in}{0.000000in}}%
\pgfpathlineto{\pgfqpoint{-0.048611in}{0.000000in}}%
\pgfusepath{stroke,fill}%
}%
\begin{pgfscope}%
\pgfsys@transformshift{0.461111in}{1.658579in}%
\pgfsys@useobject{currentmarker}{}%
\end{pgfscope}%
\end{pgfscope}%
\begin{pgfscope}%
\pgftext[x=0.294444in,y=1.605818in,left,base]{\rmfamily\fontsize{10.000000}{12.000000}\selectfont \(\displaystyle 3\)}%
\end{pgfscope}%
\begin{pgfscope}%
\pgfsetbuttcap%
\pgfsetroundjoin%
\definecolor{currentfill}{rgb}{0.000000,0.000000,0.000000}%
\pgfsetfillcolor{currentfill}%
\pgfsetlinewidth{0.803000pt}%
\definecolor{currentstroke}{rgb}{0.000000,0.000000,0.000000}%
\pgfsetstrokecolor{currentstroke}%
\pgfsetdash{}{0pt}%
\pgfsys@defobject{currentmarker}{\pgfqpoint{-0.048611in}{0.000000in}}{\pgfqpoint{0.000000in}{0.000000in}}{%
\pgfpathmoveto{\pgfqpoint{0.000000in}{0.000000in}}%
\pgfpathlineto{\pgfqpoint{-0.048611in}{0.000000in}}%
\pgfusepath{stroke,fill}%
}%
\begin{pgfscope}%
\pgfsys@transformshift{0.461111in}{2.036079in}%
\pgfsys@useobject{currentmarker}{}%
\end{pgfscope}%
\end{pgfscope}%
\begin{pgfscope}%
\pgftext[x=0.294444in,y=1.983318in,left,base]{\rmfamily\fontsize{10.000000}{12.000000}\selectfont \(\displaystyle 4\)}%
\end{pgfscope}%
\begin{pgfscope}%
\pgftext[x=0.238889in,y=1.281079in,,bottom,rotate=90.000000]{\rmfamily\fontsize{10.000000}{12.000000}\selectfont \(\displaystyle \omega_\mathrm{r} / |\Omega_\mathrm{ce}|\)}%
\end{pgfscope}%
\begin{pgfscope}%
\pgfpathrectangle{\pgfqpoint{0.461111in}{0.526079in}}{\pgfqpoint{2.325000in}{1.510000in}} %
\pgfusepath{clip}%
\pgfsetrectcap%
\pgfsetroundjoin%
\pgfsetlinewidth{1.003750pt}%
\definecolor{currentstroke}{rgb}{1.000000,0.549020,0.000000}%
\pgfsetstrokecolor{currentstroke}%
\pgfsetdash{}{0pt}%
\pgfpathmoveto{\pgfqpoint{0.577361in}{1.503843in}}%
\pgfpathlineto{\pgfqpoint{0.599672in}{1.507607in}}%
\pgfpathlineto{\pgfqpoint{0.621983in}{1.512030in}}%
\pgfpathlineto{\pgfqpoint{0.644293in}{1.517111in}}%
\pgfpathlineto{\pgfqpoint{0.666604in}{1.522848in}}%
\pgfpathlineto{\pgfqpoint{0.688914in}{1.529241in}}%
\pgfpathlineto{\pgfqpoint{0.711225in}{1.536288in}}%
\pgfpathlineto{\pgfqpoint{0.733536in}{1.543985in}}%
\pgfpathlineto{\pgfqpoint{0.755846in}{1.552329in}}%
\pgfpathlineto{\pgfqpoint{0.778157in}{1.561316in}}%
\pgfpathlineto{\pgfqpoint{0.800467in}{1.570940in}}%
\pgfpathlineto{\pgfqpoint{0.822778in}{1.581196in}}%
\pgfpathlineto{\pgfqpoint{0.845089in}{1.592077in}}%
\pgfpathlineto{\pgfqpoint{0.867399in}{1.603574in}}%
\pgfpathlineto{\pgfqpoint{0.889710in}{1.615677in}}%
\pgfpathlineto{\pgfqpoint{0.912020in}{1.628378in}}%
\pgfpathlineto{\pgfqpoint{0.934331in}{1.641665in}}%
\pgfpathlineto{\pgfqpoint{0.956642in}{1.655525in}}%
\pgfpathlineto{\pgfqpoint{0.978952in}{1.669946in}}%
\pgfpathlineto{\pgfqpoint{1.001263in}{1.684914in}}%
\pgfpathlineto{\pgfqpoint{1.023574in}{1.700415in}}%
\pgfpathlineto{\pgfqpoint{1.045884in}{1.716435in}}%
\pgfpathlineto{\pgfqpoint{1.068195in}{1.732957in}}%
\pgfpathlineto{\pgfqpoint{1.090505in}{1.749965in}}%
\pgfpathlineto{\pgfqpoint{1.112816in}{1.767445in}}%
\pgfpathlineto{\pgfqpoint{1.135127in}{1.785379in}}%
\pgfpathlineto{\pgfqpoint{1.157437in}{1.803752in}}%
\pgfpathlineto{\pgfqpoint{1.179748in}{1.822547in}}%
\pgfpathlineto{\pgfqpoint{1.202058in}{1.841748in}}%
\pgfpathlineto{\pgfqpoint{1.224369in}{1.861340in}}%
\pgfpathlineto{\pgfqpoint{1.246680in}{1.881305in}}%
\pgfpathlineto{\pgfqpoint{1.268990in}{1.901629in}}%
\pgfpathlineto{\pgfqpoint{1.291301in}{1.922297in}}%
\pgfpathlineto{\pgfqpoint{1.313611in}{1.943294in}}%
\pgfpathlineto{\pgfqpoint{1.335922in}{1.964606in}}%
\pgfpathlineto{\pgfqpoint{1.358233in}{1.986219in}}%
\pgfpathlineto{\pgfqpoint{1.380543in}{2.008119in}}%
\pgfpathlineto{\pgfqpoint{1.402854in}{2.030294in}}%
\pgfpathlineto{\pgfqpoint{1.422417in}{2.049968in}}%
\pgfusepath{stroke}%
\end{pgfscope}%
\begin{pgfscope}%
\pgfpathrectangle{\pgfqpoint{0.461111in}{0.526079in}}{\pgfqpoint{2.325000in}{1.510000in}} %
\pgfusepath{clip}%
\pgfsetrectcap%
\pgfsetroundjoin%
\pgfsetlinewidth{1.003750pt}%
\definecolor{currentstroke}{rgb}{0.501961,0.000000,0.501961}%
\pgfsetstrokecolor{currentstroke}%
\pgfsetdash{}{0pt}%
\pgfpathmoveto{\pgfqpoint{0.577361in}{1.140663in}}%
\pgfpathlineto{\pgfqpoint{0.599672in}{1.150067in}}%
\pgfpathlineto{\pgfqpoint{0.621983in}{1.160858in}}%
\pgfpathlineto{\pgfqpoint{0.644293in}{1.172938in}}%
\pgfpathlineto{\pgfqpoint{0.666604in}{1.186207in}}%
\pgfpathlineto{\pgfqpoint{0.688914in}{1.200570in}}%
\pgfpathlineto{\pgfqpoint{0.711225in}{1.215935in}}%
\pgfpathlineto{\pgfqpoint{0.733536in}{1.232216in}}%
\pgfpathlineto{\pgfqpoint{0.755846in}{1.249333in}}%
\pgfpathlineto{\pgfqpoint{0.778157in}{1.267214in}}%
\pgfpathlineto{\pgfqpoint{0.800467in}{1.285791in}}%
\pgfpathlineto{\pgfqpoint{0.822778in}{1.305005in}}%
\pgfpathlineto{\pgfqpoint{0.845089in}{1.324801in}}%
\pgfpathlineto{\pgfqpoint{0.867399in}{1.345129in}}%
\pgfpathlineto{\pgfqpoint{0.889710in}{1.365946in}}%
\pgfpathlineto{\pgfqpoint{0.912020in}{1.387211in}}%
\pgfpathlineto{\pgfqpoint{0.934331in}{1.408889in}}%
\pgfpathlineto{\pgfqpoint{0.956642in}{1.430947in}}%
\pgfpathlineto{\pgfqpoint{0.978952in}{1.453356in}}%
\pgfpathlineto{\pgfqpoint{1.001263in}{1.476091in}}%
\pgfpathlineto{\pgfqpoint{1.023574in}{1.499126in}}%
\pgfpathlineto{\pgfqpoint{1.045884in}{1.522441in}}%
\pgfpathlineto{\pgfqpoint{1.068195in}{1.546016in}}%
\pgfpathlineto{\pgfqpoint{1.090505in}{1.569833in}}%
\pgfpathlineto{\pgfqpoint{1.112816in}{1.593876in}}%
\pgfpathlineto{\pgfqpoint{1.135127in}{1.618131in}}%
\pgfpathlineto{\pgfqpoint{1.157437in}{1.642583in}}%
\pgfpathlineto{\pgfqpoint{1.179748in}{1.667221in}}%
\pgfpathlineto{\pgfqpoint{1.202058in}{1.692033in}}%
\pgfpathlineto{\pgfqpoint{1.224369in}{1.717009in}}%
\pgfpathlineto{\pgfqpoint{1.246680in}{1.742139in}}%
\pgfpathlineto{\pgfqpoint{1.268990in}{1.767415in}}%
\pgfpathlineto{\pgfqpoint{1.291301in}{1.792827in}}%
\pgfpathlineto{\pgfqpoint{1.313611in}{1.818370in}}%
\pgfpathlineto{\pgfqpoint{1.335922in}{1.844035in}}%
\pgfpathlineto{\pgfqpoint{1.358233in}{1.869816in}}%
\pgfpathlineto{\pgfqpoint{1.380543in}{1.895707in}}%
\pgfpathlineto{\pgfqpoint{1.402854in}{1.921702in}}%
\pgfpathlineto{\pgfqpoint{1.425164in}{1.947797in}}%
\pgfpathlineto{\pgfqpoint{1.447475in}{1.973986in}}%
\pgfpathlineto{\pgfqpoint{1.469786in}{2.000265in}}%
\pgfpathlineto{\pgfqpoint{1.492096in}{2.026629in}}%
\pgfpathlineto{\pgfqpoint{1.511786in}{2.049968in}}%
\pgfusepath{stroke}%
\end{pgfscope}%
\begin{pgfscope}%
\pgfpathrectangle{\pgfqpoint{0.461111in}{0.526079in}}{\pgfqpoint{2.325000in}{1.510000in}} %
\pgfusepath{clip}%
\pgfsetrectcap%
\pgfsetroundjoin%
\pgfsetlinewidth{1.003750pt}%
\definecolor{currentstroke}{rgb}{0.627451,0.321569,0.176471}%
\pgfsetstrokecolor{currentstroke}%
\pgfsetdash{}{0pt}%
\pgfpathmoveto{\pgfqpoint{0.577361in}{0.540500in}}%
\pgfpathlineto{\pgfqpoint{0.599672in}{0.546184in}}%
\pgfpathlineto{\pgfqpoint{0.621983in}{0.552605in}}%
\pgfpathlineto{\pgfqpoint{0.644293in}{0.559665in}}%
\pgfpathlineto{\pgfqpoint{0.666604in}{0.567270in}}%
\pgfpathlineto{\pgfqpoint{0.688914in}{0.575325in}}%
\pgfpathlineto{\pgfqpoint{0.711225in}{0.583746in}}%
\pgfpathlineto{\pgfqpoint{0.733536in}{0.592455in}}%
\pgfpathlineto{\pgfqpoint{0.755846in}{0.601388in}}%
\pgfpathlineto{\pgfqpoint{0.778157in}{0.610479in}}%
\pgfpathlineto{\pgfqpoint{0.800467in}{0.619657in}}%
\pgfpathlineto{\pgfqpoint{0.822778in}{0.628839in}}%
\pgfpathlineto{\pgfqpoint{0.845089in}{0.637932in}}%
\pgfpathlineto{\pgfqpoint{0.867399in}{0.646860in}}%
\pgfpathlineto{\pgfqpoint{0.889710in}{0.655568in}}%
\pgfpathlineto{\pgfqpoint{0.912020in}{0.664030in}}%
\pgfpathlineto{\pgfqpoint{0.934331in}{0.672239in}}%
\pgfpathlineto{\pgfqpoint{0.956642in}{0.680201in}}%
\pgfpathlineto{\pgfqpoint{0.978952in}{0.687923in}}%
\pgfpathlineto{\pgfqpoint{1.001263in}{0.695417in}}%
\pgfpathlineto{\pgfqpoint{1.023574in}{0.702690in}}%
\pgfpathlineto{\pgfqpoint{1.045884in}{0.709748in}}%
\pgfpathlineto{\pgfqpoint{1.068195in}{0.716593in}}%
\pgfpathlineto{\pgfqpoint{1.090505in}{0.723228in}}%
\pgfpathlineto{\pgfqpoint{1.112816in}{0.729652in}}%
\pgfpathlineto{\pgfqpoint{1.135127in}{0.735865in}}%
\pgfpathlineto{\pgfqpoint{1.157437in}{0.741868in}}%
\pgfpathlineto{\pgfqpoint{1.179748in}{0.747660in}}%
\pgfpathlineto{\pgfqpoint{1.202058in}{0.753243in}}%
\pgfpathlineto{\pgfqpoint{1.224369in}{0.758618in}}%
\pgfpathlineto{\pgfqpoint{1.246680in}{0.763788in}}%
\pgfpathlineto{\pgfqpoint{1.268990in}{0.768756in}}%
\pgfpathlineto{\pgfqpoint{1.291301in}{0.773526in}}%
\pgfpathlineto{\pgfqpoint{1.313611in}{0.778104in}}%
\pgfpathlineto{\pgfqpoint{1.335922in}{0.782494in}}%
\pgfpathlineto{\pgfqpoint{1.358233in}{0.786702in}}%
\pgfpathlineto{\pgfqpoint{1.380543in}{0.790734in}}%
\pgfpathlineto{\pgfqpoint{1.402854in}{0.794596in}}%
\pgfpathlineto{\pgfqpoint{1.425164in}{0.798295in}}%
\pgfpathlineto{\pgfqpoint{1.447475in}{0.801837in}}%
\pgfpathlineto{\pgfqpoint{1.469786in}{0.805229in}}%
\pgfpathlineto{\pgfqpoint{1.492096in}{0.808476in}}%
\pgfpathlineto{\pgfqpoint{1.514407in}{0.811585in}}%
\pgfpathlineto{\pgfqpoint{1.536717in}{0.814562in}}%
\pgfpathlineto{\pgfqpoint{1.559028in}{0.817414in}}%
\pgfpathlineto{\pgfqpoint{1.581339in}{0.820145in}}%
\pgfpathlineto{\pgfqpoint{1.603649in}{0.822761in}}%
\pgfpathlineto{\pgfqpoint{1.625960in}{0.825269in}}%
\pgfpathlineto{\pgfqpoint{1.648270in}{0.827672in}}%
\pgfpathlineto{\pgfqpoint{1.670581in}{0.829976in}}%
\pgfpathlineto{\pgfqpoint{1.692892in}{0.832185in}}%
\pgfpathlineto{\pgfqpoint{1.715202in}{0.834304in}}%
\pgfpathlineto{\pgfqpoint{1.737513in}{0.836338in}}%
\pgfpathlineto{\pgfqpoint{1.759824in}{0.838289in}}%
\pgfpathlineto{\pgfqpoint{1.782134in}{0.840163in}}%
\pgfpathlineto{\pgfqpoint{1.804445in}{0.841963in}}%
\pgfpathlineto{\pgfqpoint{1.826755in}{0.843692in}}%
\pgfpathlineto{\pgfqpoint{1.849066in}{0.845354in}}%
\pgfpathlineto{\pgfqpoint{1.871377in}{0.846952in}}%
\pgfpathlineto{\pgfqpoint{1.893687in}{0.848489in}}%
\pgfpathlineto{\pgfqpoint{1.915998in}{0.849967in}}%
\pgfpathlineto{\pgfqpoint{1.938308in}{0.851390in}}%
\pgfpathlineto{\pgfqpoint{1.960619in}{0.852760in}}%
\pgfpathlineto{\pgfqpoint{1.982930in}{0.854079in}}%
\pgfpathlineto{\pgfqpoint{2.005240in}{0.855350in}}%
\pgfpathlineto{\pgfqpoint{2.027551in}{0.856575in}}%
\pgfpathlineto{\pgfqpoint{2.049861in}{0.857756in}}%
\pgfpathlineto{\pgfqpoint{2.072172in}{0.858895in}}%
\pgfpathlineto{\pgfqpoint{2.094483in}{0.859994in}}%
\pgfpathlineto{\pgfqpoint{2.116793in}{0.861054in}}%
\pgfpathlineto{\pgfqpoint{2.139104in}{0.862077in}}%
\pgfpathlineto{\pgfqpoint{2.161414in}{0.863066in}}%
\pgfpathlineto{\pgfqpoint{2.183725in}{0.864021in}}%
\pgfpathlineto{\pgfqpoint{2.206036in}{0.864943in}}%
\pgfpathlineto{\pgfqpoint{2.228346in}{0.865835in}}%
\pgfpathlineto{\pgfqpoint{2.250657in}{0.866697in}}%
\pgfpathlineto{\pgfqpoint{2.272967in}{0.867532in}}%
\pgfpathlineto{\pgfqpoint{2.295278in}{0.868339in}}%
\pgfpathlineto{\pgfqpoint{2.317589in}{0.869120in}}%
\pgfpathlineto{\pgfqpoint{2.339899in}{0.869876in}}%
\pgfpathlineto{\pgfqpoint{2.362210in}{0.870608in}}%
\pgfpathlineto{\pgfqpoint{2.384520in}{0.871317in}}%
\pgfpathlineto{\pgfqpoint{2.406831in}{0.872004in}}%
\pgfpathlineto{\pgfqpoint{2.429142in}{0.872670in}}%
\pgfpathlineto{\pgfqpoint{2.451452in}{0.873316in}}%
\pgfpathlineto{\pgfqpoint{2.473763in}{0.873942in}}%
\pgfpathlineto{\pgfqpoint{2.496074in}{0.874550in}}%
\pgfpathlineto{\pgfqpoint{2.518384in}{0.875139in}}%
\pgfpathlineto{\pgfqpoint{2.540695in}{0.875711in}}%
\pgfpathlineto{\pgfqpoint{2.563005in}{0.876267in}}%
\pgfpathlineto{\pgfqpoint{2.585316in}{0.876806in}}%
\pgfpathlineto{\pgfqpoint{2.607627in}{0.877330in}}%
\pgfpathlineto{\pgfqpoint{2.629937in}{0.877839in}}%
\pgfpathlineto{\pgfqpoint{2.652248in}{0.878333in}}%
\pgfpathlineto{\pgfqpoint{2.674558in}{0.878814in}}%
\pgfpathlineto{\pgfqpoint{2.696869in}{0.879282in}}%
\pgfpathlineto{\pgfqpoint{2.719180in}{0.879736in}}%
\pgfpathlineto{\pgfqpoint{2.741490in}{0.880178in}}%
\pgfpathlineto{\pgfqpoint{2.763801in}{0.880609in}}%
\pgfpathlineto{\pgfqpoint{2.786111in}{0.881027in}}%
\pgfusepath{stroke}%
\end{pgfscope}%
\begin{pgfscope}%
\pgfpathrectangle{\pgfqpoint{0.461111in}{0.526079in}}{\pgfqpoint{2.325000in}{1.510000in}} %
\pgfusepath{clip}%
\pgfsetbuttcap%
\pgfsetroundjoin%
\pgfsetlinewidth{0.501875pt}%
\definecolor{currentstroke}{rgb}{0.000000,0.000000,0.000000}%
\pgfsetstrokecolor{currentstroke}%
\pgfsetdash{{1.850000pt}{0.800000pt}}{0.000000pt}%
\pgfpathmoveto{\pgfqpoint{0.490174in}{0.903579in}}%
\pgfpathlineto{\pgfqpoint{0.563948in}{0.903579in}}%
\pgfpathlineto{\pgfqpoint{0.637722in}{0.903579in}}%
\pgfpathlineto{\pgfqpoint{0.711496in}{0.903579in}}%
\pgfpathlineto{\pgfqpoint{0.785270in}{0.903579in}}%
\pgfpathlineto{\pgfqpoint{0.859044in}{0.903579in}}%
\pgfpathlineto{\pgfqpoint{0.932818in}{0.903579in}}%
\pgfpathlineto{\pgfqpoint{1.006592in}{0.903579in}}%
\pgfpathlineto{\pgfqpoint{1.080366in}{0.903579in}}%
\pgfpathlineto{\pgfqpoint{1.154140in}{0.903579in}}%
\pgfpathlineto{\pgfqpoint{1.227914in}{0.903579in}}%
\pgfpathlineto{\pgfqpoint{1.301688in}{0.903579in}}%
\pgfpathlineto{\pgfqpoint{1.375462in}{0.903579in}}%
\pgfpathlineto{\pgfqpoint{1.449236in}{0.903579in}}%
\pgfpathlineto{\pgfqpoint{1.523010in}{0.903579in}}%
\pgfpathlineto{\pgfqpoint{1.596784in}{0.903579in}}%
\pgfpathlineto{\pgfqpoint{1.670559in}{0.903579in}}%
\pgfpathlineto{\pgfqpoint{1.744333in}{0.903579in}}%
\pgfpathlineto{\pgfqpoint{1.818107in}{0.903579in}}%
\pgfpathlineto{\pgfqpoint{1.891881in}{0.903579in}}%
\pgfpathlineto{\pgfqpoint{1.965655in}{0.903579in}}%
\pgfpathlineto{\pgfqpoint{2.039429in}{0.903579in}}%
\pgfpathlineto{\pgfqpoint{2.113203in}{0.903579in}}%
\pgfpathlineto{\pgfqpoint{2.186977in}{0.903579in}}%
\pgfpathlineto{\pgfqpoint{2.260751in}{0.903579in}}%
\pgfpathlineto{\pgfqpoint{2.334525in}{0.903579in}}%
\pgfpathlineto{\pgfqpoint{2.408299in}{0.903579in}}%
\pgfpathlineto{\pgfqpoint{2.482073in}{0.903579in}}%
\pgfpathlineto{\pgfqpoint{2.555847in}{0.903579in}}%
\pgfpathlineto{\pgfqpoint{2.629621in}{0.903579in}}%
\pgfpathlineto{\pgfqpoint{2.703395in}{0.903579in}}%
\pgfpathlineto{\pgfqpoint{2.777169in}{0.903579in}}%
\pgfpathlineto{\pgfqpoint{2.800000in}{0.903579in}}%
\pgfusepath{stroke}%
\end{pgfscope}%
\begin{pgfscope}%
\pgfsetrectcap%
\pgfsetmiterjoin%
\pgfsetlinewidth{0.803000pt}%
\definecolor{currentstroke}{rgb}{0.000000,0.000000,0.000000}%
\pgfsetstrokecolor{currentstroke}%
\pgfsetdash{}{0pt}%
\pgfpathmoveto{\pgfqpoint{0.461111in}{0.526079in}}%
\pgfpathlineto{\pgfqpoint{0.461111in}{2.036079in}}%
\pgfusepath{stroke}%
\end{pgfscope}%
\begin{pgfscope}%
\pgfsetrectcap%
\pgfsetmiterjoin%
\pgfsetlinewidth{0.803000pt}%
\definecolor{currentstroke}{rgb}{0.000000,0.000000,0.000000}%
\pgfsetstrokecolor{currentstroke}%
\pgfsetdash{}{0pt}%
\pgfpathmoveto{\pgfqpoint{2.786111in}{0.526079in}}%
\pgfpathlineto{\pgfqpoint{2.786111in}{2.036079in}}%
\pgfusepath{stroke}%
\end{pgfscope}%
\begin{pgfscope}%
\pgfsetrectcap%
\pgfsetmiterjoin%
\pgfsetlinewidth{0.803000pt}%
\definecolor{currentstroke}{rgb}{0.000000,0.000000,0.000000}%
\pgfsetstrokecolor{currentstroke}%
\pgfsetdash{}{0pt}%
\pgfpathmoveto{\pgfqpoint{0.461111in}{0.526079in}}%
\pgfpathlineto{\pgfqpoint{2.786111in}{0.526079in}}%
\pgfusepath{stroke}%
\end{pgfscope}%
\begin{pgfscope}%
\pgfsetrectcap%
\pgfsetmiterjoin%
\pgfsetlinewidth{0.803000pt}%
\definecolor{currentstroke}{rgb}{0.000000,0.000000,0.000000}%
\pgfsetstrokecolor{currentstroke}%
\pgfsetdash{}{0pt}%
\pgfpathmoveto{\pgfqpoint{0.461111in}{2.036079in}}%
\pgfpathlineto{\pgfqpoint{2.786111in}{2.036079in}}%
\pgfusepath{stroke}%
\end{pgfscope}%
\begin{pgfscope}%
\pgfsetbuttcap%
\pgfsetmiterjoin%
\definecolor{currentfill}{rgb}{1.000000,1.000000,1.000000}%
\pgfsetfillcolor{currentfill}%
\pgfsetfillopacity{0.800000}%
\pgfsetlinewidth{1.003750pt}%
\definecolor{currentstroke}{rgb}{0.800000,0.800000,0.800000}%
\pgfsetstrokecolor{currentstroke}%
\pgfsetstrokeopacity{0.800000}%
\pgfsetdash{}{0pt}%
\pgfpathmoveto{\pgfqpoint{1.642299in}{1.103707in}}%
\pgfpathlineto{\pgfqpoint{2.688889in}{1.103707in}}%
\pgfpathquadraticcurveto{\pgfqpoint{2.716667in}{1.103707in}}{\pgfqpoint{2.716667in}{1.131485in}}%
\pgfpathlineto{\pgfqpoint{2.716667in}{1.938857in}}%
\pgfpathquadraticcurveto{\pgfqpoint{2.716667in}{1.966635in}}{\pgfqpoint{2.688889in}{1.966635in}}%
\pgfpathlineto{\pgfqpoint{1.642299in}{1.966635in}}%
\pgfpathquadraticcurveto{\pgfqpoint{1.614521in}{1.966635in}}{\pgfqpoint{1.614521in}{1.938857in}}%
\pgfpathlineto{\pgfqpoint{1.614521in}{1.131485in}}%
\pgfpathquadraticcurveto{\pgfqpoint{1.614521in}{1.103707in}}{\pgfqpoint{1.642299in}{1.103707in}}%
\pgfpathclose%
\pgfusepath{stroke,fill}%
\end{pgfscope}%
\begin{pgfscope}%
\pgfsetrectcap%
\pgfsetroundjoin%
\pgfsetlinewidth{1.003750pt}%
\definecolor{currentstroke}{rgb}{1.000000,0.549020,0.000000}%
\pgfsetstrokecolor{currentstroke}%
\pgfsetdash{}{0pt}%
\pgfpathmoveto{\pgfqpoint{1.670077in}{1.854168in}}%
\pgfpathlineto{\pgfqpoint{1.947855in}{1.854168in}}%
\pgfusepath{stroke}%
\end{pgfscope}%
\begin{pgfscope}%
\pgftext[x=2.058966in,y=1.805556in,left,base]{\rmfamily\fontsize{10.000000}{12.000000}\selectfont \(\displaystyle \mathrm{R}\) - wave}%
\end{pgfscope}%
\begin{pgfscope}%
\pgfsetrectcap%
\pgfsetroundjoin%
\pgfsetlinewidth{1.003750pt}%
\definecolor{currentstroke}{rgb}{0.501961,0.000000,0.501961}%
\pgfsetstrokecolor{currentstroke}%
\pgfsetdash{}{0pt}%
\pgfpathmoveto{\pgfqpoint{1.670077in}{1.650310in}}%
\pgfpathlineto{\pgfqpoint{1.947855in}{1.650310in}}%
\pgfusepath{stroke}%
\end{pgfscope}%
\begin{pgfscope}%
\pgftext[x=2.058966in,y=1.601699in,left,base]{\rmfamily\fontsize{10.000000}{12.000000}\selectfont L - wave}%
\end{pgfscope}%
\begin{pgfscope}%
\pgfsetrectcap%
\pgfsetroundjoin%
\pgfsetlinewidth{1.003750pt}%
\definecolor{currentstroke}{rgb}{0.627451,0.321569,0.176471}%
\pgfsetstrokecolor{currentstroke}%
\pgfsetdash{}{0pt}%
\pgfpathmoveto{\pgfqpoint{1.670077in}{1.446453in}}%
\pgfpathlineto{\pgfqpoint{1.947855in}{1.446453in}}%
\pgfusepath{stroke}%
\end{pgfscope}%
\begin{pgfscope}%
\pgftext[x=2.058966in,y=1.397842in,left,base]{\rmfamily\fontsize{10.000000}{12.000000}\selectfont \(\displaystyle \mathrm{R}\) - wave}%
\end{pgfscope}%
\begin{pgfscope}%
\pgfsetbuttcap%
\pgfsetroundjoin%
\pgfsetlinewidth{0.501875pt}%
\definecolor{currentstroke}{rgb}{0.000000,0.000000,0.000000}%
\pgfsetstrokecolor{currentstroke}%
\pgfsetdash{{1.850000pt}{0.800000pt}}{0.000000pt}%
\pgfpathmoveto{\pgfqpoint{1.670077in}{1.242596in}}%
\pgfpathlineto{\pgfqpoint{1.947855in}{1.242596in}}%
\pgfusepath{stroke}%
\end{pgfscope}%
\begin{pgfscope}%
\pgftext[x=2.058966in,y=1.193985in,left,base]{\rmfamily\fontsize{10.000000}{12.000000}\selectfont \(\displaystyle |\Omega_\mathrm{ce}|\)}%
\end{pgfscope}%
\end{pgfpicture}%
\makeatother%
\endgroup%
}
\subfigure[]{%% Creator: Matplotlib, PGF backend
%%
%% To include the figure in your LaTeX document, write
%%   \input{<filename>.pgf}
%%
%% Make sure the required packages are loaded in your preamble
%%   \usepackage{pgf}
%%
%% Figures using additional raster images can only be included by \input if
%% they are in the same directory as the main LaTeX file. For loading figures
%% from other directories you can use the `import` package
%%   \usepackage{import}
%% and then include the figures with
%%   \import{<path to file>}{<filename>.pgf}
%%
%% Matplotlib used the following preamble
%%   \usepackage{fontspec}
%%   \setmainfont{DejaVu Serif}
%%   \setsansfont{DejaVu Sans}
%%   \setmonofont{DejaVu Sans Mono}
%%
\begingroup%
\makeatletter%
\begin{pgfpicture}%
\pgfpathrectangle{\pgfpointorigin}{\pgfqpoint{3.491637in}{2.184691in}}%
\pgfusepath{use as bounding box, clip}%
\begin{pgfscope}%
\pgfsetbuttcap%
\pgfsetmiterjoin%
\definecolor{currentfill}{rgb}{1.000000,1.000000,1.000000}%
\pgfsetfillcolor{currentfill}%
\pgfsetlinewidth{0.000000pt}%
\definecolor{currentstroke}{rgb}{1.000000,1.000000,1.000000}%
\pgfsetstrokecolor{currentstroke}%
\pgfsetdash{}{0pt}%
\pgfpathmoveto{\pgfqpoint{0.000000in}{0.000000in}}%
\pgfpathlineto{\pgfqpoint{3.491637in}{0.000000in}}%
\pgfpathlineto{\pgfqpoint{3.491637in}{2.184691in}}%
\pgfpathlineto{\pgfqpoint{0.000000in}{2.184691in}}%
\pgfpathclose%
\pgfusepath{fill}%
\end{pgfscope}%
\begin{pgfscope}%
\pgfsetbuttcap%
\pgfsetmiterjoin%
\definecolor{currentfill}{rgb}{1.000000,1.000000,1.000000}%
\pgfsetfillcolor{currentfill}%
\pgfsetlinewidth{0.000000pt}%
\definecolor{currentstroke}{rgb}{0.000000,0.000000,0.000000}%
\pgfsetstrokecolor{currentstroke}%
\pgfsetstrokeopacity{0.000000}%
\pgfsetdash{}{0pt}%
\pgfpathmoveto{\pgfqpoint{0.708026in}{0.526079in}}%
\pgfpathlineto{\pgfqpoint{3.343026in}{0.526079in}}%
\pgfpathlineto{\pgfqpoint{3.343026in}{2.036079in}}%
\pgfpathlineto{\pgfqpoint{0.708026in}{2.036079in}}%
\pgfpathclose%
\pgfusepath{fill}%
\end{pgfscope}%
\begin{pgfscope}%
\pgfsetbuttcap%
\pgfsetroundjoin%
\definecolor{currentfill}{rgb}{0.000000,0.000000,0.000000}%
\pgfsetfillcolor{currentfill}%
\pgfsetlinewidth{0.803000pt}%
\definecolor{currentstroke}{rgb}{0.000000,0.000000,0.000000}%
\pgfsetstrokecolor{currentstroke}%
\pgfsetdash{}{0pt}%
\pgfsys@defobject{currentmarker}{\pgfqpoint{0.000000in}{-0.048611in}}{\pgfqpoint{0.000000in}{0.000000in}}{%
\pgfpathmoveto{\pgfqpoint{0.000000in}{0.000000in}}%
\pgfpathlineto{\pgfqpoint{0.000000in}{-0.048611in}}%
\pgfusepath{stroke,fill}%
}%
\begin{pgfscope}%
\pgfsys@transformshift{0.708026in}{0.526079in}%
\pgfsys@useobject{currentmarker}{}%
\end{pgfscope}%
\end{pgfscope}%
\begin{pgfscope}%
\pgftext[x=0.708026in,y=0.428857in,,top]{\rmfamily\fontsize{10.000000}{12.000000}\selectfont \(\displaystyle 0\)}%
\end{pgfscope}%
\begin{pgfscope}%
\pgfsetbuttcap%
\pgfsetroundjoin%
\definecolor{currentfill}{rgb}{0.000000,0.000000,0.000000}%
\pgfsetfillcolor{currentfill}%
\pgfsetlinewidth{0.803000pt}%
\definecolor{currentstroke}{rgb}{0.000000,0.000000,0.000000}%
\pgfsetstrokecolor{currentstroke}%
\pgfsetdash{}{0pt}%
\pgfsys@defobject{currentmarker}{\pgfqpoint{0.000000in}{-0.048611in}}{\pgfqpoint{0.000000in}{0.000000in}}{%
\pgfpathmoveto{\pgfqpoint{0.000000in}{0.000000in}}%
\pgfpathlineto{\pgfqpoint{0.000000in}{-0.048611in}}%
\pgfusepath{stroke,fill}%
}%
\begin{pgfscope}%
\pgfsys@transformshift{1.366776in}{0.526079in}%
\pgfsys@useobject{currentmarker}{}%
\end{pgfscope}%
\end{pgfscope}%
\begin{pgfscope}%
\pgftext[x=1.366776in,y=0.428857in,,top]{\rmfamily\fontsize{10.000000}{12.000000}\selectfont \(\displaystyle 2\)}%
\end{pgfscope}%
\begin{pgfscope}%
\pgfsetbuttcap%
\pgfsetroundjoin%
\definecolor{currentfill}{rgb}{0.000000,0.000000,0.000000}%
\pgfsetfillcolor{currentfill}%
\pgfsetlinewidth{0.803000pt}%
\definecolor{currentstroke}{rgb}{0.000000,0.000000,0.000000}%
\pgfsetstrokecolor{currentstroke}%
\pgfsetdash{}{0pt}%
\pgfsys@defobject{currentmarker}{\pgfqpoint{0.000000in}{-0.048611in}}{\pgfqpoint{0.000000in}{0.000000in}}{%
\pgfpathmoveto{\pgfqpoint{0.000000in}{0.000000in}}%
\pgfpathlineto{\pgfqpoint{0.000000in}{-0.048611in}}%
\pgfusepath{stroke,fill}%
}%
\begin{pgfscope}%
\pgfsys@transformshift{2.025526in}{0.526079in}%
\pgfsys@useobject{currentmarker}{}%
\end{pgfscope}%
\end{pgfscope}%
\begin{pgfscope}%
\pgftext[x=2.025526in,y=0.428857in,,top]{\rmfamily\fontsize{10.000000}{12.000000}\selectfont \(\displaystyle 4\)}%
\end{pgfscope}%
\begin{pgfscope}%
\pgfsetbuttcap%
\pgfsetroundjoin%
\definecolor{currentfill}{rgb}{0.000000,0.000000,0.000000}%
\pgfsetfillcolor{currentfill}%
\pgfsetlinewidth{0.803000pt}%
\definecolor{currentstroke}{rgb}{0.000000,0.000000,0.000000}%
\pgfsetstrokecolor{currentstroke}%
\pgfsetdash{}{0pt}%
\pgfsys@defobject{currentmarker}{\pgfqpoint{0.000000in}{-0.048611in}}{\pgfqpoint{0.000000in}{0.000000in}}{%
\pgfpathmoveto{\pgfqpoint{0.000000in}{0.000000in}}%
\pgfpathlineto{\pgfqpoint{0.000000in}{-0.048611in}}%
\pgfusepath{stroke,fill}%
}%
\begin{pgfscope}%
\pgfsys@transformshift{2.684276in}{0.526079in}%
\pgfsys@useobject{currentmarker}{}%
\end{pgfscope}%
\end{pgfscope}%
\begin{pgfscope}%
\pgftext[x=2.684276in,y=0.428857in,,top]{\rmfamily\fontsize{10.000000}{12.000000}\selectfont \(\displaystyle 6\)}%
\end{pgfscope}%
\begin{pgfscope}%
\pgfsetbuttcap%
\pgfsetroundjoin%
\definecolor{currentfill}{rgb}{0.000000,0.000000,0.000000}%
\pgfsetfillcolor{currentfill}%
\pgfsetlinewidth{0.803000pt}%
\definecolor{currentstroke}{rgb}{0.000000,0.000000,0.000000}%
\pgfsetstrokecolor{currentstroke}%
\pgfsetdash{}{0pt}%
\pgfsys@defobject{currentmarker}{\pgfqpoint{0.000000in}{-0.048611in}}{\pgfqpoint{0.000000in}{0.000000in}}{%
\pgfpathmoveto{\pgfqpoint{0.000000in}{0.000000in}}%
\pgfpathlineto{\pgfqpoint{0.000000in}{-0.048611in}}%
\pgfusepath{stroke,fill}%
}%
\begin{pgfscope}%
\pgfsys@transformshift{3.343026in}{0.526079in}%
\pgfsys@useobject{currentmarker}{}%
\end{pgfscope}%
\end{pgfscope}%
\begin{pgfscope}%
\pgftext[x=3.343026in,y=0.428857in,,top]{\rmfamily\fontsize{10.000000}{12.000000}\selectfont \(\displaystyle 8\)}%
\end{pgfscope}%
\begin{pgfscope}%
\pgftext[x=2.025526in,y=0.238889in,,top]{\rmfamily\fontsize{10.000000}{12.000000}\selectfont \(\displaystyle kc / |\Omega_\mathrm{ce}|\)}%
\end{pgfscope}%
\begin{pgfscope}%
\pgfsetbuttcap%
\pgfsetroundjoin%
\definecolor{currentfill}{rgb}{0.000000,0.000000,0.000000}%
\pgfsetfillcolor{currentfill}%
\pgfsetlinewidth{0.803000pt}%
\definecolor{currentstroke}{rgb}{0.000000,0.000000,0.000000}%
\pgfsetstrokecolor{currentstroke}%
\pgfsetdash{}{0pt}%
\pgfsys@defobject{currentmarker}{\pgfqpoint{-0.048611in}{0.000000in}}{\pgfqpoint{0.000000in}{0.000000in}}{%
\pgfpathmoveto{\pgfqpoint{0.000000in}{0.000000in}}%
\pgfpathlineto{\pgfqpoint{-0.048611in}{0.000000in}}%
\pgfusepath{stroke,fill}%
}%
\begin{pgfscope}%
\pgfsys@transformshift{0.708026in}{0.596518in}%
\pgfsys@useobject{currentmarker}{}%
\end{pgfscope}%
\end{pgfscope}%
\begin{pgfscope}%
\pgftext[x=0.294444in,y=0.543757in,left,base]{\rmfamily\fontsize{10.000000}{12.000000}\selectfont \(\displaystyle 0.000\)}%
\end{pgfscope}%
\begin{pgfscope}%
\pgfsetbuttcap%
\pgfsetroundjoin%
\definecolor{currentfill}{rgb}{0.000000,0.000000,0.000000}%
\pgfsetfillcolor{currentfill}%
\pgfsetlinewidth{0.803000pt}%
\definecolor{currentstroke}{rgb}{0.000000,0.000000,0.000000}%
\pgfsetstrokecolor{currentstroke}%
\pgfsetdash{}{0pt}%
\pgfsys@defobject{currentmarker}{\pgfqpoint{-0.048611in}{0.000000in}}{\pgfqpoint{0.000000in}{0.000000in}}{%
\pgfpathmoveto{\pgfqpoint{0.000000in}{0.000000in}}%
\pgfpathlineto{\pgfqpoint{-0.048611in}{0.000000in}}%
\pgfusepath{stroke,fill}%
}%
\begin{pgfscope}%
\pgfsys@transformshift{0.708026in}{1.038821in}%
\pgfsys@useobject{currentmarker}{}%
\end{pgfscope}%
\end{pgfscope}%
\begin{pgfscope}%
\pgftext[x=0.294444in,y=0.986059in,left,base]{\rmfamily\fontsize{10.000000}{12.000000}\selectfont \(\displaystyle 0.002\)}%
\end{pgfscope}%
\begin{pgfscope}%
\pgfsetbuttcap%
\pgfsetroundjoin%
\definecolor{currentfill}{rgb}{0.000000,0.000000,0.000000}%
\pgfsetfillcolor{currentfill}%
\pgfsetlinewidth{0.803000pt}%
\definecolor{currentstroke}{rgb}{0.000000,0.000000,0.000000}%
\pgfsetstrokecolor{currentstroke}%
\pgfsetdash{}{0pt}%
\pgfsys@defobject{currentmarker}{\pgfqpoint{-0.048611in}{0.000000in}}{\pgfqpoint{0.000000in}{0.000000in}}{%
\pgfpathmoveto{\pgfqpoint{0.000000in}{0.000000in}}%
\pgfpathlineto{\pgfqpoint{-0.048611in}{0.000000in}}%
\pgfusepath{stroke,fill}%
}%
\begin{pgfscope}%
\pgfsys@transformshift{0.708026in}{1.481123in}%
\pgfsys@useobject{currentmarker}{}%
\end{pgfscope}%
\end{pgfscope}%
\begin{pgfscope}%
\pgftext[x=0.294444in,y=1.428362in,left,base]{\rmfamily\fontsize{10.000000}{12.000000}\selectfont \(\displaystyle 0.004\)}%
\end{pgfscope}%
\begin{pgfscope}%
\pgfsetbuttcap%
\pgfsetroundjoin%
\definecolor{currentfill}{rgb}{0.000000,0.000000,0.000000}%
\pgfsetfillcolor{currentfill}%
\pgfsetlinewidth{0.803000pt}%
\definecolor{currentstroke}{rgb}{0.000000,0.000000,0.000000}%
\pgfsetstrokecolor{currentstroke}%
\pgfsetdash{}{0pt}%
\pgfsys@defobject{currentmarker}{\pgfqpoint{-0.048611in}{0.000000in}}{\pgfqpoint{0.000000in}{0.000000in}}{%
\pgfpathmoveto{\pgfqpoint{0.000000in}{0.000000in}}%
\pgfpathlineto{\pgfqpoint{-0.048611in}{0.000000in}}%
\pgfusepath{stroke,fill}%
}%
\begin{pgfscope}%
\pgfsys@transformshift{0.708026in}{1.923426in}%
\pgfsys@useobject{currentmarker}{}%
\end{pgfscope}%
\end{pgfscope}%
\begin{pgfscope}%
\pgftext[x=0.294444in,y=1.870664in,left,base]{\rmfamily\fontsize{10.000000}{12.000000}\selectfont \(\displaystyle 0.006\)}%
\end{pgfscope}%
\begin{pgfscope}%
\pgftext[x=0.238889in,y=1.281079in,,bottom,rotate=90.000000]{\rmfamily\fontsize{10.000000}{12.000000}\selectfont \(\displaystyle \gamma / |\Omega_\mathrm{ce}|\)}%
\end{pgfscope}%
\begin{pgfscope}%
\pgfpathrectangle{\pgfqpoint{0.708026in}{0.526079in}}{\pgfqpoint{2.635000in}{1.510000in}} %
\pgfusepath{clip}%
\pgfsetrectcap%
\pgfsetroundjoin%
\pgfsetlinewidth{1.003750pt}%
\definecolor{currentstroke}{rgb}{0.627451,0.321569,0.176471}%
\pgfsetstrokecolor{currentstroke}%
\pgfsetdash{}{0pt}%
\pgfpathmoveto{\pgfqpoint{0.839776in}{0.596518in}}%
\pgfpathlineto{\pgfqpoint{0.865061in}{0.596518in}}%
\pgfpathlineto{\pgfqpoint{0.890346in}{0.596518in}}%
\pgfpathlineto{\pgfqpoint{0.915632in}{0.596518in}}%
\pgfpathlineto{\pgfqpoint{0.940917in}{0.596518in}}%
\pgfpathlineto{\pgfqpoint{0.966202in}{0.596527in}}%
\pgfpathlineto{\pgfqpoint{0.991488in}{0.596717in}}%
\pgfpathlineto{\pgfqpoint{1.016773in}{0.598452in}}%
\pgfpathlineto{\pgfqpoint{1.042059in}{0.607122in}}%
\pgfpathlineto{\pgfqpoint{1.067344in}{0.634871in}}%
\pgfpathlineto{\pgfqpoint{1.092629in}{0.698318in}}%
\pgfpathlineto{\pgfqpoint{1.117915in}{0.810099in}}%
\pgfpathlineto{\pgfqpoint{1.143200in}{0.970285in}}%
\pgfpathlineto{\pgfqpoint{1.168485in}{1.164607in}}%
\pgfpathlineto{\pgfqpoint{1.193771in}{1.370191in}}%
\pgfpathlineto{\pgfqpoint{1.219056in}{1.563739in}}%
\pgfpathlineto{\pgfqpoint{1.244341in}{1.727360in}}%
\pgfpathlineto{\pgfqpoint{1.269627in}{1.850659in}}%
\pgfpathlineto{\pgfqpoint{1.294912in}{1.930138in}}%
\pgfpathlineto{\pgfqpoint{1.320197in}{1.967441in}}%
\pgfpathlineto{\pgfqpoint{1.345483in}{1.967443in}}%
\pgfpathlineto{\pgfqpoint{1.370768in}{1.936656in}}%
\pgfpathlineto{\pgfqpoint{1.396053in}{1.882067in}}%
\pgfpathlineto{\pgfqpoint{1.421339in}{1.810376in}}%
\pgfpathlineto{\pgfqpoint{1.446624in}{1.727552in}}%
\pgfpathlineto{\pgfqpoint{1.471910in}{1.638620in}}%
\pgfpathlineto{\pgfqpoint{1.497195in}{1.547607in}}%
\pgfpathlineto{\pgfqpoint{1.522480in}{1.457591in}}%
\pgfpathlineto{\pgfqpoint{1.547766in}{1.370805in}}%
\pgfpathlineto{\pgfqpoint{1.573051in}{1.288766in}}%
\pgfpathlineto{\pgfqpoint{1.598336in}{1.212415in}}%
\pgfpathlineto{\pgfqpoint{1.623622in}{1.142238in}}%
\pgfpathlineto{\pgfqpoint{1.648907in}{1.078387in}}%
\pgfpathlineto{\pgfqpoint{1.674192in}{1.020769in}}%
\pgfpathlineto{\pgfqpoint{1.699478in}{0.969126in}}%
\pgfpathlineto{\pgfqpoint{1.724763in}{0.923095in}}%
\pgfpathlineto{\pgfqpoint{1.750048in}{0.882252in}}%
\pgfpathlineto{\pgfqpoint{1.775334in}{0.846149in}}%
\pgfpathlineto{\pgfqpoint{1.800619in}{0.814331in}}%
\pgfpathlineto{\pgfqpoint{1.825904in}{0.786361in}}%
\pgfpathlineto{\pgfqpoint{1.851190in}{0.761822in}}%
\pgfpathlineto{\pgfqpoint{1.876475in}{0.740329in}}%
\pgfpathlineto{\pgfqpoint{1.901761in}{0.721528in}}%
\pgfpathlineto{\pgfqpoint{1.927046in}{0.705098in}}%
\pgfpathlineto{\pgfqpoint{1.952331in}{0.690753in}}%
\pgfpathlineto{\pgfqpoint{1.977617in}{0.678236in}}%
\pgfpathlineto{\pgfqpoint{2.002902in}{0.667318in}}%
\pgfpathlineto{\pgfqpoint{2.028187in}{0.657799in}}%
\pgfpathlineto{\pgfqpoint{2.053473in}{0.649502in}}%
\pgfpathlineto{\pgfqpoint{2.078758in}{0.642272in}}%
\pgfpathlineto{\pgfqpoint{2.104043in}{0.635972in}}%
\pgfpathlineto{\pgfqpoint{2.129329in}{0.630483in}}%
\pgfpathlineto{\pgfqpoint{2.154614in}{0.625702in}}%
\pgfpathlineto{\pgfqpoint{2.179899in}{0.621537in}}%
\pgfpathlineto{\pgfqpoint{2.205185in}{0.617910in}}%
\pgfpathlineto{\pgfqpoint{2.230470in}{0.614751in}}%
\pgfpathlineto{\pgfqpoint{2.255755in}{0.612001in}}%
\pgfpathlineto{\pgfqpoint{2.281041in}{0.609608in}}%
\pgfpathlineto{\pgfqpoint{2.306326in}{0.607525in}}%
\pgfpathlineto{\pgfqpoint{2.331612in}{0.605714in}}%
\pgfpathlineto{\pgfqpoint{2.356897in}{0.604139in}}%
\pgfpathlineto{\pgfqpoint{2.382182in}{0.602772in}}%
\pgfpathlineto{\pgfqpoint{2.407468in}{0.601584in}}%
\pgfpathlineto{\pgfqpoint{2.432753in}{0.600554in}}%
\pgfpathlineto{\pgfqpoint{2.458038in}{0.599662in}}%
\pgfpathlineto{\pgfqpoint{2.483324in}{0.598890in}}%
\pgfpathlineto{\pgfqpoint{2.508609in}{0.598224in}}%
\pgfpathlineto{\pgfqpoint{2.533894in}{0.597649in}}%
\pgfpathlineto{\pgfqpoint{2.559180in}{0.597154in}}%
\pgfpathlineto{\pgfqpoint{2.584465in}{0.596730in}}%
\pgfpathlineto{\pgfqpoint{2.609750in}{0.596367in}}%
\pgfpathlineto{\pgfqpoint{2.635036in}{0.596057in}}%
\pgfpathlineto{\pgfqpoint{2.660321in}{0.595795in}}%
\pgfpathlineto{\pgfqpoint{2.685607in}{0.595573in}}%
\pgfpathlineto{\pgfqpoint{2.710892in}{0.595387in}}%
\pgfpathlineto{\pgfqpoint{2.736177in}{0.595231in}}%
\pgfpathlineto{\pgfqpoint{2.761463in}{0.595104in}}%
\pgfpathlineto{\pgfqpoint{2.786748in}{0.594999in}}%
\pgfpathlineto{\pgfqpoint{2.812033in}{0.594916in}}%
\pgfpathlineto{\pgfqpoint{2.837319in}{0.594850in}}%
\pgfpathlineto{\pgfqpoint{2.862604in}{0.594799in}}%
\pgfpathlineto{\pgfqpoint{2.887889in}{0.594763in}}%
\pgfpathlineto{\pgfqpoint{2.913175in}{0.594737in}}%
\pgfpathlineto{\pgfqpoint{2.938460in}{0.594722in}}%
\pgfpathlineto{\pgfqpoint{2.963745in}{0.594716in}}%
\pgfpathlineto{\pgfqpoint{2.989031in}{0.594717in}}%
\pgfpathlineto{\pgfqpoint{3.014316in}{0.594724in}}%
\pgfpathlineto{\pgfqpoint{3.039601in}{0.594737in}}%
\pgfpathlineto{\pgfqpoint{3.064887in}{0.594754in}}%
\pgfpathlineto{\pgfqpoint{3.090172in}{0.594775in}}%
\pgfpathlineto{\pgfqpoint{3.115458in}{0.594800in}}%
\pgfpathlineto{\pgfqpoint{3.140743in}{0.594827in}}%
\pgfpathlineto{\pgfqpoint{3.166028in}{0.594856in}}%
\pgfpathlineto{\pgfqpoint{3.191314in}{0.594887in}}%
\pgfpathlineto{\pgfqpoint{3.216599in}{0.594919in}}%
\pgfpathlineto{\pgfqpoint{3.241884in}{0.594953in}}%
\pgfpathlineto{\pgfqpoint{3.267170in}{0.594987in}}%
\pgfpathlineto{\pgfqpoint{3.292455in}{0.595022in}}%
\pgfpathlineto{\pgfqpoint{3.317740in}{0.595057in}}%
\pgfpathlineto{\pgfqpoint{3.343026in}{0.595092in}}%
\pgfusepath{stroke}%
\end{pgfscope}%
\begin{pgfscope}%
\pgfpathrectangle{\pgfqpoint{0.708026in}{0.526079in}}{\pgfqpoint{2.635000in}{1.510000in}} %
\pgfusepath{clip}%
\pgfsetbuttcap%
\pgfsetroundjoin%
\pgfsetlinewidth{0.501875pt}%
\definecolor{currentstroke}{rgb}{0.000000,0.000000,0.000000}%
\pgfsetstrokecolor{currentstroke}%
\pgfsetdash{{1.850000pt}{0.800000pt}}{0.000000pt}%
\pgfpathmoveto{\pgfqpoint{0.694137in}{0.596518in}}%
\pgfpathlineto{\pgfqpoint{0.727988in}{0.596518in}}%
\pgfpathlineto{\pgfqpoint{0.767912in}{0.596518in}}%
\pgfpathlineto{\pgfqpoint{0.807836in}{0.596518in}}%
\pgfpathlineto{\pgfqpoint{0.847761in}{0.596518in}}%
\pgfpathlineto{\pgfqpoint{0.887685in}{0.596518in}}%
\pgfpathlineto{\pgfqpoint{0.927609in}{0.596518in}}%
\pgfpathlineto{\pgfqpoint{0.967533in}{0.596518in}}%
\pgfpathlineto{\pgfqpoint{1.007458in}{0.596518in}}%
\pgfpathlineto{\pgfqpoint{1.047382in}{0.596518in}}%
\pgfpathlineto{\pgfqpoint{1.087306in}{0.596518in}}%
\pgfpathlineto{\pgfqpoint{1.127230in}{0.596518in}}%
\pgfpathlineto{\pgfqpoint{1.167154in}{0.596518in}}%
\pgfpathlineto{\pgfqpoint{1.207079in}{0.596518in}}%
\pgfpathlineto{\pgfqpoint{1.247003in}{0.596518in}}%
\pgfpathlineto{\pgfqpoint{1.286927in}{0.596518in}}%
\pgfpathlineto{\pgfqpoint{1.326851in}{0.596518in}}%
\pgfpathlineto{\pgfqpoint{1.366776in}{0.596518in}}%
\pgfpathlineto{\pgfqpoint{1.406700in}{0.596518in}}%
\pgfpathlineto{\pgfqpoint{1.446624in}{0.596518in}}%
\pgfpathlineto{\pgfqpoint{1.486548in}{0.596518in}}%
\pgfpathlineto{\pgfqpoint{1.526473in}{0.596518in}}%
\pgfpathlineto{\pgfqpoint{1.566397in}{0.596518in}}%
\pgfpathlineto{\pgfqpoint{1.606321in}{0.596518in}}%
\pgfpathlineto{\pgfqpoint{1.646245in}{0.596518in}}%
\pgfpathlineto{\pgfqpoint{1.686170in}{0.596518in}}%
\pgfpathlineto{\pgfqpoint{1.726094in}{0.596518in}}%
\pgfpathlineto{\pgfqpoint{1.766018in}{0.596518in}}%
\pgfpathlineto{\pgfqpoint{1.805942in}{0.596518in}}%
\pgfpathlineto{\pgfqpoint{1.845867in}{0.596518in}}%
\pgfpathlineto{\pgfqpoint{1.885791in}{0.596518in}}%
\pgfpathlineto{\pgfqpoint{1.925715in}{0.596518in}}%
\pgfpathlineto{\pgfqpoint{1.965639in}{0.596518in}}%
\pgfpathlineto{\pgfqpoint{2.005564in}{0.596518in}}%
\pgfpathlineto{\pgfqpoint{2.045488in}{0.596518in}}%
\pgfpathlineto{\pgfqpoint{2.085412in}{0.596518in}}%
\pgfpathlineto{\pgfqpoint{2.125336in}{0.596518in}}%
\pgfpathlineto{\pgfqpoint{2.165261in}{0.596518in}}%
\pgfpathlineto{\pgfqpoint{2.205185in}{0.596518in}}%
\pgfpathlineto{\pgfqpoint{2.245109in}{0.596518in}}%
\pgfpathlineto{\pgfqpoint{2.285033in}{0.596518in}}%
\pgfpathlineto{\pgfqpoint{2.324958in}{0.596518in}}%
\pgfpathlineto{\pgfqpoint{2.364882in}{0.596518in}}%
\pgfpathlineto{\pgfqpoint{2.404806in}{0.596518in}}%
\pgfpathlineto{\pgfqpoint{2.444730in}{0.596518in}}%
\pgfpathlineto{\pgfqpoint{2.484654in}{0.596518in}}%
\pgfpathlineto{\pgfqpoint{2.524579in}{0.596518in}}%
\pgfpathlineto{\pgfqpoint{2.564503in}{0.596518in}}%
\pgfpathlineto{\pgfqpoint{2.604427in}{0.596518in}}%
\pgfpathlineto{\pgfqpoint{2.644351in}{0.596518in}}%
\pgfpathlineto{\pgfqpoint{2.684276in}{0.596518in}}%
\pgfpathlineto{\pgfqpoint{2.724200in}{0.596518in}}%
\pgfpathlineto{\pgfqpoint{2.764124in}{0.596518in}}%
\pgfpathlineto{\pgfqpoint{2.804048in}{0.596518in}}%
\pgfpathlineto{\pgfqpoint{2.843973in}{0.596518in}}%
\pgfpathlineto{\pgfqpoint{2.883897in}{0.596518in}}%
\pgfpathlineto{\pgfqpoint{2.923821in}{0.596518in}}%
\pgfpathlineto{\pgfqpoint{2.963745in}{0.596518in}}%
\pgfpathlineto{\pgfqpoint{3.003670in}{0.596518in}}%
\pgfpathlineto{\pgfqpoint{3.043594in}{0.596518in}}%
\pgfpathlineto{\pgfqpoint{3.083518in}{0.596518in}}%
\pgfpathlineto{\pgfqpoint{3.123442in}{0.596518in}}%
\pgfpathlineto{\pgfqpoint{3.163367in}{0.596518in}}%
\pgfpathlineto{\pgfqpoint{3.203291in}{0.596518in}}%
\pgfpathlineto{\pgfqpoint{3.243215in}{0.596518in}}%
\pgfpathlineto{\pgfqpoint{3.283139in}{0.596518in}}%
\pgfpathlineto{\pgfqpoint{3.323064in}{0.596518in}}%
\pgfpathlineto{\pgfqpoint{3.356915in}{0.596518in}}%
\pgfusepath{stroke}%
\end{pgfscope}%
\begin{pgfscope}%
\pgfsetrectcap%
\pgfsetmiterjoin%
\pgfsetlinewidth{0.803000pt}%
\definecolor{currentstroke}{rgb}{0.000000,0.000000,0.000000}%
\pgfsetstrokecolor{currentstroke}%
\pgfsetdash{}{0pt}%
\pgfpathmoveto{\pgfqpoint{0.708026in}{0.526079in}}%
\pgfpathlineto{\pgfqpoint{0.708026in}{2.036079in}}%
\pgfusepath{stroke}%
\end{pgfscope}%
\begin{pgfscope}%
\pgfsetrectcap%
\pgfsetmiterjoin%
\pgfsetlinewidth{0.803000pt}%
\definecolor{currentstroke}{rgb}{0.000000,0.000000,0.000000}%
\pgfsetstrokecolor{currentstroke}%
\pgfsetdash{}{0pt}%
\pgfpathmoveto{\pgfqpoint{3.343026in}{0.526079in}}%
\pgfpathlineto{\pgfqpoint{3.343026in}{2.036079in}}%
\pgfusepath{stroke}%
\end{pgfscope}%
\begin{pgfscope}%
\pgfsetrectcap%
\pgfsetmiterjoin%
\pgfsetlinewidth{0.803000pt}%
\definecolor{currentstroke}{rgb}{0.000000,0.000000,0.000000}%
\pgfsetstrokecolor{currentstroke}%
\pgfsetdash{}{0pt}%
\pgfpathmoveto{\pgfqpoint{0.708026in}{0.526079in}}%
\pgfpathlineto{\pgfqpoint{3.343026in}{0.526079in}}%
\pgfusepath{stroke}%
\end{pgfscope}%
\begin{pgfscope}%
\pgfsetrectcap%
\pgfsetmiterjoin%
\pgfsetlinewidth{0.803000pt}%
\definecolor{currentstroke}{rgb}{0.000000,0.000000,0.000000}%
\pgfsetstrokecolor{currentstroke}%
\pgfsetdash{}{0pt}%
\pgfpathmoveto{\pgfqpoint{0.708026in}{2.036079in}}%
\pgfpathlineto{\pgfqpoint{3.343026in}{2.036079in}}%
\pgfusepath{stroke}%
\end{pgfscope}%
\begin{pgfscope}%
\pgfsetbuttcap%
\pgfsetmiterjoin%
\definecolor{currentfill}{rgb}{1.000000,1.000000,1.000000}%
\pgfsetfillcolor{currentfill}%
\pgfsetfillopacity{0.800000}%
\pgfsetlinewidth{1.003750pt}%
\definecolor{currentstroke}{rgb}{0.800000,0.800000,0.800000}%
\pgfsetstrokecolor{currentstroke}%
\pgfsetstrokeopacity{0.800000}%
\pgfsetdash{}{0pt}%
\pgfpathmoveto{\pgfqpoint{2.199213in}{1.721111in}}%
\pgfpathlineto{\pgfqpoint{3.245803in}{1.721111in}}%
\pgfpathquadraticcurveto{\pgfqpoint{3.273581in}{1.721111in}}{\pgfqpoint{3.273581in}{1.748889in}}%
\pgfpathlineto{\pgfqpoint{3.273581in}{1.938857in}}%
\pgfpathquadraticcurveto{\pgfqpoint{3.273581in}{1.966635in}}{\pgfqpoint{3.245803in}{1.966635in}}%
\pgfpathlineto{\pgfqpoint{2.199213in}{1.966635in}}%
\pgfpathquadraticcurveto{\pgfqpoint{2.171436in}{1.966635in}}{\pgfqpoint{2.171436in}{1.938857in}}%
\pgfpathlineto{\pgfqpoint{2.171436in}{1.748889in}}%
\pgfpathquadraticcurveto{\pgfqpoint{2.171436in}{1.721111in}}{\pgfqpoint{2.199213in}{1.721111in}}%
\pgfpathclose%
\pgfusepath{stroke,fill}%
\end{pgfscope}%
\begin{pgfscope}%
\pgfsetrectcap%
\pgfsetroundjoin%
\pgfsetlinewidth{1.003750pt}%
\definecolor{currentstroke}{rgb}{0.627451,0.321569,0.176471}%
\pgfsetstrokecolor{currentstroke}%
\pgfsetdash{}{0pt}%
\pgfpathmoveto{\pgfqpoint{2.226991in}{1.854167in}}%
\pgfpathlineto{\pgfqpoint{2.504769in}{1.854167in}}%
\pgfusepath{stroke}%
\end{pgfscope}%
\begin{pgfscope}%
\pgftext[x=2.615880in,y=1.805556in,left,base]{\rmfamily\fontsize{10.000000}{12.000000}\selectfont \(\displaystyle \mathrm{R}\) - wave}%
\end{pgfscope}%
\end{pgfpicture}%
\makeatother%
\endgroup%
}
\caption{(a) Numerical solutions of the dispersion relation (\ref{eq_dispersion_relation}) (real part) for $\Omega_\mr{pe}=2|\Omega_\mr{ce}|$, $\nu_\mr{h}=0.005$, $\vpar=0.2c$ and $\vperp=0.6c$. (b) Same for the imaginary part. Here, only the solution corresponding for the R-wave below the electron cyclotron frequency $|\Omega_\mr{ce}|$ is shown since the imaginary parts of the other two branches are close to zero.\label{fig_solutions_dispersion}}
\end{figure}
The other two solutions correspond to right-handed (R) and left-handed (L) circularly polarized waves (transversal waves with perpendicular perturbations with respect to the background magnetic field only), respectively. The dispersion relation for these types of waves for an arbitrary hot electron equilibrium distribution function reads
\begin{align}
D_{\mr{R/L}}(k,\omega)=1-\frac{c^2k^2}{\omega^2}-\frac{\Omega_\mr{pe}^2}{\omega(\omega\pm\Omega_\mr{ce})}+\nu_\mr{h}\frac{\Omega_\mr{pe}^2}{\omega}\int\frac{v_\perp}{2}\frac{\hat{G}F_\mr{h}^0}{\omega\pm\Omega_\mr{ce}-kv_\parallel}\mr{d}^3\mb{v}\overset{!}{=}0,
\end{align}
where $\nu_\mr{h}=n_{\mr{h}0}/n_{\mr{c}0}$ is the ratio between hot and cold electron number densities and $F_\mr{h}^0$ is the velocity part of the equilibrium distribution function, i.e. $f_\mr{h}^0(v_\perp,v_\parallel)=n_{\mr{h}0}F_\mr{h}^0(v_\perp,v_\parallel)$ and $\hat{G}$ is a differential operator measuring the anisotropy of the distribution function in velocity space \citep{Brambilla1998, Xiaoetal1998}:
\begin{align}
\hat{G}=\frac{\pa}{\pa v_\perp}+\frac{k}{\omega}\left(v_\perp\frac{\pa}{\pa v_\parallel}-v_\parallel\frac{\pa}{\pa v_\perp}\right).
\end{align}
In order to satisfy the steady-state Vlasov equation with the background magnetic field $\mb{B}_0$ it is straightforward to show the equilibrium distribution function must be rotationally symmetric around the magnetic field and therefore only depends on $v_\perp^2=v_x^2+v_y^2$ and $v_\parallel=v_z$. For the special case of an anisotropic Maxwellian with generally different thermal velocities in parallel and perpendicular direction,
\begin{align}
F_\mr{h}^0(v_\perp,v_\parallel)=\frac{1}{(2\pi)^{3/2}v_{\mr{th}\parallel}v_{\mr{th}\perp}^2}\exp\left(-\frac{v_\perp^2}{2v_{\mr{th}\perp}^2}-\frac{v_\parallel^2}{2v_{\mr{th}\parallel}^2}\right),\label{eq_anisotropic_Maxwellian}
\end{align} 
the dispersion relation transfers to
\begin{align}
D_{\mr{R/L}}(k,\omega)=1-\frac{c^2k^2}{\omega^2}-\frac{\Omega_\mr{pe}^2}{\omega(\omega\pm\Omega_\mr{ce})}+\nu_\mr{h}\frac{\Omega_\mr{pe}^2}{\omega^2}\left[\frac{\omega}{k\sqrt{2}v_{\mr{th}\parallel}}Z(\xi^\pm)-\left(1-\frac{v_{\mr{th}\perp}^2}{v_{\mr{th}\parallel}^2}\right)(1+\xi^\pm Z(\xi^\pm))\right]=0,\label{eq_dispersion_relation}
\end{align}
where $\xi^\pm=(\omega\pm\Omega_\mr{ce})/k\sqrt{2}v_{\mr{th}\parallel}$ and $Z$ is the plasma dispersion function given by
\begin{align}
Z(\xi)=\sqrt{\pi}\mr{e}^{-\xi^2}\left(i-\frac{2}{\sqrt{\pi}}\int_0^\xi\mr{e}^{t^2}\mr{d}t\right)=\sqrt{\pi}\mr{e}^{-\xi^2}(i-\mr{erfi}(\xi))\label{eq_plasma_dispersion_function}.
\end{align}
In the absence of energetic electrons ($\nu_\mr{h}\rightarrow0$), the dispersion relation (\ref{eq_dispersion_relation}) transfers to the well-known cold plasma dispersion relation for electron waves, which only provides solutions with real oscillation frequencies $\omega_r:=\mr{Im}(\omega)$ for all wave numbers $k$. This means that there is no wave growth or damping due to an imaginary part $\gamma:=\mr{Im}(\omega)$ \citep{Brambilla1998}. However, depending on the temperature anisotropy of $F_\mr{h}^0$, the dispersion relation (\ref{eq_dispersion_relation}) provides solutions with $\gamma\neq0$ which is shown in fig. \ref{fig_solutions_dispersion}, where we plot the real frequency $\omega_\mr{r}$ on the left-hand side and the growth rate $\gamma$ on the right-hand side. One can see that there are two solutions for R-waves and one solution for L-waves, which is known from the cold plasma theory. However, due to interaction of waves with fast electrons that meet the resonance condition $\omega=kv_\parallel\mp\Omega_\mr{ce}$, the lower branch below the electron cyclotron frequency becomes unstable for a certain range of wave numbers if the temperature anisotropy is sufficiently large.

We shall use these results for the verification of the developed numerical algorithms.

\section{Numerical methods}
In this section, we apply two kinds of numerical methods on the electron hybrid model which we have just discussed on the continuous level and for which the linear dispersion relation (\ref{eq_dispersion_relation}) is available. We shall restrict ourselves on this case and consequently consider perpendicular perturbations only. This means that we neglect the $z$-components of the fields $\tilde{\mb{E}}$, $\tilde{\mb{B}}$ and $\tilde{\mb{j}}_\mr{c}$ in the model (\ref{eq_model_linearized}). However, we retain the $z$-components of the hot electrons, since this is crucial to obtain resonant particles which then leads to wave-particle interaction and energy transfers between waves and particles. We start with an intuitive application of a combination of classical finite elements for solving field equations and the classical particle-in-cell methods for solving the Vlasov equation followed by applying structure-preserving finite element particle-in-cell methods. 

\subsection{Standard finite element particle-in-cell}
\label{sec_standard}
As a first step, we write the momentum balance equation (\ref{eq_model_linearized_1}), Faraday's law (\ref{eq_model_linearized_3}) and Amp\'{e}res law (\ref{eq_model_linearized_4}) in the compact form
\begin{subequations}
\label{eq_compact}
\begin{align}
&\frac{\pa\mb{U}}{\pa t}+A_1\frac{\pa\mb{U}}{\pa z}+A_2\mb{U}=\mb{S},\\
&\mb{U}(0,t)=\mb{U}(L,t),
\end{align}
\end{subequations}
for the vector of unknowns $\mb{U}=(\tilde{E}_x,\tilde{E}_y,\tilde{B}_x,\tilde{B}_y,\tilde{j}_{\mr{c}x},\tilde{j}_{\mr{c}y})$ and impose periodic boundary conditions on the domain $\Omega=(0,L)$, where $L$ is the length of the computational domain. The constant matrices $A_1,A_2\in\mathbb{R}^6$ and the source term $\mb{S}$ are
\begin{subequations}
\begin{align}
A_1&=
\begin{pmatrix}
0 &0  &0 &c^2  &0 &0 \\
0 &0  &-c^2 &0 &0 &0 \\
0 &-1  &0 &0 &0 &0  \\
1 &0  &0 &0 &0 &0  \\
0 &0  &0 &0 &0 &0   \\
0 &0  &0 &0 &0 &0 
\end{pmatrix},
\end{align}
\begin{align}
A_2&=
\begin{pmatrix}
0 &0 &0 &0 &\mu_0c^2 &0 \\
0 &0 &0 &0 &0 &\mu_0c^2 \\
0 &0 &0 &0 &0 &0 \\
0 &0 &0 &0 &0 &0 \\
-\epsilon_0\Omega_{\mr{pe}}^2 &0 &0 &0 &0 &-\Omega_{\mr{ce}} \\
0 &-\epsilon_0\Omega_{\mr{pe}}^2 &0 &0 &\Omega_{\mr{ce}} &0 \\
\end{pmatrix},
\end{align}
\begin{align}
\textbf{S}&=
\begin{pmatrix}
-\mu_0c^2 j_{\mr{h}x} \\
-\mu_0c^2 j_{\mr{h}y} \\
0 \\
0 \\
0 \\
0
\end{pmatrix}\label{eq_source_term}.
\end{align}
\end{subequations}
Following classical finite element methods (see \citep{Doneaetal2003}, for instance), the corresponding weak formulation of the problem is obtained by multiplying the equation with a test function $V\in H^1$ and integrating over the domain $\Omega$. We shall need the spaces $L^2$ of square integrable functions and $H^1$ of square integrable function with the first derivative being in $L^2$. The problem then reads: Find $\mb{U}\in\underbrace{H^1\times\ldots\times H^1}_{\text{6 times}}$ such that
\begin{align}
\int_0^L\frac{\pa \mb{U}}{\pa t}V\mr{d}z+A_1\int_0^L\frac{\pa\mb{U}}{\pa z}V\mr{d}z+A_2\int_0^L\mb{U}V\mr{d}z=\int_0^L\mb{S}V\mr{d}z\quad\quad\forall\,V\in H^1.
\end{align}
As a next step, we replace the function space $H^1$ by a finite-dimensional subspace $\mathcal{S}_h\subset H^1$ in which we look for the approximate solution $\mb{U}_h$ of the problem (\ref{eq_compact}). In addition to that, we use the same subspace for the trail function $\mb{U}_h$ and the test function $V_h$ (Bubnov-Galerkin-method). This leads to the following discrete version of the above problem: Find $\mb{U}\in\underbrace{S_h\times\ldots\times S_h}_{\text{6 times}}$ such that
\begin{align}
\int_0^L\frac{\pa \mb{U}_h}{\pa t}V_h\mr{d}z+A_1\int_0^L\frac{\pa\mb{U}_h}{\pa z}V_h\mr{d}z+A_2\int_0^L\mb{U}_hV_h\mr{d}z=\int_0^L\mb{S}V_h\mr{d}z\quad\quad\forall\,V_h\in\mathcal{S}_h.\label{eq_weak_discrete}
\end{align}
Expanding trail and test function in a basis of $\mathcal{S}_h$ denoted by $(\varphi_j)_{j=0,\ldots,N-1}$, where $N$ is the dimension of $\mathcal{S}_h$, as
\begin{align}
\mb{U}_h(z,t)=\sum_{j=0}^{N-1}\mb{u}_j(t)\varphi_j(z),\quad\quad V_h(z)=\sum_{j=0}^{N-1}v_j\varphi_j(z),\label{eq_expansion}
\end{align}
and plugging these expressions in the discrete weak formulation (\ref{eq_weak_discrete}) yields
\begin{align}
\sum_{i,j=0}^{N-1}v_i\frac{\mr{d}\mb{u}_j}{\mr{d}t}\underbrace{\int_0^L\varphi_i\varphi_j\mr{d}z}_{=:m_{ij}}+A_1\sum_{i,j=0}^{N-1}v_i\mb{u}_j\underbrace{\int_0^L\varphi_i\varphi_j^\prime\mr{d}z}_{=:c_{ij}}+A_2\sum_{i,j=0}^{N-1}v_i\mb{u}_j\underbrace{\int_0^L\varphi_i\varphi_j\mr{d}z}_{=:m_{ij}}=\sum_{i=0}^{N-1}v_i\int_0^L\mb{S}\varphi_i\mr{d}z,\label{eq_matrix_formulation_1}
\end{align}
where we have defined the entries of the mass matrix $\mathbb{M}:=(m_{ij})_{i,j=0,\ldots,N-1}$ and the advection matrix $\mathbb{C}:=(c_{ij})_{i,j=0,\ldots,N-1}$. With this, (\ref{eq_matrix_formulation_1}) can be expressed equivalently in the following semi-discrete block matrix form:
\begin{align}
\mathbb{V}\mathbb{M}_\mr{b}\frac{\mr{d}\mb{u}}{\mr{d}t}+\mathbb{V}\tilde{\mathbb{C}}\mb{u}+\mathbb{V}\tilde{\mathbb{M}}\mb{u}=\mathbb{V}\mathbb{S}.\label{eq_matrix_formulation_2}
\end{align}
In this matrix formulation, the vector $\mb{u}$ contains all the unknown finite element coefficients of the expansion (\ref{eq_expansion}), i.e. $\mb{u}=(\mb{u}_0,\mb{u}_1,\ldots,\mb{u}_{N-1})^\top$ and every $\mb{u}_j=(e_{xj},e_{yj},b_{xj},b_{yj},j_{cxj},j_{cyj})$ contains the respective coefficients of all six physical quantities which makes $\mb{u}$ a vector of total length $6N$. The block matrix $\mathbb{V}$ for the coefficients of the test function $V_h$ is
\begin{align}
\mathbb{V}:=\begin{pmatrix}
v_0I_6 &0 &\cdots &0 \\
0 &v_1I_6 &\cdots &0 \\
\vdots &\vdots &\ddots &\vdots \\
0 &0 &\cdots &v_{N-1}I_6\\
\end{pmatrix}\quad\in\mathbb{R}^{6N\times6N},
\end{align} 
where $I_6$ denotes the $6\times6$ identity matrix. Furthermore, the block matrices $\tilde{\mathbb{M}}$ and $\tilde{\mathbb{C}}$ are given by
\begin{align}
\tilde{\mathbb{M}}&:=\begin{pmatrix}
m_{0,0}A_1 &m_{0,1}A_1 &\cdots &m_{0,N-1}A_1 \\
m_{1,0}A_1 &m_{1,1}A_1 &\cdots &m_{1,N-1}A_1 \\
\vdots &\vdots &\ddots &\vdots \\
m_{N-1,0}A_1 &m_{N-1,1}A_1 &\cdots &m_{N-1,N-1}A_1 \\
\end{pmatrix}\quad\in\mathbb{R}^{6N\times6N},\label{eq_def_block_1}
\end{align}
\begin{align}
\tilde{\mathbb{C}}&:=\begin{pmatrix}
c_{0,0}A_2 &c_{0,1}A_2 &\cdots &c_{0,N-1}A_2 \\
c_{1,0}A_2 &c_{1,1}A_2 &\cdots &c_{1,N-1}A_2 \\
\vdots &\vdots &\ddots &\vdots \\
c_{N-1,0}A_2 &c_{N-1,1}A_2 &\cdots &c_{N-1,N-1}A_2 \\
\end{pmatrix}\quad\in\mathbb{R}^{6N\times6N},\label{eq_def_block_2}
\end{align}
respectively. The missing matrix $\mathbb{M}_\text{b}$ and the vector $\mathbb{S}$ are
\begin{align}
&\mathbb{M}_\text{b}:=\begin{pmatrix}
m_{0,0}I_6 &m_{0,1}I_6 &\cdots &m_{0,N-1}I_6 \\
m_{1,0}I_6 &m_{1,1}I_6 &\cdots &m_{1,N-1}I_6 \\
\vdots &\vdots &\ddots &\vdots \\
m_{N-1,0}I_6 &m_{N-1,1}I_6 &\cdots &m_{N-1,N-1}I_6 \\
\end{pmatrix}\quad\in\mathbb{R}^{6N\times6N},\label{eq_def_block_3}
\end{align}
and
\begin{align}
&\mathbb{S}:=\begin{pmatrix}
\int_0^L\mathbf{S}\varphi_0(z)\text{d}z \\
\vdots \\
\int_0^L\mathbf{S}\varphi_{N-1}(z)\text{d}z
\end{pmatrix}\quad\in\mathbb{R}^{6N}.\label{eq_def_righthandside}
\end{align}
Since we want (\ref{eq_matrix_formulation_2}) to be true for all $\mathbb{V}$, we finally end up the with semi-discrete system
\begin{align}
\mathbb{M}_\mr{b}\frac{\mr{d}\mb{u}}{\mr{d}t}=-\tilde{\mathbb{C}}\mb{u}-\tilde{\mathbb{M}}\mb{u}+\mathbb{S}
\end{align}
for the time evolution of all finite element coefficients $\mb{u}\in\mathbb{R}^{6N}$.

Having done the spatial discretization, the next step is to apply a time stepping scheme on above system. Here, we use a second-order Crank-Nicolson scheme \citep{Cranketal1947} which consists of applying a mid-point rule on the quantities on the right-hand side. Denoting the time step by $n$, i.e. $t_n=n\Delta t$, the fully discrete matrix formulation for advancing $\mb{u}^n\rightarrow\mb{u}^{n+1}$ then reads
\begin{align}
\left(\mathbb{M}_\mr{b}+\frac{1}{2}\Delta t\tilde{\mathbb{C}}+\frac{1}{2}\Delta t\tilde{\mathbb{M}}\right)\mb{u}^{n+1}=\left(\mathbb{M}_\mr{b}-\frac{1}{2}\Delta t\tilde{\mathbb{C}}-\frac{1}{2}\Delta t\tilde{\mathbb{M}}\right)\mb{u}^n+\frac{1}{2}\Delta t\left(\mathbb{S}^{n+1}+\mathbb{S}^n\right).\label{eq_Crank_Nicolson}
\end{align}
We immediately see that this time stepping scheme involves the inversion of the matrix on the left-hand side. However, since this must be done only once in the very beginning of a simulation, this is not considered to be a problem.

Let us now construct a basis of the finite-dimensional sub-space $\mathcal{S}_h$ with $\dim\mathcal{S}_h=N$. We do this with a family of B-splines \citep{Ratnanietal2012}, which are piecewise polynomials of degree $p$. The set of basis functions is fully determined by a sequence of $m+1$ points (or knots) $0=z_0\leq z_1\leq\ldots\leq z_m=L$ which defines a knot vector $T=(z_0,z_1,\ldots,z_m)$. For degree $p=0$ the basis functions $(\varphi_j^{p=0})_{j=0,\ldots,m-1}$ are defined by
\begin{align}
\varphi_{j}^0(z)=\begin{cases}
1\quad z\in [z_j,\,z_{j+1})\\0 \quad\text{else}.
\end{cases}\label{eq_def_Bsplines_0}
\end{align}
Higher degrees are defined by the following recursion formula:
\begin{align}
\varphi_j^p(z)=w_j^p(z)\varphi_j^{p-1}(z)+(1-w_{j+1}^p)\varphi^{p-1}_{j+1}(z), \quad\quad w_j^p(z)=\frac{z-z_j}{z_{j+p}-z_j}.\label{eq_def_Bsplines_higher}
\end{align}
If the knot vector $T$ contains $r$ repeated knots one says that this knot has multiplicity $r$. Using multiple knots at the boundaries enables the application of Dirichlet boundary conditions by enforcing all the interior splines to vanish at the boundaries and setting the first and last spline there to one. This can be achieved by using $r=p+1$ equal knots for the left and right boundary, respectively. In this case $\dim\mathcal{S}_h=m-p$. However, since we are using periodic boundary conditions, we need a periodic basis. This can be achieved by extending the knot vector over the boundaries by $p$ additional points. The result is shown in fig. \ref{fig_Bsplines_periodic} for generic degrees $p=1$ and $p=2$. In this case $\dim\mathcal{S}_h=m-2p$. Note in fig. \ref{fig_Bsplines_periodic}, that B-splines which leave the domain at one boundary come back at the other boundary which can be seen by the respective color codings.
\begin{figure}[!t]
\centering
\subfigure[]{\input{01_Figures/Bsplines_p=1.pgf}}
\subfigure[]{%% Creator: Matplotlib, PGF backend
%%
%% To include the figure in your LaTeX document, write
%%   \input{<filename>.pgf}
%%
%% Make sure the required packages are loaded in your preamble
%%   \usepackage{pgf}
%%
%% Figures using additional raster images can only be included by \input if
%% they are in the same directory as the main LaTeX file. For loading figures
%% from other directories you can use the `import` package
%%   \usepackage{import}
%% and then include the figures with
%%   \import{<path to file>}{<filename>.pgf}
%%
%% Matplotlib used the following preamble
%%   \usepackage{fontspec}
%%   \setmainfont{DejaVu Serif}
%%   \setsansfont{DejaVu Sans}
%%   \setmonofont{DejaVu Sans Mono}
%%
\begingroup%
\makeatletter%
\begin{pgfpicture}%
\pgfpathrectangle{\pgfpointorigin}{\pgfqpoint{3.198427in}{2.331214in}}%
\pgfusepath{use as bounding box, clip}%
\begin{pgfscope}%
\pgfsetbuttcap%
\pgfsetmiterjoin%
\definecolor{currentfill}{rgb}{1.000000,1.000000,1.000000}%
\pgfsetfillcolor{currentfill}%
\pgfsetlinewidth{0.000000pt}%
\definecolor{currentstroke}{rgb}{1.000000,1.000000,1.000000}%
\pgfsetstrokecolor{currentstroke}%
\pgfsetdash{}{0pt}%
\pgfpathmoveto{\pgfqpoint{0.000000in}{0.000000in}}%
\pgfpathlineto{\pgfqpoint{3.198427in}{0.000000in}}%
\pgfpathlineto{\pgfqpoint{3.198427in}{2.331214in}}%
\pgfpathlineto{\pgfqpoint{0.000000in}{2.331214in}}%
\pgfpathclose%
\pgfusepath{fill}%
\end{pgfscope}%
\begin{pgfscope}%
\pgfsetbuttcap%
\pgfsetmiterjoin%
\definecolor{currentfill}{rgb}{1.000000,1.000000,1.000000}%
\pgfsetfillcolor{currentfill}%
\pgfsetlinewidth{0.000000pt}%
\definecolor{currentstroke}{rgb}{0.000000,0.000000,0.000000}%
\pgfsetstrokecolor{currentstroke}%
\pgfsetstrokeopacity{0.000000}%
\pgfsetdash{}{0pt}%
\pgfpathmoveto{\pgfqpoint{0.374692in}{0.521603in}}%
\pgfpathlineto{\pgfqpoint{3.009692in}{0.521603in}}%
\pgfpathlineto{\pgfqpoint{3.009692in}{2.182603in}}%
\pgfpathlineto{\pgfqpoint{0.374692in}{2.182603in}}%
\pgfpathclose%
\pgfusepath{fill}%
\end{pgfscope}%
\begin{pgfscope}%
\pgfsetbuttcap%
\pgfsetroundjoin%
\definecolor{currentfill}{rgb}{0.000000,0.000000,0.000000}%
\pgfsetfillcolor{currentfill}%
\pgfsetlinewidth{0.803000pt}%
\definecolor{currentstroke}{rgb}{0.000000,0.000000,0.000000}%
\pgfsetstrokecolor{currentstroke}%
\pgfsetdash{}{0pt}%
\pgfsys@defobject{currentmarker}{\pgfqpoint{0.000000in}{-0.048611in}}{\pgfqpoint{0.000000in}{0.000000in}}{%
\pgfpathmoveto{\pgfqpoint{0.000000in}{0.000000in}}%
\pgfpathlineto{\pgfqpoint{0.000000in}{-0.048611in}}%
\pgfusepath{stroke,fill}%
}%
\begin{pgfscope}%
\pgfsys@transformshift{0.374692in}{0.521603in}%
\pgfsys@useobject{currentmarker}{}%
\end{pgfscope}%
\end{pgfscope}%
\begin{pgfscope}%
\pgftext[x=0.374692in,y=0.424381in,,top]{\rmfamily\fontsize{10.000000}{12.000000}\selectfont \(\displaystyle 0.0\)}%
\end{pgfscope}%
\begin{pgfscope}%
\pgfsetbuttcap%
\pgfsetroundjoin%
\definecolor{currentfill}{rgb}{0.000000,0.000000,0.000000}%
\pgfsetfillcolor{currentfill}%
\pgfsetlinewidth{0.803000pt}%
\definecolor{currentstroke}{rgb}{0.000000,0.000000,0.000000}%
\pgfsetstrokecolor{currentstroke}%
\pgfsetdash{}{0pt}%
\pgfsys@defobject{currentmarker}{\pgfqpoint{0.000000in}{-0.048611in}}{\pgfqpoint{0.000000in}{0.000000in}}{%
\pgfpathmoveto{\pgfqpoint{0.000000in}{0.000000in}}%
\pgfpathlineto{\pgfqpoint{0.000000in}{-0.048611in}}%
\pgfusepath{stroke,fill}%
}%
\begin{pgfscope}%
\pgfsys@transformshift{0.901692in}{0.521603in}%
\pgfsys@useobject{currentmarker}{}%
\end{pgfscope}%
\end{pgfscope}%
\begin{pgfscope}%
\pgftext[x=0.901692in,y=0.424381in,,top]{\rmfamily\fontsize{10.000000}{12.000000}\selectfont \(\displaystyle 0.2\)}%
\end{pgfscope}%
\begin{pgfscope}%
\pgfsetbuttcap%
\pgfsetroundjoin%
\definecolor{currentfill}{rgb}{0.000000,0.000000,0.000000}%
\pgfsetfillcolor{currentfill}%
\pgfsetlinewidth{0.803000pt}%
\definecolor{currentstroke}{rgb}{0.000000,0.000000,0.000000}%
\pgfsetstrokecolor{currentstroke}%
\pgfsetdash{}{0pt}%
\pgfsys@defobject{currentmarker}{\pgfqpoint{0.000000in}{-0.048611in}}{\pgfqpoint{0.000000in}{0.000000in}}{%
\pgfpathmoveto{\pgfqpoint{0.000000in}{0.000000in}}%
\pgfpathlineto{\pgfqpoint{0.000000in}{-0.048611in}}%
\pgfusepath{stroke,fill}%
}%
\begin{pgfscope}%
\pgfsys@transformshift{1.428692in}{0.521603in}%
\pgfsys@useobject{currentmarker}{}%
\end{pgfscope}%
\end{pgfscope}%
\begin{pgfscope}%
\pgftext[x=1.428692in,y=0.424381in,,top]{\rmfamily\fontsize{10.000000}{12.000000}\selectfont \(\displaystyle 0.4\)}%
\end{pgfscope}%
\begin{pgfscope}%
\pgfsetbuttcap%
\pgfsetroundjoin%
\definecolor{currentfill}{rgb}{0.000000,0.000000,0.000000}%
\pgfsetfillcolor{currentfill}%
\pgfsetlinewidth{0.803000pt}%
\definecolor{currentstroke}{rgb}{0.000000,0.000000,0.000000}%
\pgfsetstrokecolor{currentstroke}%
\pgfsetdash{}{0pt}%
\pgfsys@defobject{currentmarker}{\pgfqpoint{0.000000in}{-0.048611in}}{\pgfqpoint{0.000000in}{0.000000in}}{%
\pgfpathmoveto{\pgfqpoint{0.000000in}{0.000000in}}%
\pgfpathlineto{\pgfqpoint{0.000000in}{-0.048611in}}%
\pgfusepath{stroke,fill}%
}%
\begin{pgfscope}%
\pgfsys@transformshift{1.955692in}{0.521603in}%
\pgfsys@useobject{currentmarker}{}%
\end{pgfscope}%
\end{pgfscope}%
\begin{pgfscope}%
\pgftext[x=1.955692in,y=0.424381in,,top]{\rmfamily\fontsize{10.000000}{12.000000}\selectfont \(\displaystyle 0.6\)}%
\end{pgfscope}%
\begin{pgfscope}%
\pgfsetbuttcap%
\pgfsetroundjoin%
\definecolor{currentfill}{rgb}{0.000000,0.000000,0.000000}%
\pgfsetfillcolor{currentfill}%
\pgfsetlinewidth{0.803000pt}%
\definecolor{currentstroke}{rgb}{0.000000,0.000000,0.000000}%
\pgfsetstrokecolor{currentstroke}%
\pgfsetdash{}{0pt}%
\pgfsys@defobject{currentmarker}{\pgfqpoint{0.000000in}{-0.048611in}}{\pgfqpoint{0.000000in}{0.000000in}}{%
\pgfpathmoveto{\pgfqpoint{0.000000in}{0.000000in}}%
\pgfpathlineto{\pgfqpoint{0.000000in}{-0.048611in}}%
\pgfusepath{stroke,fill}%
}%
\begin{pgfscope}%
\pgfsys@transformshift{2.482692in}{0.521603in}%
\pgfsys@useobject{currentmarker}{}%
\end{pgfscope}%
\end{pgfscope}%
\begin{pgfscope}%
\pgftext[x=2.482692in,y=0.424381in,,top]{\rmfamily\fontsize{10.000000}{12.000000}\selectfont \(\displaystyle 0.8\)}%
\end{pgfscope}%
\begin{pgfscope}%
\pgfsetbuttcap%
\pgfsetroundjoin%
\definecolor{currentfill}{rgb}{0.000000,0.000000,0.000000}%
\pgfsetfillcolor{currentfill}%
\pgfsetlinewidth{0.803000pt}%
\definecolor{currentstroke}{rgb}{0.000000,0.000000,0.000000}%
\pgfsetstrokecolor{currentstroke}%
\pgfsetdash{}{0pt}%
\pgfsys@defobject{currentmarker}{\pgfqpoint{0.000000in}{-0.048611in}}{\pgfqpoint{0.000000in}{0.000000in}}{%
\pgfpathmoveto{\pgfqpoint{0.000000in}{0.000000in}}%
\pgfpathlineto{\pgfqpoint{0.000000in}{-0.048611in}}%
\pgfusepath{stroke,fill}%
}%
\begin{pgfscope}%
\pgfsys@transformshift{3.009692in}{0.521603in}%
\pgfsys@useobject{currentmarker}{}%
\end{pgfscope}%
\end{pgfscope}%
\begin{pgfscope}%
\pgftext[x=3.009692in,y=0.424381in,,top]{\rmfamily\fontsize{10.000000}{12.000000}\selectfont \(\displaystyle 1.0\)}%
\end{pgfscope}%
\begin{pgfscope}%
\pgftext[x=1.692192in,y=0.234413in,,top]{\rmfamily\fontsize{10.000000}{12.000000}\selectfont \(\displaystyle z\)}%
\end{pgfscope}%
\begin{pgfscope}%
\pgfsetbuttcap%
\pgfsetroundjoin%
\definecolor{currentfill}{rgb}{0.000000,0.000000,0.000000}%
\pgfsetfillcolor{currentfill}%
\pgfsetlinewidth{0.803000pt}%
\definecolor{currentstroke}{rgb}{0.000000,0.000000,0.000000}%
\pgfsetstrokecolor{currentstroke}%
\pgfsetdash{}{0pt}%
\pgfsys@defobject{currentmarker}{\pgfqpoint{-0.048611in}{0.000000in}}{\pgfqpoint{0.000000in}{0.000000in}}{%
\pgfpathmoveto{\pgfqpoint{0.000000in}{0.000000in}}%
\pgfpathlineto{\pgfqpoint{-0.048611in}{0.000000in}}%
\pgfusepath{stroke,fill}%
}%
\begin{pgfscope}%
\pgfsys@transformshift{0.374692in}{0.729228in}%
\pgfsys@useobject{currentmarker}{}%
\end{pgfscope}%
\end{pgfscope}%
\begin{pgfscope}%
\pgftext[x=0.100000in,y=0.676467in,left,base]{\rmfamily\fontsize{10.000000}{12.000000}\selectfont \(\displaystyle 0.0\)}%
\end{pgfscope}%
\begin{pgfscope}%
\pgfsetbuttcap%
\pgfsetroundjoin%
\definecolor{currentfill}{rgb}{0.000000,0.000000,0.000000}%
\pgfsetfillcolor{currentfill}%
\pgfsetlinewidth{0.803000pt}%
\definecolor{currentstroke}{rgb}{0.000000,0.000000,0.000000}%
\pgfsetstrokecolor{currentstroke}%
\pgfsetdash{}{0pt}%
\pgfsys@defobject{currentmarker}{\pgfqpoint{-0.048611in}{0.000000in}}{\pgfqpoint{0.000000in}{0.000000in}}{%
\pgfpathmoveto{\pgfqpoint{0.000000in}{0.000000in}}%
\pgfpathlineto{\pgfqpoint{-0.048611in}{0.000000in}}%
\pgfusepath{stroke,fill}%
}%
\begin{pgfscope}%
\pgfsys@transformshift{0.374692in}{1.144478in}%
\pgfsys@useobject{currentmarker}{}%
\end{pgfscope}%
\end{pgfscope}%
\begin{pgfscope}%
\pgftext[x=0.100000in,y=1.091717in,left,base]{\rmfamily\fontsize{10.000000}{12.000000}\selectfont \(\displaystyle 0.5\)}%
\end{pgfscope}%
\begin{pgfscope}%
\pgfsetbuttcap%
\pgfsetroundjoin%
\definecolor{currentfill}{rgb}{0.000000,0.000000,0.000000}%
\pgfsetfillcolor{currentfill}%
\pgfsetlinewidth{0.803000pt}%
\definecolor{currentstroke}{rgb}{0.000000,0.000000,0.000000}%
\pgfsetstrokecolor{currentstroke}%
\pgfsetdash{}{0pt}%
\pgfsys@defobject{currentmarker}{\pgfqpoint{-0.048611in}{0.000000in}}{\pgfqpoint{0.000000in}{0.000000in}}{%
\pgfpathmoveto{\pgfqpoint{0.000000in}{0.000000in}}%
\pgfpathlineto{\pgfqpoint{-0.048611in}{0.000000in}}%
\pgfusepath{stroke,fill}%
}%
\begin{pgfscope}%
\pgfsys@transformshift{0.374692in}{1.559728in}%
\pgfsys@useobject{currentmarker}{}%
\end{pgfscope}%
\end{pgfscope}%
\begin{pgfscope}%
\pgftext[x=0.100000in,y=1.506967in,left,base]{\rmfamily\fontsize{10.000000}{12.000000}\selectfont \(\displaystyle 1.0\)}%
\end{pgfscope}%
\begin{pgfscope}%
\pgfsetbuttcap%
\pgfsetroundjoin%
\definecolor{currentfill}{rgb}{0.000000,0.000000,0.000000}%
\pgfsetfillcolor{currentfill}%
\pgfsetlinewidth{0.803000pt}%
\definecolor{currentstroke}{rgb}{0.000000,0.000000,0.000000}%
\pgfsetstrokecolor{currentstroke}%
\pgfsetdash{}{0pt}%
\pgfsys@defobject{currentmarker}{\pgfqpoint{-0.048611in}{0.000000in}}{\pgfqpoint{0.000000in}{0.000000in}}{%
\pgfpathmoveto{\pgfqpoint{0.000000in}{0.000000in}}%
\pgfpathlineto{\pgfqpoint{-0.048611in}{0.000000in}}%
\pgfusepath{stroke,fill}%
}%
\begin{pgfscope}%
\pgfsys@transformshift{0.374692in}{1.974978in}%
\pgfsys@useobject{currentmarker}{}%
\end{pgfscope}%
\end{pgfscope}%
\begin{pgfscope}%
\pgftext[x=0.100000in,y=1.922217in,left,base]{\rmfamily\fontsize{10.000000}{12.000000}\selectfont \(\displaystyle 1.5\)}%
\end{pgfscope}%
\begin{pgfscope}%
\pgfpathrectangle{\pgfqpoint{0.374692in}{0.521603in}}{\pgfqpoint{2.635000in}{1.661000in}} %
\pgfusepath{clip}%
\pgfsetrectcap%
\pgfsetroundjoin%
\pgfsetlinewidth{1.003750pt}%
\definecolor{currentstroke}{rgb}{0.121569,0.466667,0.705882}%
\pgfsetstrokecolor{currentstroke}%
\pgfsetdash{}{0pt}%
\pgfpathmoveto{\pgfqpoint{0.374692in}{1.144478in}}%
\pgfpathlineto{\pgfqpoint{0.380015in}{1.136132in}}%
\pgfpathlineto{\pgfqpoint{0.385338in}{1.127870in}}%
\pgfpathlineto{\pgfqpoint{0.390662in}{1.119693in}}%
\pgfpathlineto{\pgfqpoint{0.395985in}{1.111601in}}%
\pgfpathlineto{\pgfqpoint{0.401308in}{1.103593in}}%
\pgfpathlineto{\pgfqpoint{0.406631in}{1.095670in}}%
\pgfpathlineto{\pgfqpoint{0.411955in}{1.087832in}}%
\pgfpathlineto{\pgfqpoint{0.417278in}{1.080079in}}%
\pgfpathlineto{\pgfqpoint{0.422601in}{1.072410in}}%
\pgfpathlineto{\pgfqpoint{0.427924in}{1.064826in}}%
\pgfpathlineto{\pgfqpoint{0.433248in}{1.057327in}}%
\pgfpathlineto{\pgfqpoint{0.438571in}{1.049913in}}%
\pgfpathlineto{\pgfqpoint{0.443894in}{1.042583in}}%
\pgfpathlineto{\pgfqpoint{0.449217in}{1.035338in}}%
\pgfpathlineto{\pgfqpoint{0.454540in}{1.028178in}}%
\pgfpathlineto{\pgfqpoint{0.459864in}{1.021102in}}%
\pgfpathlineto{\pgfqpoint{0.465187in}{1.014112in}}%
\pgfpathlineto{\pgfqpoint{0.470510in}{1.007206in}}%
\pgfpathlineto{\pgfqpoint{0.475833in}{1.000384in}}%
\pgfpathlineto{\pgfqpoint{0.481157in}{0.993648in}}%
\pgfpathlineto{\pgfqpoint{0.486480in}{0.986996in}}%
\pgfpathlineto{\pgfqpoint{0.491803in}{0.980429in}}%
\pgfpathlineto{\pgfqpoint{0.497126in}{0.973947in}}%
\pgfpathlineto{\pgfqpoint{0.502450in}{0.967549in}}%
\pgfpathlineto{\pgfqpoint{0.507773in}{0.961236in}}%
\pgfpathlineto{\pgfqpoint{0.513096in}{0.955008in}}%
\pgfpathlineto{\pgfqpoint{0.518419in}{0.948865in}}%
\pgfpathlineto{\pgfqpoint{0.523742in}{0.942806in}}%
\pgfpathlineto{\pgfqpoint{0.529066in}{0.936832in}}%
\pgfpathlineto{\pgfqpoint{0.534389in}{0.930943in}}%
\pgfpathlineto{\pgfqpoint{0.539712in}{0.925139in}}%
\pgfpathlineto{\pgfqpoint{0.545035in}{0.919419in}}%
\pgfpathlineto{\pgfqpoint{0.550359in}{0.913784in}}%
\pgfpathlineto{\pgfqpoint{0.555682in}{0.908234in}}%
\pgfpathlineto{\pgfqpoint{0.561005in}{0.902768in}}%
\pgfpathlineto{\pgfqpoint{0.566328in}{0.897387in}}%
\pgfpathlineto{\pgfqpoint{0.571652in}{0.892091in}}%
\pgfpathlineto{\pgfqpoint{0.576975in}{0.886880in}}%
\pgfpathlineto{\pgfqpoint{0.582298in}{0.881754in}}%
\pgfpathlineto{\pgfqpoint{0.587621in}{0.876712in}}%
\pgfpathlineto{\pgfqpoint{0.592944in}{0.871755in}}%
\pgfpathlineto{\pgfqpoint{0.598268in}{0.866882in}}%
\pgfpathlineto{\pgfqpoint{0.603591in}{0.862095in}}%
\pgfpathlineto{\pgfqpoint{0.608914in}{0.857392in}}%
\pgfpathlineto{\pgfqpoint{0.614237in}{0.852774in}}%
\pgfpathlineto{\pgfqpoint{0.619561in}{0.848240in}}%
\pgfpathlineto{\pgfqpoint{0.624884in}{0.843792in}}%
\pgfpathlineto{\pgfqpoint{0.630207in}{0.839428in}}%
\pgfpathlineto{\pgfqpoint{0.635530in}{0.835149in}}%
\pgfpathlineto{\pgfqpoint{0.640854in}{0.830954in}}%
\pgfpathlineto{\pgfqpoint{0.646177in}{0.826844in}}%
\pgfpathlineto{\pgfqpoint{0.651500in}{0.822820in}}%
\pgfpathlineto{\pgfqpoint{0.656823in}{0.818879in}}%
\pgfpathlineto{\pgfqpoint{0.662146in}{0.815024in}}%
\pgfpathlineto{\pgfqpoint{0.667470in}{0.811253in}}%
\pgfpathlineto{\pgfqpoint{0.672793in}{0.807567in}}%
\pgfpathlineto{\pgfqpoint{0.678116in}{0.803966in}}%
\pgfpathlineto{\pgfqpoint{0.683439in}{0.800449in}}%
\pgfpathlineto{\pgfqpoint{0.688763in}{0.797017in}}%
\pgfpathlineto{\pgfqpoint{0.694086in}{0.793670in}}%
\pgfpathlineto{\pgfqpoint{0.699409in}{0.790408in}}%
\pgfpathlineto{\pgfqpoint{0.704732in}{0.787230in}}%
\pgfpathlineto{\pgfqpoint{0.710056in}{0.784137in}}%
\pgfpathlineto{\pgfqpoint{0.715379in}{0.781129in}}%
\pgfpathlineto{\pgfqpoint{0.720702in}{0.778206in}}%
\pgfpathlineto{\pgfqpoint{0.726025in}{0.775367in}}%
\pgfpathlineto{\pgfqpoint{0.731349in}{0.772613in}}%
\pgfpathlineto{\pgfqpoint{0.736672in}{0.769944in}}%
\pgfpathlineto{\pgfqpoint{0.741995in}{0.767360in}}%
\pgfpathlineto{\pgfqpoint{0.747318in}{0.764860in}}%
\pgfpathlineto{\pgfqpoint{0.752641in}{0.762445in}}%
\pgfpathlineto{\pgfqpoint{0.757965in}{0.760115in}}%
\pgfpathlineto{\pgfqpoint{0.763288in}{0.757869in}}%
\pgfpathlineto{\pgfqpoint{0.768611in}{0.755708in}}%
\pgfpathlineto{\pgfqpoint{0.773934in}{0.753632in}}%
\pgfpathlineto{\pgfqpoint{0.779258in}{0.751641in}}%
\pgfpathlineto{\pgfqpoint{0.784581in}{0.749735in}}%
\pgfpathlineto{\pgfqpoint{0.789904in}{0.747913in}}%
\pgfpathlineto{\pgfqpoint{0.795227in}{0.746176in}}%
\pgfpathlineto{\pgfqpoint{0.800551in}{0.744523in}}%
\pgfpathlineto{\pgfqpoint{0.805874in}{0.742956in}}%
\pgfpathlineto{\pgfqpoint{0.811197in}{0.741473in}}%
\pgfpathlineto{\pgfqpoint{0.816520in}{0.740075in}}%
\pgfpathlineto{\pgfqpoint{0.821843in}{0.738761in}}%
\pgfpathlineto{\pgfqpoint{0.827167in}{0.737532in}}%
\pgfpathlineto{\pgfqpoint{0.832490in}{0.736389in}}%
\pgfpathlineto{\pgfqpoint{0.837813in}{0.735329in}}%
\pgfpathlineto{\pgfqpoint{0.843136in}{0.734355in}}%
\pgfpathlineto{\pgfqpoint{0.848460in}{0.733465in}}%
\pgfpathlineto{\pgfqpoint{0.853783in}{0.732660in}}%
\pgfpathlineto{\pgfqpoint{0.859106in}{0.731940in}}%
\pgfpathlineto{\pgfqpoint{0.864429in}{0.731304in}}%
\pgfpathlineto{\pgfqpoint{0.869753in}{0.730754in}}%
\pgfpathlineto{\pgfqpoint{0.875076in}{0.730288in}}%
\pgfpathlineto{\pgfqpoint{0.880399in}{0.729906in}}%
\pgfpathlineto{\pgfqpoint{0.885722in}{0.729610in}}%
\pgfpathlineto{\pgfqpoint{0.891045in}{0.729398in}}%
\pgfpathlineto{\pgfqpoint{0.896369in}{0.729271in}}%
\pgfpathlineto{\pgfqpoint{0.901692in}{0.729228in}}%
\pgfusepath{stroke}%
\end{pgfscope}%
\begin{pgfscope}%
\pgfpathrectangle{\pgfqpoint{0.374692in}{0.521603in}}{\pgfqpoint{2.635000in}{1.661000in}} %
\pgfusepath{clip}%
\pgfsetrectcap%
\pgfsetroundjoin%
\pgfsetlinewidth{1.003750pt}%
\definecolor{currentstroke}{rgb}{1.000000,0.498039,0.054902}%
\pgfsetstrokecolor{currentstroke}%
\pgfsetdash{}{0pt}%
\pgfpathmoveto{\pgfqpoint{0.374692in}{1.144478in}}%
\pgfpathlineto{\pgfqpoint{0.380015in}{1.152782in}}%
\pgfpathlineto{\pgfqpoint{0.385338in}{1.160917in}}%
\pgfpathlineto{\pgfqpoint{0.390662in}{1.168882in}}%
\pgfpathlineto{\pgfqpoint{0.395985in}{1.176678in}}%
\pgfpathlineto{\pgfqpoint{0.401308in}{1.184304in}}%
\pgfpathlineto{\pgfqpoint{0.406631in}{1.191761in}}%
\pgfpathlineto{\pgfqpoint{0.411955in}{1.199048in}}%
\pgfpathlineto{\pgfqpoint{0.417278in}{1.206166in}}%
\pgfpathlineto{\pgfqpoint{0.422601in}{1.213115in}}%
\pgfpathlineto{\pgfqpoint{0.427924in}{1.219894in}}%
\pgfpathlineto{\pgfqpoint{0.433248in}{1.226503in}}%
\pgfpathlineto{\pgfqpoint{0.438571in}{1.232943in}}%
\pgfpathlineto{\pgfqpoint{0.443894in}{1.239213in}}%
\pgfpathlineto{\pgfqpoint{0.449217in}{1.245314in}}%
\pgfpathlineto{\pgfqpoint{0.454540in}{1.251246in}}%
\pgfpathlineto{\pgfqpoint{0.459864in}{1.257008in}}%
\pgfpathlineto{\pgfqpoint{0.465187in}{1.262601in}}%
\pgfpathlineto{\pgfqpoint{0.470510in}{1.268024in}}%
\pgfpathlineto{\pgfqpoint{0.475833in}{1.273277in}}%
\pgfpathlineto{\pgfqpoint{0.481157in}{1.278362in}}%
\pgfpathlineto{\pgfqpoint{0.486480in}{1.283276in}}%
\pgfpathlineto{\pgfqpoint{0.491803in}{1.288022in}}%
\pgfpathlineto{\pgfqpoint{0.497126in}{1.292597in}}%
\pgfpathlineto{\pgfqpoint{0.502450in}{1.297004in}}%
\pgfpathlineto{\pgfqpoint{0.507773in}{1.301240in}}%
\pgfpathlineto{\pgfqpoint{0.513096in}{1.305308in}}%
\pgfpathlineto{\pgfqpoint{0.518419in}{1.309206in}}%
\pgfpathlineto{\pgfqpoint{0.523742in}{1.312934in}}%
\pgfpathlineto{\pgfqpoint{0.529066in}{1.316493in}}%
\pgfpathlineto{\pgfqpoint{0.534389in}{1.319882in}}%
\pgfpathlineto{\pgfqpoint{0.539712in}{1.323102in}}%
\pgfpathlineto{\pgfqpoint{0.545035in}{1.326153in}}%
\pgfpathlineto{\pgfqpoint{0.550359in}{1.329034in}}%
\pgfpathlineto{\pgfqpoint{0.555682in}{1.331745in}}%
\pgfpathlineto{\pgfqpoint{0.561005in}{1.334288in}}%
\pgfpathlineto{\pgfqpoint{0.566328in}{1.336660in}}%
\pgfpathlineto{\pgfqpoint{0.571652in}{1.338863in}}%
\pgfpathlineto{\pgfqpoint{0.576975in}{1.340897in}}%
\pgfpathlineto{\pgfqpoint{0.582298in}{1.342761in}}%
\pgfpathlineto{\pgfqpoint{0.587621in}{1.344456in}}%
\pgfpathlineto{\pgfqpoint{0.592944in}{1.345981in}}%
\pgfpathlineto{\pgfqpoint{0.598268in}{1.347337in}}%
\pgfpathlineto{\pgfqpoint{0.603591in}{1.348523in}}%
\pgfpathlineto{\pgfqpoint{0.608914in}{1.349540in}}%
\pgfpathlineto{\pgfqpoint{0.614237in}{1.350387in}}%
\pgfpathlineto{\pgfqpoint{0.619561in}{1.351065in}}%
\pgfpathlineto{\pgfqpoint{0.624884in}{1.351574in}}%
\pgfpathlineto{\pgfqpoint{0.630207in}{1.351913in}}%
\pgfpathlineto{\pgfqpoint{0.635530in}{1.352082in}}%
\pgfpathlineto{\pgfqpoint{0.640854in}{1.352082in}}%
\pgfpathlineto{\pgfqpoint{0.646177in}{1.351913in}}%
\pgfpathlineto{\pgfqpoint{0.651500in}{1.351574in}}%
\pgfpathlineto{\pgfqpoint{0.656823in}{1.351065in}}%
\pgfpathlineto{\pgfqpoint{0.662146in}{1.350387in}}%
\pgfpathlineto{\pgfqpoint{0.667470in}{1.349540in}}%
\pgfpathlineto{\pgfqpoint{0.672793in}{1.348523in}}%
\pgfpathlineto{\pgfqpoint{0.678116in}{1.347337in}}%
\pgfpathlineto{\pgfqpoint{0.683439in}{1.345981in}}%
\pgfpathlineto{\pgfqpoint{0.688763in}{1.344456in}}%
\pgfpathlineto{\pgfqpoint{0.694086in}{1.342761in}}%
\pgfpathlineto{\pgfqpoint{0.699409in}{1.340897in}}%
\pgfpathlineto{\pgfqpoint{0.704732in}{1.338863in}}%
\pgfpathlineto{\pgfqpoint{0.710056in}{1.336660in}}%
\pgfpathlineto{\pgfqpoint{0.715379in}{1.334288in}}%
\pgfpathlineto{\pgfqpoint{0.720702in}{1.331745in}}%
\pgfpathlineto{\pgfqpoint{0.726025in}{1.329034in}}%
\pgfpathlineto{\pgfqpoint{0.731349in}{1.326153in}}%
\pgfpathlineto{\pgfqpoint{0.736672in}{1.323102in}}%
\pgfpathlineto{\pgfqpoint{0.741995in}{1.319882in}}%
\pgfpathlineto{\pgfqpoint{0.747318in}{1.316493in}}%
\pgfpathlineto{\pgfqpoint{0.752641in}{1.312934in}}%
\pgfpathlineto{\pgfqpoint{0.757965in}{1.309206in}}%
\pgfpathlineto{\pgfqpoint{0.763288in}{1.305308in}}%
\pgfpathlineto{\pgfqpoint{0.768611in}{1.301240in}}%
\pgfpathlineto{\pgfqpoint{0.773934in}{1.297004in}}%
\pgfpathlineto{\pgfqpoint{0.779258in}{1.292597in}}%
\pgfpathlineto{\pgfqpoint{0.784581in}{1.288022in}}%
\pgfpathlineto{\pgfqpoint{0.789904in}{1.283276in}}%
\pgfpathlineto{\pgfqpoint{0.795227in}{1.278362in}}%
\pgfpathlineto{\pgfqpoint{0.800551in}{1.273277in}}%
\pgfpathlineto{\pgfqpoint{0.805874in}{1.268024in}}%
\pgfpathlineto{\pgfqpoint{0.811197in}{1.262601in}}%
\pgfpathlineto{\pgfqpoint{0.816520in}{1.257008in}}%
\pgfpathlineto{\pgfqpoint{0.821843in}{1.251246in}}%
\pgfpathlineto{\pgfqpoint{0.827167in}{1.245314in}}%
\pgfpathlineto{\pgfqpoint{0.832490in}{1.239213in}}%
\pgfpathlineto{\pgfqpoint{0.837813in}{1.232943in}}%
\pgfpathlineto{\pgfqpoint{0.843136in}{1.226503in}}%
\pgfpathlineto{\pgfqpoint{0.848460in}{1.219894in}}%
\pgfpathlineto{\pgfqpoint{0.853783in}{1.213115in}}%
\pgfpathlineto{\pgfqpoint{0.859106in}{1.206166in}}%
\pgfpathlineto{\pgfqpoint{0.864429in}{1.199048in}}%
\pgfpathlineto{\pgfqpoint{0.869753in}{1.191761in}}%
\pgfpathlineto{\pgfqpoint{0.875076in}{1.184304in}}%
\pgfpathlineto{\pgfqpoint{0.880399in}{1.176678in}}%
\pgfpathlineto{\pgfqpoint{0.885722in}{1.168882in}}%
\pgfpathlineto{\pgfqpoint{0.891045in}{1.160917in}}%
\pgfpathlineto{\pgfqpoint{0.896369in}{1.152782in}}%
\pgfpathlineto{\pgfqpoint{0.901692in}{1.144478in}}%
\pgfusepath{stroke}%
\end{pgfscope}%
\begin{pgfscope}%
\pgfpathrectangle{\pgfqpoint{0.374692in}{0.521603in}}{\pgfqpoint{2.635000in}{1.661000in}} %
\pgfusepath{clip}%
\pgfsetrectcap%
\pgfsetroundjoin%
\pgfsetlinewidth{1.003750pt}%
\definecolor{currentstroke}{rgb}{0.172549,0.627451,0.172549}%
\pgfsetstrokecolor{currentstroke}%
\pgfsetdash{}{0pt}%
\pgfpathmoveto{\pgfqpoint{0.374692in}{0.729228in}}%
\pgfpathlineto{\pgfqpoint{0.380015in}{0.729271in}}%
\pgfpathlineto{\pgfqpoint{0.385338in}{0.729398in}}%
\pgfpathlineto{\pgfqpoint{0.390662in}{0.729610in}}%
\pgfpathlineto{\pgfqpoint{0.395985in}{0.729906in}}%
\pgfpathlineto{\pgfqpoint{0.401308in}{0.730288in}}%
\pgfpathlineto{\pgfqpoint{0.406631in}{0.730754in}}%
\pgfpathlineto{\pgfqpoint{0.411955in}{0.731304in}}%
\pgfpathlineto{\pgfqpoint{0.417278in}{0.731940in}}%
\pgfpathlineto{\pgfqpoint{0.422601in}{0.732660in}}%
\pgfpathlineto{\pgfqpoint{0.427924in}{0.733465in}}%
\pgfpathlineto{\pgfqpoint{0.433248in}{0.734355in}}%
\pgfpathlineto{\pgfqpoint{0.438571in}{0.735329in}}%
\pgfpathlineto{\pgfqpoint{0.443894in}{0.736389in}}%
\pgfpathlineto{\pgfqpoint{0.449217in}{0.737532in}}%
\pgfpathlineto{\pgfqpoint{0.454540in}{0.738761in}}%
\pgfpathlineto{\pgfqpoint{0.459864in}{0.740075in}}%
\pgfpathlineto{\pgfqpoint{0.465187in}{0.741473in}}%
\pgfpathlineto{\pgfqpoint{0.470510in}{0.742956in}}%
\pgfpathlineto{\pgfqpoint{0.475833in}{0.744523in}}%
\pgfpathlineto{\pgfqpoint{0.481157in}{0.746176in}}%
\pgfpathlineto{\pgfqpoint{0.486480in}{0.747913in}}%
\pgfpathlineto{\pgfqpoint{0.491803in}{0.749735in}}%
\pgfpathlineto{\pgfqpoint{0.497126in}{0.751641in}}%
\pgfpathlineto{\pgfqpoint{0.502450in}{0.753632in}}%
\pgfpathlineto{\pgfqpoint{0.507773in}{0.755708in}}%
\pgfpathlineto{\pgfqpoint{0.513096in}{0.757869in}}%
\pgfpathlineto{\pgfqpoint{0.518419in}{0.760115in}}%
\pgfpathlineto{\pgfqpoint{0.523742in}{0.762445in}}%
\pgfpathlineto{\pgfqpoint{0.529066in}{0.764860in}}%
\pgfpathlineto{\pgfqpoint{0.534389in}{0.767360in}}%
\pgfpathlineto{\pgfqpoint{0.539712in}{0.769944in}}%
\pgfpathlineto{\pgfqpoint{0.545035in}{0.772613in}}%
\pgfpathlineto{\pgfqpoint{0.550359in}{0.775367in}}%
\pgfpathlineto{\pgfqpoint{0.555682in}{0.778206in}}%
\pgfpathlineto{\pgfqpoint{0.561005in}{0.781129in}}%
\pgfpathlineto{\pgfqpoint{0.566328in}{0.784137in}}%
\pgfpathlineto{\pgfqpoint{0.571652in}{0.787230in}}%
\pgfpathlineto{\pgfqpoint{0.576975in}{0.790408in}}%
\pgfpathlineto{\pgfqpoint{0.582298in}{0.793670in}}%
\pgfpathlineto{\pgfqpoint{0.587621in}{0.797017in}}%
\pgfpathlineto{\pgfqpoint{0.592944in}{0.800449in}}%
\pgfpathlineto{\pgfqpoint{0.598268in}{0.803966in}}%
\pgfpathlineto{\pgfqpoint{0.603591in}{0.807567in}}%
\pgfpathlineto{\pgfqpoint{0.608914in}{0.811253in}}%
\pgfpathlineto{\pgfqpoint{0.614237in}{0.815024in}}%
\pgfpathlineto{\pgfqpoint{0.619561in}{0.818879in}}%
\pgfpathlineto{\pgfqpoint{0.624884in}{0.822820in}}%
\pgfpathlineto{\pgfqpoint{0.630207in}{0.826844in}}%
\pgfpathlineto{\pgfqpoint{0.635530in}{0.830954in}}%
\pgfpathlineto{\pgfqpoint{0.640854in}{0.835149in}}%
\pgfpathlineto{\pgfqpoint{0.646177in}{0.839428in}}%
\pgfpathlineto{\pgfqpoint{0.651500in}{0.843792in}}%
\pgfpathlineto{\pgfqpoint{0.656823in}{0.848240in}}%
\pgfpathlineto{\pgfqpoint{0.662146in}{0.852774in}}%
\pgfpathlineto{\pgfqpoint{0.667470in}{0.857392in}}%
\pgfpathlineto{\pgfqpoint{0.672793in}{0.862095in}}%
\pgfpathlineto{\pgfqpoint{0.678116in}{0.866882in}}%
\pgfpathlineto{\pgfqpoint{0.683439in}{0.871755in}}%
\pgfpathlineto{\pgfqpoint{0.688763in}{0.876712in}}%
\pgfpathlineto{\pgfqpoint{0.694086in}{0.881754in}}%
\pgfpathlineto{\pgfqpoint{0.699409in}{0.886880in}}%
\pgfpathlineto{\pgfqpoint{0.704732in}{0.892091in}}%
\pgfpathlineto{\pgfqpoint{0.710056in}{0.897387in}}%
\pgfpathlineto{\pgfqpoint{0.715379in}{0.902768in}}%
\pgfpathlineto{\pgfqpoint{0.720702in}{0.908234in}}%
\pgfpathlineto{\pgfqpoint{0.726025in}{0.913784in}}%
\pgfpathlineto{\pgfqpoint{0.731349in}{0.919419in}}%
\pgfpathlineto{\pgfqpoint{0.736672in}{0.925139in}}%
\pgfpathlineto{\pgfqpoint{0.741995in}{0.930943in}}%
\pgfpathlineto{\pgfqpoint{0.747318in}{0.936832in}}%
\pgfpathlineto{\pgfqpoint{0.752641in}{0.942806in}}%
\pgfpathlineto{\pgfqpoint{0.757965in}{0.948865in}}%
\pgfpathlineto{\pgfqpoint{0.763288in}{0.955008in}}%
\pgfpathlineto{\pgfqpoint{0.768611in}{0.961236in}}%
\pgfpathlineto{\pgfqpoint{0.773934in}{0.967549in}}%
\pgfpathlineto{\pgfqpoint{0.779258in}{0.973947in}}%
\pgfpathlineto{\pgfqpoint{0.784581in}{0.980429in}}%
\pgfpathlineto{\pgfqpoint{0.789904in}{0.986996in}}%
\pgfpathlineto{\pgfqpoint{0.795227in}{0.993648in}}%
\pgfpathlineto{\pgfqpoint{0.800551in}{1.000384in}}%
\pgfpathlineto{\pgfqpoint{0.805874in}{1.007206in}}%
\pgfpathlineto{\pgfqpoint{0.811197in}{1.014112in}}%
\pgfpathlineto{\pgfqpoint{0.816520in}{1.021102in}}%
\pgfpathlineto{\pgfqpoint{0.821843in}{1.028178in}}%
\pgfpathlineto{\pgfqpoint{0.827167in}{1.035338in}}%
\pgfpathlineto{\pgfqpoint{0.832490in}{1.042583in}}%
\pgfpathlineto{\pgfqpoint{0.837813in}{1.049913in}}%
\pgfpathlineto{\pgfqpoint{0.843136in}{1.057327in}}%
\pgfpathlineto{\pgfqpoint{0.848460in}{1.064826in}}%
\pgfpathlineto{\pgfqpoint{0.853783in}{1.072410in}}%
\pgfpathlineto{\pgfqpoint{0.859106in}{1.080079in}}%
\pgfpathlineto{\pgfqpoint{0.864429in}{1.087832in}}%
\pgfpathlineto{\pgfqpoint{0.869753in}{1.095670in}}%
\pgfpathlineto{\pgfqpoint{0.875076in}{1.103593in}}%
\pgfpathlineto{\pgfqpoint{0.880399in}{1.111601in}}%
\pgfpathlineto{\pgfqpoint{0.885722in}{1.119693in}}%
\pgfpathlineto{\pgfqpoint{0.891045in}{1.127870in}}%
\pgfpathlineto{\pgfqpoint{0.896369in}{1.136132in}}%
\pgfpathlineto{\pgfqpoint{0.901692in}{1.144478in}}%
\pgfusepath{stroke}%
\end{pgfscope}%
\begin{pgfscope}%
\pgfpathrectangle{\pgfqpoint{0.374692in}{0.521603in}}{\pgfqpoint{2.635000in}{1.661000in}} %
\pgfusepath{clip}%
\pgfsetrectcap%
\pgfsetroundjoin%
\pgfsetlinewidth{1.003750pt}%
\definecolor{currentstroke}{rgb}{1.000000,0.498039,0.054902}%
\pgfsetstrokecolor{currentstroke}%
\pgfsetdash{}{0pt}%
\pgfpathmoveto{\pgfqpoint{0.901692in}{1.144478in}}%
\pgfpathlineto{\pgfqpoint{0.907015in}{1.136132in}}%
\pgfpathlineto{\pgfqpoint{0.912338in}{1.127870in}}%
\pgfpathlineto{\pgfqpoint{0.917662in}{1.119693in}}%
\pgfpathlineto{\pgfqpoint{0.922985in}{1.111601in}}%
\pgfpathlineto{\pgfqpoint{0.928308in}{1.103593in}}%
\pgfpathlineto{\pgfqpoint{0.933631in}{1.095670in}}%
\pgfpathlineto{\pgfqpoint{0.938955in}{1.087832in}}%
\pgfpathlineto{\pgfqpoint{0.944278in}{1.080079in}}%
\pgfpathlineto{\pgfqpoint{0.949601in}{1.072410in}}%
\pgfpathlineto{\pgfqpoint{0.954924in}{1.064826in}}%
\pgfpathlineto{\pgfqpoint{0.960248in}{1.057327in}}%
\pgfpathlineto{\pgfqpoint{0.965571in}{1.049913in}}%
\pgfpathlineto{\pgfqpoint{0.970894in}{1.042583in}}%
\pgfpathlineto{\pgfqpoint{0.976217in}{1.035338in}}%
\pgfpathlineto{\pgfqpoint{0.981540in}{1.028178in}}%
\pgfpathlineto{\pgfqpoint{0.986864in}{1.021102in}}%
\pgfpathlineto{\pgfqpoint{0.992187in}{1.014112in}}%
\pgfpathlineto{\pgfqpoint{0.997510in}{1.007206in}}%
\pgfpathlineto{\pgfqpoint{1.002833in}{1.000384in}}%
\pgfpathlineto{\pgfqpoint{1.008157in}{0.993648in}}%
\pgfpathlineto{\pgfqpoint{1.013480in}{0.986996in}}%
\pgfpathlineto{\pgfqpoint{1.018803in}{0.980429in}}%
\pgfpathlineto{\pgfqpoint{1.024126in}{0.973947in}}%
\pgfpathlineto{\pgfqpoint{1.029450in}{0.967549in}}%
\pgfpathlineto{\pgfqpoint{1.034773in}{0.961236in}}%
\pgfpathlineto{\pgfqpoint{1.040096in}{0.955008in}}%
\pgfpathlineto{\pgfqpoint{1.045419in}{0.948865in}}%
\pgfpathlineto{\pgfqpoint{1.050742in}{0.942806in}}%
\pgfpathlineto{\pgfqpoint{1.056066in}{0.936832in}}%
\pgfpathlineto{\pgfqpoint{1.061389in}{0.930943in}}%
\pgfpathlineto{\pgfqpoint{1.066712in}{0.925139in}}%
\pgfpathlineto{\pgfqpoint{1.072035in}{0.919419in}}%
\pgfpathlineto{\pgfqpoint{1.077359in}{0.913784in}}%
\pgfpathlineto{\pgfqpoint{1.082682in}{0.908234in}}%
\pgfpathlineto{\pgfqpoint{1.088005in}{0.902768in}}%
\pgfpathlineto{\pgfqpoint{1.093328in}{0.897387in}}%
\pgfpathlineto{\pgfqpoint{1.098652in}{0.892091in}}%
\pgfpathlineto{\pgfqpoint{1.103975in}{0.886880in}}%
\pgfpathlineto{\pgfqpoint{1.109298in}{0.881754in}}%
\pgfpathlineto{\pgfqpoint{1.114621in}{0.876712in}}%
\pgfpathlineto{\pgfqpoint{1.119944in}{0.871755in}}%
\pgfpathlineto{\pgfqpoint{1.125268in}{0.866882in}}%
\pgfpathlineto{\pgfqpoint{1.130591in}{0.862095in}}%
\pgfpathlineto{\pgfqpoint{1.135914in}{0.857392in}}%
\pgfpathlineto{\pgfqpoint{1.141237in}{0.852774in}}%
\pgfpathlineto{\pgfqpoint{1.146561in}{0.848240in}}%
\pgfpathlineto{\pgfqpoint{1.151884in}{0.843792in}}%
\pgfpathlineto{\pgfqpoint{1.157207in}{0.839428in}}%
\pgfpathlineto{\pgfqpoint{1.162530in}{0.835149in}}%
\pgfpathlineto{\pgfqpoint{1.167854in}{0.830954in}}%
\pgfpathlineto{\pgfqpoint{1.173177in}{0.826844in}}%
\pgfpathlineto{\pgfqpoint{1.178500in}{0.822820in}}%
\pgfpathlineto{\pgfqpoint{1.183823in}{0.818879in}}%
\pgfpathlineto{\pgfqpoint{1.189146in}{0.815024in}}%
\pgfpathlineto{\pgfqpoint{1.194470in}{0.811253in}}%
\pgfpathlineto{\pgfqpoint{1.199793in}{0.807567in}}%
\pgfpathlineto{\pgfqpoint{1.205116in}{0.803966in}}%
\pgfpathlineto{\pgfqpoint{1.210439in}{0.800449in}}%
\pgfpathlineto{\pgfqpoint{1.215763in}{0.797017in}}%
\pgfpathlineto{\pgfqpoint{1.221086in}{0.793670in}}%
\pgfpathlineto{\pgfqpoint{1.226409in}{0.790408in}}%
\pgfpathlineto{\pgfqpoint{1.231732in}{0.787230in}}%
\pgfpathlineto{\pgfqpoint{1.237056in}{0.784137in}}%
\pgfpathlineto{\pgfqpoint{1.242379in}{0.781129in}}%
\pgfpathlineto{\pgfqpoint{1.247702in}{0.778206in}}%
\pgfpathlineto{\pgfqpoint{1.253025in}{0.775367in}}%
\pgfpathlineto{\pgfqpoint{1.258349in}{0.772613in}}%
\pgfpathlineto{\pgfqpoint{1.263672in}{0.769944in}}%
\pgfpathlineto{\pgfqpoint{1.268995in}{0.767360in}}%
\pgfpathlineto{\pgfqpoint{1.274318in}{0.764860in}}%
\pgfpathlineto{\pgfqpoint{1.279641in}{0.762445in}}%
\pgfpathlineto{\pgfqpoint{1.284965in}{0.760115in}}%
\pgfpathlineto{\pgfqpoint{1.290288in}{0.757869in}}%
\pgfpathlineto{\pgfqpoint{1.295611in}{0.755708in}}%
\pgfpathlineto{\pgfqpoint{1.300934in}{0.753632in}}%
\pgfpathlineto{\pgfqpoint{1.306258in}{0.751641in}}%
\pgfpathlineto{\pgfqpoint{1.311581in}{0.749735in}}%
\pgfpathlineto{\pgfqpoint{1.316904in}{0.747913in}}%
\pgfpathlineto{\pgfqpoint{1.322227in}{0.746176in}}%
\pgfpathlineto{\pgfqpoint{1.327551in}{0.744523in}}%
\pgfpathlineto{\pgfqpoint{1.332874in}{0.742956in}}%
\pgfpathlineto{\pgfqpoint{1.338197in}{0.741473in}}%
\pgfpathlineto{\pgfqpoint{1.343520in}{0.740075in}}%
\pgfpathlineto{\pgfqpoint{1.348843in}{0.738761in}}%
\pgfpathlineto{\pgfqpoint{1.354167in}{0.737532in}}%
\pgfpathlineto{\pgfqpoint{1.359490in}{0.736389in}}%
\pgfpathlineto{\pgfqpoint{1.364813in}{0.735329in}}%
\pgfpathlineto{\pgfqpoint{1.370136in}{0.734355in}}%
\pgfpathlineto{\pgfqpoint{1.375460in}{0.733465in}}%
\pgfpathlineto{\pgfqpoint{1.380783in}{0.732660in}}%
\pgfpathlineto{\pgfqpoint{1.386106in}{0.731940in}}%
\pgfpathlineto{\pgfqpoint{1.391429in}{0.731304in}}%
\pgfpathlineto{\pgfqpoint{1.396753in}{0.730754in}}%
\pgfpathlineto{\pgfqpoint{1.402076in}{0.730288in}}%
\pgfpathlineto{\pgfqpoint{1.407399in}{0.729906in}}%
\pgfpathlineto{\pgfqpoint{1.412722in}{0.729610in}}%
\pgfpathlineto{\pgfqpoint{1.418045in}{0.729398in}}%
\pgfpathlineto{\pgfqpoint{1.423369in}{0.729271in}}%
\pgfpathlineto{\pgfqpoint{1.428692in}{0.729228in}}%
\pgfusepath{stroke}%
\end{pgfscope}%
\begin{pgfscope}%
\pgfpathrectangle{\pgfqpoint{0.374692in}{0.521603in}}{\pgfqpoint{2.635000in}{1.661000in}} %
\pgfusepath{clip}%
\pgfsetrectcap%
\pgfsetroundjoin%
\pgfsetlinewidth{1.003750pt}%
\definecolor{currentstroke}{rgb}{0.172549,0.627451,0.172549}%
\pgfsetstrokecolor{currentstroke}%
\pgfsetdash{}{0pt}%
\pgfpathmoveto{\pgfqpoint{0.901692in}{1.144478in}}%
\pgfpathlineto{\pgfqpoint{0.907015in}{1.152782in}}%
\pgfpathlineto{\pgfqpoint{0.912338in}{1.160917in}}%
\pgfpathlineto{\pgfqpoint{0.917662in}{1.168882in}}%
\pgfpathlineto{\pgfqpoint{0.922985in}{1.176678in}}%
\pgfpathlineto{\pgfqpoint{0.928308in}{1.184304in}}%
\pgfpathlineto{\pgfqpoint{0.933631in}{1.191761in}}%
\pgfpathlineto{\pgfqpoint{0.938955in}{1.199048in}}%
\pgfpathlineto{\pgfqpoint{0.944278in}{1.206166in}}%
\pgfpathlineto{\pgfqpoint{0.949601in}{1.213115in}}%
\pgfpathlineto{\pgfqpoint{0.954924in}{1.219894in}}%
\pgfpathlineto{\pgfqpoint{0.960248in}{1.226503in}}%
\pgfpathlineto{\pgfqpoint{0.965571in}{1.232943in}}%
\pgfpathlineto{\pgfqpoint{0.970894in}{1.239213in}}%
\pgfpathlineto{\pgfqpoint{0.976217in}{1.245314in}}%
\pgfpathlineto{\pgfqpoint{0.981540in}{1.251246in}}%
\pgfpathlineto{\pgfqpoint{0.986864in}{1.257008in}}%
\pgfpathlineto{\pgfqpoint{0.992187in}{1.262601in}}%
\pgfpathlineto{\pgfqpoint{0.997510in}{1.268024in}}%
\pgfpathlineto{\pgfqpoint{1.002833in}{1.273277in}}%
\pgfpathlineto{\pgfqpoint{1.008157in}{1.278362in}}%
\pgfpathlineto{\pgfqpoint{1.013480in}{1.283276in}}%
\pgfpathlineto{\pgfqpoint{1.018803in}{1.288022in}}%
\pgfpathlineto{\pgfqpoint{1.024126in}{1.292597in}}%
\pgfpathlineto{\pgfqpoint{1.029450in}{1.297004in}}%
\pgfpathlineto{\pgfqpoint{1.034773in}{1.301240in}}%
\pgfpathlineto{\pgfqpoint{1.040096in}{1.305308in}}%
\pgfpathlineto{\pgfqpoint{1.045419in}{1.309206in}}%
\pgfpathlineto{\pgfqpoint{1.050742in}{1.312934in}}%
\pgfpathlineto{\pgfqpoint{1.056066in}{1.316493in}}%
\pgfpathlineto{\pgfqpoint{1.061389in}{1.319882in}}%
\pgfpathlineto{\pgfqpoint{1.066712in}{1.323102in}}%
\pgfpathlineto{\pgfqpoint{1.072035in}{1.326153in}}%
\pgfpathlineto{\pgfqpoint{1.077359in}{1.329034in}}%
\pgfpathlineto{\pgfqpoint{1.082682in}{1.331745in}}%
\pgfpathlineto{\pgfqpoint{1.088005in}{1.334288in}}%
\pgfpathlineto{\pgfqpoint{1.093328in}{1.336660in}}%
\pgfpathlineto{\pgfqpoint{1.098652in}{1.338863in}}%
\pgfpathlineto{\pgfqpoint{1.103975in}{1.340897in}}%
\pgfpathlineto{\pgfqpoint{1.109298in}{1.342761in}}%
\pgfpathlineto{\pgfqpoint{1.114621in}{1.344456in}}%
\pgfpathlineto{\pgfqpoint{1.119944in}{1.345981in}}%
\pgfpathlineto{\pgfqpoint{1.125268in}{1.347337in}}%
\pgfpathlineto{\pgfqpoint{1.130591in}{1.348523in}}%
\pgfpathlineto{\pgfqpoint{1.135914in}{1.349540in}}%
\pgfpathlineto{\pgfqpoint{1.141237in}{1.350387in}}%
\pgfpathlineto{\pgfqpoint{1.146561in}{1.351065in}}%
\pgfpathlineto{\pgfqpoint{1.151884in}{1.351574in}}%
\pgfpathlineto{\pgfqpoint{1.157207in}{1.351913in}}%
\pgfpathlineto{\pgfqpoint{1.162530in}{1.352082in}}%
\pgfpathlineto{\pgfqpoint{1.167854in}{1.352082in}}%
\pgfpathlineto{\pgfqpoint{1.173177in}{1.351913in}}%
\pgfpathlineto{\pgfqpoint{1.178500in}{1.351574in}}%
\pgfpathlineto{\pgfqpoint{1.183823in}{1.351065in}}%
\pgfpathlineto{\pgfqpoint{1.189146in}{1.350387in}}%
\pgfpathlineto{\pgfqpoint{1.194470in}{1.349540in}}%
\pgfpathlineto{\pgfqpoint{1.199793in}{1.348523in}}%
\pgfpathlineto{\pgfqpoint{1.205116in}{1.347337in}}%
\pgfpathlineto{\pgfqpoint{1.210439in}{1.345981in}}%
\pgfpathlineto{\pgfqpoint{1.215763in}{1.344456in}}%
\pgfpathlineto{\pgfqpoint{1.221086in}{1.342761in}}%
\pgfpathlineto{\pgfqpoint{1.226409in}{1.340897in}}%
\pgfpathlineto{\pgfqpoint{1.231732in}{1.338863in}}%
\pgfpathlineto{\pgfqpoint{1.237056in}{1.336660in}}%
\pgfpathlineto{\pgfqpoint{1.242379in}{1.334288in}}%
\pgfpathlineto{\pgfqpoint{1.247702in}{1.331745in}}%
\pgfpathlineto{\pgfqpoint{1.253025in}{1.329034in}}%
\pgfpathlineto{\pgfqpoint{1.258349in}{1.326153in}}%
\pgfpathlineto{\pgfqpoint{1.263672in}{1.323102in}}%
\pgfpathlineto{\pgfqpoint{1.268995in}{1.319882in}}%
\pgfpathlineto{\pgfqpoint{1.274318in}{1.316493in}}%
\pgfpathlineto{\pgfqpoint{1.279641in}{1.312934in}}%
\pgfpathlineto{\pgfqpoint{1.284965in}{1.309206in}}%
\pgfpathlineto{\pgfqpoint{1.290288in}{1.305308in}}%
\pgfpathlineto{\pgfqpoint{1.295611in}{1.301240in}}%
\pgfpathlineto{\pgfqpoint{1.300934in}{1.297004in}}%
\pgfpathlineto{\pgfqpoint{1.306258in}{1.292597in}}%
\pgfpathlineto{\pgfqpoint{1.311581in}{1.288022in}}%
\pgfpathlineto{\pgfqpoint{1.316904in}{1.283276in}}%
\pgfpathlineto{\pgfqpoint{1.322227in}{1.278362in}}%
\pgfpathlineto{\pgfqpoint{1.327551in}{1.273277in}}%
\pgfpathlineto{\pgfqpoint{1.332874in}{1.268024in}}%
\pgfpathlineto{\pgfqpoint{1.338197in}{1.262601in}}%
\pgfpathlineto{\pgfqpoint{1.343520in}{1.257008in}}%
\pgfpathlineto{\pgfqpoint{1.348843in}{1.251246in}}%
\pgfpathlineto{\pgfqpoint{1.354167in}{1.245314in}}%
\pgfpathlineto{\pgfqpoint{1.359490in}{1.239213in}}%
\pgfpathlineto{\pgfqpoint{1.364813in}{1.232943in}}%
\pgfpathlineto{\pgfqpoint{1.370136in}{1.226503in}}%
\pgfpathlineto{\pgfqpoint{1.375460in}{1.219894in}}%
\pgfpathlineto{\pgfqpoint{1.380783in}{1.213115in}}%
\pgfpathlineto{\pgfqpoint{1.386106in}{1.206166in}}%
\pgfpathlineto{\pgfqpoint{1.391429in}{1.199048in}}%
\pgfpathlineto{\pgfqpoint{1.396753in}{1.191761in}}%
\pgfpathlineto{\pgfqpoint{1.402076in}{1.184304in}}%
\pgfpathlineto{\pgfqpoint{1.407399in}{1.176678in}}%
\pgfpathlineto{\pgfqpoint{1.412722in}{1.168882in}}%
\pgfpathlineto{\pgfqpoint{1.418045in}{1.160917in}}%
\pgfpathlineto{\pgfqpoint{1.423369in}{1.152782in}}%
\pgfpathlineto{\pgfqpoint{1.428692in}{1.144478in}}%
\pgfusepath{stroke}%
\end{pgfscope}%
\begin{pgfscope}%
\pgfpathrectangle{\pgfqpoint{0.374692in}{0.521603in}}{\pgfqpoint{2.635000in}{1.661000in}} %
\pgfusepath{clip}%
\pgfsetrectcap%
\pgfsetroundjoin%
\pgfsetlinewidth{1.003750pt}%
\definecolor{currentstroke}{rgb}{0.839216,0.152941,0.156863}%
\pgfsetstrokecolor{currentstroke}%
\pgfsetdash{}{0pt}%
\pgfpathmoveto{\pgfqpoint{0.901692in}{0.729228in}}%
\pgfpathlineto{\pgfqpoint{0.907015in}{0.729271in}}%
\pgfpathlineto{\pgfqpoint{0.912338in}{0.729398in}}%
\pgfpathlineto{\pgfqpoint{0.917662in}{0.729610in}}%
\pgfpathlineto{\pgfqpoint{0.922985in}{0.729906in}}%
\pgfpathlineto{\pgfqpoint{0.928308in}{0.730288in}}%
\pgfpathlineto{\pgfqpoint{0.933631in}{0.730754in}}%
\pgfpathlineto{\pgfqpoint{0.938955in}{0.731304in}}%
\pgfpathlineto{\pgfqpoint{0.944278in}{0.731940in}}%
\pgfpathlineto{\pgfqpoint{0.949601in}{0.732660in}}%
\pgfpathlineto{\pgfqpoint{0.954924in}{0.733465in}}%
\pgfpathlineto{\pgfqpoint{0.960248in}{0.734355in}}%
\pgfpathlineto{\pgfqpoint{0.965571in}{0.735329in}}%
\pgfpathlineto{\pgfqpoint{0.970894in}{0.736389in}}%
\pgfpathlineto{\pgfqpoint{0.976217in}{0.737532in}}%
\pgfpathlineto{\pgfqpoint{0.981540in}{0.738761in}}%
\pgfpathlineto{\pgfqpoint{0.986864in}{0.740075in}}%
\pgfpathlineto{\pgfqpoint{0.992187in}{0.741473in}}%
\pgfpathlineto{\pgfqpoint{0.997510in}{0.742956in}}%
\pgfpathlineto{\pgfqpoint{1.002833in}{0.744523in}}%
\pgfpathlineto{\pgfqpoint{1.008157in}{0.746176in}}%
\pgfpathlineto{\pgfqpoint{1.013480in}{0.747913in}}%
\pgfpathlineto{\pgfqpoint{1.018803in}{0.749735in}}%
\pgfpathlineto{\pgfqpoint{1.024126in}{0.751641in}}%
\pgfpathlineto{\pgfqpoint{1.029450in}{0.753632in}}%
\pgfpathlineto{\pgfqpoint{1.034773in}{0.755708in}}%
\pgfpathlineto{\pgfqpoint{1.040096in}{0.757869in}}%
\pgfpathlineto{\pgfqpoint{1.045419in}{0.760115in}}%
\pgfpathlineto{\pgfqpoint{1.050742in}{0.762445in}}%
\pgfpathlineto{\pgfqpoint{1.056066in}{0.764860in}}%
\pgfpathlineto{\pgfqpoint{1.061389in}{0.767360in}}%
\pgfpathlineto{\pgfqpoint{1.066712in}{0.769944in}}%
\pgfpathlineto{\pgfqpoint{1.072035in}{0.772613in}}%
\pgfpathlineto{\pgfqpoint{1.077359in}{0.775367in}}%
\pgfpathlineto{\pgfqpoint{1.082682in}{0.778206in}}%
\pgfpathlineto{\pgfqpoint{1.088005in}{0.781129in}}%
\pgfpathlineto{\pgfqpoint{1.093328in}{0.784137in}}%
\pgfpathlineto{\pgfqpoint{1.098652in}{0.787230in}}%
\pgfpathlineto{\pgfqpoint{1.103975in}{0.790408in}}%
\pgfpathlineto{\pgfqpoint{1.109298in}{0.793670in}}%
\pgfpathlineto{\pgfqpoint{1.114621in}{0.797017in}}%
\pgfpathlineto{\pgfqpoint{1.119944in}{0.800449in}}%
\pgfpathlineto{\pgfqpoint{1.125268in}{0.803966in}}%
\pgfpathlineto{\pgfqpoint{1.130591in}{0.807567in}}%
\pgfpathlineto{\pgfqpoint{1.135914in}{0.811253in}}%
\pgfpathlineto{\pgfqpoint{1.141237in}{0.815024in}}%
\pgfpathlineto{\pgfqpoint{1.146561in}{0.818879in}}%
\pgfpathlineto{\pgfqpoint{1.151884in}{0.822820in}}%
\pgfpathlineto{\pgfqpoint{1.157207in}{0.826844in}}%
\pgfpathlineto{\pgfqpoint{1.162530in}{0.830954in}}%
\pgfpathlineto{\pgfqpoint{1.167854in}{0.835149in}}%
\pgfpathlineto{\pgfqpoint{1.173177in}{0.839428in}}%
\pgfpathlineto{\pgfqpoint{1.178500in}{0.843792in}}%
\pgfpathlineto{\pgfqpoint{1.183823in}{0.848240in}}%
\pgfpathlineto{\pgfqpoint{1.189146in}{0.852774in}}%
\pgfpathlineto{\pgfqpoint{1.194470in}{0.857392in}}%
\pgfpathlineto{\pgfqpoint{1.199793in}{0.862095in}}%
\pgfpathlineto{\pgfqpoint{1.205116in}{0.866882in}}%
\pgfpathlineto{\pgfqpoint{1.210439in}{0.871755in}}%
\pgfpathlineto{\pgfqpoint{1.215763in}{0.876712in}}%
\pgfpathlineto{\pgfqpoint{1.221086in}{0.881754in}}%
\pgfpathlineto{\pgfqpoint{1.226409in}{0.886880in}}%
\pgfpathlineto{\pgfqpoint{1.231732in}{0.892091in}}%
\pgfpathlineto{\pgfqpoint{1.237056in}{0.897387in}}%
\pgfpathlineto{\pgfqpoint{1.242379in}{0.902768in}}%
\pgfpathlineto{\pgfqpoint{1.247702in}{0.908234in}}%
\pgfpathlineto{\pgfqpoint{1.253025in}{0.913784in}}%
\pgfpathlineto{\pgfqpoint{1.258349in}{0.919419in}}%
\pgfpathlineto{\pgfqpoint{1.263672in}{0.925139in}}%
\pgfpathlineto{\pgfqpoint{1.268995in}{0.930943in}}%
\pgfpathlineto{\pgfqpoint{1.274318in}{0.936832in}}%
\pgfpathlineto{\pgfqpoint{1.279641in}{0.942806in}}%
\pgfpathlineto{\pgfqpoint{1.284965in}{0.948865in}}%
\pgfpathlineto{\pgfqpoint{1.290288in}{0.955008in}}%
\pgfpathlineto{\pgfqpoint{1.295611in}{0.961236in}}%
\pgfpathlineto{\pgfqpoint{1.300934in}{0.967549in}}%
\pgfpathlineto{\pgfqpoint{1.306258in}{0.973947in}}%
\pgfpathlineto{\pgfqpoint{1.311581in}{0.980429in}}%
\pgfpathlineto{\pgfqpoint{1.316904in}{0.986996in}}%
\pgfpathlineto{\pgfqpoint{1.322227in}{0.993648in}}%
\pgfpathlineto{\pgfqpoint{1.327551in}{1.000384in}}%
\pgfpathlineto{\pgfqpoint{1.332874in}{1.007206in}}%
\pgfpathlineto{\pgfqpoint{1.338197in}{1.014112in}}%
\pgfpathlineto{\pgfqpoint{1.343520in}{1.021102in}}%
\pgfpathlineto{\pgfqpoint{1.348843in}{1.028178in}}%
\pgfpathlineto{\pgfqpoint{1.354167in}{1.035338in}}%
\pgfpathlineto{\pgfqpoint{1.359490in}{1.042583in}}%
\pgfpathlineto{\pgfqpoint{1.364813in}{1.049913in}}%
\pgfpathlineto{\pgfqpoint{1.370136in}{1.057327in}}%
\pgfpathlineto{\pgfqpoint{1.375460in}{1.064826in}}%
\pgfpathlineto{\pgfqpoint{1.380783in}{1.072410in}}%
\pgfpathlineto{\pgfqpoint{1.386106in}{1.080079in}}%
\pgfpathlineto{\pgfqpoint{1.391429in}{1.087832in}}%
\pgfpathlineto{\pgfqpoint{1.396753in}{1.095670in}}%
\pgfpathlineto{\pgfqpoint{1.402076in}{1.103593in}}%
\pgfpathlineto{\pgfqpoint{1.407399in}{1.111601in}}%
\pgfpathlineto{\pgfqpoint{1.412722in}{1.119693in}}%
\pgfpathlineto{\pgfqpoint{1.418045in}{1.127870in}}%
\pgfpathlineto{\pgfqpoint{1.423369in}{1.136132in}}%
\pgfpathlineto{\pgfqpoint{1.428692in}{1.144478in}}%
\pgfusepath{stroke}%
\end{pgfscope}%
\begin{pgfscope}%
\pgfpathrectangle{\pgfqpoint{0.374692in}{0.521603in}}{\pgfqpoint{2.635000in}{1.661000in}} %
\pgfusepath{clip}%
\pgfsetbuttcap%
\pgfsetroundjoin%
\pgfsetlinewidth{1.505625pt}%
\definecolor{currentstroke}{rgb}{0.000000,0.000000,0.000000}%
\pgfsetstrokecolor{currentstroke}%
\pgfsetdash{{5.550000pt}{2.400000pt}}{0.000000pt}%
\pgfpathmoveto{\pgfqpoint{0.901692in}{0.729228in}}%
\pgfpathlineto{\pgfqpoint{0.901692in}{0.798437in}}%
\pgfpathlineto{\pgfqpoint{0.901692in}{0.867645in}}%
\pgfpathlineto{\pgfqpoint{0.901692in}{0.936853in}}%
\pgfpathlineto{\pgfqpoint{0.901692in}{1.006062in}}%
\pgfpathlineto{\pgfqpoint{0.901692in}{1.075270in}}%
\pgfpathlineto{\pgfqpoint{0.901692in}{1.144478in}}%
\pgfpathlineto{\pgfqpoint{0.901692in}{1.213687in}}%
\pgfpathlineto{\pgfqpoint{0.901692in}{1.282895in}}%
\pgfpathlineto{\pgfqpoint{0.901692in}{1.352103in}}%
\pgfusepath{stroke}%
\end{pgfscope}%
\begin{pgfscope}%
\pgfpathrectangle{\pgfqpoint{0.374692in}{0.521603in}}{\pgfqpoint{2.635000in}{1.661000in}} %
\pgfusepath{clip}%
\pgfsetrectcap%
\pgfsetroundjoin%
\pgfsetlinewidth{1.003750pt}%
\definecolor{currentstroke}{rgb}{0.172549,0.627451,0.172549}%
\pgfsetstrokecolor{currentstroke}%
\pgfsetdash{}{0pt}%
\pgfpathmoveto{\pgfqpoint{1.428692in}{1.144478in}}%
\pgfpathlineto{\pgfqpoint{1.434015in}{1.136132in}}%
\pgfpathlineto{\pgfqpoint{1.439338in}{1.127870in}}%
\pgfpathlineto{\pgfqpoint{1.444662in}{1.119693in}}%
\pgfpathlineto{\pgfqpoint{1.449985in}{1.111601in}}%
\pgfpathlineto{\pgfqpoint{1.455308in}{1.103593in}}%
\pgfpathlineto{\pgfqpoint{1.460631in}{1.095670in}}%
\pgfpathlineto{\pgfqpoint{1.465955in}{1.087832in}}%
\pgfpathlineto{\pgfqpoint{1.471278in}{1.080079in}}%
\pgfpathlineto{\pgfqpoint{1.476601in}{1.072410in}}%
\pgfpathlineto{\pgfqpoint{1.481924in}{1.064826in}}%
\pgfpathlineto{\pgfqpoint{1.487248in}{1.057327in}}%
\pgfpathlineto{\pgfqpoint{1.492571in}{1.049913in}}%
\pgfpathlineto{\pgfqpoint{1.497894in}{1.042583in}}%
\pgfpathlineto{\pgfqpoint{1.503217in}{1.035338in}}%
\pgfpathlineto{\pgfqpoint{1.508540in}{1.028178in}}%
\pgfpathlineto{\pgfqpoint{1.513864in}{1.021102in}}%
\pgfpathlineto{\pgfqpoint{1.519187in}{1.014112in}}%
\pgfpathlineto{\pgfqpoint{1.524510in}{1.007206in}}%
\pgfpathlineto{\pgfqpoint{1.529833in}{1.000384in}}%
\pgfpathlineto{\pgfqpoint{1.535157in}{0.993648in}}%
\pgfpathlineto{\pgfqpoint{1.540480in}{0.986996in}}%
\pgfpathlineto{\pgfqpoint{1.545803in}{0.980429in}}%
\pgfpathlineto{\pgfqpoint{1.551126in}{0.973947in}}%
\pgfpathlineto{\pgfqpoint{1.556450in}{0.967549in}}%
\pgfpathlineto{\pgfqpoint{1.561773in}{0.961236in}}%
\pgfpathlineto{\pgfqpoint{1.567096in}{0.955008in}}%
\pgfpathlineto{\pgfqpoint{1.572419in}{0.948865in}}%
\pgfpathlineto{\pgfqpoint{1.577742in}{0.942806in}}%
\pgfpathlineto{\pgfqpoint{1.583066in}{0.936832in}}%
\pgfpathlineto{\pgfqpoint{1.588389in}{0.930943in}}%
\pgfpathlineto{\pgfqpoint{1.593712in}{0.925139in}}%
\pgfpathlineto{\pgfqpoint{1.599035in}{0.919419in}}%
\pgfpathlineto{\pgfqpoint{1.604359in}{0.913784in}}%
\pgfpathlineto{\pgfqpoint{1.609682in}{0.908234in}}%
\pgfpathlineto{\pgfqpoint{1.615005in}{0.902768in}}%
\pgfpathlineto{\pgfqpoint{1.620328in}{0.897387in}}%
\pgfpathlineto{\pgfqpoint{1.625652in}{0.892091in}}%
\pgfpathlineto{\pgfqpoint{1.630975in}{0.886880in}}%
\pgfpathlineto{\pgfqpoint{1.636298in}{0.881754in}}%
\pgfpathlineto{\pgfqpoint{1.641621in}{0.876712in}}%
\pgfpathlineto{\pgfqpoint{1.646944in}{0.871755in}}%
\pgfpathlineto{\pgfqpoint{1.652268in}{0.866882in}}%
\pgfpathlineto{\pgfqpoint{1.657591in}{0.862095in}}%
\pgfpathlineto{\pgfqpoint{1.662914in}{0.857392in}}%
\pgfpathlineto{\pgfqpoint{1.668237in}{0.852774in}}%
\pgfpathlineto{\pgfqpoint{1.673561in}{0.848240in}}%
\pgfpathlineto{\pgfqpoint{1.678884in}{0.843792in}}%
\pgfpathlineto{\pgfqpoint{1.684207in}{0.839428in}}%
\pgfpathlineto{\pgfqpoint{1.689530in}{0.835149in}}%
\pgfpathlineto{\pgfqpoint{1.694854in}{0.830954in}}%
\pgfpathlineto{\pgfqpoint{1.700177in}{0.826844in}}%
\pgfpathlineto{\pgfqpoint{1.705500in}{0.822820in}}%
\pgfpathlineto{\pgfqpoint{1.710823in}{0.818879in}}%
\pgfpathlineto{\pgfqpoint{1.716146in}{0.815024in}}%
\pgfpathlineto{\pgfqpoint{1.721470in}{0.811253in}}%
\pgfpathlineto{\pgfqpoint{1.726793in}{0.807567in}}%
\pgfpathlineto{\pgfqpoint{1.732116in}{0.803966in}}%
\pgfpathlineto{\pgfqpoint{1.737439in}{0.800449in}}%
\pgfpathlineto{\pgfqpoint{1.742763in}{0.797017in}}%
\pgfpathlineto{\pgfqpoint{1.748086in}{0.793670in}}%
\pgfpathlineto{\pgfqpoint{1.753409in}{0.790408in}}%
\pgfpathlineto{\pgfqpoint{1.758732in}{0.787230in}}%
\pgfpathlineto{\pgfqpoint{1.764056in}{0.784137in}}%
\pgfpathlineto{\pgfqpoint{1.769379in}{0.781129in}}%
\pgfpathlineto{\pgfqpoint{1.774702in}{0.778206in}}%
\pgfpathlineto{\pgfqpoint{1.780025in}{0.775367in}}%
\pgfpathlineto{\pgfqpoint{1.785349in}{0.772613in}}%
\pgfpathlineto{\pgfqpoint{1.790672in}{0.769944in}}%
\pgfpathlineto{\pgfqpoint{1.795995in}{0.767360in}}%
\pgfpathlineto{\pgfqpoint{1.801318in}{0.764860in}}%
\pgfpathlineto{\pgfqpoint{1.806641in}{0.762445in}}%
\pgfpathlineto{\pgfqpoint{1.811965in}{0.760115in}}%
\pgfpathlineto{\pgfqpoint{1.817288in}{0.757869in}}%
\pgfpathlineto{\pgfqpoint{1.822611in}{0.755708in}}%
\pgfpathlineto{\pgfqpoint{1.827934in}{0.753632in}}%
\pgfpathlineto{\pgfqpoint{1.833258in}{0.751641in}}%
\pgfpathlineto{\pgfqpoint{1.838581in}{0.749735in}}%
\pgfpathlineto{\pgfqpoint{1.843904in}{0.747913in}}%
\pgfpathlineto{\pgfqpoint{1.849227in}{0.746176in}}%
\pgfpathlineto{\pgfqpoint{1.854551in}{0.744523in}}%
\pgfpathlineto{\pgfqpoint{1.859874in}{0.742956in}}%
\pgfpathlineto{\pgfqpoint{1.865197in}{0.741473in}}%
\pgfpathlineto{\pgfqpoint{1.870520in}{0.740075in}}%
\pgfpathlineto{\pgfqpoint{1.875843in}{0.738761in}}%
\pgfpathlineto{\pgfqpoint{1.881167in}{0.737532in}}%
\pgfpathlineto{\pgfqpoint{1.886490in}{0.736389in}}%
\pgfpathlineto{\pgfqpoint{1.891813in}{0.735329in}}%
\pgfpathlineto{\pgfqpoint{1.897136in}{0.734355in}}%
\pgfpathlineto{\pgfqpoint{1.902460in}{0.733465in}}%
\pgfpathlineto{\pgfqpoint{1.907783in}{0.732660in}}%
\pgfpathlineto{\pgfqpoint{1.913106in}{0.731940in}}%
\pgfpathlineto{\pgfqpoint{1.918429in}{0.731304in}}%
\pgfpathlineto{\pgfqpoint{1.923753in}{0.730754in}}%
\pgfpathlineto{\pgfqpoint{1.929076in}{0.730288in}}%
\pgfpathlineto{\pgfqpoint{1.934399in}{0.729906in}}%
\pgfpathlineto{\pgfqpoint{1.939722in}{0.729610in}}%
\pgfpathlineto{\pgfqpoint{1.945045in}{0.729398in}}%
\pgfpathlineto{\pgfqpoint{1.950369in}{0.729271in}}%
\pgfpathlineto{\pgfqpoint{1.955692in}{0.729228in}}%
\pgfusepath{stroke}%
\end{pgfscope}%
\begin{pgfscope}%
\pgfpathrectangle{\pgfqpoint{0.374692in}{0.521603in}}{\pgfqpoint{2.635000in}{1.661000in}} %
\pgfusepath{clip}%
\pgfsetrectcap%
\pgfsetroundjoin%
\pgfsetlinewidth{1.003750pt}%
\definecolor{currentstroke}{rgb}{0.839216,0.152941,0.156863}%
\pgfsetstrokecolor{currentstroke}%
\pgfsetdash{}{0pt}%
\pgfpathmoveto{\pgfqpoint{1.428692in}{1.144478in}}%
\pgfpathlineto{\pgfqpoint{1.434015in}{1.152782in}}%
\pgfpathlineto{\pgfqpoint{1.439338in}{1.160917in}}%
\pgfpathlineto{\pgfqpoint{1.444662in}{1.168882in}}%
\pgfpathlineto{\pgfqpoint{1.449985in}{1.176678in}}%
\pgfpathlineto{\pgfqpoint{1.455308in}{1.184304in}}%
\pgfpathlineto{\pgfqpoint{1.460631in}{1.191761in}}%
\pgfpathlineto{\pgfqpoint{1.465955in}{1.199048in}}%
\pgfpathlineto{\pgfqpoint{1.471278in}{1.206166in}}%
\pgfpathlineto{\pgfqpoint{1.476601in}{1.213115in}}%
\pgfpathlineto{\pgfqpoint{1.481924in}{1.219894in}}%
\pgfpathlineto{\pgfqpoint{1.487248in}{1.226503in}}%
\pgfpathlineto{\pgfqpoint{1.492571in}{1.232943in}}%
\pgfpathlineto{\pgfqpoint{1.497894in}{1.239213in}}%
\pgfpathlineto{\pgfqpoint{1.503217in}{1.245314in}}%
\pgfpathlineto{\pgfqpoint{1.508540in}{1.251246in}}%
\pgfpathlineto{\pgfqpoint{1.513864in}{1.257008in}}%
\pgfpathlineto{\pgfqpoint{1.519187in}{1.262601in}}%
\pgfpathlineto{\pgfqpoint{1.524510in}{1.268024in}}%
\pgfpathlineto{\pgfqpoint{1.529833in}{1.273277in}}%
\pgfpathlineto{\pgfqpoint{1.535157in}{1.278362in}}%
\pgfpathlineto{\pgfqpoint{1.540480in}{1.283276in}}%
\pgfpathlineto{\pgfqpoint{1.545803in}{1.288022in}}%
\pgfpathlineto{\pgfqpoint{1.551126in}{1.292597in}}%
\pgfpathlineto{\pgfqpoint{1.556450in}{1.297004in}}%
\pgfpathlineto{\pgfqpoint{1.561773in}{1.301240in}}%
\pgfpathlineto{\pgfqpoint{1.567096in}{1.305308in}}%
\pgfpathlineto{\pgfqpoint{1.572419in}{1.309206in}}%
\pgfpathlineto{\pgfqpoint{1.577742in}{1.312934in}}%
\pgfpathlineto{\pgfqpoint{1.583066in}{1.316493in}}%
\pgfpathlineto{\pgfqpoint{1.588389in}{1.319882in}}%
\pgfpathlineto{\pgfqpoint{1.593712in}{1.323102in}}%
\pgfpathlineto{\pgfqpoint{1.599035in}{1.326153in}}%
\pgfpathlineto{\pgfqpoint{1.604359in}{1.329034in}}%
\pgfpathlineto{\pgfqpoint{1.609682in}{1.331745in}}%
\pgfpathlineto{\pgfqpoint{1.615005in}{1.334288in}}%
\pgfpathlineto{\pgfqpoint{1.620328in}{1.336660in}}%
\pgfpathlineto{\pgfqpoint{1.625652in}{1.338863in}}%
\pgfpathlineto{\pgfqpoint{1.630975in}{1.340897in}}%
\pgfpathlineto{\pgfqpoint{1.636298in}{1.342761in}}%
\pgfpathlineto{\pgfqpoint{1.641621in}{1.344456in}}%
\pgfpathlineto{\pgfqpoint{1.646944in}{1.345981in}}%
\pgfpathlineto{\pgfqpoint{1.652268in}{1.347337in}}%
\pgfpathlineto{\pgfqpoint{1.657591in}{1.348523in}}%
\pgfpathlineto{\pgfqpoint{1.662914in}{1.349540in}}%
\pgfpathlineto{\pgfqpoint{1.668237in}{1.350387in}}%
\pgfpathlineto{\pgfqpoint{1.673561in}{1.351065in}}%
\pgfpathlineto{\pgfqpoint{1.678884in}{1.351574in}}%
\pgfpathlineto{\pgfqpoint{1.684207in}{1.351913in}}%
\pgfpathlineto{\pgfqpoint{1.689530in}{1.352082in}}%
\pgfpathlineto{\pgfqpoint{1.694854in}{1.352082in}}%
\pgfpathlineto{\pgfqpoint{1.700177in}{1.351913in}}%
\pgfpathlineto{\pgfqpoint{1.705500in}{1.351574in}}%
\pgfpathlineto{\pgfqpoint{1.710823in}{1.351065in}}%
\pgfpathlineto{\pgfqpoint{1.716146in}{1.350387in}}%
\pgfpathlineto{\pgfqpoint{1.721470in}{1.349540in}}%
\pgfpathlineto{\pgfqpoint{1.726793in}{1.348523in}}%
\pgfpathlineto{\pgfqpoint{1.732116in}{1.347337in}}%
\pgfpathlineto{\pgfqpoint{1.737439in}{1.345981in}}%
\pgfpathlineto{\pgfqpoint{1.742763in}{1.344456in}}%
\pgfpathlineto{\pgfqpoint{1.748086in}{1.342761in}}%
\pgfpathlineto{\pgfqpoint{1.753409in}{1.340897in}}%
\pgfpathlineto{\pgfqpoint{1.758732in}{1.338863in}}%
\pgfpathlineto{\pgfqpoint{1.764056in}{1.336660in}}%
\pgfpathlineto{\pgfqpoint{1.769379in}{1.334288in}}%
\pgfpathlineto{\pgfqpoint{1.774702in}{1.331745in}}%
\pgfpathlineto{\pgfqpoint{1.780025in}{1.329034in}}%
\pgfpathlineto{\pgfqpoint{1.785349in}{1.326153in}}%
\pgfpathlineto{\pgfqpoint{1.790672in}{1.323102in}}%
\pgfpathlineto{\pgfqpoint{1.795995in}{1.319882in}}%
\pgfpathlineto{\pgfqpoint{1.801318in}{1.316493in}}%
\pgfpathlineto{\pgfqpoint{1.806641in}{1.312934in}}%
\pgfpathlineto{\pgfqpoint{1.811965in}{1.309206in}}%
\pgfpathlineto{\pgfqpoint{1.817288in}{1.305308in}}%
\pgfpathlineto{\pgfqpoint{1.822611in}{1.301240in}}%
\pgfpathlineto{\pgfqpoint{1.827934in}{1.297004in}}%
\pgfpathlineto{\pgfqpoint{1.833258in}{1.292597in}}%
\pgfpathlineto{\pgfqpoint{1.838581in}{1.288022in}}%
\pgfpathlineto{\pgfqpoint{1.843904in}{1.283276in}}%
\pgfpathlineto{\pgfqpoint{1.849227in}{1.278362in}}%
\pgfpathlineto{\pgfqpoint{1.854551in}{1.273277in}}%
\pgfpathlineto{\pgfqpoint{1.859874in}{1.268024in}}%
\pgfpathlineto{\pgfqpoint{1.865197in}{1.262601in}}%
\pgfpathlineto{\pgfqpoint{1.870520in}{1.257008in}}%
\pgfpathlineto{\pgfqpoint{1.875843in}{1.251246in}}%
\pgfpathlineto{\pgfqpoint{1.881167in}{1.245314in}}%
\pgfpathlineto{\pgfqpoint{1.886490in}{1.239213in}}%
\pgfpathlineto{\pgfqpoint{1.891813in}{1.232943in}}%
\pgfpathlineto{\pgfqpoint{1.897136in}{1.226503in}}%
\pgfpathlineto{\pgfqpoint{1.902460in}{1.219894in}}%
\pgfpathlineto{\pgfqpoint{1.907783in}{1.213115in}}%
\pgfpathlineto{\pgfqpoint{1.913106in}{1.206166in}}%
\pgfpathlineto{\pgfqpoint{1.918429in}{1.199048in}}%
\pgfpathlineto{\pgfqpoint{1.923753in}{1.191761in}}%
\pgfpathlineto{\pgfqpoint{1.929076in}{1.184304in}}%
\pgfpathlineto{\pgfqpoint{1.934399in}{1.176678in}}%
\pgfpathlineto{\pgfqpoint{1.939722in}{1.168882in}}%
\pgfpathlineto{\pgfqpoint{1.945045in}{1.160917in}}%
\pgfpathlineto{\pgfqpoint{1.950369in}{1.152782in}}%
\pgfpathlineto{\pgfqpoint{1.955692in}{1.144478in}}%
\pgfusepath{stroke}%
\end{pgfscope}%
\begin{pgfscope}%
\pgfpathrectangle{\pgfqpoint{0.374692in}{0.521603in}}{\pgfqpoint{2.635000in}{1.661000in}} %
\pgfusepath{clip}%
\pgfsetrectcap%
\pgfsetroundjoin%
\pgfsetlinewidth{1.003750pt}%
\definecolor{currentstroke}{rgb}{0.580392,0.403922,0.741176}%
\pgfsetstrokecolor{currentstroke}%
\pgfsetdash{}{0pt}%
\pgfpathmoveto{\pgfqpoint{1.428692in}{0.729228in}}%
\pgfpathlineto{\pgfqpoint{1.434015in}{0.729271in}}%
\pgfpathlineto{\pgfqpoint{1.439338in}{0.729398in}}%
\pgfpathlineto{\pgfqpoint{1.444662in}{0.729610in}}%
\pgfpathlineto{\pgfqpoint{1.449985in}{0.729906in}}%
\pgfpathlineto{\pgfqpoint{1.455308in}{0.730288in}}%
\pgfpathlineto{\pgfqpoint{1.460631in}{0.730754in}}%
\pgfpathlineto{\pgfqpoint{1.465955in}{0.731304in}}%
\pgfpathlineto{\pgfqpoint{1.471278in}{0.731940in}}%
\pgfpathlineto{\pgfqpoint{1.476601in}{0.732660in}}%
\pgfpathlineto{\pgfqpoint{1.481924in}{0.733465in}}%
\pgfpathlineto{\pgfqpoint{1.487248in}{0.734355in}}%
\pgfpathlineto{\pgfqpoint{1.492571in}{0.735329in}}%
\pgfpathlineto{\pgfqpoint{1.497894in}{0.736389in}}%
\pgfpathlineto{\pgfqpoint{1.503217in}{0.737532in}}%
\pgfpathlineto{\pgfqpoint{1.508540in}{0.738761in}}%
\pgfpathlineto{\pgfqpoint{1.513864in}{0.740075in}}%
\pgfpathlineto{\pgfqpoint{1.519187in}{0.741473in}}%
\pgfpathlineto{\pgfqpoint{1.524510in}{0.742956in}}%
\pgfpathlineto{\pgfqpoint{1.529833in}{0.744523in}}%
\pgfpathlineto{\pgfqpoint{1.535157in}{0.746176in}}%
\pgfpathlineto{\pgfqpoint{1.540480in}{0.747913in}}%
\pgfpathlineto{\pgfqpoint{1.545803in}{0.749735in}}%
\pgfpathlineto{\pgfqpoint{1.551126in}{0.751641in}}%
\pgfpathlineto{\pgfqpoint{1.556450in}{0.753632in}}%
\pgfpathlineto{\pgfqpoint{1.561773in}{0.755708in}}%
\pgfpathlineto{\pgfqpoint{1.567096in}{0.757869in}}%
\pgfpathlineto{\pgfqpoint{1.572419in}{0.760115in}}%
\pgfpathlineto{\pgfqpoint{1.577742in}{0.762445in}}%
\pgfpathlineto{\pgfqpoint{1.583066in}{0.764860in}}%
\pgfpathlineto{\pgfqpoint{1.588389in}{0.767360in}}%
\pgfpathlineto{\pgfqpoint{1.593712in}{0.769944in}}%
\pgfpathlineto{\pgfqpoint{1.599035in}{0.772613in}}%
\pgfpathlineto{\pgfqpoint{1.604359in}{0.775367in}}%
\pgfpathlineto{\pgfqpoint{1.609682in}{0.778206in}}%
\pgfpathlineto{\pgfqpoint{1.615005in}{0.781129in}}%
\pgfpathlineto{\pgfqpoint{1.620328in}{0.784137in}}%
\pgfpathlineto{\pgfqpoint{1.625652in}{0.787230in}}%
\pgfpathlineto{\pgfqpoint{1.630975in}{0.790408in}}%
\pgfpathlineto{\pgfqpoint{1.636298in}{0.793670in}}%
\pgfpathlineto{\pgfqpoint{1.641621in}{0.797017in}}%
\pgfpathlineto{\pgfqpoint{1.646944in}{0.800449in}}%
\pgfpathlineto{\pgfqpoint{1.652268in}{0.803966in}}%
\pgfpathlineto{\pgfqpoint{1.657591in}{0.807567in}}%
\pgfpathlineto{\pgfqpoint{1.662914in}{0.811253in}}%
\pgfpathlineto{\pgfqpoint{1.668237in}{0.815024in}}%
\pgfpathlineto{\pgfqpoint{1.673561in}{0.818879in}}%
\pgfpathlineto{\pgfqpoint{1.678884in}{0.822820in}}%
\pgfpathlineto{\pgfqpoint{1.684207in}{0.826844in}}%
\pgfpathlineto{\pgfqpoint{1.689530in}{0.830954in}}%
\pgfpathlineto{\pgfqpoint{1.694854in}{0.835149in}}%
\pgfpathlineto{\pgfqpoint{1.700177in}{0.839428in}}%
\pgfpathlineto{\pgfqpoint{1.705500in}{0.843792in}}%
\pgfpathlineto{\pgfqpoint{1.710823in}{0.848240in}}%
\pgfpathlineto{\pgfqpoint{1.716146in}{0.852774in}}%
\pgfpathlineto{\pgfqpoint{1.721470in}{0.857392in}}%
\pgfpathlineto{\pgfqpoint{1.726793in}{0.862095in}}%
\pgfpathlineto{\pgfqpoint{1.732116in}{0.866882in}}%
\pgfpathlineto{\pgfqpoint{1.737439in}{0.871755in}}%
\pgfpathlineto{\pgfqpoint{1.742763in}{0.876712in}}%
\pgfpathlineto{\pgfqpoint{1.748086in}{0.881754in}}%
\pgfpathlineto{\pgfqpoint{1.753409in}{0.886880in}}%
\pgfpathlineto{\pgfqpoint{1.758732in}{0.892091in}}%
\pgfpathlineto{\pgfqpoint{1.764056in}{0.897387in}}%
\pgfpathlineto{\pgfqpoint{1.769379in}{0.902768in}}%
\pgfpathlineto{\pgfqpoint{1.774702in}{0.908234in}}%
\pgfpathlineto{\pgfqpoint{1.780025in}{0.913784in}}%
\pgfpathlineto{\pgfqpoint{1.785349in}{0.919419in}}%
\pgfpathlineto{\pgfqpoint{1.790672in}{0.925139in}}%
\pgfpathlineto{\pgfqpoint{1.795995in}{0.930943in}}%
\pgfpathlineto{\pgfqpoint{1.801318in}{0.936832in}}%
\pgfpathlineto{\pgfqpoint{1.806641in}{0.942806in}}%
\pgfpathlineto{\pgfqpoint{1.811965in}{0.948865in}}%
\pgfpathlineto{\pgfqpoint{1.817288in}{0.955008in}}%
\pgfpathlineto{\pgfqpoint{1.822611in}{0.961236in}}%
\pgfpathlineto{\pgfqpoint{1.827934in}{0.967549in}}%
\pgfpathlineto{\pgfqpoint{1.833258in}{0.973947in}}%
\pgfpathlineto{\pgfqpoint{1.838581in}{0.980429in}}%
\pgfpathlineto{\pgfqpoint{1.843904in}{0.986996in}}%
\pgfpathlineto{\pgfqpoint{1.849227in}{0.993648in}}%
\pgfpathlineto{\pgfqpoint{1.854551in}{1.000384in}}%
\pgfpathlineto{\pgfqpoint{1.859874in}{1.007206in}}%
\pgfpathlineto{\pgfqpoint{1.865197in}{1.014112in}}%
\pgfpathlineto{\pgfqpoint{1.870520in}{1.021102in}}%
\pgfpathlineto{\pgfqpoint{1.875843in}{1.028178in}}%
\pgfpathlineto{\pgfqpoint{1.881167in}{1.035338in}}%
\pgfpathlineto{\pgfqpoint{1.886490in}{1.042583in}}%
\pgfpathlineto{\pgfqpoint{1.891813in}{1.049913in}}%
\pgfpathlineto{\pgfqpoint{1.897136in}{1.057327in}}%
\pgfpathlineto{\pgfqpoint{1.902460in}{1.064826in}}%
\pgfpathlineto{\pgfqpoint{1.907783in}{1.072410in}}%
\pgfpathlineto{\pgfqpoint{1.913106in}{1.080079in}}%
\pgfpathlineto{\pgfqpoint{1.918429in}{1.087832in}}%
\pgfpathlineto{\pgfqpoint{1.923753in}{1.095670in}}%
\pgfpathlineto{\pgfqpoint{1.929076in}{1.103593in}}%
\pgfpathlineto{\pgfqpoint{1.934399in}{1.111601in}}%
\pgfpathlineto{\pgfqpoint{1.939722in}{1.119693in}}%
\pgfpathlineto{\pgfqpoint{1.945045in}{1.127870in}}%
\pgfpathlineto{\pgfqpoint{1.950369in}{1.136132in}}%
\pgfpathlineto{\pgfqpoint{1.955692in}{1.144478in}}%
\pgfusepath{stroke}%
\end{pgfscope}%
\begin{pgfscope}%
\pgfpathrectangle{\pgfqpoint{0.374692in}{0.521603in}}{\pgfqpoint{2.635000in}{1.661000in}} %
\pgfusepath{clip}%
\pgfsetbuttcap%
\pgfsetroundjoin%
\pgfsetlinewidth{1.505625pt}%
\definecolor{currentstroke}{rgb}{0.000000,0.000000,0.000000}%
\pgfsetstrokecolor{currentstroke}%
\pgfsetdash{{5.550000pt}{2.400000pt}}{0.000000pt}%
\pgfpathmoveto{\pgfqpoint{1.428692in}{0.729228in}}%
\pgfpathlineto{\pgfqpoint{1.428692in}{0.798437in}}%
\pgfpathlineto{\pgfqpoint{1.428692in}{0.867645in}}%
\pgfpathlineto{\pgfqpoint{1.428692in}{0.936853in}}%
\pgfpathlineto{\pgfqpoint{1.428692in}{1.006062in}}%
\pgfpathlineto{\pgfqpoint{1.428692in}{1.075270in}}%
\pgfpathlineto{\pgfqpoint{1.428692in}{1.144478in}}%
\pgfpathlineto{\pgfqpoint{1.428692in}{1.213687in}}%
\pgfpathlineto{\pgfqpoint{1.428692in}{1.282895in}}%
\pgfpathlineto{\pgfqpoint{1.428692in}{1.352103in}}%
\pgfusepath{stroke}%
\end{pgfscope}%
\begin{pgfscope}%
\pgfpathrectangle{\pgfqpoint{0.374692in}{0.521603in}}{\pgfqpoint{2.635000in}{1.661000in}} %
\pgfusepath{clip}%
\pgfsetrectcap%
\pgfsetroundjoin%
\pgfsetlinewidth{1.003750pt}%
\definecolor{currentstroke}{rgb}{0.839216,0.152941,0.156863}%
\pgfsetstrokecolor{currentstroke}%
\pgfsetdash{}{0pt}%
\pgfpathmoveto{\pgfqpoint{1.955692in}{1.144478in}}%
\pgfpathlineto{\pgfqpoint{1.961015in}{1.136132in}}%
\pgfpathlineto{\pgfqpoint{1.966338in}{1.127870in}}%
\pgfpathlineto{\pgfqpoint{1.971662in}{1.119693in}}%
\pgfpathlineto{\pgfqpoint{1.976985in}{1.111601in}}%
\pgfpathlineto{\pgfqpoint{1.982308in}{1.103593in}}%
\pgfpathlineto{\pgfqpoint{1.987631in}{1.095670in}}%
\pgfpathlineto{\pgfqpoint{1.992955in}{1.087832in}}%
\pgfpathlineto{\pgfqpoint{1.998278in}{1.080079in}}%
\pgfpathlineto{\pgfqpoint{2.003601in}{1.072410in}}%
\pgfpathlineto{\pgfqpoint{2.008924in}{1.064826in}}%
\pgfpathlineto{\pgfqpoint{2.014248in}{1.057327in}}%
\pgfpathlineto{\pgfqpoint{2.019571in}{1.049913in}}%
\pgfpathlineto{\pgfqpoint{2.024894in}{1.042583in}}%
\pgfpathlineto{\pgfqpoint{2.030217in}{1.035338in}}%
\pgfpathlineto{\pgfqpoint{2.035540in}{1.028178in}}%
\pgfpathlineto{\pgfqpoint{2.040864in}{1.021102in}}%
\pgfpathlineto{\pgfqpoint{2.046187in}{1.014112in}}%
\pgfpathlineto{\pgfqpoint{2.051510in}{1.007206in}}%
\pgfpathlineto{\pgfqpoint{2.056833in}{1.000384in}}%
\pgfpathlineto{\pgfqpoint{2.062157in}{0.993648in}}%
\pgfpathlineto{\pgfqpoint{2.067480in}{0.986996in}}%
\pgfpathlineto{\pgfqpoint{2.072803in}{0.980429in}}%
\pgfpathlineto{\pgfqpoint{2.078126in}{0.973947in}}%
\pgfpathlineto{\pgfqpoint{2.083450in}{0.967549in}}%
\pgfpathlineto{\pgfqpoint{2.088773in}{0.961236in}}%
\pgfpathlineto{\pgfqpoint{2.094096in}{0.955008in}}%
\pgfpathlineto{\pgfqpoint{2.099419in}{0.948865in}}%
\pgfpathlineto{\pgfqpoint{2.104742in}{0.942806in}}%
\pgfpathlineto{\pgfqpoint{2.110066in}{0.936832in}}%
\pgfpathlineto{\pgfqpoint{2.115389in}{0.930943in}}%
\pgfpathlineto{\pgfqpoint{2.120712in}{0.925139in}}%
\pgfpathlineto{\pgfqpoint{2.126035in}{0.919419in}}%
\pgfpathlineto{\pgfqpoint{2.131359in}{0.913784in}}%
\pgfpathlineto{\pgfqpoint{2.136682in}{0.908234in}}%
\pgfpathlineto{\pgfqpoint{2.142005in}{0.902768in}}%
\pgfpathlineto{\pgfqpoint{2.147328in}{0.897387in}}%
\pgfpathlineto{\pgfqpoint{2.152652in}{0.892091in}}%
\pgfpathlineto{\pgfqpoint{2.157975in}{0.886880in}}%
\pgfpathlineto{\pgfqpoint{2.163298in}{0.881754in}}%
\pgfpathlineto{\pgfqpoint{2.168621in}{0.876712in}}%
\pgfpathlineto{\pgfqpoint{2.173944in}{0.871755in}}%
\pgfpathlineto{\pgfqpoint{2.179268in}{0.866882in}}%
\pgfpathlineto{\pgfqpoint{2.184591in}{0.862095in}}%
\pgfpathlineto{\pgfqpoint{2.189914in}{0.857392in}}%
\pgfpathlineto{\pgfqpoint{2.195237in}{0.852774in}}%
\pgfpathlineto{\pgfqpoint{2.200561in}{0.848240in}}%
\pgfpathlineto{\pgfqpoint{2.205884in}{0.843792in}}%
\pgfpathlineto{\pgfqpoint{2.211207in}{0.839428in}}%
\pgfpathlineto{\pgfqpoint{2.216530in}{0.835149in}}%
\pgfpathlineto{\pgfqpoint{2.221854in}{0.830954in}}%
\pgfpathlineto{\pgfqpoint{2.227177in}{0.826844in}}%
\pgfpathlineto{\pgfqpoint{2.232500in}{0.822820in}}%
\pgfpathlineto{\pgfqpoint{2.237823in}{0.818879in}}%
\pgfpathlineto{\pgfqpoint{2.243146in}{0.815024in}}%
\pgfpathlineto{\pgfqpoint{2.248470in}{0.811253in}}%
\pgfpathlineto{\pgfqpoint{2.253793in}{0.807567in}}%
\pgfpathlineto{\pgfqpoint{2.259116in}{0.803966in}}%
\pgfpathlineto{\pgfqpoint{2.264439in}{0.800449in}}%
\pgfpathlineto{\pgfqpoint{2.269763in}{0.797017in}}%
\pgfpathlineto{\pgfqpoint{2.275086in}{0.793670in}}%
\pgfpathlineto{\pgfqpoint{2.280409in}{0.790408in}}%
\pgfpathlineto{\pgfqpoint{2.285732in}{0.787230in}}%
\pgfpathlineto{\pgfqpoint{2.291056in}{0.784137in}}%
\pgfpathlineto{\pgfqpoint{2.296379in}{0.781129in}}%
\pgfpathlineto{\pgfqpoint{2.301702in}{0.778206in}}%
\pgfpathlineto{\pgfqpoint{2.307025in}{0.775367in}}%
\pgfpathlineto{\pgfqpoint{2.312349in}{0.772613in}}%
\pgfpathlineto{\pgfqpoint{2.317672in}{0.769944in}}%
\pgfpathlineto{\pgfqpoint{2.322995in}{0.767360in}}%
\pgfpathlineto{\pgfqpoint{2.328318in}{0.764860in}}%
\pgfpathlineto{\pgfqpoint{2.333641in}{0.762445in}}%
\pgfpathlineto{\pgfqpoint{2.338965in}{0.760115in}}%
\pgfpathlineto{\pgfqpoint{2.344288in}{0.757869in}}%
\pgfpathlineto{\pgfqpoint{2.349611in}{0.755708in}}%
\pgfpathlineto{\pgfqpoint{2.354934in}{0.753632in}}%
\pgfpathlineto{\pgfqpoint{2.360258in}{0.751641in}}%
\pgfpathlineto{\pgfqpoint{2.365581in}{0.749735in}}%
\pgfpathlineto{\pgfqpoint{2.370904in}{0.747913in}}%
\pgfpathlineto{\pgfqpoint{2.376227in}{0.746176in}}%
\pgfpathlineto{\pgfqpoint{2.381551in}{0.744523in}}%
\pgfpathlineto{\pgfqpoint{2.386874in}{0.742956in}}%
\pgfpathlineto{\pgfqpoint{2.392197in}{0.741473in}}%
\pgfpathlineto{\pgfqpoint{2.397520in}{0.740075in}}%
\pgfpathlineto{\pgfqpoint{2.402843in}{0.738761in}}%
\pgfpathlineto{\pgfqpoint{2.408167in}{0.737532in}}%
\pgfpathlineto{\pgfqpoint{2.413490in}{0.736389in}}%
\pgfpathlineto{\pgfqpoint{2.418813in}{0.735329in}}%
\pgfpathlineto{\pgfqpoint{2.424136in}{0.734355in}}%
\pgfpathlineto{\pgfqpoint{2.429460in}{0.733465in}}%
\pgfpathlineto{\pgfqpoint{2.434783in}{0.732660in}}%
\pgfpathlineto{\pgfqpoint{2.440106in}{0.731940in}}%
\pgfpathlineto{\pgfqpoint{2.445429in}{0.731304in}}%
\pgfpathlineto{\pgfqpoint{2.450753in}{0.730754in}}%
\pgfpathlineto{\pgfqpoint{2.456076in}{0.730288in}}%
\pgfpathlineto{\pgfqpoint{2.461399in}{0.729906in}}%
\pgfpathlineto{\pgfqpoint{2.466722in}{0.729610in}}%
\pgfpathlineto{\pgfqpoint{2.472045in}{0.729398in}}%
\pgfpathlineto{\pgfqpoint{2.477369in}{0.729271in}}%
\pgfpathlineto{\pgfqpoint{2.482692in}{0.729228in}}%
\pgfusepath{stroke}%
\end{pgfscope}%
\begin{pgfscope}%
\pgfpathrectangle{\pgfqpoint{0.374692in}{0.521603in}}{\pgfqpoint{2.635000in}{1.661000in}} %
\pgfusepath{clip}%
\pgfsetrectcap%
\pgfsetroundjoin%
\pgfsetlinewidth{1.003750pt}%
\definecolor{currentstroke}{rgb}{0.580392,0.403922,0.741176}%
\pgfsetstrokecolor{currentstroke}%
\pgfsetdash{}{0pt}%
\pgfpathmoveto{\pgfqpoint{1.955692in}{1.144478in}}%
\pgfpathlineto{\pgfqpoint{1.961015in}{1.152782in}}%
\pgfpathlineto{\pgfqpoint{1.966338in}{1.160917in}}%
\pgfpathlineto{\pgfqpoint{1.971662in}{1.168882in}}%
\pgfpathlineto{\pgfqpoint{1.976985in}{1.176678in}}%
\pgfpathlineto{\pgfqpoint{1.982308in}{1.184304in}}%
\pgfpathlineto{\pgfqpoint{1.987631in}{1.191761in}}%
\pgfpathlineto{\pgfqpoint{1.992955in}{1.199048in}}%
\pgfpathlineto{\pgfqpoint{1.998278in}{1.206166in}}%
\pgfpathlineto{\pgfqpoint{2.003601in}{1.213115in}}%
\pgfpathlineto{\pgfqpoint{2.008924in}{1.219894in}}%
\pgfpathlineto{\pgfqpoint{2.014248in}{1.226503in}}%
\pgfpathlineto{\pgfqpoint{2.019571in}{1.232943in}}%
\pgfpathlineto{\pgfqpoint{2.024894in}{1.239213in}}%
\pgfpathlineto{\pgfqpoint{2.030217in}{1.245314in}}%
\pgfpathlineto{\pgfqpoint{2.035540in}{1.251246in}}%
\pgfpathlineto{\pgfqpoint{2.040864in}{1.257008in}}%
\pgfpathlineto{\pgfqpoint{2.046187in}{1.262601in}}%
\pgfpathlineto{\pgfqpoint{2.051510in}{1.268024in}}%
\pgfpathlineto{\pgfqpoint{2.056833in}{1.273277in}}%
\pgfpathlineto{\pgfqpoint{2.062157in}{1.278362in}}%
\pgfpathlineto{\pgfqpoint{2.067480in}{1.283276in}}%
\pgfpathlineto{\pgfqpoint{2.072803in}{1.288022in}}%
\pgfpathlineto{\pgfqpoint{2.078126in}{1.292597in}}%
\pgfpathlineto{\pgfqpoint{2.083450in}{1.297004in}}%
\pgfpathlineto{\pgfqpoint{2.088773in}{1.301240in}}%
\pgfpathlineto{\pgfqpoint{2.094096in}{1.305308in}}%
\pgfpathlineto{\pgfqpoint{2.099419in}{1.309206in}}%
\pgfpathlineto{\pgfqpoint{2.104742in}{1.312934in}}%
\pgfpathlineto{\pgfqpoint{2.110066in}{1.316493in}}%
\pgfpathlineto{\pgfqpoint{2.115389in}{1.319882in}}%
\pgfpathlineto{\pgfqpoint{2.120712in}{1.323102in}}%
\pgfpathlineto{\pgfqpoint{2.126035in}{1.326153in}}%
\pgfpathlineto{\pgfqpoint{2.131359in}{1.329034in}}%
\pgfpathlineto{\pgfqpoint{2.136682in}{1.331745in}}%
\pgfpathlineto{\pgfqpoint{2.142005in}{1.334288in}}%
\pgfpathlineto{\pgfqpoint{2.147328in}{1.336660in}}%
\pgfpathlineto{\pgfqpoint{2.152652in}{1.338863in}}%
\pgfpathlineto{\pgfqpoint{2.157975in}{1.340897in}}%
\pgfpathlineto{\pgfqpoint{2.163298in}{1.342761in}}%
\pgfpathlineto{\pgfqpoint{2.168621in}{1.344456in}}%
\pgfpathlineto{\pgfqpoint{2.173944in}{1.345981in}}%
\pgfpathlineto{\pgfqpoint{2.179268in}{1.347337in}}%
\pgfpathlineto{\pgfqpoint{2.184591in}{1.348523in}}%
\pgfpathlineto{\pgfqpoint{2.189914in}{1.349540in}}%
\pgfpathlineto{\pgfqpoint{2.195237in}{1.350387in}}%
\pgfpathlineto{\pgfqpoint{2.200561in}{1.351065in}}%
\pgfpathlineto{\pgfqpoint{2.205884in}{1.351574in}}%
\pgfpathlineto{\pgfqpoint{2.211207in}{1.351913in}}%
\pgfpathlineto{\pgfqpoint{2.216530in}{1.352082in}}%
\pgfpathlineto{\pgfqpoint{2.221854in}{1.352082in}}%
\pgfpathlineto{\pgfqpoint{2.227177in}{1.351913in}}%
\pgfpathlineto{\pgfqpoint{2.232500in}{1.351574in}}%
\pgfpathlineto{\pgfqpoint{2.237823in}{1.351065in}}%
\pgfpathlineto{\pgfqpoint{2.243146in}{1.350387in}}%
\pgfpathlineto{\pgfqpoint{2.248470in}{1.349540in}}%
\pgfpathlineto{\pgfqpoint{2.253793in}{1.348523in}}%
\pgfpathlineto{\pgfqpoint{2.259116in}{1.347337in}}%
\pgfpathlineto{\pgfqpoint{2.264439in}{1.345981in}}%
\pgfpathlineto{\pgfqpoint{2.269763in}{1.344456in}}%
\pgfpathlineto{\pgfqpoint{2.275086in}{1.342761in}}%
\pgfpathlineto{\pgfqpoint{2.280409in}{1.340897in}}%
\pgfpathlineto{\pgfqpoint{2.285732in}{1.338863in}}%
\pgfpathlineto{\pgfqpoint{2.291056in}{1.336660in}}%
\pgfpathlineto{\pgfqpoint{2.296379in}{1.334288in}}%
\pgfpathlineto{\pgfqpoint{2.301702in}{1.331745in}}%
\pgfpathlineto{\pgfqpoint{2.307025in}{1.329034in}}%
\pgfpathlineto{\pgfqpoint{2.312349in}{1.326153in}}%
\pgfpathlineto{\pgfqpoint{2.317672in}{1.323102in}}%
\pgfpathlineto{\pgfqpoint{2.322995in}{1.319882in}}%
\pgfpathlineto{\pgfqpoint{2.328318in}{1.316493in}}%
\pgfpathlineto{\pgfqpoint{2.333641in}{1.312934in}}%
\pgfpathlineto{\pgfqpoint{2.338965in}{1.309206in}}%
\pgfpathlineto{\pgfqpoint{2.344288in}{1.305308in}}%
\pgfpathlineto{\pgfqpoint{2.349611in}{1.301240in}}%
\pgfpathlineto{\pgfqpoint{2.354934in}{1.297004in}}%
\pgfpathlineto{\pgfqpoint{2.360258in}{1.292597in}}%
\pgfpathlineto{\pgfqpoint{2.365581in}{1.288022in}}%
\pgfpathlineto{\pgfqpoint{2.370904in}{1.283276in}}%
\pgfpathlineto{\pgfqpoint{2.376227in}{1.278362in}}%
\pgfpathlineto{\pgfqpoint{2.381551in}{1.273277in}}%
\pgfpathlineto{\pgfqpoint{2.386874in}{1.268024in}}%
\pgfpathlineto{\pgfqpoint{2.392197in}{1.262601in}}%
\pgfpathlineto{\pgfqpoint{2.397520in}{1.257008in}}%
\pgfpathlineto{\pgfqpoint{2.402843in}{1.251246in}}%
\pgfpathlineto{\pgfqpoint{2.408167in}{1.245314in}}%
\pgfpathlineto{\pgfqpoint{2.413490in}{1.239213in}}%
\pgfpathlineto{\pgfqpoint{2.418813in}{1.232943in}}%
\pgfpathlineto{\pgfqpoint{2.424136in}{1.226503in}}%
\pgfpathlineto{\pgfqpoint{2.429460in}{1.219894in}}%
\pgfpathlineto{\pgfqpoint{2.434783in}{1.213115in}}%
\pgfpathlineto{\pgfqpoint{2.440106in}{1.206166in}}%
\pgfpathlineto{\pgfqpoint{2.445429in}{1.199048in}}%
\pgfpathlineto{\pgfqpoint{2.450753in}{1.191761in}}%
\pgfpathlineto{\pgfqpoint{2.456076in}{1.184304in}}%
\pgfpathlineto{\pgfqpoint{2.461399in}{1.176678in}}%
\pgfpathlineto{\pgfqpoint{2.466722in}{1.168882in}}%
\pgfpathlineto{\pgfqpoint{2.472045in}{1.160917in}}%
\pgfpathlineto{\pgfqpoint{2.477369in}{1.152782in}}%
\pgfpathlineto{\pgfqpoint{2.482692in}{1.144478in}}%
\pgfusepath{stroke}%
\end{pgfscope}%
\begin{pgfscope}%
\pgfpathrectangle{\pgfqpoint{0.374692in}{0.521603in}}{\pgfqpoint{2.635000in}{1.661000in}} %
\pgfusepath{clip}%
\pgfsetrectcap%
\pgfsetroundjoin%
\pgfsetlinewidth{1.003750pt}%
\definecolor{currentstroke}{rgb}{0.121569,0.466667,0.705882}%
\pgfsetstrokecolor{currentstroke}%
\pgfsetdash{}{0pt}%
\pgfpathmoveto{\pgfqpoint{1.955692in}{0.729228in}}%
\pgfpathlineto{\pgfqpoint{1.961015in}{0.729271in}}%
\pgfpathlineto{\pgfqpoint{1.966338in}{0.729398in}}%
\pgfpathlineto{\pgfqpoint{1.971662in}{0.729610in}}%
\pgfpathlineto{\pgfqpoint{1.976985in}{0.729906in}}%
\pgfpathlineto{\pgfqpoint{1.982308in}{0.730288in}}%
\pgfpathlineto{\pgfqpoint{1.987631in}{0.730754in}}%
\pgfpathlineto{\pgfqpoint{1.992955in}{0.731304in}}%
\pgfpathlineto{\pgfqpoint{1.998278in}{0.731940in}}%
\pgfpathlineto{\pgfqpoint{2.003601in}{0.732660in}}%
\pgfpathlineto{\pgfqpoint{2.008924in}{0.733465in}}%
\pgfpathlineto{\pgfqpoint{2.014248in}{0.734355in}}%
\pgfpathlineto{\pgfqpoint{2.019571in}{0.735329in}}%
\pgfpathlineto{\pgfqpoint{2.024894in}{0.736389in}}%
\pgfpathlineto{\pgfqpoint{2.030217in}{0.737532in}}%
\pgfpathlineto{\pgfqpoint{2.035540in}{0.738761in}}%
\pgfpathlineto{\pgfqpoint{2.040864in}{0.740075in}}%
\pgfpathlineto{\pgfqpoint{2.046187in}{0.741473in}}%
\pgfpathlineto{\pgfqpoint{2.051510in}{0.742956in}}%
\pgfpathlineto{\pgfqpoint{2.056833in}{0.744523in}}%
\pgfpathlineto{\pgfqpoint{2.062157in}{0.746176in}}%
\pgfpathlineto{\pgfqpoint{2.067480in}{0.747913in}}%
\pgfpathlineto{\pgfqpoint{2.072803in}{0.749735in}}%
\pgfpathlineto{\pgfqpoint{2.078126in}{0.751641in}}%
\pgfpathlineto{\pgfqpoint{2.083450in}{0.753632in}}%
\pgfpathlineto{\pgfqpoint{2.088773in}{0.755708in}}%
\pgfpathlineto{\pgfqpoint{2.094096in}{0.757869in}}%
\pgfpathlineto{\pgfqpoint{2.099419in}{0.760115in}}%
\pgfpathlineto{\pgfqpoint{2.104742in}{0.762445in}}%
\pgfpathlineto{\pgfqpoint{2.110066in}{0.764860in}}%
\pgfpathlineto{\pgfqpoint{2.115389in}{0.767360in}}%
\pgfpathlineto{\pgfqpoint{2.120712in}{0.769944in}}%
\pgfpathlineto{\pgfqpoint{2.126035in}{0.772613in}}%
\pgfpathlineto{\pgfqpoint{2.131359in}{0.775367in}}%
\pgfpathlineto{\pgfqpoint{2.136682in}{0.778206in}}%
\pgfpathlineto{\pgfqpoint{2.142005in}{0.781129in}}%
\pgfpathlineto{\pgfqpoint{2.147328in}{0.784137in}}%
\pgfpathlineto{\pgfqpoint{2.152652in}{0.787230in}}%
\pgfpathlineto{\pgfqpoint{2.157975in}{0.790408in}}%
\pgfpathlineto{\pgfqpoint{2.163298in}{0.793670in}}%
\pgfpathlineto{\pgfqpoint{2.168621in}{0.797017in}}%
\pgfpathlineto{\pgfqpoint{2.173944in}{0.800449in}}%
\pgfpathlineto{\pgfqpoint{2.179268in}{0.803966in}}%
\pgfpathlineto{\pgfqpoint{2.184591in}{0.807567in}}%
\pgfpathlineto{\pgfqpoint{2.189914in}{0.811253in}}%
\pgfpathlineto{\pgfqpoint{2.195237in}{0.815024in}}%
\pgfpathlineto{\pgfqpoint{2.200561in}{0.818879in}}%
\pgfpathlineto{\pgfqpoint{2.205884in}{0.822820in}}%
\pgfpathlineto{\pgfqpoint{2.211207in}{0.826844in}}%
\pgfpathlineto{\pgfqpoint{2.216530in}{0.830954in}}%
\pgfpathlineto{\pgfqpoint{2.221854in}{0.835149in}}%
\pgfpathlineto{\pgfqpoint{2.227177in}{0.839428in}}%
\pgfpathlineto{\pgfqpoint{2.232500in}{0.843792in}}%
\pgfpathlineto{\pgfqpoint{2.237823in}{0.848240in}}%
\pgfpathlineto{\pgfqpoint{2.243146in}{0.852774in}}%
\pgfpathlineto{\pgfqpoint{2.248470in}{0.857392in}}%
\pgfpathlineto{\pgfqpoint{2.253793in}{0.862095in}}%
\pgfpathlineto{\pgfqpoint{2.259116in}{0.866882in}}%
\pgfpathlineto{\pgfqpoint{2.264439in}{0.871755in}}%
\pgfpathlineto{\pgfqpoint{2.269763in}{0.876712in}}%
\pgfpathlineto{\pgfqpoint{2.275086in}{0.881754in}}%
\pgfpathlineto{\pgfqpoint{2.280409in}{0.886880in}}%
\pgfpathlineto{\pgfqpoint{2.285732in}{0.892091in}}%
\pgfpathlineto{\pgfqpoint{2.291056in}{0.897387in}}%
\pgfpathlineto{\pgfqpoint{2.296379in}{0.902768in}}%
\pgfpathlineto{\pgfqpoint{2.301702in}{0.908234in}}%
\pgfpathlineto{\pgfqpoint{2.307025in}{0.913784in}}%
\pgfpathlineto{\pgfqpoint{2.312349in}{0.919419in}}%
\pgfpathlineto{\pgfqpoint{2.317672in}{0.925139in}}%
\pgfpathlineto{\pgfqpoint{2.322995in}{0.930943in}}%
\pgfpathlineto{\pgfqpoint{2.328318in}{0.936832in}}%
\pgfpathlineto{\pgfqpoint{2.333641in}{0.942806in}}%
\pgfpathlineto{\pgfqpoint{2.338965in}{0.948865in}}%
\pgfpathlineto{\pgfqpoint{2.344288in}{0.955008in}}%
\pgfpathlineto{\pgfqpoint{2.349611in}{0.961236in}}%
\pgfpathlineto{\pgfqpoint{2.354934in}{0.967549in}}%
\pgfpathlineto{\pgfqpoint{2.360258in}{0.973947in}}%
\pgfpathlineto{\pgfqpoint{2.365581in}{0.980429in}}%
\pgfpathlineto{\pgfqpoint{2.370904in}{0.986996in}}%
\pgfpathlineto{\pgfqpoint{2.376227in}{0.993648in}}%
\pgfpathlineto{\pgfqpoint{2.381551in}{1.000384in}}%
\pgfpathlineto{\pgfqpoint{2.386874in}{1.007206in}}%
\pgfpathlineto{\pgfqpoint{2.392197in}{1.014112in}}%
\pgfpathlineto{\pgfqpoint{2.397520in}{1.021102in}}%
\pgfpathlineto{\pgfqpoint{2.402843in}{1.028178in}}%
\pgfpathlineto{\pgfqpoint{2.408167in}{1.035338in}}%
\pgfpathlineto{\pgfqpoint{2.413490in}{1.042583in}}%
\pgfpathlineto{\pgfqpoint{2.418813in}{1.049913in}}%
\pgfpathlineto{\pgfqpoint{2.424136in}{1.057327in}}%
\pgfpathlineto{\pgfqpoint{2.429460in}{1.064826in}}%
\pgfpathlineto{\pgfqpoint{2.434783in}{1.072410in}}%
\pgfpathlineto{\pgfqpoint{2.440106in}{1.080079in}}%
\pgfpathlineto{\pgfqpoint{2.445429in}{1.087832in}}%
\pgfpathlineto{\pgfqpoint{2.450753in}{1.095670in}}%
\pgfpathlineto{\pgfqpoint{2.456076in}{1.103593in}}%
\pgfpathlineto{\pgfqpoint{2.461399in}{1.111601in}}%
\pgfpathlineto{\pgfqpoint{2.466722in}{1.119693in}}%
\pgfpathlineto{\pgfqpoint{2.472045in}{1.127870in}}%
\pgfpathlineto{\pgfqpoint{2.477369in}{1.136132in}}%
\pgfpathlineto{\pgfqpoint{2.482692in}{1.144478in}}%
\pgfusepath{stroke}%
\end{pgfscope}%
\begin{pgfscope}%
\pgfpathrectangle{\pgfqpoint{0.374692in}{0.521603in}}{\pgfqpoint{2.635000in}{1.661000in}} %
\pgfusepath{clip}%
\pgfsetbuttcap%
\pgfsetroundjoin%
\pgfsetlinewidth{1.505625pt}%
\definecolor{currentstroke}{rgb}{0.000000,0.000000,0.000000}%
\pgfsetstrokecolor{currentstroke}%
\pgfsetdash{{5.550000pt}{2.400000pt}}{0.000000pt}%
\pgfpathmoveto{\pgfqpoint{1.955692in}{0.729228in}}%
\pgfpathlineto{\pgfqpoint{1.955692in}{0.798437in}}%
\pgfpathlineto{\pgfqpoint{1.955692in}{0.867645in}}%
\pgfpathlineto{\pgfqpoint{1.955692in}{0.936853in}}%
\pgfpathlineto{\pgfqpoint{1.955692in}{1.006062in}}%
\pgfpathlineto{\pgfqpoint{1.955692in}{1.075270in}}%
\pgfpathlineto{\pgfqpoint{1.955692in}{1.144478in}}%
\pgfpathlineto{\pgfqpoint{1.955692in}{1.213687in}}%
\pgfpathlineto{\pgfqpoint{1.955692in}{1.282895in}}%
\pgfpathlineto{\pgfqpoint{1.955692in}{1.352103in}}%
\pgfusepath{stroke}%
\end{pgfscope}%
\begin{pgfscope}%
\pgfpathrectangle{\pgfqpoint{0.374692in}{0.521603in}}{\pgfqpoint{2.635000in}{1.661000in}} %
\pgfusepath{clip}%
\pgfsetrectcap%
\pgfsetroundjoin%
\pgfsetlinewidth{1.003750pt}%
\definecolor{currentstroke}{rgb}{0.580392,0.403922,0.741176}%
\pgfsetstrokecolor{currentstroke}%
\pgfsetdash{}{0pt}%
\pgfpathmoveto{\pgfqpoint{2.482692in}{1.144478in}}%
\pgfpathlineto{\pgfqpoint{2.488015in}{1.136132in}}%
\pgfpathlineto{\pgfqpoint{2.493338in}{1.127870in}}%
\pgfpathlineto{\pgfqpoint{2.498662in}{1.119693in}}%
\pgfpathlineto{\pgfqpoint{2.503985in}{1.111601in}}%
\pgfpathlineto{\pgfqpoint{2.509308in}{1.103593in}}%
\pgfpathlineto{\pgfqpoint{2.514631in}{1.095670in}}%
\pgfpathlineto{\pgfqpoint{2.519955in}{1.087832in}}%
\pgfpathlineto{\pgfqpoint{2.525278in}{1.080079in}}%
\pgfpathlineto{\pgfqpoint{2.530601in}{1.072410in}}%
\pgfpathlineto{\pgfqpoint{2.535924in}{1.064826in}}%
\pgfpathlineto{\pgfqpoint{2.541248in}{1.057327in}}%
\pgfpathlineto{\pgfqpoint{2.546571in}{1.049913in}}%
\pgfpathlineto{\pgfqpoint{2.551894in}{1.042583in}}%
\pgfpathlineto{\pgfqpoint{2.557217in}{1.035338in}}%
\pgfpathlineto{\pgfqpoint{2.562540in}{1.028178in}}%
\pgfpathlineto{\pgfqpoint{2.567864in}{1.021102in}}%
\pgfpathlineto{\pgfqpoint{2.573187in}{1.014112in}}%
\pgfpathlineto{\pgfqpoint{2.578510in}{1.007206in}}%
\pgfpathlineto{\pgfqpoint{2.583833in}{1.000384in}}%
\pgfpathlineto{\pgfqpoint{2.589157in}{0.993648in}}%
\pgfpathlineto{\pgfqpoint{2.594480in}{0.986996in}}%
\pgfpathlineto{\pgfqpoint{2.599803in}{0.980429in}}%
\pgfpathlineto{\pgfqpoint{2.605126in}{0.973947in}}%
\pgfpathlineto{\pgfqpoint{2.610450in}{0.967549in}}%
\pgfpathlineto{\pgfqpoint{2.615773in}{0.961236in}}%
\pgfpathlineto{\pgfqpoint{2.621096in}{0.955008in}}%
\pgfpathlineto{\pgfqpoint{2.626419in}{0.948865in}}%
\pgfpathlineto{\pgfqpoint{2.631742in}{0.942806in}}%
\pgfpathlineto{\pgfqpoint{2.637066in}{0.936832in}}%
\pgfpathlineto{\pgfqpoint{2.642389in}{0.930943in}}%
\pgfpathlineto{\pgfqpoint{2.647712in}{0.925139in}}%
\pgfpathlineto{\pgfqpoint{2.653035in}{0.919419in}}%
\pgfpathlineto{\pgfqpoint{2.658359in}{0.913784in}}%
\pgfpathlineto{\pgfqpoint{2.663682in}{0.908234in}}%
\pgfpathlineto{\pgfqpoint{2.669005in}{0.902768in}}%
\pgfpathlineto{\pgfqpoint{2.674328in}{0.897387in}}%
\pgfpathlineto{\pgfqpoint{2.679652in}{0.892091in}}%
\pgfpathlineto{\pgfqpoint{2.684975in}{0.886880in}}%
\pgfpathlineto{\pgfqpoint{2.690298in}{0.881754in}}%
\pgfpathlineto{\pgfqpoint{2.695621in}{0.876712in}}%
\pgfpathlineto{\pgfqpoint{2.700944in}{0.871755in}}%
\pgfpathlineto{\pgfqpoint{2.706268in}{0.866882in}}%
\pgfpathlineto{\pgfqpoint{2.711591in}{0.862095in}}%
\pgfpathlineto{\pgfqpoint{2.716914in}{0.857392in}}%
\pgfpathlineto{\pgfqpoint{2.722237in}{0.852774in}}%
\pgfpathlineto{\pgfqpoint{2.727561in}{0.848240in}}%
\pgfpathlineto{\pgfqpoint{2.732884in}{0.843792in}}%
\pgfpathlineto{\pgfqpoint{2.738207in}{0.839428in}}%
\pgfpathlineto{\pgfqpoint{2.743530in}{0.835149in}}%
\pgfpathlineto{\pgfqpoint{2.748854in}{0.830954in}}%
\pgfpathlineto{\pgfqpoint{2.754177in}{0.826844in}}%
\pgfpathlineto{\pgfqpoint{2.759500in}{0.822820in}}%
\pgfpathlineto{\pgfqpoint{2.764823in}{0.818879in}}%
\pgfpathlineto{\pgfqpoint{2.770146in}{0.815024in}}%
\pgfpathlineto{\pgfqpoint{2.775470in}{0.811253in}}%
\pgfpathlineto{\pgfqpoint{2.780793in}{0.807567in}}%
\pgfpathlineto{\pgfqpoint{2.786116in}{0.803966in}}%
\pgfpathlineto{\pgfqpoint{2.791439in}{0.800449in}}%
\pgfpathlineto{\pgfqpoint{2.796763in}{0.797017in}}%
\pgfpathlineto{\pgfqpoint{2.802086in}{0.793670in}}%
\pgfpathlineto{\pgfqpoint{2.807409in}{0.790408in}}%
\pgfpathlineto{\pgfqpoint{2.812732in}{0.787230in}}%
\pgfpathlineto{\pgfqpoint{2.818056in}{0.784137in}}%
\pgfpathlineto{\pgfqpoint{2.823379in}{0.781129in}}%
\pgfpathlineto{\pgfqpoint{2.828702in}{0.778206in}}%
\pgfpathlineto{\pgfqpoint{2.834025in}{0.775367in}}%
\pgfpathlineto{\pgfqpoint{2.839349in}{0.772613in}}%
\pgfpathlineto{\pgfqpoint{2.844672in}{0.769944in}}%
\pgfpathlineto{\pgfqpoint{2.849995in}{0.767360in}}%
\pgfpathlineto{\pgfqpoint{2.855318in}{0.764860in}}%
\pgfpathlineto{\pgfqpoint{2.860641in}{0.762445in}}%
\pgfpathlineto{\pgfqpoint{2.865965in}{0.760115in}}%
\pgfpathlineto{\pgfqpoint{2.871288in}{0.757869in}}%
\pgfpathlineto{\pgfqpoint{2.876611in}{0.755708in}}%
\pgfpathlineto{\pgfqpoint{2.881934in}{0.753632in}}%
\pgfpathlineto{\pgfqpoint{2.887258in}{0.751641in}}%
\pgfpathlineto{\pgfqpoint{2.892581in}{0.749735in}}%
\pgfpathlineto{\pgfqpoint{2.897904in}{0.747913in}}%
\pgfpathlineto{\pgfqpoint{2.903227in}{0.746176in}}%
\pgfpathlineto{\pgfqpoint{2.908551in}{0.744523in}}%
\pgfpathlineto{\pgfqpoint{2.913874in}{0.742956in}}%
\pgfpathlineto{\pgfqpoint{2.919197in}{0.741473in}}%
\pgfpathlineto{\pgfqpoint{2.924520in}{0.740075in}}%
\pgfpathlineto{\pgfqpoint{2.929843in}{0.738761in}}%
\pgfpathlineto{\pgfqpoint{2.935167in}{0.737532in}}%
\pgfpathlineto{\pgfqpoint{2.940490in}{0.736389in}}%
\pgfpathlineto{\pgfqpoint{2.945813in}{0.735329in}}%
\pgfpathlineto{\pgfqpoint{2.951136in}{0.734355in}}%
\pgfpathlineto{\pgfqpoint{2.956460in}{0.733465in}}%
\pgfpathlineto{\pgfqpoint{2.961783in}{0.732660in}}%
\pgfpathlineto{\pgfqpoint{2.967106in}{0.731940in}}%
\pgfpathlineto{\pgfqpoint{2.972429in}{0.731304in}}%
\pgfpathlineto{\pgfqpoint{2.977753in}{0.730754in}}%
\pgfpathlineto{\pgfqpoint{2.983076in}{0.730288in}}%
\pgfpathlineto{\pgfqpoint{2.988399in}{0.729906in}}%
\pgfpathlineto{\pgfqpoint{2.993722in}{0.729610in}}%
\pgfpathlineto{\pgfqpoint{2.999045in}{0.729398in}}%
\pgfpathlineto{\pgfqpoint{3.004369in}{0.729271in}}%
\pgfpathlineto{\pgfqpoint{3.009692in}{0.729228in}}%
\pgfusepath{stroke}%
\end{pgfscope}%
\begin{pgfscope}%
\pgfpathrectangle{\pgfqpoint{0.374692in}{0.521603in}}{\pgfqpoint{2.635000in}{1.661000in}} %
\pgfusepath{clip}%
\pgfsetrectcap%
\pgfsetroundjoin%
\pgfsetlinewidth{1.003750pt}%
\definecolor{currentstroke}{rgb}{0.121569,0.466667,0.705882}%
\pgfsetstrokecolor{currentstroke}%
\pgfsetdash{}{0pt}%
\pgfpathmoveto{\pgfqpoint{2.482692in}{1.144478in}}%
\pgfpathlineto{\pgfqpoint{2.488015in}{1.152782in}}%
\pgfpathlineto{\pgfqpoint{2.493338in}{1.160917in}}%
\pgfpathlineto{\pgfqpoint{2.498662in}{1.168882in}}%
\pgfpathlineto{\pgfqpoint{2.503985in}{1.176678in}}%
\pgfpathlineto{\pgfqpoint{2.509308in}{1.184304in}}%
\pgfpathlineto{\pgfqpoint{2.514631in}{1.191761in}}%
\pgfpathlineto{\pgfqpoint{2.519955in}{1.199048in}}%
\pgfpathlineto{\pgfqpoint{2.525278in}{1.206166in}}%
\pgfpathlineto{\pgfqpoint{2.530601in}{1.213115in}}%
\pgfpathlineto{\pgfqpoint{2.535924in}{1.219894in}}%
\pgfpathlineto{\pgfqpoint{2.541248in}{1.226503in}}%
\pgfpathlineto{\pgfqpoint{2.546571in}{1.232943in}}%
\pgfpathlineto{\pgfqpoint{2.551894in}{1.239213in}}%
\pgfpathlineto{\pgfqpoint{2.557217in}{1.245314in}}%
\pgfpathlineto{\pgfqpoint{2.562540in}{1.251246in}}%
\pgfpathlineto{\pgfqpoint{2.567864in}{1.257008in}}%
\pgfpathlineto{\pgfqpoint{2.573187in}{1.262601in}}%
\pgfpathlineto{\pgfqpoint{2.578510in}{1.268024in}}%
\pgfpathlineto{\pgfqpoint{2.583833in}{1.273277in}}%
\pgfpathlineto{\pgfqpoint{2.589157in}{1.278362in}}%
\pgfpathlineto{\pgfqpoint{2.594480in}{1.283276in}}%
\pgfpathlineto{\pgfqpoint{2.599803in}{1.288022in}}%
\pgfpathlineto{\pgfqpoint{2.605126in}{1.292597in}}%
\pgfpathlineto{\pgfqpoint{2.610450in}{1.297004in}}%
\pgfpathlineto{\pgfqpoint{2.615773in}{1.301240in}}%
\pgfpathlineto{\pgfqpoint{2.621096in}{1.305308in}}%
\pgfpathlineto{\pgfqpoint{2.626419in}{1.309206in}}%
\pgfpathlineto{\pgfqpoint{2.631742in}{1.312934in}}%
\pgfpathlineto{\pgfqpoint{2.637066in}{1.316493in}}%
\pgfpathlineto{\pgfqpoint{2.642389in}{1.319882in}}%
\pgfpathlineto{\pgfqpoint{2.647712in}{1.323102in}}%
\pgfpathlineto{\pgfqpoint{2.653035in}{1.326153in}}%
\pgfpathlineto{\pgfqpoint{2.658359in}{1.329034in}}%
\pgfpathlineto{\pgfqpoint{2.663682in}{1.331745in}}%
\pgfpathlineto{\pgfqpoint{2.669005in}{1.334288in}}%
\pgfpathlineto{\pgfqpoint{2.674328in}{1.336660in}}%
\pgfpathlineto{\pgfqpoint{2.679652in}{1.338863in}}%
\pgfpathlineto{\pgfqpoint{2.684975in}{1.340897in}}%
\pgfpathlineto{\pgfqpoint{2.690298in}{1.342761in}}%
\pgfpathlineto{\pgfqpoint{2.695621in}{1.344456in}}%
\pgfpathlineto{\pgfqpoint{2.700944in}{1.345981in}}%
\pgfpathlineto{\pgfqpoint{2.706268in}{1.347337in}}%
\pgfpathlineto{\pgfqpoint{2.711591in}{1.348523in}}%
\pgfpathlineto{\pgfqpoint{2.716914in}{1.349540in}}%
\pgfpathlineto{\pgfqpoint{2.722237in}{1.350387in}}%
\pgfpathlineto{\pgfqpoint{2.727561in}{1.351065in}}%
\pgfpathlineto{\pgfqpoint{2.732884in}{1.351574in}}%
\pgfpathlineto{\pgfqpoint{2.738207in}{1.351913in}}%
\pgfpathlineto{\pgfqpoint{2.743530in}{1.352082in}}%
\pgfpathlineto{\pgfqpoint{2.748854in}{1.352082in}}%
\pgfpathlineto{\pgfqpoint{2.754177in}{1.351913in}}%
\pgfpathlineto{\pgfqpoint{2.759500in}{1.351574in}}%
\pgfpathlineto{\pgfqpoint{2.764823in}{1.351065in}}%
\pgfpathlineto{\pgfqpoint{2.770146in}{1.350387in}}%
\pgfpathlineto{\pgfqpoint{2.775470in}{1.349540in}}%
\pgfpathlineto{\pgfqpoint{2.780793in}{1.348523in}}%
\pgfpathlineto{\pgfqpoint{2.786116in}{1.347337in}}%
\pgfpathlineto{\pgfqpoint{2.791439in}{1.345981in}}%
\pgfpathlineto{\pgfqpoint{2.796763in}{1.344456in}}%
\pgfpathlineto{\pgfqpoint{2.802086in}{1.342761in}}%
\pgfpathlineto{\pgfqpoint{2.807409in}{1.340897in}}%
\pgfpathlineto{\pgfqpoint{2.812732in}{1.338863in}}%
\pgfpathlineto{\pgfqpoint{2.818056in}{1.336660in}}%
\pgfpathlineto{\pgfqpoint{2.823379in}{1.334288in}}%
\pgfpathlineto{\pgfqpoint{2.828702in}{1.331745in}}%
\pgfpathlineto{\pgfqpoint{2.834025in}{1.329034in}}%
\pgfpathlineto{\pgfqpoint{2.839349in}{1.326153in}}%
\pgfpathlineto{\pgfqpoint{2.844672in}{1.323102in}}%
\pgfpathlineto{\pgfqpoint{2.849995in}{1.319882in}}%
\pgfpathlineto{\pgfqpoint{2.855318in}{1.316493in}}%
\pgfpathlineto{\pgfqpoint{2.860641in}{1.312934in}}%
\pgfpathlineto{\pgfqpoint{2.865965in}{1.309206in}}%
\pgfpathlineto{\pgfqpoint{2.871288in}{1.305308in}}%
\pgfpathlineto{\pgfqpoint{2.876611in}{1.301240in}}%
\pgfpathlineto{\pgfqpoint{2.881934in}{1.297004in}}%
\pgfpathlineto{\pgfqpoint{2.887258in}{1.292597in}}%
\pgfpathlineto{\pgfqpoint{2.892581in}{1.288022in}}%
\pgfpathlineto{\pgfqpoint{2.897904in}{1.283276in}}%
\pgfpathlineto{\pgfqpoint{2.903227in}{1.278362in}}%
\pgfpathlineto{\pgfqpoint{2.908551in}{1.273277in}}%
\pgfpathlineto{\pgfqpoint{2.913874in}{1.268024in}}%
\pgfpathlineto{\pgfqpoint{2.919197in}{1.262601in}}%
\pgfpathlineto{\pgfqpoint{2.924520in}{1.257008in}}%
\pgfpathlineto{\pgfqpoint{2.929843in}{1.251246in}}%
\pgfpathlineto{\pgfqpoint{2.935167in}{1.245314in}}%
\pgfpathlineto{\pgfqpoint{2.940490in}{1.239213in}}%
\pgfpathlineto{\pgfqpoint{2.945813in}{1.232943in}}%
\pgfpathlineto{\pgfqpoint{2.951136in}{1.226503in}}%
\pgfpathlineto{\pgfqpoint{2.956460in}{1.219894in}}%
\pgfpathlineto{\pgfqpoint{2.961783in}{1.213115in}}%
\pgfpathlineto{\pgfqpoint{2.967106in}{1.206166in}}%
\pgfpathlineto{\pgfqpoint{2.972429in}{1.199048in}}%
\pgfpathlineto{\pgfqpoint{2.977753in}{1.191761in}}%
\pgfpathlineto{\pgfqpoint{2.983076in}{1.184304in}}%
\pgfpathlineto{\pgfqpoint{2.988399in}{1.176678in}}%
\pgfpathlineto{\pgfqpoint{2.993722in}{1.168882in}}%
\pgfpathlineto{\pgfqpoint{2.999045in}{1.160917in}}%
\pgfpathlineto{\pgfqpoint{3.004369in}{1.152782in}}%
\pgfpathlineto{\pgfqpoint{3.009692in}{1.144478in}}%
\pgfusepath{stroke}%
\end{pgfscope}%
\begin{pgfscope}%
\pgfpathrectangle{\pgfqpoint{0.374692in}{0.521603in}}{\pgfqpoint{2.635000in}{1.661000in}} %
\pgfusepath{clip}%
\pgfsetrectcap%
\pgfsetroundjoin%
\pgfsetlinewidth{1.003750pt}%
\definecolor{currentstroke}{rgb}{1.000000,0.498039,0.054902}%
\pgfsetstrokecolor{currentstroke}%
\pgfsetdash{}{0pt}%
\pgfpathmoveto{\pgfqpoint{2.482692in}{0.729228in}}%
\pgfpathlineto{\pgfqpoint{2.488015in}{0.729271in}}%
\pgfpathlineto{\pgfqpoint{2.493338in}{0.729398in}}%
\pgfpathlineto{\pgfqpoint{2.498662in}{0.729610in}}%
\pgfpathlineto{\pgfqpoint{2.503985in}{0.729906in}}%
\pgfpathlineto{\pgfqpoint{2.509308in}{0.730288in}}%
\pgfpathlineto{\pgfqpoint{2.514631in}{0.730754in}}%
\pgfpathlineto{\pgfqpoint{2.519955in}{0.731304in}}%
\pgfpathlineto{\pgfqpoint{2.525278in}{0.731940in}}%
\pgfpathlineto{\pgfqpoint{2.530601in}{0.732660in}}%
\pgfpathlineto{\pgfqpoint{2.535924in}{0.733465in}}%
\pgfpathlineto{\pgfqpoint{2.541248in}{0.734355in}}%
\pgfpathlineto{\pgfqpoint{2.546571in}{0.735329in}}%
\pgfpathlineto{\pgfqpoint{2.551894in}{0.736389in}}%
\pgfpathlineto{\pgfqpoint{2.557217in}{0.737532in}}%
\pgfpathlineto{\pgfqpoint{2.562540in}{0.738761in}}%
\pgfpathlineto{\pgfqpoint{2.567864in}{0.740075in}}%
\pgfpathlineto{\pgfqpoint{2.573187in}{0.741473in}}%
\pgfpathlineto{\pgfqpoint{2.578510in}{0.742956in}}%
\pgfpathlineto{\pgfqpoint{2.583833in}{0.744523in}}%
\pgfpathlineto{\pgfqpoint{2.589157in}{0.746176in}}%
\pgfpathlineto{\pgfqpoint{2.594480in}{0.747913in}}%
\pgfpathlineto{\pgfqpoint{2.599803in}{0.749735in}}%
\pgfpathlineto{\pgfqpoint{2.605126in}{0.751641in}}%
\pgfpathlineto{\pgfqpoint{2.610450in}{0.753632in}}%
\pgfpathlineto{\pgfqpoint{2.615773in}{0.755708in}}%
\pgfpathlineto{\pgfqpoint{2.621096in}{0.757869in}}%
\pgfpathlineto{\pgfqpoint{2.626419in}{0.760115in}}%
\pgfpathlineto{\pgfqpoint{2.631742in}{0.762445in}}%
\pgfpathlineto{\pgfqpoint{2.637066in}{0.764860in}}%
\pgfpathlineto{\pgfqpoint{2.642389in}{0.767360in}}%
\pgfpathlineto{\pgfqpoint{2.647712in}{0.769944in}}%
\pgfpathlineto{\pgfqpoint{2.653035in}{0.772613in}}%
\pgfpathlineto{\pgfqpoint{2.658359in}{0.775367in}}%
\pgfpathlineto{\pgfqpoint{2.663682in}{0.778206in}}%
\pgfpathlineto{\pgfqpoint{2.669005in}{0.781129in}}%
\pgfpathlineto{\pgfqpoint{2.674328in}{0.784137in}}%
\pgfpathlineto{\pgfqpoint{2.679652in}{0.787230in}}%
\pgfpathlineto{\pgfqpoint{2.684975in}{0.790408in}}%
\pgfpathlineto{\pgfqpoint{2.690298in}{0.793670in}}%
\pgfpathlineto{\pgfqpoint{2.695621in}{0.797017in}}%
\pgfpathlineto{\pgfqpoint{2.700944in}{0.800449in}}%
\pgfpathlineto{\pgfqpoint{2.706268in}{0.803966in}}%
\pgfpathlineto{\pgfqpoint{2.711591in}{0.807567in}}%
\pgfpathlineto{\pgfqpoint{2.716914in}{0.811253in}}%
\pgfpathlineto{\pgfqpoint{2.722237in}{0.815024in}}%
\pgfpathlineto{\pgfqpoint{2.727561in}{0.818879in}}%
\pgfpathlineto{\pgfqpoint{2.732884in}{0.822820in}}%
\pgfpathlineto{\pgfqpoint{2.738207in}{0.826844in}}%
\pgfpathlineto{\pgfqpoint{2.743530in}{0.830954in}}%
\pgfpathlineto{\pgfqpoint{2.748854in}{0.835149in}}%
\pgfpathlineto{\pgfqpoint{2.754177in}{0.839428in}}%
\pgfpathlineto{\pgfqpoint{2.759500in}{0.843792in}}%
\pgfpathlineto{\pgfqpoint{2.764823in}{0.848240in}}%
\pgfpathlineto{\pgfqpoint{2.770146in}{0.852774in}}%
\pgfpathlineto{\pgfqpoint{2.775470in}{0.857392in}}%
\pgfpathlineto{\pgfqpoint{2.780793in}{0.862095in}}%
\pgfpathlineto{\pgfqpoint{2.786116in}{0.866882in}}%
\pgfpathlineto{\pgfqpoint{2.791439in}{0.871755in}}%
\pgfpathlineto{\pgfqpoint{2.796763in}{0.876712in}}%
\pgfpathlineto{\pgfqpoint{2.802086in}{0.881754in}}%
\pgfpathlineto{\pgfqpoint{2.807409in}{0.886880in}}%
\pgfpathlineto{\pgfqpoint{2.812732in}{0.892091in}}%
\pgfpathlineto{\pgfqpoint{2.818056in}{0.897387in}}%
\pgfpathlineto{\pgfqpoint{2.823379in}{0.902768in}}%
\pgfpathlineto{\pgfqpoint{2.828702in}{0.908234in}}%
\pgfpathlineto{\pgfqpoint{2.834025in}{0.913784in}}%
\pgfpathlineto{\pgfqpoint{2.839349in}{0.919419in}}%
\pgfpathlineto{\pgfqpoint{2.844672in}{0.925139in}}%
\pgfpathlineto{\pgfqpoint{2.849995in}{0.930943in}}%
\pgfpathlineto{\pgfqpoint{2.855318in}{0.936832in}}%
\pgfpathlineto{\pgfqpoint{2.860641in}{0.942806in}}%
\pgfpathlineto{\pgfqpoint{2.865965in}{0.948865in}}%
\pgfpathlineto{\pgfqpoint{2.871288in}{0.955008in}}%
\pgfpathlineto{\pgfqpoint{2.876611in}{0.961236in}}%
\pgfpathlineto{\pgfqpoint{2.881934in}{0.967549in}}%
\pgfpathlineto{\pgfqpoint{2.887258in}{0.973947in}}%
\pgfpathlineto{\pgfqpoint{2.892581in}{0.980429in}}%
\pgfpathlineto{\pgfqpoint{2.897904in}{0.986996in}}%
\pgfpathlineto{\pgfqpoint{2.903227in}{0.993648in}}%
\pgfpathlineto{\pgfqpoint{2.908551in}{1.000384in}}%
\pgfpathlineto{\pgfqpoint{2.913874in}{1.007206in}}%
\pgfpathlineto{\pgfqpoint{2.919197in}{1.014112in}}%
\pgfpathlineto{\pgfqpoint{2.924520in}{1.021102in}}%
\pgfpathlineto{\pgfqpoint{2.929843in}{1.028178in}}%
\pgfpathlineto{\pgfqpoint{2.935167in}{1.035338in}}%
\pgfpathlineto{\pgfqpoint{2.940490in}{1.042583in}}%
\pgfpathlineto{\pgfqpoint{2.945813in}{1.049913in}}%
\pgfpathlineto{\pgfqpoint{2.951136in}{1.057327in}}%
\pgfpathlineto{\pgfqpoint{2.956460in}{1.064826in}}%
\pgfpathlineto{\pgfqpoint{2.961783in}{1.072410in}}%
\pgfpathlineto{\pgfqpoint{2.967106in}{1.080079in}}%
\pgfpathlineto{\pgfqpoint{2.972429in}{1.087832in}}%
\pgfpathlineto{\pgfqpoint{2.977753in}{1.095670in}}%
\pgfpathlineto{\pgfqpoint{2.983076in}{1.103593in}}%
\pgfpathlineto{\pgfqpoint{2.988399in}{1.111601in}}%
\pgfpathlineto{\pgfqpoint{2.993722in}{1.119693in}}%
\pgfpathlineto{\pgfqpoint{2.999045in}{1.127870in}}%
\pgfpathlineto{\pgfqpoint{3.004369in}{1.136132in}}%
\pgfpathlineto{\pgfqpoint{3.009692in}{1.144478in}}%
\pgfusepath{stroke}%
\end{pgfscope}%
\begin{pgfscope}%
\pgfpathrectangle{\pgfqpoint{0.374692in}{0.521603in}}{\pgfqpoint{2.635000in}{1.661000in}} %
\pgfusepath{clip}%
\pgfsetbuttcap%
\pgfsetroundjoin%
\pgfsetlinewidth{1.505625pt}%
\definecolor{currentstroke}{rgb}{0.000000,0.000000,0.000000}%
\pgfsetstrokecolor{currentstroke}%
\pgfsetdash{{5.550000pt}{2.400000pt}}{0.000000pt}%
\pgfpathmoveto{\pgfqpoint{2.482692in}{0.729228in}}%
\pgfpathlineto{\pgfqpoint{2.482692in}{0.798437in}}%
\pgfpathlineto{\pgfqpoint{2.482692in}{0.867645in}}%
\pgfpathlineto{\pgfqpoint{2.482692in}{0.936853in}}%
\pgfpathlineto{\pgfqpoint{2.482692in}{1.006062in}}%
\pgfpathlineto{\pgfqpoint{2.482692in}{1.075270in}}%
\pgfpathlineto{\pgfqpoint{2.482692in}{1.144478in}}%
\pgfpathlineto{\pgfqpoint{2.482692in}{1.213687in}}%
\pgfpathlineto{\pgfqpoint{2.482692in}{1.282895in}}%
\pgfpathlineto{\pgfqpoint{2.482692in}{1.352103in}}%
\pgfusepath{stroke}%
\end{pgfscope}%
\begin{pgfscope}%
\pgfpathrectangle{\pgfqpoint{0.374692in}{0.521603in}}{\pgfqpoint{2.635000in}{1.661000in}} %
\pgfusepath{clip}%
\pgfsetbuttcap%
\pgfsetroundjoin%
\definecolor{currentfill}{rgb}{1.000000,0.000000,0.000000}%
\pgfsetfillcolor{currentfill}%
\pgfsetlinewidth{1.003750pt}%
\definecolor{currentstroke}{rgb}{1.000000,0.000000,0.000000}%
\pgfsetstrokecolor{currentstroke}%
\pgfsetdash{}{0pt}%
\pgfsys@defobject{currentmarker}{\pgfqpoint{-0.020833in}{-0.020833in}}{\pgfqpoint{0.020833in}{0.020833in}}{%
\pgfpathmoveto{\pgfqpoint{0.000000in}{-0.020833in}}%
\pgfpathcurveto{\pgfqpoint{0.005525in}{-0.020833in}}{\pgfqpoint{0.010825in}{-0.018638in}}{\pgfqpoint{0.014731in}{-0.014731in}}%
\pgfpathcurveto{\pgfqpoint{0.018638in}{-0.010825in}}{\pgfqpoint{0.020833in}{-0.005525in}}{\pgfqpoint{0.020833in}{0.000000in}}%
\pgfpathcurveto{\pgfqpoint{0.020833in}{0.005525in}}{\pgfqpoint{0.018638in}{0.010825in}}{\pgfqpoint{0.014731in}{0.014731in}}%
\pgfpathcurveto{\pgfqpoint{0.010825in}{0.018638in}}{\pgfqpoint{0.005525in}{0.020833in}}{\pgfqpoint{0.000000in}{0.020833in}}%
\pgfpathcurveto{\pgfqpoint{-0.005525in}{0.020833in}}{\pgfqpoint{-0.010825in}{0.018638in}}{\pgfqpoint{-0.014731in}{0.014731in}}%
\pgfpathcurveto{\pgfqpoint{-0.018638in}{0.010825in}}{\pgfqpoint{-0.020833in}{0.005525in}}{\pgfqpoint{-0.020833in}{0.000000in}}%
\pgfpathcurveto{\pgfqpoint{-0.020833in}{-0.005525in}}{\pgfqpoint{-0.018638in}{-0.010825in}}{\pgfqpoint{-0.014731in}{-0.014731in}}%
\pgfpathcurveto{\pgfqpoint{-0.010825in}{-0.018638in}}{\pgfqpoint{-0.005525in}{-0.020833in}}{\pgfqpoint{0.000000in}{-0.020833in}}%
\pgfpathclose%
\pgfusepath{stroke,fill}%
}%
\begin{pgfscope}%
\pgfsys@transformshift{0.434086in}{0.729228in}%
\pgfsys@useobject{currentmarker}{}%
\end{pgfscope}%
\begin{pgfscope}%
\pgfsys@transformshift{0.638192in}{0.729228in}%
\pgfsys@useobject{currentmarker}{}%
\end{pgfscope}%
\begin{pgfscope}%
\pgfsys@transformshift{0.842298in}{0.729228in}%
\pgfsys@useobject{currentmarker}{}%
\end{pgfscope}%
\begin{pgfscope}%
\pgfsys@transformshift{0.961086in}{0.729228in}%
\pgfsys@useobject{currentmarker}{}%
\end{pgfscope}%
\begin{pgfscope}%
\pgfsys@transformshift{1.165192in}{0.729228in}%
\pgfsys@useobject{currentmarker}{}%
\end{pgfscope}%
\begin{pgfscope}%
\pgfsys@transformshift{1.369298in}{0.729228in}%
\pgfsys@useobject{currentmarker}{}%
\end{pgfscope}%
\begin{pgfscope}%
\pgfsys@transformshift{1.488086in}{0.729228in}%
\pgfsys@useobject{currentmarker}{}%
\end{pgfscope}%
\begin{pgfscope}%
\pgfsys@transformshift{1.692192in}{0.729228in}%
\pgfsys@useobject{currentmarker}{}%
\end{pgfscope}%
\begin{pgfscope}%
\pgfsys@transformshift{1.896298in}{0.729228in}%
\pgfsys@useobject{currentmarker}{}%
\end{pgfscope}%
\begin{pgfscope}%
\pgfsys@transformshift{2.015086in}{0.729228in}%
\pgfsys@useobject{currentmarker}{}%
\end{pgfscope}%
\begin{pgfscope}%
\pgfsys@transformshift{2.219192in}{0.729228in}%
\pgfsys@useobject{currentmarker}{}%
\end{pgfscope}%
\begin{pgfscope}%
\pgfsys@transformshift{2.423298in}{0.729228in}%
\pgfsys@useobject{currentmarker}{}%
\end{pgfscope}%
\begin{pgfscope}%
\pgfsys@transformshift{2.542086in}{0.729228in}%
\pgfsys@useobject{currentmarker}{}%
\end{pgfscope}%
\begin{pgfscope}%
\pgfsys@transformshift{2.746192in}{0.729228in}%
\pgfsys@useobject{currentmarker}{}%
\end{pgfscope}%
\begin{pgfscope}%
\pgfsys@transformshift{2.950298in}{0.729228in}%
\pgfsys@useobject{currentmarker}{}%
\end{pgfscope}%
\end{pgfscope}%
\begin{pgfscope}%
\pgfsetrectcap%
\pgfsetmiterjoin%
\pgfsetlinewidth{0.803000pt}%
\definecolor{currentstroke}{rgb}{0.000000,0.000000,0.000000}%
\pgfsetstrokecolor{currentstroke}%
\pgfsetdash{}{0pt}%
\pgfpathmoveto{\pgfqpoint{0.374692in}{0.521603in}}%
\pgfpathlineto{\pgfqpoint{0.374692in}{2.182603in}}%
\pgfusepath{stroke}%
\end{pgfscope}%
\begin{pgfscope}%
\pgfsetrectcap%
\pgfsetmiterjoin%
\pgfsetlinewidth{0.803000pt}%
\definecolor{currentstroke}{rgb}{0.000000,0.000000,0.000000}%
\pgfsetstrokecolor{currentstroke}%
\pgfsetdash{}{0pt}%
\pgfpathmoveto{\pgfqpoint{3.009692in}{0.521603in}}%
\pgfpathlineto{\pgfqpoint{3.009692in}{2.182603in}}%
\pgfusepath{stroke}%
\end{pgfscope}%
\begin{pgfscope}%
\pgfsetrectcap%
\pgfsetmiterjoin%
\pgfsetlinewidth{0.803000pt}%
\definecolor{currentstroke}{rgb}{0.000000,0.000000,0.000000}%
\pgfsetstrokecolor{currentstroke}%
\pgfsetdash{}{0pt}%
\pgfpathmoveto{\pgfqpoint{0.374692in}{0.521603in}}%
\pgfpathlineto{\pgfqpoint{3.009692in}{0.521603in}}%
\pgfusepath{stroke}%
\end{pgfscope}%
\begin{pgfscope}%
\pgfsetrectcap%
\pgfsetmiterjoin%
\pgfsetlinewidth{0.803000pt}%
\definecolor{currentstroke}{rgb}{0.000000,0.000000,0.000000}%
\pgfsetstrokecolor{currentstroke}%
\pgfsetdash{}{0pt}%
\pgfpathmoveto{\pgfqpoint{0.374692in}{2.182603in}}%
\pgfpathlineto{\pgfqpoint{3.009692in}{2.182603in}}%
\pgfusepath{stroke}%
\end{pgfscope}%
\begin{pgfscope}%
\pgfsetbuttcap%
\pgfsetmiterjoin%
\definecolor{currentfill}{rgb}{1.000000,1.000000,1.000000}%
\pgfsetfillcolor{currentfill}%
\pgfsetfillopacity{0.800000}%
\pgfsetlinewidth{1.003750pt}%
\definecolor{currentstroke}{rgb}{0.800000,0.800000,0.800000}%
\pgfsetstrokecolor{currentstroke}%
\pgfsetstrokeopacity{0.800000}%
\pgfsetdash{}{0pt}%
\pgfpathmoveto{\pgfqpoint{0.845236in}{1.661811in}}%
\pgfpathlineto{\pgfqpoint{2.912470in}{1.661811in}}%
\pgfpathquadraticcurveto{\pgfqpoint{2.940247in}{1.661811in}}{\pgfqpoint{2.940247in}{1.689589in}}%
\pgfpathlineto{\pgfqpoint{2.940247in}{2.085381in}}%
\pgfpathquadraticcurveto{\pgfqpoint{2.940247in}{2.113159in}}{\pgfqpoint{2.912470in}{2.113159in}}%
\pgfpathlineto{\pgfqpoint{0.845236in}{2.113159in}}%
\pgfpathquadraticcurveto{\pgfqpoint{0.817458in}{2.113159in}}{\pgfqpoint{0.817458in}{2.085381in}}%
\pgfpathlineto{\pgfqpoint{0.817458in}{1.689589in}}%
\pgfpathquadraticcurveto{\pgfqpoint{0.817458in}{1.661811in}}{\pgfqpoint{0.845236in}{1.661811in}}%
\pgfpathclose%
\pgfusepath{stroke,fill}%
\end{pgfscope}%
\begin{pgfscope}%
\pgfsetbuttcap%
\pgfsetroundjoin%
\pgfsetlinewidth{1.505625pt}%
\definecolor{currentstroke}{rgb}{0.000000,0.000000,0.000000}%
\pgfsetstrokecolor{currentstroke}%
\pgfsetdash{{5.550000pt}{2.400000pt}}{0.000000pt}%
\pgfpathmoveto{\pgfqpoint{0.873014in}{2.000691in}}%
\pgfpathlineto{\pgfqpoint{1.150792in}{2.000691in}}%
\pgfusepath{stroke}%
\end{pgfscope}%
\begin{pgfscope}%
\pgftext[x=1.261903in,y=1.952080in,left,base]{\rmfamily\fontsize{10.000000}{12.000000}\selectfont element boundaries}%
\end{pgfscope}%
\begin{pgfscope}%
\pgfsetbuttcap%
\pgfsetroundjoin%
\definecolor{currentfill}{rgb}{1.000000,0.000000,0.000000}%
\pgfsetfillcolor{currentfill}%
\pgfsetlinewidth{1.003750pt}%
\definecolor{currentstroke}{rgb}{1.000000,0.000000,0.000000}%
\pgfsetstrokecolor{currentstroke}%
\pgfsetdash{}{0pt}%
\pgfsys@defobject{currentmarker}{\pgfqpoint{-0.020833in}{-0.020833in}}{\pgfqpoint{0.020833in}{0.020833in}}{%
\pgfpathmoveto{\pgfqpoint{0.000000in}{-0.020833in}}%
\pgfpathcurveto{\pgfqpoint{0.005525in}{-0.020833in}}{\pgfqpoint{0.010825in}{-0.018638in}}{\pgfqpoint{0.014731in}{-0.014731in}}%
\pgfpathcurveto{\pgfqpoint{0.018638in}{-0.010825in}}{\pgfqpoint{0.020833in}{-0.005525in}}{\pgfqpoint{0.020833in}{0.000000in}}%
\pgfpathcurveto{\pgfqpoint{0.020833in}{0.005525in}}{\pgfqpoint{0.018638in}{0.010825in}}{\pgfqpoint{0.014731in}{0.014731in}}%
\pgfpathcurveto{\pgfqpoint{0.010825in}{0.018638in}}{\pgfqpoint{0.005525in}{0.020833in}}{\pgfqpoint{0.000000in}{0.020833in}}%
\pgfpathcurveto{\pgfqpoint{-0.005525in}{0.020833in}}{\pgfqpoint{-0.010825in}{0.018638in}}{\pgfqpoint{-0.014731in}{0.014731in}}%
\pgfpathcurveto{\pgfqpoint{-0.018638in}{0.010825in}}{\pgfqpoint{-0.020833in}{0.005525in}}{\pgfqpoint{-0.020833in}{0.000000in}}%
\pgfpathcurveto{\pgfqpoint{-0.020833in}{-0.005525in}}{\pgfqpoint{-0.018638in}{-0.010825in}}{\pgfqpoint{-0.014731in}{-0.014731in}}%
\pgfpathcurveto{\pgfqpoint{-0.010825in}{-0.018638in}}{\pgfqpoint{-0.005525in}{-0.020833in}}{\pgfqpoint{0.000000in}{-0.020833in}}%
\pgfpathclose%
\pgfusepath{stroke,fill}%
}%
\begin{pgfscope}%
\pgfsys@transformshift{1.011903in}{1.796834in}%
\pgfsys@useobject{currentmarker}{}%
\end{pgfscope}%
\end{pgfscope}%
\begin{pgfscope}%
\pgftext[x=1.261903in,y=1.748223in,left,base]{\rmfamily\fontsize{10.000000}{12.000000}\selectfont Gauss-Legendre points}%
\end{pgfscope}%
\end{pgfpicture}%
\makeatother%
\endgroup%
}
\caption{(a) Example for a periodic B-spline basis of degree $p=1$ on a domain of length $L=1$ discretized by $N_\mr{el}=5$ elements and the corresponding Gauss-Legendre quadrature points. In this special case, a B-spline basis is equivalent to the basis of linear Lagrange finite elements. (b) Same for degree $p=2$.\label{fig_Bsplines_periodic}}
\end{figure}

The elements of the discretized domain are naturally related to the knot sequence by simply using all interior knots together with the boundaries of the domain as the element boundaries which we denote by $(c_k)_{k=0,\ldots,N_\text{el}}$, where $N_\text{el}$ is the total number of elements and $c_0=0$ and $c_{N_\text{el}}=L$. 

Let us note some important properties of a B-spline basis \citep{Ratnanietal2012}:
\begin{itemize}
\item B-splines are piecewise polynomials of degree $p$,
\item B-splines are non-negative,
\item Compact support: there are exactly $p+1$ non-vanishing B-splines in each element and the support of the B-spline $\varphi_j^p$ is contained in $[z_j,\ldots,z_{j+p+1}]$,
\item B-splines form a partition of unity: $\sum_{j=0}^{N-1}\varphi_j^p(z)=1,\quad\forall z\in\mathbb{R}$.
\item It a knot $z_m$ has multiplicity $r$ then the B-spline is $\mathcal{C}^(p-r)$ at $z_m$
\end{itemize}
Since B-splines are piecewise polynomials, all matrices (mass and advection matrix) can be computed exactly by using a quadrature rule of sufficient order. Here, we use the Gauss-Legendre quadrature rule with $p+1$ quadrature points per element which allows us to integrate exactly polynomials of an order up to $2p+1$.

Finally, we use a classical particle-in-cell solver \citep{Birdsalletal2004} to treat the source term and thus discretize the distribution function $f_\mr{h}$ in a sum of Dirac masses in the four-dimensional phase space:
\begin{align}
f_\mr{h}(z,\mb{v},t)\approx\sum_{k=1}^{N_\mr{p}}w_k\delta(z-z_k(t))\delta(\mb{v}-\mb{v}_k(t)),
\end{align}
where $N_\mb{p}$ is the number of particles, $w_k$ is the weight of the $k$-th particle and $\mb{v}_k=\mb{v}_k(t)$ and $z_k=z_k(t)$ are the particles' velocities and positions, respectively, satisfying the equations of motion
\begin{subequations}
\label{eq_motion_particles}
\begin{alignat}{2}
&\frac{\mr{d}\mb{v}_k}{\mr{d}t}=\frac{q_\mr{e}}{m_\mr{e}}\left[\mb{E}(z_k(t),t)+\mb{v}_k(t)\times\mb{B}(z_k(t),t)\right],\quad\quad &\mb{v}_k(0)=\mb{v}_k^0,\\
&\frac{\mr{d}z_k}{\mr{d}t}=v_{kz}, &z_k(0)=z_k^0.
\end{alignat}
\end{subequations}
We solve this set of ordinary differential equations in time with the classical Boris method which uses a staggered grid for positions and velocities, i.e. positions are computed at integer time steps ($z_k^n \rightarrow z_k^{n+1}$), whereas velocities are computed at interleaved time steps ($\mb{v}_k^{n-1/2}\rightarrow\mb{v}_k^{n+1/2}$) \citep{Boris1970, Birdsalletal2004, Qinetal2013}. With this particle approach, the integrals over the current contribution from the energetic electrons appearing in (\ref{eq_def_righthandside}) can be estimated with the usual Monte Carlo interpretation in the following manner \citep{Aydemir1994}:
\begin{align}
\int_0^Lj_{\mr{h}x/y}\varphi_j(z)\mr{d}z\approx q_\mr{e}\sum_{k=1}^{N_\mr{p}}\left[v_{kx/y}(t)\frac{1}{N_\mr{p}}\frac{f_\mr{h}^0(z_k^0,\mb{v}_k^0)}{g_\mr{h}^0(z_k^0,\mb{v}_k^0)}\varphi_j(z_k(t))\right]:=q_\mr{e}\sum_{k=1}^{N_\mr{p}}\left[v_{kx/y}(t)w_k\varphi_j(z_k(t))\right],\label{eq_hotcurrent_weak}
\end{align} 
where we have defined the particles' weights which are determined from the initial distribution function $f_\mr{h}^0$ and the sampling distribution $g_\mr{h}^0$ from which the initial particles are drawn. Throughout this work we shall entirely use the sampling distribution 
\begin{align}
g_\mr{h}^0(z,v_x,v_y,v_z)=\frac{1}{L}\frac{1}{(2\pi)^{3/2}v_{\mr{th}\parallel}v_{\mr{th}\perp}^2}\exp\left(-\frac{v_x^2+v_y^2}{2v_{\mr{th}\perp}^2}-\frac{v_z^2}{2v_{\mr{th}\parallel}^2}\right),\label{eq_sampling_distribution}
\end{align}
i.e. we sample uniformly in real space and normally in every velocity direction using standard random number generators. With this particular choice $w_k=n_{\mr{h}0}L/N_\mr{p}$ for the anisotropic Maxwellian $f_\mr{h}^0=n_{\mr{h}0}\cdot(\ref{eq_anisotropic_Maxwellian})$. Finally, since the Boris method computes positions at integer time steps and velocities at interleaved time steps, we have to approximate the entries of the average vector $\Delta t/2\left(\mathbb{S}^{n+1}+\mathbb{S}^n\right)$ appearing on the right-hand side of (\ref{eq_Crank_Nicolson}) due to the Crank-Nicolson discretization in the following manner:
\begin{align}
-\frac{\mu_0c^2q_\mr{e}\Delta t}{2}\sum_{k=1}^{N_\mr{p}}w_k\left[v_{kx/y}^{n+1}\varphi_j(z_k^{n+1})+v_{kx/y}^{n}\varphi_j(z_k^{n})\right]\approx-\mu_0c^2q_\mr{e}\Delta t\sum_{k=1}^{N_\mr{p}}w_kv_{kx/y}^{n+1/2}\varphi_j\left(\frac{1}{2}(z_k^{n+1}+z_k^n)\right).\label{eq_average_source_term}
\end{align}

Now we have everything to write down an algorithm for numerically solving the model (\ref{eq_model_linearized}) with perpendicular perturbations only:
\begin{enumerate}
\item Create a periodic B-spline basis of degree $p$ on a domain of length $L$ discretized by $N_\mr{el}$ elements (see (\ref{eq_def_Bsplines_0}) and (\ref{eq_def_Bsplines_higher})).
\item Assemble the mass matrix $\mathbb{M}$ and advection matrix $\mathbb{C}$ and from this, assemble the block matrices $\tilde{\mathbb{M}}$ (\ref{eq_def_block_1}), $\tilde{\mathbb{C}}$ (\ref{eq_def_block_2}) and $\mathbb{M}_\mr{b}$ (\ref{eq_def_block_3}).
\item Load the initial field $\mb{U}(z,t=0)$ and perform a $L^2$-projection \citep{Brambleetal2002} in order to get the $6N$ initial coefficients $\mb{u}^0$.
\item Sample the initial positions $(z_k^0)_{k=1,\ldots,N_\mr{p}}$ and velocities $(v_{kx}^0,v_{ky}^0,v_{kz}^0)_{k=1,\ldots,N_\mr{p}}$ according to the sampling distribution (\ref{eq_sampling_distribution}) by using a random number generator and compute the weights $w_k=n_{\mr{h}0}L/N_\mr{p}$.
\item Compute the electric and magnetic field at the particle positions by noting that
\begin{subequations}
\label{eq_fields_particles}
\begin{align}
&B_{x/y}(z_k^n,t^n)=\tilde{B}_{hx/y}(z_k^n,t^n)=\sum_{j=0}^{N-1}b_{x/y}^n\varphi_j(z_k^n),\\
&B_z(z_k^n,t^n)=B_0,\\
&E_{x/y}(z_k^n,t^n)=\tilde{E}_{hx/y}(z_k^n,t^n)=\sum_{j=0}^{N-1}e_{x/y}^n\varphi_j(z_k^n),\\
&E_z(z_k^n,t^n)=0.
\end{align}
\end{subequations}
\item Compute the particle velocities $(v_{kx}^{-1/2},v_{ky}^{-1/2},v_{kz}^{-1/2})_{k=1,\ldots,N_\mr{p}}$ by applying the Boris algorithm with the time step $\tilde{\Delta t}=-\Delta t/2$.
\item Start the time loop:
	\begin{enumerate}[label*=\arabic*]
	\item Update the particle positions ($z_k^n \rightarrow z_k^{n+1}$) and 	velocities ($\mb{v}_k^n\rightarrow\mb{v}_k^{n+1}$) by applying the Boris algorithm 	with the time step $\Delta t$.\label{eq_time_loop_1}
	\item Assemble the source term $\Delta t/2\left(\mathbb{S}^{n+1}+\mathbb{S}^n\right)$ in the scheme (\ref{eq_Crank_Nicolson}) according to formula (\ref{eq_average_source_term}).
	\item Update the finite element coefficients ($\mb{u}^n\rightarrow\mb{u}^{n+1}$) according to the scheme (\ref{eq_Crank_Nicolson}) with the time step $\Delta t$.
	\item Compute the new fields at the particle positions according to formulas (\ref{eq_fields_particles}).
	\item Go to 7.1
	\end{enumerate} 
\end{enumerate}
\newpage

\begin{wrapfigure}{r}{8cm}
\vspace{-0.9cm}
\centering
\includegraphics[scale=0.2]{01_Figures/deRham1D.pdf}
\caption{Commuting diagram for function spaces in one spatial dimension with continuous spaces in the upper and discrete spaces in the lower line connected via the projectros $\Pi_0$ and $\Pi_1$.\vspace{-0.3cm} \label{fig_commuting_diagram}}
\end{wrapfigure}

\subsection{Geometric finite element particle-in-cell}
\label{sec_geometric}
In this section, we apply a structure-preserving finite element particle-in-cell method on the same model. The main difference compared to the previous approach is that we look for approximate solutions of the fields in different subspaces which are related to the continuous spaces according to a commuting diagram. In one spatial dimension, this diagram takes the form depicted in fig. \ref{fig_commuting_diagram}, where the upper line represents the sequence of spaces involved in Maxwell's equations and the lower line the finite-dimensional counterparts which are related to the continuous spaces by the projectors $\Pi_0$ and $\Pi_1$. In analogy to the previous section, we assume the domain to be $\Omega=(0,L)$ and impose periodic boundary conditions on all quantities. In order to obtain a weak formulation, we multiply by (this time different) test functions $F_x$, $F_y$, $C_x$, $C_y$, $O_x$, $O_y$ and integrate over $\Omega$. This results in
\begin{subequations}
\begin{align}
    &\int_0^L\frac{\partial\tilde{E}_x}{\partial t}F_x\mathrm{d}z+c^2\int_0^L\frac{\partial\tilde{B}_y}{\partial z}F_x\mathrm{d}z+\mu_0c^2\int_0^L\tilde{j}_{\text{c}x}F_x\mathrm{d}z=-\mu_0c^2\int_0^Lj_{\text{h}x}F_x\mathrm{d}z, \\
    &\int_0^L\frac{\partial\tilde{E}_y}{\partial t}F_y\mathrm{d}z-c^2\int_0^L\frac{\partial\tilde{B}_x}{\partial z}F_y\mathrm{d}z+\mu_0c^2\int_0^L\tilde{j}_{\text{c}y}F_y\mathrm{d}z=-\mu_0c^2\int_0^Lj_{\text{h}y}F_y\mathrm{d}z, \\
    &\int_0^L\frac{\partial\tilde{B}_x}{\partial t}C_x\mathrm{d}z-\int_0^L\frac{\partial\tilde{E}_y}{\partial z}C_x\mathrm{d}z=0, \\
    &\int_0^L\frac{\partial\tilde{B}_y}{\partial t}C_y\mathrm{d}z+\int_0^L\frac{\partial\tilde{E}_x}{\partial z}C_y\mathrm{d}z=0, \\
    &\int_0^L\frac{\partial\tilde{j}_{\text{c}x}}{\partial t}O_x\mathrm{d}z-\epsilon_0\Omega_\mathrm{pe}^2\int_0^L\tilde{E}_xO_x\mathrm{d}z-\Omega_\mathrm{ce}\int_0^L\tilde{j}_{\text{c}y}O_x\mathrm{d}z=0, \\
    &\int_0^L\frac{\partial\tilde{j}_{\text{c}y}}{\partial t}O_y\mathrm{d}z-\epsilon_0\Omega_\mathrm{pe}^2\int_0^L\tilde{E}_yO_y\mathrm{d}z+\Omega_\mathrm{ce}\int_0^L\tilde{j}_{\text{c}x}O_y\mathrm{d}z=0.
\end{align}
\end{subequations}
Note that this procedure is actually not necessary for the last two equations since they do not involve spatial derivatives and are thus ODEs in time. However, for reasons of clarity, we continue with the above formulation. We will see later that all matrices due to the spatial discretization cancel out. Obviously, we should look for $\tilde{\mb{E}}$ and $\tilde{\mb{j}}_\mr{c}$ in the same space since they are never connected via spatial derivatives in the same equation. The opposite is true for the magnetic field because in Maxwell's equations $\tilde{\mb{B}}$ is connected with the other two quantities via a spatial derivative and therefore $\tilde{\mb{B}}$ must be in different spaces if we want to satisfy the diagram in fig. \ref{fig_commuting_diagram}. This means that there are two options: Either we choose $\tilde{\mb{B}}\in L^2$ and $\tilde{\mb{E}}$, $\tilde{\mb{j}}_\mr{c}\in H^1$ or vice versa. We follow \citep{Krausetal2017} and choose the former option. This leads to the following weak formulation: 
find $(\tilde{E}_x,\tilde{E}_y,\tilde{B}_x,\tilde{B}_y,\tilde{j}_{\text{c}x},\tilde{j}_{\text{c}y})\in H^1\times H^1\times L^2\times L^2\times H^1\times H^1$ such that
\begin{subequations}
\begin{alignat}{2}
	&\int_0^L\frac{\partial\tilde{E}_x}{\partial t}F_x\mathrm{d}z-c^2\int_0^L\tilde{B}_y\frac{\partial F_x}{\partial z}\mathrm{d}z+\mu_0c^2\int_0^L\tilde{j}_{\text{c}x}F_x\mathrm{d}z=-\mu_0c^2\int_0^Lj_{\text{h}x}F_x\mathrm{d}z \quad\quad&&\forall\,F_x\in H^1,\\
	&\int_0^L\frac{\partial\tilde{E}_y}{\partial t}F_y\mathrm{d}z+c^2\int_0^L\tilde{B}_x\frac{\partial F_y}{\partial z}\mathrm{d}z+\mu_0c^2\int_0^L\tilde{j}_{\text{c}y}F_y\mathrm{d}z=-\mu_0c^2\int_0^Lj_{\text{h}y}F_y\mathrm{d}z &&\forall\,F_y\in H^1,\\
	&\int_0^L\frac{\partial\tilde{B}_x}{\partial t}C_x\mathrm{d}z-\int_0^L\frac{\partial\tilde{E}_y}{\partial z}C_x\mathrm{d}z=0 &&\forall\,C_x\in L^2,\\
	&\int_0^L\frac{\partial\tilde{B}_y}{\partial t}C_y\mathrm{d}z+\int_0^L\frac{\partial\tilde{E}_x}{\partial z}C_y\mathrm{d}z=0 &&\forall\,C_y\in L^2,\\  
	&\int_0^L\frac{\partial\tilde{j}_{\text{c}x}}{\partial t}O_x\mathrm{d}z-\epsilon_0\Omega_\mathrm{pe}^2\int_0^L\tilde{E}_xO_x\mathrm{d}z-\Omega_\mathrm{ce}\int_0^L\tilde{j}_{\text{c}y}O_x\mathrm{d}z=0 &&\forall\,O_x\in H^1,\\
	&\int_0^L\frac{\partial\tilde{j}_{\text{c}y}}{\partial t}O_y\mathrm{d}z-\epsilon_0\Omega_\mathrm{pe}^2\int_0^L\tilde{E}_yO_y\mathrm{d}z+\Omega_\mathrm{ce}\int_0^L\tilde{j}_{\text{c}x}O_y\mathrm{d}z=0 &&\forall\,O_y\in H^1.
\end{alignat}
\end{subequations} 
Due to this particular choice, we have integrated by parts the terms involving the magnetic field in the first two equations in order to shift he derivative from $\tilde{\mb{B}}$ to the test functions $F_{x/y}$ (this changes the sign). This has the consequence that these equations will be solved in a weak sense, whereas the other equations will be solved in a strong sense. As a next step, we replace the spaces  $H^1$ and $L^2$ by their finite-dimensional counterparts $V_0\subset H^1$ and $V_1\subset L^2$ and denote the dimensions by $\dim V_0=N_0$ and $\dim V_1=N_1$ and the set of basis functions by $(\varphi^0_j)_{j=0,\ldots,N_0-1}$ and $(\varphi^1_{j+1/2})_{j=0,\ldots,N_1-1}$, respectively. The discrete version of above problem then reads: find $(\tilde{E}_{hx},\tilde{E}_{hy},\tilde{B}_{hx},\tilde{B}_{hy},\tilde{j}_{\text{c}x}^h,\tilde{j}_{\text{c}y}^h)\in V_0\times V_0\times V_1\times V_1\times V_0\times V_0$ such that
\begin{subequations}
\label{eq_weak_gem_discrete}
\begin{alignat}{2}
	&\int_0^L\frac{\partial\tilde{E}_{hx}}{\partial t}F_{hx}\mathrm{d}z-c^2\int_0^L\tilde{B}_{hy}\frac{\partial F_{hx}}{\partial z}\mathrm{d}z+\mu_0c^2\int_0^L\tilde{j}_{\text{c}x}^hF_{hx}\mathrm{d}z=-\mu_0c^2\int_0^Lj_{\text{h}x}F_{hx}\mathrm{d}z \quad\quad&&\forall\,F_{hx}\in V_0,\label{eq_weak_gem_discrete_1}\\
	&\int_0^L\frac{\partial\tilde{E}_{hy}}{\partial t}F_{hy}\mathrm{d}z+c^2\int_0^L\tilde{B}_{hx}\frac{\partial F_{hy}}{\partial z}\mathrm{d}z+\mu_0c^2\int_0^L\tilde{j}_{\text{c}y}^hF_{hy}\mathrm{d}z=-\mu_0c^2\int_0^Lj_{\text{h}y}F_{hy}\mathrm{d}z &&\forall\,F_{hy}\in V_0,\\
	&\int_0^L\frac{\partial\tilde{B}_{hx}}{\partial t}C_{hx}\mathrm{d}z-\int_0^L\frac{\partial\tilde{E}_{hy}}{\partial z}C_{hx}\mathrm{d}z=0 &&\forall\,C_{hx}\in V_1,\\
	&\int_0^L\frac{\partial\tilde{B}_{hy}}{\partial t}C_{hy}\mathrm{d}z+\int_0^L\frac{\partial\tilde{E}_{hx}}{\partial z}C_{hy}\mathrm{d}z=0 &&\forall\,C_{hy}\in V_1,\\  
	&\int_0^L\frac{\partial\tilde{j}_{\text{c}x}^h}{\partial t}O_{hx}\mathrm{d}z-\epsilon_0\Omega_\mathrm{pe}^2\int_0^L\tilde{E}_{hx}O_{hx}\mathrm{d}z-\Omega_\mathrm{ce}\int_0^L\tilde{j}_{\text{c}y}^hO_{hx}\mathrm{d}z=0 &&\forall\,O_{hx}\in V_0,\\
	&\int_0^L\frac{\partial\tilde{j}_{\text{c}y}^h}{\partial t}O_{hy}\mathrm{d}z-\epsilon_0\Omega_\mathrm{pe}^2\int_0^L\tilde{E}_{hy}O_{hy}\mathrm{d}z+\Omega_\mathrm{ce}\int_0^L\tilde{j}_{\text{c}x}^hO_{hy}\mathrm{d}z=0 &&\forall\,O_{hy}\in V_0.
\end{alignat}
\end{subequations}

There are multiple possibilities to construct the commuting diagram shown in fig. \ref{fig_commuting_diagram}. The general procedure for this is to define a basis for the first subspace $V_0$, then to look for an appropriate basis for the next space $V_1$ in order to satisfy the sequence for differential operators in the lower line,  and finally to find the projectors such that the diagram is commuting. For the space $V_0$, we choose standard Lagrange finite elements of degree $p$ which are most easily defined on a reference element $I=[-1,1]$ together with a mapping $F_k:I\rightarrow\Omega_k$, $s\in I\mapsto\Omega_k$ on elements $\Omega_k=[c_k,c_{k+1}]$ on the physical domain $\Omega$, where $(c_k)_{k=0,\ldots,N_\mr{el}}$ denotes the element boundaries of $N_\mr{el}$ elements. The mapping $F_k$ and its inverse $F_k^{-1}$ are simply given by
\begin{subequations}
\label{eq_mapping}
\begin{align}
&z=F_k(s):=c_k+\frac{s+1}{2}(c_{k+1}-c_k),\\
&s=F_k(z)^{-1}:=\frac{2(z-c_k)}{c_{k+1}-c_k}-1,
\end{align}
\end{subequations}
and the Lagrange \textit{shape} functions $(\eta_n(s))_{n=0,\ldots,p}$ of degree $p$ in the reference element $I$ are created from a sequence of knots $s_0=-1<\ldots<s_p=1$ and are defined by $\eta_n(s_m)=\delta_{nm}$, which leads to the well-known formula
\begin{align}
\eta_n(s)=\prod_{m\neq n}\frac{s-s_m}{s_n-s_m}.\label{eq_def_Lagrange_shape}
\end{align}
\begin{figure}[!t]
\centering
\subfigure[]{%% Creator: Matplotlib, PGF backend
%%
%% To include the figure in your LaTeX document, write
%%   \input{<filename>.pgf}
%%
%% Make sure the required packages are loaded in your preamble
%%   \usepackage{pgf}
%%
%% Figures using additional raster images can only be included by \input if
%% they are in the same directory as the main LaTeX file. For loading figures
%% from other directories you can use the `import` package
%%   \usepackage{import}
%% and then include the figures with
%%   \import{<path to file>}{<filename>.pgf}
%%
%% Matplotlib used the following preamble
%%   \usepackage{fontspec}
%%   \setmainfont{DejaVu Serif}
%%   \setsansfont{DejaVu Sans}
%%   \setmonofont{DejaVu Sans Mono}
%%
\begingroup%
\makeatletter%
\begin{pgfpicture}%
\pgfpathrectangle{\pgfpointorigin}{\pgfqpoint{3.198427in}{2.633214in}}%
\pgfusepath{use as bounding box, clip}%
\begin{pgfscope}%
\pgfsetbuttcap%
\pgfsetmiterjoin%
\definecolor{currentfill}{rgb}{1.000000,1.000000,1.000000}%
\pgfsetfillcolor{currentfill}%
\pgfsetlinewidth{0.000000pt}%
\definecolor{currentstroke}{rgb}{1.000000,1.000000,1.000000}%
\pgfsetstrokecolor{currentstroke}%
\pgfsetdash{}{0pt}%
\pgfpathmoveto{\pgfqpoint{0.000000in}{0.000000in}}%
\pgfpathlineto{\pgfqpoint{3.198427in}{0.000000in}}%
\pgfpathlineto{\pgfqpoint{3.198427in}{2.633214in}}%
\pgfpathlineto{\pgfqpoint{0.000000in}{2.633214in}}%
\pgfpathclose%
\pgfusepath{fill}%
\end{pgfscope}%
\begin{pgfscope}%
\pgfsetbuttcap%
\pgfsetmiterjoin%
\definecolor{currentfill}{rgb}{1.000000,1.000000,1.000000}%
\pgfsetfillcolor{currentfill}%
\pgfsetlinewidth{0.000000pt}%
\definecolor{currentstroke}{rgb}{0.000000,0.000000,0.000000}%
\pgfsetstrokecolor{currentstroke}%
\pgfsetstrokeopacity{0.000000}%
\pgfsetdash{}{0pt}%
\pgfpathmoveto{\pgfqpoint{0.374692in}{0.521603in}}%
\pgfpathlineto{\pgfqpoint{3.009692in}{0.521603in}}%
\pgfpathlineto{\pgfqpoint{3.009692in}{2.484603in}}%
\pgfpathlineto{\pgfqpoint{0.374692in}{2.484603in}}%
\pgfpathclose%
\pgfusepath{fill}%
\end{pgfscope}%
\begin{pgfscope}%
\pgfsetbuttcap%
\pgfsetroundjoin%
\definecolor{currentfill}{rgb}{0.000000,0.000000,0.000000}%
\pgfsetfillcolor{currentfill}%
\pgfsetlinewidth{0.803000pt}%
\definecolor{currentstroke}{rgb}{0.000000,0.000000,0.000000}%
\pgfsetstrokecolor{currentstroke}%
\pgfsetdash{}{0pt}%
\pgfsys@defobject{currentmarker}{\pgfqpoint{0.000000in}{-0.048611in}}{\pgfqpoint{0.000000in}{0.000000in}}{%
\pgfpathmoveto{\pgfqpoint{0.000000in}{0.000000in}}%
\pgfpathlineto{\pgfqpoint{0.000000in}{-0.048611in}}%
\pgfusepath{stroke,fill}%
}%
\begin{pgfscope}%
\pgfsys@transformshift{0.374692in}{0.521603in}%
\pgfsys@useobject{currentmarker}{}%
\end{pgfscope}%
\end{pgfscope}%
\begin{pgfscope}%
\pgftext[x=0.374692in,y=0.424381in,,top]{\rmfamily\fontsize{10.000000}{12.000000}\selectfont \(\displaystyle 0.0\)}%
\end{pgfscope}%
\begin{pgfscope}%
\pgfsetbuttcap%
\pgfsetroundjoin%
\definecolor{currentfill}{rgb}{0.000000,0.000000,0.000000}%
\pgfsetfillcolor{currentfill}%
\pgfsetlinewidth{0.803000pt}%
\definecolor{currentstroke}{rgb}{0.000000,0.000000,0.000000}%
\pgfsetstrokecolor{currentstroke}%
\pgfsetdash{}{0pt}%
\pgfsys@defobject{currentmarker}{\pgfqpoint{0.000000in}{-0.048611in}}{\pgfqpoint{0.000000in}{0.000000in}}{%
\pgfpathmoveto{\pgfqpoint{0.000000in}{0.000000in}}%
\pgfpathlineto{\pgfqpoint{0.000000in}{-0.048611in}}%
\pgfusepath{stroke,fill}%
}%
\begin{pgfscope}%
\pgfsys@transformshift{0.901692in}{0.521603in}%
\pgfsys@useobject{currentmarker}{}%
\end{pgfscope}%
\end{pgfscope}%
\begin{pgfscope}%
\pgftext[x=0.901692in,y=0.424381in,,top]{\rmfamily\fontsize{10.000000}{12.000000}\selectfont \(\displaystyle 0.2\)}%
\end{pgfscope}%
\begin{pgfscope}%
\pgfsetbuttcap%
\pgfsetroundjoin%
\definecolor{currentfill}{rgb}{0.000000,0.000000,0.000000}%
\pgfsetfillcolor{currentfill}%
\pgfsetlinewidth{0.803000pt}%
\definecolor{currentstroke}{rgb}{0.000000,0.000000,0.000000}%
\pgfsetstrokecolor{currentstroke}%
\pgfsetdash{}{0pt}%
\pgfsys@defobject{currentmarker}{\pgfqpoint{0.000000in}{-0.048611in}}{\pgfqpoint{0.000000in}{0.000000in}}{%
\pgfpathmoveto{\pgfqpoint{0.000000in}{0.000000in}}%
\pgfpathlineto{\pgfqpoint{0.000000in}{-0.048611in}}%
\pgfusepath{stroke,fill}%
}%
\begin{pgfscope}%
\pgfsys@transformshift{1.428692in}{0.521603in}%
\pgfsys@useobject{currentmarker}{}%
\end{pgfscope}%
\end{pgfscope}%
\begin{pgfscope}%
\pgftext[x=1.428692in,y=0.424381in,,top]{\rmfamily\fontsize{10.000000}{12.000000}\selectfont \(\displaystyle 0.4\)}%
\end{pgfscope}%
\begin{pgfscope}%
\pgfsetbuttcap%
\pgfsetroundjoin%
\definecolor{currentfill}{rgb}{0.000000,0.000000,0.000000}%
\pgfsetfillcolor{currentfill}%
\pgfsetlinewidth{0.803000pt}%
\definecolor{currentstroke}{rgb}{0.000000,0.000000,0.000000}%
\pgfsetstrokecolor{currentstroke}%
\pgfsetdash{}{0pt}%
\pgfsys@defobject{currentmarker}{\pgfqpoint{0.000000in}{-0.048611in}}{\pgfqpoint{0.000000in}{0.000000in}}{%
\pgfpathmoveto{\pgfqpoint{0.000000in}{0.000000in}}%
\pgfpathlineto{\pgfqpoint{0.000000in}{-0.048611in}}%
\pgfusepath{stroke,fill}%
}%
\begin{pgfscope}%
\pgfsys@transformshift{1.955692in}{0.521603in}%
\pgfsys@useobject{currentmarker}{}%
\end{pgfscope}%
\end{pgfscope}%
\begin{pgfscope}%
\pgftext[x=1.955692in,y=0.424381in,,top]{\rmfamily\fontsize{10.000000}{12.000000}\selectfont \(\displaystyle 0.6\)}%
\end{pgfscope}%
\begin{pgfscope}%
\pgfsetbuttcap%
\pgfsetroundjoin%
\definecolor{currentfill}{rgb}{0.000000,0.000000,0.000000}%
\pgfsetfillcolor{currentfill}%
\pgfsetlinewidth{0.803000pt}%
\definecolor{currentstroke}{rgb}{0.000000,0.000000,0.000000}%
\pgfsetstrokecolor{currentstroke}%
\pgfsetdash{}{0pt}%
\pgfsys@defobject{currentmarker}{\pgfqpoint{0.000000in}{-0.048611in}}{\pgfqpoint{0.000000in}{0.000000in}}{%
\pgfpathmoveto{\pgfqpoint{0.000000in}{0.000000in}}%
\pgfpathlineto{\pgfqpoint{0.000000in}{-0.048611in}}%
\pgfusepath{stroke,fill}%
}%
\begin{pgfscope}%
\pgfsys@transformshift{2.482692in}{0.521603in}%
\pgfsys@useobject{currentmarker}{}%
\end{pgfscope}%
\end{pgfscope}%
\begin{pgfscope}%
\pgftext[x=2.482692in,y=0.424381in,,top]{\rmfamily\fontsize{10.000000}{12.000000}\selectfont \(\displaystyle 0.8\)}%
\end{pgfscope}%
\begin{pgfscope}%
\pgfsetbuttcap%
\pgfsetroundjoin%
\definecolor{currentfill}{rgb}{0.000000,0.000000,0.000000}%
\pgfsetfillcolor{currentfill}%
\pgfsetlinewidth{0.803000pt}%
\definecolor{currentstroke}{rgb}{0.000000,0.000000,0.000000}%
\pgfsetstrokecolor{currentstroke}%
\pgfsetdash{}{0pt}%
\pgfsys@defobject{currentmarker}{\pgfqpoint{0.000000in}{-0.048611in}}{\pgfqpoint{0.000000in}{0.000000in}}{%
\pgfpathmoveto{\pgfqpoint{0.000000in}{0.000000in}}%
\pgfpathlineto{\pgfqpoint{0.000000in}{-0.048611in}}%
\pgfusepath{stroke,fill}%
}%
\begin{pgfscope}%
\pgfsys@transformshift{3.009692in}{0.521603in}%
\pgfsys@useobject{currentmarker}{}%
\end{pgfscope}%
\end{pgfscope}%
\begin{pgfscope}%
\pgftext[x=3.009692in,y=0.424381in,,top]{\rmfamily\fontsize{10.000000}{12.000000}\selectfont \(\displaystyle 1.0\)}%
\end{pgfscope}%
\begin{pgfscope}%
\pgftext[x=1.692192in,y=0.234413in,,top]{\rmfamily\fontsize{10.000000}{12.000000}\selectfont \(\displaystyle z\)}%
\end{pgfscope}%
\begin{pgfscope}%
\pgfsetbuttcap%
\pgfsetroundjoin%
\definecolor{currentfill}{rgb}{0.000000,0.000000,0.000000}%
\pgfsetfillcolor{currentfill}%
\pgfsetlinewidth{0.803000pt}%
\definecolor{currentstroke}{rgb}{0.000000,0.000000,0.000000}%
\pgfsetstrokecolor{currentstroke}%
\pgfsetdash{}{0pt}%
\pgfsys@defobject{currentmarker}{\pgfqpoint{-0.048611in}{0.000000in}}{\pgfqpoint{0.000000in}{0.000000in}}{%
\pgfpathmoveto{\pgfqpoint{0.000000in}{0.000000in}}%
\pgfpathlineto{\pgfqpoint{-0.048611in}{0.000000in}}%
\pgfusepath{stroke,fill}%
}%
\begin{pgfscope}%
\pgfsys@transformshift{0.374692in}{0.685187in}%
\pgfsys@useobject{currentmarker}{}%
\end{pgfscope}%
\end{pgfscope}%
\begin{pgfscope}%
\pgftext[x=0.100000in,y=0.632425in,left,base]{\rmfamily\fontsize{10.000000}{12.000000}\selectfont \(\displaystyle 0.0\)}%
\end{pgfscope}%
\begin{pgfscope}%
\pgfsetbuttcap%
\pgfsetroundjoin%
\definecolor{currentfill}{rgb}{0.000000,0.000000,0.000000}%
\pgfsetfillcolor{currentfill}%
\pgfsetlinewidth{0.803000pt}%
\definecolor{currentstroke}{rgb}{0.000000,0.000000,0.000000}%
\pgfsetstrokecolor{currentstroke}%
\pgfsetdash{}{0pt}%
\pgfsys@defobject{currentmarker}{\pgfqpoint{-0.048611in}{0.000000in}}{\pgfqpoint{0.000000in}{0.000000in}}{%
\pgfpathmoveto{\pgfqpoint{0.000000in}{0.000000in}}%
\pgfpathlineto{\pgfqpoint{-0.048611in}{0.000000in}}%
\pgfusepath{stroke,fill}%
}%
\begin{pgfscope}%
\pgfsys@transformshift{0.374692in}{1.012353in}%
\pgfsys@useobject{currentmarker}{}%
\end{pgfscope}%
\end{pgfscope}%
\begin{pgfscope}%
\pgftext[x=0.100000in,y=0.959592in,left,base]{\rmfamily\fontsize{10.000000}{12.000000}\selectfont \(\displaystyle 0.5\)}%
\end{pgfscope}%
\begin{pgfscope}%
\pgfsetbuttcap%
\pgfsetroundjoin%
\definecolor{currentfill}{rgb}{0.000000,0.000000,0.000000}%
\pgfsetfillcolor{currentfill}%
\pgfsetlinewidth{0.803000pt}%
\definecolor{currentstroke}{rgb}{0.000000,0.000000,0.000000}%
\pgfsetstrokecolor{currentstroke}%
\pgfsetdash{}{0pt}%
\pgfsys@defobject{currentmarker}{\pgfqpoint{-0.048611in}{0.000000in}}{\pgfqpoint{0.000000in}{0.000000in}}{%
\pgfpathmoveto{\pgfqpoint{0.000000in}{0.000000in}}%
\pgfpathlineto{\pgfqpoint{-0.048611in}{0.000000in}}%
\pgfusepath{stroke,fill}%
}%
\begin{pgfscope}%
\pgfsys@transformshift{0.374692in}{1.339520in}%
\pgfsys@useobject{currentmarker}{}%
\end{pgfscope}%
\end{pgfscope}%
\begin{pgfscope}%
\pgftext[x=0.100000in,y=1.286758in,left,base]{\rmfamily\fontsize{10.000000}{12.000000}\selectfont \(\displaystyle 1.0\)}%
\end{pgfscope}%
\begin{pgfscope}%
\pgfsetbuttcap%
\pgfsetroundjoin%
\definecolor{currentfill}{rgb}{0.000000,0.000000,0.000000}%
\pgfsetfillcolor{currentfill}%
\pgfsetlinewidth{0.803000pt}%
\definecolor{currentstroke}{rgb}{0.000000,0.000000,0.000000}%
\pgfsetstrokecolor{currentstroke}%
\pgfsetdash{}{0pt}%
\pgfsys@defobject{currentmarker}{\pgfqpoint{-0.048611in}{0.000000in}}{\pgfqpoint{0.000000in}{0.000000in}}{%
\pgfpathmoveto{\pgfqpoint{0.000000in}{0.000000in}}%
\pgfpathlineto{\pgfqpoint{-0.048611in}{0.000000in}}%
\pgfusepath{stroke,fill}%
}%
\begin{pgfscope}%
\pgfsys@transformshift{0.374692in}{1.666687in}%
\pgfsys@useobject{currentmarker}{}%
\end{pgfscope}%
\end{pgfscope}%
\begin{pgfscope}%
\pgftext[x=0.100000in,y=1.613925in,left,base]{\rmfamily\fontsize{10.000000}{12.000000}\selectfont \(\displaystyle 1.5\)}%
\end{pgfscope}%
\begin{pgfscope}%
\pgfsetbuttcap%
\pgfsetroundjoin%
\definecolor{currentfill}{rgb}{0.000000,0.000000,0.000000}%
\pgfsetfillcolor{currentfill}%
\pgfsetlinewidth{0.803000pt}%
\definecolor{currentstroke}{rgb}{0.000000,0.000000,0.000000}%
\pgfsetstrokecolor{currentstroke}%
\pgfsetdash{}{0pt}%
\pgfsys@defobject{currentmarker}{\pgfqpoint{-0.048611in}{0.000000in}}{\pgfqpoint{0.000000in}{0.000000in}}{%
\pgfpathmoveto{\pgfqpoint{0.000000in}{0.000000in}}%
\pgfpathlineto{\pgfqpoint{-0.048611in}{0.000000in}}%
\pgfusepath{stroke,fill}%
}%
\begin{pgfscope}%
\pgfsys@transformshift{0.374692in}{1.993853in}%
\pgfsys@useobject{currentmarker}{}%
\end{pgfscope}%
\end{pgfscope}%
\begin{pgfscope}%
\pgftext[x=0.100000in,y=1.941092in,left,base]{\rmfamily\fontsize{10.000000}{12.000000}\selectfont \(\displaystyle 2.0\)}%
\end{pgfscope}%
\begin{pgfscope}%
\pgfsetbuttcap%
\pgfsetroundjoin%
\definecolor{currentfill}{rgb}{0.000000,0.000000,0.000000}%
\pgfsetfillcolor{currentfill}%
\pgfsetlinewidth{0.803000pt}%
\definecolor{currentstroke}{rgb}{0.000000,0.000000,0.000000}%
\pgfsetstrokecolor{currentstroke}%
\pgfsetdash{}{0pt}%
\pgfsys@defobject{currentmarker}{\pgfqpoint{-0.048611in}{0.000000in}}{\pgfqpoint{0.000000in}{0.000000in}}{%
\pgfpathmoveto{\pgfqpoint{0.000000in}{0.000000in}}%
\pgfpathlineto{\pgfqpoint{-0.048611in}{0.000000in}}%
\pgfusepath{stroke,fill}%
}%
\begin{pgfscope}%
\pgfsys@transformshift{0.374692in}{2.321020in}%
\pgfsys@useobject{currentmarker}{}%
\end{pgfscope}%
\end{pgfscope}%
\begin{pgfscope}%
\pgftext[x=0.100000in,y=2.268258in,left,base]{\rmfamily\fontsize{10.000000}{12.000000}\selectfont \(\displaystyle 2.5\)}%
\end{pgfscope}%
\begin{pgfscope}%
\pgfpathrectangle{\pgfqpoint{0.374692in}{0.521603in}}{\pgfqpoint{2.635000in}{1.963000in}} %
\pgfusepath{clip}%
\pgfsetrectcap%
\pgfsetroundjoin%
\pgfsetlinewidth{1.003750pt}%
\definecolor{currentstroke}{rgb}{0.121569,0.466667,0.705882}%
\pgfsetstrokecolor{currentstroke}%
\pgfsetdash{}{0pt}%
\pgfpathmoveto{\pgfqpoint{0.374692in}{1.339520in}}%
\pgfpathlineto{\pgfqpoint{0.383564in}{1.319825in}}%
\pgfpathlineto{\pgfqpoint{0.392436in}{1.300398in}}%
\pgfpathlineto{\pgfqpoint{0.401308in}{1.281237in}}%
\pgfpathlineto{\pgfqpoint{0.410180in}{1.262343in}}%
\pgfpathlineto{\pgfqpoint{0.419052in}{1.243717in}}%
\pgfpathlineto{\pgfqpoint{0.427924in}{1.225357in}}%
\pgfpathlineto{\pgfqpoint{0.436796in}{1.207265in}}%
\pgfpathlineto{\pgfqpoint{0.445668in}{1.189439in}}%
\pgfpathlineto{\pgfqpoint{0.454540in}{1.171881in}}%
\pgfpathlineto{\pgfqpoint{0.463412in}{1.154590in}}%
\pgfpathlineto{\pgfqpoint{0.472285in}{1.137565in}}%
\pgfpathlineto{\pgfqpoint{0.481157in}{1.120808in}}%
\pgfpathlineto{\pgfqpoint{0.490029in}{1.104318in}}%
\pgfpathlineto{\pgfqpoint{0.498901in}{1.088095in}}%
\pgfpathlineto{\pgfqpoint{0.507773in}{1.072139in}}%
\pgfpathlineto{\pgfqpoint{0.516645in}{1.056450in}}%
\pgfpathlineto{\pgfqpoint{0.525517in}{1.041028in}}%
\pgfpathlineto{\pgfqpoint{0.534389in}{1.025873in}}%
\pgfpathlineto{\pgfqpoint{0.543261in}{1.010985in}}%
\pgfpathlineto{\pgfqpoint{0.552133in}{0.996364in}}%
\pgfpathlineto{\pgfqpoint{0.561005in}{0.982010in}}%
\pgfpathlineto{\pgfqpoint{0.569877in}{0.967923in}}%
\pgfpathlineto{\pgfqpoint{0.578749in}{0.954104in}}%
\pgfpathlineto{\pgfqpoint{0.587621in}{0.940551in}}%
\pgfpathlineto{\pgfqpoint{0.596493in}{0.927265in}}%
\pgfpathlineto{\pgfqpoint{0.605365in}{0.914247in}}%
\pgfpathlineto{\pgfqpoint{0.614237in}{0.901495in}}%
\pgfpathlineto{\pgfqpoint{0.623109in}{0.889011in}}%
\pgfpathlineto{\pgfqpoint{0.631982in}{0.876793in}}%
\pgfpathlineto{\pgfqpoint{0.640854in}{0.864843in}}%
\pgfpathlineto{\pgfqpoint{0.649726in}{0.853160in}}%
\pgfpathlineto{\pgfqpoint{0.658598in}{0.841743in}}%
\pgfpathlineto{\pgfqpoint{0.667470in}{0.830594in}}%
\pgfpathlineto{\pgfqpoint{0.676342in}{0.819712in}}%
\pgfpathlineto{\pgfqpoint{0.685214in}{0.809097in}}%
\pgfpathlineto{\pgfqpoint{0.694086in}{0.798749in}}%
\pgfpathlineto{\pgfqpoint{0.702958in}{0.788668in}}%
\pgfpathlineto{\pgfqpoint{0.711830in}{0.778854in}}%
\pgfpathlineto{\pgfqpoint{0.720702in}{0.769307in}}%
\pgfpathlineto{\pgfqpoint{0.729574in}{0.760027in}}%
\pgfpathlineto{\pgfqpoint{0.738446in}{0.751014in}}%
\pgfpathlineto{\pgfqpoint{0.747318in}{0.742268in}}%
\pgfpathlineto{\pgfqpoint{0.756190in}{0.733789in}}%
\pgfpathlineto{\pgfqpoint{0.765062in}{0.725578in}}%
\pgfpathlineto{\pgfqpoint{0.773934in}{0.717633in}}%
\pgfpathlineto{\pgfqpoint{0.782806in}{0.709955in}}%
\pgfpathlineto{\pgfqpoint{0.791678in}{0.702545in}}%
\pgfpathlineto{\pgfqpoint{0.800551in}{0.695401in}}%
\pgfpathlineto{\pgfqpoint{0.809423in}{0.688525in}}%
\pgfpathlineto{\pgfqpoint{0.818295in}{0.681915in}}%
\pgfpathlineto{\pgfqpoint{0.827167in}{0.675573in}}%
\pgfpathlineto{\pgfqpoint{0.836039in}{0.669498in}}%
\pgfpathlineto{\pgfqpoint{0.844911in}{0.663689in}}%
\pgfpathlineto{\pgfqpoint{0.853783in}{0.658148in}}%
\pgfpathlineto{\pgfqpoint{0.862655in}{0.652874in}}%
\pgfpathlineto{\pgfqpoint{0.871527in}{0.647867in}}%
\pgfpathlineto{\pgfqpoint{0.880399in}{0.643127in}}%
\pgfpathlineto{\pgfqpoint{0.889271in}{0.638654in}}%
\pgfpathlineto{\pgfqpoint{0.898143in}{0.634448in}}%
\pgfpathlineto{\pgfqpoint{0.907015in}{0.630509in}}%
\pgfpathlineto{\pgfqpoint{0.915887in}{0.626837in}}%
\pgfpathlineto{\pgfqpoint{0.924759in}{0.623432in}}%
\pgfpathlineto{\pgfqpoint{0.933631in}{0.620294in}}%
\pgfpathlineto{\pgfqpoint{0.942503in}{0.617423in}}%
\pgfpathlineto{\pgfqpoint{0.951375in}{0.614820in}}%
\pgfpathlineto{\pgfqpoint{0.960248in}{0.612483in}}%
\pgfpathlineto{\pgfqpoint{0.969120in}{0.610413in}}%
\pgfpathlineto{\pgfqpoint{0.977992in}{0.608611in}}%
\pgfpathlineto{\pgfqpoint{0.986864in}{0.607075in}}%
\pgfpathlineto{\pgfqpoint{0.995736in}{0.605807in}}%
\pgfpathlineto{\pgfqpoint{1.004608in}{0.604805in}}%
\pgfpathlineto{\pgfqpoint{1.013480in}{0.604071in}}%
\pgfpathlineto{\pgfqpoint{1.022352in}{0.603604in}}%
\pgfpathlineto{\pgfqpoint{1.031224in}{0.603403in}}%
\pgfpathlineto{\pgfqpoint{1.040096in}{0.603470in}}%
\pgfpathlineto{\pgfqpoint{1.048968in}{0.603804in}}%
\pgfpathlineto{\pgfqpoint{1.057840in}{0.604405in}}%
\pgfpathlineto{\pgfqpoint{1.066712in}{0.605273in}}%
\pgfpathlineto{\pgfqpoint{1.075584in}{0.606408in}}%
\pgfpathlineto{\pgfqpoint{1.084456in}{0.607810in}}%
\pgfpathlineto{\pgfqpoint{1.093328in}{0.609479in}}%
\pgfpathlineto{\pgfqpoint{1.102200in}{0.611415in}}%
\pgfpathlineto{\pgfqpoint{1.111072in}{0.613618in}}%
\pgfpathlineto{\pgfqpoint{1.119944in}{0.616088in}}%
\pgfpathlineto{\pgfqpoint{1.128817in}{0.618825in}}%
\pgfpathlineto{\pgfqpoint{1.137689in}{0.621830in}}%
\pgfpathlineto{\pgfqpoint{1.146561in}{0.625101in}}%
\pgfpathlineto{\pgfqpoint{1.155433in}{0.628639in}}%
\pgfpathlineto{\pgfqpoint{1.164305in}{0.632445in}}%
\pgfpathlineto{\pgfqpoint{1.173177in}{0.636517in}}%
\pgfpathlineto{\pgfqpoint{1.182049in}{0.640857in}}%
\pgfpathlineto{\pgfqpoint{1.190921in}{0.645463in}}%
\pgfpathlineto{\pgfqpoint{1.199793in}{0.650337in}}%
\pgfpathlineto{\pgfqpoint{1.208665in}{0.655478in}}%
\pgfpathlineto{\pgfqpoint{1.217537in}{0.660885in}}%
\pgfpathlineto{\pgfqpoint{1.226409in}{0.666560in}}%
\pgfpathlineto{\pgfqpoint{1.235281in}{0.672502in}}%
\pgfpathlineto{\pgfqpoint{1.244153in}{0.678711in}}%
\pgfpathlineto{\pgfqpoint{1.253025in}{0.685187in}}%
\pgfusepath{stroke}%
\end{pgfscope}%
\begin{pgfscope}%
\pgfpathrectangle{\pgfqpoint{0.374692in}{0.521603in}}{\pgfqpoint{2.635000in}{1.963000in}} %
\pgfusepath{clip}%
\pgfsetrectcap%
\pgfsetroundjoin%
\pgfsetlinewidth{1.003750pt}%
\definecolor{currentstroke}{rgb}{1.000000,0.498039,0.054902}%
\pgfsetstrokecolor{currentstroke}%
\pgfsetdash{}{0pt}%
\pgfpathmoveto{\pgfqpoint{0.374692in}{0.685187in}}%
\pgfpathlineto{\pgfqpoint{0.383564in}{0.711357in}}%
\pgfpathlineto{\pgfqpoint{0.392436in}{0.736994in}}%
\pgfpathlineto{\pgfqpoint{0.401308in}{0.762096in}}%
\pgfpathlineto{\pgfqpoint{0.410180in}{0.786665in}}%
\pgfpathlineto{\pgfqpoint{0.419052in}{0.810699in}}%
\pgfpathlineto{\pgfqpoint{0.427924in}{0.834199in}}%
\pgfpathlineto{\pgfqpoint{0.436796in}{0.857165in}}%
\pgfpathlineto{\pgfqpoint{0.445668in}{0.879597in}}%
\pgfpathlineto{\pgfqpoint{0.454540in}{0.901495in}}%
\pgfpathlineto{\pgfqpoint{0.463412in}{0.922859in}}%
\pgfpathlineto{\pgfqpoint{0.472285in}{0.943689in}}%
\pgfpathlineto{\pgfqpoint{0.481157in}{0.963984in}}%
\pgfpathlineto{\pgfqpoint{0.490029in}{0.983746in}}%
\pgfpathlineto{\pgfqpoint{0.498901in}{1.002973in}}%
\pgfpathlineto{\pgfqpoint{0.507773in}{1.021667in}}%
\pgfpathlineto{\pgfqpoint{0.516645in}{1.039826in}}%
\pgfpathlineto{\pgfqpoint{0.525517in}{1.057451in}}%
\pgfpathlineto{\pgfqpoint{0.534389in}{1.074542in}}%
\pgfpathlineto{\pgfqpoint{0.543261in}{1.091099in}}%
\pgfpathlineto{\pgfqpoint{0.552133in}{1.107122in}}%
\pgfpathlineto{\pgfqpoint{0.561005in}{1.122611in}}%
\pgfpathlineto{\pgfqpoint{0.569877in}{1.137565in}}%
\pgfpathlineto{\pgfqpoint{0.578749in}{1.151986in}}%
\pgfpathlineto{\pgfqpoint{0.587621in}{1.165872in}}%
\pgfpathlineto{\pgfqpoint{0.596493in}{1.179225in}}%
\pgfpathlineto{\pgfqpoint{0.605365in}{1.192043in}}%
\pgfpathlineto{\pgfqpoint{0.614237in}{1.204327in}}%
\pgfpathlineto{\pgfqpoint{0.623109in}{1.216077in}}%
\pgfpathlineto{\pgfqpoint{0.631982in}{1.227293in}}%
\pgfpathlineto{\pgfqpoint{0.640854in}{1.237975in}}%
\pgfpathlineto{\pgfqpoint{0.649726in}{1.248123in}}%
\pgfpathlineto{\pgfqpoint{0.658598in}{1.257737in}}%
\pgfpathlineto{\pgfqpoint{0.667470in}{1.266816in}}%
\pgfpathlineto{\pgfqpoint{0.676342in}{1.275362in}}%
\pgfpathlineto{\pgfqpoint{0.685214in}{1.283373in}}%
\pgfpathlineto{\pgfqpoint{0.694086in}{1.290851in}}%
\pgfpathlineto{\pgfqpoint{0.702958in}{1.297794in}}%
\pgfpathlineto{\pgfqpoint{0.711830in}{1.304203in}}%
\pgfpathlineto{\pgfqpoint{0.720702in}{1.310078in}}%
\pgfpathlineto{\pgfqpoint{0.729574in}{1.315419in}}%
\pgfpathlineto{\pgfqpoint{0.738446in}{1.320226in}}%
\pgfpathlineto{\pgfqpoint{0.747318in}{1.324499in}}%
\pgfpathlineto{\pgfqpoint{0.756190in}{1.328237in}}%
\pgfpathlineto{\pgfqpoint{0.765062in}{1.331442in}}%
\pgfpathlineto{\pgfqpoint{0.773934in}{1.334112in}}%
\pgfpathlineto{\pgfqpoint{0.782806in}{1.336249in}}%
\pgfpathlineto{\pgfqpoint{0.791678in}{1.337851in}}%
\pgfpathlineto{\pgfqpoint{0.800551in}{1.338919in}}%
\pgfpathlineto{\pgfqpoint{0.809423in}{1.339453in}}%
\pgfpathlineto{\pgfqpoint{0.818295in}{1.339453in}}%
\pgfpathlineto{\pgfqpoint{0.827167in}{1.338919in}}%
\pgfpathlineto{\pgfqpoint{0.836039in}{1.337851in}}%
\pgfpathlineto{\pgfqpoint{0.844911in}{1.336249in}}%
\pgfpathlineto{\pgfqpoint{0.853783in}{1.334112in}}%
\pgfpathlineto{\pgfqpoint{0.862655in}{1.331442in}}%
\pgfpathlineto{\pgfqpoint{0.871527in}{1.328237in}}%
\pgfpathlineto{\pgfqpoint{0.880399in}{1.324499in}}%
\pgfpathlineto{\pgfqpoint{0.889271in}{1.320226in}}%
\pgfpathlineto{\pgfqpoint{0.898143in}{1.315419in}}%
\pgfpathlineto{\pgfqpoint{0.907015in}{1.310078in}}%
\pgfpathlineto{\pgfqpoint{0.915887in}{1.304203in}}%
\pgfpathlineto{\pgfqpoint{0.924759in}{1.297794in}}%
\pgfpathlineto{\pgfqpoint{0.933631in}{1.290851in}}%
\pgfpathlineto{\pgfqpoint{0.942503in}{1.283373in}}%
\pgfpathlineto{\pgfqpoint{0.951375in}{1.275362in}}%
\pgfpathlineto{\pgfqpoint{0.960248in}{1.266816in}}%
\pgfpathlineto{\pgfqpoint{0.969120in}{1.257737in}}%
\pgfpathlineto{\pgfqpoint{0.977992in}{1.248123in}}%
\pgfpathlineto{\pgfqpoint{0.986864in}{1.237975in}}%
\pgfpathlineto{\pgfqpoint{0.995736in}{1.227293in}}%
\pgfpathlineto{\pgfqpoint{1.004608in}{1.216077in}}%
\pgfpathlineto{\pgfqpoint{1.013480in}{1.204327in}}%
\pgfpathlineto{\pgfqpoint{1.022352in}{1.192043in}}%
\pgfpathlineto{\pgfqpoint{1.031224in}{1.179225in}}%
\pgfpathlineto{\pgfqpoint{1.040096in}{1.165872in}}%
\pgfpathlineto{\pgfqpoint{1.048968in}{1.151986in}}%
\pgfpathlineto{\pgfqpoint{1.057840in}{1.137565in}}%
\pgfpathlineto{\pgfqpoint{1.066712in}{1.122611in}}%
\pgfpathlineto{\pgfqpoint{1.075584in}{1.107122in}}%
\pgfpathlineto{\pgfqpoint{1.084456in}{1.091099in}}%
\pgfpathlineto{\pgfqpoint{1.093328in}{1.074542in}}%
\pgfpathlineto{\pgfqpoint{1.102200in}{1.057451in}}%
\pgfpathlineto{\pgfqpoint{1.111072in}{1.039826in}}%
\pgfpathlineto{\pgfqpoint{1.119944in}{1.021667in}}%
\pgfpathlineto{\pgfqpoint{1.128817in}{1.002973in}}%
\pgfpathlineto{\pgfqpoint{1.137689in}{0.983746in}}%
\pgfpathlineto{\pgfqpoint{1.146561in}{0.963984in}}%
\pgfpathlineto{\pgfqpoint{1.155433in}{0.943689in}}%
\pgfpathlineto{\pgfqpoint{1.164305in}{0.922859in}}%
\pgfpathlineto{\pgfqpoint{1.173177in}{0.901495in}}%
\pgfpathlineto{\pgfqpoint{1.182049in}{0.879597in}}%
\pgfpathlineto{\pgfqpoint{1.190921in}{0.857165in}}%
\pgfpathlineto{\pgfqpoint{1.199793in}{0.834199in}}%
\pgfpathlineto{\pgfqpoint{1.208665in}{0.810699in}}%
\pgfpathlineto{\pgfqpoint{1.217537in}{0.786665in}}%
\pgfpathlineto{\pgfqpoint{1.226409in}{0.762096in}}%
\pgfpathlineto{\pgfqpoint{1.235281in}{0.736994in}}%
\pgfpathlineto{\pgfqpoint{1.244153in}{0.711357in}}%
\pgfpathlineto{\pgfqpoint{1.253025in}{0.685187in}}%
\pgfusepath{stroke}%
\end{pgfscope}%
\begin{pgfscope}%
\pgfpathrectangle{\pgfqpoint{0.374692in}{0.521603in}}{\pgfqpoint{2.635000in}{1.963000in}} %
\pgfusepath{clip}%
\pgfsetrectcap%
\pgfsetroundjoin%
\pgfsetlinewidth{1.003750pt}%
\definecolor{currentstroke}{rgb}{0.172549,0.627451,0.172549}%
\pgfsetstrokecolor{currentstroke}%
\pgfsetdash{}{0pt}%
\pgfpathmoveto{\pgfqpoint{0.374692in}{0.685187in}}%
\pgfpathlineto{\pgfqpoint{0.383564in}{0.678711in}}%
\pgfpathlineto{\pgfqpoint{0.392436in}{0.672502in}}%
\pgfpathlineto{\pgfqpoint{0.401308in}{0.666560in}}%
\pgfpathlineto{\pgfqpoint{0.410180in}{0.660885in}}%
\pgfpathlineto{\pgfqpoint{0.419052in}{0.655478in}}%
\pgfpathlineto{\pgfqpoint{0.427924in}{0.650337in}}%
\pgfpathlineto{\pgfqpoint{0.436796in}{0.645463in}}%
\pgfpathlineto{\pgfqpoint{0.445668in}{0.640857in}}%
\pgfpathlineto{\pgfqpoint{0.454540in}{0.636517in}}%
\pgfpathlineto{\pgfqpoint{0.463412in}{0.632445in}}%
\pgfpathlineto{\pgfqpoint{0.472285in}{0.628639in}}%
\pgfpathlineto{\pgfqpoint{0.481157in}{0.625101in}}%
\pgfpathlineto{\pgfqpoint{0.490029in}{0.621830in}}%
\pgfpathlineto{\pgfqpoint{0.498901in}{0.618825in}}%
\pgfpathlineto{\pgfqpoint{0.507773in}{0.616088in}}%
\pgfpathlineto{\pgfqpoint{0.516645in}{0.613618in}}%
\pgfpathlineto{\pgfqpoint{0.525517in}{0.611415in}}%
\pgfpathlineto{\pgfqpoint{0.534389in}{0.609479in}}%
\pgfpathlineto{\pgfqpoint{0.543261in}{0.607810in}}%
\pgfpathlineto{\pgfqpoint{0.552133in}{0.606408in}}%
\pgfpathlineto{\pgfqpoint{0.561005in}{0.605273in}}%
\pgfpathlineto{\pgfqpoint{0.569877in}{0.604405in}}%
\pgfpathlineto{\pgfqpoint{0.578749in}{0.603804in}}%
\pgfpathlineto{\pgfqpoint{0.587621in}{0.603470in}}%
\pgfpathlineto{\pgfqpoint{0.596493in}{0.603403in}}%
\pgfpathlineto{\pgfqpoint{0.605365in}{0.603604in}}%
\pgfpathlineto{\pgfqpoint{0.614237in}{0.604071in}}%
\pgfpathlineto{\pgfqpoint{0.623109in}{0.604805in}}%
\pgfpathlineto{\pgfqpoint{0.631982in}{0.605807in}}%
\pgfpathlineto{\pgfqpoint{0.640854in}{0.607075in}}%
\pgfpathlineto{\pgfqpoint{0.649726in}{0.608611in}}%
\pgfpathlineto{\pgfqpoint{0.658598in}{0.610413in}}%
\pgfpathlineto{\pgfqpoint{0.667470in}{0.612483in}}%
\pgfpathlineto{\pgfqpoint{0.676342in}{0.614820in}}%
\pgfpathlineto{\pgfqpoint{0.685214in}{0.617423in}}%
\pgfpathlineto{\pgfqpoint{0.694086in}{0.620294in}}%
\pgfpathlineto{\pgfqpoint{0.702958in}{0.623432in}}%
\pgfpathlineto{\pgfqpoint{0.711830in}{0.626837in}}%
\pgfpathlineto{\pgfqpoint{0.720702in}{0.630509in}}%
\pgfpathlineto{\pgfqpoint{0.729574in}{0.634448in}}%
\pgfpathlineto{\pgfqpoint{0.738446in}{0.638654in}}%
\pgfpathlineto{\pgfqpoint{0.747318in}{0.643127in}}%
\pgfpathlineto{\pgfqpoint{0.756190in}{0.647867in}}%
\pgfpathlineto{\pgfqpoint{0.765062in}{0.652874in}}%
\pgfpathlineto{\pgfqpoint{0.773934in}{0.658148in}}%
\pgfpathlineto{\pgfqpoint{0.782806in}{0.663689in}}%
\pgfpathlineto{\pgfqpoint{0.791678in}{0.669498in}}%
\pgfpathlineto{\pgfqpoint{0.800551in}{0.675573in}}%
\pgfpathlineto{\pgfqpoint{0.809423in}{0.681915in}}%
\pgfpathlineto{\pgfqpoint{0.818295in}{0.688525in}}%
\pgfpathlineto{\pgfqpoint{0.827167in}{0.695401in}}%
\pgfpathlineto{\pgfqpoint{0.836039in}{0.702545in}}%
\pgfpathlineto{\pgfqpoint{0.844911in}{0.709955in}}%
\pgfpathlineto{\pgfqpoint{0.853783in}{0.717633in}}%
\pgfpathlineto{\pgfqpoint{0.862655in}{0.725578in}}%
\pgfpathlineto{\pgfqpoint{0.871527in}{0.733789in}}%
\pgfpathlineto{\pgfqpoint{0.880399in}{0.742268in}}%
\pgfpathlineto{\pgfqpoint{0.889271in}{0.751014in}}%
\pgfpathlineto{\pgfqpoint{0.898143in}{0.760027in}}%
\pgfpathlineto{\pgfqpoint{0.907015in}{0.769307in}}%
\pgfpathlineto{\pgfqpoint{0.915887in}{0.778854in}}%
\pgfpathlineto{\pgfqpoint{0.924759in}{0.788668in}}%
\pgfpathlineto{\pgfqpoint{0.933631in}{0.798749in}}%
\pgfpathlineto{\pgfqpoint{0.942503in}{0.809097in}}%
\pgfpathlineto{\pgfqpoint{0.951375in}{0.819712in}}%
\pgfpathlineto{\pgfqpoint{0.960248in}{0.830594in}}%
\pgfpathlineto{\pgfqpoint{0.969120in}{0.841743in}}%
\pgfpathlineto{\pgfqpoint{0.977992in}{0.853160in}}%
\pgfpathlineto{\pgfqpoint{0.986864in}{0.864843in}}%
\pgfpathlineto{\pgfqpoint{0.995736in}{0.876793in}}%
\pgfpathlineto{\pgfqpoint{1.004608in}{0.889011in}}%
\pgfpathlineto{\pgfqpoint{1.013480in}{0.901495in}}%
\pgfpathlineto{\pgfqpoint{1.022352in}{0.914247in}}%
\pgfpathlineto{\pgfqpoint{1.031224in}{0.927265in}}%
\pgfpathlineto{\pgfqpoint{1.040096in}{0.940551in}}%
\pgfpathlineto{\pgfqpoint{1.048968in}{0.954104in}}%
\pgfpathlineto{\pgfqpoint{1.057840in}{0.967923in}}%
\pgfpathlineto{\pgfqpoint{1.066712in}{0.982010in}}%
\pgfpathlineto{\pgfqpoint{1.075584in}{0.996364in}}%
\pgfpathlineto{\pgfqpoint{1.084456in}{1.010985in}}%
\pgfpathlineto{\pgfqpoint{1.093328in}{1.025873in}}%
\pgfpathlineto{\pgfqpoint{1.102200in}{1.041028in}}%
\pgfpathlineto{\pgfqpoint{1.111072in}{1.056450in}}%
\pgfpathlineto{\pgfqpoint{1.119944in}{1.072139in}}%
\pgfpathlineto{\pgfqpoint{1.128817in}{1.088095in}}%
\pgfpathlineto{\pgfqpoint{1.137689in}{1.104318in}}%
\pgfpathlineto{\pgfqpoint{1.146561in}{1.120808in}}%
\pgfpathlineto{\pgfqpoint{1.155433in}{1.137565in}}%
\pgfpathlineto{\pgfqpoint{1.164305in}{1.154590in}}%
\pgfpathlineto{\pgfqpoint{1.173177in}{1.171881in}}%
\pgfpathlineto{\pgfqpoint{1.182049in}{1.189439in}}%
\pgfpathlineto{\pgfqpoint{1.190921in}{1.207265in}}%
\pgfpathlineto{\pgfqpoint{1.199793in}{1.225357in}}%
\pgfpathlineto{\pgfqpoint{1.208665in}{1.243717in}}%
\pgfpathlineto{\pgfqpoint{1.217537in}{1.262343in}}%
\pgfpathlineto{\pgfqpoint{1.226409in}{1.281237in}}%
\pgfpathlineto{\pgfqpoint{1.235281in}{1.300398in}}%
\pgfpathlineto{\pgfqpoint{1.244153in}{1.319825in}}%
\pgfpathlineto{\pgfqpoint{1.253025in}{1.339520in}}%
\pgfusepath{stroke}%
\end{pgfscope}%
\begin{pgfscope}%
\pgfpathrectangle{\pgfqpoint{0.374692in}{0.521603in}}{\pgfqpoint{2.635000in}{1.963000in}} %
\pgfusepath{clip}%
\pgfsetbuttcap%
\pgfsetroundjoin%
\pgfsetlinewidth{1.505625pt}%
\definecolor{currentstroke}{rgb}{0.000000,0.000000,0.000000}%
\pgfsetstrokecolor{currentstroke}%
\pgfsetdash{{5.550000pt}{2.400000pt}}{0.000000pt}%
\pgfpathmoveto{\pgfqpoint{1.253025in}{0.521603in}}%
\pgfpathlineto{\pgfqpoint{1.253025in}{0.630659in}}%
\pgfpathlineto{\pgfqpoint{1.253025in}{0.739714in}}%
\pgfpathlineto{\pgfqpoint{1.253025in}{0.848770in}}%
\pgfpathlineto{\pgfqpoint{1.253025in}{0.957826in}}%
\pgfpathlineto{\pgfqpoint{1.253025in}{1.066881in}}%
\pgfpathlineto{\pgfqpoint{1.253025in}{1.175937in}}%
\pgfpathlineto{\pgfqpoint{1.253025in}{1.284992in}}%
\pgfpathlineto{\pgfqpoint{1.253025in}{1.394048in}}%
\pgfpathlineto{\pgfqpoint{1.253025in}{1.503103in}}%
\pgfusepath{stroke}%
\end{pgfscope}%
\begin{pgfscope}%
\pgfpathrectangle{\pgfqpoint{0.374692in}{0.521603in}}{\pgfqpoint{2.635000in}{1.963000in}} %
\pgfusepath{clip}%
\pgfsetrectcap%
\pgfsetroundjoin%
\pgfsetlinewidth{1.003750pt}%
\definecolor{currentstroke}{rgb}{0.172549,0.627451,0.172549}%
\pgfsetstrokecolor{currentstroke}%
\pgfsetdash{}{0pt}%
\pgfpathmoveto{\pgfqpoint{1.253025in}{1.339520in}}%
\pgfpathlineto{\pgfqpoint{1.261897in}{1.319825in}}%
\pgfpathlineto{\pgfqpoint{1.270769in}{1.300398in}}%
\pgfpathlineto{\pgfqpoint{1.279641in}{1.281237in}}%
\pgfpathlineto{\pgfqpoint{1.288513in}{1.262343in}}%
\pgfpathlineto{\pgfqpoint{1.297386in}{1.243717in}}%
\pgfpathlineto{\pgfqpoint{1.306258in}{1.225357in}}%
\pgfpathlineto{\pgfqpoint{1.315130in}{1.207265in}}%
\pgfpathlineto{\pgfqpoint{1.324002in}{1.189439in}}%
\pgfpathlineto{\pgfqpoint{1.332874in}{1.171881in}}%
\pgfpathlineto{\pgfqpoint{1.341746in}{1.154590in}}%
\pgfpathlineto{\pgfqpoint{1.350618in}{1.137565in}}%
\pgfpathlineto{\pgfqpoint{1.359490in}{1.120808in}}%
\pgfpathlineto{\pgfqpoint{1.368362in}{1.104318in}}%
\pgfpathlineto{\pgfqpoint{1.377234in}{1.088095in}}%
\pgfpathlineto{\pgfqpoint{1.386106in}{1.072139in}}%
\pgfpathlineto{\pgfqpoint{1.394978in}{1.056450in}}%
\pgfpathlineto{\pgfqpoint{1.403850in}{1.041028in}}%
\pgfpathlineto{\pgfqpoint{1.412722in}{1.025873in}}%
\pgfpathlineto{\pgfqpoint{1.421594in}{1.010985in}}%
\pgfpathlineto{\pgfqpoint{1.430466in}{0.996364in}}%
\pgfpathlineto{\pgfqpoint{1.439338in}{0.982010in}}%
\pgfpathlineto{\pgfqpoint{1.448210in}{0.967923in}}%
\pgfpathlineto{\pgfqpoint{1.457083in}{0.954104in}}%
\pgfpathlineto{\pgfqpoint{1.465955in}{0.940551in}}%
\pgfpathlineto{\pgfqpoint{1.474827in}{0.927265in}}%
\pgfpathlineto{\pgfqpoint{1.483699in}{0.914247in}}%
\pgfpathlineto{\pgfqpoint{1.492571in}{0.901495in}}%
\pgfpathlineto{\pgfqpoint{1.501443in}{0.889011in}}%
\pgfpathlineto{\pgfqpoint{1.510315in}{0.876793in}}%
\pgfpathlineto{\pgfqpoint{1.519187in}{0.864843in}}%
\pgfpathlineto{\pgfqpoint{1.528059in}{0.853160in}}%
\pgfpathlineto{\pgfqpoint{1.536931in}{0.841743in}}%
\pgfpathlineto{\pgfqpoint{1.545803in}{0.830594in}}%
\pgfpathlineto{\pgfqpoint{1.554675in}{0.819712in}}%
\pgfpathlineto{\pgfqpoint{1.563547in}{0.809097in}}%
\pgfpathlineto{\pgfqpoint{1.572419in}{0.798749in}}%
\pgfpathlineto{\pgfqpoint{1.581291in}{0.788668in}}%
\pgfpathlineto{\pgfqpoint{1.590163in}{0.778854in}}%
\pgfpathlineto{\pgfqpoint{1.599035in}{0.769307in}}%
\pgfpathlineto{\pgfqpoint{1.607907in}{0.760027in}}%
\pgfpathlineto{\pgfqpoint{1.616779in}{0.751014in}}%
\pgfpathlineto{\pgfqpoint{1.625652in}{0.742268in}}%
\pgfpathlineto{\pgfqpoint{1.634524in}{0.733789in}}%
\pgfpathlineto{\pgfqpoint{1.643396in}{0.725578in}}%
\pgfpathlineto{\pgfqpoint{1.652268in}{0.717633in}}%
\pgfpathlineto{\pgfqpoint{1.661140in}{0.709955in}}%
\pgfpathlineto{\pgfqpoint{1.670012in}{0.702545in}}%
\pgfpathlineto{\pgfqpoint{1.678884in}{0.695401in}}%
\pgfpathlineto{\pgfqpoint{1.687756in}{0.688525in}}%
\pgfpathlineto{\pgfqpoint{1.696628in}{0.681915in}}%
\pgfpathlineto{\pgfqpoint{1.705500in}{0.675573in}}%
\pgfpathlineto{\pgfqpoint{1.714372in}{0.669498in}}%
\pgfpathlineto{\pgfqpoint{1.723244in}{0.663689in}}%
\pgfpathlineto{\pgfqpoint{1.732116in}{0.658148in}}%
\pgfpathlineto{\pgfqpoint{1.740988in}{0.652874in}}%
\pgfpathlineto{\pgfqpoint{1.749860in}{0.647867in}}%
\pgfpathlineto{\pgfqpoint{1.758732in}{0.643127in}}%
\pgfpathlineto{\pgfqpoint{1.767604in}{0.638654in}}%
\pgfpathlineto{\pgfqpoint{1.776476in}{0.634448in}}%
\pgfpathlineto{\pgfqpoint{1.785349in}{0.630509in}}%
\pgfpathlineto{\pgfqpoint{1.794221in}{0.626837in}}%
\pgfpathlineto{\pgfqpoint{1.803093in}{0.623432in}}%
\pgfpathlineto{\pgfqpoint{1.811965in}{0.620294in}}%
\pgfpathlineto{\pgfqpoint{1.820837in}{0.617423in}}%
\pgfpathlineto{\pgfqpoint{1.829709in}{0.614820in}}%
\pgfpathlineto{\pgfqpoint{1.838581in}{0.612483in}}%
\pgfpathlineto{\pgfqpoint{1.847453in}{0.610413in}}%
\pgfpathlineto{\pgfqpoint{1.856325in}{0.608611in}}%
\pgfpathlineto{\pgfqpoint{1.865197in}{0.607075in}}%
\pgfpathlineto{\pgfqpoint{1.874069in}{0.605807in}}%
\pgfpathlineto{\pgfqpoint{1.882941in}{0.604805in}}%
\pgfpathlineto{\pgfqpoint{1.891813in}{0.604071in}}%
\pgfpathlineto{\pgfqpoint{1.900685in}{0.603604in}}%
\pgfpathlineto{\pgfqpoint{1.909557in}{0.603403in}}%
\pgfpathlineto{\pgfqpoint{1.918429in}{0.603470in}}%
\pgfpathlineto{\pgfqpoint{1.927301in}{0.603804in}}%
\pgfpathlineto{\pgfqpoint{1.936173in}{0.604405in}}%
\pgfpathlineto{\pgfqpoint{1.945045in}{0.605273in}}%
\pgfpathlineto{\pgfqpoint{1.953918in}{0.606408in}}%
\pgfpathlineto{\pgfqpoint{1.962790in}{0.607810in}}%
\pgfpathlineto{\pgfqpoint{1.971662in}{0.609479in}}%
\pgfpathlineto{\pgfqpoint{1.980534in}{0.611415in}}%
\pgfpathlineto{\pgfqpoint{1.989406in}{0.613618in}}%
\pgfpathlineto{\pgfqpoint{1.998278in}{0.616088in}}%
\pgfpathlineto{\pgfqpoint{2.007150in}{0.618825in}}%
\pgfpathlineto{\pgfqpoint{2.016022in}{0.621830in}}%
\pgfpathlineto{\pgfqpoint{2.024894in}{0.625101in}}%
\pgfpathlineto{\pgfqpoint{2.033766in}{0.628639in}}%
\pgfpathlineto{\pgfqpoint{2.042638in}{0.632445in}}%
\pgfpathlineto{\pgfqpoint{2.051510in}{0.636517in}}%
\pgfpathlineto{\pgfqpoint{2.060382in}{0.640857in}}%
\pgfpathlineto{\pgfqpoint{2.069254in}{0.645463in}}%
\pgfpathlineto{\pgfqpoint{2.078126in}{0.650337in}}%
\pgfpathlineto{\pgfqpoint{2.086998in}{0.655478in}}%
\pgfpathlineto{\pgfqpoint{2.095870in}{0.660885in}}%
\pgfpathlineto{\pgfqpoint{2.104742in}{0.666560in}}%
\pgfpathlineto{\pgfqpoint{2.113615in}{0.672502in}}%
\pgfpathlineto{\pgfqpoint{2.122487in}{0.678711in}}%
\pgfpathlineto{\pgfqpoint{2.131359in}{0.685187in}}%
\pgfusepath{stroke}%
\end{pgfscope}%
\begin{pgfscope}%
\pgfpathrectangle{\pgfqpoint{0.374692in}{0.521603in}}{\pgfqpoint{2.635000in}{1.963000in}} %
\pgfusepath{clip}%
\pgfsetrectcap%
\pgfsetroundjoin%
\pgfsetlinewidth{1.003750pt}%
\definecolor{currentstroke}{rgb}{0.839216,0.152941,0.156863}%
\pgfsetstrokecolor{currentstroke}%
\pgfsetdash{}{0pt}%
\pgfpathmoveto{\pgfqpoint{1.253025in}{0.685187in}}%
\pgfpathlineto{\pgfqpoint{1.261897in}{0.711357in}}%
\pgfpathlineto{\pgfqpoint{1.270769in}{0.736994in}}%
\pgfpathlineto{\pgfqpoint{1.279641in}{0.762096in}}%
\pgfpathlineto{\pgfqpoint{1.288513in}{0.786665in}}%
\pgfpathlineto{\pgfqpoint{1.297386in}{0.810699in}}%
\pgfpathlineto{\pgfqpoint{1.306258in}{0.834199in}}%
\pgfpathlineto{\pgfqpoint{1.315130in}{0.857165in}}%
\pgfpathlineto{\pgfqpoint{1.324002in}{0.879597in}}%
\pgfpathlineto{\pgfqpoint{1.332874in}{0.901495in}}%
\pgfpathlineto{\pgfqpoint{1.341746in}{0.922859in}}%
\pgfpathlineto{\pgfqpoint{1.350618in}{0.943689in}}%
\pgfpathlineto{\pgfqpoint{1.359490in}{0.963984in}}%
\pgfpathlineto{\pgfqpoint{1.368362in}{0.983746in}}%
\pgfpathlineto{\pgfqpoint{1.377234in}{1.002973in}}%
\pgfpathlineto{\pgfqpoint{1.386106in}{1.021667in}}%
\pgfpathlineto{\pgfqpoint{1.394978in}{1.039826in}}%
\pgfpathlineto{\pgfqpoint{1.403850in}{1.057451in}}%
\pgfpathlineto{\pgfqpoint{1.412722in}{1.074542in}}%
\pgfpathlineto{\pgfqpoint{1.421594in}{1.091099in}}%
\pgfpathlineto{\pgfqpoint{1.430466in}{1.107122in}}%
\pgfpathlineto{\pgfqpoint{1.439338in}{1.122611in}}%
\pgfpathlineto{\pgfqpoint{1.448210in}{1.137565in}}%
\pgfpathlineto{\pgfqpoint{1.457083in}{1.151986in}}%
\pgfpathlineto{\pgfqpoint{1.465955in}{1.165872in}}%
\pgfpathlineto{\pgfqpoint{1.474827in}{1.179225in}}%
\pgfpathlineto{\pgfqpoint{1.483699in}{1.192043in}}%
\pgfpathlineto{\pgfqpoint{1.492571in}{1.204327in}}%
\pgfpathlineto{\pgfqpoint{1.501443in}{1.216077in}}%
\pgfpathlineto{\pgfqpoint{1.510315in}{1.227293in}}%
\pgfpathlineto{\pgfqpoint{1.519187in}{1.237975in}}%
\pgfpathlineto{\pgfqpoint{1.528059in}{1.248123in}}%
\pgfpathlineto{\pgfqpoint{1.536931in}{1.257737in}}%
\pgfpathlineto{\pgfqpoint{1.545803in}{1.266816in}}%
\pgfpathlineto{\pgfqpoint{1.554675in}{1.275362in}}%
\pgfpathlineto{\pgfqpoint{1.563547in}{1.283373in}}%
\pgfpathlineto{\pgfqpoint{1.572419in}{1.290851in}}%
\pgfpathlineto{\pgfqpoint{1.581291in}{1.297794in}}%
\pgfpathlineto{\pgfqpoint{1.590163in}{1.304203in}}%
\pgfpathlineto{\pgfqpoint{1.599035in}{1.310078in}}%
\pgfpathlineto{\pgfqpoint{1.607907in}{1.315419in}}%
\pgfpathlineto{\pgfqpoint{1.616779in}{1.320226in}}%
\pgfpathlineto{\pgfqpoint{1.625652in}{1.324499in}}%
\pgfpathlineto{\pgfqpoint{1.634524in}{1.328237in}}%
\pgfpathlineto{\pgfqpoint{1.643396in}{1.331442in}}%
\pgfpathlineto{\pgfqpoint{1.652268in}{1.334112in}}%
\pgfpathlineto{\pgfqpoint{1.661140in}{1.336249in}}%
\pgfpathlineto{\pgfqpoint{1.670012in}{1.337851in}}%
\pgfpathlineto{\pgfqpoint{1.678884in}{1.338919in}}%
\pgfpathlineto{\pgfqpoint{1.687756in}{1.339453in}}%
\pgfpathlineto{\pgfqpoint{1.696628in}{1.339453in}}%
\pgfpathlineto{\pgfqpoint{1.705500in}{1.338919in}}%
\pgfpathlineto{\pgfqpoint{1.714372in}{1.337851in}}%
\pgfpathlineto{\pgfqpoint{1.723244in}{1.336249in}}%
\pgfpathlineto{\pgfqpoint{1.732116in}{1.334112in}}%
\pgfpathlineto{\pgfqpoint{1.740988in}{1.331442in}}%
\pgfpathlineto{\pgfqpoint{1.749860in}{1.328237in}}%
\pgfpathlineto{\pgfqpoint{1.758732in}{1.324499in}}%
\pgfpathlineto{\pgfqpoint{1.767604in}{1.320226in}}%
\pgfpathlineto{\pgfqpoint{1.776476in}{1.315419in}}%
\pgfpathlineto{\pgfqpoint{1.785349in}{1.310078in}}%
\pgfpathlineto{\pgfqpoint{1.794221in}{1.304203in}}%
\pgfpathlineto{\pgfqpoint{1.803093in}{1.297794in}}%
\pgfpathlineto{\pgfqpoint{1.811965in}{1.290851in}}%
\pgfpathlineto{\pgfqpoint{1.820837in}{1.283373in}}%
\pgfpathlineto{\pgfqpoint{1.829709in}{1.275362in}}%
\pgfpathlineto{\pgfqpoint{1.838581in}{1.266816in}}%
\pgfpathlineto{\pgfqpoint{1.847453in}{1.257737in}}%
\pgfpathlineto{\pgfqpoint{1.856325in}{1.248123in}}%
\pgfpathlineto{\pgfqpoint{1.865197in}{1.237975in}}%
\pgfpathlineto{\pgfqpoint{1.874069in}{1.227293in}}%
\pgfpathlineto{\pgfqpoint{1.882941in}{1.216077in}}%
\pgfpathlineto{\pgfqpoint{1.891813in}{1.204327in}}%
\pgfpathlineto{\pgfqpoint{1.900685in}{1.192043in}}%
\pgfpathlineto{\pgfqpoint{1.909557in}{1.179225in}}%
\pgfpathlineto{\pgfqpoint{1.918429in}{1.165872in}}%
\pgfpathlineto{\pgfqpoint{1.927301in}{1.151986in}}%
\pgfpathlineto{\pgfqpoint{1.936173in}{1.137565in}}%
\pgfpathlineto{\pgfqpoint{1.945045in}{1.122611in}}%
\pgfpathlineto{\pgfqpoint{1.953918in}{1.107122in}}%
\pgfpathlineto{\pgfqpoint{1.962790in}{1.091099in}}%
\pgfpathlineto{\pgfqpoint{1.971662in}{1.074542in}}%
\pgfpathlineto{\pgfqpoint{1.980534in}{1.057451in}}%
\pgfpathlineto{\pgfqpoint{1.989406in}{1.039826in}}%
\pgfpathlineto{\pgfqpoint{1.998278in}{1.021667in}}%
\pgfpathlineto{\pgfqpoint{2.007150in}{1.002973in}}%
\pgfpathlineto{\pgfqpoint{2.016022in}{0.983746in}}%
\pgfpathlineto{\pgfqpoint{2.024894in}{0.963984in}}%
\pgfpathlineto{\pgfqpoint{2.033766in}{0.943689in}}%
\pgfpathlineto{\pgfqpoint{2.042638in}{0.922859in}}%
\pgfpathlineto{\pgfqpoint{2.051510in}{0.901495in}}%
\pgfpathlineto{\pgfqpoint{2.060382in}{0.879597in}}%
\pgfpathlineto{\pgfqpoint{2.069254in}{0.857165in}}%
\pgfpathlineto{\pgfqpoint{2.078126in}{0.834199in}}%
\pgfpathlineto{\pgfqpoint{2.086998in}{0.810699in}}%
\pgfpathlineto{\pgfqpoint{2.095870in}{0.786665in}}%
\pgfpathlineto{\pgfqpoint{2.104742in}{0.762096in}}%
\pgfpathlineto{\pgfqpoint{2.113615in}{0.736994in}}%
\pgfpathlineto{\pgfqpoint{2.122487in}{0.711357in}}%
\pgfpathlineto{\pgfqpoint{2.131359in}{0.685187in}}%
\pgfusepath{stroke}%
\end{pgfscope}%
\begin{pgfscope}%
\pgfpathrectangle{\pgfqpoint{0.374692in}{0.521603in}}{\pgfqpoint{2.635000in}{1.963000in}} %
\pgfusepath{clip}%
\pgfsetrectcap%
\pgfsetroundjoin%
\pgfsetlinewidth{1.003750pt}%
\definecolor{currentstroke}{rgb}{0.580392,0.403922,0.741176}%
\pgfsetstrokecolor{currentstroke}%
\pgfsetdash{}{0pt}%
\pgfpathmoveto{\pgfqpoint{1.253025in}{0.685187in}}%
\pgfpathlineto{\pgfqpoint{1.261897in}{0.678711in}}%
\pgfpathlineto{\pgfqpoint{1.270769in}{0.672502in}}%
\pgfpathlineto{\pgfqpoint{1.279641in}{0.666560in}}%
\pgfpathlineto{\pgfqpoint{1.288513in}{0.660885in}}%
\pgfpathlineto{\pgfqpoint{1.297386in}{0.655478in}}%
\pgfpathlineto{\pgfqpoint{1.306258in}{0.650337in}}%
\pgfpathlineto{\pgfqpoint{1.315130in}{0.645463in}}%
\pgfpathlineto{\pgfqpoint{1.324002in}{0.640857in}}%
\pgfpathlineto{\pgfqpoint{1.332874in}{0.636517in}}%
\pgfpathlineto{\pgfqpoint{1.341746in}{0.632445in}}%
\pgfpathlineto{\pgfqpoint{1.350618in}{0.628639in}}%
\pgfpathlineto{\pgfqpoint{1.359490in}{0.625101in}}%
\pgfpathlineto{\pgfqpoint{1.368362in}{0.621830in}}%
\pgfpathlineto{\pgfqpoint{1.377234in}{0.618825in}}%
\pgfpathlineto{\pgfqpoint{1.386106in}{0.616088in}}%
\pgfpathlineto{\pgfqpoint{1.394978in}{0.613618in}}%
\pgfpathlineto{\pgfqpoint{1.403850in}{0.611415in}}%
\pgfpathlineto{\pgfqpoint{1.412722in}{0.609479in}}%
\pgfpathlineto{\pgfqpoint{1.421594in}{0.607810in}}%
\pgfpathlineto{\pgfqpoint{1.430466in}{0.606408in}}%
\pgfpathlineto{\pgfqpoint{1.439338in}{0.605273in}}%
\pgfpathlineto{\pgfqpoint{1.448210in}{0.604405in}}%
\pgfpathlineto{\pgfqpoint{1.457083in}{0.603804in}}%
\pgfpathlineto{\pgfqpoint{1.465955in}{0.603470in}}%
\pgfpathlineto{\pgfqpoint{1.474827in}{0.603403in}}%
\pgfpathlineto{\pgfqpoint{1.483699in}{0.603604in}}%
\pgfpathlineto{\pgfqpoint{1.492571in}{0.604071in}}%
\pgfpathlineto{\pgfqpoint{1.501443in}{0.604805in}}%
\pgfpathlineto{\pgfqpoint{1.510315in}{0.605807in}}%
\pgfpathlineto{\pgfqpoint{1.519187in}{0.607075in}}%
\pgfpathlineto{\pgfqpoint{1.528059in}{0.608611in}}%
\pgfpathlineto{\pgfqpoint{1.536931in}{0.610413in}}%
\pgfpathlineto{\pgfqpoint{1.545803in}{0.612483in}}%
\pgfpathlineto{\pgfqpoint{1.554675in}{0.614820in}}%
\pgfpathlineto{\pgfqpoint{1.563547in}{0.617423in}}%
\pgfpathlineto{\pgfqpoint{1.572419in}{0.620294in}}%
\pgfpathlineto{\pgfqpoint{1.581291in}{0.623432in}}%
\pgfpathlineto{\pgfqpoint{1.590163in}{0.626837in}}%
\pgfpathlineto{\pgfqpoint{1.599035in}{0.630509in}}%
\pgfpathlineto{\pgfqpoint{1.607907in}{0.634448in}}%
\pgfpathlineto{\pgfqpoint{1.616779in}{0.638654in}}%
\pgfpathlineto{\pgfqpoint{1.625652in}{0.643127in}}%
\pgfpathlineto{\pgfqpoint{1.634524in}{0.647867in}}%
\pgfpathlineto{\pgfqpoint{1.643396in}{0.652874in}}%
\pgfpathlineto{\pgfqpoint{1.652268in}{0.658148in}}%
\pgfpathlineto{\pgfqpoint{1.661140in}{0.663689in}}%
\pgfpathlineto{\pgfqpoint{1.670012in}{0.669498in}}%
\pgfpathlineto{\pgfqpoint{1.678884in}{0.675573in}}%
\pgfpathlineto{\pgfqpoint{1.687756in}{0.681915in}}%
\pgfpathlineto{\pgfqpoint{1.696628in}{0.688525in}}%
\pgfpathlineto{\pgfqpoint{1.705500in}{0.695401in}}%
\pgfpathlineto{\pgfqpoint{1.714372in}{0.702545in}}%
\pgfpathlineto{\pgfqpoint{1.723244in}{0.709955in}}%
\pgfpathlineto{\pgfqpoint{1.732116in}{0.717633in}}%
\pgfpathlineto{\pgfqpoint{1.740988in}{0.725578in}}%
\pgfpathlineto{\pgfqpoint{1.749860in}{0.733789in}}%
\pgfpathlineto{\pgfqpoint{1.758732in}{0.742268in}}%
\pgfpathlineto{\pgfqpoint{1.767604in}{0.751014in}}%
\pgfpathlineto{\pgfqpoint{1.776476in}{0.760027in}}%
\pgfpathlineto{\pgfqpoint{1.785349in}{0.769307in}}%
\pgfpathlineto{\pgfqpoint{1.794221in}{0.778854in}}%
\pgfpathlineto{\pgfqpoint{1.803093in}{0.788668in}}%
\pgfpathlineto{\pgfqpoint{1.811965in}{0.798749in}}%
\pgfpathlineto{\pgfqpoint{1.820837in}{0.809097in}}%
\pgfpathlineto{\pgfqpoint{1.829709in}{0.819712in}}%
\pgfpathlineto{\pgfqpoint{1.838581in}{0.830594in}}%
\pgfpathlineto{\pgfqpoint{1.847453in}{0.841743in}}%
\pgfpathlineto{\pgfqpoint{1.856325in}{0.853160in}}%
\pgfpathlineto{\pgfqpoint{1.865197in}{0.864843in}}%
\pgfpathlineto{\pgfqpoint{1.874069in}{0.876793in}}%
\pgfpathlineto{\pgfqpoint{1.882941in}{0.889011in}}%
\pgfpathlineto{\pgfqpoint{1.891813in}{0.901495in}}%
\pgfpathlineto{\pgfqpoint{1.900685in}{0.914247in}}%
\pgfpathlineto{\pgfqpoint{1.909557in}{0.927265in}}%
\pgfpathlineto{\pgfqpoint{1.918429in}{0.940551in}}%
\pgfpathlineto{\pgfqpoint{1.927301in}{0.954104in}}%
\pgfpathlineto{\pgfqpoint{1.936173in}{0.967923in}}%
\pgfpathlineto{\pgfqpoint{1.945045in}{0.982010in}}%
\pgfpathlineto{\pgfqpoint{1.953918in}{0.996364in}}%
\pgfpathlineto{\pgfqpoint{1.962790in}{1.010985in}}%
\pgfpathlineto{\pgfqpoint{1.971662in}{1.025873in}}%
\pgfpathlineto{\pgfqpoint{1.980534in}{1.041028in}}%
\pgfpathlineto{\pgfqpoint{1.989406in}{1.056450in}}%
\pgfpathlineto{\pgfqpoint{1.998278in}{1.072139in}}%
\pgfpathlineto{\pgfqpoint{2.007150in}{1.088095in}}%
\pgfpathlineto{\pgfqpoint{2.016022in}{1.104318in}}%
\pgfpathlineto{\pgfqpoint{2.024894in}{1.120808in}}%
\pgfpathlineto{\pgfqpoint{2.033766in}{1.137565in}}%
\pgfpathlineto{\pgfqpoint{2.042638in}{1.154590in}}%
\pgfpathlineto{\pgfqpoint{2.051510in}{1.171881in}}%
\pgfpathlineto{\pgfqpoint{2.060382in}{1.189439in}}%
\pgfpathlineto{\pgfqpoint{2.069254in}{1.207265in}}%
\pgfpathlineto{\pgfqpoint{2.078126in}{1.225357in}}%
\pgfpathlineto{\pgfqpoint{2.086998in}{1.243717in}}%
\pgfpathlineto{\pgfqpoint{2.095870in}{1.262343in}}%
\pgfpathlineto{\pgfqpoint{2.104742in}{1.281237in}}%
\pgfpathlineto{\pgfqpoint{2.113615in}{1.300398in}}%
\pgfpathlineto{\pgfqpoint{2.122487in}{1.319825in}}%
\pgfpathlineto{\pgfqpoint{2.131359in}{1.339520in}}%
\pgfusepath{stroke}%
\end{pgfscope}%
\begin{pgfscope}%
\pgfpathrectangle{\pgfqpoint{0.374692in}{0.521603in}}{\pgfqpoint{2.635000in}{1.963000in}} %
\pgfusepath{clip}%
\pgfsetrectcap%
\pgfsetroundjoin%
\pgfsetlinewidth{1.003750pt}%
\definecolor{currentstroke}{rgb}{0.580392,0.403922,0.741176}%
\pgfsetstrokecolor{currentstroke}%
\pgfsetdash{}{0pt}%
\pgfpathmoveto{\pgfqpoint{2.131359in}{1.339520in}}%
\pgfpathlineto{\pgfqpoint{2.140231in}{1.319825in}}%
\pgfpathlineto{\pgfqpoint{2.149103in}{1.300398in}}%
\pgfpathlineto{\pgfqpoint{2.157975in}{1.281237in}}%
\pgfpathlineto{\pgfqpoint{2.166847in}{1.262343in}}%
\pgfpathlineto{\pgfqpoint{2.175719in}{1.243717in}}%
\pgfpathlineto{\pgfqpoint{2.184591in}{1.225357in}}%
\pgfpathlineto{\pgfqpoint{2.193463in}{1.207265in}}%
\pgfpathlineto{\pgfqpoint{2.202335in}{1.189439in}}%
\pgfpathlineto{\pgfqpoint{2.211207in}{1.171881in}}%
\pgfpathlineto{\pgfqpoint{2.220079in}{1.154590in}}%
\pgfpathlineto{\pgfqpoint{2.228951in}{1.137565in}}%
\pgfpathlineto{\pgfqpoint{2.237823in}{1.120808in}}%
\pgfpathlineto{\pgfqpoint{2.246695in}{1.104318in}}%
\pgfpathlineto{\pgfqpoint{2.255567in}{1.088095in}}%
\pgfpathlineto{\pgfqpoint{2.264439in}{1.072139in}}%
\pgfpathlineto{\pgfqpoint{2.273311in}{1.056450in}}%
\pgfpathlineto{\pgfqpoint{2.282184in}{1.041028in}}%
\pgfpathlineto{\pgfqpoint{2.291056in}{1.025873in}}%
\pgfpathlineto{\pgfqpoint{2.299928in}{1.010985in}}%
\pgfpathlineto{\pgfqpoint{2.308800in}{0.996364in}}%
\pgfpathlineto{\pgfqpoint{2.317672in}{0.982010in}}%
\pgfpathlineto{\pgfqpoint{2.326544in}{0.967923in}}%
\pgfpathlineto{\pgfqpoint{2.335416in}{0.954104in}}%
\pgfpathlineto{\pgfqpoint{2.344288in}{0.940551in}}%
\pgfpathlineto{\pgfqpoint{2.353160in}{0.927265in}}%
\pgfpathlineto{\pgfqpoint{2.362032in}{0.914247in}}%
\pgfpathlineto{\pgfqpoint{2.370904in}{0.901495in}}%
\pgfpathlineto{\pgfqpoint{2.379776in}{0.889011in}}%
\pgfpathlineto{\pgfqpoint{2.388648in}{0.876793in}}%
\pgfpathlineto{\pgfqpoint{2.397520in}{0.864843in}}%
\pgfpathlineto{\pgfqpoint{2.406392in}{0.853160in}}%
\pgfpathlineto{\pgfqpoint{2.415264in}{0.841743in}}%
\pgfpathlineto{\pgfqpoint{2.424136in}{0.830594in}}%
\pgfpathlineto{\pgfqpoint{2.433008in}{0.819712in}}%
\pgfpathlineto{\pgfqpoint{2.441880in}{0.809097in}}%
\pgfpathlineto{\pgfqpoint{2.450753in}{0.798749in}}%
\pgfpathlineto{\pgfqpoint{2.459625in}{0.788668in}}%
\pgfpathlineto{\pgfqpoint{2.468497in}{0.778854in}}%
\pgfpathlineto{\pgfqpoint{2.477369in}{0.769307in}}%
\pgfpathlineto{\pgfqpoint{2.486241in}{0.760027in}}%
\pgfpathlineto{\pgfqpoint{2.495113in}{0.751014in}}%
\pgfpathlineto{\pgfqpoint{2.503985in}{0.742268in}}%
\pgfpathlineto{\pgfqpoint{2.512857in}{0.733789in}}%
\pgfpathlineto{\pgfqpoint{2.521729in}{0.725578in}}%
\pgfpathlineto{\pgfqpoint{2.530601in}{0.717633in}}%
\pgfpathlineto{\pgfqpoint{2.539473in}{0.709955in}}%
\pgfpathlineto{\pgfqpoint{2.548345in}{0.702545in}}%
\pgfpathlineto{\pgfqpoint{2.557217in}{0.695401in}}%
\pgfpathlineto{\pgfqpoint{2.566089in}{0.688525in}}%
\pgfpathlineto{\pgfqpoint{2.574961in}{0.681915in}}%
\pgfpathlineto{\pgfqpoint{2.583833in}{0.675573in}}%
\pgfpathlineto{\pgfqpoint{2.592705in}{0.669498in}}%
\pgfpathlineto{\pgfqpoint{2.601577in}{0.663689in}}%
\pgfpathlineto{\pgfqpoint{2.610450in}{0.658148in}}%
\pgfpathlineto{\pgfqpoint{2.619322in}{0.652874in}}%
\pgfpathlineto{\pgfqpoint{2.628194in}{0.647867in}}%
\pgfpathlineto{\pgfqpoint{2.637066in}{0.643127in}}%
\pgfpathlineto{\pgfqpoint{2.645938in}{0.638654in}}%
\pgfpathlineto{\pgfqpoint{2.654810in}{0.634448in}}%
\pgfpathlineto{\pgfqpoint{2.663682in}{0.630509in}}%
\pgfpathlineto{\pgfqpoint{2.672554in}{0.626837in}}%
\pgfpathlineto{\pgfqpoint{2.681426in}{0.623432in}}%
\pgfpathlineto{\pgfqpoint{2.690298in}{0.620294in}}%
\pgfpathlineto{\pgfqpoint{2.699170in}{0.617423in}}%
\pgfpathlineto{\pgfqpoint{2.708042in}{0.614820in}}%
\pgfpathlineto{\pgfqpoint{2.716914in}{0.612483in}}%
\pgfpathlineto{\pgfqpoint{2.725786in}{0.610413in}}%
\pgfpathlineto{\pgfqpoint{2.734658in}{0.608611in}}%
\pgfpathlineto{\pgfqpoint{2.743530in}{0.607075in}}%
\pgfpathlineto{\pgfqpoint{2.752402in}{0.605807in}}%
\pgfpathlineto{\pgfqpoint{2.761274in}{0.604805in}}%
\pgfpathlineto{\pgfqpoint{2.770146in}{0.604071in}}%
\pgfpathlineto{\pgfqpoint{2.779019in}{0.603604in}}%
\pgfpathlineto{\pgfqpoint{2.787891in}{0.603403in}}%
\pgfpathlineto{\pgfqpoint{2.796763in}{0.603470in}}%
\pgfpathlineto{\pgfqpoint{2.805635in}{0.603804in}}%
\pgfpathlineto{\pgfqpoint{2.814507in}{0.604405in}}%
\pgfpathlineto{\pgfqpoint{2.823379in}{0.605273in}}%
\pgfpathlineto{\pgfqpoint{2.832251in}{0.606408in}}%
\pgfpathlineto{\pgfqpoint{2.841123in}{0.607810in}}%
\pgfpathlineto{\pgfqpoint{2.849995in}{0.609479in}}%
\pgfpathlineto{\pgfqpoint{2.858867in}{0.611415in}}%
\pgfpathlineto{\pgfqpoint{2.867739in}{0.613618in}}%
\pgfpathlineto{\pgfqpoint{2.876611in}{0.616088in}}%
\pgfpathlineto{\pgfqpoint{2.885483in}{0.618825in}}%
\pgfpathlineto{\pgfqpoint{2.894355in}{0.621830in}}%
\pgfpathlineto{\pgfqpoint{2.903227in}{0.625101in}}%
\pgfpathlineto{\pgfqpoint{2.912099in}{0.628639in}}%
\pgfpathlineto{\pgfqpoint{2.920971in}{0.632445in}}%
\pgfpathlineto{\pgfqpoint{2.929843in}{0.636517in}}%
\pgfpathlineto{\pgfqpoint{2.938716in}{0.640857in}}%
\pgfpathlineto{\pgfqpoint{2.947588in}{0.645463in}}%
\pgfpathlineto{\pgfqpoint{2.956460in}{0.650337in}}%
\pgfpathlineto{\pgfqpoint{2.965332in}{0.655478in}}%
\pgfpathlineto{\pgfqpoint{2.974204in}{0.660885in}}%
\pgfpathlineto{\pgfqpoint{2.983076in}{0.666560in}}%
\pgfpathlineto{\pgfqpoint{2.991948in}{0.672502in}}%
\pgfpathlineto{\pgfqpoint{3.000820in}{0.678711in}}%
\pgfpathlineto{\pgfqpoint{3.009692in}{0.685187in}}%
\pgfusepath{stroke}%
\end{pgfscope}%
\begin{pgfscope}%
\pgfpathrectangle{\pgfqpoint{0.374692in}{0.521603in}}{\pgfqpoint{2.635000in}{1.963000in}} %
\pgfusepath{clip}%
\pgfsetrectcap%
\pgfsetroundjoin%
\pgfsetlinewidth{1.003750pt}%
\definecolor{currentstroke}{rgb}{0.549020,0.337255,0.294118}%
\pgfsetstrokecolor{currentstroke}%
\pgfsetdash{}{0pt}%
\pgfpathmoveto{\pgfqpoint{2.131359in}{0.685187in}}%
\pgfpathlineto{\pgfqpoint{2.140231in}{0.711357in}}%
\pgfpathlineto{\pgfqpoint{2.149103in}{0.736994in}}%
\pgfpathlineto{\pgfqpoint{2.157975in}{0.762096in}}%
\pgfpathlineto{\pgfqpoint{2.166847in}{0.786665in}}%
\pgfpathlineto{\pgfqpoint{2.175719in}{0.810699in}}%
\pgfpathlineto{\pgfqpoint{2.184591in}{0.834199in}}%
\pgfpathlineto{\pgfqpoint{2.193463in}{0.857165in}}%
\pgfpathlineto{\pgfqpoint{2.202335in}{0.879597in}}%
\pgfpathlineto{\pgfqpoint{2.211207in}{0.901495in}}%
\pgfpathlineto{\pgfqpoint{2.220079in}{0.922859in}}%
\pgfpathlineto{\pgfqpoint{2.228951in}{0.943689in}}%
\pgfpathlineto{\pgfqpoint{2.237823in}{0.963984in}}%
\pgfpathlineto{\pgfqpoint{2.246695in}{0.983746in}}%
\pgfpathlineto{\pgfqpoint{2.255567in}{1.002973in}}%
\pgfpathlineto{\pgfqpoint{2.264439in}{1.021667in}}%
\pgfpathlineto{\pgfqpoint{2.273311in}{1.039826in}}%
\pgfpathlineto{\pgfqpoint{2.282184in}{1.057451in}}%
\pgfpathlineto{\pgfqpoint{2.291056in}{1.074542in}}%
\pgfpathlineto{\pgfqpoint{2.299928in}{1.091099in}}%
\pgfpathlineto{\pgfqpoint{2.308800in}{1.107122in}}%
\pgfpathlineto{\pgfqpoint{2.317672in}{1.122611in}}%
\pgfpathlineto{\pgfqpoint{2.326544in}{1.137565in}}%
\pgfpathlineto{\pgfqpoint{2.335416in}{1.151986in}}%
\pgfpathlineto{\pgfqpoint{2.344288in}{1.165872in}}%
\pgfpathlineto{\pgfqpoint{2.353160in}{1.179225in}}%
\pgfpathlineto{\pgfqpoint{2.362032in}{1.192043in}}%
\pgfpathlineto{\pgfqpoint{2.370904in}{1.204327in}}%
\pgfpathlineto{\pgfqpoint{2.379776in}{1.216077in}}%
\pgfpathlineto{\pgfqpoint{2.388648in}{1.227293in}}%
\pgfpathlineto{\pgfqpoint{2.397520in}{1.237975in}}%
\pgfpathlineto{\pgfqpoint{2.406392in}{1.248123in}}%
\pgfpathlineto{\pgfqpoint{2.415264in}{1.257737in}}%
\pgfpathlineto{\pgfqpoint{2.424136in}{1.266816in}}%
\pgfpathlineto{\pgfqpoint{2.433008in}{1.275362in}}%
\pgfpathlineto{\pgfqpoint{2.441880in}{1.283373in}}%
\pgfpathlineto{\pgfqpoint{2.450753in}{1.290851in}}%
\pgfpathlineto{\pgfqpoint{2.459625in}{1.297794in}}%
\pgfpathlineto{\pgfqpoint{2.468497in}{1.304203in}}%
\pgfpathlineto{\pgfqpoint{2.477369in}{1.310078in}}%
\pgfpathlineto{\pgfqpoint{2.486241in}{1.315419in}}%
\pgfpathlineto{\pgfqpoint{2.495113in}{1.320226in}}%
\pgfpathlineto{\pgfqpoint{2.503985in}{1.324499in}}%
\pgfpathlineto{\pgfqpoint{2.512857in}{1.328237in}}%
\pgfpathlineto{\pgfqpoint{2.521729in}{1.331442in}}%
\pgfpathlineto{\pgfqpoint{2.530601in}{1.334112in}}%
\pgfpathlineto{\pgfqpoint{2.539473in}{1.336249in}}%
\pgfpathlineto{\pgfqpoint{2.548345in}{1.337851in}}%
\pgfpathlineto{\pgfqpoint{2.557217in}{1.338919in}}%
\pgfpathlineto{\pgfqpoint{2.566089in}{1.339453in}}%
\pgfpathlineto{\pgfqpoint{2.574961in}{1.339453in}}%
\pgfpathlineto{\pgfqpoint{2.583833in}{1.338919in}}%
\pgfpathlineto{\pgfqpoint{2.592705in}{1.337851in}}%
\pgfpathlineto{\pgfqpoint{2.601577in}{1.336249in}}%
\pgfpathlineto{\pgfqpoint{2.610450in}{1.334112in}}%
\pgfpathlineto{\pgfqpoint{2.619322in}{1.331442in}}%
\pgfpathlineto{\pgfqpoint{2.628194in}{1.328237in}}%
\pgfpathlineto{\pgfqpoint{2.637066in}{1.324499in}}%
\pgfpathlineto{\pgfqpoint{2.645938in}{1.320226in}}%
\pgfpathlineto{\pgfqpoint{2.654810in}{1.315419in}}%
\pgfpathlineto{\pgfqpoint{2.663682in}{1.310078in}}%
\pgfpathlineto{\pgfqpoint{2.672554in}{1.304203in}}%
\pgfpathlineto{\pgfqpoint{2.681426in}{1.297794in}}%
\pgfpathlineto{\pgfqpoint{2.690298in}{1.290851in}}%
\pgfpathlineto{\pgfqpoint{2.699170in}{1.283373in}}%
\pgfpathlineto{\pgfqpoint{2.708042in}{1.275362in}}%
\pgfpathlineto{\pgfqpoint{2.716914in}{1.266816in}}%
\pgfpathlineto{\pgfqpoint{2.725786in}{1.257737in}}%
\pgfpathlineto{\pgfqpoint{2.734658in}{1.248123in}}%
\pgfpathlineto{\pgfqpoint{2.743530in}{1.237975in}}%
\pgfpathlineto{\pgfqpoint{2.752402in}{1.227293in}}%
\pgfpathlineto{\pgfqpoint{2.761274in}{1.216077in}}%
\pgfpathlineto{\pgfqpoint{2.770146in}{1.204327in}}%
\pgfpathlineto{\pgfqpoint{2.779019in}{1.192043in}}%
\pgfpathlineto{\pgfqpoint{2.787891in}{1.179225in}}%
\pgfpathlineto{\pgfqpoint{2.796763in}{1.165872in}}%
\pgfpathlineto{\pgfqpoint{2.805635in}{1.151986in}}%
\pgfpathlineto{\pgfqpoint{2.814507in}{1.137565in}}%
\pgfpathlineto{\pgfqpoint{2.823379in}{1.122611in}}%
\pgfpathlineto{\pgfqpoint{2.832251in}{1.107122in}}%
\pgfpathlineto{\pgfqpoint{2.841123in}{1.091099in}}%
\pgfpathlineto{\pgfqpoint{2.849995in}{1.074542in}}%
\pgfpathlineto{\pgfqpoint{2.858867in}{1.057451in}}%
\pgfpathlineto{\pgfqpoint{2.867739in}{1.039826in}}%
\pgfpathlineto{\pgfqpoint{2.876611in}{1.021667in}}%
\pgfpathlineto{\pgfqpoint{2.885483in}{1.002973in}}%
\pgfpathlineto{\pgfqpoint{2.894355in}{0.983746in}}%
\pgfpathlineto{\pgfqpoint{2.903227in}{0.963984in}}%
\pgfpathlineto{\pgfqpoint{2.912099in}{0.943689in}}%
\pgfpathlineto{\pgfqpoint{2.920971in}{0.922859in}}%
\pgfpathlineto{\pgfqpoint{2.929843in}{0.901495in}}%
\pgfpathlineto{\pgfqpoint{2.938716in}{0.879597in}}%
\pgfpathlineto{\pgfqpoint{2.947588in}{0.857165in}}%
\pgfpathlineto{\pgfqpoint{2.956460in}{0.834199in}}%
\pgfpathlineto{\pgfqpoint{2.965332in}{0.810699in}}%
\pgfpathlineto{\pgfqpoint{2.974204in}{0.786665in}}%
\pgfpathlineto{\pgfqpoint{2.983076in}{0.762096in}}%
\pgfpathlineto{\pgfqpoint{2.991948in}{0.736994in}}%
\pgfpathlineto{\pgfqpoint{3.000820in}{0.711357in}}%
\pgfpathlineto{\pgfqpoint{3.009692in}{0.685187in}}%
\pgfusepath{stroke}%
\end{pgfscope}%
\begin{pgfscope}%
\pgfpathrectangle{\pgfqpoint{0.374692in}{0.521603in}}{\pgfqpoint{2.635000in}{1.963000in}} %
\pgfusepath{clip}%
\pgfsetrectcap%
\pgfsetroundjoin%
\pgfsetlinewidth{1.003750pt}%
\definecolor{currentstroke}{rgb}{0.121569,0.466667,0.705882}%
\pgfsetstrokecolor{currentstroke}%
\pgfsetdash{}{0pt}%
\pgfpathmoveto{\pgfqpoint{2.131359in}{0.685187in}}%
\pgfpathlineto{\pgfqpoint{2.140231in}{0.678711in}}%
\pgfpathlineto{\pgfqpoint{2.149103in}{0.672502in}}%
\pgfpathlineto{\pgfqpoint{2.157975in}{0.666560in}}%
\pgfpathlineto{\pgfqpoint{2.166847in}{0.660885in}}%
\pgfpathlineto{\pgfqpoint{2.175719in}{0.655478in}}%
\pgfpathlineto{\pgfqpoint{2.184591in}{0.650337in}}%
\pgfpathlineto{\pgfqpoint{2.193463in}{0.645463in}}%
\pgfpathlineto{\pgfqpoint{2.202335in}{0.640857in}}%
\pgfpathlineto{\pgfqpoint{2.211207in}{0.636517in}}%
\pgfpathlineto{\pgfqpoint{2.220079in}{0.632445in}}%
\pgfpathlineto{\pgfqpoint{2.228951in}{0.628639in}}%
\pgfpathlineto{\pgfqpoint{2.237823in}{0.625101in}}%
\pgfpathlineto{\pgfqpoint{2.246695in}{0.621830in}}%
\pgfpathlineto{\pgfqpoint{2.255567in}{0.618825in}}%
\pgfpathlineto{\pgfqpoint{2.264439in}{0.616088in}}%
\pgfpathlineto{\pgfqpoint{2.273311in}{0.613618in}}%
\pgfpathlineto{\pgfqpoint{2.282184in}{0.611415in}}%
\pgfpathlineto{\pgfqpoint{2.291056in}{0.609479in}}%
\pgfpathlineto{\pgfqpoint{2.299928in}{0.607810in}}%
\pgfpathlineto{\pgfqpoint{2.308800in}{0.606408in}}%
\pgfpathlineto{\pgfqpoint{2.317672in}{0.605273in}}%
\pgfpathlineto{\pgfqpoint{2.326544in}{0.604405in}}%
\pgfpathlineto{\pgfqpoint{2.335416in}{0.603804in}}%
\pgfpathlineto{\pgfqpoint{2.344288in}{0.603470in}}%
\pgfpathlineto{\pgfqpoint{2.353160in}{0.603403in}}%
\pgfpathlineto{\pgfqpoint{2.362032in}{0.603604in}}%
\pgfpathlineto{\pgfqpoint{2.370904in}{0.604071in}}%
\pgfpathlineto{\pgfqpoint{2.379776in}{0.604805in}}%
\pgfpathlineto{\pgfqpoint{2.388648in}{0.605807in}}%
\pgfpathlineto{\pgfqpoint{2.397520in}{0.607075in}}%
\pgfpathlineto{\pgfqpoint{2.406392in}{0.608611in}}%
\pgfpathlineto{\pgfqpoint{2.415264in}{0.610413in}}%
\pgfpathlineto{\pgfqpoint{2.424136in}{0.612483in}}%
\pgfpathlineto{\pgfqpoint{2.433008in}{0.614820in}}%
\pgfpathlineto{\pgfqpoint{2.441880in}{0.617423in}}%
\pgfpathlineto{\pgfqpoint{2.450753in}{0.620294in}}%
\pgfpathlineto{\pgfqpoint{2.459625in}{0.623432in}}%
\pgfpathlineto{\pgfqpoint{2.468497in}{0.626837in}}%
\pgfpathlineto{\pgfqpoint{2.477369in}{0.630509in}}%
\pgfpathlineto{\pgfqpoint{2.486241in}{0.634448in}}%
\pgfpathlineto{\pgfqpoint{2.495113in}{0.638654in}}%
\pgfpathlineto{\pgfqpoint{2.503985in}{0.643127in}}%
\pgfpathlineto{\pgfqpoint{2.512857in}{0.647867in}}%
\pgfpathlineto{\pgfqpoint{2.521729in}{0.652874in}}%
\pgfpathlineto{\pgfqpoint{2.530601in}{0.658148in}}%
\pgfpathlineto{\pgfqpoint{2.539473in}{0.663689in}}%
\pgfpathlineto{\pgfqpoint{2.548345in}{0.669498in}}%
\pgfpathlineto{\pgfqpoint{2.557217in}{0.675573in}}%
\pgfpathlineto{\pgfqpoint{2.566089in}{0.681915in}}%
\pgfpathlineto{\pgfqpoint{2.574961in}{0.688525in}}%
\pgfpathlineto{\pgfqpoint{2.583833in}{0.695401in}}%
\pgfpathlineto{\pgfqpoint{2.592705in}{0.702545in}}%
\pgfpathlineto{\pgfqpoint{2.601577in}{0.709955in}}%
\pgfpathlineto{\pgfqpoint{2.610450in}{0.717633in}}%
\pgfpathlineto{\pgfqpoint{2.619322in}{0.725578in}}%
\pgfpathlineto{\pgfqpoint{2.628194in}{0.733789in}}%
\pgfpathlineto{\pgfqpoint{2.637066in}{0.742268in}}%
\pgfpathlineto{\pgfqpoint{2.645938in}{0.751014in}}%
\pgfpathlineto{\pgfqpoint{2.654810in}{0.760027in}}%
\pgfpathlineto{\pgfqpoint{2.663682in}{0.769307in}}%
\pgfpathlineto{\pgfqpoint{2.672554in}{0.778854in}}%
\pgfpathlineto{\pgfqpoint{2.681426in}{0.788668in}}%
\pgfpathlineto{\pgfqpoint{2.690298in}{0.798749in}}%
\pgfpathlineto{\pgfqpoint{2.699170in}{0.809097in}}%
\pgfpathlineto{\pgfqpoint{2.708042in}{0.819712in}}%
\pgfpathlineto{\pgfqpoint{2.716914in}{0.830594in}}%
\pgfpathlineto{\pgfqpoint{2.725786in}{0.841743in}}%
\pgfpathlineto{\pgfqpoint{2.734658in}{0.853160in}}%
\pgfpathlineto{\pgfqpoint{2.743530in}{0.864843in}}%
\pgfpathlineto{\pgfqpoint{2.752402in}{0.876793in}}%
\pgfpathlineto{\pgfqpoint{2.761274in}{0.889011in}}%
\pgfpathlineto{\pgfqpoint{2.770146in}{0.901495in}}%
\pgfpathlineto{\pgfqpoint{2.779019in}{0.914247in}}%
\pgfpathlineto{\pgfqpoint{2.787891in}{0.927265in}}%
\pgfpathlineto{\pgfqpoint{2.796763in}{0.940551in}}%
\pgfpathlineto{\pgfqpoint{2.805635in}{0.954104in}}%
\pgfpathlineto{\pgfqpoint{2.814507in}{0.967923in}}%
\pgfpathlineto{\pgfqpoint{2.823379in}{0.982010in}}%
\pgfpathlineto{\pgfqpoint{2.832251in}{0.996364in}}%
\pgfpathlineto{\pgfqpoint{2.841123in}{1.010985in}}%
\pgfpathlineto{\pgfqpoint{2.849995in}{1.025873in}}%
\pgfpathlineto{\pgfqpoint{2.858867in}{1.041028in}}%
\pgfpathlineto{\pgfqpoint{2.867739in}{1.056450in}}%
\pgfpathlineto{\pgfqpoint{2.876611in}{1.072139in}}%
\pgfpathlineto{\pgfqpoint{2.885483in}{1.088095in}}%
\pgfpathlineto{\pgfqpoint{2.894355in}{1.104318in}}%
\pgfpathlineto{\pgfqpoint{2.903227in}{1.120808in}}%
\pgfpathlineto{\pgfqpoint{2.912099in}{1.137565in}}%
\pgfpathlineto{\pgfqpoint{2.920971in}{1.154590in}}%
\pgfpathlineto{\pgfqpoint{2.929843in}{1.171881in}}%
\pgfpathlineto{\pgfqpoint{2.938716in}{1.189439in}}%
\pgfpathlineto{\pgfqpoint{2.947588in}{1.207265in}}%
\pgfpathlineto{\pgfqpoint{2.956460in}{1.225357in}}%
\pgfpathlineto{\pgfqpoint{2.965332in}{1.243717in}}%
\pgfpathlineto{\pgfqpoint{2.974204in}{1.262343in}}%
\pgfpathlineto{\pgfqpoint{2.983076in}{1.281237in}}%
\pgfpathlineto{\pgfqpoint{2.991948in}{1.300398in}}%
\pgfpathlineto{\pgfqpoint{3.000820in}{1.319825in}}%
\pgfpathlineto{\pgfqpoint{3.009692in}{1.339520in}}%
\pgfusepath{stroke}%
\end{pgfscope}%
\begin{pgfscope}%
\pgfpathrectangle{\pgfqpoint{0.374692in}{0.521603in}}{\pgfqpoint{2.635000in}{1.963000in}} %
\pgfusepath{clip}%
\pgfsetbuttcap%
\pgfsetroundjoin%
\pgfsetlinewidth{1.505625pt}%
\definecolor{currentstroke}{rgb}{0.000000,0.000000,0.000000}%
\pgfsetstrokecolor{currentstroke}%
\pgfsetdash{{5.550000pt}{2.400000pt}}{0.000000pt}%
\pgfpathmoveto{\pgfqpoint{2.131359in}{0.521603in}}%
\pgfpathlineto{\pgfqpoint{2.131359in}{0.630659in}}%
\pgfpathlineto{\pgfqpoint{2.131359in}{0.739714in}}%
\pgfpathlineto{\pgfqpoint{2.131359in}{0.848770in}}%
\pgfpathlineto{\pgfqpoint{2.131359in}{0.957826in}}%
\pgfpathlineto{\pgfqpoint{2.131359in}{1.066881in}}%
\pgfpathlineto{\pgfqpoint{2.131359in}{1.175937in}}%
\pgfpathlineto{\pgfqpoint{2.131359in}{1.284992in}}%
\pgfpathlineto{\pgfqpoint{2.131359in}{1.394048in}}%
\pgfpathlineto{\pgfqpoint{2.131359in}{1.503103in}}%
\pgfusepath{stroke}%
\end{pgfscope}%
\begin{pgfscope}%
\pgfpathrectangle{\pgfqpoint{0.374692in}{0.521603in}}{\pgfqpoint{2.635000in}{1.963000in}} %
\pgfusepath{clip}%
\pgfsetbuttcap%
\pgfsetroundjoin%
\definecolor{currentfill}{rgb}{1.000000,0.000000,0.000000}%
\pgfsetfillcolor{currentfill}%
\pgfsetlinewidth{1.003750pt}%
\definecolor{currentstroke}{rgb}{1.000000,0.000000,0.000000}%
\pgfsetstrokecolor{currentstroke}%
\pgfsetdash{}{0pt}%
\pgfsys@defobject{currentmarker}{\pgfqpoint{-0.020833in}{-0.020833in}}{\pgfqpoint{0.020833in}{0.020833in}}{%
\pgfpathmoveto{\pgfqpoint{0.000000in}{-0.020833in}}%
\pgfpathcurveto{\pgfqpoint{0.005525in}{-0.020833in}}{\pgfqpoint{0.010825in}{-0.018638in}}{\pgfqpoint{0.014731in}{-0.014731in}}%
\pgfpathcurveto{\pgfqpoint{0.018638in}{-0.010825in}}{\pgfqpoint{0.020833in}{-0.005525in}}{\pgfqpoint{0.020833in}{0.000000in}}%
\pgfpathcurveto{\pgfqpoint{0.020833in}{0.005525in}}{\pgfqpoint{0.018638in}{0.010825in}}{\pgfqpoint{0.014731in}{0.014731in}}%
\pgfpathcurveto{\pgfqpoint{0.010825in}{0.018638in}}{\pgfqpoint{0.005525in}{0.020833in}}{\pgfqpoint{0.000000in}{0.020833in}}%
\pgfpathcurveto{\pgfqpoint{-0.005525in}{0.020833in}}{\pgfqpoint{-0.010825in}{0.018638in}}{\pgfqpoint{-0.014731in}{0.014731in}}%
\pgfpathcurveto{\pgfqpoint{-0.018638in}{0.010825in}}{\pgfqpoint{-0.020833in}{0.005525in}}{\pgfqpoint{-0.020833in}{0.000000in}}%
\pgfpathcurveto{\pgfqpoint{-0.020833in}{-0.005525in}}{\pgfqpoint{-0.018638in}{-0.010825in}}{\pgfqpoint{-0.014731in}{-0.014731in}}%
\pgfpathcurveto{\pgfqpoint{-0.010825in}{-0.018638in}}{\pgfqpoint{-0.005525in}{-0.020833in}}{\pgfqpoint{0.000000in}{-0.020833in}}%
\pgfpathclose%
\pgfusepath{stroke,fill}%
}%
\begin{pgfscope}%
\pgfsys@transformshift{0.374692in}{0.685187in}%
\pgfsys@useobject{currentmarker}{}%
\end{pgfscope}%
\begin{pgfscope}%
\pgfsys@transformshift{0.813859in}{0.685187in}%
\pgfsys@useobject{currentmarker}{}%
\end{pgfscope}%
\begin{pgfscope}%
\pgfsys@transformshift{1.253025in}{0.685187in}%
\pgfsys@useobject{currentmarker}{}%
\end{pgfscope}%
\end{pgfscope}%
\begin{pgfscope}%
\pgfpathrectangle{\pgfqpoint{0.374692in}{0.521603in}}{\pgfqpoint{2.635000in}{1.963000in}} %
\pgfusepath{clip}%
\pgfsetbuttcap%
\pgfsetroundjoin%
\definecolor{currentfill}{rgb}{1.000000,0.000000,0.000000}%
\pgfsetfillcolor{currentfill}%
\pgfsetlinewidth{1.003750pt}%
\definecolor{currentstroke}{rgb}{1.000000,0.000000,0.000000}%
\pgfsetstrokecolor{currentstroke}%
\pgfsetdash{}{0pt}%
\pgfsys@defobject{currentmarker}{\pgfqpoint{-0.020833in}{-0.020833in}}{\pgfqpoint{0.020833in}{0.020833in}}{%
\pgfpathmoveto{\pgfqpoint{0.000000in}{-0.020833in}}%
\pgfpathcurveto{\pgfqpoint{0.005525in}{-0.020833in}}{\pgfqpoint{0.010825in}{-0.018638in}}{\pgfqpoint{0.014731in}{-0.014731in}}%
\pgfpathcurveto{\pgfqpoint{0.018638in}{-0.010825in}}{\pgfqpoint{0.020833in}{-0.005525in}}{\pgfqpoint{0.020833in}{0.000000in}}%
\pgfpathcurveto{\pgfqpoint{0.020833in}{0.005525in}}{\pgfqpoint{0.018638in}{0.010825in}}{\pgfqpoint{0.014731in}{0.014731in}}%
\pgfpathcurveto{\pgfqpoint{0.010825in}{0.018638in}}{\pgfqpoint{0.005525in}{0.020833in}}{\pgfqpoint{0.000000in}{0.020833in}}%
\pgfpathcurveto{\pgfqpoint{-0.005525in}{0.020833in}}{\pgfqpoint{-0.010825in}{0.018638in}}{\pgfqpoint{-0.014731in}{0.014731in}}%
\pgfpathcurveto{\pgfqpoint{-0.018638in}{0.010825in}}{\pgfqpoint{-0.020833in}{0.005525in}}{\pgfqpoint{-0.020833in}{0.000000in}}%
\pgfpathcurveto{\pgfqpoint{-0.020833in}{-0.005525in}}{\pgfqpoint{-0.018638in}{-0.010825in}}{\pgfqpoint{-0.014731in}{-0.014731in}}%
\pgfpathcurveto{\pgfqpoint{-0.010825in}{-0.018638in}}{\pgfqpoint{-0.005525in}{-0.020833in}}{\pgfqpoint{0.000000in}{-0.020833in}}%
\pgfpathclose%
\pgfusepath{stroke,fill}%
}%
\begin{pgfscope}%
\pgfsys@transformshift{0.374692in}{0.685187in}%
\pgfsys@useobject{currentmarker}{}%
\end{pgfscope}%
\begin{pgfscope}%
\pgfsys@transformshift{0.813859in}{0.685187in}%
\pgfsys@useobject{currentmarker}{}%
\end{pgfscope}%
\begin{pgfscope}%
\pgfsys@transformshift{1.253025in}{0.685187in}%
\pgfsys@useobject{currentmarker}{}%
\end{pgfscope}%
\end{pgfscope}%
\begin{pgfscope}%
\pgfpathrectangle{\pgfqpoint{0.374692in}{0.521603in}}{\pgfqpoint{2.635000in}{1.963000in}} %
\pgfusepath{clip}%
\pgfsetbuttcap%
\pgfsetroundjoin%
\definecolor{currentfill}{rgb}{1.000000,0.000000,0.000000}%
\pgfsetfillcolor{currentfill}%
\pgfsetlinewidth{1.003750pt}%
\definecolor{currentstroke}{rgb}{1.000000,0.000000,0.000000}%
\pgfsetstrokecolor{currentstroke}%
\pgfsetdash{}{0pt}%
\pgfsys@defobject{currentmarker}{\pgfqpoint{-0.020833in}{-0.020833in}}{\pgfqpoint{0.020833in}{0.020833in}}{%
\pgfpathmoveto{\pgfqpoint{0.000000in}{-0.020833in}}%
\pgfpathcurveto{\pgfqpoint{0.005525in}{-0.020833in}}{\pgfqpoint{0.010825in}{-0.018638in}}{\pgfqpoint{0.014731in}{-0.014731in}}%
\pgfpathcurveto{\pgfqpoint{0.018638in}{-0.010825in}}{\pgfqpoint{0.020833in}{-0.005525in}}{\pgfqpoint{0.020833in}{0.000000in}}%
\pgfpathcurveto{\pgfqpoint{0.020833in}{0.005525in}}{\pgfqpoint{0.018638in}{0.010825in}}{\pgfqpoint{0.014731in}{0.014731in}}%
\pgfpathcurveto{\pgfqpoint{0.010825in}{0.018638in}}{\pgfqpoint{0.005525in}{0.020833in}}{\pgfqpoint{0.000000in}{0.020833in}}%
\pgfpathcurveto{\pgfqpoint{-0.005525in}{0.020833in}}{\pgfqpoint{-0.010825in}{0.018638in}}{\pgfqpoint{-0.014731in}{0.014731in}}%
\pgfpathcurveto{\pgfqpoint{-0.018638in}{0.010825in}}{\pgfqpoint{-0.020833in}{0.005525in}}{\pgfqpoint{-0.020833in}{0.000000in}}%
\pgfpathcurveto{\pgfqpoint{-0.020833in}{-0.005525in}}{\pgfqpoint{-0.018638in}{-0.010825in}}{\pgfqpoint{-0.014731in}{-0.014731in}}%
\pgfpathcurveto{\pgfqpoint{-0.010825in}{-0.018638in}}{\pgfqpoint{-0.005525in}{-0.020833in}}{\pgfqpoint{0.000000in}{-0.020833in}}%
\pgfpathclose%
\pgfusepath{stroke,fill}%
}%
\begin{pgfscope}%
\pgfsys@transformshift{1.253025in}{0.685187in}%
\pgfsys@useobject{currentmarker}{}%
\end{pgfscope}%
\begin{pgfscope}%
\pgfsys@transformshift{1.692192in}{0.685187in}%
\pgfsys@useobject{currentmarker}{}%
\end{pgfscope}%
\begin{pgfscope}%
\pgfsys@transformshift{2.131359in}{0.685187in}%
\pgfsys@useobject{currentmarker}{}%
\end{pgfscope}%
\end{pgfscope}%
\begin{pgfscope}%
\pgfpathrectangle{\pgfqpoint{0.374692in}{0.521603in}}{\pgfqpoint{2.635000in}{1.963000in}} %
\pgfusepath{clip}%
\pgfsetbuttcap%
\pgfsetroundjoin%
\definecolor{currentfill}{rgb}{1.000000,0.000000,0.000000}%
\pgfsetfillcolor{currentfill}%
\pgfsetlinewidth{1.003750pt}%
\definecolor{currentstroke}{rgb}{1.000000,0.000000,0.000000}%
\pgfsetstrokecolor{currentstroke}%
\pgfsetdash{}{0pt}%
\pgfsys@defobject{currentmarker}{\pgfqpoint{-0.020833in}{-0.020833in}}{\pgfqpoint{0.020833in}{0.020833in}}{%
\pgfpathmoveto{\pgfqpoint{0.000000in}{-0.020833in}}%
\pgfpathcurveto{\pgfqpoint{0.005525in}{-0.020833in}}{\pgfqpoint{0.010825in}{-0.018638in}}{\pgfqpoint{0.014731in}{-0.014731in}}%
\pgfpathcurveto{\pgfqpoint{0.018638in}{-0.010825in}}{\pgfqpoint{0.020833in}{-0.005525in}}{\pgfqpoint{0.020833in}{0.000000in}}%
\pgfpathcurveto{\pgfqpoint{0.020833in}{0.005525in}}{\pgfqpoint{0.018638in}{0.010825in}}{\pgfqpoint{0.014731in}{0.014731in}}%
\pgfpathcurveto{\pgfqpoint{0.010825in}{0.018638in}}{\pgfqpoint{0.005525in}{0.020833in}}{\pgfqpoint{0.000000in}{0.020833in}}%
\pgfpathcurveto{\pgfqpoint{-0.005525in}{0.020833in}}{\pgfqpoint{-0.010825in}{0.018638in}}{\pgfqpoint{-0.014731in}{0.014731in}}%
\pgfpathcurveto{\pgfqpoint{-0.018638in}{0.010825in}}{\pgfqpoint{-0.020833in}{0.005525in}}{\pgfqpoint{-0.020833in}{0.000000in}}%
\pgfpathcurveto{\pgfqpoint{-0.020833in}{-0.005525in}}{\pgfqpoint{-0.018638in}{-0.010825in}}{\pgfqpoint{-0.014731in}{-0.014731in}}%
\pgfpathcurveto{\pgfqpoint{-0.010825in}{-0.018638in}}{\pgfqpoint{-0.005525in}{-0.020833in}}{\pgfqpoint{0.000000in}{-0.020833in}}%
\pgfpathclose%
\pgfusepath{stroke,fill}%
}%
\begin{pgfscope}%
\pgfsys@transformshift{1.253025in}{0.685187in}%
\pgfsys@useobject{currentmarker}{}%
\end{pgfscope}%
\begin{pgfscope}%
\pgfsys@transformshift{1.692192in}{0.685187in}%
\pgfsys@useobject{currentmarker}{}%
\end{pgfscope}%
\begin{pgfscope}%
\pgfsys@transformshift{2.131359in}{0.685187in}%
\pgfsys@useobject{currentmarker}{}%
\end{pgfscope}%
\end{pgfscope}%
\begin{pgfscope}%
\pgfpathrectangle{\pgfqpoint{0.374692in}{0.521603in}}{\pgfqpoint{2.635000in}{1.963000in}} %
\pgfusepath{clip}%
\pgfsetbuttcap%
\pgfsetroundjoin%
\definecolor{currentfill}{rgb}{0.000000,0.000000,0.000000}%
\pgfsetfillcolor{currentfill}%
\pgfsetlinewidth{1.003750pt}%
\definecolor{currentstroke}{rgb}{0.000000,0.000000,0.000000}%
\pgfsetstrokecolor{currentstroke}%
\pgfsetdash{}{0pt}%
\pgfsys@defobject{currentmarker}{\pgfqpoint{-0.020833in}{-0.020833in}}{\pgfqpoint{0.020833in}{0.020833in}}{%
\pgfpathmoveto{\pgfqpoint{0.000000in}{-0.020833in}}%
\pgfpathcurveto{\pgfqpoint{0.005525in}{-0.020833in}}{\pgfqpoint{0.010825in}{-0.018638in}}{\pgfqpoint{0.014731in}{-0.014731in}}%
\pgfpathcurveto{\pgfqpoint{0.018638in}{-0.010825in}}{\pgfqpoint{0.020833in}{-0.005525in}}{\pgfqpoint{0.020833in}{0.000000in}}%
\pgfpathcurveto{\pgfqpoint{0.020833in}{0.005525in}}{\pgfqpoint{0.018638in}{0.010825in}}{\pgfqpoint{0.014731in}{0.014731in}}%
\pgfpathcurveto{\pgfqpoint{0.010825in}{0.018638in}}{\pgfqpoint{0.005525in}{0.020833in}}{\pgfqpoint{0.000000in}{0.020833in}}%
\pgfpathcurveto{\pgfqpoint{-0.005525in}{0.020833in}}{\pgfqpoint{-0.010825in}{0.018638in}}{\pgfqpoint{-0.014731in}{0.014731in}}%
\pgfpathcurveto{\pgfqpoint{-0.018638in}{0.010825in}}{\pgfqpoint{-0.020833in}{0.005525in}}{\pgfqpoint{-0.020833in}{0.000000in}}%
\pgfpathcurveto{\pgfqpoint{-0.020833in}{-0.005525in}}{\pgfqpoint{-0.018638in}{-0.010825in}}{\pgfqpoint{-0.014731in}{-0.014731in}}%
\pgfpathcurveto{\pgfqpoint{-0.010825in}{-0.018638in}}{\pgfqpoint{-0.005525in}{-0.020833in}}{\pgfqpoint{0.000000in}{-0.020833in}}%
\pgfpathclose%
\pgfusepath{stroke,fill}%
}%
\begin{pgfscope}%
\pgfsys@transformshift{2.131359in}{0.685187in}%
\pgfsys@useobject{currentmarker}{}%
\end{pgfscope}%
\begin{pgfscope}%
\pgfsys@transformshift{2.570525in}{0.685187in}%
\pgfsys@useobject{currentmarker}{}%
\end{pgfscope}%
\end{pgfscope}%
\begin{pgfscope}%
\pgfpathrectangle{\pgfqpoint{0.374692in}{0.521603in}}{\pgfqpoint{2.635000in}{1.963000in}} %
\pgfusepath{clip}%
\pgfsetbuttcap%
\pgfsetroundjoin%
\definecolor{currentfill}{rgb}{1.000000,0.000000,0.000000}%
\pgfsetfillcolor{currentfill}%
\pgfsetlinewidth{1.003750pt}%
\definecolor{currentstroke}{rgb}{1.000000,0.000000,0.000000}%
\pgfsetstrokecolor{currentstroke}%
\pgfsetdash{}{0pt}%
\pgfsys@defobject{currentmarker}{\pgfqpoint{-0.020833in}{-0.020833in}}{\pgfqpoint{0.020833in}{0.020833in}}{%
\pgfpathmoveto{\pgfqpoint{0.000000in}{-0.020833in}}%
\pgfpathcurveto{\pgfqpoint{0.005525in}{-0.020833in}}{\pgfqpoint{0.010825in}{-0.018638in}}{\pgfqpoint{0.014731in}{-0.014731in}}%
\pgfpathcurveto{\pgfqpoint{0.018638in}{-0.010825in}}{\pgfqpoint{0.020833in}{-0.005525in}}{\pgfqpoint{0.020833in}{0.000000in}}%
\pgfpathcurveto{\pgfqpoint{0.020833in}{0.005525in}}{\pgfqpoint{0.018638in}{0.010825in}}{\pgfqpoint{0.014731in}{0.014731in}}%
\pgfpathcurveto{\pgfqpoint{0.010825in}{0.018638in}}{\pgfqpoint{0.005525in}{0.020833in}}{\pgfqpoint{0.000000in}{0.020833in}}%
\pgfpathcurveto{\pgfqpoint{-0.005525in}{0.020833in}}{\pgfqpoint{-0.010825in}{0.018638in}}{\pgfqpoint{-0.014731in}{0.014731in}}%
\pgfpathcurveto{\pgfqpoint{-0.018638in}{0.010825in}}{\pgfqpoint{-0.020833in}{0.005525in}}{\pgfqpoint{-0.020833in}{0.000000in}}%
\pgfpathcurveto{\pgfqpoint{-0.020833in}{-0.005525in}}{\pgfqpoint{-0.018638in}{-0.010825in}}{\pgfqpoint{-0.014731in}{-0.014731in}}%
\pgfpathcurveto{\pgfqpoint{-0.010825in}{-0.018638in}}{\pgfqpoint{-0.005525in}{-0.020833in}}{\pgfqpoint{0.000000in}{-0.020833in}}%
\pgfpathclose%
\pgfusepath{stroke,fill}%
}%
\begin{pgfscope}%
\pgfsys@transformshift{2.131359in}{0.685187in}%
\pgfsys@useobject{currentmarker}{}%
\end{pgfscope}%
\begin{pgfscope}%
\pgfsys@transformshift{2.570525in}{0.685187in}%
\pgfsys@useobject{currentmarker}{}%
\end{pgfscope}%
\end{pgfscope}%
\begin{pgfscope}%
\pgfsetrectcap%
\pgfsetmiterjoin%
\pgfsetlinewidth{0.803000pt}%
\definecolor{currentstroke}{rgb}{0.000000,0.000000,0.000000}%
\pgfsetstrokecolor{currentstroke}%
\pgfsetdash{}{0pt}%
\pgfpathmoveto{\pgfqpoint{0.374692in}{0.521603in}}%
\pgfpathlineto{\pgfqpoint{0.374692in}{2.484603in}}%
\pgfusepath{stroke}%
\end{pgfscope}%
\begin{pgfscope}%
\pgfsetrectcap%
\pgfsetmiterjoin%
\pgfsetlinewidth{0.803000pt}%
\definecolor{currentstroke}{rgb}{0.000000,0.000000,0.000000}%
\pgfsetstrokecolor{currentstroke}%
\pgfsetdash{}{0pt}%
\pgfpathmoveto{\pgfqpoint{3.009692in}{0.521603in}}%
\pgfpathlineto{\pgfqpoint{3.009692in}{2.484603in}}%
\pgfusepath{stroke}%
\end{pgfscope}%
\begin{pgfscope}%
\pgfsetrectcap%
\pgfsetmiterjoin%
\pgfsetlinewidth{0.803000pt}%
\definecolor{currentstroke}{rgb}{0.000000,0.000000,0.000000}%
\pgfsetstrokecolor{currentstroke}%
\pgfsetdash{}{0pt}%
\pgfpathmoveto{\pgfqpoint{0.374692in}{0.521603in}}%
\pgfpathlineto{\pgfqpoint{3.009692in}{0.521603in}}%
\pgfusepath{stroke}%
\end{pgfscope}%
\begin{pgfscope}%
\pgfsetrectcap%
\pgfsetmiterjoin%
\pgfsetlinewidth{0.803000pt}%
\definecolor{currentstroke}{rgb}{0.000000,0.000000,0.000000}%
\pgfsetstrokecolor{currentstroke}%
\pgfsetdash{}{0pt}%
\pgfpathmoveto{\pgfqpoint{0.374692in}{2.484603in}}%
\pgfpathlineto{\pgfqpoint{3.009692in}{2.484603in}}%
\pgfusepath{stroke}%
\end{pgfscope}%
\begin{pgfscope}%
\pgfsetbuttcap%
\pgfsetmiterjoin%
\definecolor{currentfill}{rgb}{1.000000,1.000000,1.000000}%
\pgfsetfillcolor{currentfill}%
\pgfsetfillopacity{0.800000}%
\pgfsetlinewidth{1.003750pt}%
\definecolor{currentstroke}{rgb}{0.800000,0.800000,0.800000}%
\pgfsetstrokecolor{currentstroke}%
\pgfsetstrokeopacity{0.800000}%
\pgfsetdash{}{0pt}%
\pgfpathmoveto{\pgfqpoint{1.450976in}{1.759954in}}%
\pgfpathlineto{\pgfqpoint{2.912470in}{1.759954in}}%
\pgfpathquadraticcurveto{\pgfqpoint{2.940247in}{1.759954in}}{\pgfqpoint{2.940247in}{1.787732in}}%
\pgfpathlineto{\pgfqpoint{2.940247in}{2.387381in}}%
\pgfpathquadraticcurveto{\pgfqpoint{2.940247in}{2.415159in}}{\pgfqpoint{2.912470in}{2.415159in}}%
\pgfpathlineto{\pgfqpoint{1.450976in}{2.415159in}}%
\pgfpathquadraticcurveto{\pgfqpoint{1.423198in}{2.415159in}}{\pgfqpoint{1.423198in}{2.387381in}}%
\pgfpathlineto{\pgfqpoint{1.423198in}{1.787732in}}%
\pgfpathquadraticcurveto{\pgfqpoint{1.423198in}{1.759954in}}{\pgfqpoint{1.450976in}{1.759954in}}%
\pgfpathclose%
\pgfusepath{stroke,fill}%
\end{pgfscope}%
\begin{pgfscope}%
\pgfsetbuttcap%
\pgfsetroundjoin%
\pgfsetlinewidth{1.505625pt}%
\definecolor{currentstroke}{rgb}{0.000000,0.000000,0.000000}%
\pgfsetstrokecolor{currentstroke}%
\pgfsetdash{{5.550000pt}{2.400000pt}}{0.000000pt}%
\pgfpathmoveto{\pgfqpoint{1.478754in}{2.302691in}}%
\pgfpathlineto{\pgfqpoint{1.756532in}{2.302691in}}%
\pgfusepath{stroke}%
\end{pgfscope}%
\begin{pgfscope}%
\pgftext[x=1.867643in,y=2.254080in,left,base]{\rmfamily\fontsize{10.000000}{12.000000}\selectfont el. boundaries}%
\end{pgfscope}%
\begin{pgfscope}%
\pgfsetbuttcap%
\pgfsetroundjoin%
\definecolor{currentfill}{rgb}{0.000000,0.000000,0.000000}%
\pgfsetfillcolor{currentfill}%
\pgfsetlinewidth{1.003750pt}%
\definecolor{currentstroke}{rgb}{0.000000,0.000000,0.000000}%
\pgfsetstrokecolor{currentstroke}%
\pgfsetdash{}{0pt}%
\pgfsys@defobject{currentmarker}{\pgfqpoint{-0.020833in}{-0.020833in}}{\pgfqpoint{0.020833in}{0.020833in}}{%
\pgfpathmoveto{\pgfqpoint{0.000000in}{-0.020833in}}%
\pgfpathcurveto{\pgfqpoint{0.005525in}{-0.020833in}}{\pgfqpoint{0.010825in}{-0.018638in}}{\pgfqpoint{0.014731in}{-0.014731in}}%
\pgfpathcurveto{\pgfqpoint{0.018638in}{-0.010825in}}{\pgfqpoint{0.020833in}{-0.005525in}}{\pgfqpoint{0.020833in}{0.000000in}}%
\pgfpathcurveto{\pgfqpoint{0.020833in}{0.005525in}}{\pgfqpoint{0.018638in}{0.010825in}}{\pgfqpoint{0.014731in}{0.014731in}}%
\pgfpathcurveto{\pgfqpoint{0.010825in}{0.018638in}}{\pgfqpoint{0.005525in}{0.020833in}}{\pgfqpoint{0.000000in}{0.020833in}}%
\pgfpathcurveto{\pgfqpoint{-0.005525in}{0.020833in}}{\pgfqpoint{-0.010825in}{0.018638in}}{\pgfqpoint{-0.014731in}{0.014731in}}%
\pgfpathcurveto{\pgfqpoint{-0.018638in}{0.010825in}}{\pgfqpoint{-0.020833in}{0.005525in}}{\pgfqpoint{-0.020833in}{0.000000in}}%
\pgfpathcurveto{\pgfqpoint{-0.020833in}{-0.005525in}}{\pgfqpoint{-0.018638in}{-0.010825in}}{\pgfqpoint{-0.014731in}{-0.014731in}}%
\pgfpathcurveto{\pgfqpoint{-0.010825in}{-0.018638in}}{\pgfqpoint{-0.005525in}{-0.020833in}}{\pgfqpoint{0.000000in}{-0.020833in}}%
\pgfpathclose%
\pgfusepath{stroke,fill}%
}%
\begin{pgfscope}%
\pgfsys@transformshift{1.617643in}{2.098834in}%
\pgfsys@useobject{currentmarker}{}%
\end{pgfscope}%
\end{pgfscope}%
\begin{pgfscope}%
\pgftext[x=1.867643in,y=2.050223in,left,base]{\rmfamily\fontsize{10.000000}{12.000000}\selectfont local knots \(\displaystyle s_m\)}%
\end{pgfscope}%
\begin{pgfscope}%
\pgfsetbuttcap%
\pgfsetroundjoin%
\definecolor{currentfill}{rgb}{1.000000,0.000000,0.000000}%
\pgfsetfillcolor{currentfill}%
\pgfsetlinewidth{1.003750pt}%
\definecolor{currentstroke}{rgb}{1.000000,0.000000,0.000000}%
\pgfsetstrokecolor{currentstroke}%
\pgfsetdash{}{0pt}%
\pgfsys@defobject{currentmarker}{\pgfqpoint{-0.020833in}{-0.020833in}}{\pgfqpoint{0.020833in}{0.020833in}}{%
\pgfpathmoveto{\pgfqpoint{0.000000in}{-0.020833in}}%
\pgfpathcurveto{\pgfqpoint{0.005525in}{-0.020833in}}{\pgfqpoint{0.010825in}{-0.018638in}}{\pgfqpoint{0.014731in}{-0.014731in}}%
\pgfpathcurveto{\pgfqpoint{0.018638in}{-0.010825in}}{\pgfqpoint{0.020833in}{-0.005525in}}{\pgfqpoint{0.020833in}{0.000000in}}%
\pgfpathcurveto{\pgfqpoint{0.020833in}{0.005525in}}{\pgfqpoint{0.018638in}{0.010825in}}{\pgfqpoint{0.014731in}{0.014731in}}%
\pgfpathcurveto{\pgfqpoint{0.010825in}{0.018638in}}{\pgfqpoint{0.005525in}{0.020833in}}{\pgfqpoint{0.000000in}{0.020833in}}%
\pgfpathcurveto{\pgfqpoint{-0.005525in}{0.020833in}}{\pgfqpoint{-0.010825in}{0.018638in}}{\pgfqpoint{-0.014731in}{0.014731in}}%
\pgfpathcurveto{\pgfqpoint{-0.018638in}{0.010825in}}{\pgfqpoint{-0.020833in}{0.005525in}}{\pgfqpoint{-0.020833in}{0.000000in}}%
\pgfpathcurveto{\pgfqpoint{-0.020833in}{-0.005525in}}{\pgfqpoint{-0.018638in}{-0.010825in}}{\pgfqpoint{-0.014731in}{-0.014731in}}%
\pgfpathcurveto{\pgfqpoint{-0.010825in}{-0.018638in}}{\pgfqpoint{-0.005525in}{-0.020833in}}{\pgfqpoint{0.000000in}{-0.020833in}}%
\pgfpathclose%
\pgfusepath{stroke,fill}%
}%
\begin{pgfscope}%
\pgfsys@transformshift{1.617643in}{1.894977in}%
\pgfsys@useobject{currentmarker}{}%
\end{pgfscope}%
\end{pgfscope}%
\begin{pgfscope}%
\pgftext[x=1.867643in,y=1.846366in,left,base]{\rmfamily\fontsize{10.000000}{12.000000}\selectfont global knots \(\displaystyle z_i\)}%
\end{pgfscope}%
\begin{pgfscope}%
\pgfsetbuttcap%
\pgfsetmiterjoin%
\definecolor{currentfill}{rgb}{1.000000,1.000000,1.000000}%
\pgfsetfillcolor{currentfill}%
\pgfsetlinewidth{0.000000pt}%
\definecolor{currentstroke}{rgb}{0.000000,0.000000,0.000000}%
\pgfsetstrokecolor{currentstroke}%
\pgfsetstrokeopacity{0.000000}%
\pgfsetdash{}{0pt}%
\pgfpathmoveto{\pgfqpoint{0.629692in}{1.756603in}}%
\pgfpathlineto{\pgfqpoint{1.309692in}{1.756603in}}%
\pgfpathlineto{\pgfqpoint{1.309692in}{2.224603in}}%
\pgfpathlineto{\pgfqpoint{0.629692in}{2.224603in}}%
\pgfpathclose%
\pgfusepath{fill}%
\end{pgfscope}%
\begin{pgfscope}%
\pgfsetbuttcap%
\pgfsetroundjoin%
\definecolor{currentfill}{rgb}{0.000000,0.000000,0.000000}%
\pgfsetfillcolor{currentfill}%
\pgfsetlinewidth{0.803000pt}%
\definecolor{currentstroke}{rgb}{0.000000,0.000000,0.000000}%
\pgfsetstrokecolor{currentstroke}%
\pgfsetdash{}{0pt}%
\pgfsys@defobject{currentmarker}{\pgfqpoint{0.000000in}{-0.048611in}}{\pgfqpoint{0.000000in}{0.000000in}}{%
\pgfpathmoveto{\pgfqpoint{0.000000in}{0.000000in}}%
\pgfpathlineto{\pgfqpoint{0.000000in}{-0.048611in}}%
\pgfusepath{stroke,fill}%
}%
\begin{pgfscope}%
\pgfsys@transformshift{0.660601in}{1.756603in}%
\pgfsys@useobject{currentmarker}{}%
\end{pgfscope}%
\end{pgfscope}%
\begin{pgfscope}%
\pgftext[x=0.660601in,y=1.659381in,,top]{\rmfamily\fontsize{10.000000}{12.000000}\selectfont \(\displaystyle -1\)}%
\end{pgfscope}%
\begin{pgfscope}%
\pgfsetbuttcap%
\pgfsetroundjoin%
\definecolor{currentfill}{rgb}{0.000000,0.000000,0.000000}%
\pgfsetfillcolor{currentfill}%
\pgfsetlinewidth{0.803000pt}%
\definecolor{currentstroke}{rgb}{0.000000,0.000000,0.000000}%
\pgfsetstrokecolor{currentstroke}%
\pgfsetdash{}{0pt}%
\pgfsys@defobject{currentmarker}{\pgfqpoint{0.000000in}{-0.048611in}}{\pgfqpoint{0.000000in}{0.000000in}}{%
\pgfpathmoveto{\pgfqpoint{0.000000in}{0.000000in}}%
\pgfpathlineto{\pgfqpoint{0.000000in}{-0.048611in}}%
\pgfusepath{stroke,fill}%
}%
\begin{pgfscope}%
\pgfsys@transformshift{0.969692in}{1.756603in}%
\pgfsys@useobject{currentmarker}{}%
\end{pgfscope}%
\end{pgfscope}%
\begin{pgfscope}%
\pgftext[x=0.969692in,y=1.659381in,,top]{\rmfamily\fontsize{10.000000}{12.000000}\selectfont \(\displaystyle 0\)}%
\end{pgfscope}%
\begin{pgfscope}%
\pgfsetbuttcap%
\pgfsetroundjoin%
\definecolor{currentfill}{rgb}{0.000000,0.000000,0.000000}%
\pgfsetfillcolor{currentfill}%
\pgfsetlinewidth{0.803000pt}%
\definecolor{currentstroke}{rgb}{0.000000,0.000000,0.000000}%
\pgfsetstrokecolor{currentstroke}%
\pgfsetdash{}{0pt}%
\pgfsys@defobject{currentmarker}{\pgfqpoint{0.000000in}{-0.048611in}}{\pgfqpoint{0.000000in}{0.000000in}}{%
\pgfpathmoveto{\pgfqpoint{0.000000in}{0.000000in}}%
\pgfpathlineto{\pgfqpoint{0.000000in}{-0.048611in}}%
\pgfusepath{stroke,fill}%
}%
\begin{pgfscope}%
\pgfsys@transformshift{1.278783in}{1.756603in}%
\pgfsys@useobject{currentmarker}{}%
\end{pgfscope}%
\end{pgfscope}%
\begin{pgfscope}%
\pgftext[x=1.278783in,y=1.659381in,,top]{\rmfamily\fontsize{10.000000}{12.000000}\selectfont \(\displaystyle 1\)}%
\end{pgfscope}%
\begin{pgfscope}%
\pgftext[x=0.969692in,y=1.469413in,,top]{\rmfamily\fontsize{10.000000}{12.000000}\selectfont s}%
\end{pgfscope}%
\begin{pgfscope}%
\pgfsetbuttcap%
\pgfsetroundjoin%
\definecolor{currentfill}{rgb}{0.000000,0.000000,0.000000}%
\pgfsetfillcolor{currentfill}%
\pgfsetlinewidth{0.803000pt}%
\definecolor{currentstroke}{rgb}{0.000000,0.000000,0.000000}%
\pgfsetstrokecolor{currentstroke}%
\pgfsetdash{}{0pt}%
\pgfsys@defobject{currentmarker}{\pgfqpoint{-0.048611in}{0.000000in}}{\pgfqpoint{0.000000in}{0.000000in}}{%
\pgfpathmoveto{\pgfqpoint{0.000000in}{0.000000in}}%
\pgfpathlineto{\pgfqpoint{-0.048611in}{0.000000in}}%
\pgfusepath{stroke,fill}%
}%
\begin{pgfscope}%
\pgfsys@transformshift{0.629692in}{1.825145in}%
\pgfsys@useobject{currentmarker}{}%
\end{pgfscope}%
\end{pgfscope}%
\begin{pgfscope}%
\pgftext[x=0.463025in,y=1.772383in,left,base]{\rmfamily\fontsize{10.000000}{12.000000}\selectfont \(\displaystyle 0\)}%
\end{pgfscope}%
\begin{pgfscope}%
\pgfsetbuttcap%
\pgfsetroundjoin%
\definecolor{currentfill}{rgb}{0.000000,0.000000,0.000000}%
\pgfsetfillcolor{currentfill}%
\pgfsetlinewidth{0.803000pt}%
\definecolor{currentstroke}{rgb}{0.000000,0.000000,0.000000}%
\pgfsetstrokecolor{currentstroke}%
\pgfsetdash{}{0pt}%
\pgfsys@defobject{currentmarker}{\pgfqpoint{-0.048611in}{0.000000in}}{\pgfqpoint{0.000000in}{0.000000in}}{%
\pgfpathmoveto{\pgfqpoint{0.000000in}{0.000000in}}%
\pgfpathlineto{\pgfqpoint{-0.048611in}{0.000000in}}%
\pgfusepath{stroke,fill}%
}%
\begin{pgfscope}%
\pgfsys@transformshift{0.629692in}{2.203331in}%
\pgfsys@useobject{currentmarker}{}%
\end{pgfscope}%
\end{pgfscope}%
\begin{pgfscope}%
\pgftext[x=0.463025in,y=2.150569in,left,base]{\rmfamily\fontsize{10.000000}{12.000000}\selectfont \(\displaystyle 1\)}%
\end{pgfscope}%
\begin{pgfscope}%
\pgfpathrectangle{\pgfqpoint{0.629692in}{1.756603in}}{\pgfqpoint{0.680000in}{0.468000in}} %
\pgfusepath{clip}%
\pgfsetrectcap%
\pgfsetroundjoin%
\pgfsetlinewidth{1.003750pt}%
\definecolor{currentstroke}{rgb}{0.121569,0.466667,0.705882}%
\pgfsetstrokecolor{currentstroke}%
\pgfsetdash{}{0pt}%
\pgfpathmoveto{\pgfqpoint{0.660601in}{2.203331in}}%
\pgfpathlineto{\pgfqpoint{0.666845in}{2.191948in}}%
\pgfpathlineto{\pgfqpoint{0.673090in}{2.180719in}}%
\pgfpathlineto{\pgfqpoint{0.679334in}{2.169645in}}%
\pgfpathlineto{\pgfqpoint{0.685578in}{2.158725in}}%
\pgfpathlineto{\pgfqpoint{0.691822in}{2.147959in}}%
\pgfpathlineto{\pgfqpoint{0.698067in}{2.137348in}}%
\pgfpathlineto{\pgfqpoint{0.704311in}{2.126891in}}%
\pgfpathlineto{\pgfqpoint{0.710555in}{2.116588in}}%
\pgfpathlineto{\pgfqpoint{0.716799in}{2.106440in}}%
\pgfpathlineto{\pgfqpoint{0.723044in}{2.096446in}}%
\pgfpathlineto{\pgfqpoint{0.729288in}{2.086606in}}%
\pgfpathlineto{\pgfqpoint{0.735532in}{2.076921in}}%
\pgfpathlineto{\pgfqpoint{0.741776in}{2.067390in}}%
\pgfpathlineto{\pgfqpoint{0.748021in}{2.058014in}}%
\pgfpathlineto{\pgfqpoint{0.754265in}{2.048792in}}%
\pgfpathlineto{\pgfqpoint{0.760509in}{2.039724in}}%
\pgfpathlineto{\pgfqpoint{0.766753in}{2.030810in}}%
\pgfpathlineto{\pgfqpoint{0.772998in}{2.022051in}}%
\pgfpathlineto{\pgfqpoint{0.779242in}{2.013447in}}%
\pgfpathlineto{\pgfqpoint{0.785486in}{2.004996in}}%
\pgfpathlineto{\pgfqpoint{0.791731in}{1.996700in}}%
\pgfpathlineto{\pgfqpoint{0.797975in}{1.988558in}}%
\pgfpathlineto{\pgfqpoint{0.804219in}{1.980571in}}%
\pgfpathlineto{\pgfqpoint{0.810463in}{1.972738in}}%
\pgfpathlineto{\pgfqpoint{0.816708in}{1.965059in}}%
\pgfpathlineto{\pgfqpoint{0.822952in}{1.957535in}}%
\pgfpathlineto{\pgfqpoint{0.829196in}{1.950165in}}%
\pgfpathlineto{\pgfqpoint{0.835440in}{1.942949in}}%
\pgfpathlineto{\pgfqpoint{0.841685in}{1.935888in}}%
\pgfpathlineto{\pgfqpoint{0.847929in}{1.928981in}}%
\pgfpathlineto{\pgfqpoint{0.854173in}{1.922228in}}%
\pgfpathlineto{\pgfqpoint{0.860417in}{1.915630in}}%
\pgfpathlineto{\pgfqpoint{0.866662in}{1.909186in}}%
\pgfpathlineto{\pgfqpoint{0.872906in}{1.902896in}}%
\pgfpathlineto{\pgfqpoint{0.879150in}{1.896761in}}%
\pgfpathlineto{\pgfqpoint{0.885394in}{1.890780in}}%
\pgfpathlineto{\pgfqpoint{0.891639in}{1.884954in}}%
\pgfpathlineto{\pgfqpoint{0.897883in}{1.879281in}}%
\pgfpathlineto{\pgfqpoint{0.904127in}{1.873763in}}%
\pgfpathlineto{\pgfqpoint{0.910371in}{1.868400in}}%
\pgfpathlineto{\pgfqpoint{0.916616in}{1.863191in}}%
\pgfpathlineto{\pgfqpoint{0.922860in}{1.858136in}}%
\pgfpathlineto{\pgfqpoint{0.929104in}{1.853235in}}%
\pgfpathlineto{\pgfqpoint{0.935349in}{1.848489in}}%
\pgfpathlineto{\pgfqpoint{0.941593in}{1.843898in}}%
\pgfpathlineto{\pgfqpoint{0.947837in}{1.839460in}}%
\pgfpathlineto{\pgfqpoint{0.954081in}{1.835177in}}%
\pgfpathlineto{\pgfqpoint{0.960326in}{1.831048in}}%
\pgfpathlineto{\pgfqpoint{0.966570in}{1.827074in}}%
\pgfpathlineto{\pgfqpoint{0.972814in}{1.823254in}}%
\pgfpathlineto{\pgfqpoint{0.979058in}{1.819588in}}%
\pgfpathlineto{\pgfqpoint{0.985303in}{1.816077in}}%
\pgfpathlineto{\pgfqpoint{0.991547in}{1.812720in}}%
\pgfpathlineto{\pgfqpoint{0.997791in}{1.809517in}}%
\pgfpathlineto{\pgfqpoint{1.004035in}{1.806469in}}%
\pgfpathlineto{\pgfqpoint{1.010280in}{1.803575in}}%
\pgfpathlineto{\pgfqpoint{1.016524in}{1.800835in}}%
\pgfpathlineto{\pgfqpoint{1.022768in}{1.798250in}}%
\pgfpathlineto{\pgfqpoint{1.029012in}{1.795819in}}%
\pgfpathlineto{\pgfqpoint{1.035257in}{1.793542in}}%
\pgfpathlineto{\pgfqpoint{1.041501in}{1.791420in}}%
\pgfpathlineto{\pgfqpoint{1.047745in}{1.789452in}}%
\pgfpathlineto{\pgfqpoint{1.053989in}{1.787638in}}%
\pgfpathlineto{\pgfqpoint{1.060234in}{1.785979in}}%
\pgfpathlineto{\pgfqpoint{1.066478in}{1.784474in}}%
\pgfpathlineto{\pgfqpoint{1.072722in}{1.783124in}}%
\pgfpathlineto{\pgfqpoint{1.078967in}{1.781928in}}%
\pgfpathlineto{\pgfqpoint{1.085211in}{1.780886in}}%
\pgfpathlineto{\pgfqpoint{1.091455in}{1.779998in}}%
\pgfpathlineto{\pgfqpoint{1.097699in}{1.779265in}}%
\pgfpathlineto{\pgfqpoint{1.103944in}{1.778686in}}%
\pgfpathlineto{\pgfqpoint{1.110188in}{1.778262in}}%
\pgfpathlineto{\pgfqpoint{1.116432in}{1.777992in}}%
\pgfpathlineto{\pgfqpoint{1.122676in}{1.777876in}}%
\pgfpathlineto{\pgfqpoint{1.128921in}{1.777915in}}%
\pgfpathlineto{\pgfqpoint{1.135165in}{1.778108in}}%
\pgfpathlineto{\pgfqpoint{1.141409in}{1.778455in}}%
\pgfpathlineto{\pgfqpoint{1.147653in}{1.778956in}}%
\pgfpathlineto{\pgfqpoint{1.153898in}{1.779612in}}%
\pgfpathlineto{\pgfqpoint{1.160142in}{1.780423in}}%
\pgfpathlineto{\pgfqpoint{1.166386in}{1.781387in}}%
\pgfpathlineto{\pgfqpoint{1.172630in}{1.782506in}}%
\pgfpathlineto{\pgfqpoint{1.178875in}{1.783780in}}%
\pgfpathlineto{\pgfqpoint{1.185119in}{1.785207in}}%
\pgfpathlineto{\pgfqpoint{1.191363in}{1.786790in}}%
\pgfpathlineto{\pgfqpoint{1.197607in}{1.788526in}}%
\pgfpathlineto{\pgfqpoint{1.203852in}{1.790417in}}%
\pgfpathlineto{\pgfqpoint{1.210096in}{1.792462in}}%
\pgfpathlineto{\pgfqpoint{1.216340in}{1.794661in}}%
\pgfpathlineto{\pgfqpoint{1.222585in}{1.797015in}}%
\pgfpathlineto{\pgfqpoint{1.228829in}{1.799523in}}%
\pgfpathlineto{\pgfqpoint{1.235073in}{1.802186in}}%
\pgfpathlineto{\pgfqpoint{1.241317in}{1.805002in}}%
\pgfpathlineto{\pgfqpoint{1.247562in}{1.807974in}}%
\pgfpathlineto{\pgfqpoint{1.253806in}{1.811099in}}%
\pgfpathlineto{\pgfqpoint{1.260050in}{1.814379in}}%
\pgfpathlineto{\pgfqpoint{1.266294in}{1.817813in}}%
\pgfpathlineto{\pgfqpoint{1.272539in}{1.821402in}}%
\pgfpathlineto{\pgfqpoint{1.278783in}{1.825145in}}%
\pgfusepath{stroke}%
\end{pgfscope}%
\begin{pgfscope}%
\pgfpathrectangle{\pgfqpoint{0.629692in}{1.756603in}}{\pgfqpoint{0.680000in}{0.468000in}} %
\pgfusepath{clip}%
\pgfsetrectcap%
\pgfsetroundjoin%
\pgfsetlinewidth{1.003750pt}%
\definecolor{currentstroke}{rgb}{1.000000,0.498039,0.054902}%
\pgfsetstrokecolor{currentstroke}%
\pgfsetdash{}{0pt}%
\pgfpathmoveto{\pgfqpoint{0.660601in}{1.825145in}}%
\pgfpathlineto{\pgfqpoint{0.666845in}{1.840270in}}%
\pgfpathlineto{\pgfqpoint{0.673090in}{1.855088in}}%
\pgfpathlineto{\pgfqpoint{0.679334in}{1.869596in}}%
\pgfpathlineto{\pgfqpoint{0.685578in}{1.883796in}}%
\pgfpathlineto{\pgfqpoint{0.691822in}{1.897687in}}%
\pgfpathlineto{\pgfqpoint{0.698067in}{1.911270in}}%
\pgfpathlineto{\pgfqpoint{0.704311in}{1.924543in}}%
\pgfpathlineto{\pgfqpoint{0.710555in}{1.937508in}}%
\pgfpathlineto{\pgfqpoint{0.716799in}{1.950165in}}%
\pgfpathlineto{\pgfqpoint{0.723044in}{1.962512in}}%
\pgfpathlineto{\pgfqpoint{0.729288in}{1.974551in}}%
\pgfpathlineto{\pgfqpoint{0.735532in}{1.986282in}}%
\pgfpathlineto{\pgfqpoint{0.741776in}{1.997703in}}%
\pgfpathlineto{\pgfqpoint{0.748021in}{2.008816in}}%
\pgfpathlineto{\pgfqpoint{0.754265in}{2.019620in}}%
\pgfpathlineto{\pgfqpoint{0.760509in}{2.030116in}}%
\pgfpathlineto{\pgfqpoint{0.766753in}{2.040303in}}%
\pgfpathlineto{\pgfqpoint{0.772998in}{2.050181in}}%
\pgfpathlineto{\pgfqpoint{0.779242in}{2.059750in}}%
\pgfpathlineto{\pgfqpoint{0.785486in}{2.069011in}}%
\pgfpathlineto{\pgfqpoint{0.791731in}{2.077963in}}%
\pgfpathlineto{\pgfqpoint{0.797975in}{2.086606in}}%
\pgfpathlineto{\pgfqpoint{0.804219in}{2.094941in}}%
\pgfpathlineto{\pgfqpoint{0.810463in}{2.102967in}}%
\pgfpathlineto{\pgfqpoint{0.816708in}{2.110684in}}%
\pgfpathlineto{\pgfqpoint{0.822952in}{2.118093in}}%
\pgfpathlineto{\pgfqpoint{0.829196in}{2.125193in}}%
\pgfpathlineto{\pgfqpoint{0.835440in}{2.131984in}}%
\pgfpathlineto{\pgfqpoint{0.841685in}{2.138467in}}%
\pgfpathlineto{\pgfqpoint{0.847929in}{2.144641in}}%
\pgfpathlineto{\pgfqpoint{0.854173in}{2.150506in}}%
\pgfpathlineto{\pgfqpoint{0.860417in}{2.156062in}}%
\pgfpathlineto{\pgfqpoint{0.866662in}{2.161310in}}%
\pgfpathlineto{\pgfqpoint{0.872906in}{2.166249in}}%
\pgfpathlineto{\pgfqpoint{0.879150in}{2.170879in}}%
\pgfpathlineto{\pgfqpoint{0.885394in}{2.175201in}}%
\pgfpathlineto{\pgfqpoint{0.891639in}{2.179214in}}%
\pgfpathlineto{\pgfqpoint{0.897883in}{2.182918in}}%
\pgfpathlineto{\pgfqpoint{0.904127in}{2.186314in}}%
\pgfpathlineto{\pgfqpoint{0.910371in}{2.189401in}}%
\pgfpathlineto{\pgfqpoint{0.916616in}{2.192179in}}%
\pgfpathlineto{\pgfqpoint{0.922860in}{2.194649in}}%
\pgfpathlineto{\pgfqpoint{0.929104in}{2.196809in}}%
\pgfpathlineto{\pgfqpoint{0.935349in}{2.198662in}}%
\pgfpathlineto{\pgfqpoint{0.941593in}{2.200205in}}%
\pgfpathlineto{\pgfqpoint{0.947837in}{2.201440in}}%
\pgfpathlineto{\pgfqpoint{0.954081in}{2.202366in}}%
\pgfpathlineto{\pgfqpoint{0.960326in}{2.202983in}}%
\pgfpathlineto{\pgfqpoint{0.966570in}{2.203292in}}%
\pgfpathlineto{\pgfqpoint{0.972814in}{2.203292in}}%
\pgfpathlineto{\pgfqpoint{0.979058in}{2.202983in}}%
\pgfpathlineto{\pgfqpoint{0.985303in}{2.202366in}}%
\pgfpathlineto{\pgfqpoint{0.991547in}{2.201440in}}%
\pgfpathlineto{\pgfqpoint{0.997791in}{2.200205in}}%
\pgfpathlineto{\pgfqpoint{1.004035in}{2.198662in}}%
\pgfpathlineto{\pgfqpoint{1.010280in}{2.196809in}}%
\pgfpathlineto{\pgfqpoint{1.016524in}{2.194649in}}%
\pgfpathlineto{\pgfqpoint{1.022768in}{2.192179in}}%
\pgfpathlineto{\pgfqpoint{1.029012in}{2.189401in}}%
\pgfpathlineto{\pgfqpoint{1.035257in}{2.186314in}}%
\pgfpathlineto{\pgfqpoint{1.041501in}{2.182918in}}%
\pgfpathlineto{\pgfqpoint{1.047745in}{2.179214in}}%
\pgfpathlineto{\pgfqpoint{1.053989in}{2.175201in}}%
\pgfpathlineto{\pgfqpoint{1.060234in}{2.170879in}}%
\pgfpathlineto{\pgfqpoint{1.066478in}{2.166249in}}%
\pgfpathlineto{\pgfqpoint{1.072722in}{2.161310in}}%
\pgfpathlineto{\pgfqpoint{1.078967in}{2.156062in}}%
\pgfpathlineto{\pgfqpoint{1.085211in}{2.150506in}}%
\pgfpathlineto{\pgfqpoint{1.091455in}{2.144641in}}%
\pgfpathlineto{\pgfqpoint{1.097699in}{2.138467in}}%
\pgfpathlineto{\pgfqpoint{1.103944in}{2.131984in}}%
\pgfpathlineto{\pgfqpoint{1.110188in}{2.125193in}}%
\pgfpathlineto{\pgfqpoint{1.116432in}{2.118093in}}%
\pgfpathlineto{\pgfqpoint{1.122676in}{2.110684in}}%
\pgfpathlineto{\pgfqpoint{1.128921in}{2.102967in}}%
\pgfpathlineto{\pgfqpoint{1.135165in}{2.094941in}}%
\pgfpathlineto{\pgfqpoint{1.141409in}{2.086606in}}%
\pgfpathlineto{\pgfqpoint{1.147653in}{2.077963in}}%
\pgfpathlineto{\pgfqpoint{1.153898in}{2.069011in}}%
\pgfpathlineto{\pgfqpoint{1.160142in}{2.059750in}}%
\pgfpathlineto{\pgfqpoint{1.166386in}{2.050181in}}%
\pgfpathlineto{\pgfqpoint{1.172630in}{2.040303in}}%
\pgfpathlineto{\pgfqpoint{1.178875in}{2.030116in}}%
\pgfpathlineto{\pgfqpoint{1.185119in}{2.019620in}}%
\pgfpathlineto{\pgfqpoint{1.191363in}{2.008816in}}%
\pgfpathlineto{\pgfqpoint{1.197607in}{1.997703in}}%
\pgfpathlineto{\pgfqpoint{1.203852in}{1.986282in}}%
\pgfpathlineto{\pgfqpoint{1.210096in}{1.974551in}}%
\pgfpathlineto{\pgfqpoint{1.216340in}{1.962512in}}%
\pgfpathlineto{\pgfqpoint{1.222585in}{1.950165in}}%
\pgfpathlineto{\pgfqpoint{1.228829in}{1.937508in}}%
\pgfpathlineto{\pgfqpoint{1.235073in}{1.924543in}}%
\pgfpathlineto{\pgfqpoint{1.241317in}{1.911270in}}%
\pgfpathlineto{\pgfqpoint{1.247562in}{1.897687in}}%
\pgfpathlineto{\pgfqpoint{1.253806in}{1.883796in}}%
\pgfpathlineto{\pgfqpoint{1.260050in}{1.869596in}}%
\pgfpathlineto{\pgfqpoint{1.266294in}{1.855088in}}%
\pgfpathlineto{\pgfqpoint{1.272539in}{1.840270in}}%
\pgfpathlineto{\pgfqpoint{1.278783in}{1.825145in}}%
\pgfusepath{stroke}%
\end{pgfscope}%
\begin{pgfscope}%
\pgfpathrectangle{\pgfqpoint{0.629692in}{1.756603in}}{\pgfqpoint{0.680000in}{0.468000in}} %
\pgfusepath{clip}%
\pgfsetrectcap%
\pgfsetroundjoin%
\pgfsetlinewidth{1.003750pt}%
\definecolor{currentstroke}{rgb}{0.172549,0.627451,0.172549}%
\pgfsetstrokecolor{currentstroke}%
\pgfsetdash{}{0pt}%
\pgfpathmoveto{\pgfqpoint{0.660601in}{1.825145in}}%
\pgfpathlineto{\pgfqpoint{0.666845in}{1.821402in}}%
\pgfpathlineto{\pgfqpoint{0.673090in}{1.817813in}}%
\pgfpathlineto{\pgfqpoint{0.679334in}{1.814379in}}%
\pgfpathlineto{\pgfqpoint{0.685578in}{1.811099in}}%
\pgfpathlineto{\pgfqpoint{0.691822in}{1.807974in}}%
\pgfpathlineto{\pgfqpoint{0.698067in}{1.805002in}}%
\pgfpathlineto{\pgfqpoint{0.704311in}{1.802186in}}%
\pgfpathlineto{\pgfqpoint{0.710555in}{1.799523in}}%
\pgfpathlineto{\pgfqpoint{0.716799in}{1.797015in}}%
\pgfpathlineto{\pgfqpoint{0.723044in}{1.794661in}}%
\pgfpathlineto{\pgfqpoint{0.729288in}{1.792462in}}%
\pgfpathlineto{\pgfqpoint{0.735532in}{1.790417in}}%
\pgfpathlineto{\pgfqpoint{0.741776in}{1.788526in}}%
\pgfpathlineto{\pgfqpoint{0.748021in}{1.786790in}}%
\pgfpathlineto{\pgfqpoint{0.754265in}{1.785207in}}%
\pgfpathlineto{\pgfqpoint{0.760509in}{1.783780in}}%
\pgfpathlineto{\pgfqpoint{0.766753in}{1.782506in}}%
\pgfpathlineto{\pgfqpoint{0.772998in}{1.781387in}}%
\pgfpathlineto{\pgfqpoint{0.779242in}{1.780423in}}%
\pgfpathlineto{\pgfqpoint{0.785486in}{1.779612in}}%
\pgfpathlineto{\pgfqpoint{0.791731in}{1.778956in}}%
\pgfpathlineto{\pgfqpoint{0.797975in}{1.778455in}}%
\pgfpathlineto{\pgfqpoint{0.804219in}{1.778108in}}%
\pgfpathlineto{\pgfqpoint{0.810463in}{1.777915in}}%
\pgfpathlineto{\pgfqpoint{0.816708in}{1.777876in}}%
\pgfpathlineto{\pgfqpoint{0.822952in}{1.777992in}}%
\pgfpathlineto{\pgfqpoint{0.829196in}{1.778262in}}%
\pgfpathlineto{\pgfqpoint{0.835440in}{1.778686in}}%
\pgfpathlineto{\pgfqpoint{0.841685in}{1.779265in}}%
\pgfpathlineto{\pgfqpoint{0.847929in}{1.779998in}}%
\pgfpathlineto{\pgfqpoint{0.854173in}{1.780886in}}%
\pgfpathlineto{\pgfqpoint{0.860417in}{1.781928in}}%
\pgfpathlineto{\pgfqpoint{0.866662in}{1.783124in}}%
\pgfpathlineto{\pgfqpoint{0.872906in}{1.784474in}}%
\pgfpathlineto{\pgfqpoint{0.879150in}{1.785979in}}%
\pgfpathlineto{\pgfqpoint{0.885394in}{1.787638in}}%
\pgfpathlineto{\pgfqpoint{0.891639in}{1.789452in}}%
\pgfpathlineto{\pgfqpoint{0.897883in}{1.791420in}}%
\pgfpathlineto{\pgfqpoint{0.904127in}{1.793542in}}%
\pgfpathlineto{\pgfqpoint{0.910371in}{1.795819in}}%
\pgfpathlineto{\pgfqpoint{0.916616in}{1.798250in}}%
\pgfpathlineto{\pgfqpoint{0.922860in}{1.800835in}}%
\pgfpathlineto{\pgfqpoint{0.929104in}{1.803575in}}%
\pgfpathlineto{\pgfqpoint{0.935349in}{1.806469in}}%
\pgfpathlineto{\pgfqpoint{0.941593in}{1.809517in}}%
\pgfpathlineto{\pgfqpoint{0.947837in}{1.812720in}}%
\pgfpathlineto{\pgfqpoint{0.954081in}{1.816077in}}%
\pgfpathlineto{\pgfqpoint{0.960326in}{1.819588in}}%
\pgfpathlineto{\pgfqpoint{0.966570in}{1.823254in}}%
\pgfpathlineto{\pgfqpoint{0.972814in}{1.827074in}}%
\pgfpathlineto{\pgfqpoint{0.979058in}{1.831048in}}%
\pgfpathlineto{\pgfqpoint{0.985303in}{1.835177in}}%
\pgfpathlineto{\pgfqpoint{0.991547in}{1.839460in}}%
\pgfpathlineto{\pgfqpoint{0.997791in}{1.843898in}}%
\pgfpathlineto{\pgfqpoint{1.004035in}{1.848489in}}%
\pgfpathlineto{\pgfqpoint{1.010280in}{1.853235in}}%
\pgfpathlineto{\pgfqpoint{1.016524in}{1.858136in}}%
\pgfpathlineto{\pgfqpoint{1.022768in}{1.863191in}}%
\pgfpathlineto{\pgfqpoint{1.029012in}{1.868400in}}%
\pgfpathlineto{\pgfqpoint{1.035257in}{1.873763in}}%
\pgfpathlineto{\pgfqpoint{1.041501in}{1.879281in}}%
\pgfpathlineto{\pgfqpoint{1.047745in}{1.884954in}}%
\pgfpathlineto{\pgfqpoint{1.053989in}{1.890780in}}%
\pgfpathlineto{\pgfqpoint{1.060234in}{1.896761in}}%
\pgfpathlineto{\pgfqpoint{1.066478in}{1.902896in}}%
\pgfpathlineto{\pgfqpoint{1.072722in}{1.909186in}}%
\pgfpathlineto{\pgfqpoint{1.078967in}{1.915630in}}%
\pgfpathlineto{\pgfqpoint{1.085211in}{1.922228in}}%
\pgfpathlineto{\pgfqpoint{1.091455in}{1.928981in}}%
\pgfpathlineto{\pgfqpoint{1.097699in}{1.935888in}}%
\pgfpathlineto{\pgfqpoint{1.103944in}{1.942949in}}%
\pgfpathlineto{\pgfqpoint{1.110188in}{1.950165in}}%
\pgfpathlineto{\pgfqpoint{1.116432in}{1.957535in}}%
\pgfpathlineto{\pgfqpoint{1.122676in}{1.965059in}}%
\pgfpathlineto{\pgfqpoint{1.128921in}{1.972738in}}%
\pgfpathlineto{\pgfqpoint{1.135165in}{1.980571in}}%
\pgfpathlineto{\pgfqpoint{1.141409in}{1.988558in}}%
\pgfpathlineto{\pgfqpoint{1.147653in}{1.996700in}}%
\pgfpathlineto{\pgfqpoint{1.153898in}{2.004996in}}%
\pgfpathlineto{\pgfqpoint{1.160142in}{2.013447in}}%
\pgfpathlineto{\pgfqpoint{1.166386in}{2.022051in}}%
\pgfpathlineto{\pgfqpoint{1.172630in}{2.030810in}}%
\pgfpathlineto{\pgfqpoint{1.178875in}{2.039724in}}%
\pgfpathlineto{\pgfqpoint{1.185119in}{2.048792in}}%
\pgfpathlineto{\pgfqpoint{1.191363in}{2.058014in}}%
\pgfpathlineto{\pgfqpoint{1.197607in}{2.067390in}}%
\pgfpathlineto{\pgfqpoint{1.203852in}{2.076921in}}%
\pgfpathlineto{\pgfqpoint{1.210096in}{2.086606in}}%
\pgfpathlineto{\pgfqpoint{1.216340in}{2.096446in}}%
\pgfpathlineto{\pgfqpoint{1.222585in}{2.106440in}}%
\pgfpathlineto{\pgfqpoint{1.228829in}{2.116588in}}%
\pgfpathlineto{\pgfqpoint{1.235073in}{2.126891in}}%
\pgfpathlineto{\pgfqpoint{1.241317in}{2.137348in}}%
\pgfpathlineto{\pgfqpoint{1.247562in}{2.147959in}}%
\pgfpathlineto{\pgfqpoint{1.253806in}{2.158725in}}%
\pgfpathlineto{\pgfqpoint{1.260050in}{2.169645in}}%
\pgfpathlineto{\pgfqpoint{1.266294in}{2.180719in}}%
\pgfpathlineto{\pgfqpoint{1.272539in}{2.191948in}}%
\pgfpathlineto{\pgfqpoint{1.278783in}{2.203331in}}%
\pgfusepath{stroke}%
\end{pgfscope}%
\begin{pgfscope}%
\pgfpathrectangle{\pgfqpoint{0.629692in}{1.756603in}}{\pgfqpoint{0.680000in}{0.468000in}} %
\pgfusepath{clip}%
\pgfsetbuttcap%
\pgfsetroundjoin%
\definecolor{currentfill}{rgb}{0.000000,0.000000,0.000000}%
\pgfsetfillcolor{currentfill}%
\pgfsetlinewidth{1.003750pt}%
\definecolor{currentstroke}{rgb}{0.000000,0.000000,0.000000}%
\pgfsetstrokecolor{currentstroke}%
\pgfsetdash{}{0pt}%
\pgfsys@defobject{currentmarker}{\pgfqpoint{-0.020833in}{-0.020833in}}{\pgfqpoint{0.020833in}{0.020833in}}{%
\pgfpathmoveto{\pgfqpoint{0.000000in}{-0.020833in}}%
\pgfpathcurveto{\pgfqpoint{0.005525in}{-0.020833in}}{\pgfqpoint{0.010825in}{-0.018638in}}{\pgfqpoint{0.014731in}{-0.014731in}}%
\pgfpathcurveto{\pgfqpoint{0.018638in}{-0.010825in}}{\pgfqpoint{0.020833in}{-0.005525in}}{\pgfqpoint{0.020833in}{0.000000in}}%
\pgfpathcurveto{\pgfqpoint{0.020833in}{0.005525in}}{\pgfqpoint{0.018638in}{0.010825in}}{\pgfqpoint{0.014731in}{0.014731in}}%
\pgfpathcurveto{\pgfqpoint{0.010825in}{0.018638in}}{\pgfqpoint{0.005525in}{0.020833in}}{\pgfqpoint{0.000000in}{0.020833in}}%
\pgfpathcurveto{\pgfqpoint{-0.005525in}{0.020833in}}{\pgfqpoint{-0.010825in}{0.018638in}}{\pgfqpoint{-0.014731in}{0.014731in}}%
\pgfpathcurveto{\pgfqpoint{-0.018638in}{0.010825in}}{\pgfqpoint{-0.020833in}{0.005525in}}{\pgfqpoint{-0.020833in}{0.000000in}}%
\pgfpathcurveto{\pgfqpoint{-0.020833in}{-0.005525in}}{\pgfqpoint{-0.018638in}{-0.010825in}}{\pgfqpoint{-0.014731in}{-0.014731in}}%
\pgfpathcurveto{\pgfqpoint{-0.010825in}{-0.018638in}}{\pgfqpoint{-0.005525in}{-0.020833in}}{\pgfqpoint{0.000000in}{-0.020833in}}%
\pgfpathclose%
\pgfusepath{stroke,fill}%
}%
\begin{pgfscope}%
\pgfsys@transformshift{0.660601in}{1.825145in}%
\pgfsys@useobject{currentmarker}{}%
\end{pgfscope}%
\begin{pgfscope}%
\pgfsys@transformshift{0.969692in}{1.825145in}%
\pgfsys@useobject{currentmarker}{}%
\end{pgfscope}%
\begin{pgfscope}%
\pgfsys@transformshift{1.278783in}{1.825145in}%
\pgfsys@useobject{currentmarker}{}%
\end{pgfscope}%
\end{pgfscope}%
\begin{pgfscope}%
\pgfsetrectcap%
\pgfsetmiterjoin%
\pgfsetlinewidth{0.803000pt}%
\definecolor{currentstroke}{rgb}{0.000000,0.000000,0.000000}%
\pgfsetstrokecolor{currentstroke}%
\pgfsetdash{}{0pt}%
\pgfpathmoveto{\pgfqpoint{0.629692in}{1.756603in}}%
\pgfpathlineto{\pgfqpoint{0.629692in}{2.224603in}}%
\pgfusepath{stroke}%
\end{pgfscope}%
\begin{pgfscope}%
\pgfsetrectcap%
\pgfsetmiterjoin%
\pgfsetlinewidth{0.803000pt}%
\definecolor{currentstroke}{rgb}{0.000000,0.000000,0.000000}%
\pgfsetstrokecolor{currentstroke}%
\pgfsetdash{}{0pt}%
\pgfpathmoveto{\pgfqpoint{1.309692in}{1.756603in}}%
\pgfpathlineto{\pgfqpoint{1.309692in}{2.224603in}}%
\pgfusepath{stroke}%
\end{pgfscope}%
\begin{pgfscope}%
\pgfsetrectcap%
\pgfsetmiterjoin%
\pgfsetlinewidth{0.803000pt}%
\definecolor{currentstroke}{rgb}{0.000000,0.000000,0.000000}%
\pgfsetstrokecolor{currentstroke}%
\pgfsetdash{}{0pt}%
\pgfpathmoveto{\pgfqpoint{0.629692in}{1.756603in}}%
\pgfpathlineto{\pgfqpoint{1.309692in}{1.756603in}}%
\pgfusepath{stroke}%
\end{pgfscope}%
\begin{pgfscope}%
\pgfsetrectcap%
\pgfsetmiterjoin%
\pgfsetlinewidth{0.803000pt}%
\definecolor{currentstroke}{rgb}{0.000000,0.000000,0.000000}%
\pgfsetstrokecolor{currentstroke}%
\pgfsetdash{}{0pt}%
\pgfpathmoveto{\pgfqpoint{0.629692in}{2.224603in}}%
\pgfpathlineto{\pgfqpoint{1.309692in}{2.224603in}}%
\pgfusepath{stroke}%
\end{pgfscope}%
\begin{pgfscope}%
\pgftext[x=0.969692in,y=2.307937in,,base]{\rmfamily\fontsize{12.000000}{14.400000}\selectfont Local \(\displaystyle \eta_n\)}%
\end{pgfscope}%
\end{pgfpicture}%
\makeatother%
\endgroup%
}
\subfigure[]{%% Creator: Matplotlib, PGF backend
%%
%% To include the figure in your LaTeX document, write
%%   \input{<filename>.pgf}
%%
%% Make sure the required packages are loaded in your preamble
%%   \usepackage{pgf}
%%
%% Figures using additional raster images can only be included by \input if
%% they are in the same directory as the main LaTeX file. For loading figures
%% from other directories you can use the `import` package
%%   \usepackage{import}
%% and then include the figures with
%%   \import{<path to file>}{<filename>.pgf}
%%
%% Matplotlib used the following preamble
%%   \usepackage{fontspec}
%%   \setmainfont{DejaVu Serif}
%%   \setsansfont{DejaVu Sans}
%%   \setmonofont{DejaVu Sans Mono}
%%
\begingroup%
\makeatletter%
\begin{pgfpicture}%
\pgfpathrectangle{\pgfpointorigin}{\pgfqpoint{3.198427in}{2.637365in}}%
\pgfusepath{use as bounding box, clip}%
\begin{pgfscope}%
\pgfsetbuttcap%
\pgfsetmiterjoin%
\definecolor{currentfill}{rgb}{1.000000,1.000000,1.000000}%
\pgfsetfillcolor{currentfill}%
\pgfsetlinewidth{0.000000pt}%
\definecolor{currentstroke}{rgb}{1.000000,1.000000,1.000000}%
\pgfsetstrokecolor{currentstroke}%
\pgfsetdash{}{0pt}%
\pgfpathmoveto{\pgfqpoint{0.000000in}{0.000000in}}%
\pgfpathlineto{\pgfqpoint{3.198427in}{0.000000in}}%
\pgfpathlineto{\pgfqpoint{3.198427in}{2.637365in}}%
\pgfpathlineto{\pgfqpoint{0.000000in}{2.637365in}}%
\pgfpathclose%
\pgfusepath{fill}%
\end{pgfscope}%
\begin{pgfscope}%
\pgfsetbuttcap%
\pgfsetmiterjoin%
\definecolor{currentfill}{rgb}{1.000000,1.000000,1.000000}%
\pgfsetfillcolor{currentfill}%
\pgfsetlinewidth{0.000000pt}%
\definecolor{currentstroke}{rgb}{0.000000,0.000000,0.000000}%
\pgfsetstrokecolor{currentstroke}%
\pgfsetstrokeopacity{0.000000}%
\pgfsetdash{}{0pt}%
\pgfpathmoveto{\pgfqpoint{0.374692in}{0.521603in}}%
\pgfpathlineto{\pgfqpoint{3.009692in}{0.521603in}}%
\pgfpathlineto{\pgfqpoint{3.009692in}{2.484603in}}%
\pgfpathlineto{\pgfqpoint{0.374692in}{2.484603in}}%
\pgfpathclose%
\pgfusepath{fill}%
\end{pgfscope}%
\begin{pgfscope}%
\pgfsetbuttcap%
\pgfsetroundjoin%
\definecolor{currentfill}{rgb}{0.000000,0.000000,0.000000}%
\pgfsetfillcolor{currentfill}%
\pgfsetlinewidth{0.803000pt}%
\definecolor{currentstroke}{rgb}{0.000000,0.000000,0.000000}%
\pgfsetstrokecolor{currentstroke}%
\pgfsetdash{}{0pt}%
\pgfsys@defobject{currentmarker}{\pgfqpoint{0.000000in}{-0.048611in}}{\pgfqpoint{0.000000in}{0.000000in}}{%
\pgfpathmoveto{\pgfqpoint{0.000000in}{0.000000in}}%
\pgfpathlineto{\pgfqpoint{0.000000in}{-0.048611in}}%
\pgfusepath{stroke,fill}%
}%
\begin{pgfscope}%
\pgfsys@transformshift{0.374692in}{0.521603in}%
\pgfsys@useobject{currentmarker}{}%
\end{pgfscope}%
\end{pgfscope}%
\begin{pgfscope}%
\pgftext[x=0.374692in,y=0.424381in,,top]{\rmfamily\fontsize{10.000000}{12.000000}\selectfont \(\displaystyle 0.0\)}%
\end{pgfscope}%
\begin{pgfscope}%
\pgfsetbuttcap%
\pgfsetroundjoin%
\definecolor{currentfill}{rgb}{0.000000,0.000000,0.000000}%
\pgfsetfillcolor{currentfill}%
\pgfsetlinewidth{0.803000pt}%
\definecolor{currentstroke}{rgb}{0.000000,0.000000,0.000000}%
\pgfsetstrokecolor{currentstroke}%
\pgfsetdash{}{0pt}%
\pgfsys@defobject{currentmarker}{\pgfqpoint{0.000000in}{-0.048611in}}{\pgfqpoint{0.000000in}{0.000000in}}{%
\pgfpathmoveto{\pgfqpoint{0.000000in}{0.000000in}}%
\pgfpathlineto{\pgfqpoint{0.000000in}{-0.048611in}}%
\pgfusepath{stroke,fill}%
}%
\begin{pgfscope}%
\pgfsys@transformshift{0.901692in}{0.521603in}%
\pgfsys@useobject{currentmarker}{}%
\end{pgfscope}%
\end{pgfscope}%
\begin{pgfscope}%
\pgftext[x=0.901692in,y=0.424381in,,top]{\rmfamily\fontsize{10.000000}{12.000000}\selectfont \(\displaystyle 0.2\)}%
\end{pgfscope}%
\begin{pgfscope}%
\pgfsetbuttcap%
\pgfsetroundjoin%
\definecolor{currentfill}{rgb}{0.000000,0.000000,0.000000}%
\pgfsetfillcolor{currentfill}%
\pgfsetlinewidth{0.803000pt}%
\definecolor{currentstroke}{rgb}{0.000000,0.000000,0.000000}%
\pgfsetstrokecolor{currentstroke}%
\pgfsetdash{}{0pt}%
\pgfsys@defobject{currentmarker}{\pgfqpoint{0.000000in}{-0.048611in}}{\pgfqpoint{0.000000in}{0.000000in}}{%
\pgfpathmoveto{\pgfqpoint{0.000000in}{0.000000in}}%
\pgfpathlineto{\pgfqpoint{0.000000in}{-0.048611in}}%
\pgfusepath{stroke,fill}%
}%
\begin{pgfscope}%
\pgfsys@transformshift{1.428692in}{0.521603in}%
\pgfsys@useobject{currentmarker}{}%
\end{pgfscope}%
\end{pgfscope}%
\begin{pgfscope}%
\pgftext[x=1.428692in,y=0.424381in,,top]{\rmfamily\fontsize{10.000000}{12.000000}\selectfont \(\displaystyle 0.4\)}%
\end{pgfscope}%
\begin{pgfscope}%
\pgfsetbuttcap%
\pgfsetroundjoin%
\definecolor{currentfill}{rgb}{0.000000,0.000000,0.000000}%
\pgfsetfillcolor{currentfill}%
\pgfsetlinewidth{0.803000pt}%
\definecolor{currentstroke}{rgb}{0.000000,0.000000,0.000000}%
\pgfsetstrokecolor{currentstroke}%
\pgfsetdash{}{0pt}%
\pgfsys@defobject{currentmarker}{\pgfqpoint{0.000000in}{-0.048611in}}{\pgfqpoint{0.000000in}{0.000000in}}{%
\pgfpathmoveto{\pgfqpoint{0.000000in}{0.000000in}}%
\pgfpathlineto{\pgfqpoint{0.000000in}{-0.048611in}}%
\pgfusepath{stroke,fill}%
}%
\begin{pgfscope}%
\pgfsys@transformshift{1.955692in}{0.521603in}%
\pgfsys@useobject{currentmarker}{}%
\end{pgfscope}%
\end{pgfscope}%
\begin{pgfscope}%
\pgftext[x=1.955692in,y=0.424381in,,top]{\rmfamily\fontsize{10.000000}{12.000000}\selectfont \(\displaystyle 0.6\)}%
\end{pgfscope}%
\begin{pgfscope}%
\pgfsetbuttcap%
\pgfsetroundjoin%
\definecolor{currentfill}{rgb}{0.000000,0.000000,0.000000}%
\pgfsetfillcolor{currentfill}%
\pgfsetlinewidth{0.803000pt}%
\definecolor{currentstroke}{rgb}{0.000000,0.000000,0.000000}%
\pgfsetstrokecolor{currentstroke}%
\pgfsetdash{}{0pt}%
\pgfsys@defobject{currentmarker}{\pgfqpoint{0.000000in}{-0.048611in}}{\pgfqpoint{0.000000in}{0.000000in}}{%
\pgfpathmoveto{\pgfqpoint{0.000000in}{0.000000in}}%
\pgfpathlineto{\pgfqpoint{0.000000in}{-0.048611in}}%
\pgfusepath{stroke,fill}%
}%
\begin{pgfscope}%
\pgfsys@transformshift{2.482692in}{0.521603in}%
\pgfsys@useobject{currentmarker}{}%
\end{pgfscope}%
\end{pgfscope}%
\begin{pgfscope}%
\pgftext[x=2.482692in,y=0.424381in,,top]{\rmfamily\fontsize{10.000000}{12.000000}\selectfont \(\displaystyle 0.8\)}%
\end{pgfscope}%
\begin{pgfscope}%
\pgfsetbuttcap%
\pgfsetroundjoin%
\definecolor{currentfill}{rgb}{0.000000,0.000000,0.000000}%
\pgfsetfillcolor{currentfill}%
\pgfsetlinewidth{0.803000pt}%
\definecolor{currentstroke}{rgb}{0.000000,0.000000,0.000000}%
\pgfsetstrokecolor{currentstroke}%
\pgfsetdash{}{0pt}%
\pgfsys@defobject{currentmarker}{\pgfqpoint{0.000000in}{-0.048611in}}{\pgfqpoint{0.000000in}{0.000000in}}{%
\pgfpathmoveto{\pgfqpoint{0.000000in}{0.000000in}}%
\pgfpathlineto{\pgfqpoint{0.000000in}{-0.048611in}}%
\pgfusepath{stroke,fill}%
}%
\begin{pgfscope}%
\pgfsys@transformshift{3.009692in}{0.521603in}%
\pgfsys@useobject{currentmarker}{}%
\end{pgfscope}%
\end{pgfscope}%
\begin{pgfscope}%
\pgftext[x=3.009692in,y=0.424381in,,top]{\rmfamily\fontsize{10.000000}{12.000000}\selectfont \(\displaystyle 1.0\)}%
\end{pgfscope}%
\begin{pgfscope}%
\pgftext[x=1.692192in,y=0.234413in,,top]{\rmfamily\fontsize{10.000000}{12.000000}\selectfont \(\displaystyle z\)}%
\end{pgfscope}%
\begin{pgfscope}%
\pgfsetbuttcap%
\pgfsetroundjoin%
\definecolor{currentfill}{rgb}{0.000000,0.000000,0.000000}%
\pgfsetfillcolor{currentfill}%
\pgfsetlinewidth{0.803000pt}%
\definecolor{currentstroke}{rgb}{0.000000,0.000000,0.000000}%
\pgfsetstrokecolor{currentstroke}%
\pgfsetdash{}{0pt}%
\pgfsys@defobject{currentmarker}{\pgfqpoint{-0.048611in}{0.000000in}}{\pgfqpoint{0.000000in}{0.000000in}}{%
\pgfpathmoveto{\pgfqpoint{0.000000in}{0.000000in}}%
\pgfpathlineto{\pgfqpoint{-0.048611in}{0.000000in}}%
\pgfusepath{stroke,fill}%
}%
\begin{pgfscope}%
\pgfsys@transformshift{0.374692in}{0.521603in}%
\pgfsys@useobject{currentmarker}{}%
\end{pgfscope}%
\end{pgfscope}%
\begin{pgfscope}%
\pgftext[x=0.100000in,y=0.468842in,left,base]{\rmfamily\fontsize{10.000000}{12.000000}\selectfont \(\displaystyle -1\)}%
\end{pgfscope}%
\begin{pgfscope}%
\pgfsetbuttcap%
\pgfsetroundjoin%
\definecolor{currentfill}{rgb}{0.000000,0.000000,0.000000}%
\pgfsetfillcolor{currentfill}%
\pgfsetlinewidth{0.803000pt}%
\definecolor{currentstroke}{rgb}{0.000000,0.000000,0.000000}%
\pgfsetstrokecolor{currentstroke}%
\pgfsetdash{}{0pt}%
\pgfsys@defobject{currentmarker}{\pgfqpoint{-0.048611in}{0.000000in}}{\pgfqpoint{0.000000in}{0.000000in}}{%
\pgfpathmoveto{\pgfqpoint{0.000000in}{0.000000in}}%
\pgfpathlineto{\pgfqpoint{-0.048611in}{0.000000in}}%
\pgfusepath{stroke,fill}%
}%
\begin{pgfscope}%
\pgfsys@transformshift{0.374692in}{0.848770in}%
\pgfsys@useobject{currentmarker}{}%
\end{pgfscope}%
\end{pgfscope}%
\begin{pgfscope}%
\pgftext[x=0.208025in,y=0.796008in,left,base]{\rmfamily\fontsize{10.000000}{12.000000}\selectfont \(\displaystyle 0\)}%
\end{pgfscope}%
\begin{pgfscope}%
\pgfsetbuttcap%
\pgfsetroundjoin%
\definecolor{currentfill}{rgb}{0.000000,0.000000,0.000000}%
\pgfsetfillcolor{currentfill}%
\pgfsetlinewidth{0.803000pt}%
\definecolor{currentstroke}{rgb}{0.000000,0.000000,0.000000}%
\pgfsetstrokecolor{currentstroke}%
\pgfsetdash{}{0pt}%
\pgfsys@defobject{currentmarker}{\pgfqpoint{-0.048611in}{0.000000in}}{\pgfqpoint{0.000000in}{0.000000in}}{%
\pgfpathmoveto{\pgfqpoint{0.000000in}{0.000000in}}%
\pgfpathlineto{\pgfqpoint{-0.048611in}{0.000000in}}%
\pgfusepath{stroke,fill}%
}%
\begin{pgfscope}%
\pgfsys@transformshift{0.374692in}{1.175937in}%
\pgfsys@useobject{currentmarker}{}%
\end{pgfscope}%
\end{pgfscope}%
\begin{pgfscope}%
\pgftext[x=0.208025in,y=1.123175in,left,base]{\rmfamily\fontsize{10.000000}{12.000000}\selectfont \(\displaystyle 1\)}%
\end{pgfscope}%
\begin{pgfscope}%
\pgfsetbuttcap%
\pgfsetroundjoin%
\definecolor{currentfill}{rgb}{0.000000,0.000000,0.000000}%
\pgfsetfillcolor{currentfill}%
\pgfsetlinewidth{0.803000pt}%
\definecolor{currentstroke}{rgb}{0.000000,0.000000,0.000000}%
\pgfsetstrokecolor{currentstroke}%
\pgfsetdash{}{0pt}%
\pgfsys@defobject{currentmarker}{\pgfqpoint{-0.048611in}{0.000000in}}{\pgfqpoint{0.000000in}{0.000000in}}{%
\pgfpathmoveto{\pgfqpoint{0.000000in}{0.000000in}}%
\pgfpathlineto{\pgfqpoint{-0.048611in}{0.000000in}}%
\pgfusepath{stroke,fill}%
}%
\begin{pgfscope}%
\pgfsys@transformshift{0.374692in}{1.503103in}%
\pgfsys@useobject{currentmarker}{}%
\end{pgfscope}%
\end{pgfscope}%
\begin{pgfscope}%
\pgftext[x=0.208025in,y=1.450342in,left,base]{\rmfamily\fontsize{10.000000}{12.000000}\selectfont \(\displaystyle 2\)}%
\end{pgfscope}%
\begin{pgfscope}%
\pgfsetbuttcap%
\pgfsetroundjoin%
\definecolor{currentfill}{rgb}{0.000000,0.000000,0.000000}%
\pgfsetfillcolor{currentfill}%
\pgfsetlinewidth{0.803000pt}%
\definecolor{currentstroke}{rgb}{0.000000,0.000000,0.000000}%
\pgfsetstrokecolor{currentstroke}%
\pgfsetdash{}{0pt}%
\pgfsys@defobject{currentmarker}{\pgfqpoint{-0.048611in}{0.000000in}}{\pgfqpoint{0.000000in}{0.000000in}}{%
\pgfpathmoveto{\pgfqpoint{0.000000in}{0.000000in}}%
\pgfpathlineto{\pgfqpoint{-0.048611in}{0.000000in}}%
\pgfusepath{stroke,fill}%
}%
\begin{pgfscope}%
\pgfsys@transformshift{0.374692in}{1.830270in}%
\pgfsys@useobject{currentmarker}{}%
\end{pgfscope}%
\end{pgfscope}%
\begin{pgfscope}%
\pgftext[x=0.208025in,y=1.777508in,left,base]{\rmfamily\fontsize{10.000000}{12.000000}\selectfont \(\displaystyle 3\)}%
\end{pgfscope}%
\begin{pgfscope}%
\pgfsetbuttcap%
\pgfsetroundjoin%
\definecolor{currentfill}{rgb}{0.000000,0.000000,0.000000}%
\pgfsetfillcolor{currentfill}%
\pgfsetlinewidth{0.803000pt}%
\definecolor{currentstroke}{rgb}{0.000000,0.000000,0.000000}%
\pgfsetstrokecolor{currentstroke}%
\pgfsetdash{}{0pt}%
\pgfsys@defobject{currentmarker}{\pgfqpoint{-0.048611in}{0.000000in}}{\pgfqpoint{0.000000in}{0.000000in}}{%
\pgfpathmoveto{\pgfqpoint{0.000000in}{0.000000in}}%
\pgfpathlineto{\pgfqpoint{-0.048611in}{0.000000in}}%
\pgfusepath{stroke,fill}%
}%
\begin{pgfscope}%
\pgfsys@transformshift{0.374692in}{2.157437in}%
\pgfsys@useobject{currentmarker}{}%
\end{pgfscope}%
\end{pgfscope}%
\begin{pgfscope}%
\pgftext[x=0.208025in,y=2.104675in,left,base]{\rmfamily\fontsize{10.000000}{12.000000}\selectfont \(\displaystyle 4\)}%
\end{pgfscope}%
\begin{pgfscope}%
\pgfsetbuttcap%
\pgfsetroundjoin%
\definecolor{currentfill}{rgb}{0.000000,0.000000,0.000000}%
\pgfsetfillcolor{currentfill}%
\pgfsetlinewidth{0.803000pt}%
\definecolor{currentstroke}{rgb}{0.000000,0.000000,0.000000}%
\pgfsetstrokecolor{currentstroke}%
\pgfsetdash{}{0pt}%
\pgfsys@defobject{currentmarker}{\pgfqpoint{-0.048611in}{0.000000in}}{\pgfqpoint{0.000000in}{0.000000in}}{%
\pgfpathmoveto{\pgfqpoint{0.000000in}{0.000000in}}%
\pgfpathlineto{\pgfqpoint{-0.048611in}{0.000000in}}%
\pgfusepath{stroke,fill}%
}%
\begin{pgfscope}%
\pgfsys@transformshift{0.374692in}{2.484603in}%
\pgfsys@useobject{currentmarker}{}%
\end{pgfscope}%
\end{pgfscope}%
\begin{pgfscope}%
\pgftext[x=0.208025in,y=2.431842in,left,base]{\rmfamily\fontsize{10.000000}{12.000000}\selectfont \(\displaystyle 5\)}%
\end{pgfscope}%
\begin{pgfscope}%
\pgfpathrectangle{\pgfqpoint{0.374692in}{0.521603in}}{\pgfqpoint{2.635000in}{1.963000in}} %
\pgfusepath{clip}%
\pgfsetbuttcap%
\pgfsetroundjoin%
\pgfsetlinewidth{1.505625pt}%
\definecolor{currentstroke}{rgb}{0.000000,0.000000,0.000000}%
\pgfsetstrokecolor{currentstroke}%
\pgfsetdash{{5.550000pt}{2.400000pt}}{0.000000pt}%
\pgfpathmoveto{\pgfqpoint{2.131359in}{0.521603in}}%
\pgfpathlineto{\pgfqpoint{2.131359in}{0.630659in}}%
\pgfpathlineto{\pgfqpoint{2.131359in}{0.739714in}}%
\pgfpathlineto{\pgfqpoint{2.131359in}{0.848770in}}%
\pgfpathlineto{\pgfqpoint{2.131359in}{0.957826in}}%
\pgfpathlineto{\pgfqpoint{2.131359in}{1.066881in}}%
\pgfpathlineto{\pgfqpoint{2.131359in}{1.175937in}}%
\pgfpathlineto{\pgfqpoint{2.131359in}{1.284992in}}%
\pgfpathlineto{\pgfqpoint{2.131359in}{1.394048in}}%
\pgfpathlineto{\pgfqpoint{2.131359in}{1.503103in}}%
\pgfusepath{stroke}%
\end{pgfscope}%
\begin{pgfscope}%
\pgfpathrectangle{\pgfqpoint{0.374692in}{0.521603in}}{\pgfqpoint{2.635000in}{1.963000in}} %
\pgfusepath{clip}%
\pgfsetrectcap%
\pgfsetroundjoin%
\pgfsetlinewidth{1.003750pt}%
\definecolor{currentstroke}{rgb}{0.121569,0.466667,0.705882}%
\pgfsetstrokecolor{currentstroke}%
\pgfsetdash{}{0pt}%
\pgfpathmoveto{\pgfqpoint{0.374692in}{1.339520in}}%
\pgfpathlineto{\pgfqpoint{0.383564in}{1.332911in}}%
\pgfpathlineto{\pgfqpoint{0.392436in}{1.326301in}}%
\pgfpathlineto{\pgfqpoint{0.401308in}{1.319692in}}%
\pgfpathlineto{\pgfqpoint{0.410180in}{1.313082in}}%
\pgfpathlineto{\pgfqpoint{0.419052in}{1.306473in}}%
\pgfpathlineto{\pgfqpoint{0.427924in}{1.299863in}}%
\pgfpathlineto{\pgfqpoint{0.436796in}{1.293254in}}%
\pgfpathlineto{\pgfqpoint{0.445668in}{1.286645in}}%
\pgfpathlineto{\pgfqpoint{0.454540in}{1.280035in}}%
\pgfpathlineto{\pgfqpoint{0.463412in}{1.273426in}}%
\pgfpathlineto{\pgfqpoint{0.472285in}{1.266816in}}%
\pgfpathlineto{\pgfqpoint{0.481157in}{1.260207in}}%
\pgfpathlineto{\pgfqpoint{0.490029in}{1.253597in}}%
\pgfpathlineto{\pgfqpoint{0.498901in}{1.246988in}}%
\pgfpathlineto{\pgfqpoint{0.507773in}{1.240379in}}%
\pgfpathlineto{\pgfqpoint{0.516645in}{1.233769in}}%
\pgfpathlineto{\pgfqpoint{0.525517in}{1.227160in}}%
\pgfpathlineto{\pgfqpoint{0.534389in}{1.220550in}}%
\pgfpathlineto{\pgfqpoint{0.543261in}{1.213941in}}%
\pgfpathlineto{\pgfqpoint{0.552133in}{1.207331in}}%
\pgfpathlineto{\pgfqpoint{0.561005in}{1.200722in}}%
\pgfpathlineto{\pgfqpoint{0.569877in}{1.194113in}}%
\pgfpathlineto{\pgfqpoint{0.578749in}{1.187503in}}%
\pgfpathlineto{\pgfqpoint{0.587621in}{1.180894in}}%
\pgfpathlineto{\pgfqpoint{0.596493in}{1.174284in}}%
\pgfpathlineto{\pgfqpoint{0.605365in}{1.167675in}}%
\pgfpathlineto{\pgfqpoint{0.614237in}{1.161065in}}%
\pgfpathlineto{\pgfqpoint{0.623109in}{1.154456in}}%
\pgfpathlineto{\pgfqpoint{0.631982in}{1.147847in}}%
\pgfpathlineto{\pgfqpoint{0.640854in}{1.141237in}}%
\pgfpathlineto{\pgfqpoint{0.649726in}{1.134628in}}%
\pgfpathlineto{\pgfqpoint{0.658598in}{1.128018in}}%
\pgfpathlineto{\pgfqpoint{0.667470in}{1.121409in}}%
\pgfpathlineto{\pgfqpoint{0.676342in}{1.114799in}}%
\pgfpathlineto{\pgfqpoint{0.685214in}{1.108190in}}%
\pgfpathlineto{\pgfqpoint{0.694086in}{1.101581in}}%
\pgfpathlineto{\pgfqpoint{0.702958in}{1.094971in}}%
\pgfpathlineto{\pgfqpoint{0.711830in}{1.088362in}}%
\pgfpathlineto{\pgfqpoint{0.720702in}{1.081752in}}%
\pgfpathlineto{\pgfqpoint{0.729574in}{1.075143in}}%
\pgfpathlineto{\pgfqpoint{0.738446in}{1.068533in}}%
\pgfpathlineto{\pgfqpoint{0.747318in}{1.061924in}}%
\pgfpathlineto{\pgfqpoint{0.756190in}{1.055315in}}%
\pgfpathlineto{\pgfqpoint{0.765062in}{1.048705in}}%
\pgfpathlineto{\pgfqpoint{0.773934in}{1.042096in}}%
\pgfpathlineto{\pgfqpoint{0.782806in}{1.035486in}}%
\pgfpathlineto{\pgfqpoint{0.791678in}{1.028877in}}%
\pgfpathlineto{\pgfqpoint{0.800551in}{1.022267in}}%
\pgfpathlineto{\pgfqpoint{0.809423in}{1.015658in}}%
\pgfpathlineto{\pgfqpoint{0.818295in}{1.009049in}}%
\pgfpathlineto{\pgfqpoint{0.827167in}{1.002439in}}%
\pgfpathlineto{\pgfqpoint{0.836039in}{0.995830in}}%
\pgfpathlineto{\pgfqpoint{0.844911in}{0.989220in}}%
\pgfpathlineto{\pgfqpoint{0.853783in}{0.982611in}}%
\pgfpathlineto{\pgfqpoint{0.862655in}{0.976001in}}%
\pgfpathlineto{\pgfqpoint{0.871527in}{0.969392in}}%
\pgfpathlineto{\pgfqpoint{0.880399in}{0.962783in}}%
\pgfpathlineto{\pgfqpoint{0.889271in}{0.956173in}}%
\pgfpathlineto{\pgfqpoint{0.898143in}{0.949564in}}%
\pgfpathlineto{\pgfqpoint{0.907015in}{0.942954in}}%
\pgfpathlineto{\pgfqpoint{0.915887in}{0.936345in}}%
\pgfpathlineto{\pgfqpoint{0.924759in}{0.929735in}}%
\pgfpathlineto{\pgfqpoint{0.933631in}{0.923126in}}%
\pgfpathlineto{\pgfqpoint{0.942503in}{0.916517in}}%
\pgfpathlineto{\pgfqpoint{0.951375in}{0.909907in}}%
\pgfpathlineto{\pgfqpoint{0.960248in}{0.903298in}}%
\pgfpathlineto{\pgfqpoint{0.969120in}{0.896688in}}%
\pgfpathlineto{\pgfqpoint{0.977992in}{0.890079in}}%
\pgfpathlineto{\pgfqpoint{0.986864in}{0.883469in}}%
\pgfpathlineto{\pgfqpoint{0.995736in}{0.876860in}}%
\pgfpathlineto{\pgfqpoint{1.004608in}{0.870251in}}%
\pgfpathlineto{\pgfqpoint{1.013480in}{0.863641in}}%
\pgfpathlineto{\pgfqpoint{1.022352in}{0.857032in}}%
\pgfpathlineto{\pgfqpoint{1.031224in}{0.850422in}}%
\pgfpathlineto{\pgfqpoint{1.040096in}{0.843813in}}%
\pgfpathlineto{\pgfqpoint{1.048968in}{0.837204in}}%
\pgfpathlineto{\pgfqpoint{1.057840in}{0.830594in}}%
\pgfpathlineto{\pgfqpoint{1.066712in}{0.823985in}}%
\pgfpathlineto{\pgfqpoint{1.075584in}{0.817375in}}%
\pgfpathlineto{\pgfqpoint{1.084456in}{0.810766in}}%
\pgfpathlineto{\pgfqpoint{1.093328in}{0.804156in}}%
\pgfpathlineto{\pgfqpoint{1.102200in}{0.797547in}}%
\pgfpathlineto{\pgfqpoint{1.111072in}{0.790938in}}%
\pgfpathlineto{\pgfqpoint{1.119944in}{0.784328in}}%
\pgfpathlineto{\pgfqpoint{1.128817in}{0.777719in}}%
\pgfpathlineto{\pgfqpoint{1.137689in}{0.771109in}}%
\pgfpathlineto{\pgfqpoint{1.146561in}{0.764500in}}%
\pgfpathlineto{\pgfqpoint{1.155433in}{0.757890in}}%
\pgfpathlineto{\pgfqpoint{1.164305in}{0.751281in}}%
\pgfpathlineto{\pgfqpoint{1.173177in}{0.744672in}}%
\pgfpathlineto{\pgfqpoint{1.182049in}{0.738062in}}%
\pgfpathlineto{\pgfqpoint{1.190921in}{0.731453in}}%
\pgfpathlineto{\pgfqpoint{1.199793in}{0.724843in}}%
\pgfpathlineto{\pgfqpoint{1.208665in}{0.718234in}}%
\pgfpathlineto{\pgfqpoint{1.217537in}{0.711624in}}%
\pgfpathlineto{\pgfqpoint{1.226409in}{0.705015in}}%
\pgfpathlineto{\pgfqpoint{1.235281in}{0.698406in}}%
\pgfpathlineto{\pgfqpoint{1.244153in}{0.691796in}}%
\pgfpathlineto{\pgfqpoint{1.253025in}{0.685187in}}%
\pgfusepath{stroke}%
\end{pgfscope}%
\begin{pgfscope}%
\pgfpathrectangle{\pgfqpoint{0.374692in}{0.521603in}}{\pgfqpoint{2.635000in}{1.963000in}} %
\pgfusepath{clip}%
\pgfsetrectcap%
\pgfsetroundjoin%
\pgfsetlinewidth{1.003750pt}%
\definecolor{currentstroke}{rgb}{1.000000,0.498039,0.054902}%
\pgfsetstrokecolor{currentstroke}%
\pgfsetdash{}{0pt}%
\pgfpathmoveto{\pgfqpoint{0.374692in}{0.685187in}}%
\pgfpathlineto{\pgfqpoint{0.383564in}{0.691796in}}%
\pgfpathlineto{\pgfqpoint{0.392436in}{0.698406in}}%
\pgfpathlineto{\pgfqpoint{0.401308in}{0.705015in}}%
\pgfpathlineto{\pgfqpoint{0.410180in}{0.711624in}}%
\pgfpathlineto{\pgfqpoint{0.419052in}{0.718234in}}%
\pgfpathlineto{\pgfqpoint{0.427924in}{0.724843in}}%
\pgfpathlineto{\pgfqpoint{0.436796in}{0.731453in}}%
\pgfpathlineto{\pgfqpoint{0.445668in}{0.738062in}}%
\pgfpathlineto{\pgfqpoint{0.454540in}{0.744672in}}%
\pgfpathlineto{\pgfqpoint{0.463412in}{0.751281in}}%
\pgfpathlineto{\pgfqpoint{0.472285in}{0.757890in}}%
\pgfpathlineto{\pgfqpoint{0.481157in}{0.764500in}}%
\pgfpathlineto{\pgfqpoint{0.490029in}{0.771109in}}%
\pgfpathlineto{\pgfqpoint{0.498901in}{0.777719in}}%
\pgfpathlineto{\pgfqpoint{0.507773in}{0.784328in}}%
\pgfpathlineto{\pgfqpoint{0.516645in}{0.790938in}}%
\pgfpathlineto{\pgfqpoint{0.525517in}{0.797547in}}%
\pgfpathlineto{\pgfqpoint{0.534389in}{0.804156in}}%
\pgfpathlineto{\pgfqpoint{0.543261in}{0.810766in}}%
\pgfpathlineto{\pgfqpoint{0.552133in}{0.817375in}}%
\pgfpathlineto{\pgfqpoint{0.561005in}{0.823985in}}%
\pgfpathlineto{\pgfqpoint{0.569877in}{0.830594in}}%
\pgfpathlineto{\pgfqpoint{0.578749in}{0.837204in}}%
\pgfpathlineto{\pgfqpoint{0.587621in}{0.843813in}}%
\pgfpathlineto{\pgfqpoint{0.596493in}{0.850422in}}%
\pgfpathlineto{\pgfqpoint{0.605365in}{0.857032in}}%
\pgfpathlineto{\pgfqpoint{0.614237in}{0.863641in}}%
\pgfpathlineto{\pgfqpoint{0.623109in}{0.870251in}}%
\pgfpathlineto{\pgfqpoint{0.631982in}{0.876860in}}%
\pgfpathlineto{\pgfqpoint{0.640854in}{0.883469in}}%
\pgfpathlineto{\pgfqpoint{0.649726in}{0.890079in}}%
\pgfpathlineto{\pgfqpoint{0.658598in}{0.896688in}}%
\pgfpathlineto{\pgfqpoint{0.667470in}{0.903298in}}%
\pgfpathlineto{\pgfqpoint{0.676342in}{0.909907in}}%
\pgfpathlineto{\pgfqpoint{0.685214in}{0.916517in}}%
\pgfpathlineto{\pgfqpoint{0.694086in}{0.923126in}}%
\pgfpathlineto{\pgfqpoint{0.702958in}{0.929735in}}%
\pgfpathlineto{\pgfqpoint{0.711830in}{0.936345in}}%
\pgfpathlineto{\pgfqpoint{0.720702in}{0.942954in}}%
\pgfpathlineto{\pgfqpoint{0.729574in}{0.949564in}}%
\pgfpathlineto{\pgfqpoint{0.738446in}{0.956173in}}%
\pgfpathlineto{\pgfqpoint{0.747318in}{0.962783in}}%
\pgfpathlineto{\pgfqpoint{0.756190in}{0.969392in}}%
\pgfpathlineto{\pgfqpoint{0.765062in}{0.976001in}}%
\pgfpathlineto{\pgfqpoint{0.773934in}{0.982611in}}%
\pgfpathlineto{\pgfqpoint{0.782806in}{0.989220in}}%
\pgfpathlineto{\pgfqpoint{0.791678in}{0.995830in}}%
\pgfpathlineto{\pgfqpoint{0.800551in}{1.002439in}}%
\pgfpathlineto{\pgfqpoint{0.809423in}{1.009049in}}%
\pgfpathlineto{\pgfqpoint{0.818295in}{1.015658in}}%
\pgfpathlineto{\pgfqpoint{0.827167in}{1.022267in}}%
\pgfpathlineto{\pgfqpoint{0.836039in}{1.028877in}}%
\pgfpathlineto{\pgfqpoint{0.844911in}{1.035486in}}%
\pgfpathlineto{\pgfqpoint{0.853783in}{1.042096in}}%
\pgfpathlineto{\pgfqpoint{0.862655in}{1.048705in}}%
\pgfpathlineto{\pgfqpoint{0.871527in}{1.055315in}}%
\pgfpathlineto{\pgfqpoint{0.880399in}{1.061924in}}%
\pgfpathlineto{\pgfqpoint{0.889271in}{1.068533in}}%
\pgfpathlineto{\pgfqpoint{0.898143in}{1.075143in}}%
\pgfpathlineto{\pgfqpoint{0.907015in}{1.081752in}}%
\pgfpathlineto{\pgfqpoint{0.915887in}{1.088362in}}%
\pgfpathlineto{\pgfqpoint{0.924759in}{1.094971in}}%
\pgfpathlineto{\pgfqpoint{0.933631in}{1.101581in}}%
\pgfpathlineto{\pgfqpoint{0.942503in}{1.108190in}}%
\pgfpathlineto{\pgfqpoint{0.951375in}{1.114799in}}%
\pgfpathlineto{\pgfqpoint{0.960248in}{1.121409in}}%
\pgfpathlineto{\pgfqpoint{0.969120in}{1.128018in}}%
\pgfpathlineto{\pgfqpoint{0.977992in}{1.134628in}}%
\pgfpathlineto{\pgfqpoint{0.986864in}{1.141237in}}%
\pgfpathlineto{\pgfqpoint{0.995736in}{1.147847in}}%
\pgfpathlineto{\pgfqpoint{1.004608in}{1.154456in}}%
\pgfpathlineto{\pgfqpoint{1.013480in}{1.161065in}}%
\pgfpathlineto{\pgfqpoint{1.022352in}{1.167675in}}%
\pgfpathlineto{\pgfqpoint{1.031224in}{1.174284in}}%
\pgfpathlineto{\pgfqpoint{1.040096in}{1.180894in}}%
\pgfpathlineto{\pgfqpoint{1.048968in}{1.187503in}}%
\pgfpathlineto{\pgfqpoint{1.057840in}{1.194113in}}%
\pgfpathlineto{\pgfqpoint{1.066712in}{1.200722in}}%
\pgfpathlineto{\pgfqpoint{1.075584in}{1.207331in}}%
\pgfpathlineto{\pgfqpoint{1.084456in}{1.213941in}}%
\pgfpathlineto{\pgfqpoint{1.093328in}{1.220550in}}%
\pgfpathlineto{\pgfqpoint{1.102200in}{1.227160in}}%
\pgfpathlineto{\pgfqpoint{1.111072in}{1.233769in}}%
\pgfpathlineto{\pgfqpoint{1.119944in}{1.240379in}}%
\pgfpathlineto{\pgfqpoint{1.128817in}{1.246988in}}%
\pgfpathlineto{\pgfqpoint{1.137689in}{1.253597in}}%
\pgfpathlineto{\pgfqpoint{1.146561in}{1.260207in}}%
\pgfpathlineto{\pgfqpoint{1.155433in}{1.266816in}}%
\pgfpathlineto{\pgfqpoint{1.164305in}{1.273426in}}%
\pgfpathlineto{\pgfqpoint{1.173177in}{1.280035in}}%
\pgfpathlineto{\pgfqpoint{1.182049in}{1.286645in}}%
\pgfpathlineto{\pgfqpoint{1.190921in}{1.293254in}}%
\pgfpathlineto{\pgfqpoint{1.199793in}{1.299863in}}%
\pgfpathlineto{\pgfqpoint{1.208665in}{1.306473in}}%
\pgfpathlineto{\pgfqpoint{1.217537in}{1.313082in}}%
\pgfpathlineto{\pgfqpoint{1.226409in}{1.319692in}}%
\pgfpathlineto{\pgfqpoint{1.235281in}{1.326301in}}%
\pgfpathlineto{\pgfqpoint{1.244153in}{1.332911in}}%
\pgfpathlineto{\pgfqpoint{1.253025in}{1.339520in}}%
\pgfusepath{stroke}%
\end{pgfscope}%
\begin{pgfscope}%
\pgfpathrectangle{\pgfqpoint{0.374692in}{0.521603in}}{\pgfqpoint{2.635000in}{1.963000in}} %
\pgfusepath{clip}%
\pgfsetbuttcap%
\pgfsetroundjoin%
\definecolor{currentfill}{rgb}{1.000000,0.000000,0.000000}%
\pgfsetfillcolor{currentfill}%
\pgfsetlinewidth{1.003750pt}%
\definecolor{currentstroke}{rgb}{1.000000,0.000000,0.000000}%
\pgfsetstrokecolor{currentstroke}%
\pgfsetdash{}{0pt}%
\pgfsys@defobject{currentmarker}{\pgfqpoint{-0.020833in}{-0.020833in}}{\pgfqpoint{0.020833in}{0.020833in}}{%
\pgfpathmoveto{\pgfqpoint{0.000000in}{-0.020833in}}%
\pgfpathcurveto{\pgfqpoint{0.005525in}{-0.020833in}}{\pgfqpoint{0.010825in}{-0.018638in}}{\pgfqpoint{0.014731in}{-0.014731in}}%
\pgfpathcurveto{\pgfqpoint{0.018638in}{-0.010825in}}{\pgfqpoint{0.020833in}{-0.005525in}}{\pgfqpoint{0.020833in}{0.000000in}}%
\pgfpathcurveto{\pgfqpoint{0.020833in}{0.005525in}}{\pgfqpoint{0.018638in}{0.010825in}}{\pgfqpoint{0.014731in}{0.014731in}}%
\pgfpathcurveto{\pgfqpoint{0.010825in}{0.018638in}}{\pgfqpoint{0.005525in}{0.020833in}}{\pgfqpoint{0.000000in}{0.020833in}}%
\pgfpathcurveto{\pgfqpoint{-0.005525in}{0.020833in}}{\pgfqpoint{-0.010825in}{0.018638in}}{\pgfqpoint{-0.014731in}{0.014731in}}%
\pgfpathcurveto{\pgfqpoint{-0.018638in}{0.010825in}}{\pgfqpoint{-0.020833in}{0.005525in}}{\pgfqpoint{-0.020833in}{0.000000in}}%
\pgfpathcurveto{\pgfqpoint{-0.020833in}{-0.005525in}}{\pgfqpoint{-0.018638in}{-0.010825in}}{\pgfqpoint{-0.014731in}{-0.014731in}}%
\pgfpathcurveto{\pgfqpoint{-0.010825in}{-0.018638in}}{\pgfqpoint{-0.005525in}{-0.020833in}}{\pgfqpoint{0.000000in}{-0.020833in}}%
\pgfpathclose%
\pgfusepath{stroke,fill}%
}%
\begin{pgfscope}%
\pgfsys@transformshift{0.374692in}{0.848770in}%
\pgfsys@useobject{currentmarker}{}%
\end{pgfscope}%
\begin{pgfscope}%
\pgfsys@transformshift{0.813859in}{0.848770in}%
\pgfsys@useobject{currentmarker}{}%
\end{pgfscope}%
\begin{pgfscope}%
\pgfsys@transformshift{1.253025in}{0.848770in}%
\pgfsys@useobject{currentmarker}{}%
\end{pgfscope}%
\end{pgfscope}%
\begin{pgfscope}%
\pgfpathrectangle{\pgfqpoint{0.374692in}{0.521603in}}{\pgfqpoint{2.635000in}{1.963000in}} %
\pgfusepath{clip}%
\pgfsetbuttcap%
\pgfsetroundjoin%
\definecolor{currentfill}{rgb}{1.000000,0.000000,0.000000}%
\pgfsetfillcolor{currentfill}%
\pgfsetlinewidth{1.003750pt}%
\definecolor{currentstroke}{rgb}{1.000000,0.000000,0.000000}%
\pgfsetstrokecolor{currentstroke}%
\pgfsetdash{}{0pt}%
\pgfsys@defobject{currentmarker}{\pgfqpoint{-0.020833in}{-0.020833in}}{\pgfqpoint{0.020833in}{0.020833in}}{%
\pgfpathmoveto{\pgfqpoint{0.000000in}{-0.020833in}}%
\pgfpathcurveto{\pgfqpoint{0.005525in}{-0.020833in}}{\pgfqpoint{0.010825in}{-0.018638in}}{\pgfqpoint{0.014731in}{-0.014731in}}%
\pgfpathcurveto{\pgfqpoint{0.018638in}{-0.010825in}}{\pgfqpoint{0.020833in}{-0.005525in}}{\pgfqpoint{0.020833in}{0.000000in}}%
\pgfpathcurveto{\pgfqpoint{0.020833in}{0.005525in}}{\pgfqpoint{0.018638in}{0.010825in}}{\pgfqpoint{0.014731in}{0.014731in}}%
\pgfpathcurveto{\pgfqpoint{0.010825in}{0.018638in}}{\pgfqpoint{0.005525in}{0.020833in}}{\pgfqpoint{0.000000in}{0.020833in}}%
\pgfpathcurveto{\pgfqpoint{-0.005525in}{0.020833in}}{\pgfqpoint{-0.010825in}{0.018638in}}{\pgfqpoint{-0.014731in}{0.014731in}}%
\pgfpathcurveto{\pgfqpoint{-0.018638in}{0.010825in}}{\pgfqpoint{-0.020833in}{0.005525in}}{\pgfqpoint{-0.020833in}{0.000000in}}%
\pgfpathcurveto{\pgfqpoint{-0.020833in}{-0.005525in}}{\pgfqpoint{-0.018638in}{-0.010825in}}{\pgfqpoint{-0.014731in}{-0.014731in}}%
\pgfpathcurveto{\pgfqpoint{-0.010825in}{-0.018638in}}{\pgfqpoint{-0.005525in}{-0.020833in}}{\pgfqpoint{0.000000in}{-0.020833in}}%
\pgfpathclose%
\pgfusepath{stroke,fill}%
}%
\begin{pgfscope}%
\pgfsys@transformshift{0.374692in}{0.848770in}%
\pgfsys@useobject{currentmarker}{}%
\end{pgfscope}%
\begin{pgfscope}%
\pgfsys@transformshift{0.813859in}{0.848770in}%
\pgfsys@useobject{currentmarker}{}%
\end{pgfscope}%
\begin{pgfscope}%
\pgfsys@transformshift{1.253025in}{0.848770in}%
\pgfsys@useobject{currentmarker}{}%
\end{pgfscope}%
\end{pgfscope}%
\begin{pgfscope}%
\pgfpathrectangle{\pgfqpoint{0.374692in}{0.521603in}}{\pgfqpoint{2.635000in}{1.963000in}} %
\pgfusepath{clip}%
\pgfsetbuttcap%
\pgfsetroundjoin%
\pgfsetlinewidth{1.505625pt}%
\definecolor{currentstroke}{rgb}{0.000000,0.000000,0.000000}%
\pgfsetstrokecolor{currentstroke}%
\pgfsetdash{{5.550000pt}{2.400000pt}}{0.000000pt}%
\pgfpathmoveto{\pgfqpoint{1.253025in}{0.521603in}}%
\pgfpathlineto{\pgfqpoint{1.253025in}{0.630659in}}%
\pgfpathlineto{\pgfqpoint{1.253025in}{0.739714in}}%
\pgfpathlineto{\pgfqpoint{1.253025in}{0.848770in}}%
\pgfpathlineto{\pgfqpoint{1.253025in}{0.957826in}}%
\pgfpathlineto{\pgfqpoint{1.253025in}{1.066881in}}%
\pgfpathlineto{\pgfqpoint{1.253025in}{1.175937in}}%
\pgfpathlineto{\pgfqpoint{1.253025in}{1.284992in}}%
\pgfpathlineto{\pgfqpoint{1.253025in}{1.394048in}}%
\pgfpathlineto{\pgfqpoint{1.253025in}{1.503103in}}%
\pgfusepath{stroke}%
\end{pgfscope}%
\begin{pgfscope}%
\pgfpathrectangle{\pgfqpoint{0.374692in}{0.521603in}}{\pgfqpoint{2.635000in}{1.963000in}} %
\pgfusepath{clip}%
\pgfsetrectcap%
\pgfsetroundjoin%
\pgfsetlinewidth{1.003750pt}%
\definecolor{currentstroke}{rgb}{0.172549,0.627451,0.172549}%
\pgfsetstrokecolor{currentstroke}%
\pgfsetdash{}{0pt}%
\pgfpathmoveto{\pgfqpoint{1.253025in}{1.339520in}}%
\pgfpathlineto{\pgfqpoint{1.261897in}{1.332911in}}%
\pgfpathlineto{\pgfqpoint{1.270769in}{1.326301in}}%
\pgfpathlineto{\pgfqpoint{1.279641in}{1.319692in}}%
\pgfpathlineto{\pgfqpoint{1.288513in}{1.313082in}}%
\pgfpathlineto{\pgfqpoint{1.297386in}{1.306473in}}%
\pgfpathlineto{\pgfqpoint{1.306258in}{1.299863in}}%
\pgfpathlineto{\pgfqpoint{1.315130in}{1.293254in}}%
\pgfpathlineto{\pgfqpoint{1.324002in}{1.286645in}}%
\pgfpathlineto{\pgfqpoint{1.332874in}{1.280035in}}%
\pgfpathlineto{\pgfqpoint{1.341746in}{1.273426in}}%
\pgfpathlineto{\pgfqpoint{1.350618in}{1.266816in}}%
\pgfpathlineto{\pgfqpoint{1.359490in}{1.260207in}}%
\pgfpathlineto{\pgfqpoint{1.368362in}{1.253597in}}%
\pgfpathlineto{\pgfqpoint{1.377234in}{1.246988in}}%
\pgfpathlineto{\pgfqpoint{1.386106in}{1.240379in}}%
\pgfpathlineto{\pgfqpoint{1.394978in}{1.233769in}}%
\pgfpathlineto{\pgfqpoint{1.403850in}{1.227160in}}%
\pgfpathlineto{\pgfqpoint{1.412722in}{1.220550in}}%
\pgfpathlineto{\pgfqpoint{1.421594in}{1.213941in}}%
\pgfpathlineto{\pgfqpoint{1.430466in}{1.207331in}}%
\pgfpathlineto{\pgfqpoint{1.439338in}{1.200722in}}%
\pgfpathlineto{\pgfqpoint{1.448210in}{1.194113in}}%
\pgfpathlineto{\pgfqpoint{1.457083in}{1.187503in}}%
\pgfpathlineto{\pgfqpoint{1.465955in}{1.180894in}}%
\pgfpathlineto{\pgfqpoint{1.474827in}{1.174284in}}%
\pgfpathlineto{\pgfqpoint{1.483699in}{1.167675in}}%
\pgfpathlineto{\pgfqpoint{1.492571in}{1.161065in}}%
\pgfpathlineto{\pgfqpoint{1.501443in}{1.154456in}}%
\pgfpathlineto{\pgfqpoint{1.510315in}{1.147847in}}%
\pgfpathlineto{\pgfqpoint{1.519187in}{1.141237in}}%
\pgfpathlineto{\pgfqpoint{1.528059in}{1.134628in}}%
\pgfpathlineto{\pgfqpoint{1.536931in}{1.128018in}}%
\pgfpathlineto{\pgfqpoint{1.545803in}{1.121409in}}%
\pgfpathlineto{\pgfqpoint{1.554675in}{1.114799in}}%
\pgfpathlineto{\pgfqpoint{1.563547in}{1.108190in}}%
\pgfpathlineto{\pgfqpoint{1.572419in}{1.101581in}}%
\pgfpathlineto{\pgfqpoint{1.581291in}{1.094971in}}%
\pgfpathlineto{\pgfqpoint{1.590163in}{1.088362in}}%
\pgfpathlineto{\pgfqpoint{1.599035in}{1.081752in}}%
\pgfpathlineto{\pgfqpoint{1.607907in}{1.075143in}}%
\pgfpathlineto{\pgfqpoint{1.616779in}{1.068533in}}%
\pgfpathlineto{\pgfqpoint{1.625652in}{1.061924in}}%
\pgfpathlineto{\pgfqpoint{1.634524in}{1.055315in}}%
\pgfpathlineto{\pgfqpoint{1.643396in}{1.048705in}}%
\pgfpathlineto{\pgfqpoint{1.652268in}{1.042096in}}%
\pgfpathlineto{\pgfqpoint{1.661140in}{1.035486in}}%
\pgfpathlineto{\pgfqpoint{1.670012in}{1.028877in}}%
\pgfpathlineto{\pgfqpoint{1.678884in}{1.022267in}}%
\pgfpathlineto{\pgfqpoint{1.687756in}{1.015658in}}%
\pgfpathlineto{\pgfqpoint{1.696628in}{1.009049in}}%
\pgfpathlineto{\pgfqpoint{1.705500in}{1.002439in}}%
\pgfpathlineto{\pgfqpoint{1.714372in}{0.995830in}}%
\pgfpathlineto{\pgfqpoint{1.723244in}{0.989220in}}%
\pgfpathlineto{\pgfqpoint{1.732116in}{0.982611in}}%
\pgfpathlineto{\pgfqpoint{1.740988in}{0.976001in}}%
\pgfpathlineto{\pgfqpoint{1.749860in}{0.969392in}}%
\pgfpathlineto{\pgfqpoint{1.758732in}{0.962783in}}%
\pgfpathlineto{\pgfqpoint{1.767604in}{0.956173in}}%
\pgfpathlineto{\pgfqpoint{1.776476in}{0.949564in}}%
\pgfpathlineto{\pgfqpoint{1.785349in}{0.942954in}}%
\pgfpathlineto{\pgfqpoint{1.794221in}{0.936345in}}%
\pgfpathlineto{\pgfqpoint{1.803093in}{0.929735in}}%
\pgfpathlineto{\pgfqpoint{1.811965in}{0.923126in}}%
\pgfpathlineto{\pgfqpoint{1.820837in}{0.916517in}}%
\pgfpathlineto{\pgfqpoint{1.829709in}{0.909907in}}%
\pgfpathlineto{\pgfqpoint{1.838581in}{0.903298in}}%
\pgfpathlineto{\pgfqpoint{1.847453in}{0.896688in}}%
\pgfpathlineto{\pgfqpoint{1.856325in}{0.890079in}}%
\pgfpathlineto{\pgfqpoint{1.865197in}{0.883469in}}%
\pgfpathlineto{\pgfqpoint{1.874069in}{0.876860in}}%
\pgfpathlineto{\pgfqpoint{1.882941in}{0.870251in}}%
\pgfpathlineto{\pgfqpoint{1.891813in}{0.863641in}}%
\pgfpathlineto{\pgfqpoint{1.900685in}{0.857032in}}%
\pgfpathlineto{\pgfqpoint{1.909557in}{0.850422in}}%
\pgfpathlineto{\pgfqpoint{1.918429in}{0.843813in}}%
\pgfpathlineto{\pgfqpoint{1.927301in}{0.837204in}}%
\pgfpathlineto{\pgfqpoint{1.936173in}{0.830594in}}%
\pgfpathlineto{\pgfqpoint{1.945045in}{0.823985in}}%
\pgfpathlineto{\pgfqpoint{1.953918in}{0.817375in}}%
\pgfpathlineto{\pgfqpoint{1.962790in}{0.810766in}}%
\pgfpathlineto{\pgfqpoint{1.971662in}{0.804156in}}%
\pgfpathlineto{\pgfqpoint{1.980534in}{0.797547in}}%
\pgfpathlineto{\pgfqpoint{1.989406in}{0.790938in}}%
\pgfpathlineto{\pgfqpoint{1.998278in}{0.784328in}}%
\pgfpathlineto{\pgfqpoint{2.007150in}{0.777719in}}%
\pgfpathlineto{\pgfqpoint{2.016022in}{0.771109in}}%
\pgfpathlineto{\pgfqpoint{2.024894in}{0.764500in}}%
\pgfpathlineto{\pgfqpoint{2.033766in}{0.757890in}}%
\pgfpathlineto{\pgfqpoint{2.042638in}{0.751281in}}%
\pgfpathlineto{\pgfqpoint{2.051510in}{0.744672in}}%
\pgfpathlineto{\pgfqpoint{2.060382in}{0.738062in}}%
\pgfpathlineto{\pgfqpoint{2.069254in}{0.731453in}}%
\pgfpathlineto{\pgfqpoint{2.078126in}{0.724843in}}%
\pgfpathlineto{\pgfqpoint{2.086998in}{0.718234in}}%
\pgfpathlineto{\pgfqpoint{2.095870in}{0.711624in}}%
\pgfpathlineto{\pgfqpoint{2.104742in}{0.705015in}}%
\pgfpathlineto{\pgfqpoint{2.113615in}{0.698406in}}%
\pgfpathlineto{\pgfqpoint{2.122487in}{0.691796in}}%
\pgfpathlineto{\pgfqpoint{2.131359in}{0.685187in}}%
\pgfusepath{stroke}%
\end{pgfscope}%
\begin{pgfscope}%
\pgfpathrectangle{\pgfqpoint{0.374692in}{0.521603in}}{\pgfqpoint{2.635000in}{1.963000in}} %
\pgfusepath{clip}%
\pgfsetrectcap%
\pgfsetroundjoin%
\pgfsetlinewidth{1.003750pt}%
\definecolor{currentstroke}{rgb}{0.839216,0.152941,0.156863}%
\pgfsetstrokecolor{currentstroke}%
\pgfsetdash{}{0pt}%
\pgfpathmoveto{\pgfqpoint{1.253025in}{0.685187in}}%
\pgfpathlineto{\pgfqpoint{1.261897in}{0.691796in}}%
\pgfpathlineto{\pgfqpoint{1.270769in}{0.698406in}}%
\pgfpathlineto{\pgfqpoint{1.279641in}{0.705015in}}%
\pgfpathlineto{\pgfqpoint{1.288513in}{0.711624in}}%
\pgfpathlineto{\pgfqpoint{1.297386in}{0.718234in}}%
\pgfpathlineto{\pgfqpoint{1.306258in}{0.724843in}}%
\pgfpathlineto{\pgfqpoint{1.315130in}{0.731453in}}%
\pgfpathlineto{\pgfqpoint{1.324002in}{0.738062in}}%
\pgfpathlineto{\pgfqpoint{1.332874in}{0.744672in}}%
\pgfpathlineto{\pgfqpoint{1.341746in}{0.751281in}}%
\pgfpathlineto{\pgfqpoint{1.350618in}{0.757890in}}%
\pgfpathlineto{\pgfqpoint{1.359490in}{0.764500in}}%
\pgfpathlineto{\pgfqpoint{1.368362in}{0.771109in}}%
\pgfpathlineto{\pgfqpoint{1.377234in}{0.777719in}}%
\pgfpathlineto{\pgfqpoint{1.386106in}{0.784328in}}%
\pgfpathlineto{\pgfqpoint{1.394978in}{0.790938in}}%
\pgfpathlineto{\pgfqpoint{1.403850in}{0.797547in}}%
\pgfpathlineto{\pgfqpoint{1.412722in}{0.804156in}}%
\pgfpathlineto{\pgfqpoint{1.421594in}{0.810766in}}%
\pgfpathlineto{\pgfqpoint{1.430466in}{0.817375in}}%
\pgfpathlineto{\pgfqpoint{1.439338in}{0.823985in}}%
\pgfpathlineto{\pgfqpoint{1.448210in}{0.830594in}}%
\pgfpathlineto{\pgfqpoint{1.457083in}{0.837204in}}%
\pgfpathlineto{\pgfqpoint{1.465955in}{0.843813in}}%
\pgfpathlineto{\pgfqpoint{1.474827in}{0.850422in}}%
\pgfpathlineto{\pgfqpoint{1.483699in}{0.857032in}}%
\pgfpathlineto{\pgfqpoint{1.492571in}{0.863641in}}%
\pgfpathlineto{\pgfqpoint{1.501443in}{0.870251in}}%
\pgfpathlineto{\pgfqpoint{1.510315in}{0.876860in}}%
\pgfpathlineto{\pgfqpoint{1.519187in}{0.883469in}}%
\pgfpathlineto{\pgfqpoint{1.528059in}{0.890079in}}%
\pgfpathlineto{\pgfqpoint{1.536931in}{0.896688in}}%
\pgfpathlineto{\pgfqpoint{1.545803in}{0.903298in}}%
\pgfpathlineto{\pgfqpoint{1.554675in}{0.909907in}}%
\pgfpathlineto{\pgfqpoint{1.563547in}{0.916517in}}%
\pgfpathlineto{\pgfqpoint{1.572419in}{0.923126in}}%
\pgfpathlineto{\pgfqpoint{1.581291in}{0.929735in}}%
\pgfpathlineto{\pgfqpoint{1.590163in}{0.936345in}}%
\pgfpathlineto{\pgfqpoint{1.599035in}{0.942954in}}%
\pgfpathlineto{\pgfqpoint{1.607907in}{0.949564in}}%
\pgfpathlineto{\pgfqpoint{1.616779in}{0.956173in}}%
\pgfpathlineto{\pgfqpoint{1.625652in}{0.962783in}}%
\pgfpathlineto{\pgfqpoint{1.634524in}{0.969392in}}%
\pgfpathlineto{\pgfqpoint{1.643396in}{0.976001in}}%
\pgfpathlineto{\pgfqpoint{1.652268in}{0.982611in}}%
\pgfpathlineto{\pgfqpoint{1.661140in}{0.989220in}}%
\pgfpathlineto{\pgfqpoint{1.670012in}{0.995830in}}%
\pgfpathlineto{\pgfqpoint{1.678884in}{1.002439in}}%
\pgfpathlineto{\pgfqpoint{1.687756in}{1.009049in}}%
\pgfpathlineto{\pgfqpoint{1.696628in}{1.015658in}}%
\pgfpathlineto{\pgfqpoint{1.705500in}{1.022267in}}%
\pgfpathlineto{\pgfqpoint{1.714372in}{1.028877in}}%
\pgfpathlineto{\pgfqpoint{1.723244in}{1.035486in}}%
\pgfpathlineto{\pgfqpoint{1.732116in}{1.042096in}}%
\pgfpathlineto{\pgfqpoint{1.740988in}{1.048705in}}%
\pgfpathlineto{\pgfqpoint{1.749860in}{1.055315in}}%
\pgfpathlineto{\pgfqpoint{1.758732in}{1.061924in}}%
\pgfpathlineto{\pgfqpoint{1.767604in}{1.068533in}}%
\pgfpathlineto{\pgfqpoint{1.776476in}{1.075143in}}%
\pgfpathlineto{\pgfqpoint{1.785349in}{1.081752in}}%
\pgfpathlineto{\pgfqpoint{1.794221in}{1.088362in}}%
\pgfpathlineto{\pgfqpoint{1.803093in}{1.094971in}}%
\pgfpathlineto{\pgfqpoint{1.811965in}{1.101581in}}%
\pgfpathlineto{\pgfqpoint{1.820837in}{1.108190in}}%
\pgfpathlineto{\pgfqpoint{1.829709in}{1.114799in}}%
\pgfpathlineto{\pgfqpoint{1.838581in}{1.121409in}}%
\pgfpathlineto{\pgfqpoint{1.847453in}{1.128018in}}%
\pgfpathlineto{\pgfqpoint{1.856325in}{1.134628in}}%
\pgfpathlineto{\pgfqpoint{1.865197in}{1.141237in}}%
\pgfpathlineto{\pgfqpoint{1.874069in}{1.147847in}}%
\pgfpathlineto{\pgfqpoint{1.882941in}{1.154456in}}%
\pgfpathlineto{\pgfqpoint{1.891813in}{1.161065in}}%
\pgfpathlineto{\pgfqpoint{1.900685in}{1.167675in}}%
\pgfpathlineto{\pgfqpoint{1.909557in}{1.174284in}}%
\pgfpathlineto{\pgfqpoint{1.918429in}{1.180894in}}%
\pgfpathlineto{\pgfqpoint{1.927301in}{1.187503in}}%
\pgfpathlineto{\pgfqpoint{1.936173in}{1.194113in}}%
\pgfpathlineto{\pgfqpoint{1.945045in}{1.200722in}}%
\pgfpathlineto{\pgfqpoint{1.953918in}{1.207331in}}%
\pgfpathlineto{\pgfqpoint{1.962790in}{1.213941in}}%
\pgfpathlineto{\pgfqpoint{1.971662in}{1.220550in}}%
\pgfpathlineto{\pgfqpoint{1.980534in}{1.227160in}}%
\pgfpathlineto{\pgfqpoint{1.989406in}{1.233769in}}%
\pgfpathlineto{\pgfqpoint{1.998278in}{1.240379in}}%
\pgfpathlineto{\pgfqpoint{2.007150in}{1.246988in}}%
\pgfpathlineto{\pgfqpoint{2.016022in}{1.253597in}}%
\pgfpathlineto{\pgfqpoint{2.024894in}{1.260207in}}%
\pgfpathlineto{\pgfqpoint{2.033766in}{1.266816in}}%
\pgfpathlineto{\pgfqpoint{2.042638in}{1.273426in}}%
\pgfpathlineto{\pgfqpoint{2.051510in}{1.280035in}}%
\pgfpathlineto{\pgfqpoint{2.060382in}{1.286645in}}%
\pgfpathlineto{\pgfqpoint{2.069254in}{1.293254in}}%
\pgfpathlineto{\pgfqpoint{2.078126in}{1.299863in}}%
\pgfpathlineto{\pgfqpoint{2.086998in}{1.306473in}}%
\pgfpathlineto{\pgfqpoint{2.095870in}{1.313082in}}%
\pgfpathlineto{\pgfqpoint{2.104742in}{1.319692in}}%
\pgfpathlineto{\pgfqpoint{2.113615in}{1.326301in}}%
\pgfpathlineto{\pgfqpoint{2.122487in}{1.332911in}}%
\pgfpathlineto{\pgfqpoint{2.131359in}{1.339520in}}%
\pgfusepath{stroke}%
\end{pgfscope}%
\begin{pgfscope}%
\pgfpathrectangle{\pgfqpoint{0.374692in}{0.521603in}}{\pgfqpoint{2.635000in}{1.963000in}} %
\pgfusepath{clip}%
\pgfsetbuttcap%
\pgfsetroundjoin%
\definecolor{currentfill}{rgb}{1.000000,0.000000,0.000000}%
\pgfsetfillcolor{currentfill}%
\pgfsetlinewidth{1.003750pt}%
\definecolor{currentstroke}{rgb}{1.000000,0.000000,0.000000}%
\pgfsetstrokecolor{currentstroke}%
\pgfsetdash{}{0pt}%
\pgfsys@defobject{currentmarker}{\pgfqpoint{-0.020833in}{-0.020833in}}{\pgfqpoint{0.020833in}{0.020833in}}{%
\pgfpathmoveto{\pgfqpoint{0.000000in}{-0.020833in}}%
\pgfpathcurveto{\pgfqpoint{0.005525in}{-0.020833in}}{\pgfqpoint{0.010825in}{-0.018638in}}{\pgfqpoint{0.014731in}{-0.014731in}}%
\pgfpathcurveto{\pgfqpoint{0.018638in}{-0.010825in}}{\pgfqpoint{0.020833in}{-0.005525in}}{\pgfqpoint{0.020833in}{0.000000in}}%
\pgfpathcurveto{\pgfqpoint{0.020833in}{0.005525in}}{\pgfqpoint{0.018638in}{0.010825in}}{\pgfqpoint{0.014731in}{0.014731in}}%
\pgfpathcurveto{\pgfqpoint{0.010825in}{0.018638in}}{\pgfqpoint{0.005525in}{0.020833in}}{\pgfqpoint{0.000000in}{0.020833in}}%
\pgfpathcurveto{\pgfqpoint{-0.005525in}{0.020833in}}{\pgfqpoint{-0.010825in}{0.018638in}}{\pgfqpoint{-0.014731in}{0.014731in}}%
\pgfpathcurveto{\pgfqpoint{-0.018638in}{0.010825in}}{\pgfqpoint{-0.020833in}{0.005525in}}{\pgfqpoint{-0.020833in}{0.000000in}}%
\pgfpathcurveto{\pgfqpoint{-0.020833in}{-0.005525in}}{\pgfqpoint{-0.018638in}{-0.010825in}}{\pgfqpoint{-0.014731in}{-0.014731in}}%
\pgfpathcurveto{\pgfqpoint{-0.010825in}{-0.018638in}}{\pgfqpoint{-0.005525in}{-0.020833in}}{\pgfqpoint{0.000000in}{-0.020833in}}%
\pgfpathclose%
\pgfusepath{stroke,fill}%
}%
\begin{pgfscope}%
\pgfsys@transformshift{1.253025in}{0.848770in}%
\pgfsys@useobject{currentmarker}{}%
\end{pgfscope}%
\begin{pgfscope}%
\pgfsys@transformshift{1.692192in}{0.848770in}%
\pgfsys@useobject{currentmarker}{}%
\end{pgfscope}%
\begin{pgfscope}%
\pgfsys@transformshift{2.131359in}{0.848770in}%
\pgfsys@useobject{currentmarker}{}%
\end{pgfscope}%
\end{pgfscope}%
\begin{pgfscope}%
\pgfpathrectangle{\pgfqpoint{0.374692in}{0.521603in}}{\pgfqpoint{2.635000in}{1.963000in}} %
\pgfusepath{clip}%
\pgfsetbuttcap%
\pgfsetroundjoin%
\definecolor{currentfill}{rgb}{1.000000,0.000000,0.000000}%
\pgfsetfillcolor{currentfill}%
\pgfsetlinewidth{1.003750pt}%
\definecolor{currentstroke}{rgb}{1.000000,0.000000,0.000000}%
\pgfsetstrokecolor{currentstroke}%
\pgfsetdash{}{0pt}%
\pgfsys@defobject{currentmarker}{\pgfqpoint{-0.020833in}{-0.020833in}}{\pgfqpoint{0.020833in}{0.020833in}}{%
\pgfpathmoveto{\pgfqpoint{0.000000in}{-0.020833in}}%
\pgfpathcurveto{\pgfqpoint{0.005525in}{-0.020833in}}{\pgfqpoint{0.010825in}{-0.018638in}}{\pgfqpoint{0.014731in}{-0.014731in}}%
\pgfpathcurveto{\pgfqpoint{0.018638in}{-0.010825in}}{\pgfqpoint{0.020833in}{-0.005525in}}{\pgfqpoint{0.020833in}{0.000000in}}%
\pgfpathcurveto{\pgfqpoint{0.020833in}{0.005525in}}{\pgfqpoint{0.018638in}{0.010825in}}{\pgfqpoint{0.014731in}{0.014731in}}%
\pgfpathcurveto{\pgfqpoint{0.010825in}{0.018638in}}{\pgfqpoint{0.005525in}{0.020833in}}{\pgfqpoint{0.000000in}{0.020833in}}%
\pgfpathcurveto{\pgfqpoint{-0.005525in}{0.020833in}}{\pgfqpoint{-0.010825in}{0.018638in}}{\pgfqpoint{-0.014731in}{0.014731in}}%
\pgfpathcurveto{\pgfqpoint{-0.018638in}{0.010825in}}{\pgfqpoint{-0.020833in}{0.005525in}}{\pgfqpoint{-0.020833in}{0.000000in}}%
\pgfpathcurveto{\pgfqpoint{-0.020833in}{-0.005525in}}{\pgfqpoint{-0.018638in}{-0.010825in}}{\pgfqpoint{-0.014731in}{-0.014731in}}%
\pgfpathcurveto{\pgfqpoint{-0.010825in}{-0.018638in}}{\pgfqpoint{-0.005525in}{-0.020833in}}{\pgfqpoint{0.000000in}{-0.020833in}}%
\pgfpathclose%
\pgfusepath{stroke,fill}%
}%
\begin{pgfscope}%
\pgfsys@transformshift{1.253025in}{0.848770in}%
\pgfsys@useobject{currentmarker}{}%
\end{pgfscope}%
\begin{pgfscope}%
\pgfsys@transformshift{1.692192in}{0.848770in}%
\pgfsys@useobject{currentmarker}{}%
\end{pgfscope}%
\begin{pgfscope}%
\pgfsys@transformshift{2.131359in}{0.848770in}%
\pgfsys@useobject{currentmarker}{}%
\end{pgfscope}%
\end{pgfscope}%
\begin{pgfscope}%
\pgfpathrectangle{\pgfqpoint{0.374692in}{0.521603in}}{\pgfqpoint{2.635000in}{1.963000in}} %
\pgfusepath{clip}%
\pgfsetrectcap%
\pgfsetroundjoin%
\pgfsetlinewidth{1.003750pt}%
\definecolor{currentstroke}{rgb}{0.580392,0.403922,0.741176}%
\pgfsetstrokecolor{currentstroke}%
\pgfsetdash{}{0pt}%
\pgfpathmoveto{\pgfqpoint{2.131359in}{1.339520in}}%
\pgfpathlineto{\pgfqpoint{2.140231in}{1.332911in}}%
\pgfpathlineto{\pgfqpoint{2.149103in}{1.326301in}}%
\pgfpathlineto{\pgfqpoint{2.157975in}{1.319692in}}%
\pgfpathlineto{\pgfqpoint{2.166847in}{1.313082in}}%
\pgfpathlineto{\pgfqpoint{2.175719in}{1.306473in}}%
\pgfpathlineto{\pgfqpoint{2.184591in}{1.299863in}}%
\pgfpathlineto{\pgfqpoint{2.193463in}{1.293254in}}%
\pgfpathlineto{\pgfqpoint{2.202335in}{1.286645in}}%
\pgfpathlineto{\pgfqpoint{2.211207in}{1.280035in}}%
\pgfpathlineto{\pgfqpoint{2.220079in}{1.273426in}}%
\pgfpathlineto{\pgfqpoint{2.228951in}{1.266816in}}%
\pgfpathlineto{\pgfqpoint{2.237823in}{1.260207in}}%
\pgfpathlineto{\pgfqpoint{2.246695in}{1.253597in}}%
\pgfpathlineto{\pgfqpoint{2.255567in}{1.246988in}}%
\pgfpathlineto{\pgfqpoint{2.264439in}{1.240379in}}%
\pgfpathlineto{\pgfqpoint{2.273311in}{1.233769in}}%
\pgfpathlineto{\pgfqpoint{2.282184in}{1.227160in}}%
\pgfpathlineto{\pgfqpoint{2.291056in}{1.220550in}}%
\pgfpathlineto{\pgfqpoint{2.299928in}{1.213941in}}%
\pgfpathlineto{\pgfqpoint{2.308800in}{1.207331in}}%
\pgfpathlineto{\pgfqpoint{2.317672in}{1.200722in}}%
\pgfpathlineto{\pgfqpoint{2.326544in}{1.194113in}}%
\pgfpathlineto{\pgfqpoint{2.335416in}{1.187503in}}%
\pgfpathlineto{\pgfqpoint{2.344288in}{1.180894in}}%
\pgfpathlineto{\pgfqpoint{2.353160in}{1.174284in}}%
\pgfpathlineto{\pgfqpoint{2.362032in}{1.167675in}}%
\pgfpathlineto{\pgfqpoint{2.370904in}{1.161065in}}%
\pgfpathlineto{\pgfqpoint{2.379776in}{1.154456in}}%
\pgfpathlineto{\pgfqpoint{2.388648in}{1.147847in}}%
\pgfpathlineto{\pgfqpoint{2.397520in}{1.141237in}}%
\pgfpathlineto{\pgfqpoint{2.406392in}{1.134628in}}%
\pgfpathlineto{\pgfqpoint{2.415264in}{1.128018in}}%
\pgfpathlineto{\pgfqpoint{2.424136in}{1.121409in}}%
\pgfpathlineto{\pgfqpoint{2.433008in}{1.114799in}}%
\pgfpathlineto{\pgfqpoint{2.441880in}{1.108190in}}%
\pgfpathlineto{\pgfqpoint{2.450753in}{1.101581in}}%
\pgfpathlineto{\pgfqpoint{2.459625in}{1.094971in}}%
\pgfpathlineto{\pgfqpoint{2.468497in}{1.088362in}}%
\pgfpathlineto{\pgfqpoint{2.477369in}{1.081752in}}%
\pgfpathlineto{\pgfqpoint{2.486241in}{1.075143in}}%
\pgfpathlineto{\pgfqpoint{2.495113in}{1.068533in}}%
\pgfpathlineto{\pgfqpoint{2.503985in}{1.061924in}}%
\pgfpathlineto{\pgfqpoint{2.512857in}{1.055315in}}%
\pgfpathlineto{\pgfqpoint{2.521729in}{1.048705in}}%
\pgfpathlineto{\pgfqpoint{2.530601in}{1.042096in}}%
\pgfpathlineto{\pgfqpoint{2.539473in}{1.035486in}}%
\pgfpathlineto{\pgfqpoint{2.548345in}{1.028877in}}%
\pgfpathlineto{\pgfqpoint{2.557217in}{1.022267in}}%
\pgfpathlineto{\pgfqpoint{2.566089in}{1.015658in}}%
\pgfpathlineto{\pgfqpoint{2.574961in}{1.009049in}}%
\pgfpathlineto{\pgfqpoint{2.583833in}{1.002439in}}%
\pgfpathlineto{\pgfqpoint{2.592705in}{0.995830in}}%
\pgfpathlineto{\pgfqpoint{2.601577in}{0.989220in}}%
\pgfpathlineto{\pgfqpoint{2.610450in}{0.982611in}}%
\pgfpathlineto{\pgfqpoint{2.619322in}{0.976001in}}%
\pgfpathlineto{\pgfqpoint{2.628194in}{0.969392in}}%
\pgfpathlineto{\pgfqpoint{2.637066in}{0.962783in}}%
\pgfpathlineto{\pgfqpoint{2.645938in}{0.956173in}}%
\pgfpathlineto{\pgfqpoint{2.654810in}{0.949564in}}%
\pgfpathlineto{\pgfqpoint{2.663682in}{0.942954in}}%
\pgfpathlineto{\pgfqpoint{2.672554in}{0.936345in}}%
\pgfpathlineto{\pgfqpoint{2.681426in}{0.929735in}}%
\pgfpathlineto{\pgfqpoint{2.690298in}{0.923126in}}%
\pgfpathlineto{\pgfqpoint{2.699170in}{0.916517in}}%
\pgfpathlineto{\pgfqpoint{2.708042in}{0.909907in}}%
\pgfpathlineto{\pgfqpoint{2.716914in}{0.903298in}}%
\pgfpathlineto{\pgfqpoint{2.725786in}{0.896688in}}%
\pgfpathlineto{\pgfqpoint{2.734658in}{0.890079in}}%
\pgfpathlineto{\pgfqpoint{2.743530in}{0.883469in}}%
\pgfpathlineto{\pgfqpoint{2.752402in}{0.876860in}}%
\pgfpathlineto{\pgfqpoint{2.761274in}{0.870251in}}%
\pgfpathlineto{\pgfqpoint{2.770146in}{0.863641in}}%
\pgfpathlineto{\pgfqpoint{2.779019in}{0.857032in}}%
\pgfpathlineto{\pgfqpoint{2.787891in}{0.850422in}}%
\pgfpathlineto{\pgfqpoint{2.796763in}{0.843813in}}%
\pgfpathlineto{\pgfqpoint{2.805635in}{0.837204in}}%
\pgfpathlineto{\pgfqpoint{2.814507in}{0.830594in}}%
\pgfpathlineto{\pgfqpoint{2.823379in}{0.823985in}}%
\pgfpathlineto{\pgfqpoint{2.832251in}{0.817375in}}%
\pgfpathlineto{\pgfqpoint{2.841123in}{0.810766in}}%
\pgfpathlineto{\pgfqpoint{2.849995in}{0.804156in}}%
\pgfpathlineto{\pgfqpoint{2.858867in}{0.797547in}}%
\pgfpathlineto{\pgfqpoint{2.867739in}{0.790938in}}%
\pgfpathlineto{\pgfqpoint{2.876611in}{0.784328in}}%
\pgfpathlineto{\pgfqpoint{2.885483in}{0.777719in}}%
\pgfpathlineto{\pgfqpoint{2.894355in}{0.771109in}}%
\pgfpathlineto{\pgfqpoint{2.903227in}{0.764500in}}%
\pgfpathlineto{\pgfqpoint{2.912099in}{0.757890in}}%
\pgfpathlineto{\pgfqpoint{2.920971in}{0.751281in}}%
\pgfpathlineto{\pgfqpoint{2.929843in}{0.744672in}}%
\pgfpathlineto{\pgfqpoint{2.938716in}{0.738062in}}%
\pgfpathlineto{\pgfqpoint{2.947588in}{0.731453in}}%
\pgfpathlineto{\pgfqpoint{2.956460in}{0.724843in}}%
\pgfpathlineto{\pgfqpoint{2.965332in}{0.718234in}}%
\pgfpathlineto{\pgfqpoint{2.974204in}{0.711624in}}%
\pgfpathlineto{\pgfqpoint{2.983076in}{0.705015in}}%
\pgfpathlineto{\pgfqpoint{2.991948in}{0.698406in}}%
\pgfpathlineto{\pgfqpoint{3.000820in}{0.691796in}}%
\pgfpathlineto{\pgfqpoint{3.009692in}{0.685187in}}%
\pgfusepath{stroke}%
\end{pgfscope}%
\begin{pgfscope}%
\pgfpathrectangle{\pgfqpoint{0.374692in}{0.521603in}}{\pgfqpoint{2.635000in}{1.963000in}} %
\pgfusepath{clip}%
\pgfsetrectcap%
\pgfsetroundjoin%
\pgfsetlinewidth{1.003750pt}%
\definecolor{currentstroke}{rgb}{0.549020,0.337255,0.294118}%
\pgfsetstrokecolor{currentstroke}%
\pgfsetdash{}{0pt}%
\pgfpathmoveto{\pgfqpoint{2.131359in}{0.685187in}}%
\pgfpathlineto{\pgfqpoint{2.140231in}{0.691796in}}%
\pgfpathlineto{\pgfqpoint{2.149103in}{0.698406in}}%
\pgfpathlineto{\pgfqpoint{2.157975in}{0.705015in}}%
\pgfpathlineto{\pgfqpoint{2.166847in}{0.711624in}}%
\pgfpathlineto{\pgfqpoint{2.175719in}{0.718234in}}%
\pgfpathlineto{\pgfqpoint{2.184591in}{0.724843in}}%
\pgfpathlineto{\pgfqpoint{2.193463in}{0.731453in}}%
\pgfpathlineto{\pgfqpoint{2.202335in}{0.738062in}}%
\pgfpathlineto{\pgfqpoint{2.211207in}{0.744672in}}%
\pgfpathlineto{\pgfqpoint{2.220079in}{0.751281in}}%
\pgfpathlineto{\pgfqpoint{2.228951in}{0.757890in}}%
\pgfpathlineto{\pgfqpoint{2.237823in}{0.764500in}}%
\pgfpathlineto{\pgfqpoint{2.246695in}{0.771109in}}%
\pgfpathlineto{\pgfqpoint{2.255567in}{0.777719in}}%
\pgfpathlineto{\pgfqpoint{2.264439in}{0.784328in}}%
\pgfpathlineto{\pgfqpoint{2.273311in}{0.790938in}}%
\pgfpathlineto{\pgfqpoint{2.282184in}{0.797547in}}%
\pgfpathlineto{\pgfqpoint{2.291056in}{0.804156in}}%
\pgfpathlineto{\pgfqpoint{2.299928in}{0.810766in}}%
\pgfpathlineto{\pgfqpoint{2.308800in}{0.817375in}}%
\pgfpathlineto{\pgfqpoint{2.317672in}{0.823985in}}%
\pgfpathlineto{\pgfqpoint{2.326544in}{0.830594in}}%
\pgfpathlineto{\pgfqpoint{2.335416in}{0.837204in}}%
\pgfpathlineto{\pgfqpoint{2.344288in}{0.843813in}}%
\pgfpathlineto{\pgfqpoint{2.353160in}{0.850422in}}%
\pgfpathlineto{\pgfqpoint{2.362032in}{0.857032in}}%
\pgfpathlineto{\pgfqpoint{2.370904in}{0.863641in}}%
\pgfpathlineto{\pgfqpoint{2.379776in}{0.870251in}}%
\pgfpathlineto{\pgfqpoint{2.388648in}{0.876860in}}%
\pgfpathlineto{\pgfqpoint{2.397520in}{0.883469in}}%
\pgfpathlineto{\pgfqpoint{2.406392in}{0.890079in}}%
\pgfpathlineto{\pgfqpoint{2.415264in}{0.896688in}}%
\pgfpathlineto{\pgfqpoint{2.424136in}{0.903298in}}%
\pgfpathlineto{\pgfqpoint{2.433008in}{0.909907in}}%
\pgfpathlineto{\pgfqpoint{2.441880in}{0.916517in}}%
\pgfpathlineto{\pgfqpoint{2.450753in}{0.923126in}}%
\pgfpathlineto{\pgfqpoint{2.459625in}{0.929735in}}%
\pgfpathlineto{\pgfqpoint{2.468497in}{0.936345in}}%
\pgfpathlineto{\pgfqpoint{2.477369in}{0.942954in}}%
\pgfpathlineto{\pgfqpoint{2.486241in}{0.949564in}}%
\pgfpathlineto{\pgfqpoint{2.495113in}{0.956173in}}%
\pgfpathlineto{\pgfqpoint{2.503985in}{0.962783in}}%
\pgfpathlineto{\pgfqpoint{2.512857in}{0.969392in}}%
\pgfpathlineto{\pgfqpoint{2.521729in}{0.976001in}}%
\pgfpathlineto{\pgfqpoint{2.530601in}{0.982611in}}%
\pgfpathlineto{\pgfqpoint{2.539473in}{0.989220in}}%
\pgfpathlineto{\pgfqpoint{2.548345in}{0.995830in}}%
\pgfpathlineto{\pgfqpoint{2.557217in}{1.002439in}}%
\pgfpathlineto{\pgfqpoint{2.566089in}{1.009049in}}%
\pgfpathlineto{\pgfqpoint{2.574961in}{1.015658in}}%
\pgfpathlineto{\pgfqpoint{2.583833in}{1.022267in}}%
\pgfpathlineto{\pgfqpoint{2.592705in}{1.028877in}}%
\pgfpathlineto{\pgfqpoint{2.601577in}{1.035486in}}%
\pgfpathlineto{\pgfqpoint{2.610450in}{1.042096in}}%
\pgfpathlineto{\pgfqpoint{2.619322in}{1.048705in}}%
\pgfpathlineto{\pgfqpoint{2.628194in}{1.055315in}}%
\pgfpathlineto{\pgfqpoint{2.637066in}{1.061924in}}%
\pgfpathlineto{\pgfqpoint{2.645938in}{1.068533in}}%
\pgfpathlineto{\pgfqpoint{2.654810in}{1.075143in}}%
\pgfpathlineto{\pgfqpoint{2.663682in}{1.081752in}}%
\pgfpathlineto{\pgfqpoint{2.672554in}{1.088362in}}%
\pgfpathlineto{\pgfqpoint{2.681426in}{1.094971in}}%
\pgfpathlineto{\pgfqpoint{2.690298in}{1.101581in}}%
\pgfpathlineto{\pgfqpoint{2.699170in}{1.108190in}}%
\pgfpathlineto{\pgfqpoint{2.708042in}{1.114799in}}%
\pgfpathlineto{\pgfqpoint{2.716914in}{1.121409in}}%
\pgfpathlineto{\pgfqpoint{2.725786in}{1.128018in}}%
\pgfpathlineto{\pgfqpoint{2.734658in}{1.134628in}}%
\pgfpathlineto{\pgfqpoint{2.743530in}{1.141237in}}%
\pgfpathlineto{\pgfqpoint{2.752402in}{1.147847in}}%
\pgfpathlineto{\pgfqpoint{2.761274in}{1.154456in}}%
\pgfpathlineto{\pgfqpoint{2.770146in}{1.161065in}}%
\pgfpathlineto{\pgfqpoint{2.779019in}{1.167675in}}%
\pgfpathlineto{\pgfqpoint{2.787891in}{1.174284in}}%
\pgfpathlineto{\pgfqpoint{2.796763in}{1.180894in}}%
\pgfpathlineto{\pgfqpoint{2.805635in}{1.187503in}}%
\pgfpathlineto{\pgfqpoint{2.814507in}{1.194113in}}%
\pgfpathlineto{\pgfqpoint{2.823379in}{1.200722in}}%
\pgfpathlineto{\pgfqpoint{2.832251in}{1.207331in}}%
\pgfpathlineto{\pgfqpoint{2.841123in}{1.213941in}}%
\pgfpathlineto{\pgfqpoint{2.849995in}{1.220550in}}%
\pgfpathlineto{\pgfqpoint{2.858867in}{1.227160in}}%
\pgfpathlineto{\pgfqpoint{2.867739in}{1.233769in}}%
\pgfpathlineto{\pgfqpoint{2.876611in}{1.240379in}}%
\pgfpathlineto{\pgfqpoint{2.885483in}{1.246988in}}%
\pgfpathlineto{\pgfqpoint{2.894355in}{1.253597in}}%
\pgfpathlineto{\pgfqpoint{2.903227in}{1.260207in}}%
\pgfpathlineto{\pgfqpoint{2.912099in}{1.266816in}}%
\pgfpathlineto{\pgfqpoint{2.920971in}{1.273426in}}%
\pgfpathlineto{\pgfqpoint{2.929843in}{1.280035in}}%
\pgfpathlineto{\pgfqpoint{2.938716in}{1.286645in}}%
\pgfpathlineto{\pgfqpoint{2.947588in}{1.293254in}}%
\pgfpathlineto{\pgfqpoint{2.956460in}{1.299863in}}%
\pgfpathlineto{\pgfqpoint{2.965332in}{1.306473in}}%
\pgfpathlineto{\pgfqpoint{2.974204in}{1.313082in}}%
\pgfpathlineto{\pgfqpoint{2.983076in}{1.319692in}}%
\pgfpathlineto{\pgfqpoint{2.991948in}{1.326301in}}%
\pgfpathlineto{\pgfqpoint{3.000820in}{1.332911in}}%
\pgfpathlineto{\pgfqpoint{3.009692in}{1.339520in}}%
\pgfusepath{stroke}%
\end{pgfscope}%
\begin{pgfscope}%
\pgfpathrectangle{\pgfqpoint{0.374692in}{0.521603in}}{\pgfqpoint{2.635000in}{1.963000in}} %
\pgfusepath{clip}%
\pgfsetbuttcap%
\pgfsetroundjoin%
\definecolor{currentfill}{rgb}{0.000000,0.000000,0.000000}%
\pgfsetfillcolor{currentfill}%
\pgfsetlinewidth{1.003750pt}%
\definecolor{currentstroke}{rgb}{0.000000,0.000000,0.000000}%
\pgfsetstrokecolor{currentstroke}%
\pgfsetdash{}{0pt}%
\pgfsys@defobject{currentmarker}{\pgfqpoint{-0.020833in}{-0.020833in}}{\pgfqpoint{0.020833in}{0.020833in}}{%
\pgfpathmoveto{\pgfqpoint{0.000000in}{-0.020833in}}%
\pgfpathcurveto{\pgfqpoint{0.005525in}{-0.020833in}}{\pgfqpoint{0.010825in}{-0.018638in}}{\pgfqpoint{0.014731in}{-0.014731in}}%
\pgfpathcurveto{\pgfqpoint{0.018638in}{-0.010825in}}{\pgfqpoint{0.020833in}{-0.005525in}}{\pgfqpoint{0.020833in}{0.000000in}}%
\pgfpathcurveto{\pgfqpoint{0.020833in}{0.005525in}}{\pgfqpoint{0.018638in}{0.010825in}}{\pgfqpoint{0.014731in}{0.014731in}}%
\pgfpathcurveto{\pgfqpoint{0.010825in}{0.018638in}}{\pgfqpoint{0.005525in}{0.020833in}}{\pgfqpoint{0.000000in}{0.020833in}}%
\pgfpathcurveto{\pgfqpoint{-0.005525in}{0.020833in}}{\pgfqpoint{-0.010825in}{0.018638in}}{\pgfqpoint{-0.014731in}{0.014731in}}%
\pgfpathcurveto{\pgfqpoint{-0.018638in}{0.010825in}}{\pgfqpoint{-0.020833in}{0.005525in}}{\pgfqpoint{-0.020833in}{0.000000in}}%
\pgfpathcurveto{\pgfqpoint{-0.020833in}{-0.005525in}}{\pgfqpoint{-0.018638in}{-0.010825in}}{\pgfqpoint{-0.014731in}{-0.014731in}}%
\pgfpathcurveto{\pgfqpoint{-0.010825in}{-0.018638in}}{\pgfqpoint{-0.005525in}{-0.020833in}}{\pgfqpoint{0.000000in}{-0.020833in}}%
\pgfpathclose%
\pgfusepath{stroke,fill}%
}%
\begin{pgfscope}%
\pgfsys@transformshift{2.131359in}{0.848770in}%
\pgfsys@useobject{currentmarker}{}%
\end{pgfscope}%
\begin{pgfscope}%
\pgfsys@transformshift{2.570525in}{0.848770in}%
\pgfsys@useobject{currentmarker}{}%
\end{pgfscope}%
\end{pgfscope}%
\begin{pgfscope}%
\pgfpathrectangle{\pgfqpoint{0.374692in}{0.521603in}}{\pgfqpoint{2.635000in}{1.963000in}} %
\pgfusepath{clip}%
\pgfsetbuttcap%
\pgfsetroundjoin%
\definecolor{currentfill}{rgb}{1.000000,0.000000,0.000000}%
\pgfsetfillcolor{currentfill}%
\pgfsetlinewidth{1.003750pt}%
\definecolor{currentstroke}{rgb}{1.000000,0.000000,0.000000}%
\pgfsetstrokecolor{currentstroke}%
\pgfsetdash{}{0pt}%
\pgfsys@defobject{currentmarker}{\pgfqpoint{-0.020833in}{-0.020833in}}{\pgfqpoint{0.020833in}{0.020833in}}{%
\pgfpathmoveto{\pgfqpoint{0.000000in}{-0.020833in}}%
\pgfpathcurveto{\pgfqpoint{0.005525in}{-0.020833in}}{\pgfqpoint{0.010825in}{-0.018638in}}{\pgfqpoint{0.014731in}{-0.014731in}}%
\pgfpathcurveto{\pgfqpoint{0.018638in}{-0.010825in}}{\pgfqpoint{0.020833in}{-0.005525in}}{\pgfqpoint{0.020833in}{0.000000in}}%
\pgfpathcurveto{\pgfqpoint{0.020833in}{0.005525in}}{\pgfqpoint{0.018638in}{0.010825in}}{\pgfqpoint{0.014731in}{0.014731in}}%
\pgfpathcurveto{\pgfqpoint{0.010825in}{0.018638in}}{\pgfqpoint{0.005525in}{0.020833in}}{\pgfqpoint{0.000000in}{0.020833in}}%
\pgfpathcurveto{\pgfqpoint{-0.005525in}{0.020833in}}{\pgfqpoint{-0.010825in}{0.018638in}}{\pgfqpoint{-0.014731in}{0.014731in}}%
\pgfpathcurveto{\pgfqpoint{-0.018638in}{0.010825in}}{\pgfqpoint{-0.020833in}{0.005525in}}{\pgfqpoint{-0.020833in}{0.000000in}}%
\pgfpathcurveto{\pgfqpoint{-0.020833in}{-0.005525in}}{\pgfqpoint{-0.018638in}{-0.010825in}}{\pgfqpoint{-0.014731in}{-0.014731in}}%
\pgfpathcurveto{\pgfqpoint{-0.010825in}{-0.018638in}}{\pgfqpoint{-0.005525in}{-0.020833in}}{\pgfqpoint{0.000000in}{-0.020833in}}%
\pgfpathclose%
\pgfusepath{stroke,fill}%
}%
\begin{pgfscope}%
\pgfsys@transformshift{2.131359in}{0.848770in}%
\pgfsys@useobject{currentmarker}{}%
\end{pgfscope}%
\begin{pgfscope}%
\pgfsys@transformshift{2.570525in}{0.848770in}%
\pgfsys@useobject{currentmarker}{}%
\end{pgfscope}%
\end{pgfscope}%
\begin{pgfscope}%
\pgfsetrectcap%
\pgfsetmiterjoin%
\pgfsetlinewidth{0.803000pt}%
\definecolor{currentstroke}{rgb}{0.000000,0.000000,0.000000}%
\pgfsetstrokecolor{currentstroke}%
\pgfsetdash{}{0pt}%
\pgfpathmoveto{\pgfqpoint{0.374692in}{0.521603in}}%
\pgfpathlineto{\pgfqpoint{0.374692in}{2.484603in}}%
\pgfusepath{stroke}%
\end{pgfscope}%
\begin{pgfscope}%
\pgfsetrectcap%
\pgfsetmiterjoin%
\pgfsetlinewidth{0.803000pt}%
\definecolor{currentstroke}{rgb}{0.000000,0.000000,0.000000}%
\pgfsetstrokecolor{currentstroke}%
\pgfsetdash{}{0pt}%
\pgfpathmoveto{\pgfqpoint{3.009692in}{0.521603in}}%
\pgfpathlineto{\pgfqpoint{3.009692in}{2.484603in}}%
\pgfusepath{stroke}%
\end{pgfscope}%
\begin{pgfscope}%
\pgfsetrectcap%
\pgfsetmiterjoin%
\pgfsetlinewidth{0.803000pt}%
\definecolor{currentstroke}{rgb}{0.000000,0.000000,0.000000}%
\pgfsetstrokecolor{currentstroke}%
\pgfsetdash{}{0pt}%
\pgfpathmoveto{\pgfqpoint{0.374692in}{0.521603in}}%
\pgfpathlineto{\pgfqpoint{3.009692in}{0.521603in}}%
\pgfusepath{stroke}%
\end{pgfscope}%
\begin{pgfscope}%
\pgfsetrectcap%
\pgfsetmiterjoin%
\pgfsetlinewidth{0.803000pt}%
\definecolor{currentstroke}{rgb}{0.000000,0.000000,0.000000}%
\pgfsetstrokecolor{currentstroke}%
\pgfsetdash{}{0pt}%
\pgfpathmoveto{\pgfqpoint{0.374692in}{2.484603in}}%
\pgfpathlineto{\pgfqpoint{3.009692in}{2.484603in}}%
\pgfusepath{stroke}%
\end{pgfscope}%
\begin{pgfscope}%
\pgfsetbuttcap%
\pgfsetmiterjoin%
\definecolor{currentfill}{rgb}{1.000000,1.000000,1.000000}%
\pgfsetfillcolor{currentfill}%
\pgfsetfillopacity{0.800000}%
\pgfsetlinewidth{1.003750pt}%
\definecolor{currentstroke}{rgb}{0.800000,0.800000,0.800000}%
\pgfsetstrokecolor{currentstroke}%
\pgfsetstrokeopacity{0.800000}%
\pgfsetdash{}{0pt}%
\pgfpathmoveto{\pgfqpoint{1.450976in}{1.759954in}}%
\pgfpathlineto{\pgfqpoint{2.912470in}{1.759954in}}%
\pgfpathquadraticcurveto{\pgfqpoint{2.940247in}{1.759954in}}{\pgfqpoint{2.940247in}{1.787732in}}%
\pgfpathlineto{\pgfqpoint{2.940247in}{2.387381in}}%
\pgfpathquadraticcurveto{\pgfqpoint{2.940247in}{2.415159in}}{\pgfqpoint{2.912470in}{2.415159in}}%
\pgfpathlineto{\pgfqpoint{1.450976in}{2.415159in}}%
\pgfpathquadraticcurveto{\pgfqpoint{1.423198in}{2.415159in}}{\pgfqpoint{1.423198in}{2.387381in}}%
\pgfpathlineto{\pgfqpoint{1.423198in}{1.787732in}}%
\pgfpathquadraticcurveto{\pgfqpoint{1.423198in}{1.759954in}}{\pgfqpoint{1.450976in}{1.759954in}}%
\pgfpathclose%
\pgfusepath{stroke,fill}%
\end{pgfscope}%
\begin{pgfscope}%
\pgfsetbuttcap%
\pgfsetroundjoin%
\pgfsetlinewidth{1.505625pt}%
\definecolor{currentstroke}{rgb}{0.000000,0.000000,0.000000}%
\pgfsetstrokecolor{currentstroke}%
\pgfsetdash{{5.550000pt}{2.400000pt}}{0.000000pt}%
\pgfpathmoveto{\pgfqpoint{1.478754in}{2.302691in}}%
\pgfpathlineto{\pgfqpoint{1.756532in}{2.302691in}}%
\pgfusepath{stroke}%
\end{pgfscope}%
\begin{pgfscope}%
\pgftext[x=1.867643in,y=2.254080in,left,base]{\rmfamily\fontsize{10.000000}{12.000000}\selectfont el. boundaries}%
\end{pgfscope}%
\begin{pgfscope}%
\pgfsetbuttcap%
\pgfsetroundjoin%
\definecolor{currentfill}{rgb}{0.000000,0.000000,0.000000}%
\pgfsetfillcolor{currentfill}%
\pgfsetlinewidth{1.003750pt}%
\definecolor{currentstroke}{rgb}{0.000000,0.000000,0.000000}%
\pgfsetstrokecolor{currentstroke}%
\pgfsetdash{}{0pt}%
\pgfsys@defobject{currentmarker}{\pgfqpoint{-0.020833in}{-0.020833in}}{\pgfqpoint{0.020833in}{0.020833in}}{%
\pgfpathmoveto{\pgfqpoint{0.000000in}{-0.020833in}}%
\pgfpathcurveto{\pgfqpoint{0.005525in}{-0.020833in}}{\pgfqpoint{0.010825in}{-0.018638in}}{\pgfqpoint{0.014731in}{-0.014731in}}%
\pgfpathcurveto{\pgfqpoint{0.018638in}{-0.010825in}}{\pgfqpoint{0.020833in}{-0.005525in}}{\pgfqpoint{0.020833in}{0.000000in}}%
\pgfpathcurveto{\pgfqpoint{0.020833in}{0.005525in}}{\pgfqpoint{0.018638in}{0.010825in}}{\pgfqpoint{0.014731in}{0.014731in}}%
\pgfpathcurveto{\pgfqpoint{0.010825in}{0.018638in}}{\pgfqpoint{0.005525in}{0.020833in}}{\pgfqpoint{0.000000in}{0.020833in}}%
\pgfpathcurveto{\pgfqpoint{-0.005525in}{0.020833in}}{\pgfqpoint{-0.010825in}{0.018638in}}{\pgfqpoint{-0.014731in}{0.014731in}}%
\pgfpathcurveto{\pgfqpoint{-0.018638in}{0.010825in}}{\pgfqpoint{-0.020833in}{0.005525in}}{\pgfqpoint{-0.020833in}{0.000000in}}%
\pgfpathcurveto{\pgfqpoint{-0.020833in}{-0.005525in}}{\pgfqpoint{-0.018638in}{-0.010825in}}{\pgfqpoint{-0.014731in}{-0.014731in}}%
\pgfpathcurveto{\pgfqpoint{-0.010825in}{-0.018638in}}{\pgfqpoint{-0.005525in}{-0.020833in}}{\pgfqpoint{0.000000in}{-0.020833in}}%
\pgfpathclose%
\pgfusepath{stroke,fill}%
}%
\begin{pgfscope}%
\pgfsys@transformshift{1.617643in}{2.098834in}%
\pgfsys@useobject{currentmarker}{}%
\end{pgfscope}%
\end{pgfscope}%
\begin{pgfscope}%
\pgftext[x=1.867643in,y=2.050223in,left,base]{\rmfamily\fontsize{10.000000}{12.000000}\selectfont local knots \(\displaystyle s_m\)}%
\end{pgfscope}%
\begin{pgfscope}%
\pgfsetbuttcap%
\pgfsetroundjoin%
\definecolor{currentfill}{rgb}{1.000000,0.000000,0.000000}%
\pgfsetfillcolor{currentfill}%
\pgfsetlinewidth{1.003750pt}%
\definecolor{currentstroke}{rgb}{1.000000,0.000000,0.000000}%
\pgfsetstrokecolor{currentstroke}%
\pgfsetdash{}{0pt}%
\pgfsys@defobject{currentmarker}{\pgfqpoint{-0.020833in}{-0.020833in}}{\pgfqpoint{0.020833in}{0.020833in}}{%
\pgfpathmoveto{\pgfqpoint{0.000000in}{-0.020833in}}%
\pgfpathcurveto{\pgfqpoint{0.005525in}{-0.020833in}}{\pgfqpoint{0.010825in}{-0.018638in}}{\pgfqpoint{0.014731in}{-0.014731in}}%
\pgfpathcurveto{\pgfqpoint{0.018638in}{-0.010825in}}{\pgfqpoint{0.020833in}{-0.005525in}}{\pgfqpoint{0.020833in}{0.000000in}}%
\pgfpathcurveto{\pgfqpoint{0.020833in}{0.005525in}}{\pgfqpoint{0.018638in}{0.010825in}}{\pgfqpoint{0.014731in}{0.014731in}}%
\pgfpathcurveto{\pgfqpoint{0.010825in}{0.018638in}}{\pgfqpoint{0.005525in}{0.020833in}}{\pgfqpoint{0.000000in}{0.020833in}}%
\pgfpathcurveto{\pgfqpoint{-0.005525in}{0.020833in}}{\pgfqpoint{-0.010825in}{0.018638in}}{\pgfqpoint{-0.014731in}{0.014731in}}%
\pgfpathcurveto{\pgfqpoint{-0.018638in}{0.010825in}}{\pgfqpoint{-0.020833in}{0.005525in}}{\pgfqpoint{-0.020833in}{0.000000in}}%
\pgfpathcurveto{\pgfqpoint{-0.020833in}{-0.005525in}}{\pgfqpoint{-0.018638in}{-0.010825in}}{\pgfqpoint{-0.014731in}{-0.014731in}}%
\pgfpathcurveto{\pgfqpoint{-0.010825in}{-0.018638in}}{\pgfqpoint{-0.005525in}{-0.020833in}}{\pgfqpoint{0.000000in}{-0.020833in}}%
\pgfpathclose%
\pgfusepath{stroke,fill}%
}%
\begin{pgfscope}%
\pgfsys@transformshift{1.617643in}{1.894977in}%
\pgfsys@useobject{currentmarker}{}%
\end{pgfscope}%
\end{pgfscope}%
\begin{pgfscope}%
\pgftext[x=1.867643in,y=1.846366in,left,base]{\rmfamily\fontsize{10.000000}{12.000000}\selectfont global knots \(\displaystyle z_i\)}%
\end{pgfscope}%
\begin{pgfscope}%
\pgfsetbuttcap%
\pgfsetmiterjoin%
\definecolor{currentfill}{rgb}{1.000000,1.000000,1.000000}%
\pgfsetfillcolor{currentfill}%
\pgfsetlinewidth{0.000000pt}%
\definecolor{currentstroke}{rgb}{0.000000,0.000000,0.000000}%
\pgfsetstrokecolor{currentstroke}%
\pgfsetstrokeopacity{0.000000}%
\pgfsetdash{}{0pt}%
\pgfpathmoveto{\pgfqpoint{0.629692in}{1.756603in}}%
\pgfpathlineto{\pgfqpoint{1.309692in}{1.756603in}}%
\pgfpathlineto{\pgfqpoint{1.309692in}{2.224603in}}%
\pgfpathlineto{\pgfqpoint{0.629692in}{2.224603in}}%
\pgfpathclose%
\pgfusepath{fill}%
\end{pgfscope}%
\begin{pgfscope}%
\pgfsetbuttcap%
\pgfsetroundjoin%
\definecolor{currentfill}{rgb}{0.000000,0.000000,0.000000}%
\pgfsetfillcolor{currentfill}%
\pgfsetlinewidth{0.803000pt}%
\definecolor{currentstroke}{rgb}{0.000000,0.000000,0.000000}%
\pgfsetstrokecolor{currentstroke}%
\pgfsetdash{}{0pt}%
\pgfsys@defobject{currentmarker}{\pgfqpoint{0.000000in}{-0.048611in}}{\pgfqpoint{0.000000in}{0.000000in}}{%
\pgfpathmoveto{\pgfqpoint{0.000000in}{0.000000in}}%
\pgfpathlineto{\pgfqpoint{0.000000in}{-0.048611in}}%
\pgfusepath{stroke,fill}%
}%
\begin{pgfscope}%
\pgfsys@transformshift{0.660601in}{1.756603in}%
\pgfsys@useobject{currentmarker}{}%
\end{pgfscope}%
\end{pgfscope}%
\begin{pgfscope}%
\pgftext[x=0.660601in,y=1.659381in,,top]{\rmfamily\fontsize{10.000000}{12.000000}\selectfont \(\displaystyle -1\)}%
\end{pgfscope}%
\begin{pgfscope}%
\pgfsetbuttcap%
\pgfsetroundjoin%
\definecolor{currentfill}{rgb}{0.000000,0.000000,0.000000}%
\pgfsetfillcolor{currentfill}%
\pgfsetlinewidth{0.803000pt}%
\definecolor{currentstroke}{rgb}{0.000000,0.000000,0.000000}%
\pgfsetstrokecolor{currentstroke}%
\pgfsetdash{}{0pt}%
\pgfsys@defobject{currentmarker}{\pgfqpoint{0.000000in}{-0.048611in}}{\pgfqpoint{0.000000in}{0.000000in}}{%
\pgfpathmoveto{\pgfqpoint{0.000000in}{0.000000in}}%
\pgfpathlineto{\pgfqpoint{0.000000in}{-0.048611in}}%
\pgfusepath{stroke,fill}%
}%
\begin{pgfscope}%
\pgfsys@transformshift{0.969692in}{1.756603in}%
\pgfsys@useobject{currentmarker}{}%
\end{pgfscope}%
\end{pgfscope}%
\begin{pgfscope}%
\pgftext[x=0.969692in,y=1.659381in,,top]{\rmfamily\fontsize{10.000000}{12.000000}\selectfont \(\displaystyle 0\)}%
\end{pgfscope}%
\begin{pgfscope}%
\pgfsetbuttcap%
\pgfsetroundjoin%
\definecolor{currentfill}{rgb}{0.000000,0.000000,0.000000}%
\pgfsetfillcolor{currentfill}%
\pgfsetlinewidth{0.803000pt}%
\definecolor{currentstroke}{rgb}{0.000000,0.000000,0.000000}%
\pgfsetstrokecolor{currentstroke}%
\pgfsetdash{}{0pt}%
\pgfsys@defobject{currentmarker}{\pgfqpoint{0.000000in}{-0.048611in}}{\pgfqpoint{0.000000in}{0.000000in}}{%
\pgfpathmoveto{\pgfqpoint{0.000000in}{0.000000in}}%
\pgfpathlineto{\pgfqpoint{0.000000in}{-0.048611in}}%
\pgfusepath{stroke,fill}%
}%
\begin{pgfscope}%
\pgfsys@transformshift{1.278783in}{1.756603in}%
\pgfsys@useobject{currentmarker}{}%
\end{pgfscope}%
\end{pgfscope}%
\begin{pgfscope}%
\pgftext[x=1.278783in,y=1.659381in,,top]{\rmfamily\fontsize{10.000000}{12.000000}\selectfont \(\displaystyle 1\)}%
\end{pgfscope}%
\begin{pgfscope}%
\pgftext[x=0.969692in,y=1.469413in,,top]{\rmfamily\fontsize{10.000000}{12.000000}\selectfont s}%
\end{pgfscope}%
\begin{pgfscope}%
\pgfsetbuttcap%
\pgfsetroundjoin%
\definecolor{currentfill}{rgb}{0.000000,0.000000,0.000000}%
\pgfsetfillcolor{currentfill}%
\pgfsetlinewidth{0.803000pt}%
\definecolor{currentstroke}{rgb}{0.000000,0.000000,0.000000}%
\pgfsetstrokecolor{currentstroke}%
\pgfsetdash{}{0pt}%
\pgfsys@defobject{currentmarker}{\pgfqpoint{-0.048611in}{0.000000in}}{\pgfqpoint{0.000000in}{0.000000in}}{%
\pgfpathmoveto{\pgfqpoint{0.000000in}{0.000000in}}%
\pgfpathlineto{\pgfqpoint{-0.048611in}{0.000000in}}%
\pgfusepath{stroke,fill}%
}%
\begin{pgfscope}%
\pgfsys@transformshift{0.629692in}{1.884240in}%
\pgfsys@useobject{currentmarker}{}%
\end{pgfscope}%
\end{pgfscope}%
\begin{pgfscope}%
\pgftext[x=0.463025in,y=1.831478in,left,base]{\rmfamily\fontsize{10.000000}{12.000000}\selectfont \(\displaystyle 0\)}%
\end{pgfscope}%
\begin{pgfscope}%
\pgfsetbuttcap%
\pgfsetroundjoin%
\definecolor{currentfill}{rgb}{0.000000,0.000000,0.000000}%
\pgfsetfillcolor{currentfill}%
\pgfsetlinewidth{0.803000pt}%
\definecolor{currentstroke}{rgb}{0.000000,0.000000,0.000000}%
\pgfsetstrokecolor{currentstroke}%
\pgfsetdash{}{0pt}%
\pgfsys@defobject{currentmarker}{\pgfqpoint{-0.048611in}{0.000000in}}{\pgfqpoint{0.000000in}{0.000000in}}{%
\pgfpathmoveto{\pgfqpoint{0.000000in}{0.000000in}}%
\pgfpathlineto{\pgfqpoint{-0.048611in}{0.000000in}}%
\pgfusepath{stroke,fill}%
}%
\begin{pgfscope}%
\pgfsys@transformshift{0.629692in}{2.096967in}%
\pgfsys@useobject{currentmarker}{}%
\end{pgfscope}%
\end{pgfscope}%
\begin{pgfscope}%
\pgftext[x=0.463025in,y=2.044205in,left,base]{\rmfamily\fontsize{10.000000}{12.000000}\selectfont \(\displaystyle 1\)}%
\end{pgfscope}%
\begin{pgfscope}%
\pgfpathrectangle{\pgfqpoint{0.629692in}{1.756603in}}{\pgfqpoint{0.680000in}{0.468000in}} %
\pgfusepath{clip}%
\pgfsetrectcap%
\pgfsetroundjoin%
\pgfsetlinewidth{1.003750pt}%
\definecolor{currentstroke}{rgb}{0.121569,0.466667,0.705882}%
\pgfsetstrokecolor{currentstroke}%
\pgfsetdash{}{0pt}%
\pgfpathmoveto{\pgfqpoint{0.660601in}{2.203331in}}%
\pgfpathlineto{\pgfqpoint{0.666845in}{2.199033in}}%
\pgfpathlineto{\pgfqpoint{0.673090in}{2.194736in}}%
\pgfpathlineto{\pgfqpoint{0.679334in}{2.190438in}}%
\pgfpathlineto{\pgfqpoint{0.685578in}{2.186141in}}%
\pgfpathlineto{\pgfqpoint{0.691822in}{2.181843in}}%
\pgfpathlineto{\pgfqpoint{0.698067in}{2.177545in}}%
\pgfpathlineto{\pgfqpoint{0.704311in}{2.173248in}}%
\pgfpathlineto{\pgfqpoint{0.710555in}{2.168950in}}%
\pgfpathlineto{\pgfqpoint{0.716799in}{2.164653in}}%
\pgfpathlineto{\pgfqpoint{0.723044in}{2.160355in}}%
\pgfpathlineto{\pgfqpoint{0.729288in}{2.156058in}}%
\pgfpathlineto{\pgfqpoint{0.735532in}{2.151760in}}%
\pgfpathlineto{\pgfqpoint{0.741776in}{2.147463in}}%
\pgfpathlineto{\pgfqpoint{0.748021in}{2.143165in}}%
\pgfpathlineto{\pgfqpoint{0.754265in}{2.138868in}}%
\pgfpathlineto{\pgfqpoint{0.760509in}{2.134570in}}%
\pgfpathlineto{\pgfqpoint{0.766753in}{2.130273in}}%
\pgfpathlineto{\pgfqpoint{0.772998in}{2.125975in}}%
\pgfpathlineto{\pgfqpoint{0.779242in}{2.121678in}}%
\pgfpathlineto{\pgfqpoint{0.785486in}{2.117380in}}%
\pgfpathlineto{\pgfqpoint{0.791731in}{2.113083in}}%
\pgfpathlineto{\pgfqpoint{0.797975in}{2.108785in}}%
\pgfpathlineto{\pgfqpoint{0.804219in}{2.104488in}}%
\pgfpathlineto{\pgfqpoint{0.810463in}{2.100190in}}%
\pgfpathlineto{\pgfqpoint{0.816708in}{2.095893in}}%
\pgfpathlineto{\pgfqpoint{0.822952in}{2.091595in}}%
\pgfpathlineto{\pgfqpoint{0.829196in}{2.087298in}}%
\pgfpathlineto{\pgfqpoint{0.835440in}{2.083000in}}%
\pgfpathlineto{\pgfqpoint{0.841685in}{2.078703in}}%
\pgfpathlineto{\pgfqpoint{0.847929in}{2.074405in}}%
\pgfpathlineto{\pgfqpoint{0.854173in}{2.070107in}}%
\pgfpathlineto{\pgfqpoint{0.860417in}{2.065810in}}%
\pgfpathlineto{\pgfqpoint{0.866662in}{2.061512in}}%
\pgfpathlineto{\pgfqpoint{0.872906in}{2.057215in}}%
\pgfpathlineto{\pgfqpoint{0.879150in}{2.052917in}}%
\pgfpathlineto{\pgfqpoint{0.885394in}{2.048620in}}%
\pgfpathlineto{\pgfqpoint{0.891639in}{2.044322in}}%
\pgfpathlineto{\pgfqpoint{0.897883in}{2.040025in}}%
\pgfpathlineto{\pgfqpoint{0.904127in}{2.035727in}}%
\pgfpathlineto{\pgfqpoint{0.910371in}{2.031430in}}%
\pgfpathlineto{\pgfqpoint{0.916616in}{2.027132in}}%
\pgfpathlineto{\pgfqpoint{0.922860in}{2.022835in}}%
\pgfpathlineto{\pgfqpoint{0.929104in}{2.018537in}}%
\pgfpathlineto{\pgfqpoint{0.935349in}{2.014240in}}%
\pgfpathlineto{\pgfqpoint{0.941593in}{2.009942in}}%
\pgfpathlineto{\pgfqpoint{0.947837in}{2.005645in}}%
\pgfpathlineto{\pgfqpoint{0.954081in}{2.001347in}}%
\pgfpathlineto{\pgfqpoint{0.960326in}{1.997050in}}%
\pgfpathlineto{\pgfqpoint{0.966570in}{1.992752in}}%
\pgfpathlineto{\pgfqpoint{0.972814in}{1.988455in}}%
\pgfpathlineto{\pgfqpoint{0.979058in}{1.984157in}}%
\pgfpathlineto{\pgfqpoint{0.985303in}{1.979860in}}%
\pgfpathlineto{\pgfqpoint{0.991547in}{1.975562in}}%
\pgfpathlineto{\pgfqpoint{0.997791in}{1.971264in}}%
\pgfpathlineto{\pgfqpoint{1.004035in}{1.966967in}}%
\pgfpathlineto{\pgfqpoint{1.010280in}{1.962669in}}%
\pgfpathlineto{\pgfqpoint{1.016524in}{1.958372in}}%
\pgfpathlineto{\pgfqpoint{1.022768in}{1.954074in}}%
\pgfpathlineto{\pgfqpoint{1.029012in}{1.949777in}}%
\pgfpathlineto{\pgfqpoint{1.035257in}{1.945479in}}%
\pgfpathlineto{\pgfqpoint{1.041501in}{1.941182in}}%
\pgfpathlineto{\pgfqpoint{1.047745in}{1.936884in}}%
\pgfpathlineto{\pgfqpoint{1.053989in}{1.932587in}}%
\pgfpathlineto{\pgfqpoint{1.060234in}{1.928289in}}%
\pgfpathlineto{\pgfqpoint{1.066478in}{1.923992in}}%
\pgfpathlineto{\pgfqpoint{1.072722in}{1.919694in}}%
\pgfpathlineto{\pgfqpoint{1.078967in}{1.915397in}}%
\pgfpathlineto{\pgfqpoint{1.085211in}{1.911099in}}%
\pgfpathlineto{\pgfqpoint{1.091455in}{1.906802in}}%
\pgfpathlineto{\pgfqpoint{1.097699in}{1.902504in}}%
\pgfpathlineto{\pgfqpoint{1.103944in}{1.898207in}}%
\pgfpathlineto{\pgfqpoint{1.110188in}{1.893909in}}%
\pgfpathlineto{\pgfqpoint{1.116432in}{1.889612in}}%
\pgfpathlineto{\pgfqpoint{1.122676in}{1.885314in}}%
\pgfpathlineto{\pgfqpoint{1.128921in}{1.881017in}}%
\pgfpathlineto{\pgfqpoint{1.135165in}{1.876719in}}%
\pgfpathlineto{\pgfqpoint{1.141409in}{1.872422in}}%
\pgfpathlineto{\pgfqpoint{1.147653in}{1.868124in}}%
\pgfpathlineto{\pgfqpoint{1.153898in}{1.863826in}}%
\pgfpathlineto{\pgfqpoint{1.160142in}{1.859529in}}%
\pgfpathlineto{\pgfqpoint{1.166386in}{1.855231in}}%
\pgfpathlineto{\pgfqpoint{1.172630in}{1.850934in}}%
\pgfpathlineto{\pgfqpoint{1.178875in}{1.846636in}}%
\pgfpathlineto{\pgfqpoint{1.185119in}{1.842339in}}%
\pgfpathlineto{\pgfqpoint{1.191363in}{1.838041in}}%
\pgfpathlineto{\pgfqpoint{1.197607in}{1.833744in}}%
\pgfpathlineto{\pgfqpoint{1.203852in}{1.829446in}}%
\pgfpathlineto{\pgfqpoint{1.210096in}{1.825149in}}%
\pgfpathlineto{\pgfqpoint{1.216340in}{1.820851in}}%
\pgfpathlineto{\pgfqpoint{1.222585in}{1.816554in}}%
\pgfpathlineto{\pgfqpoint{1.228829in}{1.812256in}}%
\pgfpathlineto{\pgfqpoint{1.235073in}{1.807959in}}%
\pgfpathlineto{\pgfqpoint{1.241317in}{1.803661in}}%
\pgfpathlineto{\pgfqpoint{1.247562in}{1.799364in}}%
\pgfpathlineto{\pgfqpoint{1.253806in}{1.795066in}}%
\pgfpathlineto{\pgfqpoint{1.260050in}{1.790769in}}%
\pgfpathlineto{\pgfqpoint{1.266294in}{1.786471in}}%
\pgfpathlineto{\pgfqpoint{1.272539in}{1.782174in}}%
\pgfpathlineto{\pgfqpoint{1.278783in}{1.777876in}}%
\pgfusepath{stroke}%
\end{pgfscope}%
\begin{pgfscope}%
\pgfpathrectangle{\pgfqpoint{0.629692in}{1.756603in}}{\pgfqpoint{0.680000in}{0.468000in}} %
\pgfusepath{clip}%
\pgfsetrectcap%
\pgfsetroundjoin%
\pgfsetlinewidth{1.003750pt}%
\definecolor{currentstroke}{rgb}{1.000000,0.498039,0.054902}%
\pgfsetstrokecolor{currentstroke}%
\pgfsetdash{}{0pt}%
\pgfpathmoveto{\pgfqpoint{0.660601in}{1.777876in}}%
\pgfpathlineto{\pgfqpoint{0.666845in}{1.782174in}}%
\pgfpathlineto{\pgfqpoint{0.673090in}{1.786471in}}%
\pgfpathlineto{\pgfqpoint{0.679334in}{1.790769in}}%
\pgfpathlineto{\pgfqpoint{0.685578in}{1.795066in}}%
\pgfpathlineto{\pgfqpoint{0.691822in}{1.799364in}}%
\pgfpathlineto{\pgfqpoint{0.698067in}{1.803661in}}%
\pgfpathlineto{\pgfqpoint{0.704311in}{1.807959in}}%
\pgfpathlineto{\pgfqpoint{0.710555in}{1.812256in}}%
\pgfpathlineto{\pgfqpoint{0.716799in}{1.816554in}}%
\pgfpathlineto{\pgfqpoint{0.723044in}{1.820851in}}%
\pgfpathlineto{\pgfqpoint{0.729288in}{1.825149in}}%
\pgfpathlineto{\pgfqpoint{0.735532in}{1.829446in}}%
\pgfpathlineto{\pgfqpoint{0.741776in}{1.833744in}}%
\pgfpathlineto{\pgfqpoint{0.748021in}{1.838041in}}%
\pgfpathlineto{\pgfqpoint{0.754265in}{1.842339in}}%
\pgfpathlineto{\pgfqpoint{0.760509in}{1.846636in}}%
\pgfpathlineto{\pgfqpoint{0.766753in}{1.850934in}}%
\pgfpathlineto{\pgfqpoint{0.772998in}{1.855231in}}%
\pgfpathlineto{\pgfqpoint{0.779242in}{1.859529in}}%
\pgfpathlineto{\pgfqpoint{0.785486in}{1.863826in}}%
\pgfpathlineto{\pgfqpoint{0.791731in}{1.868124in}}%
\pgfpathlineto{\pgfqpoint{0.797975in}{1.872422in}}%
\pgfpathlineto{\pgfqpoint{0.804219in}{1.876719in}}%
\pgfpathlineto{\pgfqpoint{0.810463in}{1.881017in}}%
\pgfpathlineto{\pgfqpoint{0.816708in}{1.885314in}}%
\pgfpathlineto{\pgfqpoint{0.822952in}{1.889612in}}%
\pgfpathlineto{\pgfqpoint{0.829196in}{1.893909in}}%
\pgfpathlineto{\pgfqpoint{0.835440in}{1.898207in}}%
\pgfpathlineto{\pgfqpoint{0.841685in}{1.902504in}}%
\pgfpathlineto{\pgfqpoint{0.847929in}{1.906802in}}%
\pgfpathlineto{\pgfqpoint{0.854173in}{1.911099in}}%
\pgfpathlineto{\pgfqpoint{0.860417in}{1.915397in}}%
\pgfpathlineto{\pgfqpoint{0.866662in}{1.919694in}}%
\pgfpathlineto{\pgfqpoint{0.872906in}{1.923992in}}%
\pgfpathlineto{\pgfqpoint{0.879150in}{1.928289in}}%
\pgfpathlineto{\pgfqpoint{0.885394in}{1.932587in}}%
\pgfpathlineto{\pgfqpoint{0.891639in}{1.936884in}}%
\pgfpathlineto{\pgfqpoint{0.897883in}{1.941182in}}%
\pgfpathlineto{\pgfqpoint{0.904127in}{1.945479in}}%
\pgfpathlineto{\pgfqpoint{0.910371in}{1.949777in}}%
\pgfpathlineto{\pgfqpoint{0.916616in}{1.954074in}}%
\pgfpathlineto{\pgfqpoint{0.922860in}{1.958372in}}%
\pgfpathlineto{\pgfqpoint{0.929104in}{1.962669in}}%
\pgfpathlineto{\pgfqpoint{0.935349in}{1.966967in}}%
\pgfpathlineto{\pgfqpoint{0.941593in}{1.971264in}}%
\pgfpathlineto{\pgfqpoint{0.947837in}{1.975562in}}%
\pgfpathlineto{\pgfqpoint{0.954081in}{1.979860in}}%
\pgfpathlineto{\pgfqpoint{0.960326in}{1.984157in}}%
\pgfpathlineto{\pgfqpoint{0.966570in}{1.988455in}}%
\pgfpathlineto{\pgfqpoint{0.972814in}{1.992752in}}%
\pgfpathlineto{\pgfqpoint{0.979058in}{1.997050in}}%
\pgfpathlineto{\pgfqpoint{0.985303in}{2.001347in}}%
\pgfpathlineto{\pgfqpoint{0.991547in}{2.005645in}}%
\pgfpathlineto{\pgfqpoint{0.997791in}{2.009942in}}%
\pgfpathlineto{\pgfqpoint{1.004035in}{2.014240in}}%
\pgfpathlineto{\pgfqpoint{1.010280in}{2.018537in}}%
\pgfpathlineto{\pgfqpoint{1.016524in}{2.022835in}}%
\pgfpathlineto{\pgfqpoint{1.022768in}{2.027132in}}%
\pgfpathlineto{\pgfqpoint{1.029012in}{2.031430in}}%
\pgfpathlineto{\pgfqpoint{1.035257in}{2.035727in}}%
\pgfpathlineto{\pgfqpoint{1.041501in}{2.040025in}}%
\pgfpathlineto{\pgfqpoint{1.047745in}{2.044322in}}%
\pgfpathlineto{\pgfqpoint{1.053989in}{2.048620in}}%
\pgfpathlineto{\pgfqpoint{1.060234in}{2.052917in}}%
\pgfpathlineto{\pgfqpoint{1.066478in}{2.057215in}}%
\pgfpathlineto{\pgfqpoint{1.072722in}{2.061512in}}%
\pgfpathlineto{\pgfqpoint{1.078967in}{2.065810in}}%
\pgfpathlineto{\pgfqpoint{1.085211in}{2.070107in}}%
\pgfpathlineto{\pgfqpoint{1.091455in}{2.074405in}}%
\pgfpathlineto{\pgfqpoint{1.097699in}{2.078703in}}%
\pgfpathlineto{\pgfqpoint{1.103944in}{2.083000in}}%
\pgfpathlineto{\pgfqpoint{1.110188in}{2.087298in}}%
\pgfpathlineto{\pgfqpoint{1.116432in}{2.091595in}}%
\pgfpathlineto{\pgfqpoint{1.122676in}{2.095893in}}%
\pgfpathlineto{\pgfqpoint{1.128921in}{2.100190in}}%
\pgfpathlineto{\pgfqpoint{1.135165in}{2.104488in}}%
\pgfpathlineto{\pgfqpoint{1.141409in}{2.108785in}}%
\pgfpathlineto{\pgfqpoint{1.147653in}{2.113083in}}%
\pgfpathlineto{\pgfqpoint{1.153898in}{2.117380in}}%
\pgfpathlineto{\pgfqpoint{1.160142in}{2.121678in}}%
\pgfpathlineto{\pgfqpoint{1.166386in}{2.125975in}}%
\pgfpathlineto{\pgfqpoint{1.172630in}{2.130273in}}%
\pgfpathlineto{\pgfqpoint{1.178875in}{2.134570in}}%
\pgfpathlineto{\pgfqpoint{1.185119in}{2.138868in}}%
\pgfpathlineto{\pgfqpoint{1.191363in}{2.143165in}}%
\pgfpathlineto{\pgfqpoint{1.197607in}{2.147463in}}%
\pgfpathlineto{\pgfqpoint{1.203852in}{2.151760in}}%
\pgfpathlineto{\pgfqpoint{1.210096in}{2.156058in}}%
\pgfpathlineto{\pgfqpoint{1.216340in}{2.160355in}}%
\pgfpathlineto{\pgfqpoint{1.222585in}{2.164653in}}%
\pgfpathlineto{\pgfqpoint{1.228829in}{2.168950in}}%
\pgfpathlineto{\pgfqpoint{1.235073in}{2.173248in}}%
\pgfpathlineto{\pgfqpoint{1.241317in}{2.177545in}}%
\pgfpathlineto{\pgfqpoint{1.247562in}{2.181843in}}%
\pgfpathlineto{\pgfqpoint{1.253806in}{2.186141in}}%
\pgfpathlineto{\pgfqpoint{1.260050in}{2.190438in}}%
\pgfpathlineto{\pgfqpoint{1.266294in}{2.194736in}}%
\pgfpathlineto{\pgfqpoint{1.272539in}{2.199033in}}%
\pgfpathlineto{\pgfqpoint{1.278783in}{2.203331in}}%
\pgfusepath{stroke}%
\end{pgfscope}%
\begin{pgfscope}%
\pgfpathrectangle{\pgfqpoint{0.629692in}{1.756603in}}{\pgfqpoint{0.680000in}{0.468000in}} %
\pgfusepath{clip}%
\pgfsetbuttcap%
\pgfsetroundjoin%
\definecolor{currentfill}{rgb}{0.000000,0.000000,0.000000}%
\pgfsetfillcolor{currentfill}%
\pgfsetlinewidth{1.003750pt}%
\definecolor{currentstroke}{rgb}{0.000000,0.000000,0.000000}%
\pgfsetstrokecolor{currentstroke}%
\pgfsetdash{}{0pt}%
\pgfsys@defobject{currentmarker}{\pgfqpoint{-0.020833in}{-0.020833in}}{\pgfqpoint{0.020833in}{0.020833in}}{%
\pgfpathmoveto{\pgfqpoint{0.000000in}{-0.020833in}}%
\pgfpathcurveto{\pgfqpoint{0.005525in}{-0.020833in}}{\pgfqpoint{0.010825in}{-0.018638in}}{\pgfqpoint{0.014731in}{-0.014731in}}%
\pgfpathcurveto{\pgfqpoint{0.018638in}{-0.010825in}}{\pgfqpoint{0.020833in}{-0.005525in}}{\pgfqpoint{0.020833in}{0.000000in}}%
\pgfpathcurveto{\pgfqpoint{0.020833in}{0.005525in}}{\pgfqpoint{0.018638in}{0.010825in}}{\pgfqpoint{0.014731in}{0.014731in}}%
\pgfpathcurveto{\pgfqpoint{0.010825in}{0.018638in}}{\pgfqpoint{0.005525in}{0.020833in}}{\pgfqpoint{0.000000in}{0.020833in}}%
\pgfpathcurveto{\pgfqpoint{-0.005525in}{0.020833in}}{\pgfqpoint{-0.010825in}{0.018638in}}{\pgfqpoint{-0.014731in}{0.014731in}}%
\pgfpathcurveto{\pgfqpoint{-0.018638in}{0.010825in}}{\pgfqpoint{-0.020833in}{0.005525in}}{\pgfqpoint{-0.020833in}{0.000000in}}%
\pgfpathcurveto{\pgfqpoint{-0.020833in}{-0.005525in}}{\pgfqpoint{-0.018638in}{-0.010825in}}{\pgfqpoint{-0.014731in}{-0.014731in}}%
\pgfpathcurveto{\pgfqpoint{-0.010825in}{-0.018638in}}{\pgfqpoint{-0.005525in}{-0.020833in}}{\pgfqpoint{0.000000in}{-0.020833in}}%
\pgfpathclose%
\pgfusepath{stroke,fill}%
}%
\begin{pgfscope}%
\pgfsys@transformshift{0.660601in}{1.884240in}%
\pgfsys@useobject{currentmarker}{}%
\end{pgfscope}%
\begin{pgfscope}%
\pgfsys@transformshift{0.969692in}{1.884240in}%
\pgfsys@useobject{currentmarker}{}%
\end{pgfscope}%
\begin{pgfscope}%
\pgfsys@transformshift{1.278783in}{1.884240in}%
\pgfsys@useobject{currentmarker}{}%
\end{pgfscope}%
\end{pgfscope}%
\begin{pgfscope}%
\pgfsetrectcap%
\pgfsetmiterjoin%
\pgfsetlinewidth{0.803000pt}%
\definecolor{currentstroke}{rgb}{0.000000,0.000000,0.000000}%
\pgfsetstrokecolor{currentstroke}%
\pgfsetdash{}{0pt}%
\pgfpathmoveto{\pgfqpoint{0.629692in}{1.756603in}}%
\pgfpathlineto{\pgfqpoint{0.629692in}{2.224603in}}%
\pgfusepath{stroke}%
\end{pgfscope}%
\begin{pgfscope}%
\pgfsetrectcap%
\pgfsetmiterjoin%
\pgfsetlinewidth{0.803000pt}%
\definecolor{currentstroke}{rgb}{0.000000,0.000000,0.000000}%
\pgfsetstrokecolor{currentstroke}%
\pgfsetdash{}{0pt}%
\pgfpathmoveto{\pgfqpoint{1.309692in}{1.756603in}}%
\pgfpathlineto{\pgfqpoint{1.309692in}{2.224603in}}%
\pgfusepath{stroke}%
\end{pgfscope}%
\begin{pgfscope}%
\pgfsetrectcap%
\pgfsetmiterjoin%
\pgfsetlinewidth{0.803000pt}%
\definecolor{currentstroke}{rgb}{0.000000,0.000000,0.000000}%
\pgfsetstrokecolor{currentstroke}%
\pgfsetdash{}{0pt}%
\pgfpathmoveto{\pgfqpoint{0.629692in}{1.756603in}}%
\pgfpathlineto{\pgfqpoint{1.309692in}{1.756603in}}%
\pgfusepath{stroke}%
\end{pgfscope}%
\begin{pgfscope}%
\pgfsetrectcap%
\pgfsetmiterjoin%
\pgfsetlinewidth{0.803000pt}%
\definecolor{currentstroke}{rgb}{0.000000,0.000000,0.000000}%
\pgfsetstrokecolor{currentstroke}%
\pgfsetdash{}{0pt}%
\pgfpathmoveto{\pgfqpoint{0.629692in}{2.224603in}}%
\pgfpathlineto{\pgfqpoint{1.309692in}{2.224603in}}%
\pgfusepath{stroke}%
\end{pgfscope}%
\begin{pgfscope}%
\pgftext[x=0.969692in,y=2.307937in,,base]{\rmfamily\fontsize{12.000000}{14.400000}\selectfont Local \(\displaystyle \chi_{n+1/2}\)}%
\end{pgfscope}%
\end{pgfpicture}%
\makeatother%
\endgroup%
}
\caption{(a) Lagrange shape functions of degree $p=2$ in the reference element $I=[-1,1]$ and the corresponding periodic basis functions on a physical domain of length $L=1$ which has been discretized by $N_\mr{el}=3$ elements of equal length. (b) Corresponding local histopolation shape and basis functions.\label{fig_Lagrange}}
\end{figure}
\hspace{-2mm} The construction of the \textit{basis} functions on the physical domain is then done by noting that we need continuity at the shared degrees of freedom at the element boundaries, which is why such shape functions form a mutual basis function. This leads to a total number of $N_0=pN_\mr{el}$ basis functions in case of periodic boundary conditions. The corresponding projector $\Pi_0$ on this basis acting on some continuous function $E\in H^1$ is defined by
\begin{align}
\Pi_0:H^1\rightarrow V_0,\quad(\Pi_0 E)(z_i)=E(z_i),\label{eq_def_projector0}
\end{align}
where $(z_i)_{i=0,\ldots,N_0-1}$ is the global knots sequence on the physical domain which satisfies $\varphi^0_i(z_j)=\delta_{ij}$. Denoting the projected function by $E_h:=\Pi_0E$ we thus have
\begin{align}
E(z_i)=E_h(z_i)=\sum_{j=0}^{N_0-1}e_j\varphi_j^0(z_i)=e_j,
\end{align}
which means that the finite element coefficients are the values of the function at the knot sequence $(z_i)_{i=0,\ldots,N_0-1}$. As a next step, we consider the space $V_1$ and define the shape functions $(\chi_{n+1/2})_{n=0,\ldots,p-1}$ in the reference element $I$ by
\begin{align}
\int_{s_m}^{s_{m+1}}\chi_{n+1/2}(s)\mr{d}s=\delta_{nm},\label{eq_def_LHP}
\end{align}
where $s_0=-1<\ldots<s_p=1$ is the same local knots sequence as for the usual Lagrange shape functions. Some simple considerations yield that the solution of these equations is given by linear combinations of first order derivatives of the Lagrange shape functions $(\eta_n(s))_{n=0,\ldots,p}$,
\begin{align}
\chi_{n+1/2}(s)=\sum_{m=n+1}^p\frac{\mr{d}}{\mr{d}s}\eta_m(s),\label{eq_def_Lagrange_histo}
\end{align}
which can be verified by plugging this in the definition (\ref{eq_def_LHP}) and using the property $\eta_n(s_m)=\delta_{nm}$. In order to get a basis on the physical domain, these shape function are just put next to each other since there are no shared degrees of freedom at the element boundaries at which continuity must be enforced. This also has the consequence that the total number of basis function is again $N_1=pN_\mr{el}$, however, in contrast to the previous case, there are now $p$ non-vanishing basis function per element (and not $p+1$). We define the corresponding projector $\Pi_1$ acting on some square integrable function $B\in L^2$ by
\begin{align}
\Pi_1:L^2\rightarrow V_1,\quad\int_{z_i}^{z_{i+1}}(\Pi_1 B)(z)\mr{d}z=\int_{z_i}^{z_{i+1}}B(z)\mr{d}z.\label{eq_def_projector1}
\end{align}
Note that $i=0,\ldots,N_0-1$ and thus $z_{N_0}=L$ is just the right end of the domain in this case (this point does actually not exist due to periodic boundary conditions). Again, denoting the projected function by $B_h:=\Pi_1B$ we have
\begin{align}
\int_{z_i}^{z_{i+1}}B(z)\mr{d}z=\int_{z_i}^{z_{i+1}}B_h(z)\mr{d}z=\sum_{j=0}^{N_1-1}b_{j+1/2}\int_{z_i}^{z_{i+1}}\varphi_{j+1/2}^1(z)\mr{d}z=\frac{c_{k+1}-c_k}{2}b_{i+1/2}\quad\quad\forall\,z_i\in[c_k,c_{k+1}).\label{eq_coefficients_V1}
\end{align}
The proof that this choice for the bases of the space $V_0$ and $V_1$ together with the projectors $\Pi_0$ (\ref{eq_def_projector0}) and $\Pi_1$ (\ref{eq_def_projector1}) is indeed valid can be shown in the following way: take $\psi\in H^1$ and note that
\begin{align}
&\int_{z_i}^{z_{i+1}}(\Pi_1\frac{\pa\psi}{\pa z})(z)\mr{d}z\underset{\underset{\text{(\ref{eq_def_projector1})}}{\uparrow}}{=}\int_{z_i}^{z_{i+1}}\frac{\pa\psi}{\pa z}(z)\mr{d}z=\psi(z_{i+1})-\psi(z_i)\underset{\underset{\text{(\ref{eq_def_projector0})}}{\uparrow}}{=}(\Pi_0\psi)(z_{i+1})-(\Pi_0\psi)(z_{i})=\int_{z_i}^{z_{i+1}}\frac{\pa}{\pa z}(\Pi_0\psi)(z)\mr{d}z,
\end{align}
which means that $\Pi_1\pa\psi/\pa z=\pa/\pa z(\Pi_0\psi)$ and hence the diagram is commuting. 

In order to obtain a matrix formulation out of the (discrete) weak formulation (\ref{eq_weak_gem_discrete}), we express all quantities in their respective basis by
\begin{align}
\tilde{E}_{hx/y}(z,t)=\sum_{j=0}^{N_0-1}e_{x/yj}(t)\varphi_j^0(z),\quad\tilde{B}_{hx/y}(z,t)=\sum_{j=0}^{N_1-1}b_{x/yj+1/2}(t)\varphi_{j+1/2}^1(z),\quad\tilde{j}_{\mr{c}x/y}^h(z,t)=\sum_{j=0}^{N_0-1}y_{x/yj}(t)\varphi_j^0(z),
\end{align}
and plug this in the weak formulation (\ref{eq_weak_gem_discrete}). The same is done for the test functions $F_{hx/y}\in V_0$, $C_{hx/y}\in V_1$ and $O_{hx/y}\in V_0$. Let us do this in an exemplary way for the $x$-component of Amper\'{e}re's law (\ref{eq_weak_gem_discrete_1}) by noting that the spatial derivative in the second term is acting on the test function $F_{hx}\in V_0$ with coefficients $(f_{j})_{j=0,\ldots,N_0-1}$. According to the diagram in fig. \ref{fig_commuting_diagram}, this has the consequence that the function $\pa F_{hx}/\pa z$ must now be part of the space $V_1$ with new coefficients $(f_{j+1/2})_{j=0,\ldots,N_1-1}$, which are given by formula (\ref{eq_coefficients_V1}):
\begin{align}
\frac{c_{k+1}-c_k}{2}f_{j+1/2}=\int_{z_j}^{z_{j+1}}\frac{\pa F_{hx}}{\pa z}\mr{d}z=\sum_{i=0}^{N_0-1}f_i\int_{z_j}^{z_{j+1}}\frac{\pa}{\pa z}\varphi_i^0(z)\mr{d}z=\sum_{i=0}^{N_0-1}f_i\left[\varphi_i^0(z_{j+1})-\varphi_i^0(z_{j})\right]=f_{j+1}-f_j.
\end{align} 
For a uniform mesh $c_{k+1}-c_k=h$ we hence get 
\begin{align}
\begin{split}
&\sum_{i,j}^{N_0-1}\frac{\mr{d}e_{xj}}{\mr{d}t}f_{xi}\underbrace{\int_0^L\varphi_i^0\varphi_j^0\mr{d}z}_{=:m_{ij}^0}-\frac{2c^2}{h}\sum_{i,j=0}^{N_1-1}b_{yj+1/2}(f_{i+1}-f_i)\underbrace{\int_0^L\varphi_{i+1/2}^1\varphi_{j+1/2}^1\mr{d}z}_{=:m_{ij}^1}+\mu_0c^2\sum_{i,j=0}^{N_0-1}y_{xj}f_{xi}\underbrace{\int_0^L\varphi_i^0\varphi_j^0\mr{d}z}_{=:m_{ij}^0}\\
=&-\mu_0c^2\sum_{i=0}^{N_0-1}f_{xi}\underbrace{\int_0^Lj_{\mr{h}x}\varphi_i^0\mr{d}z}_{=:\bar{j}_{\mr{h}xi}}
\end{split}
\end{align}
for the $x$-component of Amper\'{e}re's law (\ref{eq_weak_gem_discrete_1}), where we have defined the entries of the two mass matrices $\mathbb{M}^0:=(m_{ij}^0)_{i,j=0,\ldots,N_0-1}$ and $\mathbb{M}^1:=(m_{ij}^1)_{i,j=0,\ldots,N_1-1}$, respectively, as well as the vector $\bar{\mb{j}}_{\mr{h}x}:=(\bar{j}_{\mr{h}xi})_{i=0,\ldots,N_0-1}$ for the right-hand side which is coupled to the particle-in-cell part of the algorithm in the exact same way as it was done in (\ref{eq_hotcurrent_weak}). All together, this leads to the equivalent matrix formulation
\begin{subequations}
\begin{align}
&\textbf{f}_x^\top\mathbb{M}^0\frac{\text{d}\textbf{e}_x}{\text{d}t}-c^2(\mathbb{G}\textbf{f}_x)^\top\mathbb{M}^1\textbf{b}_y+\mu_0c^2\textbf{f}_x^\top\mathbb{M}^0\textbf{y}_x=-\mu_0c^2q_\mr{e}\textbf{f}_x^\top\mathbb{Q}^0\mathbb{W}\mb{V}_x\\
\Leftrightarrow\quad&\mathbb{M}^0\frac{\text{d}\textbf{e}_x}{\text{d}t}-c^2\mathbb{G}^\top\mathbb{M}^1\textbf{b}_y+\mu_0c^2q_\mr{e}\mathbb{M}^0\textbf{y}_x=-\mu_0c^2\mathbb{Q}^0\mathbb{W}\mb{V}_x,
\end{align}
\end{subequations}
since we want this to be true for all $\mb{f}_x$. Furthermore, we have introduced the vectors $\mb{Z}=(z_1\ldots,z_{N_\mr{p}})^\top\in\mathbb{R}^{N_\mr{p}}$ and $\mb{V}_x=(v_{1x}\ldots,v_{N_\mr{p}x})^\top\in\mathbb{R}^{N_\mr{p}}$ holding the particles' positions and velocities, respectively. The matrices $\mathbb{Q}^0\in\mathbb{R}^{N_0\times N_\mr{p}}$ and $\mathbb{W}\in\mathbb{R}^{N_\mr{p}\times N_\mr{p}}$ are
\begin{subequations}
\begin{align}
&\mathbb{Q}^0=\mathbb{Q}^0(\mb{Z}):=(\varphi_i^0(z_k))_{i=0,\ldots,N_0-1,k=1\ldots,N_\mr{p}},\label{eq_def_Q0}\\
&\mathbb{W}:=\mr{diag}(w_1,\ldots,w_{N_\mr{p}}),\label{eq_def_W}
\end{align}
\end{subequations}
and simply result from writing (\ref{eq_hotcurrent_weak}) in terms of matrix-vector multiplications. Finally, we have introduced the discrete gradient matrix
\begin{align}
\mathbb{G}:=\frac{2}{h}
\begin{pmatrix}
 -1 & 1  &  &  &  \\
  & -1 & 1 & &   \\
  &  & \ddots & \ddots &  \\
  &  &   & -1 & 1 \\
  1 &  &  &    & -1  
\end{pmatrix} \quad\in\mathbb{R}^{N_1\times N_0},\label{eq_discrete_gradient}
\end{align}
where the last row is due to periodic boundary conditions and thus $f_{N_0}=f_0$, for instance.

Doing the same for the other equations in (\ref{eq_weak_gem_discrete}) as well as for the equations of motion for the particles, (\ref{eq_motion_particles}) leads to the following semi-discrete system for the ten variables $\mb{\mb{u}}=(\mb{e}_x,\mb{e}_y,\mb{b}_x,\mb{b}_y,\mb{y}_x,\mb{y}_y,\mb{Z},\mb{V}_x,\mb{V}_y,\mb{V}_z)\in\mathbb{R}^{4N_0+2N_1+4N_\mr{p}}$:
\begin{subequations}
\label{eq_semi}
\begin{align}
    &\frac{\mathrm{d}\textbf{e}_x}{\mathrm{d} t}=c^2(\mathbb{M}^0)^{-1}\mathbb{G}^\top\mathbb{M}^1\textbf{b}_y-\mu_0c^2\textbf{y}_x-\mu_0c^2q_\text{e}(\mathbb{M}^0)^{-1}\mathbb{Q}^0\mathbb{W}\textbf{V}_x,\label{eq_semi1}\\
    &\frac{\mathrm{d}\textbf{e}_y}{\mathrm{d} t}=-c^2(\mathbb{M}^0)^{-1}\mathbb{G}^\top\mathbb{M}^1\textbf{b}_x-\mu_0c^2\textbf{y}_y-\mu_0c^2q_\text{e}(\mathbb{M}^0)^{-1}\mathbb{Q}^0\mathbb{W}\textbf{V}_y,\label{eq_semi2}\\
    &\frac{\mathrm{d}\textbf{b}_x}{\mathrm{d} t}=\mathbb{G}\textbf{e}_y,\label{eq_semi3}\\
    &\frac{\mathrm{d}\textbf{b}_y}{\mathrm{d} t}=-\mathbb{G}\textbf{e}_x,\label{eq_semi4}\\
    &\frac{\mathrm{d}\textbf{y}_x}{\mathrm{d} t}=\epsilon_0\Omega_\text{pe}^2\textbf{e}_x+\Omega_\text{ce}\textbf{y}_y,\label{eq_semi5}\\
    &\frac{\mathrm{d}\textbf{y}_y}{\mathrm{d} t}=\epsilon_0\Omega_\text{pe}^2\textbf{e}_y-\Omega_\text{ce}\textbf{y}_x,\label{eq_semi6}\\
    &\frac{\mathrm{d}\textbf{Z}}{\mathrm{d} t}=\textbf{V}_z,\label{eq_semi7}\\
    &\frac{\mathrm{d}\textbf{V}_x}{\mathrm{d} t}=\frac{q_\text{e}}{m_\text{e}}[(\mathbb{Q}^0)^\top\textbf{e}_x-\mathbb{B}_y\textbf{V}_z+B_0\textbf{V}_y],\label{eq_semi8}\\
    &\frac{\mathrm{d}\textbf{V}_y}{\mathrm{d} t}=\frac{q_\text{e}}{m_\text{e}}[(\mathbb{Q}^0)^\top\textbf{e}_y+\mathbb{B}_x\textbf{V}_z-B_0\textbf{V}_x], \label{eq_semi9}\\
    &\frac{\mathrm{d}\textbf{V}_z}{\mathrm{d} t}=\frac{q_\text{e}}{m_\text{e}}[\mathbb{B}_y\textbf{V}_x-\mathbb{B}_x\textbf{V}_y],\label{eq_semi10},
\end{align}
\end{subequations}
where the matrices $\mathbb{Q}^1\in\mathbb{R}^{N_1\times N_\mr{p}}$ and $\mathbb{B}_{x/y}\in\mathbb{R}^{N_\mr{p}\times N_\mr{p}}$ defined by
\begin{align}
&\mathbb{Q}^1=\mathbb{Q}^1(\mb{Z}):=(\varphi_{i+1/2}^1(z_k))_{i=0,\ldots,N_1-1,k=1\ldots,N_\mr{p}},\label{eq_def_Q1}\\
&\mathbb{B}_{x/y}=\mathbb{B}_{x/y}(\mb{Z},\mb{b}_{x/y}):=\mr{diag}\left[(\mathbb{Q}^1)^\top(\mb{Z})\mb{b}_{x/y}\right]\label{eq_def_Bxy}
\end{align}
arise naturally after writing the particles' equations of motion in matrix-vector form and noting that the discrete electric and magnetic field can be expressed in their respective bases (see (\ref{eq_fields_particles})).

In order to analyze the above semi-discrete system of equations, we define the system's discrete Hamiltonian $H_h:\mathbb{R}^n\rightarrow\mathbb{R}$, $\mb{u}\mapsto H_h(\mb{u})$ ($n=4N_0+2N_1+4N_\mr{p}$) by replacing the continuous functions in the energy (\ref{eq_total_energy}) by their discrete counterparts. This results in
\begin{align}
\label{eq_discrete_Hamiltonian}
\begin{split}
H_h(\mb{u}):=&\underbrace{\frac{\epsilon_0}{2}(\mb{e}_x^\top\mathbb{M}^0\mb{e}_x+\mb{e}_y^\top\mathbb{M}^0\mb{e}_y)}_{H_E}+\underbrace{\frac{1}{2\mu_0}(\mb{b}_x^\top\mathbb{M}^1\mb{b}_x+\mb{b}_y^\top\mathbb{M}^1\mb{b}_y)}_{H_B}+\underbrace{\frac{1}{2\epsilon_0\Omega_\mr{pe}^2}(\mb{y}_x^\top\mathbb{M}^0\mb{y}_x+\mb{y}_y^\top\mathbb{M}^0\mb{y}_y)}_{H_Y}\\
+&\underbrace{\frac{m_\mr{e}}{2}\mb{V}_x^\top\mathbb{W}\mb{V}_x}_{H_x}+\underbrace{\frac{m_\mr{e}}{2}\mb{V}_y^\top\mathbb{W}\mb{V}_y}_{H_y}+\underbrace{\frac{m_\mr{e}}{2}\mb{V}_z^\top\mathbb{W}\mb{V}_z}_{H_z}.
\end{split}
\end{align} 
Using this discrete Hamiltonian, it is straightforward to show that the semi-discrete system (\ref{eq_semi}) can be equivalently written in a compact, non-canonical Hamiltonian structure \citep{Morrison2017} for the combined variable $\mb{u}$:
\begin{align}
\frac{\mr{d}\mb{u}}{\mr{d}t}=\mathbb{J}(\mb{u})\nabla_\mb{u}H_h(\mb{u}).\label{eq_Hamiltonian_structure}
\end{align}
The Poisson matrix $\mathbb{J}$ is skew-symmetric, i.e. $\mathbb{J}^\top=-\mathbb{J}$, and thus the system (\ref{eq_semi}) conserves exactly the discrete Hamiltonian (\ref{eq_discrete_Hamiltonian}) which can easily be seen by noting that
\begin{align}
\frac{\mr{d}}{\mr{d}t}H_h(\mb{u})=\nabla_\mb{u}H_h^\top\frac{\mr{d}\mb{u}}{\mr{d}t}=\nabla_\mb{u}H_h^\top\mathbb{J}(\mb{u})\nabla_\mb{u}H_h(\mb{u})=-\nabla_\mb{u}H_h^\top\mathbb{J}(\mb{u})\nabla_\mb{u}H_h(\mb{u})=0.
\end{align}
We again follow \citep{Krausetal2017} and choose a splitting scheme for the time integration. For Hamiltonian systems of the form (\ref{eq_Hamiltonian_structure}) there are in principle two options: Either one splits the Poisson matrix and keeps the full Hamiltonian. If each of the subsystems can then be solved analytically, this yields exact energy conservation. Or one splits the Hamiltonian while keeping the full Poisson matrix. This yields Poisson integrators which have the advantage that some invariants, the so-called Casimir invariants of Hamiltonian systems, are preserved exactly even on the fully discretized level. For reasons of stability, the latter option is often preferred, which is why we shall apply this method and consequently split the Hamiltonian (\ref{eq_discrete_Hamiltonian}) into the three parts 
\begin{align}
H_h=H_E+H_B+H_Y+H_x+H_y+H_z,
\end{align} 
in order to obtain six subsystems which still have the form (\ref{eq_Hamiltonian_structure}), however, with a simpler Hamiltonian, respectively. We find that each of the subsystems can be solved analytically in the way listed in the appendix, which means that we get a set of six Poisson integrators denoted by $\Phi_{\Delta t}^E$, $\Phi_{\Delta t}^B$, $\Phi_{\Delta t}^Y$, $\Phi_{\Delta t}^x$, $\Phi_{\Delta t}^y$ and $\Phi_{\Delta t}^z$, which can be applied successively in some specific order to advance $\mb{u}$ by a time step $\Delta t$. The easiest composition is the first-order Lie-Trotter splitting \citep{Trotter1959}, which consists of simply applying each integrator on after the other:
\begin{align}
\Phi_{\Delta t}^L:=\Phi_{\Delta t}^z\circ\Phi_{\Delta t}^y\circ\Phi_{\Delta t}^x\circ\Phi_{\Delta t}^Y\circ\Phi_{\Delta t}^B\circ\Phi_{\Delta t}^E.\label{eq_LieTrotter}
\end{align}
It is important to note that the input to each integrator must be the output of the previous integrator which has the consequence that if the magnetic field coefficients $\mb{b}_x$ and $\mb{b}_y$ change, for instance, the matrices $\mathbb{B}_{x/y}=\mathbb{B}_{x/y}(\mb{Z},\mb{b}_{x/y})$ need to be updated. Furthermore, we use the second order, symmetric Strang splitting \citep{Strang1968}
\begin{align}
\Phi_{\Delta t}^S:=\Phi_{\Delta t/2}^z\circ\Phi_{\Delta t/2}^y\circ\Phi_{\Delta t/2}^x\circ\Phi_{\Delta t/2}^Y\circ\Phi_{\Delta t/2}^B\circ\Phi_{\Delta t/2}^E\circ\Phi_{\Delta t}^E\circ\Phi_{\Delta t}^B\circ\Phi_{\Delta t}^Y\circ\Phi_{\Delta t}^x\circ\Phi_{\Delta t}^y\circ\Phi_{\Delta t}^z.\label{eq_Strang}
\end{align} 
Higher order splitting schemes can e.g. be found in \citep{McLachlanetal2012}.

Finally, like it was done in the previous section, we want to summarize the algorithm for numerically solving the hybrid model (\ref{eq_model_linearized}) with perpendicular perturbations only:
\begin{enumerate}
\item Create a periodic basis of Lagrange polynomials $(\varphi_i^0(z))_{i=0,\ldots,N_0-1}$ of degree $p$ on a domain $L$ discretized by $N_\text{el}$ elements using the definition of the shape functions (\ref{eq_def_Lagrange_shape}) on the reference element $I=[-1,1]$ and the formulas (\ref{eq_mapping}) for transformations on the physical domain. This results in $N_0=pN_\text{el}$.
\item Create the corresponding basis of Lagrange histopolation polynomials $(\varphi_{i+1/2}^1(z))$ $_{i=0,\ldots,N_1-1}$ using the definition of the shape functions (\ref{eq_def_Lagrange_histo}) on the reference element $I=[-1,1]$ and the formulas (\ref{eq_mapping}) for transformations on the physical domain. This results in $N_1=pN_\text{el}$.
\item Assemble the global mass matrices $\mathbb{M}^0$ and $\mathbb{M}^1$.
\item Load the initial fields $\tilde{E}_x(z,t=0)$, $\tilde{E}_y(z,t=0)$, $\tilde{B}_x(z,t=0)$, $\tilde{B}_y(z,t=0)$, $\tilde{j}_{\text{c}x}(z,t=0)$, $\tilde{j}_{\text{c}y}(z,t=0)$ and use the projectors $\Pi_0$ (\ref{eq_def_projector0}) and $\Pi_1$ (\ref{eq_def_projector1}) in order to get the initial finite element coefficients $\textbf{e}_x^0$, $\textbf{e}_y^0$, $\textbf{b}_x^0$, $\textbf{b}_y^0$, $\textbf{y}_x^0$, $\textbf{y}_y^0$.
\item Sample the initial positions $(z_k^0)_{k=1,\ldots,N_\mr{p}}$ and velocities $(v_{kx}^0,v_{ky}^0,v_{kz}^0)_{k=1,\ldots,N_\mr{p}}$ according to the sampling distribution (\ref{eq_sampling_distribution}) by using a random number generator and compute the weights $w_k=n_{\mr{h}0}L/N_\mr{p}$.
\item Assemble the matrices $\mathbb{G}$ (\ref{eq_discrete_gradient}), $\mathbb{Q}^0(\textbf{Z}^0)$ (\ref{eq_def_Q0}), $\mathbb{Q}^1(\textbf{Z}^0)$ (\ref{eq_def_Q1}), $\mathbb{B}_x(\textbf{Z}^0,\textbf{b}_x^0)$ (\ref{eq_def_Bxy}), $\mathbb{B}_y(\textbf{Z}^0,\textbf{b}_y^0)$ (\ref{eq_def_Bxy}) and $\mathbb{W}$ (\ref{eq_def_W}).
\item Start the time loop:
	\begin{enumerate}[label*=\arabic*]
	\item Apply one of the time integrators (\ref{eq_LieTrotter}) (Lie-Trotter) or (\ref{eq_Strang}) (Strang) for a time step $\Delta t$ in order to update $\textbf{e}_x^n$, $\textbf{e}_y^n$, $\textbf{b}_x^n$, $\textbf{b}_y^n$, $\textbf{y}_x^n$, $\textbf{y}_y^n$, $\textbf{Z}^n$, $\textbf{V}_x^n$, $\textbf{V}_y^n$, $\textbf{V}_z^n$ $\rightarrow$ $\textbf{e}_x^{n+1}$, $\textbf{e}_y^{n+1}$, $\textbf{b}_x^{n+1}$, $\textbf{b}_y^{n+1}$, $\textbf{y}_x^{n+1}$, $\textbf{y}_y^{n+1}$, $\textbf{Z}^{n+1}$, $\textbf{V}_x^{n+1}$, $\textbf{V}_y^{n+1}$, $\textbf{V}_z^{n+1}$. The single integrators are listed in \ref{sec_appendix}.
	\item Go to 7.1
	\end{enumerate}
\end{enumerate} 
\newpage
\section{Numerical results}
\begin{wraptable}{r}{7.4cm}
\vspace{-0.75cm}
\caption{\label{tab_parameters}Parameters for test run. In case of the structure-preserving code, the polynomial degree refers to the Lagrange polynomials that span the space $V_0$.}
\vspace{0.2cm}
\centering
\begin{tabular}{|l|l|}
\hline
\textbf{Parameter} & \textbf{Value} \\
\hline
Parallel thermal velocity $v_{\mr{th}\parallel}$ & $0.2c$ \\
\hline
Perpendicular thermal velocity $v_{\mr{th}\perp}$ & $0.53c$ \\
\hline
Density ratio $\nu_\mr{h}=n_{\mr{h}0}/n_{\mr{c}0}$ & $0.06$ \\
\hline
Cold plasma frequency $\Omega_\mr{pe}$ & $2|\Omega_\mr{ce}|$ \\
\hline 
Wavenumber of perturbation $k$ & $2|\Omega_\mr{ce}|/c$ \\
\hline
Amplitude of perturbation $a$ & $10^{-4}B_0$ \\
\hline
Length of computational domain $L$ & $2\pi/k$ \\
\hline
Number of elements $N_\mr{el}$ & 32 \\
\hline
Polynomial degree $p$ & 1 \\
\hline
Number of particles $N_\mr{p}$ & $10^6$ \\
\hline
Time step & $0.0125|\Omega_\mr{ce}|$ \\
\hline
\end{tabular}
\end{wraptable}


In this section, we present results for a single test run obtained with the two developed algorithms explained in the previous sections. For this test run, we initialize the codes as follows: We choose an anisotropic Maxwellian for the energetic electrons and perturb the $x$-component of the magnetic wave field by 
\begin{align}
\tilde{B}_x(z,t=0)=a\sin(kz),
\end{align}
in order to seed the instability for one particular $k$-mode. The amplitude $a$ is chosen with respect to the background magnetic field such that it is small enough to start in the linear phase, but large enough to reach the nonlinear phase within a reasonable simulation time. All other field quantities are initially zero, i.e. there is to electric field and cold plasma current at $t=0$. All parameters of the run are given in tab. \ref{tab_parameters}. Note that we have chosen a polynomial degree of $p=1$ in order to get basis functions which are as similar as possible for the two codes since B-splines and Lagrange polynomials are the same for this degree (see fig. \ref{fig_Bsplines_periodic}a). This difference between the two codes then is, that the magnetic field is still expressed with piecewise linear functions in the case of standard finite elements, but with piecewise constant functions in the case of geometric structure-preserving finite elements.  
\begin{figure}[!t]
\centering
\subfigure[]{%% Creator: Matplotlib, PGF backend
%%
%% To include the figure in your LaTeX document, write
%%   \input{<filename>.pgf}
%%
%% Make sure the required packages are loaded in your preamble
%%   \usepackage{pgf}
%%
%% Figures using additional raster images can only be included by \input if
%% they are in the same directory as the main LaTeX file. For loading figures
%% from other directories you can use the `import` package
%%   \usepackage{import}
%% and then include the figures with
%%   \import{<path to file>}{<filename>.pgf}
%%
%% Matplotlib used the following preamble
%%   \usepackage{fontspec}
%%   \setmainfont{DejaVu Serif}
%%   \setsansfont{DejaVu Sans}
%%   \setmonofont{DejaVu Sans Mono}
%%
\begingroup%
\makeatletter%
\begin{pgfpicture}%
\pgfpathrectangle{\pgfpointorigin}{\pgfqpoint{3.208836in}{2.339841in}}%
\pgfusepath{use as bounding box, clip}%
\begin{pgfscope}%
\pgfsetbuttcap%
\pgfsetmiterjoin%
\definecolor{currentfill}{rgb}{1.000000,1.000000,1.000000}%
\pgfsetfillcolor{currentfill}%
\pgfsetlinewidth{0.000000pt}%
\definecolor{currentstroke}{rgb}{1.000000,1.000000,1.000000}%
\pgfsetstrokecolor{currentstroke}%
\pgfsetdash{}{0pt}%
\pgfpathmoveto{\pgfqpoint{0.000000in}{0.000000in}}%
\pgfpathlineto{\pgfqpoint{3.208836in}{0.000000in}}%
\pgfpathlineto{\pgfqpoint{3.208836in}{2.339841in}}%
\pgfpathlineto{\pgfqpoint{0.000000in}{2.339841in}}%
\pgfpathclose%
\pgfusepath{fill}%
\end{pgfscope}%
\begin{pgfscope}%
\pgfsetbuttcap%
\pgfsetmiterjoin%
\definecolor{currentfill}{rgb}{1.000000,1.000000,1.000000}%
\pgfsetfillcolor{currentfill}%
\pgfsetlinewidth{0.000000pt}%
\definecolor{currentstroke}{rgb}{0.000000,0.000000,0.000000}%
\pgfsetstrokecolor{currentstroke}%
\pgfsetstrokeopacity{0.000000}%
\pgfsetdash{}{0pt}%
\pgfpathmoveto{\pgfqpoint{0.679669in}{0.526079in}}%
\pgfpathlineto{\pgfqpoint{3.004669in}{0.526079in}}%
\pgfpathlineto{\pgfqpoint{3.004669in}{2.187079in}}%
\pgfpathlineto{\pgfqpoint{0.679669in}{2.187079in}}%
\pgfpathclose%
\pgfusepath{fill}%
\end{pgfscope}%
\begin{pgfscope}%
\pgfsetbuttcap%
\pgfsetroundjoin%
\definecolor{currentfill}{rgb}{0.000000,0.000000,0.000000}%
\pgfsetfillcolor{currentfill}%
\pgfsetlinewidth{0.803000pt}%
\definecolor{currentstroke}{rgb}{0.000000,0.000000,0.000000}%
\pgfsetstrokecolor{currentstroke}%
\pgfsetdash{}{0pt}%
\pgfsys@defobject{currentmarker}{\pgfqpoint{0.000000in}{-0.048611in}}{\pgfqpoint{0.000000in}{0.000000in}}{%
\pgfpathmoveto{\pgfqpoint{0.000000in}{0.000000in}}%
\pgfpathlineto{\pgfqpoint{0.000000in}{-0.048611in}}%
\pgfusepath{stroke,fill}%
}%
\begin{pgfscope}%
\pgfsys@transformshift{0.679669in}{0.526079in}%
\pgfsys@useobject{currentmarker}{}%
\end{pgfscope}%
\end{pgfscope}%
\begin{pgfscope}%
\pgftext[x=0.679669in,y=0.428857in,,top]{\rmfamily\fontsize{10.000000}{12.000000}\selectfont \(\displaystyle 0\)}%
\end{pgfscope}%
\begin{pgfscope}%
\pgfsetbuttcap%
\pgfsetroundjoin%
\definecolor{currentfill}{rgb}{0.000000,0.000000,0.000000}%
\pgfsetfillcolor{currentfill}%
\pgfsetlinewidth{0.803000pt}%
\definecolor{currentstroke}{rgb}{0.000000,0.000000,0.000000}%
\pgfsetstrokecolor{currentstroke}%
\pgfsetdash{}{0pt}%
\pgfsys@defobject{currentmarker}{\pgfqpoint{0.000000in}{-0.048611in}}{\pgfqpoint{0.000000in}{0.000000in}}{%
\pgfpathmoveto{\pgfqpoint{0.000000in}{0.000000in}}%
\pgfpathlineto{\pgfqpoint{0.000000in}{-0.048611in}}%
\pgfusepath{stroke,fill}%
}%
\begin{pgfscope}%
\pgfsys@transformshift{1.260919in}{0.526079in}%
\pgfsys@useobject{currentmarker}{}%
\end{pgfscope}%
\end{pgfscope}%
\begin{pgfscope}%
\pgftext[x=1.260919in,y=0.428857in,,top]{\rmfamily\fontsize{10.000000}{12.000000}\selectfont \(\displaystyle 50\)}%
\end{pgfscope}%
\begin{pgfscope}%
\pgfsetbuttcap%
\pgfsetroundjoin%
\definecolor{currentfill}{rgb}{0.000000,0.000000,0.000000}%
\pgfsetfillcolor{currentfill}%
\pgfsetlinewidth{0.803000pt}%
\definecolor{currentstroke}{rgb}{0.000000,0.000000,0.000000}%
\pgfsetstrokecolor{currentstroke}%
\pgfsetdash{}{0pt}%
\pgfsys@defobject{currentmarker}{\pgfqpoint{0.000000in}{-0.048611in}}{\pgfqpoint{0.000000in}{0.000000in}}{%
\pgfpathmoveto{\pgfqpoint{0.000000in}{0.000000in}}%
\pgfpathlineto{\pgfqpoint{0.000000in}{-0.048611in}}%
\pgfusepath{stroke,fill}%
}%
\begin{pgfscope}%
\pgfsys@transformshift{1.842169in}{0.526079in}%
\pgfsys@useobject{currentmarker}{}%
\end{pgfscope}%
\end{pgfscope}%
\begin{pgfscope}%
\pgftext[x=1.842169in,y=0.428857in,,top]{\rmfamily\fontsize{10.000000}{12.000000}\selectfont \(\displaystyle 100\)}%
\end{pgfscope}%
\begin{pgfscope}%
\pgfsetbuttcap%
\pgfsetroundjoin%
\definecolor{currentfill}{rgb}{0.000000,0.000000,0.000000}%
\pgfsetfillcolor{currentfill}%
\pgfsetlinewidth{0.803000pt}%
\definecolor{currentstroke}{rgb}{0.000000,0.000000,0.000000}%
\pgfsetstrokecolor{currentstroke}%
\pgfsetdash{}{0pt}%
\pgfsys@defobject{currentmarker}{\pgfqpoint{0.000000in}{-0.048611in}}{\pgfqpoint{0.000000in}{0.000000in}}{%
\pgfpathmoveto{\pgfqpoint{0.000000in}{0.000000in}}%
\pgfpathlineto{\pgfqpoint{0.000000in}{-0.048611in}}%
\pgfusepath{stroke,fill}%
}%
\begin{pgfscope}%
\pgfsys@transformshift{2.423419in}{0.526079in}%
\pgfsys@useobject{currentmarker}{}%
\end{pgfscope}%
\end{pgfscope}%
\begin{pgfscope}%
\pgftext[x=2.423419in,y=0.428857in,,top]{\rmfamily\fontsize{10.000000}{12.000000}\selectfont \(\displaystyle 150\)}%
\end{pgfscope}%
\begin{pgfscope}%
\pgfsetbuttcap%
\pgfsetroundjoin%
\definecolor{currentfill}{rgb}{0.000000,0.000000,0.000000}%
\pgfsetfillcolor{currentfill}%
\pgfsetlinewidth{0.803000pt}%
\definecolor{currentstroke}{rgb}{0.000000,0.000000,0.000000}%
\pgfsetstrokecolor{currentstroke}%
\pgfsetdash{}{0pt}%
\pgfsys@defobject{currentmarker}{\pgfqpoint{0.000000in}{-0.048611in}}{\pgfqpoint{0.000000in}{0.000000in}}{%
\pgfpathmoveto{\pgfqpoint{0.000000in}{0.000000in}}%
\pgfpathlineto{\pgfqpoint{0.000000in}{-0.048611in}}%
\pgfusepath{stroke,fill}%
}%
\begin{pgfscope}%
\pgfsys@transformshift{3.004669in}{0.526079in}%
\pgfsys@useobject{currentmarker}{}%
\end{pgfscope}%
\end{pgfscope}%
\begin{pgfscope}%
\pgftext[x=3.004669in,y=0.428857in,,top]{\rmfamily\fontsize{10.000000}{12.000000}\selectfont \(\displaystyle 200\)}%
\end{pgfscope}%
\begin{pgfscope}%
\pgftext[x=1.842169in,y=0.238889in,,top]{\rmfamily\fontsize{10.000000}{12.000000}\selectfont \(\displaystyle t|\Omega_\mathrm{ce}|\)}%
\end{pgfscope}%
\begin{pgfscope}%
\pgfsetbuttcap%
\pgfsetroundjoin%
\definecolor{currentfill}{rgb}{0.000000,0.000000,0.000000}%
\pgfsetfillcolor{currentfill}%
\pgfsetlinewidth{0.803000pt}%
\definecolor{currentstroke}{rgb}{0.000000,0.000000,0.000000}%
\pgfsetstrokecolor{currentstroke}%
\pgfsetdash{}{0pt}%
\pgfsys@defobject{currentmarker}{\pgfqpoint{-0.048611in}{0.000000in}}{\pgfqpoint{0.000000in}{0.000000in}}{%
\pgfpathmoveto{\pgfqpoint{0.000000in}{0.000000in}}%
\pgfpathlineto{\pgfqpoint{-0.048611in}{0.000000in}}%
\pgfusepath{stroke,fill}%
}%
\begin{pgfscope}%
\pgfsys@transformshift{0.679669in}{0.526079in}%
\pgfsys@useobject{currentmarker}{}%
\end{pgfscope}%
\end{pgfscope}%
\begin{pgfscope}%
\pgftext[x=0.294444in,y=0.473318in,left,base]{\rmfamily\fontsize{10.000000}{12.000000}\selectfont \(\displaystyle 10^{-8}\)}%
\end{pgfscope}%
\begin{pgfscope}%
\pgfsetbuttcap%
\pgfsetroundjoin%
\definecolor{currentfill}{rgb}{0.000000,0.000000,0.000000}%
\pgfsetfillcolor{currentfill}%
\pgfsetlinewidth{0.803000pt}%
\definecolor{currentstroke}{rgb}{0.000000,0.000000,0.000000}%
\pgfsetstrokecolor{currentstroke}%
\pgfsetdash{}{0pt}%
\pgfsys@defobject{currentmarker}{\pgfqpoint{-0.048611in}{0.000000in}}{\pgfqpoint{0.000000in}{0.000000in}}{%
\pgfpathmoveto{\pgfqpoint{0.000000in}{0.000000in}}%
\pgfpathlineto{\pgfqpoint{-0.048611in}{0.000000in}}%
\pgfusepath{stroke,fill}%
}%
\begin{pgfscope}%
\pgfsys@transformshift{0.679669in}{1.079746in}%
\pgfsys@useobject{currentmarker}{}%
\end{pgfscope}%
\end{pgfscope}%
\begin{pgfscope}%
\pgftext[x=0.294444in,y=1.026985in,left,base]{\rmfamily\fontsize{10.000000}{12.000000}\selectfont \(\displaystyle 10^{-6}\)}%
\end{pgfscope}%
\begin{pgfscope}%
\pgfsetbuttcap%
\pgfsetroundjoin%
\definecolor{currentfill}{rgb}{0.000000,0.000000,0.000000}%
\pgfsetfillcolor{currentfill}%
\pgfsetlinewidth{0.803000pt}%
\definecolor{currentstroke}{rgb}{0.000000,0.000000,0.000000}%
\pgfsetstrokecolor{currentstroke}%
\pgfsetdash{}{0pt}%
\pgfsys@defobject{currentmarker}{\pgfqpoint{-0.048611in}{0.000000in}}{\pgfqpoint{0.000000in}{0.000000in}}{%
\pgfpathmoveto{\pgfqpoint{0.000000in}{0.000000in}}%
\pgfpathlineto{\pgfqpoint{-0.048611in}{0.000000in}}%
\pgfusepath{stroke,fill}%
}%
\begin{pgfscope}%
\pgfsys@transformshift{0.679669in}{1.633413in}%
\pgfsys@useobject{currentmarker}{}%
\end{pgfscope}%
\end{pgfscope}%
\begin{pgfscope}%
\pgftext[x=0.294444in,y=1.580651in,left,base]{\rmfamily\fontsize{10.000000}{12.000000}\selectfont \(\displaystyle 10^{-4}\)}%
\end{pgfscope}%
\begin{pgfscope}%
\pgfsetbuttcap%
\pgfsetroundjoin%
\definecolor{currentfill}{rgb}{0.000000,0.000000,0.000000}%
\pgfsetfillcolor{currentfill}%
\pgfsetlinewidth{0.803000pt}%
\definecolor{currentstroke}{rgb}{0.000000,0.000000,0.000000}%
\pgfsetstrokecolor{currentstroke}%
\pgfsetdash{}{0pt}%
\pgfsys@defobject{currentmarker}{\pgfqpoint{-0.048611in}{0.000000in}}{\pgfqpoint{0.000000in}{0.000000in}}{%
\pgfpathmoveto{\pgfqpoint{0.000000in}{0.000000in}}%
\pgfpathlineto{\pgfqpoint{-0.048611in}{0.000000in}}%
\pgfusepath{stroke,fill}%
}%
\begin{pgfscope}%
\pgfsys@transformshift{0.679669in}{2.187079in}%
\pgfsys@useobject{currentmarker}{}%
\end{pgfscope}%
\end{pgfscope}%
\begin{pgfscope}%
\pgftext[x=0.294444in,y=2.134318in,left,base]{\rmfamily\fontsize{10.000000}{12.000000}\selectfont \(\displaystyle 10^{-2}\)}%
\end{pgfscope}%
\begin{pgfscope}%
\pgftext[x=0.238889in,y=1.356579in,,bottom,rotate=90.000000]{\rmfamily\fontsize{10.000000}{12.000000}\selectfont \(\displaystyle \mathcal{E} / \mathcal{E}(0)\)}%
\end{pgfscope}%
\begin{pgfscope}%
\pgfpathrectangle{\pgfqpoint{0.679669in}{0.526079in}}{\pgfqpoint{2.325000in}{1.661000in}} %
\pgfusepath{clip}%
\pgfsetrectcap%
\pgfsetroundjoin%
\pgfsetlinewidth{1.003750pt}%
\definecolor{currentstroke}{rgb}{1.000000,0.549020,0.000000}%
\pgfsetstrokecolor{currentstroke}%
\pgfsetdash{}{0pt}%
\pgfpathmoveto{\pgfqpoint{0.679669in}{0.675416in}}%
\pgfpathlineto{\pgfqpoint{0.680105in}{0.677265in}}%
\pgfpathlineto{\pgfqpoint{0.680832in}{0.718864in}}%
\pgfpathlineto{\pgfqpoint{0.684029in}{0.901156in}}%
\pgfpathlineto{\pgfqpoint{0.686935in}{0.956631in}}%
\pgfpathlineto{\pgfqpoint{0.687661in}{0.950870in}}%
\pgfpathlineto{\pgfqpoint{0.690277in}{0.934083in}}%
\pgfpathlineto{\pgfqpoint{0.690858in}{0.930261in}}%
\pgfpathlineto{\pgfqpoint{0.691585in}{0.935932in}}%
\pgfpathlineto{\pgfqpoint{0.692457in}{0.945005in}}%
\pgfpathlineto{\pgfqpoint{0.693474in}{0.941107in}}%
\pgfpathlineto{\pgfqpoint{0.694200in}{0.949735in}}%
\pgfpathlineto{\pgfqpoint{0.698124in}{1.002568in}}%
\pgfpathlineto{\pgfqpoint{0.698414in}{1.000484in}}%
\pgfpathlineto{\pgfqpoint{0.699722in}{0.950152in}}%
\pgfpathlineto{\pgfqpoint{0.700449in}{0.934867in}}%
\pgfpathlineto{\pgfqpoint{0.701175in}{0.945716in}}%
\pgfpathlineto{\pgfqpoint{0.704227in}{0.981246in}}%
\pgfpathlineto{\pgfqpoint{0.704808in}{0.976493in}}%
\pgfpathlineto{\pgfqpoint{0.706407in}{0.914210in}}%
\pgfpathlineto{\pgfqpoint{0.706843in}{0.906521in}}%
\pgfpathlineto{\pgfqpoint{0.707424in}{0.921357in}}%
\pgfpathlineto{\pgfqpoint{0.709168in}{0.976910in}}%
\pgfpathlineto{\pgfqpoint{0.710039in}{0.976392in}}%
\pgfpathlineto{\pgfqpoint{0.711202in}{0.987362in}}%
\pgfpathlineto{\pgfqpoint{0.711783in}{0.979023in}}%
\pgfpathlineto{\pgfqpoint{0.713382in}{0.926780in}}%
\pgfpathlineto{\pgfqpoint{0.714254in}{0.946929in}}%
\pgfpathlineto{\pgfqpoint{0.716724in}{0.992692in}}%
\pgfpathlineto{\pgfqpoint{0.717014in}{0.991167in}}%
\pgfpathlineto{\pgfqpoint{0.718468in}{0.959381in}}%
\pgfpathlineto{\pgfqpoint{0.719921in}{0.942285in}}%
\pgfpathlineto{\pgfqpoint{0.720357in}{0.946966in}}%
\pgfpathlineto{\pgfqpoint{0.722246in}{0.997221in}}%
\pgfpathlineto{\pgfqpoint{0.723408in}{0.988971in}}%
\pgfpathlineto{\pgfqpoint{0.724425in}{0.961586in}}%
\pgfpathlineto{\pgfqpoint{0.725443in}{0.924037in}}%
\pgfpathlineto{\pgfqpoint{0.726169in}{0.950832in}}%
\pgfpathlineto{\pgfqpoint{0.727332in}{0.985679in}}%
\pgfpathlineto{\pgfqpoint{0.728058in}{0.976213in}}%
\pgfpathlineto{\pgfqpoint{0.730819in}{0.938864in}}%
\pgfpathlineto{\pgfqpoint{0.731255in}{0.944333in}}%
\pgfpathlineto{\pgfqpoint{0.734016in}{1.002044in}}%
\pgfpathlineto{\pgfqpoint{0.734452in}{0.998220in}}%
\pgfpathlineto{\pgfqpoint{0.735760in}{0.970034in}}%
\pgfpathlineto{\pgfqpoint{0.737068in}{0.973342in}}%
\pgfpathlineto{\pgfqpoint{0.737213in}{0.973164in}}%
\pgfpathlineto{\pgfqpoint{0.737504in}{0.974887in}}%
\pgfpathlineto{\pgfqpoint{0.739538in}{1.027736in}}%
\pgfpathlineto{\pgfqpoint{0.740991in}{1.010225in}}%
\pgfpathlineto{\pgfqpoint{0.742154in}{1.004696in}}%
\pgfpathlineto{\pgfqpoint{0.742735in}{1.007763in}}%
\pgfpathlineto{\pgfqpoint{0.744624in}{1.041063in}}%
\pgfpathlineto{\pgfqpoint{0.745932in}{1.032653in}}%
\pgfpathlineto{\pgfqpoint{0.746658in}{1.027551in}}%
\pgfpathlineto{\pgfqpoint{0.748111in}{0.993290in}}%
\pgfpathlineto{\pgfqpoint{0.748983in}{1.010454in}}%
\pgfpathlineto{\pgfqpoint{0.751744in}{1.057791in}}%
\pgfpathlineto{\pgfqpoint{0.752180in}{1.054150in}}%
\pgfpathlineto{\pgfqpoint{0.755232in}{1.012353in}}%
\pgfpathlineto{\pgfqpoint{0.755668in}{1.019588in}}%
\pgfpathlineto{\pgfqpoint{0.757266in}{1.060792in}}%
\pgfpathlineto{\pgfqpoint{0.758138in}{1.054471in}}%
\pgfpathlineto{\pgfqpoint{0.760027in}{1.034419in}}%
\pgfpathlineto{\pgfqpoint{0.760754in}{1.042912in}}%
\pgfpathlineto{\pgfqpoint{0.763950in}{1.085926in}}%
\pgfpathlineto{\pgfqpoint{0.764386in}{1.081689in}}%
\pgfpathlineto{\pgfqpoint{0.766275in}{1.038802in}}%
\pgfpathlineto{\pgfqpoint{0.767438in}{1.050670in}}%
\pgfpathlineto{\pgfqpoint{0.769763in}{1.099882in}}%
\pgfpathlineto{\pgfqpoint{0.771216in}{1.089595in}}%
\pgfpathlineto{\pgfqpoint{0.772814in}{1.032078in}}%
\pgfpathlineto{\pgfqpoint{0.773832in}{1.062539in}}%
\pgfpathlineto{\pgfqpoint{0.776302in}{1.099452in}}%
\pgfpathlineto{\pgfqpoint{0.776447in}{1.099031in}}%
\pgfpathlineto{\pgfqpoint{0.777319in}{1.081257in}}%
\pgfpathlineto{\pgfqpoint{0.778918in}{1.037575in}}%
\pgfpathlineto{\pgfqpoint{0.779644in}{1.055148in}}%
\pgfpathlineto{\pgfqpoint{0.781243in}{1.108508in}}%
\pgfpathlineto{\pgfqpoint{0.781969in}{1.098138in}}%
\pgfpathlineto{\pgfqpoint{0.784585in}{1.041646in}}%
\pgfpathlineto{\pgfqpoint{0.785166in}{1.049458in}}%
\pgfpathlineto{\pgfqpoint{0.786910in}{1.122744in}}%
\pgfpathlineto{\pgfqpoint{0.787927in}{1.101521in}}%
\pgfpathlineto{\pgfqpoint{0.789961in}{1.031288in}}%
\pgfpathlineto{\pgfqpoint{0.790688in}{1.058656in}}%
\pgfpathlineto{\pgfqpoint{0.793158in}{1.126792in}}%
\pgfpathlineto{\pgfqpoint{0.793304in}{1.126111in}}%
\pgfpathlineto{\pgfqpoint{0.794175in}{1.099101in}}%
\pgfpathlineto{\pgfqpoint{0.795483in}{1.050429in}}%
\pgfpathlineto{\pgfqpoint{0.796210in}{1.067908in}}%
\pgfpathlineto{\pgfqpoint{0.798825in}{1.133687in}}%
\pgfpathlineto{\pgfqpoint{0.799261in}{1.131081in}}%
\pgfpathlineto{\pgfqpoint{0.800714in}{1.088817in}}%
\pgfpathlineto{\pgfqpoint{0.801732in}{1.065011in}}%
\pgfpathlineto{\pgfqpoint{0.802313in}{1.081793in}}%
\pgfpathlineto{\pgfqpoint{0.804057in}{1.151675in}}%
\pgfpathlineto{\pgfqpoint{0.804783in}{1.140746in}}%
\pgfpathlineto{\pgfqpoint{0.807689in}{1.085873in}}%
\pgfpathlineto{\pgfqpoint{0.808125in}{1.093552in}}%
\pgfpathlineto{\pgfqpoint{0.810305in}{1.172009in}}%
\pgfpathlineto{\pgfqpoint{0.811032in}{1.155286in}}%
\pgfpathlineto{\pgfqpoint{0.812485in}{1.085960in}}%
\pgfpathlineto{\pgfqpoint{0.813502in}{1.108815in}}%
\pgfpathlineto{\pgfqpoint{0.816554in}{1.171203in}}%
\pgfpathlineto{\pgfqpoint{0.816989in}{1.164591in}}%
\pgfpathlineto{\pgfqpoint{0.818733in}{1.073578in}}%
\pgfpathlineto{\pgfqpoint{0.819750in}{1.118062in}}%
\pgfpathlineto{\pgfqpoint{0.822366in}{1.183197in}}%
\pgfpathlineto{\pgfqpoint{0.822511in}{1.183275in}}%
\pgfpathlineto{\pgfqpoint{0.822657in}{1.182748in}}%
\pgfpathlineto{\pgfqpoint{0.823383in}{1.168148in}}%
\pgfpathlineto{\pgfqpoint{0.825127in}{1.079950in}}%
\pgfpathlineto{\pgfqpoint{0.825999in}{1.124595in}}%
\pgfpathlineto{\pgfqpoint{0.827888in}{1.177807in}}%
\pgfpathlineto{\pgfqpoint{0.828324in}{1.175657in}}%
\pgfpathlineto{\pgfqpoint{0.829777in}{1.145891in}}%
\pgfpathlineto{\pgfqpoint{0.831230in}{1.109147in}}%
\pgfpathlineto{\pgfqpoint{0.831811in}{1.123982in}}%
\pgfpathlineto{\pgfqpoint{0.833410in}{1.178456in}}%
\pgfpathlineto{\pgfqpoint{0.834136in}{1.168416in}}%
\pgfpathlineto{\pgfqpoint{0.837043in}{1.122414in}}%
\pgfpathlineto{\pgfqpoint{0.837333in}{1.123124in}}%
\pgfpathlineto{\pgfqpoint{0.838205in}{1.145320in}}%
\pgfpathlineto{\pgfqpoint{0.839513in}{1.183594in}}%
\pgfpathlineto{\pgfqpoint{0.840239in}{1.171526in}}%
\pgfpathlineto{\pgfqpoint{0.841983in}{1.120946in}}%
\pgfpathlineto{\pgfqpoint{0.842855in}{1.133699in}}%
\pgfpathlineto{\pgfqpoint{0.845907in}{1.174913in}}%
\pgfpathlineto{\pgfqpoint{0.846197in}{1.173940in}}%
\pgfpathlineto{\pgfqpoint{0.847214in}{1.146655in}}%
\pgfpathlineto{\pgfqpoint{0.847941in}{1.129105in}}%
\pgfpathlineto{\pgfqpoint{0.848668in}{1.142218in}}%
\pgfpathlineto{\pgfqpoint{0.851283in}{1.183050in}}%
\pgfpathlineto{\pgfqpoint{0.851719in}{1.183633in}}%
\pgfpathlineto{\pgfqpoint{0.852155in}{1.182294in}}%
\pgfpathlineto{\pgfqpoint{0.853172in}{1.162350in}}%
\pgfpathlineto{\pgfqpoint{0.854044in}{1.144156in}}%
\pgfpathlineto{\pgfqpoint{0.854771in}{1.157574in}}%
\pgfpathlineto{\pgfqpoint{0.856660in}{1.201170in}}%
\pgfpathlineto{\pgfqpoint{0.857241in}{1.194530in}}%
\pgfpathlineto{\pgfqpoint{0.860293in}{1.150740in}}%
\pgfpathlineto{\pgfqpoint{0.860583in}{1.151967in}}%
\pgfpathlineto{\pgfqpoint{0.862472in}{1.197136in}}%
\pgfpathlineto{\pgfqpoint{0.863489in}{1.179543in}}%
\pgfpathlineto{\pgfqpoint{0.865524in}{1.143261in}}%
\pgfpathlineto{\pgfqpoint{0.866105in}{1.144269in}}%
\pgfpathlineto{\pgfqpoint{0.866977in}{1.156235in}}%
\pgfpathlineto{\pgfqpoint{0.869011in}{1.199098in}}%
\pgfpathlineto{\pgfqpoint{0.869738in}{1.191145in}}%
\pgfpathlineto{\pgfqpoint{0.871336in}{1.142598in}}%
\pgfpathlineto{\pgfqpoint{0.872208in}{1.158063in}}%
\pgfpathlineto{\pgfqpoint{0.875114in}{1.207704in}}%
\pgfpathlineto{\pgfqpoint{0.875696in}{1.205359in}}%
\pgfpathlineto{\pgfqpoint{0.876858in}{1.175322in}}%
\pgfpathlineto{\pgfqpoint{0.877585in}{1.156505in}}%
\pgfpathlineto{\pgfqpoint{0.878311in}{1.172350in}}%
\pgfpathlineto{\pgfqpoint{0.880055in}{1.216946in}}%
\pgfpathlineto{\pgfqpoint{0.880782in}{1.212919in}}%
\pgfpathlineto{\pgfqpoint{0.882816in}{1.191430in}}%
\pgfpathlineto{\pgfqpoint{0.883688in}{1.175060in}}%
\pgfpathlineto{\pgfqpoint{0.884414in}{1.185523in}}%
\pgfpathlineto{\pgfqpoint{0.886013in}{1.214397in}}%
\pgfpathlineto{\pgfqpoint{0.886594in}{1.209020in}}%
\pgfpathlineto{\pgfqpoint{0.890227in}{1.169995in}}%
\pgfpathlineto{\pgfqpoint{0.890518in}{1.171612in}}%
\pgfpathlineto{\pgfqpoint{0.892261in}{1.215841in}}%
\pgfpathlineto{\pgfqpoint{0.893133in}{1.195688in}}%
\pgfpathlineto{\pgfqpoint{0.894150in}{1.173273in}}%
\pgfpathlineto{\pgfqpoint{0.895022in}{1.181951in}}%
\pgfpathlineto{\pgfqpoint{0.898510in}{1.235095in}}%
\pgfpathlineto{\pgfqpoint{0.899236in}{1.223534in}}%
\pgfpathlineto{\pgfqpoint{0.900544in}{1.175228in}}%
\pgfpathlineto{\pgfqpoint{0.901416in}{1.198905in}}%
\pgfpathlineto{\pgfqpoint{0.903886in}{1.235189in}}%
\pgfpathlineto{\pgfqpoint{0.904758in}{1.240606in}}%
\pgfpathlineto{\pgfqpoint{0.905339in}{1.234494in}}%
\pgfpathlineto{\pgfqpoint{0.907083in}{1.175910in}}%
\pgfpathlineto{\pgfqpoint{0.907955in}{1.198495in}}%
\pgfpathlineto{\pgfqpoint{0.909554in}{1.240160in}}%
\pgfpathlineto{\pgfqpoint{0.910280in}{1.235691in}}%
\pgfpathlineto{\pgfqpoint{0.911297in}{1.235947in}}%
\pgfpathlineto{\pgfqpoint{0.912314in}{1.220315in}}%
\pgfpathlineto{\pgfqpoint{0.913477in}{1.185823in}}%
\pgfpathlineto{\pgfqpoint{0.914204in}{1.202964in}}%
\pgfpathlineto{\pgfqpoint{0.915802in}{1.245928in}}%
\pgfpathlineto{\pgfqpoint{0.916529in}{1.235725in}}%
\pgfpathlineto{\pgfqpoint{0.919871in}{1.181764in}}%
\pgfpathlineto{\pgfqpoint{0.920016in}{1.181808in}}%
\pgfpathlineto{\pgfqpoint{0.920743in}{1.203510in}}%
\pgfpathlineto{\pgfqpoint{0.922050in}{1.247244in}}%
\pgfpathlineto{\pgfqpoint{0.922777in}{1.232888in}}%
\pgfpathlineto{\pgfqpoint{0.924230in}{1.190405in}}%
\pgfpathlineto{\pgfqpoint{0.925102in}{1.200206in}}%
\pgfpathlineto{\pgfqpoint{0.928589in}{1.260816in}}%
\pgfpathlineto{\pgfqpoint{0.929461in}{1.241334in}}%
\pgfpathlineto{\pgfqpoint{0.930769in}{1.198512in}}%
\pgfpathlineto{\pgfqpoint{0.931496in}{1.213056in}}%
\pgfpathlineto{\pgfqpoint{0.934838in}{1.261412in}}%
\pgfpathlineto{\pgfqpoint{0.935419in}{1.254906in}}%
\pgfpathlineto{\pgfqpoint{0.937308in}{1.197169in}}%
\pgfpathlineto{\pgfqpoint{0.938180in}{1.218204in}}%
\pgfpathlineto{\pgfqpoint{0.940796in}{1.251354in}}%
\pgfpathlineto{\pgfqpoint{0.941086in}{1.251991in}}%
\pgfpathlineto{\pgfqpoint{0.941522in}{1.249196in}}%
\pgfpathlineto{\pgfqpoint{0.943411in}{1.197695in}}%
\pgfpathlineto{\pgfqpoint{0.944283in}{1.223377in}}%
\pgfpathlineto{\pgfqpoint{0.945736in}{1.263937in}}%
\pgfpathlineto{\pgfqpoint{0.946463in}{1.258128in}}%
\pgfpathlineto{\pgfqpoint{0.949660in}{1.210143in}}%
\pgfpathlineto{\pgfqpoint{0.950241in}{1.221471in}}%
\pgfpathlineto{\pgfqpoint{0.951839in}{1.271473in}}%
\pgfpathlineto{\pgfqpoint{0.952566in}{1.262184in}}%
\pgfpathlineto{\pgfqpoint{0.955618in}{1.226733in}}%
\pgfpathlineto{\pgfqpoint{0.955763in}{1.226529in}}%
\pgfpathlineto{\pgfqpoint{0.956054in}{1.227381in}}%
\pgfpathlineto{\pgfqpoint{0.957071in}{1.250086in}}%
\pgfpathlineto{\pgfqpoint{0.958088in}{1.272680in}}%
\pgfpathlineto{\pgfqpoint{0.958814in}{1.262919in}}%
\pgfpathlineto{\pgfqpoint{0.960413in}{1.220237in}}%
\pgfpathlineto{\pgfqpoint{0.961285in}{1.232818in}}%
\pgfpathlineto{\pgfqpoint{0.964336in}{1.281489in}}%
\pgfpathlineto{\pgfqpoint{0.964918in}{1.273547in}}%
\pgfpathlineto{\pgfqpoint{0.966661in}{1.221395in}}%
\pgfpathlineto{\pgfqpoint{0.967533in}{1.239494in}}%
\pgfpathlineto{\pgfqpoint{0.970004in}{1.271025in}}%
\pgfpathlineto{\pgfqpoint{0.970585in}{1.273874in}}%
\pgfpathlineto{\pgfqpoint{0.971166in}{1.268021in}}%
\pgfpathlineto{\pgfqpoint{0.972910in}{1.217331in}}%
\pgfpathlineto{\pgfqpoint{0.973782in}{1.240644in}}%
\pgfpathlineto{\pgfqpoint{0.975380in}{1.276271in}}%
\pgfpathlineto{\pgfqpoint{0.976107in}{1.273345in}}%
\pgfpathlineto{\pgfqpoint{0.977850in}{1.256925in}}%
\pgfpathlineto{\pgfqpoint{0.979304in}{1.223510in}}%
\pgfpathlineto{\pgfqpoint{0.980030in}{1.242169in}}%
\pgfpathlineto{\pgfqpoint{0.981338in}{1.281633in}}%
\pgfpathlineto{\pgfqpoint{0.982210in}{1.271499in}}%
\pgfpathlineto{\pgfqpoint{0.984680in}{1.253422in}}%
\pgfpathlineto{\pgfqpoint{0.985552in}{1.247706in}}%
\pgfpathlineto{\pgfqpoint{0.986133in}{1.252377in}}%
\pgfpathlineto{\pgfqpoint{0.987877in}{1.291730in}}%
\pgfpathlineto{\pgfqpoint{0.988749in}{1.277225in}}%
\pgfpathlineto{\pgfqpoint{0.990057in}{1.251179in}}%
\pgfpathlineto{\pgfqpoint{0.990783in}{1.257468in}}%
\pgfpathlineto{\pgfqpoint{0.994125in}{1.289969in}}%
\pgfpathlineto{\pgfqpoint{0.994852in}{1.280868in}}%
\pgfpathlineto{\pgfqpoint{0.996305in}{1.240790in}}%
\pgfpathlineto{\pgfqpoint{0.997032in}{1.254760in}}%
\pgfpathlineto{\pgfqpoint{0.998194in}{1.276684in}}%
\pgfpathlineto{\pgfqpoint{0.999066in}{1.273405in}}%
\pgfpathlineto{\pgfqpoint{0.999502in}{1.275740in}}%
\pgfpathlineto{\pgfqpoint{1.000664in}{1.289549in}}%
\pgfpathlineto{\pgfqpoint{1.001246in}{1.281967in}}%
\pgfpathlineto{\pgfqpoint{1.002554in}{1.247670in}}%
\pgfpathlineto{\pgfqpoint{1.003425in}{1.263366in}}%
\pgfpathlineto{\pgfqpoint{1.004879in}{1.286974in}}%
\pgfpathlineto{\pgfqpoint{1.005605in}{1.282896in}}%
\pgfpathlineto{\pgfqpoint{1.009093in}{1.250591in}}%
\pgfpathlineto{\pgfqpoint{1.009529in}{1.258890in}}%
\pgfpathlineto{\pgfqpoint{1.011127in}{1.299931in}}%
\pgfpathlineto{\pgfqpoint{1.011854in}{1.290797in}}%
\pgfpathlineto{\pgfqpoint{1.012871in}{1.278984in}}%
\pgfpathlineto{\pgfqpoint{1.013888in}{1.280292in}}%
\pgfpathlineto{\pgfqpoint{1.015196in}{1.270426in}}%
\pgfpathlineto{\pgfqpoint{1.015777in}{1.275214in}}%
\pgfpathlineto{\pgfqpoint{1.017666in}{1.306714in}}%
\pgfpathlineto{\pgfqpoint{1.018393in}{1.293565in}}%
\pgfpathlineto{\pgfqpoint{1.019555in}{1.265858in}}%
\pgfpathlineto{\pgfqpoint{1.020282in}{1.276346in}}%
\pgfpathlineto{\pgfqpoint{1.021008in}{1.283767in}}%
\pgfpathlineto{\pgfqpoint{1.022171in}{1.283049in}}%
\pgfpathlineto{\pgfqpoint{1.023769in}{1.304743in}}%
\pgfpathlineto{\pgfqpoint{1.024641in}{1.294846in}}%
\pgfpathlineto{\pgfqpoint{1.026094in}{1.266233in}}%
\pgfpathlineto{\pgfqpoint{1.026821in}{1.276091in}}%
\pgfpathlineto{\pgfqpoint{1.028274in}{1.297439in}}%
\pgfpathlineto{\pgfqpoint{1.029146in}{1.295553in}}%
\pgfpathlineto{\pgfqpoint{1.030308in}{1.304139in}}%
\pgfpathlineto{\pgfqpoint{1.030889in}{1.298408in}}%
\pgfpathlineto{\pgfqpoint{1.032197in}{1.265719in}}%
\pgfpathlineto{\pgfqpoint{1.032924in}{1.280466in}}%
\pgfpathlineto{\pgfqpoint{1.034377in}{1.307826in}}%
\pgfpathlineto{\pgfqpoint{1.035104in}{1.305637in}}%
\pgfpathlineto{\pgfqpoint{1.038446in}{1.283400in}}%
\pgfpathlineto{\pgfqpoint{1.038882in}{1.280439in}}%
\pgfpathlineto{\pgfqpoint{1.039318in}{1.286910in}}%
\pgfpathlineto{\pgfqpoint{1.040771in}{1.324770in}}%
\pgfpathlineto{\pgfqpoint{1.041643in}{1.313040in}}%
\pgfpathlineto{\pgfqpoint{1.043968in}{1.294105in}}%
\pgfpathlineto{\pgfqpoint{1.044694in}{1.289594in}}%
\pgfpathlineto{\pgfqpoint{1.045275in}{1.293773in}}%
\pgfpathlineto{\pgfqpoint{1.047164in}{1.322234in}}%
\pgfpathlineto{\pgfqpoint{1.047891in}{1.313203in}}%
\pgfpathlineto{\pgfqpoint{1.049344in}{1.284143in}}%
\pgfpathlineto{\pgfqpoint{1.050071in}{1.293628in}}%
\pgfpathlineto{\pgfqpoint{1.053268in}{1.324904in}}%
\pgfpathlineto{\pgfqpoint{1.053413in}{1.325244in}}%
\pgfpathlineto{\pgfqpoint{1.053704in}{1.323413in}}%
\pgfpathlineto{\pgfqpoint{1.055593in}{1.286685in}}%
\pgfpathlineto{\pgfqpoint{1.056610in}{1.302447in}}%
\pgfpathlineto{\pgfqpoint{1.057918in}{1.320849in}}%
\pgfpathlineto{\pgfqpoint{1.058644in}{1.316878in}}%
\pgfpathlineto{\pgfqpoint{1.059080in}{1.315455in}}%
\pgfpathlineto{\pgfqpoint{1.059952in}{1.317801in}}%
\pgfpathlineto{\pgfqpoint{1.060388in}{1.315744in}}%
\pgfpathlineto{\pgfqpoint{1.061841in}{1.291295in}}%
\pgfpathlineto{\pgfqpoint{1.062568in}{1.303916in}}%
\pgfpathlineto{\pgfqpoint{1.064021in}{1.333953in}}%
\pgfpathlineto{\pgfqpoint{1.064747in}{1.327319in}}%
\pgfpathlineto{\pgfqpoint{1.067654in}{1.309412in}}%
\pgfpathlineto{\pgfqpoint{1.068235in}{1.307250in}}%
\pgfpathlineto{\pgfqpoint{1.068816in}{1.310462in}}%
\pgfpathlineto{\pgfqpoint{1.070414in}{1.339757in}}%
\pgfpathlineto{\pgfqpoint{1.071286in}{1.329407in}}%
\pgfpathlineto{\pgfqpoint{1.072449in}{1.315158in}}%
\pgfpathlineto{\pgfqpoint{1.073321in}{1.317211in}}%
\pgfpathlineto{\pgfqpoint{1.073902in}{1.315454in}}%
\pgfpathlineto{\pgfqpoint{1.074629in}{1.312339in}}%
\pgfpathlineto{\pgfqpoint{1.075210in}{1.314857in}}%
\pgfpathlineto{\pgfqpoint{1.076808in}{1.330179in}}%
\pgfpathlineto{\pgfqpoint{1.077535in}{1.323150in}}%
\pgfpathlineto{\pgfqpoint{1.078697in}{1.307150in}}%
\pgfpathlineto{\pgfqpoint{1.079424in}{1.314391in}}%
\pgfpathlineto{\pgfqpoint{1.081894in}{1.333321in}}%
\pgfpathlineto{\pgfqpoint{1.083057in}{1.339664in}}%
\pgfpathlineto{\pgfqpoint{1.083638in}{1.334973in}}%
\pgfpathlineto{\pgfqpoint{1.085091in}{1.314866in}}%
\pgfpathlineto{\pgfqpoint{1.085818in}{1.322170in}}%
\pgfpathlineto{\pgfqpoint{1.087416in}{1.339848in}}%
\pgfpathlineto{\pgfqpoint{1.088143in}{1.336676in}}%
\pgfpathlineto{\pgfqpoint{1.089014in}{1.333068in}}%
\pgfpathlineto{\pgfqpoint{1.089886in}{1.334789in}}%
\pgfpathlineto{\pgfqpoint{1.090468in}{1.332237in}}%
\pgfpathlineto{\pgfqpoint{1.091339in}{1.324987in}}%
\pgfpathlineto{\pgfqpoint{1.091921in}{1.329650in}}%
\pgfpathlineto{\pgfqpoint{1.093374in}{1.348637in}}%
\pgfpathlineto{\pgfqpoint{1.094100in}{1.344378in}}%
\pgfpathlineto{\pgfqpoint{1.095699in}{1.336017in}}%
\pgfpathlineto{\pgfqpoint{1.096425in}{1.337190in}}%
\pgfpathlineto{\pgfqpoint{1.097007in}{1.337700in}}%
\pgfpathlineto{\pgfqpoint{1.097588in}{1.336317in}}%
\pgfpathlineto{\pgfqpoint{1.098024in}{1.335495in}}%
\pgfpathlineto{\pgfqpoint{1.098605in}{1.337268in}}%
\pgfpathlineto{\pgfqpoint{1.100058in}{1.347166in}}%
\pgfpathlineto{\pgfqpoint{1.100785in}{1.343697in}}%
\pgfpathlineto{\pgfqpoint{1.101947in}{1.333934in}}%
\pgfpathlineto{\pgfqpoint{1.102674in}{1.338523in}}%
\pgfpathlineto{\pgfqpoint{1.103255in}{1.341447in}}%
\pgfpathlineto{\pgfqpoint{1.104272in}{1.339614in}}%
\pgfpathlineto{\pgfqpoint{1.105289in}{1.345491in}}%
\pgfpathlineto{\pgfqpoint{1.106016in}{1.347412in}}%
\pgfpathlineto{\pgfqpoint{1.106888in}{1.346434in}}%
\pgfpathlineto{\pgfqpoint{1.108050in}{1.339996in}}%
\pgfpathlineto{\pgfqpoint{1.108486in}{1.338340in}}%
\pgfpathlineto{\pgfqpoint{1.109068in}{1.341517in}}%
\pgfpathlineto{\pgfqpoint{1.111538in}{1.351751in}}%
\pgfpathlineto{\pgfqpoint{1.112700in}{1.356100in}}%
\pgfpathlineto{\pgfqpoint{1.113427in}{1.353798in}}%
\pgfpathlineto{\pgfqpoint{1.114589in}{1.347639in}}%
\pgfpathlineto{\pgfqpoint{1.115171in}{1.351159in}}%
\pgfpathlineto{\pgfqpoint{1.117786in}{1.362759in}}%
\pgfpathlineto{\pgfqpoint{1.118513in}{1.361787in}}%
\pgfpathlineto{\pgfqpoint{1.121129in}{1.353791in}}%
\pgfpathlineto{\pgfqpoint{1.121564in}{1.355689in}}%
\pgfpathlineto{\pgfqpoint{1.123163in}{1.372771in}}%
\pgfpathlineto{\pgfqpoint{1.124035in}{1.367962in}}%
\pgfpathlineto{\pgfqpoint{1.126360in}{1.362813in}}%
\pgfpathlineto{\pgfqpoint{1.127377in}{1.361075in}}%
\pgfpathlineto{\pgfqpoint{1.127958in}{1.362602in}}%
\pgfpathlineto{\pgfqpoint{1.129702in}{1.372900in}}%
\pgfpathlineto{\pgfqpoint{1.130574in}{1.368538in}}%
\pgfpathlineto{\pgfqpoint{1.131591in}{1.360862in}}%
\pgfpathlineto{\pgfqpoint{1.132318in}{1.364256in}}%
\pgfpathlineto{\pgfqpoint{1.135514in}{1.379043in}}%
\pgfpathlineto{\pgfqpoint{1.135805in}{1.379470in}}%
\pgfpathlineto{\pgfqpoint{1.136241in}{1.377483in}}%
\pgfpathlineto{\pgfqpoint{1.137549in}{1.369272in}}%
\pgfpathlineto{\pgfqpoint{1.138421in}{1.371902in}}%
\pgfpathlineto{\pgfqpoint{1.140455in}{1.384577in}}%
\pgfpathlineto{\pgfqpoint{1.142780in}{1.383933in}}%
\pgfpathlineto{\pgfqpoint{1.144088in}{1.376101in}}%
\pgfpathlineto{\pgfqpoint{1.144814in}{1.381129in}}%
\pgfpathlineto{\pgfqpoint{1.146413in}{1.392332in}}%
\pgfpathlineto{\pgfqpoint{1.146994in}{1.390923in}}%
\pgfpathlineto{\pgfqpoint{1.148883in}{1.386536in}}%
\pgfpathlineto{\pgfqpoint{1.149464in}{1.386809in}}%
\pgfpathlineto{\pgfqpoint{1.151499in}{1.389901in}}%
\pgfpathlineto{\pgfqpoint{1.153097in}{1.400909in}}%
\pgfpathlineto{\pgfqpoint{1.153824in}{1.395017in}}%
\pgfpathlineto{\pgfqpoint{1.154550in}{1.389330in}}%
\pgfpathlineto{\pgfqpoint{1.155422in}{1.393050in}}%
\pgfpathlineto{\pgfqpoint{1.159055in}{1.404015in}}%
\pgfpathlineto{\pgfqpoint{1.159491in}{1.404096in}}%
\pgfpathlineto{\pgfqpoint{1.159927in}{1.402807in}}%
\pgfpathlineto{\pgfqpoint{1.161235in}{1.394664in}}%
\pgfpathlineto{\pgfqpoint{1.161961in}{1.398112in}}%
\pgfpathlineto{\pgfqpoint{1.163414in}{1.407937in}}%
\pgfpathlineto{\pgfqpoint{1.164286in}{1.406512in}}%
\pgfpathlineto{\pgfqpoint{1.164577in}{1.406348in}}%
\pgfpathlineto{\pgfqpoint{1.165158in}{1.407678in}}%
\pgfpathlineto{\pgfqpoint{1.165885in}{1.409176in}}%
\pgfpathlineto{\pgfqpoint{1.166466in}{1.407180in}}%
\pgfpathlineto{\pgfqpoint{1.167338in}{1.403604in}}%
\pgfpathlineto{\pgfqpoint{1.168064in}{1.405904in}}%
\pgfpathlineto{\pgfqpoint{1.170099in}{1.416205in}}%
\pgfpathlineto{\pgfqpoint{1.170825in}{1.413884in}}%
\pgfpathlineto{\pgfqpoint{1.171697in}{1.410154in}}%
\pgfpathlineto{\pgfqpoint{1.172569in}{1.412651in}}%
\pgfpathlineto{\pgfqpoint{1.175911in}{1.424200in}}%
\pgfpathlineto{\pgfqpoint{1.176783in}{1.421361in}}%
\pgfpathlineto{\pgfqpoint{1.178382in}{1.415658in}}%
\pgfpathlineto{\pgfqpoint{1.178963in}{1.416802in}}%
\pgfpathlineto{\pgfqpoint{1.182450in}{1.427215in}}%
\pgfpathlineto{\pgfqpoint{1.182886in}{1.426416in}}%
\pgfpathlineto{\pgfqpoint{1.184485in}{1.417177in}}%
\pgfpathlineto{\pgfqpoint{1.185211in}{1.421086in}}%
\pgfpathlineto{\pgfqpoint{1.187827in}{1.430772in}}%
\pgfpathlineto{\pgfqpoint{1.188263in}{1.430672in}}%
\pgfpathlineto{\pgfqpoint{1.188554in}{1.429940in}}%
\pgfpathlineto{\pgfqpoint{1.190733in}{1.425294in}}%
\pgfpathlineto{\pgfqpoint{1.191169in}{1.426101in}}%
\pgfpathlineto{\pgfqpoint{1.193058in}{1.438645in}}%
\pgfpathlineto{\pgfqpoint{1.194802in}{1.436178in}}%
\pgfpathlineto{\pgfqpoint{1.196255in}{1.435203in}}%
\pgfpathlineto{\pgfqpoint{1.196982in}{1.436387in}}%
\pgfpathlineto{\pgfqpoint{1.199452in}{1.446415in}}%
\pgfpathlineto{\pgfqpoint{1.200324in}{1.444157in}}%
\pgfpathlineto{\pgfqpoint{1.201486in}{1.441019in}}%
\pgfpathlineto{\pgfqpoint{1.202213in}{1.442349in}}%
\pgfpathlineto{\pgfqpoint{1.205555in}{1.453607in}}%
\pgfpathlineto{\pgfqpoint{1.206282in}{1.450975in}}%
\pgfpathlineto{\pgfqpoint{1.208025in}{1.447780in}}%
\pgfpathlineto{\pgfqpoint{1.208316in}{1.447891in}}%
\pgfpathlineto{\pgfqpoint{1.209333in}{1.450744in}}%
\pgfpathlineto{\pgfqpoint{1.212094in}{1.455934in}}%
\pgfpathlineto{\pgfqpoint{1.212821in}{1.454159in}}%
\pgfpathlineto{\pgfqpoint{1.213983in}{1.449808in}}%
\pgfpathlineto{\pgfqpoint{1.214710in}{1.451747in}}%
\pgfpathlineto{\pgfqpoint{1.217616in}{1.458709in}}%
\pgfpathlineto{\pgfqpoint{1.218779in}{1.457131in}}%
\pgfpathlineto{\pgfqpoint{1.220377in}{1.455822in}}%
\pgfpathlineto{\pgfqpoint{1.220813in}{1.456262in}}%
\pgfpathlineto{\pgfqpoint{1.221975in}{1.462406in}}%
\pgfpathlineto{\pgfqpoint{1.222847in}{1.466033in}}%
\pgfpathlineto{\pgfqpoint{1.223719in}{1.463854in}}%
\pgfpathlineto{\pgfqpoint{1.225608in}{1.463239in}}%
\pgfpathlineto{\pgfqpoint{1.226335in}{1.462809in}}%
\pgfpathlineto{\pgfqpoint{1.226771in}{1.463827in}}%
\pgfpathlineto{\pgfqpoint{1.229386in}{1.471610in}}%
\pgfpathlineto{\pgfqpoint{1.230113in}{1.470864in}}%
\pgfpathlineto{\pgfqpoint{1.231711in}{1.467935in}}%
\pgfpathlineto{\pgfqpoint{1.232583in}{1.469101in}}%
\pgfpathlineto{\pgfqpoint{1.234618in}{1.473915in}}%
\pgfpathlineto{\pgfqpoint{1.235344in}{1.475406in}}%
\pgfpathlineto{\pgfqpoint{1.235925in}{1.473950in}}%
\pgfpathlineto{\pgfqpoint{1.237088in}{1.470992in}}%
\pgfpathlineto{\pgfqpoint{1.237814in}{1.471935in}}%
\pgfpathlineto{\pgfqpoint{1.242464in}{1.477085in}}%
\pgfpathlineto{\pgfqpoint{1.243772in}{1.475076in}}%
\pgfpathlineto{\pgfqpoint{1.244354in}{1.476697in}}%
\pgfpathlineto{\pgfqpoint{1.247114in}{1.482814in}}%
\pgfpathlineto{\pgfqpoint{1.250747in}{1.487481in}}%
\pgfpathlineto{\pgfqpoint{1.252636in}{1.491553in}}%
\pgfpathlineto{\pgfqpoint{1.253218in}{1.490666in}}%
\pgfpathlineto{\pgfqpoint{1.254525in}{1.488728in}}%
\pgfpathlineto{\pgfqpoint{1.255252in}{1.489427in}}%
\pgfpathlineto{\pgfqpoint{1.257577in}{1.493852in}}%
\pgfpathlineto{\pgfqpoint{1.258304in}{1.494473in}}%
\pgfpathlineto{\pgfqpoint{1.259030in}{1.493675in}}%
\pgfpathlineto{\pgfqpoint{1.261210in}{1.490908in}}%
\pgfpathlineto{\pgfqpoint{1.261791in}{1.491971in}}%
\pgfpathlineto{\pgfqpoint{1.264843in}{1.497847in}}%
\pgfpathlineto{\pgfqpoint{1.265279in}{1.497466in}}%
\pgfpathlineto{\pgfqpoint{1.267022in}{1.494617in}}%
\pgfpathlineto{\pgfqpoint{1.267749in}{1.495969in}}%
\pgfpathlineto{\pgfqpoint{1.270800in}{1.501618in}}%
\pgfpathlineto{\pgfqpoint{1.271091in}{1.501387in}}%
\pgfpathlineto{\pgfqpoint{1.273271in}{1.499806in}}%
\pgfpathlineto{\pgfqpoint{1.273707in}{1.500614in}}%
\pgfpathlineto{\pgfqpoint{1.276904in}{1.506125in}}%
\pgfpathlineto{\pgfqpoint{1.277049in}{1.506071in}}%
\pgfpathlineto{\pgfqpoint{1.279519in}{1.506355in}}%
\pgfpathlineto{\pgfqpoint{1.281699in}{1.510738in}}%
\pgfpathlineto{\pgfqpoint{1.282861in}{1.509661in}}%
\pgfpathlineto{\pgfqpoint{1.284460in}{1.509051in}}%
\pgfpathlineto{\pgfqpoint{1.284896in}{1.509568in}}%
\pgfpathlineto{\pgfqpoint{1.288238in}{1.513461in}}%
\pgfpathlineto{\pgfqpoint{1.288964in}{1.512475in}}%
\pgfpathlineto{\pgfqpoint{1.290418in}{1.511352in}}%
\pgfpathlineto{\pgfqpoint{1.290854in}{1.511955in}}%
\pgfpathlineto{\pgfqpoint{1.294922in}{1.517257in}}%
\pgfpathlineto{\pgfqpoint{1.296811in}{1.516060in}}%
\pgfpathlineto{\pgfqpoint{1.297247in}{1.517215in}}%
\pgfpathlineto{\pgfqpoint{1.300299in}{1.522600in}}%
\pgfpathlineto{\pgfqpoint{1.313522in}{1.534173in}}%
\pgfpathlineto{\pgfqpoint{1.315702in}{1.535980in}}%
\pgfpathlineto{\pgfqpoint{1.324857in}{1.546435in}}%
\pgfpathlineto{\pgfqpoint{1.326891in}{1.547337in}}%
\pgfpathlineto{\pgfqpoint{1.331105in}{1.552641in}}%
\pgfpathlineto{\pgfqpoint{1.333575in}{1.553776in}}%
\pgfpathlineto{\pgfqpoint{1.337208in}{1.557143in}}%
\pgfpathlineto{\pgfqpoint{1.346944in}{1.565197in}}%
\pgfpathlineto{\pgfqpoint{1.348252in}{1.566061in}}%
\pgfpathlineto{\pgfqpoint{1.348979in}{1.565607in}}%
\pgfpathlineto{\pgfqpoint{1.350286in}{1.566891in}}%
\pgfpathlineto{\pgfqpoint{1.354355in}{1.571181in}}%
\pgfpathlineto{\pgfqpoint{1.357116in}{1.573897in}}%
\pgfpathlineto{\pgfqpoint{1.360022in}{1.576982in}}%
\pgfpathlineto{\pgfqpoint{1.363219in}{1.580144in}}%
\pgfpathlineto{\pgfqpoint{1.366125in}{1.582884in}}%
\pgfpathlineto{\pgfqpoint{1.368305in}{1.583797in}}%
\pgfpathlineto{\pgfqpoint{1.401436in}{1.613529in}}%
\pgfpathlineto{\pgfqpoint{1.403180in}{1.614282in}}%
\pgfpathlineto{\pgfqpoint{1.408702in}{1.618739in}}%
\pgfpathlineto{\pgfqpoint{1.439072in}{1.644543in}}%
\pgfpathlineto{\pgfqpoint{1.442850in}{1.648640in}}%
\pgfpathlineto{\pgfqpoint{1.448663in}{1.652597in}}%
\pgfpathlineto{\pgfqpoint{1.472930in}{1.674866in}}%
\pgfpathlineto{\pgfqpoint{1.486008in}{1.685292in}}%
\pgfpathlineto{\pgfqpoint{1.503591in}{1.699511in}}%
\pgfpathlineto{\pgfqpoint{1.511583in}{1.706350in}}%
\pgfpathlineto{\pgfqpoint{1.517977in}{1.712171in}}%
\pgfpathlineto{\pgfqpoint{1.533380in}{1.724935in}}%
\pgfpathlineto{\pgfqpoint{1.544424in}{1.734299in}}%
\pgfpathlineto{\pgfqpoint{1.549800in}{1.738886in}}%
\pgfpathlineto{\pgfqpoint{1.568255in}{1.752458in}}%
\pgfpathlineto{\pgfqpoint{1.574939in}{1.757726in}}%
\pgfpathlineto{\pgfqpoint{1.583368in}{1.763107in}}%
\pgfpathlineto{\pgfqpoint{1.597318in}{1.772859in}}%
\pgfpathlineto{\pgfqpoint{1.607635in}{1.780238in}}%
\pgfpathlineto{\pgfqpoint{1.617661in}{1.786523in}}%
\pgfpathlineto{\pgfqpoint{1.740305in}{1.860900in}}%
\pgfpathlineto{\pgfqpoint{1.771983in}{1.877103in}}%
\pgfpathlineto{\pgfqpoint{1.787241in}{1.884354in}}%
\pgfpathlineto{\pgfqpoint{1.833596in}{1.903868in}}%
\pgfpathlineto{\pgfqpoint{1.888814in}{1.923457in}}%
\pgfpathlineto{\pgfqpoint{1.912791in}{1.931399in}}%
\pgfpathlineto{\pgfqpoint{1.928485in}{1.936672in}}%
\pgfpathlineto{\pgfqpoint{1.953188in}{1.943843in}}%
\pgfpathlineto{\pgfqpoint{1.971207in}{1.948455in}}%
\pgfpathlineto{\pgfqpoint{1.998525in}{1.954946in}}%
\pgfpathlineto{\pgfqpoint{2.131196in}{1.977567in}}%
\pgfpathlineto{\pgfqpoint{2.163600in}{1.981929in}}%
\pgfpathlineto{\pgfqpoint{2.202108in}{1.985456in}}%
\pgfpathlineto{\pgfqpoint{2.243668in}{1.988651in}}%
\pgfpathlineto{\pgfqpoint{2.284936in}{1.991254in}}%
\pgfpathlineto{\pgfqpoint{2.300775in}{1.991675in}}%
\pgfpathlineto{\pgfqpoint{2.401186in}{1.993699in}}%
\pgfpathlineto{\pgfqpoint{2.413829in}{1.994123in}}%
\pgfpathlineto{\pgfqpoint{2.427197in}{1.994007in}}%
\pgfpathlineto{\pgfqpoint{2.454371in}{1.994123in}}%
\pgfpathlineto{\pgfqpoint{2.475296in}{1.993987in}}%
\pgfpathlineto{\pgfqpoint{2.482416in}{1.993495in}}%
\pgfpathlineto{\pgfqpoint{2.506974in}{1.993506in}}%
\pgfpathlineto{\pgfqpoint{2.518454in}{1.993844in}}%
\pgfpathlineto{\pgfqpoint{2.541994in}{1.993173in}}%
\pgfpathlineto{\pgfqpoint{2.564808in}{1.992940in}}%
\pgfpathlineto{\pgfqpoint{2.650979in}{1.992597in}}%
\pgfpathlineto{\pgfqpoint{2.714625in}{1.993012in}}%
\pgfpathlineto{\pgfqpoint{2.734533in}{1.992660in}}%
\pgfpathlineto{\pgfqpoint{2.768972in}{1.991181in}}%
\pgfpathlineto{\pgfqpoint{2.784085in}{1.990958in}}%
\pgfpathlineto{\pgfqpoint{2.825789in}{1.988642in}}%
\pgfpathlineto{\pgfqpoint{2.859502in}{1.986976in}}%
\pgfpathlineto{\pgfqpoint{2.869238in}{1.986976in}}%
\pgfpathlineto{\pgfqpoint{2.896702in}{1.986002in}}%
\pgfpathlineto{\pgfqpoint{2.940441in}{1.983675in}}%
\pgfpathlineto{\pgfqpoint{2.947997in}{1.983559in}}%
\pgfpathlineto{\pgfqpoint{2.957297in}{1.983635in}}%
\pgfpathlineto{\pgfqpoint{2.970811in}{1.982905in}}%
\pgfpathlineto{\pgfqpoint{2.982582in}{1.982740in}}%
\pgfpathlineto{\pgfqpoint{2.996822in}{1.982334in}}%
\pgfpathlineto{\pgfqpoint{3.004814in}{1.982330in}}%
\pgfpathlineto{\pgfqpoint{3.004814in}{1.982330in}}%
\pgfusepath{stroke}%
\end{pgfscope}%
\begin{pgfscope}%
\pgfpathrectangle{\pgfqpoint{0.679669in}{0.526079in}}{\pgfqpoint{2.325000in}{1.661000in}} %
\pgfusepath{clip}%
\pgfsetrectcap%
\pgfsetroundjoin%
\pgfsetlinewidth{1.003750pt}%
\definecolor{currentstroke}{rgb}{0.501961,0.000000,0.501961}%
\pgfsetstrokecolor{currentstroke}%
\pgfsetdash{}{0pt}%
\pgfpathmoveto{\pgfqpoint{0.680109in}{0.512191in}}%
\pgfpathlineto{\pgfqpoint{0.682285in}{0.814056in}}%
\pgfpathlineto{\pgfqpoint{0.684900in}{0.866351in}}%
\pgfpathlineto{\pgfqpoint{0.687225in}{0.882227in}}%
\pgfpathlineto{\pgfqpoint{0.690277in}{0.917214in}}%
\pgfpathlineto{\pgfqpoint{0.690713in}{0.919433in}}%
\pgfpathlineto{\pgfqpoint{0.691149in}{0.915648in}}%
\pgfpathlineto{\pgfqpoint{0.692602in}{0.882298in}}%
\pgfpathlineto{\pgfqpoint{0.693474in}{0.899396in}}%
\pgfpathlineto{\pgfqpoint{0.699286in}{1.006286in}}%
\pgfpathlineto{\pgfqpoint{0.700885in}{1.046254in}}%
\pgfpathlineto{\pgfqpoint{0.701466in}{1.043331in}}%
\pgfpathlineto{\pgfqpoint{0.703791in}{1.002854in}}%
\pgfpathlineto{\pgfqpoint{0.704663in}{0.993057in}}%
\pgfpathlineto{\pgfqpoint{0.705389in}{1.000324in}}%
\pgfpathlineto{\pgfqpoint{0.706552in}{1.015780in}}%
\pgfpathlineto{\pgfqpoint{0.707279in}{1.006816in}}%
\pgfpathlineto{\pgfqpoint{0.711202in}{0.906988in}}%
\pgfpathlineto{\pgfqpoint{0.711783in}{0.920409in}}%
\pgfpathlineto{\pgfqpoint{0.713382in}{0.974356in}}%
\pgfpathlineto{\pgfqpoint{0.714254in}{0.966511in}}%
\pgfpathlineto{\pgfqpoint{0.715125in}{0.961156in}}%
\pgfpathlineto{\pgfqpoint{0.715852in}{0.965784in}}%
\pgfpathlineto{\pgfqpoint{0.717305in}{0.999768in}}%
\pgfpathlineto{\pgfqpoint{0.720211in}{1.056753in}}%
\pgfpathlineto{\pgfqpoint{0.720357in}{1.056151in}}%
\pgfpathlineto{\pgfqpoint{0.721519in}{1.032277in}}%
\pgfpathlineto{\pgfqpoint{0.723554in}{1.017172in}}%
\pgfpathlineto{\pgfqpoint{0.723699in}{1.017385in}}%
\pgfpathlineto{\pgfqpoint{0.724571in}{1.027983in}}%
\pgfpathlineto{\pgfqpoint{0.725152in}{1.034471in}}%
\pgfpathlineto{\pgfqpoint{0.725733in}{1.026001in}}%
\pgfpathlineto{\pgfqpoint{0.727622in}{0.955327in}}%
\pgfpathlineto{\pgfqpoint{0.729075in}{0.963795in}}%
\pgfpathlineto{\pgfqpoint{0.729366in}{0.964406in}}%
\pgfpathlineto{\pgfqpoint{0.730964in}{0.994235in}}%
\pgfpathlineto{\pgfqpoint{0.731836in}{0.977565in}}%
\pgfpathlineto{\pgfqpoint{0.733580in}{0.953419in}}%
\pgfpathlineto{\pgfqpoint{0.734016in}{0.958421in}}%
\pgfpathlineto{\pgfqpoint{0.737504in}{1.067379in}}%
\pgfpathlineto{\pgfqpoint{0.738375in}{1.056836in}}%
\pgfpathlineto{\pgfqpoint{0.738957in}{1.050345in}}%
\pgfpathlineto{\pgfqpoint{0.739683in}{1.059765in}}%
\pgfpathlineto{\pgfqpoint{0.741718in}{1.093198in}}%
\pgfpathlineto{\pgfqpoint{0.742299in}{1.091314in}}%
\pgfpathlineto{\pgfqpoint{0.743461in}{1.066077in}}%
\pgfpathlineto{\pgfqpoint{0.746513in}{1.004718in}}%
\pgfpathlineto{\pgfqpoint{0.747094in}{1.015985in}}%
\pgfpathlineto{\pgfqpoint{0.749855in}{1.064438in}}%
\pgfpathlineto{\pgfqpoint{0.755086in}{1.101468in}}%
\pgfpathlineto{\pgfqpoint{0.755522in}{1.099308in}}%
\pgfpathlineto{\pgfqpoint{0.761625in}{1.029143in}}%
\pgfpathlineto{\pgfqpoint{0.762497in}{1.033705in}}%
\pgfpathlineto{\pgfqpoint{0.763224in}{1.032056in}}%
\pgfpathlineto{\pgfqpoint{0.763805in}{1.037383in}}%
\pgfpathlineto{\pgfqpoint{0.767729in}{1.138233in}}%
\pgfpathlineto{\pgfqpoint{0.768600in}{1.129430in}}%
\pgfpathlineto{\pgfqpoint{0.769618in}{1.119020in}}%
\pgfpathlineto{\pgfqpoint{0.770344in}{1.123803in}}%
\pgfpathlineto{\pgfqpoint{0.772088in}{1.148657in}}%
\pgfpathlineto{\pgfqpoint{0.772524in}{1.153221in}}%
\pgfpathlineto{\pgfqpoint{0.773250in}{1.143446in}}%
\pgfpathlineto{\pgfqpoint{0.776447in}{1.031875in}}%
\pgfpathlineto{\pgfqpoint{0.777464in}{1.060466in}}%
\pgfpathlineto{\pgfqpoint{0.779208in}{1.096997in}}%
\pgfpathlineto{\pgfqpoint{0.779644in}{1.091285in}}%
\pgfpathlineto{\pgfqpoint{0.781097in}{1.047707in}}%
\pgfpathlineto{\pgfqpoint{0.781824in}{1.066777in}}%
\pgfpathlineto{\pgfqpoint{0.784875in}{1.153952in}}%
\pgfpathlineto{\pgfqpoint{0.785021in}{1.153527in}}%
\pgfpathlineto{\pgfqpoint{0.785893in}{1.130601in}}%
\pgfpathlineto{\pgfqpoint{0.787055in}{1.085951in}}%
\pgfpathlineto{\pgfqpoint{0.787927in}{1.102015in}}%
\pgfpathlineto{\pgfqpoint{0.789816in}{1.143567in}}%
\pgfpathlineto{\pgfqpoint{0.790543in}{1.132777in}}%
\pgfpathlineto{\pgfqpoint{0.793158in}{1.023213in}}%
\pgfpathlineto{\pgfqpoint{0.793885in}{1.055547in}}%
\pgfpathlineto{\pgfqpoint{0.795629in}{1.132393in}}%
\pgfpathlineto{\pgfqpoint{0.796210in}{1.125028in}}%
\pgfpathlineto{\pgfqpoint{0.798680in}{1.038282in}}%
\pgfpathlineto{\pgfqpoint{0.799697in}{1.076467in}}%
\pgfpathlineto{\pgfqpoint{0.801877in}{1.163899in}}%
\pgfpathlineto{\pgfqpoint{0.802458in}{1.157417in}}%
\pgfpathlineto{\pgfqpoint{0.803911in}{1.108582in}}%
\pgfpathlineto{\pgfqpoint{0.804929in}{1.129864in}}%
\pgfpathlineto{\pgfqpoint{0.807689in}{1.172721in}}%
\pgfpathlineto{\pgfqpoint{0.807980in}{1.170393in}}%
\pgfpathlineto{\pgfqpoint{0.808997in}{1.128244in}}%
\pgfpathlineto{\pgfqpoint{0.810450in}{1.036859in}}%
\pgfpathlineto{\pgfqpoint{0.811032in}{1.071539in}}%
\pgfpathlineto{\pgfqpoint{0.812630in}{1.153699in}}%
\pgfpathlineto{\pgfqpoint{0.813211in}{1.144594in}}%
\pgfpathlineto{\pgfqpoint{0.816263in}{1.033051in}}%
\pgfpathlineto{\pgfqpoint{0.817135in}{1.082277in}}%
\pgfpathlineto{\pgfqpoint{0.819024in}{1.187829in}}%
\pgfpathlineto{\pgfqpoint{0.819605in}{1.180217in}}%
\pgfpathlineto{\pgfqpoint{0.822511in}{1.119889in}}%
\pgfpathlineto{\pgfqpoint{0.823093in}{1.130956in}}%
\pgfpathlineto{\pgfqpoint{0.824982in}{1.202681in}}%
\pgfpathlineto{\pgfqpoint{0.825708in}{1.184255in}}%
\pgfpathlineto{\pgfqpoint{0.827888in}{1.076122in}}%
\pgfpathlineto{\pgfqpoint{0.828905in}{1.098906in}}%
\pgfpathlineto{\pgfqpoint{0.830939in}{1.154878in}}%
\pgfpathlineto{\pgfqpoint{0.831521in}{1.145363in}}%
\pgfpathlineto{\pgfqpoint{0.832829in}{1.031500in}}%
\pgfpathlineto{\pgfqpoint{0.833410in}{0.994288in}}%
\pgfpathlineto{\pgfqpoint{0.834136in}{1.047085in}}%
\pgfpathlineto{\pgfqpoint{0.836752in}{1.159996in}}%
\pgfpathlineto{\pgfqpoint{0.837479in}{1.167213in}}%
\pgfpathlineto{\pgfqpoint{0.838205in}{1.160339in}}%
\pgfpathlineto{\pgfqpoint{0.839222in}{1.139412in}}%
\pgfpathlineto{\pgfqpoint{0.839949in}{1.155821in}}%
\pgfpathlineto{\pgfqpoint{0.841838in}{1.210928in}}%
\pgfpathlineto{\pgfqpoint{0.842564in}{1.205196in}}%
\pgfpathlineto{\pgfqpoint{0.845761in}{1.138325in}}%
\pgfpathlineto{\pgfqpoint{0.846197in}{1.133564in}}%
\pgfpathlineto{\pgfqpoint{0.846779in}{1.143275in}}%
\pgfpathlineto{\pgfqpoint{0.847941in}{1.167755in}}%
\pgfpathlineto{\pgfqpoint{0.848668in}{1.155205in}}%
\pgfpathlineto{\pgfqpoint{0.850993in}{1.090832in}}%
\pgfpathlineto{\pgfqpoint{0.851719in}{1.095599in}}%
\pgfpathlineto{\pgfqpoint{0.853027in}{1.144857in}}%
\pgfpathlineto{\pgfqpoint{0.854335in}{1.183382in}}%
\pgfpathlineto{\pgfqpoint{0.855061in}{1.173989in}}%
\pgfpathlineto{\pgfqpoint{0.856369in}{1.147324in}}%
\pgfpathlineto{\pgfqpoint{0.857096in}{1.160965in}}%
\pgfpathlineto{\pgfqpoint{0.860002in}{1.212462in}}%
\pgfpathlineto{\pgfqpoint{0.860147in}{1.212529in}}%
\pgfpathlineto{\pgfqpoint{0.860438in}{1.211035in}}%
\pgfpathlineto{\pgfqpoint{0.861455in}{1.179204in}}%
\pgfpathlineto{\pgfqpoint{0.862763in}{1.132134in}}%
\pgfpathlineto{\pgfqpoint{0.863489in}{1.143888in}}%
\pgfpathlineto{\pgfqpoint{0.864652in}{1.159825in}}%
\pgfpathlineto{\pgfqpoint{0.865379in}{1.154026in}}%
\pgfpathlineto{\pgfqpoint{0.867122in}{1.115697in}}%
\pgfpathlineto{\pgfqpoint{0.868866in}{1.030505in}}%
\pgfpathlineto{\pgfqpoint{0.869593in}{1.064974in}}%
\pgfpathlineto{\pgfqpoint{0.871482in}{1.164138in}}%
\pgfpathlineto{\pgfqpoint{0.872208in}{1.157837in}}%
\pgfpathlineto{\pgfqpoint{0.874824in}{1.122475in}}%
\pgfpathlineto{\pgfqpoint{0.875696in}{1.134745in}}%
\pgfpathlineto{\pgfqpoint{0.877730in}{1.202697in}}%
\pgfpathlineto{\pgfqpoint{0.878602in}{1.181544in}}%
\pgfpathlineto{\pgfqpoint{0.880055in}{1.131808in}}%
\pgfpathlineto{\pgfqpoint{0.881072in}{1.138402in}}%
\pgfpathlineto{\pgfqpoint{0.882235in}{1.147720in}}%
\pgfpathlineto{\pgfqpoint{0.883543in}{1.176524in}}%
\pgfpathlineto{\pgfqpoint{0.884269in}{1.160554in}}%
\pgfpathlineto{\pgfqpoint{0.886158in}{1.049168in}}%
\pgfpathlineto{\pgfqpoint{0.887030in}{1.091650in}}%
\pgfpathlineto{\pgfqpoint{0.889210in}{1.129869in}}%
\pgfpathlineto{\pgfqpoint{0.890372in}{1.154440in}}%
\pgfpathlineto{\pgfqpoint{0.890954in}{1.141572in}}%
\pgfpathlineto{\pgfqpoint{0.892116in}{1.086606in}}%
\pgfpathlineto{\pgfqpoint{0.892697in}{1.116160in}}%
\pgfpathlineto{\pgfqpoint{0.894877in}{1.200530in}}%
\pgfpathlineto{\pgfqpoint{0.895168in}{1.200237in}}%
\pgfpathlineto{\pgfqpoint{0.896039in}{1.194535in}}%
\pgfpathlineto{\pgfqpoint{0.898074in}{1.144659in}}%
\pgfpathlineto{\pgfqpoint{0.898655in}{1.133839in}}%
\pgfpathlineto{\pgfqpoint{0.899236in}{1.155010in}}%
\pgfpathlineto{\pgfqpoint{0.900544in}{1.213438in}}%
\pgfpathlineto{\pgfqpoint{0.901271in}{1.197916in}}%
\pgfpathlineto{\pgfqpoint{0.904758in}{1.090248in}}%
\pgfpathlineto{\pgfqpoint{0.905049in}{1.095982in}}%
\pgfpathlineto{\pgfqpoint{0.907083in}{1.201865in}}%
\pgfpathlineto{\pgfqpoint{0.908100in}{1.173037in}}%
\pgfpathlineto{\pgfqpoint{0.909263in}{1.118555in}}%
\pgfpathlineto{\pgfqpoint{0.910135in}{1.144287in}}%
\pgfpathlineto{\pgfqpoint{0.913477in}{1.232643in}}%
\pgfpathlineto{\pgfqpoint{0.913768in}{1.230811in}}%
\pgfpathlineto{\pgfqpoint{0.914930in}{1.185487in}}%
\pgfpathlineto{\pgfqpoint{0.915802in}{1.153016in}}%
\pgfpathlineto{\pgfqpoint{0.916529in}{1.169335in}}%
\pgfpathlineto{\pgfqpoint{0.918854in}{1.203474in}}%
\pgfpathlineto{\pgfqpoint{0.919725in}{1.211422in}}%
\pgfpathlineto{\pgfqpoint{0.920161in}{1.206314in}}%
\pgfpathlineto{\pgfqpoint{0.921179in}{1.134276in}}%
\pgfpathlineto{\pgfqpoint{0.922196in}{1.033284in}}%
\pgfpathlineto{\pgfqpoint{0.922922in}{1.104126in}}%
\pgfpathlineto{\pgfqpoint{0.924375in}{1.174000in}}%
\pgfpathlineto{\pgfqpoint{0.924957in}{1.171245in}}%
\pgfpathlineto{\pgfqpoint{0.925393in}{1.170111in}}%
\pgfpathlineto{\pgfqpoint{0.926119in}{1.173085in}}%
\pgfpathlineto{\pgfqpoint{0.926555in}{1.174352in}}%
\pgfpathlineto{\pgfqpoint{0.926991in}{1.171008in}}%
\pgfpathlineto{\pgfqpoint{0.928444in}{1.121884in}}%
\pgfpathlineto{\pgfqpoint{0.929171in}{1.158145in}}%
\pgfpathlineto{\pgfqpoint{0.930769in}{1.241236in}}%
\pgfpathlineto{\pgfqpoint{0.931496in}{1.232867in}}%
\pgfpathlineto{\pgfqpoint{0.934983in}{1.149667in}}%
\pgfpathlineto{\pgfqpoint{0.935855in}{1.175551in}}%
\pgfpathlineto{\pgfqpoint{0.937163in}{1.215186in}}%
\pgfpathlineto{\pgfqpoint{0.937744in}{1.201748in}}%
\pgfpathlineto{\pgfqpoint{0.940796in}{1.103768in}}%
\pgfpathlineto{\pgfqpoint{0.941232in}{1.110614in}}%
\pgfpathlineto{\pgfqpoint{0.943557in}{1.228785in}}%
\pgfpathlineto{\pgfqpoint{0.944574in}{1.196613in}}%
\pgfpathlineto{\pgfqpoint{0.945736in}{1.153088in}}%
\pgfpathlineto{\pgfqpoint{0.946463in}{1.168243in}}%
\pgfpathlineto{\pgfqpoint{0.949660in}{1.231908in}}%
\pgfpathlineto{\pgfqpoint{0.949950in}{1.228480in}}%
\pgfpathlineto{\pgfqpoint{0.950968in}{1.162701in}}%
\pgfpathlineto{\pgfqpoint{0.951839in}{1.097930in}}%
\pgfpathlineto{\pgfqpoint{0.952711in}{1.134797in}}%
\pgfpathlineto{\pgfqpoint{0.955182in}{1.190143in}}%
\pgfpathlineto{\pgfqpoint{0.955618in}{1.191243in}}%
\pgfpathlineto{\pgfqpoint{0.956054in}{1.188428in}}%
\pgfpathlineto{\pgfqpoint{0.956925in}{1.151186in}}%
\pgfpathlineto{\pgfqpoint{0.958088in}{1.047548in}}%
\pgfpathlineto{\pgfqpoint{0.958814in}{1.108953in}}%
\pgfpathlineto{\pgfqpoint{0.960558in}{1.215766in}}%
\pgfpathlineto{\pgfqpoint{0.961139in}{1.211894in}}%
\pgfpathlineto{\pgfqpoint{0.963029in}{1.178586in}}%
\pgfpathlineto{\pgfqpoint{0.964191in}{1.122176in}}%
\pgfpathlineto{\pgfqpoint{0.964918in}{1.153628in}}%
\pgfpathlineto{\pgfqpoint{0.966661in}{1.241604in}}%
\pgfpathlineto{\pgfqpoint{0.967243in}{1.232168in}}%
\pgfpathlineto{\pgfqpoint{0.970730in}{1.119870in}}%
\pgfpathlineto{\pgfqpoint{0.971166in}{1.134009in}}%
\pgfpathlineto{\pgfqpoint{0.972910in}{1.214989in}}%
\pgfpathlineto{\pgfqpoint{0.973636in}{1.191822in}}%
\pgfpathlineto{\pgfqpoint{0.975235in}{1.085834in}}%
\pgfpathlineto{\pgfqpoint{0.976252in}{1.111771in}}%
\pgfpathlineto{\pgfqpoint{0.979304in}{1.224981in}}%
\pgfpathlineto{\pgfqpoint{0.980175in}{1.197604in}}%
\pgfpathlineto{\pgfqpoint{0.981338in}{1.122426in}}%
\pgfpathlineto{\pgfqpoint{0.982210in}{1.159029in}}%
\pgfpathlineto{\pgfqpoint{0.984535in}{1.201505in}}%
\pgfpathlineto{\pgfqpoint{0.985552in}{1.211488in}}%
\pgfpathlineto{\pgfqpoint{0.986133in}{1.203121in}}%
\pgfpathlineto{\pgfqpoint{0.987441in}{1.093494in}}%
\pgfpathlineto{\pgfqpoint{0.987877in}{1.065205in}}%
\pgfpathlineto{\pgfqpoint{0.988604in}{1.112949in}}%
\pgfpathlineto{\pgfqpoint{0.989911in}{1.187006in}}%
\pgfpathlineto{\pgfqpoint{0.990638in}{1.176285in}}%
\pgfpathlineto{\pgfqpoint{0.994125in}{1.056328in}}%
\pgfpathlineto{\pgfqpoint{0.994707in}{1.100191in}}%
\pgfpathlineto{\pgfqpoint{0.996305in}{1.218531in}}%
\pgfpathlineto{\pgfqpoint{0.997032in}{1.201237in}}%
\pgfpathlineto{\pgfqpoint{0.998194in}{1.157927in}}%
\pgfpathlineto{\pgfqpoint{0.999211in}{1.168749in}}%
\pgfpathlineto{\pgfqpoint{0.999938in}{1.147914in}}%
\pgfpathlineto{\pgfqpoint{1.000664in}{1.124456in}}%
\pgfpathlineto{\pgfqpoint{1.001246in}{1.152970in}}%
\pgfpathlineto{\pgfqpoint{1.002554in}{1.221746in}}%
\pgfpathlineto{\pgfqpoint{1.003280in}{1.202021in}}%
\pgfpathlineto{\pgfqpoint{1.005024in}{1.114873in}}%
\pgfpathlineto{\pgfqpoint{1.005896in}{1.133938in}}%
\pgfpathlineto{\pgfqpoint{1.009093in}{1.208376in}}%
\pgfpathlineto{\pgfqpoint{1.007204in}{1.131704in}}%
\pgfpathlineto{\pgfqpoint{1.009674in}{1.186552in}}%
\pgfpathlineto{\pgfqpoint{1.011127in}{1.049730in}}%
\pgfpathlineto{\pgfqpoint{1.011999in}{1.124602in}}%
\pgfpathlineto{\pgfqpoint{1.014324in}{1.186402in}}%
\pgfpathlineto{\pgfqpoint{1.015196in}{1.201157in}}%
\pgfpathlineto{\pgfqpoint{1.015777in}{1.192509in}}%
\pgfpathlineto{\pgfqpoint{1.017521in}{1.090054in}}%
\pgfpathlineto{\pgfqpoint{1.018393in}{1.153113in}}%
\pgfpathlineto{\pgfqpoint{1.019555in}{1.215352in}}%
\pgfpathlineto{\pgfqpoint{1.020282in}{1.195457in}}%
\pgfpathlineto{\pgfqpoint{1.023769in}{1.076220in}}%
\pgfpathlineto{\pgfqpoint{1.024205in}{1.092376in}}%
\pgfpathlineto{\pgfqpoint{1.025949in}{1.203791in}}%
\pgfpathlineto{\pgfqpoint{1.026675in}{1.187508in}}%
\pgfpathlineto{\pgfqpoint{1.028129in}{1.118204in}}%
\pgfpathlineto{\pgfqpoint{1.029146in}{1.140442in}}%
\pgfpathlineto{\pgfqpoint{1.029436in}{1.138246in}}%
\pgfpathlineto{\pgfqpoint{1.030308in}{1.114349in}}%
\pgfpathlineto{\pgfqpoint{1.030744in}{1.131637in}}%
\pgfpathlineto{\pgfqpoint{1.032197in}{1.223523in}}%
\pgfpathlineto{\pgfqpoint{1.032924in}{1.198935in}}%
\pgfpathlineto{\pgfqpoint{1.034232in}{1.124864in}}%
\pgfpathlineto{\pgfqpoint{1.035104in}{1.142569in}}%
\pgfpathlineto{\pgfqpoint{1.038736in}{1.225806in}}%
\pgfpathlineto{\pgfqpoint{1.039172in}{1.219218in}}%
\pgfpathlineto{\pgfqpoint{1.040335in}{1.096402in}}%
\pgfpathlineto{\pgfqpoint{1.040771in}{1.052959in}}%
\pgfpathlineto{\pgfqpoint{1.041497in}{1.118860in}}%
\pgfpathlineto{\pgfqpoint{1.043677in}{1.193970in}}%
\pgfpathlineto{\pgfqpoint{1.044694in}{1.208953in}}%
\pgfpathlineto{\pgfqpoint{1.045275in}{1.199474in}}%
\pgfpathlineto{\pgfqpoint{1.046729in}{1.087684in}}%
\pgfpathlineto{\pgfqpoint{1.047164in}{1.068937in}}%
\pgfpathlineto{\pgfqpoint{1.047746in}{1.113195in}}%
\pgfpathlineto{\pgfqpoint{1.049344in}{1.223005in}}%
\pgfpathlineto{\pgfqpoint{1.050071in}{1.207428in}}%
\pgfpathlineto{\pgfqpoint{1.053268in}{1.104765in}}%
\pgfpathlineto{\pgfqpoint{1.053704in}{1.120780in}}%
\pgfpathlineto{\pgfqpoint{1.055593in}{1.236048in}}%
\pgfpathlineto{\pgfqpoint{1.056319in}{1.218715in}}%
\pgfpathlineto{\pgfqpoint{1.057918in}{1.134019in}}%
\pgfpathlineto{\pgfqpoint{1.059080in}{1.157460in}}%
\pgfpathlineto{\pgfqpoint{1.059516in}{1.154810in}}%
\pgfpathlineto{\pgfqpoint{1.059952in}{1.152359in}}%
\pgfpathlineto{\pgfqpoint{1.060243in}{1.157423in}}%
\pgfpathlineto{\pgfqpoint{1.061841in}{1.228418in}}%
\pgfpathlineto{\pgfqpoint{1.062713in}{1.193532in}}%
\pgfpathlineto{\pgfqpoint{1.064021in}{1.078196in}}%
\pgfpathlineto{\pgfqpoint{1.064747in}{1.126840in}}%
\pgfpathlineto{\pgfqpoint{1.067363in}{1.204460in}}%
\pgfpathlineto{\pgfqpoint{1.068380in}{1.216690in}}%
\pgfpathlineto{\pgfqpoint{1.068961in}{1.205552in}}%
\pgfpathlineto{\pgfqpoint{1.070414in}{1.102862in}}%
\pgfpathlineto{\pgfqpoint{1.071286in}{1.166184in}}%
\pgfpathlineto{\pgfqpoint{1.072304in}{1.209197in}}%
\pgfpathlineto{\pgfqpoint{1.073175in}{1.201646in}}%
\pgfpathlineto{\pgfqpoint{1.073466in}{1.200939in}}%
\pgfpathlineto{\pgfqpoint{1.073902in}{1.204189in}}%
\pgfpathlineto{\pgfqpoint{1.074483in}{1.208811in}}%
\pgfpathlineto{\pgfqpoint{1.075064in}{1.201140in}}%
\pgfpathlineto{\pgfqpoint{1.076808in}{1.121377in}}%
\pgfpathlineto{\pgfqpoint{1.077535in}{1.159093in}}%
\pgfpathlineto{\pgfqpoint{1.078843in}{1.218989in}}%
\pgfpathlineto{\pgfqpoint{1.079569in}{1.202217in}}%
\pgfpathlineto{\pgfqpoint{1.082911in}{1.113638in}}%
\pgfpathlineto{\pgfqpoint{1.083493in}{1.138926in}}%
\pgfpathlineto{\pgfqpoint{1.085091in}{1.223361in}}%
\pgfpathlineto{\pgfqpoint{1.085818in}{1.203120in}}%
\pgfpathlineto{\pgfqpoint{1.087271in}{1.136522in}}%
\pgfpathlineto{\pgfqpoint{1.087997in}{1.159253in}}%
\pgfpathlineto{\pgfqpoint{1.090613in}{1.210200in}}%
\pgfpathlineto{\pgfqpoint{1.091339in}{1.224372in}}%
\pgfpathlineto{\pgfqpoint{1.091921in}{1.212155in}}%
\pgfpathlineto{\pgfqpoint{1.093374in}{1.130392in}}%
\pgfpathlineto{\pgfqpoint{1.094391in}{1.158143in}}%
\pgfpathlineto{\pgfqpoint{1.095554in}{1.175260in}}%
\pgfpathlineto{\pgfqpoint{1.096135in}{1.170090in}}%
\pgfpathlineto{\pgfqpoint{1.099332in}{1.096243in}}%
\pgfpathlineto{\pgfqpoint{1.100058in}{1.061531in}}%
\pgfpathlineto{\pgfqpoint{1.100639in}{1.090683in}}%
\pgfpathlineto{\pgfqpoint{1.102093in}{1.170882in}}%
\pgfpathlineto{\pgfqpoint{1.102819in}{1.155018in}}%
\pgfpathlineto{\pgfqpoint{1.103110in}{1.151151in}}%
\pgfpathlineto{\pgfqpoint{1.103836in}{1.162978in}}%
\pgfpathlineto{\pgfqpoint{1.104272in}{1.168226in}}%
\pgfpathlineto{\pgfqpoint{1.104854in}{1.157888in}}%
\pgfpathlineto{\pgfqpoint{1.105871in}{1.137037in}}%
\pgfpathlineto{\pgfqpoint{1.106597in}{1.145812in}}%
\pgfpathlineto{\pgfqpoint{1.108486in}{1.190611in}}%
\pgfpathlineto{\pgfqpoint{1.109213in}{1.171744in}}%
\pgfpathlineto{\pgfqpoint{1.111683in}{1.125297in}}%
\pgfpathlineto{\pgfqpoint{1.112700in}{1.101139in}}%
\pgfpathlineto{\pgfqpoint{1.113282in}{1.118609in}}%
\pgfpathlineto{\pgfqpoint{1.114735in}{1.177732in}}%
\pgfpathlineto{\pgfqpoint{1.115461in}{1.156076in}}%
\pgfpathlineto{\pgfqpoint{1.116769in}{1.116326in}}%
\pgfpathlineto{\pgfqpoint{1.117641in}{1.116718in}}%
\pgfpathlineto{\pgfqpoint{1.118222in}{1.124284in}}%
\pgfpathlineto{\pgfqpoint{1.121274in}{1.202322in}}%
\pgfpathlineto{\pgfqpoint{1.121855in}{1.186301in}}%
\pgfpathlineto{\pgfqpoint{1.123018in}{1.112573in}}%
\pgfpathlineto{\pgfqpoint{1.123889in}{1.150976in}}%
\pgfpathlineto{\pgfqpoint{1.126069in}{1.193440in}}%
\pgfpathlineto{\pgfqpoint{1.126650in}{1.194519in}}%
\pgfpathlineto{\pgfqpoint{1.127232in}{1.192458in}}%
\pgfpathlineto{\pgfqpoint{1.128249in}{1.168701in}}%
\pgfpathlineto{\pgfqpoint{1.129702in}{1.108972in}}%
\pgfpathlineto{\pgfqpoint{1.130429in}{1.136610in}}%
\pgfpathlineto{\pgfqpoint{1.131736in}{1.190564in}}%
\pgfpathlineto{\pgfqpoint{1.132463in}{1.171673in}}%
\pgfpathlineto{\pgfqpoint{1.135660in}{1.098562in}}%
\pgfpathlineto{\pgfqpoint{1.136096in}{1.113023in}}%
\pgfpathlineto{\pgfqpoint{1.137839in}{1.190665in}}%
\pgfpathlineto{\pgfqpoint{1.138566in}{1.184894in}}%
\pgfpathlineto{\pgfqpoint{1.139583in}{1.150047in}}%
\pgfpathlineto{\pgfqpoint{1.140310in}{1.116023in}}%
\pgfpathlineto{\pgfqpoint{1.141036in}{1.141552in}}%
\pgfpathlineto{\pgfqpoint{1.141763in}{1.162317in}}%
\pgfpathlineto{\pgfqpoint{1.142780in}{1.152012in}}%
\pgfpathlineto{\pgfqpoint{1.144233in}{1.206783in}}%
\pgfpathlineto{\pgfqpoint{1.144960in}{1.181211in}}%
\pgfpathlineto{\pgfqpoint{1.146268in}{1.118083in}}%
\pgfpathlineto{\pgfqpoint{1.147139in}{1.139635in}}%
\pgfpathlineto{\pgfqpoint{1.149610in}{1.182316in}}%
\pgfpathlineto{\pgfqpoint{1.150191in}{1.181148in}}%
\pgfpathlineto{\pgfqpoint{1.151499in}{1.172234in}}%
\pgfpathlineto{\pgfqpoint{1.152952in}{1.094007in}}%
\pgfpathlineto{\pgfqpoint{1.153679in}{1.147566in}}%
\pgfpathlineto{\pgfqpoint{1.154696in}{1.199169in}}%
\pgfpathlineto{\pgfqpoint{1.155422in}{1.183089in}}%
\pgfpathlineto{\pgfqpoint{1.158619in}{1.142130in}}%
\pgfpathlineto{\pgfqpoint{1.159055in}{1.141752in}}%
\pgfpathlineto{\pgfqpoint{1.159346in}{1.142922in}}%
\pgfpathlineto{\pgfqpoint{1.160218in}{1.173624in}}%
\pgfpathlineto{\pgfqpoint{1.161380in}{1.215146in}}%
\pgfpathlineto{\pgfqpoint{1.162107in}{1.196788in}}%
\pgfpathlineto{\pgfqpoint{1.163414in}{1.150515in}}%
\pgfpathlineto{\pgfqpoint{1.164141in}{1.164386in}}%
\pgfpathlineto{\pgfqpoint{1.164577in}{1.169394in}}%
\pgfpathlineto{\pgfqpoint{1.165304in}{1.159767in}}%
\pgfpathlineto{\pgfqpoint{1.165594in}{1.156298in}}%
\pgfpathlineto{\pgfqpoint{1.166175in}{1.168302in}}%
\pgfpathlineto{\pgfqpoint{1.167483in}{1.212447in}}%
\pgfpathlineto{\pgfqpoint{1.168210in}{1.200115in}}%
\pgfpathlineto{\pgfqpoint{1.170099in}{1.138404in}}%
\pgfpathlineto{\pgfqpoint{1.170680in}{1.159393in}}%
\pgfpathlineto{\pgfqpoint{1.171843in}{1.203830in}}%
\pgfpathlineto{\pgfqpoint{1.172860in}{1.196557in}}%
\pgfpathlineto{\pgfqpoint{1.173877in}{1.207910in}}%
\pgfpathlineto{\pgfqpoint{1.174458in}{1.198231in}}%
\pgfpathlineto{\pgfqpoint{1.175911in}{1.131702in}}%
\pgfpathlineto{\pgfqpoint{1.176638in}{1.163132in}}%
\pgfpathlineto{\pgfqpoint{1.178527in}{1.226524in}}%
\pgfpathlineto{\pgfqpoint{1.179108in}{1.221681in}}%
\pgfpathlineto{\pgfqpoint{1.182305in}{1.172581in}}%
\pgfpathlineto{\pgfqpoint{1.182741in}{1.178213in}}%
\pgfpathlineto{\pgfqpoint{1.184485in}{1.239898in}}%
\pgfpathlineto{\pgfqpoint{1.185211in}{1.218473in}}%
\pgfpathlineto{\pgfqpoint{1.187827in}{1.141792in}}%
\pgfpathlineto{\pgfqpoint{1.188118in}{1.144228in}}%
\pgfpathlineto{\pgfqpoint{1.191024in}{1.219393in}}%
\pgfpathlineto{\pgfqpoint{1.191750in}{1.202165in}}%
\pgfpathlineto{\pgfqpoint{1.192913in}{1.151308in}}%
\pgfpathlineto{\pgfqpoint{1.193639in}{1.179939in}}%
\pgfpathlineto{\pgfqpoint{1.196255in}{1.234587in}}%
\pgfpathlineto{\pgfqpoint{1.196691in}{1.236737in}}%
\pgfpathlineto{\pgfqpoint{1.197272in}{1.229927in}}%
\pgfpathlineto{\pgfqpoint{1.199307in}{1.169147in}}%
\pgfpathlineto{\pgfqpoint{1.200324in}{1.193293in}}%
\pgfpathlineto{\pgfqpoint{1.201632in}{1.223218in}}%
\pgfpathlineto{\pgfqpoint{1.202358in}{1.216859in}}%
\pgfpathlineto{\pgfqpoint{1.204829in}{1.165637in}}%
\pgfpathlineto{\pgfqpoint{1.205410in}{1.136942in}}%
\pgfpathlineto{\pgfqpoint{1.206136in}{1.168196in}}%
\pgfpathlineto{\pgfqpoint{1.208316in}{1.222392in}}%
\pgfpathlineto{\pgfqpoint{1.208461in}{1.222214in}}%
\pgfpathlineto{\pgfqpoint{1.209188in}{1.210025in}}%
\pgfpathlineto{\pgfqpoint{1.211949in}{1.166122in}}%
\pgfpathlineto{\pgfqpoint{1.212094in}{1.166590in}}%
\pgfpathlineto{\pgfqpoint{1.212966in}{1.201226in}}%
\pgfpathlineto{\pgfqpoint{1.214129in}{1.236119in}}%
\pgfpathlineto{\pgfqpoint{1.214855in}{1.223235in}}%
\pgfpathlineto{\pgfqpoint{1.217325in}{1.176008in}}%
\pgfpathlineto{\pgfqpoint{1.217761in}{1.178451in}}%
\pgfpathlineto{\pgfqpoint{1.220522in}{1.228575in}}%
\pgfpathlineto{\pgfqpoint{1.221394in}{1.215995in}}%
\pgfpathlineto{\pgfqpoint{1.222702in}{1.158987in}}%
\pgfpathlineto{\pgfqpoint{1.223429in}{1.184351in}}%
\pgfpathlineto{\pgfqpoint{1.225899in}{1.231707in}}%
\pgfpathlineto{\pgfqpoint{1.226480in}{1.237088in}}%
\pgfpathlineto{\pgfqpoint{1.227061in}{1.229163in}}%
\pgfpathlineto{\pgfqpoint{1.229096in}{1.186624in}}%
\pgfpathlineto{\pgfqpoint{1.229822in}{1.196136in}}%
\pgfpathlineto{\pgfqpoint{1.231566in}{1.236749in}}%
\pgfpathlineto{\pgfqpoint{1.232438in}{1.228073in}}%
\pgfpathlineto{\pgfqpoint{1.235199in}{1.170526in}}%
\pgfpathlineto{\pgfqpoint{1.235925in}{1.192869in}}%
\pgfpathlineto{\pgfqpoint{1.237233in}{1.228386in}}%
\pgfpathlineto{\pgfqpoint{1.237960in}{1.224032in}}%
\pgfpathlineto{\pgfqpoint{1.239413in}{1.202105in}}%
\pgfpathlineto{\pgfqpoint{1.240285in}{1.184844in}}%
\pgfpathlineto{\pgfqpoint{1.241011in}{1.193820in}}%
\pgfpathlineto{\pgfqpoint{1.243772in}{1.240384in}}%
\pgfpathlineto{\pgfqpoint{1.244354in}{1.230356in}}%
\pgfpathlineto{\pgfqpoint{1.245661in}{1.189666in}}%
\pgfpathlineto{\pgfqpoint{1.246533in}{1.203327in}}%
\pgfpathlineto{\pgfqpoint{1.249294in}{1.244306in}}%
\pgfpathlineto{\pgfqpoint{1.249875in}{1.240982in}}%
\pgfpathlineto{\pgfqpoint{1.252346in}{1.193127in}}%
\pgfpathlineto{\pgfqpoint{1.253218in}{1.209970in}}%
\pgfpathlineto{\pgfqpoint{1.254525in}{1.235585in}}%
\pgfpathlineto{\pgfqpoint{1.255397in}{1.232419in}}%
\pgfpathlineto{\pgfqpoint{1.256560in}{1.223255in}}%
\pgfpathlineto{\pgfqpoint{1.258158in}{1.186652in}}%
\pgfpathlineto{\pgfqpoint{1.259030in}{1.200296in}}%
\pgfpathlineto{\pgfqpoint{1.261210in}{1.243052in}}%
\pgfpathlineto{\pgfqpoint{1.261936in}{1.235612in}}%
\pgfpathlineto{\pgfqpoint{1.264407in}{1.203168in}}%
\pgfpathlineto{\pgfqpoint{1.264988in}{1.207996in}}%
\pgfpathlineto{\pgfqpoint{1.267022in}{1.253582in}}%
\pgfpathlineto{\pgfqpoint{1.268039in}{1.240294in}}%
\pgfpathlineto{\pgfqpoint{1.270219in}{1.209618in}}%
\pgfpathlineto{\pgfqpoint{1.270655in}{1.212244in}}%
\pgfpathlineto{\pgfqpoint{1.273125in}{1.254121in}}%
\pgfpathlineto{\pgfqpoint{1.273997in}{1.242476in}}%
\pgfpathlineto{\pgfqpoint{1.275596in}{1.207222in}}%
\pgfpathlineto{\pgfqpoint{1.276468in}{1.213630in}}%
\pgfpathlineto{\pgfqpoint{1.279374in}{1.240590in}}%
\pgfpathlineto{\pgfqpoint{1.279810in}{1.237729in}}%
\pgfpathlineto{\pgfqpoint{1.281263in}{1.212892in}}%
\pgfpathlineto{\pgfqpoint{1.282135in}{1.223027in}}%
\pgfpathlineto{\pgfqpoint{1.284460in}{1.258292in}}%
\pgfpathlineto{\pgfqpoint{1.285186in}{1.255101in}}%
\pgfpathlineto{\pgfqpoint{1.287075in}{1.242994in}}%
\pgfpathlineto{\pgfqpoint{1.287947in}{1.231671in}}%
\pgfpathlineto{\pgfqpoint{1.288674in}{1.240370in}}%
\pgfpathlineto{\pgfqpoint{1.290272in}{1.261319in}}%
\pgfpathlineto{\pgfqpoint{1.290999in}{1.255523in}}%
\pgfpathlineto{\pgfqpoint{1.294196in}{1.218118in}}%
\pgfpathlineto{\pgfqpoint{1.294486in}{1.219261in}}%
\pgfpathlineto{\pgfqpoint{1.295794in}{1.244462in}}%
\pgfpathlineto{\pgfqpoint{1.296666in}{1.256435in}}%
\pgfpathlineto{\pgfqpoint{1.297247in}{1.248662in}}%
\pgfpathlineto{\pgfqpoint{1.298555in}{1.218711in}}%
\pgfpathlineto{\pgfqpoint{1.299427in}{1.227261in}}%
\pgfpathlineto{\pgfqpoint{1.303060in}{1.257331in}}%
\pgfpathlineto{\pgfqpoint{1.303350in}{1.256238in}}%
\pgfpathlineto{\pgfqpoint{1.305239in}{1.227344in}}%
\pgfpathlineto{\pgfqpoint{1.306402in}{1.242342in}}%
\pgfpathlineto{\pgfqpoint{1.309018in}{1.254663in}}%
\pgfpathlineto{\pgfqpoint{1.309163in}{1.254918in}}%
\pgfpathlineto{\pgfqpoint{1.309599in}{1.252862in}}%
\pgfpathlineto{\pgfqpoint{1.311633in}{1.221056in}}%
\pgfpathlineto{\pgfqpoint{1.312650in}{1.234771in}}%
\pgfpathlineto{\pgfqpoint{1.313813in}{1.246605in}}%
\pgfpathlineto{\pgfqpoint{1.314539in}{1.244263in}}%
\pgfpathlineto{\pgfqpoint{1.316429in}{1.231633in}}%
\pgfpathlineto{\pgfqpoint{1.317882in}{1.212164in}}%
\pgfpathlineto{\pgfqpoint{1.318463in}{1.221445in}}%
\pgfpathlineto{\pgfqpoint{1.320207in}{1.251006in}}%
\pgfpathlineto{\pgfqpoint{1.320933in}{1.250529in}}%
\pgfpathlineto{\pgfqpoint{1.321660in}{1.247652in}}%
\pgfpathlineto{\pgfqpoint{1.324421in}{1.235901in}}%
\pgfpathlineto{\pgfqpoint{1.324566in}{1.236314in}}%
\pgfpathlineto{\pgfqpoint{1.325729in}{1.256399in}}%
\pgfpathlineto{\pgfqpoint{1.326164in}{1.260250in}}%
\pgfpathlineto{\pgfqpoint{1.327036in}{1.254264in}}%
\pgfpathlineto{\pgfqpoint{1.328925in}{1.228726in}}%
\pgfpathlineto{\pgfqpoint{1.329943in}{1.236319in}}%
\pgfpathlineto{\pgfqpoint{1.331541in}{1.251528in}}%
\pgfpathlineto{\pgfqpoint{1.332558in}{1.264721in}}%
\pgfpathlineto{\pgfqpoint{1.333430in}{1.260171in}}%
\pgfpathlineto{\pgfqpoint{1.334883in}{1.244337in}}%
\pgfpathlineto{\pgfqpoint{1.335610in}{1.251501in}}%
\pgfpathlineto{\pgfqpoint{1.338661in}{1.283023in}}%
\pgfpathlineto{\pgfqpoint{1.339243in}{1.279498in}}%
\pgfpathlineto{\pgfqpoint{1.341422in}{1.251921in}}%
\pgfpathlineto{\pgfqpoint{1.342294in}{1.260923in}}%
\pgfpathlineto{\pgfqpoint{1.343166in}{1.267270in}}%
\pgfpathlineto{\pgfqpoint{1.343893in}{1.262107in}}%
\pgfpathlineto{\pgfqpoint{1.347525in}{1.225559in}}%
\pgfpathlineto{\pgfqpoint{1.348107in}{1.234463in}}%
\pgfpathlineto{\pgfqpoint{1.349705in}{1.266693in}}%
\pgfpathlineto{\pgfqpoint{1.350577in}{1.260639in}}%
\pgfpathlineto{\pgfqpoint{1.352321in}{1.250655in}}%
\pgfpathlineto{\pgfqpoint{1.352757in}{1.252081in}}%
\pgfpathlineto{\pgfqpoint{1.355954in}{1.282718in}}%
\pgfpathlineto{\pgfqpoint{1.356825in}{1.272484in}}%
\pgfpathlineto{\pgfqpoint{1.357988in}{1.254638in}}%
\pgfpathlineto{\pgfqpoint{1.358860in}{1.259992in}}%
\pgfpathlineto{\pgfqpoint{1.361766in}{1.277265in}}%
\pgfpathlineto{\pgfqpoint{1.362493in}{1.274321in}}%
\pgfpathlineto{\pgfqpoint{1.364672in}{1.246031in}}%
\pgfpathlineto{\pgfqpoint{1.365689in}{1.256237in}}%
\pgfpathlineto{\pgfqpoint{1.368450in}{1.284445in}}%
\pgfpathlineto{\pgfqpoint{1.368741in}{1.283906in}}%
\pgfpathlineto{\pgfqpoint{1.369613in}{1.271383in}}%
\pgfpathlineto{\pgfqpoint{1.370485in}{1.257487in}}%
\pgfpathlineto{\pgfqpoint{1.371211in}{1.267371in}}%
\pgfpathlineto{\pgfqpoint{1.373391in}{1.296218in}}%
\pgfpathlineto{\pgfqpoint{1.373827in}{1.293608in}}%
\pgfpathlineto{\pgfqpoint{1.376879in}{1.263309in}}%
\pgfpathlineto{\pgfqpoint{1.377314in}{1.266379in}}%
\pgfpathlineto{\pgfqpoint{1.379349in}{1.290830in}}%
\pgfpathlineto{\pgfqpoint{1.380075in}{1.287041in}}%
\pgfpathlineto{\pgfqpoint{1.381819in}{1.267730in}}%
\pgfpathlineto{\pgfqpoint{1.382836in}{1.270378in}}%
\pgfpathlineto{\pgfqpoint{1.383999in}{1.284350in}}%
\pgfpathlineto{\pgfqpoint{1.385888in}{1.301352in}}%
\pgfpathlineto{\pgfqpoint{1.386324in}{1.299412in}}%
\pgfpathlineto{\pgfqpoint{1.387632in}{1.285390in}}%
\pgfpathlineto{\pgfqpoint{1.388504in}{1.292200in}}%
\pgfpathlineto{\pgfqpoint{1.391410in}{1.315574in}}%
\pgfpathlineto{\pgfqpoint{1.391846in}{1.314685in}}%
\pgfpathlineto{\pgfqpoint{1.393154in}{1.300033in}}%
\pgfpathlineto{\pgfqpoint{1.394316in}{1.286494in}}%
\pgfpathlineto{\pgfqpoint{1.395043in}{1.292516in}}%
\pgfpathlineto{\pgfqpoint{1.396350in}{1.305093in}}%
\pgfpathlineto{\pgfqpoint{1.397077in}{1.301700in}}%
\pgfpathlineto{\pgfqpoint{1.400274in}{1.276649in}}%
\pgfpathlineto{\pgfqpoint{1.400855in}{1.282544in}}%
\pgfpathlineto{\pgfqpoint{1.402889in}{1.310831in}}%
\pgfpathlineto{\pgfqpoint{1.403761in}{1.309191in}}%
\pgfpathlineto{\pgfqpoint{1.404924in}{1.306204in}}%
\pgfpathlineto{\pgfqpoint{1.405505in}{1.307916in}}%
\pgfpathlineto{\pgfqpoint{1.408847in}{1.330465in}}%
\pgfpathlineto{\pgfqpoint{1.409429in}{1.324846in}}%
\pgfpathlineto{\pgfqpoint{1.411899in}{1.307834in}}%
\pgfpathlineto{\pgfqpoint{1.413788in}{1.305643in}}%
\pgfpathlineto{\pgfqpoint{1.414079in}{1.306417in}}%
\pgfpathlineto{\pgfqpoint{1.415241in}{1.312889in}}%
\pgfpathlineto{\pgfqpoint{1.415968in}{1.308367in}}%
\pgfpathlineto{\pgfqpoint{1.417421in}{1.299836in}}%
\pgfpathlineto{\pgfqpoint{1.418002in}{1.301412in}}%
\pgfpathlineto{\pgfqpoint{1.421635in}{1.331981in}}%
\pgfpathlineto{\pgfqpoint{1.422797in}{1.321773in}}%
\pgfpathlineto{\pgfqpoint{1.423960in}{1.316885in}}%
\pgfpathlineto{\pgfqpoint{1.424541in}{1.319051in}}%
\pgfpathlineto{\pgfqpoint{1.425994in}{1.332174in}}%
\pgfpathlineto{\pgfqpoint{1.426866in}{1.327395in}}%
\pgfpathlineto{\pgfqpoint{1.428900in}{1.309424in}}%
\pgfpathlineto{\pgfqpoint{1.429772in}{1.300984in}}%
\pgfpathlineto{\pgfqpoint{1.430499in}{1.306697in}}%
\pgfpathlineto{\pgfqpoint{1.432243in}{1.323107in}}%
\pgfpathlineto{\pgfqpoint{1.432969in}{1.319422in}}%
\pgfpathlineto{\pgfqpoint{1.433696in}{1.316688in}}%
\pgfpathlineto{\pgfqpoint{1.434713in}{1.318068in}}%
\pgfpathlineto{\pgfqpoint{1.435439in}{1.317031in}}%
\pgfpathlineto{\pgfqpoint{1.435875in}{1.318297in}}%
\pgfpathlineto{\pgfqpoint{1.437910in}{1.341217in}}%
\pgfpathlineto{\pgfqpoint{1.438636in}{1.345571in}}%
\pgfpathlineto{\pgfqpoint{1.439363in}{1.340372in}}%
\pgfpathlineto{\pgfqpoint{1.440235in}{1.334652in}}%
\pgfpathlineto{\pgfqpoint{1.440961in}{1.338377in}}%
\pgfpathlineto{\pgfqpoint{1.441397in}{1.339660in}}%
\pgfpathlineto{\pgfqpoint{1.442269in}{1.337195in}}%
\pgfpathlineto{\pgfqpoint{1.442560in}{1.337004in}}%
\pgfpathlineto{\pgfqpoint{1.442996in}{1.338698in}}%
\pgfpathlineto{\pgfqpoint{1.444013in}{1.343701in}}%
\pgfpathlineto{\pgfqpoint{1.444739in}{1.341294in}}%
\pgfpathlineto{\pgfqpoint{1.446919in}{1.322063in}}%
\pgfpathlineto{\pgfqpoint{1.447936in}{1.328979in}}%
\pgfpathlineto{\pgfqpoint{1.448518in}{1.331143in}}%
\pgfpathlineto{\pgfqpoint{1.449389in}{1.328538in}}%
\pgfpathlineto{\pgfqpoint{1.449825in}{1.329421in}}%
\pgfpathlineto{\pgfqpoint{1.451133in}{1.337015in}}%
\pgfpathlineto{\pgfqpoint{1.451860in}{1.334072in}}%
\pgfpathlineto{\pgfqpoint{1.452877in}{1.329312in}}%
\pgfpathlineto{\pgfqpoint{1.453458in}{1.332574in}}%
\pgfpathlineto{\pgfqpoint{1.456655in}{1.350898in}}%
\pgfpathlineto{\pgfqpoint{1.458399in}{1.350201in}}%
\pgfpathlineto{\pgfqpoint{1.459707in}{1.342105in}}%
\pgfpathlineto{\pgfqpoint{1.460433in}{1.348215in}}%
\pgfpathlineto{\pgfqpoint{1.461305in}{1.355145in}}%
\pgfpathlineto{\pgfqpoint{1.462032in}{1.350864in}}%
\pgfpathlineto{\pgfqpoint{1.464502in}{1.339998in}}%
\pgfpathlineto{\pgfqpoint{1.464938in}{1.340782in}}%
\pgfpathlineto{\pgfqpoint{1.466972in}{1.351647in}}%
\pgfpathlineto{\pgfqpoint{1.467989in}{1.358387in}}%
\pgfpathlineto{\pgfqpoint{1.468861in}{1.356498in}}%
\pgfpathlineto{\pgfqpoint{1.469733in}{1.355016in}}%
\pgfpathlineto{\pgfqpoint{1.470314in}{1.356693in}}%
\pgfpathlineto{\pgfqpoint{1.473947in}{1.373456in}}%
\pgfpathlineto{\pgfqpoint{1.474238in}{1.372732in}}%
\pgfpathlineto{\pgfqpoint{1.476708in}{1.363698in}}%
\pgfpathlineto{\pgfqpoint{1.477435in}{1.365232in}}%
\pgfpathlineto{\pgfqpoint{1.478016in}{1.366566in}}%
\pgfpathlineto{\pgfqpoint{1.478597in}{1.364601in}}%
\pgfpathlineto{\pgfqpoint{1.482230in}{1.348303in}}%
\pgfpathlineto{\pgfqpoint{1.482375in}{1.348471in}}%
\pgfpathlineto{\pgfqpoint{1.483538in}{1.356492in}}%
\pgfpathlineto{\pgfqpoint{1.484700in}{1.361313in}}%
\pgfpathlineto{\pgfqpoint{1.485427in}{1.360279in}}%
\pgfpathlineto{\pgfqpoint{1.485863in}{1.359820in}}%
\pgfpathlineto{\pgfqpoint{1.486444in}{1.361285in}}%
\pgfpathlineto{\pgfqpoint{1.491239in}{1.379215in}}%
\pgfpathlineto{\pgfqpoint{1.491385in}{1.379074in}}%
\pgfpathlineto{\pgfqpoint{1.492838in}{1.373489in}}%
\pgfpathlineto{\pgfqpoint{1.494146in}{1.376514in}}%
\pgfpathlineto{\pgfqpoint{1.499377in}{1.363789in}}%
\pgfpathlineto{\pgfqpoint{1.499958in}{1.367243in}}%
\pgfpathlineto{\pgfqpoint{1.501121in}{1.373434in}}%
\pgfpathlineto{\pgfqpoint{1.501993in}{1.372459in}}%
\pgfpathlineto{\pgfqpoint{1.502574in}{1.372121in}}%
\pgfpathlineto{\pgfqpoint{1.503155in}{1.373353in}}%
\pgfpathlineto{\pgfqpoint{1.504463in}{1.377670in}}%
\pgfpathlineto{\pgfqpoint{1.505044in}{1.375754in}}%
\pgfpathlineto{\pgfqpoint{1.505625in}{1.373869in}}%
\pgfpathlineto{\pgfqpoint{1.506352in}{1.376167in}}%
\pgfpathlineto{\pgfqpoint{1.507514in}{1.380606in}}%
\pgfpathlineto{\pgfqpoint{1.508241in}{1.379136in}}%
\pgfpathlineto{\pgfqpoint{1.509549in}{1.375055in}}%
\pgfpathlineto{\pgfqpoint{1.510275in}{1.376873in}}%
\pgfpathlineto{\pgfqpoint{1.510857in}{1.377826in}}%
\pgfpathlineto{\pgfqpoint{1.511438in}{1.375508in}}%
\pgfpathlineto{\pgfqpoint{1.512310in}{1.371966in}}%
\pgfpathlineto{\pgfqpoint{1.513036in}{1.374691in}}%
\pgfpathlineto{\pgfqpoint{1.514635in}{1.379013in}}%
\pgfpathlineto{\pgfqpoint{1.515216in}{1.378452in}}%
\pgfpathlineto{\pgfqpoint{1.516379in}{1.376792in}}%
\pgfpathlineto{\pgfqpoint{1.516960in}{1.378144in}}%
\pgfpathlineto{\pgfqpoint{1.521174in}{1.390641in}}%
\pgfpathlineto{\pgfqpoint{1.521755in}{1.389623in}}%
\pgfpathlineto{\pgfqpoint{1.522627in}{1.388336in}}%
\pgfpathlineto{\pgfqpoint{1.523208in}{1.389646in}}%
\pgfpathlineto{\pgfqpoint{1.526841in}{1.397184in}}%
\pgfpathlineto{\pgfqpoint{1.528004in}{1.395191in}}%
\pgfpathlineto{\pgfqpoint{1.529457in}{1.390727in}}%
\pgfpathlineto{\pgfqpoint{1.530183in}{1.392894in}}%
\pgfpathlineto{\pgfqpoint{1.530910in}{1.395498in}}%
\pgfpathlineto{\pgfqpoint{1.531636in}{1.392824in}}%
\pgfpathlineto{\pgfqpoint{1.533816in}{1.388367in}}%
\pgfpathlineto{\pgfqpoint{1.535414in}{1.387937in}}%
\pgfpathlineto{\pgfqpoint{1.535705in}{1.388392in}}%
\pgfpathlineto{\pgfqpoint{1.537158in}{1.397189in}}%
\pgfpathlineto{\pgfqpoint{1.537739in}{1.398521in}}%
\pgfpathlineto{\pgfqpoint{1.538466in}{1.396487in}}%
\pgfpathlineto{\pgfqpoint{1.539047in}{1.395650in}}%
\pgfpathlineto{\pgfqpoint{1.539629in}{1.397094in}}%
\pgfpathlineto{\pgfqpoint{1.543843in}{1.410583in}}%
\pgfpathlineto{\pgfqpoint{1.543988in}{1.410474in}}%
\pgfpathlineto{\pgfqpoint{1.545441in}{1.405610in}}%
\pgfpathlineto{\pgfqpoint{1.546313in}{1.408079in}}%
\pgfpathlineto{\pgfqpoint{1.547766in}{1.412670in}}%
\pgfpathlineto{\pgfqpoint{1.548493in}{1.410705in}}%
\pgfpathlineto{\pgfqpoint{1.549364in}{1.407660in}}%
\pgfpathlineto{\pgfqpoint{1.550382in}{1.409012in}}%
\pgfpathlineto{\pgfqpoint{1.551399in}{1.406447in}}%
\pgfpathlineto{\pgfqpoint{1.551835in}{1.405948in}}%
\pgfpathlineto{\pgfqpoint{1.552416in}{1.407749in}}%
\pgfpathlineto{\pgfqpoint{1.554014in}{1.415961in}}%
\pgfpathlineto{\pgfqpoint{1.554741in}{1.412639in}}%
\pgfpathlineto{\pgfqpoint{1.555613in}{1.409057in}}%
\pgfpathlineto{\pgfqpoint{1.556485in}{1.410819in}}%
\pgfpathlineto{\pgfqpoint{1.560408in}{1.419217in}}%
\pgfpathlineto{\pgfqpoint{1.560989in}{1.417434in}}%
\pgfpathlineto{\pgfqpoint{1.562152in}{1.410306in}}%
\pgfpathlineto{\pgfqpoint{1.562879in}{1.413694in}}%
\pgfpathlineto{\pgfqpoint{1.563896in}{1.417809in}}%
\pgfpathlineto{\pgfqpoint{1.564768in}{1.416765in}}%
\pgfpathlineto{\pgfqpoint{1.565785in}{1.418134in}}%
\pgfpathlineto{\pgfqpoint{1.567093in}{1.419569in}}%
\pgfpathlineto{\pgfqpoint{1.567674in}{1.418562in}}%
\pgfpathlineto{\pgfqpoint{1.568836in}{1.414138in}}%
\pgfpathlineto{\pgfqpoint{1.569563in}{1.417160in}}%
\pgfpathlineto{\pgfqpoint{1.570871in}{1.424403in}}%
\pgfpathlineto{\pgfqpoint{1.571743in}{1.423137in}}%
\pgfpathlineto{\pgfqpoint{1.572324in}{1.422798in}}%
\pgfpathlineto{\pgfqpoint{1.572905in}{1.423680in}}%
\pgfpathlineto{\pgfqpoint{1.573922in}{1.425153in}}%
\pgfpathlineto{\pgfqpoint{1.574649in}{1.423992in}}%
\pgfpathlineto{\pgfqpoint{1.575230in}{1.423393in}}%
\pgfpathlineto{\pgfqpoint{1.575666in}{1.424524in}}%
\pgfpathlineto{\pgfqpoint{1.578136in}{1.432341in}}%
\pgfpathlineto{\pgfqpoint{1.578718in}{1.431704in}}%
\pgfpathlineto{\pgfqpoint{1.579589in}{1.430765in}}%
\pgfpathlineto{\pgfqpoint{1.580171in}{1.431933in}}%
\pgfpathlineto{\pgfqpoint{1.580607in}{1.432395in}}%
\pgfpathlineto{\pgfqpoint{1.581188in}{1.430678in}}%
\pgfpathlineto{\pgfqpoint{1.582060in}{1.427681in}}%
\pgfpathlineto{\pgfqpoint{1.582932in}{1.429485in}}%
\pgfpathlineto{\pgfqpoint{1.583513in}{1.430087in}}%
\pgfpathlineto{\pgfqpoint{1.584385in}{1.429237in}}%
\pgfpathlineto{\pgfqpoint{1.586129in}{1.427161in}}%
\pgfpathlineto{\pgfqpoint{1.586855in}{1.428670in}}%
\pgfpathlineto{\pgfqpoint{1.587291in}{1.428734in}}%
\pgfpathlineto{\pgfqpoint{1.587727in}{1.427684in}}%
\pgfpathlineto{\pgfqpoint{1.588308in}{1.426515in}}%
\pgfpathlineto{\pgfqpoint{1.588889in}{1.428026in}}%
\pgfpathlineto{\pgfqpoint{1.591214in}{1.437671in}}%
\pgfpathlineto{\pgfqpoint{1.592086in}{1.436215in}}%
\pgfpathlineto{\pgfqpoint{1.592377in}{1.435971in}}%
\pgfpathlineto{\pgfqpoint{1.592958in}{1.437315in}}%
\pgfpathlineto{\pgfqpoint{1.596155in}{1.443744in}}%
\pgfpathlineto{\pgfqpoint{1.597027in}{1.443718in}}%
\pgfpathlineto{\pgfqpoint{1.597463in}{1.443149in}}%
\pgfpathlineto{\pgfqpoint{1.599352in}{1.437009in}}%
\pgfpathlineto{\pgfqpoint{1.600660in}{1.439835in}}%
\pgfpathlineto{\pgfqpoint{1.603711in}{1.443982in}}%
\pgfpathlineto{\pgfqpoint{1.603857in}{1.443862in}}%
\pgfpathlineto{\pgfqpoint{1.605455in}{1.440357in}}%
\pgfpathlineto{\pgfqpoint{1.606182in}{1.442272in}}%
\pgfpathlineto{\pgfqpoint{1.608507in}{1.450055in}}%
\pgfpathlineto{\pgfqpoint{1.609233in}{1.449240in}}%
\pgfpathlineto{\pgfqpoint{1.611994in}{1.444189in}}%
\pgfpathlineto{\pgfqpoint{1.612575in}{1.445866in}}%
\pgfpathlineto{\pgfqpoint{1.614029in}{1.451435in}}%
\pgfpathlineto{\pgfqpoint{1.614900in}{1.450181in}}%
\pgfpathlineto{\pgfqpoint{1.616935in}{1.444827in}}%
\pgfpathlineto{\pgfqpoint{1.617952in}{1.446583in}}%
\pgfpathlineto{\pgfqpoint{1.619696in}{1.454116in}}%
\pgfpathlineto{\pgfqpoint{1.620422in}{1.456281in}}%
\pgfpathlineto{\pgfqpoint{1.621149in}{1.454228in}}%
\pgfpathlineto{\pgfqpoint{1.622602in}{1.449881in}}%
\pgfpathlineto{\pgfqpoint{1.623183in}{1.451318in}}%
\pgfpathlineto{\pgfqpoint{1.626380in}{1.462103in}}%
\pgfpathlineto{\pgfqpoint{1.626816in}{1.461566in}}%
\pgfpathlineto{\pgfqpoint{1.628850in}{1.455134in}}%
\pgfpathlineto{\pgfqpoint{1.629577in}{1.457359in}}%
\pgfpathlineto{\pgfqpoint{1.631030in}{1.464880in}}%
\pgfpathlineto{\pgfqpoint{1.632047in}{1.463504in}}%
\pgfpathlineto{\pgfqpoint{1.633500in}{1.460193in}}%
\pgfpathlineto{\pgfqpoint{1.634663in}{1.457250in}}%
\pgfpathlineto{\pgfqpoint{1.635389in}{1.458343in}}%
\pgfpathlineto{\pgfqpoint{1.637279in}{1.466923in}}%
\pgfpathlineto{\pgfqpoint{1.638296in}{1.464027in}}%
\pgfpathlineto{\pgfqpoint{1.640475in}{1.459304in}}%
\pgfpathlineto{\pgfqpoint{1.640766in}{1.459573in}}%
\pgfpathlineto{\pgfqpoint{1.642219in}{1.464606in}}%
\pgfpathlineto{\pgfqpoint{1.643527in}{1.469856in}}%
\pgfpathlineto{\pgfqpoint{1.644254in}{1.467824in}}%
\pgfpathlineto{\pgfqpoint{1.645997in}{1.460851in}}%
\pgfpathlineto{\pgfqpoint{1.646724in}{1.461742in}}%
\pgfpathlineto{\pgfqpoint{1.649921in}{1.471019in}}%
\pgfpathlineto{\pgfqpoint{1.650793in}{1.468799in}}%
\pgfpathlineto{\pgfqpoint{1.652100in}{1.464141in}}%
\pgfpathlineto{\pgfqpoint{1.652682in}{1.466233in}}%
\pgfpathlineto{\pgfqpoint{1.655443in}{1.476806in}}%
\pgfpathlineto{\pgfqpoint{1.655879in}{1.476413in}}%
\pgfpathlineto{\pgfqpoint{1.658204in}{1.471197in}}%
\pgfpathlineto{\pgfqpoint{1.659075in}{1.474240in}}%
\pgfpathlineto{\pgfqpoint{1.661110in}{1.482556in}}%
\pgfpathlineto{\pgfqpoint{1.661691in}{1.481258in}}%
\pgfpathlineto{\pgfqpoint{1.664597in}{1.473506in}}%
\pgfpathlineto{\pgfqpoint{1.665033in}{1.474666in}}%
\pgfpathlineto{\pgfqpoint{1.667358in}{1.482678in}}%
\pgfpathlineto{\pgfqpoint{1.667939in}{1.481488in}}%
\pgfpathlineto{\pgfqpoint{1.669538in}{1.474632in}}%
\pgfpathlineto{\pgfqpoint{1.670555in}{1.476400in}}%
\pgfpathlineto{\pgfqpoint{1.673316in}{1.486055in}}%
\pgfpathlineto{\pgfqpoint{1.674333in}{1.483276in}}%
\pgfpathlineto{\pgfqpoint{1.675350in}{1.480240in}}%
\pgfpathlineto{\pgfqpoint{1.676077in}{1.481788in}}%
\pgfpathlineto{\pgfqpoint{1.679564in}{1.490549in}}%
\pgfpathlineto{\pgfqpoint{1.679855in}{1.490366in}}%
\pgfpathlineto{\pgfqpoint{1.681018in}{1.487000in}}%
\pgfpathlineto{\pgfqpoint{1.681599in}{1.485861in}}%
\pgfpathlineto{\pgfqpoint{1.682180in}{1.487454in}}%
\pgfpathlineto{\pgfqpoint{1.684360in}{1.494914in}}%
\pgfpathlineto{\pgfqpoint{1.684941in}{1.494340in}}%
\pgfpathlineto{\pgfqpoint{1.687702in}{1.490161in}}%
\pgfpathlineto{\pgfqpoint{1.688429in}{1.491692in}}%
\pgfpathlineto{\pgfqpoint{1.690318in}{1.497085in}}%
\pgfpathlineto{\pgfqpoint{1.691044in}{1.495385in}}%
\pgfpathlineto{\pgfqpoint{1.693369in}{1.492462in}}%
\pgfpathlineto{\pgfqpoint{1.694532in}{1.493518in}}%
\pgfpathlineto{\pgfqpoint{1.696421in}{1.497863in}}%
\pgfpathlineto{\pgfqpoint{1.697147in}{1.496401in}}%
\pgfpathlineto{\pgfqpoint{1.698891in}{1.493244in}}%
\pgfpathlineto{\pgfqpoint{1.699472in}{1.494058in}}%
\pgfpathlineto{\pgfqpoint{1.702814in}{1.500437in}}%
\pgfpathlineto{\pgfqpoint{1.703105in}{1.500156in}}%
\pgfpathlineto{\pgfqpoint{1.704413in}{1.497088in}}%
\pgfpathlineto{\pgfqpoint{1.705139in}{1.498674in}}%
\pgfpathlineto{\pgfqpoint{1.707755in}{1.505731in}}%
\pgfpathlineto{\pgfqpoint{1.708482in}{1.504820in}}%
\pgfpathlineto{\pgfqpoint{1.710807in}{1.503520in}}%
\pgfpathlineto{\pgfqpoint{1.711097in}{1.503743in}}%
\pgfpathlineto{\pgfqpoint{1.712405in}{1.506462in}}%
\pgfpathlineto{\pgfqpoint{1.713422in}{1.509355in}}%
\pgfpathlineto{\pgfqpoint{1.714149in}{1.507526in}}%
\pgfpathlineto{\pgfqpoint{1.716474in}{1.504751in}}%
\pgfpathlineto{\pgfqpoint{1.717491in}{1.506766in}}%
\pgfpathlineto{\pgfqpoint{1.720107in}{1.509614in}}%
\pgfpathlineto{\pgfqpoint{1.720252in}{1.509446in}}%
\pgfpathlineto{\pgfqpoint{1.721996in}{1.505528in}}%
\pgfpathlineto{\pgfqpoint{1.722868in}{1.507565in}}%
\pgfpathlineto{\pgfqpoint{1.725774in}{1.511907in}}%
\pgfpathlineto{\pgfqpoint{1.729116in}{1.512167in}}%
\pgfpathlineto{\pgfqpoint{1.732022in}{1.516147in}}%
\pgfpathlineto{\pgfqpoint{1.733185in}{1.516266in}}%
\pgfpathlineto{\pgfqpoint{1.733621in}{1.515737in}}%
\pgfpathlineto{\pgfqpoint{1.734493in}{1.515462in}}%
\pgfpathlineto{\pgfqpoint{1.734929in}{1.516148in}}%
\pgfpathlineto{\pgfqpoint{1.737835in}{1.520469in}}%
\pgfpathlineto{\pgfqpoint{1.738125in}{1.520252in}}%
\pgfpathlineto{\pgfqpoint{1.740596in}{1.519095in}}%
\pgfpathlineto{\pgfqpoint{1.741613in}{1.519621in}}%
\pgfpathlineto{\pgfqpoint{1.741758in}{1.519992in}}%
\pgfpathlineto{\pgfqpoint{1.743066in}{1.522415in}}%
\pgfpathlineto{\pgfqpoint{1.743793in}{1.521659in}}%
\pgfpathlineto{\pgfqpoint{1.746408in}{1.520664in}}%
\pgfpathlineto{\pgfqpoint{1.748152in}{1.521988in}}%
\pgfpathlineto{\pgfqpoint{1.750622in}{1.523315in}}%
\pgfpathlineto{\pgfqpoint{1.750913in}{1.522919in}}%
\pgfpathlineto{\pgfqpoint{1.751785in}{1.521622in}}%
\pgfpathlineto{\pgfqpoint{1.752511in}{1.523144in}}%
\pgfpathlineto{\pgfqpoint{1.754982in}{1.525796in}}%
\pgfpathlineto{\pgfqpoint{1.758033in}{1.528218in}}%
\pgfpathlineto{\pgfqpoint{1.759486in}{1.529512in}}%
\pgfpathlineto{\pgfqpoint{1.760939in}{1.531680in}}%
\pgfpathlineto{\pgfqpoint{1.761811in}{1.530908in}}%
\pgfpathlineto{\pgfqpoint{1.765008in}{1.532441in}}%
\pgfpathlineto{\pgfqpoint{1.767333in}{1.533793in}}%
\pgfpathlineto{\pgfqpoint{1.772129in}{1.534243in}}%
\pgfpathlineto{\pgfqpoint{1.773436in}{1.535990in}}%
\pgfpathlineto{\pgfqpoint{1.774308in}{1.535064in}}%
\pgfpathlineto{\pgfqpoint{1.776633in}{1.535837in}}%
\pgfpathlineto{\pgfqpoint{1.780411in}{1.537068in}}%
\pgfpathlineto{\pgfqpoint{1.782446in}{1.537844in}}%
\pgfpathlineto{\pgfqpoint{1.785061in}{1.540045in}}%
\pgfpathlineto{\pgfqpoint{1.790293in}{1.544388in}}%
\pgfpathlineto{\pgfqpoint{1.795088in}{1.544545in}}%
\pgfpathlineto{\pgfqpoint{1.796396in}{1.545448in}}%
\pgfpathlineto{\pgfqpoint{1.797122in}{1.544714in}}%
\pgfpathlineto{\pgfqpoint{1.801772in}{1.545375in}}%
\pgfpathlineto{\pgfqpoint{1.806277in}{1.548638in}}%
\pgfpathlineto{\pgfqpoint{1.809764in}{1.550278in}}%
\pgfpathlineto{\pgfqpoint{1.816013in}{1.553543in}}%
\pgfpathlineto{\pgfqpoint{1.817757in}{1.555180in}}%
\pgfpathlineto{\pgfqpoint{1.820808in}{1.556582in}}%
\pgfpathlineto{\pgfqpoint{1.824005in}{1.557354in}}%
\pgfpathlineto{\pgfqpoint{1.827783in}{1.558317in}}%
\pgfpathlineto{\pgfqpoint{1.844930in}{1.564778in}}%
\pgfpathlineto{\pgfqpoint{1.847546in}{1.566031in}}%
\pgfpathlineto{\pgfqpoint{1.864547in}{1.571587in}}%
\pgfpathlineto{\pgfqpoint{1.866436in}{1.574154in}}%
\pgfpathlineto{\pgfqpoint{1.867308in}{1.573432in}}%
\pgfpathlineto{\pgfqpoint{1.868761in}{1.573395in}}%
\pgfpathlineto{\pgfqpoint{1.869052in}{1.573812in}}%
\pgfpathlineto{\pgfqpoint{1.872685in}{1.577574in}}%
\pgfpathlineto{\pgfqpoint{1.875446in}{1.576931in}}%
\pgfpathlineto{\pgfqpoint{1.875591in}{1.577125in}}%
\pgfpathlineto{\pgfqpoint{1.878061in}{1.578148in}}%
\pgfpathlineto{\pgfqpoint{1.886199in}{1.577552in}}%
\pgfpathlineto{\pgfqpoint{1.887652in}{1.578465in}}%
\pgfpathlineto{\pgfqpoint{1.889686in}{1.581240in}}%
\pgfpathlineto{\pgfqpoint{1.890704in}{1.579976in}}%
\pgfpathlineto{\pgfqpoint{1.892011in}{1.580974in}}%
\pgfpathlineto{\pgfqpoint{1.895935in}{1.585589in}}%
\pgfpathlineto{\pgfqpoint{1.896371in}{1.584919in}}%
\pgfpathlineto{\pgfqpoint{1.897679in}{1.583513in}}%
\pgfpathlineto{\pgfqpoint{1.898260in}{1.584143in}}%
\pgfpathlineto{\pgfqpoint{1.901021in}{1.585548in}}%
\pgfpathlineto{\pgfqpoint{1.905525in}{1.587007in}}%
\pgfpathlineto{\pgfqpoint{1.906688in}{1.586151in}}%
\pgfpathlineto{\pgfqpoint{1.909304in}{1.586033in}}%
\pgfpathlineto{\pgfqpoint{1.913808in}{1.586923in}}%
\pgfpathlineto{\pgfqpoint{1.914680in}{1.587145in}}%
\pgfpathlineto{\pgfqpoint{1.914971in}{1.587714in}}%
\pgfpathlineto{\pgfqpoint{1.918022in}{1.590984in}}%
\pgfpathlineto{\pgfqpoint{1.919475in}{1.590394in}}%
\pgfpathlineto{\pgfqpoint{1.921655in}{1.591204in}}%
\pgfpathlineto{\pgfqpoint{1.922818in}{1.593481in}}%
\pgfpathlineto{\pgfqpoint{1.923689in}{1.592745in}}%
\pgfpathlineto{\pgfqpoint{1.928194in}{1.594736in}}%
\pgfpathlineto{\pgfqpoint{1.929211in}{1.596099in}}%
\pgfpathlineto{\pgfqpoint{1.930083in}{1.595401in}}%
\pgfpathlineto{\pgfqpoint{1.933861in}{1.594443in}}%
\pgfpathlineto{\pgfqpoint{1.935896in}{1.595590in}}%
\pgfpathlineto{\pgfqpoint{1.936622in}{1.594414in}}%
\pgfpathlineto{\pgfqpoint{1.937930in}{1.593796in}}%
\pgfpathlineto{\pgfqpoint{1.938511in}{1.594305in}}%
\pgfpathlineto{\pgfqpoint{1.943016in}{1.596490in}}%
\pgfpathlineto{\pgfqpoint{1.944324in}{1.596371in}}%
\pgfpathlineto{\pgfqpoint{1.944614in}{1.596859in}}%
\pgfpathlineto{\pgfqpoint{1.947521in}{1.599063in}}%
\pgfpathlineto{\pgfqpoint{1.951735in}{1.600044in}}%
\pgfpathlineto{\pgfqpoint{1.955513in}{1.600426in}}%
\pgfpathlineto{\pgfqpoint{1.956821in}{1.600065in}}%
\pgfpathlineto{\pgfqpoint{1.957257in}{1.600976in}}%
\pgfpathlineto{\pgfqpoint{1.959291in}{1.601891in}}%
\pgfpathlineto{\pgfqpoint{1.965249in}{1.602813in}}%
\pgfpathlineto{\pgfqpoint{1.965975in}{1.602728in}}%
\pgfpathlineto{\pgfqpoint{1.966411in}{1.601956in}}%
\pgfpathlineto{\pgfqpoint{1.967283in}{1.601114in}}%
\pgfpathlineto{\pgfqpoint{1.968155in}{1.601789in}}%
\pgfpathlineto{\pgfqpoint{1.984430in}{1.606949in}}%
\pgfpathlineto{\pgfqpoint{1.986755in}{1.607059in}}%
\pgfpathlineto{\pgfqpoint{1.990969in}{1.607856in}}%
\pgfpathlineto{\pgfqpoint{1.991841in}{1.608021in}}%
\pgfpathlineto{\pgfqpoint{1.992277in}{1.608800in}}%
\pgfpathlineto{\pgfqpoint{1.994893in}{1.610552in}}%
\pgfpathlineto{\pgfqpoint{2.004919in}{1.611990in}}%
\pgfpathlineto{\pgfqpoint{2.006954in}{1.612602in}}%
\pgfpathlineto{\pgfqpoint{2.013783in}{1.614343in}}%
\pgfpathlineto{\pgfqpoint{2.022066in}{1.613100in}}%
\pgfpathlineto{\pgfqpoint{2.032093in}{1.618336in}}%
\pgfpathlineto{\pgfqpoint{2.037179in}{1.618740in}}%
\pgfpathlineto{\pgfqpoint{2.039504in}{1.618797in}}%
\pgfpathlineto{\pgfqpoint{2.040666in}{1.619445in}}%
\pgfpathlineto{\pgfqpoint{2.041393in}{1.618867in}}%
\pgfpathlineto{\pgfqpoint{2.045752in}{1.620300in}}%
\pgfpathlineto{\pgfqpoint{2.047060in}{1.622183in}}%
\pgfpathlineto{\pgfqpoint{2.048077in}{1.621559in}}%
\pgfpathlineto{\pgfqpoint{2.052727in}{1.623756in}}%
\pgfpathlineto{\pgfqpoint{2.054180in}{1.623018in}}%
\pgfpathlineto{\pgfqpoint{2.056360in}{1.622228in}}%
\pgfpathlineto{\pgfqpoint{2.056796in}{1.622953in}}%
\pgfpathlineto{\pgfqpoint{2.059557in}{1.624469in}}%
\pgfpathlineto{\pgfqpoint{2.062608in}{1.624843in}}%
\pgfpathlineto{\pgfqpoint{2.065369in}{1.627080in}}%
\pgfpathlineto{\pgfqpoint{2.075396in}{1.623443in}}%
\pgfpathlineto{\pgfqpoint{2.077721in}{1.624361in}}%
\pgfpathlineto{\pgfqpoint{2.078157in}{1.623844in}}%
\pgfpathlineto{\pgfqpoint{2.078738in}{1.623953in}}%
\pgfpathlineto{\pgfqpoint{2.079174in}{1.624843in}}%
\pgfpathlineto{\pgfqpoint{2.082371in}{1.628452in}}%
\pgfpathlineto{\pgfqpoint{2.084841in}{1.628311in}}%
\pgfpathlineto{\pgfqpoint{2.086004in}{1.629509in}}%
\pgfpathlineto{\pgfqpoint{2.088329in}{1.629944in}}%
\pgfpathlineto{\pgfqpoint{2.091671in}{1.627172in}}%
\pgfpathlineto{\pgfqpoint{2.092397in}{1.628186in}}%
\pgfpathlineto{\pgfqpoint{2.095158in}{1.628588in}}%
\pgfpathlineto{\pgfqpoint{2.098646in}{1.629553in}}%
\pgfpathlineto{\pgfqpoint{2.100535in}{1.632114in}}%
\pgfpathlineto{\pgfqpoint{2.101261in}{1.631088in}}%
\pgfpathlineto{\pgfqpoint{2.103005in}{1.631429in}}%
\pgfpathlineto{\pgfqpoint{2.104894in}{1.632048in}}%
\pgfpathlineto{\pgfqpoint{2.106783in}{1.633593in}}%
\pgfpathlineto{\pgfqpoint{2.107364in}{1.632243in}}%
\pgfpathlineto{\pgfqpoint{2.108382in}{1.630477in}}%
\pgfpathlineto{\pgfqpoint{2.109254in}{1.631302in}}%
\pgfpathlineto{\pgfqpoint{2.113177in}{1.632942in}}%
\pgfpathlineto{\pgfqpoint{2.113468in}{1.632485in}}%
\pgfpathlineto{\pgfqpoint{2.114485in}{1.631774in}}%
\pgfpathlineto{\pgfqpoint{2.115211in}{1.632555in}}%
\pgfpathlineto{\pgfqpoint{2.119280in}{1.635923in}}%
\pgfpathlineto{\pgfqpoint{2.122622in}{1.635755in}}%
\pgfpathlineto{\pgfqpoint{2.123639in}{1.635062in}}%
\pgfpathlineto{\pgfqpoint{2.123785in}{1.634791in}}%
\pgfpathlineto{\pgfqpoint{2.126836in}{1.632369in}}%
\pgfpathlineto{\pgfqpoint{2.128871in}{1.633339in}}%
\pgfpathlineto{\pgfqpoint{2.129452in}{1.633966in}}%
\pgfpathlineto{\pgfqpoint{2.130179in}{1.632537in}}%
\pgfpathlineto{\pgfqpoint{2.130905in}{1.631721in}}%
\pgfpathlineto{\pgfqpoint{2.131922in}{1.632389in}}%
\pgfpathlineto{\pgfqpoint{2.133375in}{1.633917in}}%
\pgfpathlineto{\pgfqpoint{2.135846in}{1.637234in}}%
\pgfpathlineto{\pgfqpoint{2.136427in}{1.636526in}}%
\pgfpathlineto{\pgfqpoint{2.138461in}{1.635946in}}%
\pgfpathlineto{\pgfqpoint{2.141949in}{1.636940in}}%
\pgfpathlineto{\pgfqpoint{2.144855in}{1.633989in}}%
\pgfpathlineto{\pgfqpoint{2.145582in}{1.635129in}}%
\pgfpathlineto{\pgfqpoint{2.146454in}{1.635963in}}%
\pgfpathlineto{\pgfqpoint{2.147325in}{1.635153in}}%
\pgfpathlineto{\pgfqpoint{2.150377in}{1.634787in}}%
\pgfpathlineto{\pgfqpoint{2.151685in}{1.636980in}}%
\pgfpathlineto{\pgfqpoint{2.152557in}{1.638338in}}%
\pgfpathlineto{\pgfqpoint{2.153429in}{1.637332in}}%
\pgfpathlineto{\pgfqpoint{2.155754in}{1.636945in}}%
\pgfpathlineto{\pgfqpoint{2.159241in}{1.638847in}}%
\pgfpathlineto{\pgfqpoint{2.159677in}{1.638181in}}%
\pgfpathlineto{\pgfqpoint{2.161130in}{1.635492in}}%
\pgfpathlineto{\pgfqpoint{2.161857in}{1.636710in}}%
\pgfpathlineto{\pgfqpoint{2.164182in}{1.638398in}}%
\pgfpathlineto{\pgfqpoint{2.165054in}{1.638623in}}%
\pgfpathlineto{\pgfqpoint{2.165780in}{1.637982in}}%
\pgfpathlineto{\pgfqpoint{2.167814in}{1.637744in}}%
\pgfpathlineto{\pgfqpoint{2.168105in}{1.638557in}}%
\pgfpathlineto{\pgfqpoint{2.169413in}{1.641015in}}%
\pgfpathlineto{\pgfqpoint{2.170285in}{1.640468in}}%
\pgfpathlineto{\pgfqpoint{2.174499in}{1.639994in}}%
\pgfpathlineto{\pgfqpoint{2.174644in}{1.640443in}}%
\pgfpathlineto{\pgfqpoint{2.175516in}{1.641644in}}%
\pgfpathlineto{\pgfqpoint{2.176243in}{1.640814in}}%
\pgfpathlineto{\pgfqpoint{2.178277in}{1.638985in}}%
\pgfpathlineto{\pgfqpoint{2.179004in}{1.639813in}}%
\pgfpathlineto{\pgfqpoint{2.180021in}{1.638297in}}%
\pgfpathlineto{\pgfqpoint{2.180457in}{1.637943in}}%
\pgfpathlineto{\pgfqpoint{2.181038in}{1.639003in}}%
\pgfpathlineto{\pgfqpoint{2.182346in}{1.640802in}}%
\pgfpathlineto{\pgfqpoint{2.182927in}{1.640266in}}%
\pgfpathlineto{\pgfqpoint{2.184380in}{1.638630in}}%
\pgfpathlineto{\pgfqpoint{2.185107in}{1.639827in}}%
\pgfpathlineto{\pgfqpoint{2.185833in}{1.640334in}}%
\pgfpathlineto{\pgfqpoint{2.186560in}{1.639511in}}%
\pgfpathlineto{\pgfqpoint{2.187868in}{1.640372in}}%
\pgfpathlineto{\pgfqpoint{2.189175in}{1.641169in}}%
\pgfpathlineto{\pgfqpoint{2.189611in}{1.640529in}}%
\pgfpathlineto{\pgfqpoint{2.190774in}{1.638578in}}%
\pgfpathlineto{\pgfqpoint{2.191500in}{1.639894in}}%
\pgfpathlineto{\pgfqpoint{2.192518in}{1.640900in}}%
\pgfpathlineto{\pgfqpoint{2.193389in}{1.640329in}}%
\pgfpathlineto{\pgfqpoint{2.195133in}{1.641291in}}%
\pgfpathlineto{\pgfqpoint{2.196296in}{1.640131in}}%
\pgfpathlineto{\pgfqpoint{2.197168in}{1.639333in}}%
\pgfpathlineto{\pgfqpoint{2.197894in}{1.640250in}}%
\pgfpathlineto{\pgfqpoint{2.199493in}{1.641566in}}%
\pgfpathlineto{\pgfqpoint{2.200074in}{1.640953in}}%
\pgfpathlineto{\pgfqpoint{2.200800in}{1.640531in}}%
\pgfpathlineto{\pgfqpoint{2.201382in}{1.641366in}}%
\pgfpathlineto{\pgfqpoint{2.202108in}{1.641958in}}%
\pgfpathlineto{\pgfqpoint{2.202835in}{1.640806in}}%
\pgfpathlineto{\pgfqpoint{2.203561in}{1.640318in}}%
\pgfpathlineto{\pgfqpoint{2.204288in}{1.641296in}}%
\pgfpathlineto{\pgfqpoint{2.205741in}{1.643109in}}%
\pgfpathlineto{\pgfqpoint{2.206322in}{1.642299in}}%
\pgfpathlineto{\pgfqpoint{2.207194in}{1.641154in}}%
\pgfpathlineto{\pgfqpoint{2.208066in}{1.642041in}}%
\pgfpathlineto{\pgfqpoint{2.209519in}{1.641276in}}%
\pgfpathlineto{\pgfqpoint{2.210972in}{1.640441in}}%
\pgfpathlineto{\pgfqpoint{2.211408in}{1.641202in}}%
\pgfpathlineto{\pgfqpoint{2.212135in}{1.642026in}}%
\pgfpathlineto{\pgfqpoint{2.212716in}{1.640750in}}%
\pgfpathlineto{\pgfqpoint{2.213588in}{1.639062in}}%
\pgfpathlineto{\pgfqpoint{2.214460in}{1.640314in}}%
\pgfpathlineto{\pgfqpoint{2.216930in}{1.640628in}}%
\pgfpathlineto{\pgfqpoint{2.217947in}{1.642013in}}%
\pgfpathlineto{\pgfqpoint{2.218819in}{1.643253in}}%
\pgfpathlineto{\pgfqpoint{2.219546in}{1.642147in}}%
\pgfpathlineto{\pgfqpoint{2.220127in}{1.641878in}}%
\pgfpathlineto{\pgfqpoint{2.220854in}{1.642930in}}%
\pgfpathlineto{\pgfqpoint{2.222597in}{1.644924in}}%
\pgfpathlineto{\pgfqpoint{2.223324in}{1.644297in}}%
\pgfpathlineto{\pgfqpoint{2.224341in}{1.645626in}}%
\pgfpathlineto{\pgfqpoint{2.225068in}{1.646537in}}%
\pgfpathlineto{\pgfqpoint{2.225939in}{1.645454in}}%
\pgfpathlineto{\pgfqpoint{2.227683in}{1.645119in}}%
\pgfpathlineto{\pgfqpoint{2.227974in}{1.645418in}}%
\pgfpathlineto{\pgfqpoint{2.228846in}{1.645861in}}%
\pgfpathlineto{\pgfqpoint{2.229427in}{1.644756in}}%
\pgfpathlineto{\pgfqpoint{2.230299in}{1.643613in}}%
\pgfpathlineto{\pgfqpoint{2.231025in}{1.644625in}}%
\pgfpathlineto{\pgfqpoint{2.231607in}{1.644836in}}%
\pgfpathlineto{\pgfqpoint{2.232188in}{1.643986in}}%
\pgfpathlineto{\pgfqpoint{2.233641in}{1.642464in}}%
\pgfpathlineto{\pgfqpoint{2.234222in}{1.643155in}}%
\pgfpathlineto{\pgfqpoint{2.235239in}{1.644075in}}%
\pgfpathlineto{\pgfqpoint{2.235966in}{1.643190in}}%
\pgfpathlineto{\pgfqpoint{2.236983in}{1.642976in}}%
\pgfpathlineto{\pgfqpoint{2.237564in}{1.643558in}}%
\pgfpathlineto{\pgfqpoint{2.239018in}{1.643606in}}%
\pgfpathlineto{\pgfqpoint{2.239308in}{1.643214in}}%
\pgfpathlineto{\pgfqpoint{2.240180in}{1.642729in}}%
\pgfpathlineto{\pgfqpoint{2.240761in}{1.643667in}}%
\pgfpathlineto{\pgfqpoint{2.241633in}{1.644461in}}%
\pgfpathlineto{\pgfqpoint{2.242360in}{1.643578in}}%
\pgfpathlineto{\pgfqpoint{2.243958in}{1.642966in}}%
\pgfpathlineto{\pgfqpoint{2.244394in}{1.643412in}}%
\pgfpathlineto{\pgfqpoint{2.245411in}{1.643841in}}%
\pgfpathlineto{\pgfqpoint{2.246138in}{1.643136in}}%
\pgfpathlineto{\pgfqpoint{2.247155in}{1.644336in}}%
\pgfpathlineto{\pgfqpoint{2.248318in}{1.644938in}}%
\pgfpathlineto{\pgfqpoint{2.248899in}{1.644571in}}%
\pgfpathlineto{\pgfqpoint{2.250643in}{1.644348in}}%
\pgfpathlineto{\pgfqpoint{2.250933in}{1.644934in}}%
\pgfpathlineto{\pgfqpoint{2.251805in}{1.645775in}}%
\pgfpathlineto{\pgfqpoint{2.252532in}{1.644820in}}%
\pgfpathlineto{\pgfqpoint{2.253985in}{1.645539in}}%
\pgfpathlineto{\pgfqpoint{2.255874in}{1.644886in}}%
\pgfpathlineto{\pgfqpoint{2.256891in}{1.644296in}}%
\pgfpathlineto{\pgfqpoint{2.257472in}{1.645151in}}%
\pgfpathlineto{\pgfqpoint{2.258199in}{1.645432in}}%
\pgfpathlineto{\pgfqpoint{2.258780in}{1.644647in}}%
\pgfpathlineto{\pgfqpoint{2.261105in}{1.643904in}}%
\pgfpathlineto{\pgfqpoint{2.262268in}{1.643350in}}%
\pgfpathlineto{\pgfqpoint{2.262413in}{1.643061in}}%
\pgfpathlineto{\pgfqpoint{2.263139in}{1.642216in}}%
\pgfpathlineto{\pgfqpoint{2.263866in}{1.643366in}}%
\pgfpathlineto{\pgfqpoint{2.265755in}{1.643822in}}%
\pgfpathlineto{\pgfqpoint{2.270260in}{1.645767in}}%
\pgfpathlineto{\pgfqpoint{2.271568in}{1.647351in}}%
\pgfpathlineto{\pgfqpoint{2.272294in}{1.646680in}}%
\pgfpathlineto{\pgfqpoint{2.273457in}{1.646320in}}%
\pgfpathlineto{\pgfqpoint{2.273893in}{1.646930in}}%
\pgfpathlineto{\pgfqpoint{2.275927in}{1.647195in}}%
\pgfpathlineto{\pgfqpoint{2.279124in}{1.646739in}}%
\pgfpathlineto{\pgfqpoint{2.279850in}{1.646442in}}%
\pgfpathlineto{\pgfqpoint{2.280432in}{1.647368in}}%
\pgfpathlineto{\pgfqpoint{2.281594in}{1.648211in}}%
\pgfpathlineto{\pgfqpoint{2.282175in}{1.647653in}}%
\pgfpathlineto{\pgfqpoint{2.283338in}{1.647570in}}%
\pgfpathlineto{\pgfqpoint{2.283774in}{1.648121in}}%
\pgfpathlineto{\pgfqpoint{2.284646in}{1.648442in}}%
\pgfpathlineto{\pgfqpoint{2.285227in}{1.647654in}}%
\pgfpathlineto{\pgfqpoint{2.286099in}{1.647310in}}%
\pgfpathlineto{\pgfqpoint{2.286680in}{1.647985in}}%
\pgfpathlineto{\pgfqpoint{2.288424in}{1.649260in}}%
\pgfpathlineto{\pgfqpoint{2.288860in}{1.648688in}}%
\pgfpathlineto{\pgfqpoint{2.289732in}{1.647779in}}%
\pgfpathlineto{\pgfqpoint{2.290458in}{1.648610in}}%
\pgfpathlineto{\pgfqpoint{2.291039in}{1.648738in}}%
\pgfpathlineto{\pgfqpoint{2.291621in}{1.647829in}}%
\pgfpathlineto{\pgfqpoint{2.292783in}{1.647321in}}%
\pgfpathlineto{\pgfqpoint{2.293364in}{1.647675in}}%
\pgfpathlineto{\pgfqpoint{2.294963in}{1.648226in}}%
\pgfpathlineto{\pgfqpoint{2.295399in}{1.647253in}}%
\pgfpathlineto{\pgfqpoint{2.296125in}{1.646168in}}%
\pgfpathlineto{\pgfqpoint{2.296852in}{1.647283in}}%
\pgfpathlineto{\pgfqpoint{2.298596in}{1.647289in}}%
\pgfpathlineto{\pgfqpoint{2.308041in}{1.649742in}}%
\pgfpathlineto{\pgfqpoint{2.309204in}{1.647834in}}%
\pgfpathlineto{\pgfqpoint{2.310075in}{1.648641in}}%
\pgfpathlineto{\pgfqpoint{2.311674in}{1.648263in}}%
\pgfpathlineto{\pgfqpoint{2.311819in}{1.648029in}}%
\pgfpathlineto{\pgfqpoint{2.312982in}{1.646490in}}%
\pgfpathlineto{\pgfqpoint{2.313708in}{1.647523in}}%
\pgfpathlineto{\pgfqpoint{2.314289in}{1.647813in}}%
\pgfpathlineto{\pgfqpoint{2.314871in}{1.646839in}}%
\pgfpathlineto{\pgfqpoint{2.315743in}{1.645936in}}%
\pgfpathlineto{\pgfqpoint{2.316469in}{1.647035in}}%
\pgfpathlineto{\pgfqpoint{2.317777in}{1.647969in}}%
\pgfpathlineto{\pgfqpoint{2.318358in}{1.647349in}}%
\pgfpathlineto{\pgfqpoint{2.319666in}{1.646644in}}%
\pgfpathlineto{\pgfqpoint{2.320247in}{1.647334in}}%
\pgfpathlineto{\pgfqpoint{2.324025in}{1.650157in}}%
\pgfpathlineto{\pgfqpoint{2.326350in}{1.648456in}}%
\pgfpathlineto{\pgfqpoint{2.326932in}{1.649213in}}%
\pgfpathlineto{\pgfqpoint{2.327949in}{1.650083in}}%
\pgfpathlineto{\pgfqpoint{2.328675in}{1.649354in}}%
\pgfpathlineto{\pgfqpoint{2.329983in}{1.650264in}}%
\pgfpathlineto{\pgfqpoint{2.331291in}{1.649270in}}%
\pgfpathlineto{\pgfqpoint{2.332744in}{1.648943in}}%
\pgfpathlineto{\pgfqpoint{2.333035in}{1.649373in}}%
\pgfpathlineto{\pgfqpoint{2.334197in}{1.651532in}}%
\pgfpathlineto{\pgfqpoint{2.334924in}{1.650270in}}%
\pgfpathlineto{\pgfqpoint{2.335650in}{1.649572in}}%
\pgfpathlineto{\pgfqpoint{2.336377in}{1.650414in}}%
\pgfpathlineto{\pgfqpoint{2.340882in}{1.652586in}}%
\pgfpathlineto{\pgfqpoint{2.341172in}{1.651967in}}%
\pgfpathlineto{\pgfqpoint{2.342044in}{1.650824in}}%
\pgfpathlineto{\pgfqpoint{2.342916in}{1.651609in}}%
\pgfpathlineto{\pgfqpoint{2.345241in}{1.650461in}}%
\pgfpathlineto{\pgfqpoint{2.345822in}{1.650117in}}%
\pgfpathlineto{\pgfqpoint{2.346404in}{1.651149in}}%
\pgfpathlineto{\pgfqpoint{2.347130in}{1.651973in}}%
\pgfpathlineto{\pgfqpoint{2.347711in}{1.650793in}}%
\pgfpathlineto{\pgfqpoint{2.349891in}{1.649120in}}%
\pgfpathlineto{\pgfqpoint{2.355122in}{1.648875in}}%
\pgfpathlineto{\pgfqpoint{2.356430in}{1.649804in}}%
\pgfpathlineto{\pgfqpoint{2.358755in}{1.651308in}}%
\pgfpathlineto{\pgfqpoint{2.361371in}{1.650793in}}%
\pgfpathlineto{\pgfqpoint{2.362243in}{1.650633in}}%
\pgfpathlineto{\pgfqpoint{2.362824in}{1.651475in}}%
\pgfpathlineto{\pgfqpoint{2.364422in}{1.652596in}}%
\pgfpathlineto{\pgfqpoint{2.364858in}{1.652132in}}%
\pgfpathlineto{\pgfqpoint{2.368491in}{1.649173in}}%
\pgfpathlineto{\pgfqpoint{2.368782in}{1.649491in}}%
\pgfpathlineto{\pgfqpoint{2.370671in}{1.650551in}}%
\pgfpathlineto{\pgfqpoint{2.370961in}{1.650291in}}%
\pgfpathlineto{\pgfqpoint{2.374594in}{1.647523in}}%
\pgfpathlineto{\pgfqpoint{2.375757in}{1.649118in}}%
\pgfpathlineto{\pgfqpoint{2.377500in}{1.649824in}}%
\pgfpathlineto{\pgfqpoint{2.377791in}{1.649355in}}%
\pgfpathlineto{\pgfqpoint{2.378808in}{1.647507in}}%
\pgfpathlineto{\pgfqpoint{2.379389in}{1.648919in}}%
\pgfpathlineto{\pgfqpoint{2.381569in}{1.650929in}}%
\pgfpathlineto{\pgfqpoint{2.383894in}{1.651623in}}%
\pgfpathlineto{\pgfqpoint{2.385202in}{1.649252in}}%
\pgfpathlineto{\pgfqpoint{2.386074in}{1.651333in}}%
\pgfpathlineto{\pgfqpoint{2.387963in}{1.652018in}}%
\pgfpathlineto{\pgfqpoint{2.390869in}{1.650258in}}%
\pgfpathlineto{\pgfqpoint{2.391596in}{1.648953in}}%
\pgfpathlineto{\pgfqpoint{2.392322in}{1.650724in}}%
\pgfpathlineto{\pgfqpoint{2.393339in}{1.652518in}}%
\pgfpathlineto{\pgfqpoint{2.394211in}{1.651929in}}%
\pgfpathlineto{\pgfqpoint{2.398280in}{1.649864in}}%
\pgfpathlineto{\pgfqpoint{2.398425in}{1.650078in}}%
\pgfpathlineto{\pgfqpoint{2.400024in}{1.652525in}}%
\pgfpathlineto{\pgfqpoint{2.400605in}{1.651566in}}%
\pgfpathlineto{\pgfqpoint{2.402204in}{1.649159in}}%
\pgfpathlineto{\pgfqpoint{2.402930in}{1.649933in}}%
\pgfpathlineto{\pgfqpoint{2.406708in}{1.652143in}}%
\pgfpathlineto{\pgfqpoint{2.408307in}{1.650183in}}%
\pgfpathlineto{\pgfqpoint{2.409033in}{1.651201in}}%
\pgfpathlineto{\pgfqpoint{2.412957in}{1.655374in}}%
\pgfpathlineto{\pgfqpoint{2.415718in}{1.653866in}}%
\pgfpathlineto{\pgfqpoint{2.416299in}{1.654973in}}%
\pgfpathlineto{\pgfqpoint{2.416880in}{1.655629in}}%
\pgfpathlineto{\pgfqpoint{2.417607in}{1.654161in}}%
\pgfpathlineto{\pgfqpoint{2.419496in}{1.653348in}}%
\pgfpathlineto{\pgfqpoint{2.422111in}{1.650425in}}%
\pgfpathlineto{\pgfqpoint{2.422547in}{1.651065in}}%
\pgfpathlineto{\pgfqpoint{2.423274in}{1.652177in}}%
\pgfpathlineto{\pgfqpoint{2.423855in}{1.650880in}}%
\pgfpathlineto{\pgfqpoint{2.424727in}{1.648672in}}%
\pgfpathlineto{\pgfqpoint{2.425744in}{1.649650in}}%
\pgfpathlineto{\pgfqpoint{2.429086in}{1.651146in}}%
\pgfpathlineto{\pgfqpoint{2.429813in}{1.652836in}}%
\pgfpathlineto{\pgfqpoint{2.430539in}{1.651260in}}%
\pgfpathlineto{\pgfqpoint{2.431411in}{1.649838in}}%
\pgfpathlineto{\pgfqpoint{2.432138in}{1.650803in}}%
\pgfpathlineto{\pgfqpoint{2.436352in}{1.654787in}}%
\pgfpathlineto{\pgfqpoint{2.436497in}{1.654622in}}%
\pgfpathlineto{\pgfqpoint{2.438241in}{1.651496in}}%
\pgfpathlineto{\pgfqpoint{2.439113in}{1.652703in}}%
\pgfpathlineto{\pgfqpoint{2.440130in}{1.653810in}}%
\pgfpathlineto{\pgfqpoint{2.440711in}{1.652977in}}%
\pgfpathlineto{\pgfqpoint{2.441583in}{1.651953in}}%
\pgfpathlineto{\pgfqpoint{2.442310in}{1.652990in}}%
\pgfpathlineto{\pgfqpoint{2.442891in}{1.653318in}}%
\pgfpathlineto{\pgfqpoint{2.443472in}{1.652180in}}%
\pgfpathlineto{\pgfqpoint{2.444780in}{1.649700in}}%
\pgfpathlineto{\pgfqpoint{2.445361in}{1.650646in}}%
\pgfpathlineto{\pgfqpoint{2.446669in}{1.652608in}}%
\pgfpathlineto{\pgfqpoint{2.447396in}{1.651704in}}%
\pgfpathlineto{\pgfqpoint{2.448413in}{1.650795in}}%
\pgfpathlineto{\pgfqpoint{2.448994in}{1.651697in}}%
\pgfpathlineto{\pgfqpoint{2.449721in}{1.652480in}}%
\pgfpathlineto{\pgfqpoint{2.450447in}{1.651164in}}%
\pgfpathlineto{\pgfqpoint{2.451029in}{1.650340in}}%
\pgfpathlineto{\pgfqpoint{2.451755in}{1.651769in}}%
\pgfpathlineto{\pgfqpoint{2.452918in}{1.653695in}}%
\pgfpathlineto{\pgfqpoint{2.453644in}{1.653054in}}%
\pgfpathlineto{\pgfqpoint{2.455097in}{1.651078in}}%
\pgfpathlineto{\pgfqpoint{2.455824in}{1.652314in}}%
\pgfpathlineto{\pgfqpoint{2.456405in}{1.652788in}}%
\pgfpathlineto{\pgfqpoint{2.456986in}{1.651722in}}%
\pgfpathlineto{\pgfqpoint{2.457568in}{1.651024in}}%
\pgfpathlineto{\pgfqpoint{2.458294in}{1.652422in}}%
\pgfpathlineto{\pgfqpoint{2.459747in}{1.653913in}}%
\pgfpathlineto{\pgfqpoint{2.460183in}{1.653455in}}%
\pgfpathlineto{\pgfqpoint{2.461636in}{1.650347in}}%
\pgfpathlineto{\pgfqpoint{2.462363in}{1.652403in}}%
\pgfpathlineto{\pgfqpoint{2.463089in}{1.653440in}}%
\pgfpathlineto{\pgfqpoint{2.464107in}{1.652754in}}%
\pgfpathlineto{\pgfqpoint{2.466722in}{1.654458in}}%
\pgfpathlineto{\pgfqpoint{2.467158in}{1.653258in}}%
\pgfpathlineto{\pgfqpoint{2.468030in}{1.651208in}}%
\pgfpathlineto{\pgfqpoint{2.468611in}{1.652895in}}%
\pgfpathlineto{\pgfqpoint{2.469483in}{1.655137in}}%
\pgfpathlineto{\pgfqpoint{2.470355in}{1.654200in}}%
\pgfpathlineto{\pgfqpoint{2.474569in}{1.649997in}}%
\pgfpathlineto{\pgfqpoint{2.475150in}{1.651497in}}%
\pgfpathlineto{\pgfqpoint{2.476168in}{1.653897in}}%
\pgfpathlineto{\pgfqpoint{2.477039in}{1.652886in}}%
\pgfpathlineto{\pgfqpoint{2.478783in}{1.650331in}}%
\pgfpathlineto{\pgfqpoint{2.479800in}{1.650880in}}%
\pgfpathlineto{\pgfqpoint{2.480963in}{1.649444in}}%
\pgfpathlineto{\pgfqpoint{2.481544in}{1.650632in}}%
\pgfpathlineto{\pgfqpoint{2.482852in}{1.653342in}}%
\pgfpathlineto{\pgfqpoint{2.483579in}{1.652562in}}%
\pgfpathlineto{\pgfqpoint{2.485032in}{1.650985in}}%
\pgfpathlineto{\pgfqpoint{2.485613in}{1.651910in}}%
\pgfpathlineto{\pgfqpoint{2.487793in}{1.652902in}}%
\pgfpathlineto{\pgfqpoint{2.489391in}{1.654728in}}%
\pgfpathlineto{\pgfqpoint{2.490118in}{1.653465in}}%
\pgfpathlineto{\pgfqpoint{2.491425in}{1.651621in}}%
\pgfpathlineto{\pgfqpoint{2.492152in}{1.652415in}}%
\pgfpathlineto{\pgfqpoint{2.493314in}{1.653433in}}%
\pgfpathlineto{\pgfqpoint{2.493896in}{1.652722in}}%
\pgfpathlineto{\pgfqpoint{2.495639in}{1.652997in}}%
\pgfpathlineto{\pgfqpoint{2.498400in}{1.651453in}}%
\pgfpathlineto{\pgfqpoint{2.498982in}{1.652916in}}%
\pgfpathlineto{\pgfqpoint{2.499708in}{1.653990in}}%
\pgfpathlineto{\pgfqpoint{2.500435in}{1.652728in}}%
\pgfpathlineto{\pgfqpoint{2.501016in}{1.651999in}}%
\pgfpathlineto{\pgfqpoint{2.501888in}{1.653176in}}%
\pgfpathlineto{\pgfqpoint{2.503341in}{1.652537in}}%
\pgfpathlineto{\pgfqpoint{2.505230in}{1.654103in}}%
\pgfpathlineto{\pgfqpoint{2.506102in}{1.655541in}}%
\pgfpathlineto{\pgfqpoint{2.506683in}{1.654261in}}%
\pgfpathlineto{\pgfqpoint{2.507700in}{1.651812in}}%
\pgfpathlineto{\pgfqpoint{2.508718in}{1.652503in}}%
\pgfpathlineto{\pgfqpoint{2.513513in}{1.650190in}}%
\pgfpathlineto{\pgfqpoint{2.514094in}{1.649365in}}%
\pgfpathlineto{\pgfqpoint{2.514966in}{1.650529in}}%
\pgfpathlineto{\pgfqpoint{2.517436in}{1.651624in}}%
\pgfpathlineto{\pgfqpoint{2.518454in}{1.652949in}}%
\pgfpathlineto{\pgfqpoint{2.519180in}{1.653746in}}%
\pgfpathlineto{\pgfqpoint{2.519907in}{1.652545in}}%
\pgfpathlineto{\pgfqpoint{2.520633in}{1.652088in}}%
\pgfpathlineto{\pgfqpoint{2.521214in}{1.653143in}}%
\pgfpathlineto{\pgfqpoint{2.522958in}{1.656013in}}%
\pgfpathlineto{\pgfqpoint{2.523539in}{1.655335in}}%
\pgfpathlineto{\pgfqpoint{2.527172in}{1.651704in}}%
\pgfpathlineto{\pgfqpoint{2.530660in}{1.650653in}}%
\pgfpathlineto{\pgfqpoint{2.533711in}{1.648580in}}%
\pgfpathlineto{\pgfqpoint{2.542721in}{1.653153in}}%
\pgfpathlineto{\pgfqpoint{2.543011in}{1.652597in}}%
\pgfpathlineto{\pgfqpoint{2.544174in}{1.650829in}}%
\pgfpathlineto{\pgfqpoint{2.544900in}{1.652173in}}%
\pgfpathlineto{\pgfqpoint{2.545772in}{1.653086in}}%
\pgfpathlineto{\pgfqpoint{2.546499in}{1.652289in}}%
\pgfpathlineto{\pgfqpoint{2.550858in}{1.650823in}}%
\pgfpathlineto{\pgfqpoint{2.551004in}{1.651177in}}%
\pgfpathlineto{\pgfqpoint{2.552311in}{1.653821in}}%
\pgfpathlineto{\pgfqpoint{2.553183in}{1.652840in}}%
\pgfpathlineto{\pgfqpoint{2.557252in}{1.651702in}}%
\pgfpathlineto{\pgfqpoint{2.557397in}{1.651943in}}%
\pgfpathlineto{\pgfqpoint{2.558560in}{1.653624in}}%
\pgfpathlineto{\pgfqpoint{2.559286in}{1.652370in}}%
\pgfpathlineto{\pgfqpoint{2.561611in}{1.649735in}}%
\pgfpathlineto{\pgfqpoint{2.562047in}{1.649978in}}%
\pgfpathlineto{\pgfqpoint{2.565825in}{1.649937in}}%
\pgfpathlineto{\pgfqpoint{2.567279in}{1.648588in}}%
\pgfpathlineto{\pgfqpoint{2.567860in}{1.649268in}}%
\pgfpathlineto{\pgfqpoint{2.571493in}{1.652833in}}%
\pgfpathlineto{\pgfqpoint{2.574689in}{1.653023in}}%
\pgfpathlineto{\pgfqpoint{2.575852in}{1.654091in}}%
\pgfpathlineto{\pgfqpoint{2.576579in}{1.653282in}}%
\pgfpathlineto{\pgfqpoint{2.580647in}{1.650838in}}%
\pgfpathlineto{\pgfqpoint{2.582391in}{1.651243in}}%
\pgfpathlineto{\pgfqpoint{2.582682in}{1.650509in}}%
\pgfpathlineto{\pgfqpoint{2.585152in}{1.647749in}}%
\pgfpathlineto{\pgfqpoint{2.586024in}{1.647280in}}%
\pgfpathlineto{\pgfqpoint{2.586605in}{1.648042in}}%
\pgfpathlineto{\pgfqpoint{2.588785in}{1.649744in}}%
\pgfpathlineto{\pgfqpoint{2.589221in}{1.649032in}}%
\pgfpathlineto{\pgfqpoint{2.590238in}{1.647611in}}%
\pgfpathlineto{\pgfqpoint{2.590964in}{1.648721in}}%
\pgfpathlineto{\pgfqpoint{2.595033in}{1.652873in}}%
\pgfpathlineto{\pgfqpoint{2.596632in}{1.650329in}}%
\pgfpathlineto{\pgfqpoint{2.597939in}{1.652156in}}%
\pgfpathlineto{\pgfqpoint{2.599683in}{1.652961in}}%
\pgfpathlineto{\pgfqpoint{2.600119in}{1.652227in}}%
\pgfpathlineto{\pgfqpoint{2.603316in}{1.649259in}}%
\pgfpathlineto{\pgfqpoint{2.603461in}{1.649417in}}%
\pgfpathlineto{\pgfqpoint{2.605496in}{1.650805in}}%
\pgfpathlineto{\pgfqpoint{2.605932in}{1.650408in}}%
\pgfpathlineto{\pgfqpoint{2.608838in}{1.649440in}}%
\pgfpathlineto{\pgfqpoint{2.610291in}{1.650237in}}%
\pgfpathlineto{\pgfqpoint{2.612035in}{1.651816in}}%
\pgfpathlineto{\pgfqpoint{2.612616in}{1.651183in}}%
\pgfpathlineto{\pgfqpoint{2.614650in}{1.650893in}}%
\pgfpathlineto{\pgfqpoint{2.619155in}{1.650460in}}%
\pgfpathlineto{\pgfqpoint{2.620463in}{1.649671in}}%
\pgfpathlineto{\pgfqpoint{2.621044in}{1.650438in}}%
\pgfpathlineto{\pgfqpoint{2.622933in}{1.650496in}}%
\pgfpathlineto{\pgfqpoint{2.624241in}{1.651716in}}%
\pgfpathlineto{\pgfqpoint{2.625113in}{1.650617in}}%
\pgfpathlineto{\pgfqpoint{2.626711in}{1.650304in}}%
\pgfpathlineto{\pgfqpoint{2.627002in}{1.650773in}}%
\pgfpathlineto{\pgfqpoint{2.629618in}{1.652900in}}%
\pgfpathlineto{\pgfqpoint{2.630635in}{1.653384in}}%
\pgfpathlineto{\pgfqpoint{2.631361in}{1.652639in}}%
\pgfpathlineto{\pgfqpoint{2.633396in}{1.651656in}}%
\pgfpathlineto{\pgfqpoint{2.633832in}{1.652575in}}%
\pgfpathlineto{\pgfqpoint{2.634413in}{1.653295in}}%
\pgfpathlineto{\pgfqpoint{2.634994in}{1.652176in}}%
\pgfpathlineto{\pgfqpoint{2.637610in}{1.649001in}}%
\pgfpathlineto{\pgfqpoint{2.639644in}{1.648748in}}%
\pgfpathlineto{\pgfqpoint{2.639789in}{1.648938in}}%
\pgfpathlineto{\pgfqpoint{2.641097in}{1.651069in}}%
\pgfpathlineto{\pgfqpoint{2.641824in}{1.649725in}}%
\pgfpathlineto{\pgfqpoint{2.642841in}{1.649002in}}%
\pgfpathlineto{\pgfqpoint{2.643568in}{1.649632in}}%
\pgfpathlineto{\pgfqpoint{2.647491in}{1.653056in}}%
\pgfpathlineto{\pgfqpoint{2.647782in}{1.652658in}}%
\pgfpathlineto{\pgfqpoint{2.649961in}{1.651430in}}%
\pgfpathlineto{\pgfqpoint{2.650107in}{1.651534in}}%
\pgfpathlineto{\pgfqpoint{2.651705in}{1.652325in}}%
\pgfpathlineto{\pgfqpoint{2.652286in}{1.651580in}}%
\pgfpathlineto{\pgfqpoint{2.655919in}{1.648919in}}%
\pgfpathlineto{\pgfqpoint{2.656355in}{1.649475in}}%
\pgfpathlineto{\pgfqpoint{2.658680in}{1.650301in}}%
\pgfpathlineto{\pgfqpoint{2.662022in}{1.650360in}}%
\pgfpathlineto{\pgfqpoint{2.664493in}{1.653188in}}%
\pgfpathlineto{\pgfqpoint{2.665510in}{1.651792in}}%
\pgfpathlineto{\pgfqpoint{2.667254in}{1.652052in}}%
\pgfpathlineto{\pgfqpoint{2.675391in}{1.649694in}}%
\pgfpathlineto{\pgfqpoint{2.677571in}{1.650658in}}%
\pgfpathlineto{\pgfqpoint{2.679750in}{1.651457in}}%
\pgfpathlineto{\pgfqpoint{2.681785in}{1.652922in}}%
\pgfpathlineto{\pgfqpoint{2.684400in}{1.652365in}}%
\pgfpathlineto{\pgfqpoint{2.685418in}{1.650604in}}%
\pgfpathlineto{\pgfqpoint{2.686144in}{1.652347in}}%
\pgfpathlineto{\pgfqpoint{2.686871in}{1.653678in}}%
\pgfpathlineto{\pgfqpoint{2.687743in}{1.652613in}}%
\pgfpathlineto{\pgfqpoint{2.691957in}{1.649564in}}%
\pgfpathlineto{\pgfqpoint{2.692102in}{1.649742in}}%
\pgfpathlineto{\pgfqpoint{2.693700in}{1.651782in}}%
\pgfpathlineto{\pgfqpoint{2.694427in}{1.650893in}}%
\pgfpathlineto{\pgfqpoint{2.695880in}{1.648877in}}%
\pgfpathlineto{\pgfqpoint{2.696607in}{1.650080in}}%
\pgfpathlineto{\pgfqpoint{2.699949in}{1.653403in}}%
\pgfpathlineto{\pgfqpoint{2.702710in}{1.651246in}}%
\pgfpathlineto{\pgfqpoint{2.703582in}{1.652374in}}%
\pgfpathlineto{\pgfqpoint{2.705180in}{1.651854in}}%
\pgfpathlineto{\pgfqpoint{2.709539in}{1.651282in}}%
\pgfpathlineto{\pgfqpoint{2.711138in}{1.653056in}}%
\pgfpathlineto{\pgfqpoint{2.711864in}{1.652015in}}%
\pgfpathlineto{\pgfqpoint{2.714916in}{1.651806in}}%
\pgfpathlineto{\pgfqpoint{2.717096in}{1.654769in}}%
\pgfpathlineto{\pgfqpoint{2.717677in}{1.653616in}}%
\pgfpathlineto{\pgfqpoint{2.720293in}{1.650105in}}%
\pgfpathlineto{\pgfqpoint{2.720729in}{1.649675in}}%
\pgfpathlineto{\pgfqpoint{2.721600in}{1.650699in}}%
\pgfpathlineto{\pgfqpoint{2.723925in}{1.651007in}}%
\pgfpathlineto{\pgfqpoint{2.724071in}{1.650491in}}%
\pgfpathlineto{\pgfqpoint{2.725088in}{1.647864in}}%
\pgfpathlineto{\pgfqpoint{2.725960in}{1.648822in}}%
\pgfpathlineto{\pgfqpoint{2.728285in}{1.650470in}}%
\pgfpathlineto{\pgfqpoint{2.730029in}{1.652041in}}%
\pgfpathlineto{\pgfqpoint{2.730610in}{1.650391in}}%
\pgfpathlineto{\pgfqpoint{2.731191in}{1.649363in}}%
\pgfpathlineto{\pgfqpoint{2.731918in}{1.650700in}}%
\pgfpathlineto{\pgfqpoint{2.734533in}{1.653697in}}%
\pgfpathlineto{\pgfqpoint{2.734824in}{1.653456in}}%
\pgfpathlineto{\pgfqpoint{2.738021in}{1.650650in}}%
\pgfpathlineto{\pgfqpoint{2.738747in}{1.651874in}}%
\pgfpathlineto{\pgfqpoint{2.740491in}{1.652182in}}%
\pgfpathlineto{\pgfqpoint{2.740636in}{1.652073in}}%
\pgfpathlineto{\pgfqpoint{2.744850in}{1.648083in}}%
\pgfpathlineto{\pgfqpoint{2.745868in}{1.649543in}}%
\pgfpathlineto{\pgfqpoint{2.746739in}{1.649166in}}%
\pgfpathlineto{\pgfqpoint{2.747030in}{1.648657in}}%
\pgfpathlineto{\pgfqpoint{2.749064in}{1.648130in}}%
\pgfpathlineto{\pgfqpoint{2.751825in}{1.649863in}}%
\pgfpathlineto{\pgfqpoint{2.752407in}{1.650548in}}%
\pgfpathlineto{\pgfqpoint{2.753133in}{1.649445in}}%
\pgfpathlineto{\pgfqpoint{2.755313in}{1.648232in}}%
\pgfpathlineto{\pgfqpoint{2.755458in}{1.648317in}}%
\pgfpathlineto{\pgfqpoint{2.758946in}{1.650845in}}%
\pgfpathlineto{\pgfqpoint{2.759236in}{1.650283in}}%
\pgfpathlineto{\pgfqpoint{2.761416in}{1.648343in}}%
\pgfpathlineto{\pgfqpoint{2.764322in}{1.648548in}}%
\pgfpathlineto{\pgfqpoint{2.765194in}{1.649992in}}%
\pgfpathlineto{\pgfqpoint{2.765921in}{1.648843in}}%
\pgfpathlineto{\pgfqpoint{2.768246in}{1.646835in}}%
\pgfpathlineto{\pgfqpoint{2.768391in}{1.647015in}}%
\pgfpathlineto{\pgfqpoint{2.769263in}{1.648448in}}%
\pgfpathlineto{\pgfqpoint{2.769989in}{1.647030in}}%
\pgfpathlineto{\pgfqpoint{2.770571in}{1.646430in}}%
\pgfpathlineto{\pgfqpoint{2.771152in}{1.647730in}}%
\pgfpathlineto{\pgfqpoint{2.771733in}{1.648332in}}%
\pgfpathlineto{\pgfqpoint{2.772314in}{1.647007in}}%
\pgfpathlineto{\pgfqpoint{2.774494in}{1.645326in}}%
\pgfpathlineto{\pgfqpoint{2.775511in}{1.646733in}}%
\pgfpathlineto{\pgfqpoint{2.776093in}{1.646970in}}%
\pgfpathlineto{\pgfqpoint{2.776819in}{1.646064in}}%
\pgfpathlineto{\pgfqpoint{2.780452in}{1.645867in}}%
\pgfpathlineto{\pgfqpoint{2.781324in}{1.647493in}}%
\pgfpathlineto{\pgfqpoint{2.782341in}{1.649525in}}%
\pgfpathlineto{\pgfqpoint{2.783213in}{1.648539in}}%
\pgfpathlineto{\pgfqpoint{2.785102in}{1.649356in}}%
\pgfpathlineto{\pgfqpoint{2.789752in}{1.649394in}}%
\pgfpathlineto{\pgfqpoint{2.790624in}{1.648151in}}%
\pgfpathlineto{\pgfqpoint{2.791350in}{1.649004in}}%
\pgfpathlineto{\pgfqpoint{2.791932in}{1.649197in}}%
\pgfpathlineto{\pgfqpoint{2.792658in}{1.648172in}}%
\pgfpathlineto{\pgfqpoint{2.794547in}{1.648244in}}%
\pgfpathlineto{\pgfqpoint{2.796582in}{1.645857in}}%
\pgfpathlineto{\pgfqpoint{2.797163in}{1.645119in}}%
\pgfpathlineto{\pgfqpoint{2.797889in}{1.646513in}}%
\pgfpathlineto{\pgfqpoint{2.798616in}{1.647345in}}%
\pgfpathlineto{\pgfqpoint{2.799633in}{1.646684in}}%
\pgfpathlineto{\pgfqpoint{2.804138in}{1.647806in}}%
\pgfpathlineto{\pgfqpoint{2.806027in}{1.648414in}}%
\pgfpathlineto{\pgfqpoint{2.806608in}{1.648295in}}%
\pgfpathlineto{\pgfqpoint{2.807044in}{1.647309in}}%
\pgfpathlineto{\pgfqpoint{2.807625in}{1.646486in}}%
\pgfpathlineto{\pgfqpoint{2.808352in}{1.647885in}}%
\pgfpathlineto{\pgfqpoint{2.808788in}{1.648278in}}%
\pgfpathlineto{\pgfqpoint{2.809514in}{1.646817in}}%
\pgfpathlineto{\pgfqpoint{2.809950in}{1.646428in}}%
\pgfpathlineto{\pgfqpoint{2.810677in}{1.647833in}}%
\pgfpathlineto{\pgfqpoint{2.811258in}{1.648261in}}%
\pgfpathlineto{\pgfqpoint{2.811985in}{1.647354in}}%
\pgfpathlineto{\pgfqpoint{2.814310in}{1.645641in}}%
\pgfpathlineto{\pgfqpoint{2.815036in}{1.646686in}}%
\pgfpathlineto{\pgfqpoint{2.818814in}{1.647324in}}%
\pgfpathlineto{\pgfqpoint{2.819977in}{1.645153in}}%
\pgfpathlineto{\pgfqpoint{2.820413in}{1.644823in}}%
\pgfpathlineto{\pgfqpoint{2.820994in}{1.646016in}}%
\pgfpathlineto{\pgfqpoint{2.823029in}{1.647238in}}%
\pgfpathlineto{\pgfqpoint{2.824482in}{1.646180in}}%
\pgfpathlineto{\pgfqpoint{2.826807in}{1.644696in}}%
\pgfpathlineto{\pgfqpoint{2.826952in}{1.644788in}}%
\pgfpathlineto{\pgfqpoint{2.828986in}{1.647092in}}%
\pgfpathlineto{\pgfqpoint{2.829858in}{1.645763in}}%
\pgfpathlineto{\pgfqpoint{2.832474in}{1.644036in}}%
\pgfpathlineto{\pgfqpoint{2.832764in}{1.644333in}}%
\pgfpathlineto{\pgfqpoint{2.835089in}{1.647591in}}%
\pgfpathlineto{\pgfqpoint{2.835816in}{1.646535in}}%
\pgfpathlineto{\pgfqpoint{2.838141in}{1.644968in}}%
\pgfpathlineto{\pgfqpoint{2.841629in}{1.648188in}}%
\pgfpathlineto{\pgfqpoint{2.842355in}{1.645642in}}%
\pgfpathlineto{\pgfqpoint{2.843082in}{1.644481in}}%
\pgfpathlineto{\pgfqpoint{2.844099in}{1.645089in}}%
\pgfpathlineto{\pgfqpoint{2.844971in}{1.644579in}}%
\pgfpathlineto{\pgfqpoint{2.845552in}{1.645687in}}%
\pgfpathlineto{\pgfqpoint{2.847441in}{1.646223in}}%
\pgfpathlineto{\pgfqpoint{2.848313in}{1.645614in}}%
\pgfpathlineto{\pgfqpoint{2.848458in}{1.645233in}}%
\pgfpathlineto{\pgfqpoint{2.849621in}{1.643129in}}%
\pgfpathlineto{\pgfqpoint{2.850493in}{1.643807in}}%
\pgfpathlineto{\pgfqpoint{2.854125in}{1.646045in}}%
\pgfpathlineto{\pgfqpoint{2.854416in}{1.645549in}}%
\pgfpathlineto{\pgfqpoint{2.855579in}{1.643176in}}%
\pgfpathlineto{\pgfqpoint{2.856305in}{1.644502in}}%
\pgfpathlineto{\pgfqpoint{2.858630in}{1.647077in}}%
\pgfpathlineto{\pgfqpoint{2.859066in}{1.646321in}}%
\pgfpathlineto{\pgfqpoint{2.861972in}{1.642429in}}%
\pgfpathlineto{\pgfqpoint{2.862989in}{1.643998in}}%
\pgfpathlineto{\pgfqpoint{2.864733in}{1.645640in}}%
\pgfpathlineto{\pgfqpoint{2.865169in}{1.645150in}}%
\pgfpathlineto{\pgfqpoint{2.867204in}{1.642603in}}%
\pgfpathlineto{\pgfqpoint{2.867785in}{1.643281in}}%
\pgfpathlineto{\pgfqpoint{2.870982in}{1.647749in}}%
\pgfpathlineto{\pgfqpoint{2.871563in}{1.646909in}}%
\pgfpathlineto{\pgfqpoint{2.873161in}{1.644627in}}%
\pgfpathlineto{\pgfqpoint{2.873888in}{1.645515in}}%
\pgfpathlineto{\pgfqpoint{2.876504in}{1.648037in}}%
\pgfpathlineto{\pgfqpoint{2.876939in}{1.647379in}}%
\pgfpathlineto{\pgfqpoint{2.879700in}{1.642524in}}%
\pgfpathlineto{\pgfqpoint{2.880282in}{1.643747in}}%
\pgfpathlineto{\pgfqpoint{2.881299in}{1.645867in}}%
\pgfpathlineto{\pgfqpoint{2.882171in}{1.645191in}}%
\pgfpathlineto{\pgfqpoint{2.883769in}{1.643483in}}%
\pgfpathlineto{\pgfqpoint{2.885949in}{1.641952in}}%
\pgfpathlineto{\pgfqpoint{2.889000in}{1.645146in}}%
\pgfpathlineto{\pgfqpoint{2.889727in}{1.643562in}}%
\pgfpathlineto{\pgfqpoint{2.890454in}{1.642563in}}%
\pgfpathlineto{\pgfqpoint{2.891180in}{1.643688in}}%
\pgfpathlineto{\pgfqpoint{2.893941in}{1.648250in}}%
\pgfpathlineto{\pgfqpoint{2.895249in}{1.646359in}}%
\pgfpathlineto{\pgfqpoint{2.896411in}{1.644237in}}%
\pgfpathlineto{\pgfqpoint{2.896993in}{1.643407in}}%
\pgfpathlineto{\pgfqpoint{2.897719in}{1.644982in}}%
\pgfpathlineto{\pgfqpoint{2.900189in}{1.647746in}}%
\pgfpathlineto{\pgfqpoint{2.901207in}{1.646028in}}%
\pgfpathlineto{\pgfqpoint{2.902950in}{1.641974in}}%
\pgfpathlineto{\pgfqpoint{2.903677in}{1.643095in}}%
\pgfpathlineto{\pgfqpoint{2.906002in}{1.645963in}}%
\pgfpathlineto{\pgfqpoint{2.906438in}{1.645373in}}%
\pgfpathlineto{\pgfqpoint{2.909344in}{1.641207in}}%
\pgfpathlineto{\pgfqpoint{2.909635in}{1.641547in}}%
\pgfpathlineto{\pgfqpoint{2.911379in}{1.645944in}}%
\pgfpathlineto{\pgfqpoint{2.912541in}{1.644270in}}%
\pgfpathlineto{\pgfqpoint{2.914866in}{1.640597in}}%
\pgfpathlineto{\pgfqpoint{2.915738in}{1.642893in}}%
\pgfpathlineto{\pgfqpoint{2.918063in}{1.646236in}}%
\pgfpathlineto{\pgfqpoint{2.918499in}{1.645275in}}%
\pgfpathlineto{\pgfqpoint{2.920824in}{1.642424in}}%
\pgfpathlineto{\pgfqpoint{2.921841in}{1.644006in}}%
\pgfpathlineto{\pgfqpoint{2.923730in}{1.648000in}}%
\pgfpathlineto{\pgfqpoint{2.924311in}{1.647144in}}%
\pgfpathlineto{\pgfqpoint{2.927072in}{1.642961in}}%
\pgfpathlineto{\pgfqpoint{2.927218in}{1.643031in}}%
\pgfpathlineto{\pgfqpoint{2.930414in}{1.646796in}}%
\pgfpathlineto{\pgfqpoint{2.930705in}{1.646112in}}%
\pgfpathlineto{\pgfqpoint{2.932158in}{1.639809in}}%
\pgfpathlineto{\pgfqpoint{2.933175in}{1.642214in}}%
\pgfpathlineto{\pgfqpoint{2.935646in}{1.644162in}}%
\pgfpathlineto{\pgfqpoint{2.936227in}{1.643482in}}%
\pgfpathlineto{\pgfqpoint{2.938261in}{1.638343in}}%
\pgfpathlineto{\pgfqpoint{2.939424in}{1.640712in}}%
\pgfpathlineto{\pgfqpoint{2.941168in}{1.645803in}}%
\pgfpathlineto{\pgfqpoint{2.942039in}{1.643661in}}%
\pgfpathlineto{\pgfqpoint{2.944510in}{1.641425in}}%
\pgfpathlineto{\pgfqpoint{2.944800in}{1.641839in}}%
\pgfpathlineto{\pgfqpoint{2.947271in}{1.649175in}}%
\pgfpathlineto{\pgfqpoint{2.948143in}{1.647752in}}%
\pgfpathlineto{\pgfqpoint{2.949886in}{1.643691in}}%
\pgfpathlineto{\pgfqpoint{2.950758in}{1.644197in}}%
\pgfpathlineto{\pgfqpoint{2.952357in}{1.645927in}}%
\pgfpathlineto{\pgfqpoint{2.953229in}{1.647911in}}%
\pgfpathlineto{\pgfqpoint{2.953810in}{1.646370in}}%
\pgfpathlineto{\pgfqpoint{2.956135in}{1.640881in}}%
\pgfpathlineto{\pgfqpoint{2.956425in}{1.641047in}}%
\pgfpathlineto{\pgfqpoint{2.959477in}{1.646350in}}%
\pgfpathlineto{\pgfqpoint{2.960785in}{1.644234in}}%
\pgfpathlineto{\pgfqpoint{2.962093in}{1.641609in}}%
\pgfpathlineto{\pgfqpoint{2.962674in}{1.643128in}}%
\pgfpathlineto{\pgfqpoint{2.965144in}{1.647434in}}%
\pgfpathlineto{\pgfqpoint{2.965580in}{1.647815in}}%
\pgfpathlineto{\pgfqpoint{2.966016in}{1.646676in}}%
\pgfpathlineto{\pgfqpoint{2.968486in}{1.641791in}}%
\pgfpathlineto{\pgfqpoint{2.969213in}{1.643278in}}%
\pgfpathlineto{\pgfqpoint{2.970375in}{1.646768in}}%
\pgfpathlineto{\pgfqpoint{2.971247in}{1.645404in}}%
\pgfpathlineto{\pgfqpoint{2.974589in}{1.642170in}}%
\pgfpathlineto{\pgfqpoint{2.974880in}{1.642662in}}%
\pgfpathlineto{\pgfqpoint{2.976333in}{1.648439in}}%
\pgfpathlineto{\pgfqpoint{2.977496in}{1.645902in}}%
\pgfpathlineto{\pgfqpoint{2.981274in}{1.644972in}}%
\pgfpathlineto{\pgfqpoint{2.982146in}{1.646719in}}%
\pgfpathlineto{\pgfqpoint{2.982727in}{1.647897in}}%
\pgfpathlineto{\pgfqpoint{2.983454in}{1.646105in}}%
\pgfpathlineto{\pgfqpoint{2.985633in}{1.640468in}}%
\pgfpathlineto{\pgfqpoint{2.986069in}{1.640949in}}%
\pgfpathlineto{\pgfqpoint{2.988685in}{1.644540in}}%
\pgfpathlineto{\pgfqpoint{2.989266in}{1.643630in}}%
\pgfpathlineto{\pgfqpoint{2.991591in}{1.638867in}}%
\pgfpathlineto{\pgfqpoint{2.992172in}{1.640080in}}%
\pgfpathlineto{\pgfqpoint{2.994788in}{1.645055in}}%
\pgfpathlineto{\pgfqpoint{2.996241in}{1.643368in}}%
\pgfpathlineto{\pgfqpoint{2.997404in}{1.641764in}}%
\pgfpathlineto{\pgfqpoint{2.997985in}{1.642629in}}%
\pgfpathlineto{\pgfqpoint{3.000310in}{1.646833in}}%
\pgfpathlineto{\pgfqpoint{3.000746in}{1.646255in}}%
\pgfpathlineto{\pgfqpoint{3.004088in}{1.640493in}}%
\pgfpathlineto{\pgfqpoint{3.004814in}{1.642436in}}%
\pgfpathlineto{\pgfqpoint{3.004814in}{1.642436in}}%
\pgfusepath{stroke}%
\end{pgfscope}%
\begin{pgfscope}%
\pgfpathrectangle{\pgfqpoint{0.679669in}{0.526079in}}{\pgfqpoint{2.325000in}{1.661000in}} %
\pgfusepath{clip}%
\pgfsetbuttcap%
\pgfsetroundjoin%
\pgfsetlinewidth{1.003750pt}%
\definecolor{currentstroke}{rgb}{0.627451,0.321569,0.176471}%
\pgfsetstrokecolor{currentstroke}%
\pgfsetdash{{3.700000pt}{1.600000pt}}{0.000000pt}%
\pgfpathmoveto{\pgfqpoint{0.682374in}{0.512191in}}%
\pgfpathlineto{\pgfqpoint{0.685918in}{0.776806in}}%
\pgfpathlineto{\pgfqpoint{0.691149in}{0.932653in}}%
\pgfpathlineto{\pgfqpoint{0.694782in}{0.983772in}}%
\pgfpathlineto{\pgfqpoint{0.695508in}{0.985091in}}%
\pgfpathlineto{\pgfqpoint{0.696235in}{0.983814in}}%
\pgfpathlineto{\pgfqpoint{0.697979in}{0.972477in}}%
\pgfpathlineto{\pgfqpoint{0.700885in}{0.944167in}}%
\pgfpathlineto{\pgfqpoint{0.701611in}{0.948087in}}%
\pgfpathlineto{\pgfqpoint{0.704808in}{0.995988in}}%
\pgfpathlineto{\pgfqpoint{0.710185in}{1.047572in}}%
\pgfpathlineto{\pgfqpoint{0.711638in}{1.050357in}}%
\pgfpathlineto{\pgfqpoint{0.712219in}{1.049631in}}%
\pgfpathlineto{\pgfqpoint{0.714544in}{1.041254in}}%
\pgfpathlineto{\pgfqpoint{0.716869in}{1.020364in}}%
\pgfpathlineto{\pgfqpoint{0.720066in}{0.985275in}}%
\pgfpathlineto{\pgfqpoint{0.720793in}{0.988009in}}%
\pgfpathlineto{\pgfqpoint{0.723118in}{1.019847in}}%
\pgfpathlineto{\pgfqpoint{0.730383in}{1.112449in}}%
\pgfpathlineto{\pgfqpoint{0.732418in}{1.117013in}}%
\pgfpathlineto{\pgfqpoint{0.732854in}{1.116521in}}%
\pgfpathlineto{\pgfqpoint{0.734161in}{1.110570in}}%
\pgfpathlineto{\pgfqpoint{0.737504in}{1.072725in}}%
\pgfpathlineto{\pgfqpoint{0.741572in}{1.041149in}}%
\pgfpathlineto{\pgfqpoint{0.741718in}{1.041260in}}%
\pgfpathlineto{\pgfqpoint{0.742735in}{1.045869in}}%
\pgfpathlineto{\pgfqpoint{0.746513in}{1.068380in}}%
\pgfpathlineto{\pgfqpoint{0.747094in}{1.067469in}}%
\pgfpathlineto{\pgfqpoint{0.750146in}{1.054033in}}%
\pgfpathlineto{\pgfqpoint{0.752907in}{1.044933in}}%
\pgfpathlineto{\pgfqpoint{0.753343in}{1.045556in}}%
\pgfpathlineto{\pgfqpoint{0.754796in}{1.054695in}}%
\pgfpathlineto{\pgfqpoint{0.761771in}{1.126157in}}%
\pgfpathlineto{\pgfqpoint{0.762933in}{1.122246in}}%
\pgfpathlineto{\pgfqpoint{0.764677in}{1.100412in}}%
\pgfpathlineto{\pgfqpoint{0.769472in}{1.021471in}}%
\pgfpathlineto{\pgfqpoint{0.770199in}{1.023143in}}%
\pgfpathlineto{\pgfqpoint{0.771652in}{1.037769in}}%
\pgfpathlineto{\pgfqpoint{0.778046in}{1.137522in}}%
\pgfpathlineto{\pgfqpoint{0.779208in}{1.136206in}}%
\pgfpathlineto{\pgfqpoint{0.780952in}{1.128686in}}%
\pgfpathlineto{\pgfqpoint{0.783422in}{1.095674in}}%
\pgfpathlineto{\pgfqpoint{0.785311in}{1.079101in}}%
\pgfpathlineto{\pgfqpoint{0.785893in}{1.080399in}}%
\pgfpathlineto{\pgfqpoint{0.789961in}{1.106612in}}%
\pgfpathlineto{\pgfqpoint{0.794321in}{1.143541in}}%
\pgfpathlineto{\pgfqpoint{0.794611in}{1.143502in}}%
\pgfpathlineto{\pgfqpoint{0.795919in}{1.144697in}}%
\pgfpathlineto{\pgfqpoint{0.798244in}{1.150496in}}%
\pgfpathlineto{\pgfqpoint{0.798971in}{1.148637in}}%
\pgfpathlineto{\pgfqpoint{0.801005in}{1.131061in}}%
\pgfpathlineto{\pgfqpoint{0.802894in}{1.121771in}}%
\pgfpathlineto{\pgfqpoint{0.803330in}{1.122131in}}%
\pgfpathlineto{\pgfqpoint{0.807108in}{1.129107in}}%
\pgfpathlineto{\pgfqpoint{0.813938in}{1.165577in}}%
\pgfpathlineto{\pgfqpoint{0.814955in}{1.166295in}}%
\pgfpathlineto{\pgfqpoint{0.815536in}{1.165219in}}%
\pgfpathlineto{\pgfqpoint{0.816844in}{1.154924in}}%
\pgfpathlineto{\pgfqpoint{0.822366in}{1.099662in}}%
\pgfpathlineto{\pgfqpoint{0.824110in}{1.094637in}}%
\pgfpathlineto{\pgfqpoint{0.824691in}{1.096998in}}%
\pgfpathlineto{\pgfqpoint{0.826289in}{1.123651in}}%
\pgfpathlineto{\pgfqpoint{0.829922in}{1.167045in}}%
\pgfpathlineto{\pgfqpoint{0.833846in}{1.192963in}}%
\pgfpathlineto{\pgfqpoint{0.835008in}{1.189755in}}%
\pgfpathlineto{\pgfqpoint{0.841547in}{1.156998in}}%
\pgfpathlineto{\pgfqpoint{0.842419in}{1.160097in}}%
\pgfpathlineto{\pgfqpoint{0.845035in}{1.190792in}}%
\pgfpathlineto{\pgfqpoint{0.849249in}{1.224325in}}%
\pgfpathlineto{\pgfqpoint{0.851138in}{1.229627in}}%
\pgfpathlineto{\pgfqpoint{0.851574in}{1.229144in}}%
\pgfpathlineto{\pgfqpoint{0.853172in}{1.222115in}}%
\pgfpathlineto{\pgfqpoint{0.859857in}{1.176689in}}%
\pgfpathlineto{\pgfqpoint{0.861019in}{1.180605in}}%
\pgfpathlineto{\pgfqpoint{0.867994in}{1.216969in}}%
\pgfpathlineto{\pgfqpoint{0.868721in}{1.215840in}}%
\pgfpathlineto{\pgfqpoint{0.870174in}{1.204604in}}%
\pgfpathlineto{\pgfqpoint{0.877875in}{1.114945in}}%
\pgfpathlineto{\pgfqpoint{0.878893in}{1.119934in}}%
\pgfpathlineto{\pgfqpoint{0.887175in}{1.182778in}}%
\pgfpathlineto{\pgfqpoint{0.888338in}{1.181756in}}%
\pgfpathlineto{\pgfqpoint{0.890227in}{1.173422in}}%
\pgfpathlineto{\pgfqpoint{0.892843in}{1.150714in}}%
\pgfpathlineto{\pgfqpoint{0.896039in}{1.117381in}}%
\pgfpathlineto{\pgfqpoint{0.896475in}{1.118098in}}%
\pgfpathlineto{\pgfqpoint{0.900835in}{1.137281in}}%
\pgfpathlineto{\pgfqpoint{0.904758in}{1.173849in}}%
\pgfpathlineto{\pgfqpoint{0.905485in}{1.172527in}}%
\pgfpathlineto{\pgfqpoint{0.908972in}{1.165056in}}%
\pgfpathlineto{\pgfqpoint{0.910135in}{1.159524in}}%
\pgfpathlineto{\pgfqpoint{0.913622in}{1.125674in}}%
\pgfpathlineto{\pgfqpoint{0.914785in}{1.133076in}}%
\pgfpathlineto{\pgfqpoint{0.922922in}{1.212725in}}%
\pgfpathlineto{\pgfqpoint{0.923504in}{1.211743in}}%
\pgfpathlineto{\pgfqpoint{0.926119in}{1.199837in}}%
\pgfpathlineto{\pgfqpoint{0.929025in}{1.171510in}}%
\pgfpathlineto{\pgfqpoint{0.931786in}{1.137448in}}%
\pgfpathlineto{\pgfqpoint{0.932368in}{1.138951in}}%
\pgfpathlineto{\pgfqpoint{0.937889in}{1.175346in}}%
\pgfpathlineto{\pgfqpoint{0.940214in}{1.188426in}}%
\pgfpathlineto{\pgfqpoint{0.940650in}{1.187792in}}%
\pgfpathlineto{\pgfqpoint{0.941958in}{1.178690in}}%
\pgfpathlineto{\pgfqpoint{0.944429in}{1.160829in}}%
\pgfpathlineto{\pgfqpoint{0.945010in}{1.161354in}}%
\pgfpathlineto{\pgfqpoint{0.945736in}{1.161513in}}%
\pgfpathlineto{\pgfqpoint{0.946172in}{1.160803in}}%
\pgfpathlineto{\pgfqpoint{0.948788in}{1.153870in}}%
\pgfpathlineto{\pgfqpoint{0.949514in}{1.155856in}}%
\pgfpathlineto{\pgfqpoint{0.951839in}{1.180759in}}%
\pgfpathlineto{\pgfqpoint{0.954455in}{1.189390in}}%
\pgfpathlineto{\pgfqpoint{0.957943in}{1.200473in}}%
\pgfpathlineto{\pgfqpoint{0.958814in}{1.197063in}}%
\pgfpathlineto{\pgfqpoint{0.961721in}{1.181374in}}%
\pgfpathlineto{\pgfqpoint{0.962302in}{1.182129in}}%
\pgfpathlineto{\pgfqpoint{0.964191in}{1.186458in}}%
\pgfpathlineto{\pgfqpoint{0.964918in}{1.184513in}}%
\pgfpathlineto{\pgfqpoint{0.966371in}{1.178930in}}%
\pgfpathlineto{\pgfqpoint{0.967097in}{1.181202in}}%
\pgfpathlineto{\pgfqpoint{0.974799in}{1.223686in}}%
\pgfpathlineto{\pgfqpoint{0.975525in}{1.224787in}}%
\pgfpathlineto{\pgfqpoint{0.976252in}{1.223420in}}%
\pgfpathlineto{\pgfqpoint{0.978286in}{1.209878in}}%
\pgfpathlineto{\pgfqpoint{0.980175in}{1.201433in}}%
\pgfpathlineto{\pgfqpoint{0.980757in}{1.201854in}}%
\pgfpathlineto{\pgfqpoint{0.981629in}{1.201450in}}%
\pgfpathlineto{\pgfqpoint{0.981919in}{1.200776in}}%
\pgfpathlineto{\pgfqpoint{0.984099in}{1.195843in}}%
\pgfpathlineto{\pgfqpoint{0.984825in}{1.196906in}}%
\pgfpathlineto{\pgfqpoint{0.986424in}{1.206899in}}%
\pgfpathlineto{\pgfqpoint{0.989621in}{1.222151in}}%
\pgfpathlineto{\pgfqpoint{0.991510in}{1.227013in}}%
\pgfpathlineto{\pgfqpoint{0.993835in}{1.232617in}}%
\pgfpathlineto{\pgfqpoint{0.994416in}{1.231650in}}%
\pgfpathlineto{\pgfqpoint{0.997032in}{1.221803in}}%
\pgfpathlineto{\pgfqpoint{0.998339in}{1.223306in}}%
\pgfpathlineto{\pgfqpoint{1.002989in}{1.231549in}}%
\pgfpathlineto{\pgfqpoint{1.007204in}{1.250039in}}%
\pgfpathlineto{\pgfqpoint{1.009819in}{1.250586in}}%
\pgfpathlineto{\pgfqpoint{1.011127in}{1.252972in}}%
\pgfpathlineto{\pgfqpoint{1.011708in}{1.252123in}}%
\pgfpathlineto{\pgfqpoint{1.015486in}{1.242107in}}%
\pgfpathlineto{\pgfqpoint{1.016504in}{1.243273in}}%
\pgfpathlineto{\pgfqpoint{1.017521in}{1.243872in}}%
\pgfpathlineto{\pgfqpoint{1.018102in}{1.242920in}}%
\pgfpathlineto{\pgfqpoint{1.019555in}{1.240291in}}%
\pgfpathlineto{\pgfqpoint{1.020282in}{1.241579in}}%
\pgfpathlineto{\pgfqpoint{1.029436in}{1.264312in}}%
\pgfpathlineto{\pgfqpoint{1.031325in}{1.264129in}}%
\pgfpathlineto{\pgfqpoint{1.032052in}{1.264238in}}%
\pgfpathlineto{\pgfqpoint{1.032488in}{1.265196in}}%
\pgfpathlineto{\pgfqpoint{1.036266in}{1.273673in}}%
\pgfpathlineto{\pgfqpoint{1.036557in}{1.273555in}}%
\pgfpathlineto{\pgfqpoint{1.038736in}{1.273500in}}%
\pgfpathlineto{\pgfqpoint{1.038882in}{1.273834in}}%
\pgfpathlineto{\pgfqpoint{1.041497in}{1.280756in}}%
\pgfpathlineto{\pgfqpoint{1.042660in}{1.279914in}}%
\pgfpathlineto{\pgfqpoint{1.044113in}{1.280935in}}%
\pgfpathlineto{\pgfqpoint{1.046293in}{1.288773in}}%
\pgfpathlineto{\pgfqpoint{1.047455in}{1.290807in}}%
\pgfpathlineto{\pgfqpoint{1.048036in}{1.290041in}}%
\pgfpathlineto{\pgfqpoint{1.049925in}{1.287144in}}%
\pgfpathlineto{\pgfqpoint{1.050652in}{1.287878in}}%
\pgfpathlineto{\pgfqpoint{1.053558in}{1.291734in}}%
\pgfpathlineto{\pgfqpoint{1.054285in}{1.290707in}}%
\pgfpathlineto{\pgfqpoint{1.055447in}{1.289666in}}%
\pgfpathlineto{\pgfqpoint{1.056029in}{1.290681in}}%
\pgfpathlineto{\pgfqpoint{1.059807in}{1.298821in}}%
\pgfpathlineto{\pgfqpoint{1.060243in}{1.298582in}}%
\pgfpathlineto{\pgfqpoint{1.061986in}{1.298137in}}%
\pgfpathlineto{\pgfqpoint{1.062277in}{1.298660in}}%
\pgfpathlineto{\pgfqpoint{1.064166in}{1.301940in}}%
\pgfpathlineto{\pgfqpoint{1.064893in}{1.300843in}}%
\pgfpathlineto{\pgfqpoint{1.068961in}{1.292010in}}%
\pgfpathlineto{\pgfqpoint{1.069833in}{1.292528in}}%
\pgfpathlineto{\pgfqpoint{1.070996in}{1.291319in}}%
\pgfpathlineto{\pgfqpoint{1.073175in}{1.287850in}}%
\pgfpathlineto{\pgfqpoint{1.073757in}{1.288355in}}%
\pgfpathlineto{\pgfqpoint{1.075500in}{1.293457in}}%
\pgfpathlineto{\pgfqpoint{1.078552in}{1.299327in}}%
\pgfpathlineto{\pgfqpoint{1.080586in}{1.303693in}}%
\pgfpathlineto{\pgfqpoint{1.082475in}{1.305496in}}%
\pgfpathlineto{\pgfqpoint{1.082911in}{1.305075in}}%
\pgfpathlineto{\pgfqpoint{1.088579in}{1.293078in}}%
\pgfpathlineto{\pgfqpoint{1.091485in}{1.286053in}}%
\pgfpathlineto{\pgfqpoint{1.091921in}{1.286519in}}%
\pgfpathlineto{\pgfqpoint{1.097007in}{1.299274in}}%
\pgfpathlineto{\pgfqpoint{1.100494in}{1.307262in}}%
\pgfpathlineto{\pgfqpoint{1.100930in}{1.306898in}}%
\pgfpathlineto{\pgfqpoint{1.104999in}{1.304170in}}%
\pgfpathlineto{\pgfqpoint{1.110375in}{1.307139in}}%
\pgfpathlineto{\pgfqpoint{1.114735in}{1.312577in}}%
\pgfpathlineto{\pgfqpoint{1.125924in}{1.329740in}}%
\pgfpathlineto{\pgfqpoint{1.136677in}{1.351296in}}%
\pgfpathlineto{\pgfqpoint{1.138857in}{1.352572in}}%
\pgfpathlineto{\pgfqpoint{1.143216in}{1.355984in}}%
\pgfpathlineto{\pgfqpoint{1.144960in}{1.357957in}}%
\pgfpathlineto{\pgfqpoint{1.152952in}{1.371742in}}%
\pgfpathlineto{\pgfqpoint{1.158474in}{1.379996in}}%
\pgfpathlineto{\pgfqpoint{1.166175in}{1.389191in}}%
\pgfpathlineto{\pgfqpoint{1.168500in}{1.392094in}}%
\pgfpathlineto{\pgfqpoint{1.176783in}{1.405505in}}%
\pgfpathlineto{\pgfqpoint{1.180125in}{1.410947in}}%
\pgfpathlineto{\pgfqpoint{1.187100in}{1.422878in}}%
\pgfpathlineto{\pgfqpoint{1.190007in}{1.424386in}}%
\pgfpathlineto{\pgfqpoint{1.194802in}{1.423443in}}%
\pgfpathlineto{\pgfqpoint{1.197272in}{1.422999in}}%
\pgfpathlineto{\pgfqpoint{1.219360in}{1.441407in}}%
\pgfpathlineto{\pgfqpoint{1.229677in}{1.445607in}}%
\pgfpathlineto{\pgfqpoint{1.244499in}{1.453564in}}%
\pgfpathlineto{\pgfqpoint{1.248132in}{1.452868in}}%
\pgfpathlineto{\pgfqpoint{1.250747in}{1.453296in}}%
\pgfpathlineto{\pgfqpoint{1.256705in}{1.460036in}}%
\pgfpathlineto{\pgfqpoint{1.263680in}{1.470532in}}%
\pgfpathlineto{\pgfqpoint{1.271818in}{1.478861in}}%
\pgfpathlineto{\pgfqpoint{1.279229in}{1.487682in}}%
\pgfpathlineto{\pgfqpoint{1.287657in}{1.499527in}}%
\pgfpathlineto{\pgfqpoint{1.295794in}{1.510556in}}%
\pgfpathlineto{\pgfqpoint{1.300880in}{1.510695in}}%
\pgfpathlineto{\pgfqpoint{1.306111in}{1.510265in}}%
\pgfpathlineto{\pgfqpoint{1.310180in}{1.512297in}}%
\pgfpathlineto{\pgfqpoint{1.319625in}{1.519082in}}%
\pgfpathlineto{\pgfqpoint{1.325874in}{1.522496in}}%
\pgfpathlineto{\pgfqpoint{1.339824in}{1.535567in}}%
\pgfpathlineto{\pgfqpoint{1.349269in}{1.547635in}}%
\pgfpathlineto{\pgfqpoint{1.362347in}{1.556845in}}%
\pgfpathlineto{\pgfqpoint{1.371647in}{1.563642in}}%
\pgfpathlineto{\pgfqpoint{1.395043in}{1.577814in}}%
\pgfpathlineto{\pgfqpoint{1.406522in}{1.590207in}}%
\pgfpathlineto{\pgfqpoint{1.411899in}{1.596356in}}%
\pgfpathlineto{\pgfqpoint{1.421635in}{1.607115in}}%
\pgfpathlineto{\pgfqpoint{1.434277in}{1.619393in}}%
\pgfpathlineto{\pgfqpoint{1.442996in}{1.623530in}}%
\pgfpathlineto{\pgfqpoint{1.456800in}{1.635174in}}%
\pgfpathlineto{\pgfqpoint{1.469443in}{1.644140in}}%
\pgfpathlineto{\pgfqpoint{1.476272in}{1.651905in}}%
\pgfpathlineto{\pgfqpoint{1.485863in}{1.663104in}}%
\pgfpathlineto{\pgfqpoint{1.494872in}{1.671640in}}%
\pgfpathlineto{\pgfqpoint{1.505044in}{1.682409in}}%
\pgfpathlineto{\pgfqpoint{1.519721in}{1.693200in}}%
\pgfpathlineto{\pgfqpoint{1.531055in}{1.701340in}}%
\pgfpathlineto{\pgfqpoint{1.540500in}{1.707026in}}%
\pgfpathlineto{\pgfqpoint{1.551544in}{1.712970in}}%
\pgfpathlineto{\pgfqpoint{1.568255in}{1.725339in}}%
\pgfpathlineto{\pgfqpoint{1.586710in}{1.739750in}}%
\pgfpathlineto{\pgfqpoint{1.595719in}{1.745342in}}%
\pgfpathlineto{\pgfqpoint{1.609669in}{1.755010in}}%
\pgfpathlineto{\pgfqpoint{1.622602in}{1.764389in}}%
\pgfpathlineto{\pgfqpoint{1.634372in}{1.772729in}}%
\pgfpathlineto{\pgfqpoint{1.647741in}{1.782451in}}%
\pgfpathlineto{\pgfqpoint{1.707900in}{1.813858in}}%
\pgfpathlineto{\pgfqpoint{1.725047in}{1.826550in}}%
\pgfpathlineto{\pgfqpoint{1.758614in}{1.847191in}}%
\pgfpathlineto{\pgfqpoint{1.776633in}{1.857464in}}%
\pgfpathlineto{\pgfqpoint{1.789130in}{1.862537in}}%
\pgfpathlineto{\pgfqpoint{1.817030in}{1.870806in}}%
\pgfpathlineto{\pgfqpoint{1.830108in}{1.875332in}}%
\pgfpathlineto{\pgfqpoint{1.940982in}{1.915724in}}%
\pgfpathlineto{\pgfqpoint{1.975566in}{1.922716in}}%
\pgfpathlineto{\pgfqpoint{1.995910in}{1.926738in}}%
\pgfpathlineto{\pgfqpoint{2.116955in}{1.949744in}}%
\pgfpathlineto{\pgfqpoint{2.214605in}{1.960261in}}%
\pgfpathlineto{\pgfqpoint{2.236547in}{1.961382in}}%
\pgfpathlineto{\pgfqpoint{2.273747in}{1.964091in}}%
\pgfpathlineto{\pgfqpoint{2.294091in}{1.965437in}}%
\pgfpathlineto{\pgfqpoint{2.326350in}{1.967765in}}%
\pgfpathlineto{\pgfqpoint{2.355558in}{1.966133in}}%
\pgfpathlineto{\pgfqpoint{2.371252in}{1.966694in}}%
\pgfpathlineto{\pgfqpoint{2.397554in}{1.966283in}}%
\pgfpathlineto{\pgfqpoint{2.427633in}{1.967464in}}%
\pgfpathlineto{\pgfqpoint{2.459166in}{1.968242in}}%
\pgfpathlineto{\pgfqpoint{2.638627in}{1.965737in}}%
\pgfpathlineto{\pgfqpoint{2.653304in}{1.966205in}}%
\pgfpathlineto{\pgfqpoint{2.703291in}{1.966645in}}%
\pgfpathlineto{\pgfqpoint{2.723054in}{1.966512in}}%
\pgfpathlineto{\pgfqpoint{2.772605in}{1.965118in}}%
\pgfpathlineto{\pgfqpoint{2.813729in}{1.961075in}}%
\pgfpathlineto{\pgfqpoint{2.876213in}{1.960118in}}%
\pgfpathlineto{\pgfqpoint{3.001618in}{1.957954in}}%
\pgfpathlineto{\pgfqpoint{3.004814in}{1.958084in}}%
\pgfpathlineto{\pgfqpoint{3.004814in}{1.958084in}}%
\pgfusepath{stroke}%
\end{pgfscope}%
\begin{pgfscope}%
\pgfpathrectangle{\pgfqpoint{0.679669in}{0.526079in}}{\pgfqpoint{2.325000in}{1.661000in}} %
\pgfusepath{clip}%
\pgfsetbuttcap%
\pgfsetroundjoin%
\pgfsetlinewidth{1.003750pt}%
\definecolor{currentstroke}{rgb}{0.000000,0.000000,0.000000}%
\pgfsetstrokecolor{currentstroke}%
\pgfsetdash{{3.700000pt}{1.600000pt}}{0.000000pt}%
\pgfpathmoveto{\pgfqpoint{0.679669in}{0.934996in}}%
\pgfpathlineto{\pgfqpoint{0.693760in}{0.948612in}}%
\pgfpathlineto{\pgfqpoint{0.707851in}{0.962228in}}%
\pgfpathlineto{\pgfqpoint{0.721942in}{0.975844in}}%
\pgfpathlineto{\pgfqpoint{0.736033in}{0.989460in}}%
\pgfpathlineto{\pgfqpoint{0.750124in}{1.003076in}}%
\pgfpathlineto{\pgfqpoint{0.764215in}{1.016693in}}%
\pgfpathlineto{\pgfqpoint{0.778306in}{1.030309in}}%
\pgfpathlineto{\pgfqpoint{0.792396in}{1.043925in}}%
\pgfpathlineto{\pgfqpoint{0.806487in}{1.057541in}}%
\pgfpathlineto{\pgfqpoint{0.820578in}{1.071157in}}%
\pgfpathlineto{\pgfqpoint{0.834669in}{1.084773in}}%
\pgfpathlineto{\pgfqpoint{0.848760in}{1.098389in}}%
\pgfpathlineto{\pgfqpoint{0.862851in}{1.112005in}}%
\pgfpathlineto{\pgfqpoint{0.876942in}{1.125622in}}%
\pgfpathlineto{\pgfqpoint{0.891033in}{1.139238in}}%
\pgfpathlineto{\pgfqpoint{0.905124in}{1.152854in}}%
\pgfpathlineto{\pgfqpoint{0.919215in}{1.166470in}}%
\pgfpathlineto{\pgfqpoint{0.933306in}{1.180086in}}%
\pgfpathlineto{\pgfqpoint{0.947396in}{1.193702in}}%
\pgfpathlineto{\pgfqpoint{0.961487in}{1.207318in}}%
\pgfpathlineto{\pgfqpoint{0.975578in}{1.220934in}}%
\pgfpathlineto{\pgfqpoint{0.989669in}{1.234551in}}%
\pgfpathlineto{\pgfqpoint{1.003760in}{1.248167in}}%
\pgfpathlineto{\pgfqpoint{1.017851in}{1.261783in}}%
\pgfpathlineto{\pgfqpoint{1.031942in}{1.275399in}}%
\pgfpathlineto{\pgfqpoint{1.046033in}{1.289015in}}%
\pgfpathlineto{\pgfqpoint{1.060124in}{1.302631in}}%
\pgfpathlineto{\pgfqpoint{1.074215in}{1.316247in}}%
\pgfpathlineto{\pgfqpoint{1.088306in}{1.329863in}}%
\pgfpathlineto{\pgfqpoint{1.102396in}{1.343480in}}%
\pgfpathlineto{\pgfqpoint{1.116487in}{1.357096in}}%
\pgfpathlineto{\pgfqpoint{1.130578in}{1.370712in}}%
\pgfpathlineto{\pgfqpoint{1.144669in}{1.384328in}}%
\pgfpathlineto{\pgfqpoint{1.158760in}{1.397944in}}%
\pgfpathlineto{\pgfqpoint{1.172851in}{1.411560in}}%
\pgfpathlineto{\pgfqpoint{1.186942in}{1.425176in}}%
\pgfpathlineto{\pgfqpoint{1.201033in}{1.438792in}}%
\pgfpathlineto{\pgfqpoint{1.215124in}{1.452409in}}%
\pgfpathlineto{\pgfqpoint{1.229215in}{1.466025in}}%
\pgfpathlineto{\pgfqpoint{1.243306in}{1.479641in}}%
\pgfpathlineto{\pgfqpoint{1.257396in}{1.493257in}}%
\pgfpathlineto{\pgfqpoint{1.271487in}{1.506873in}}%
\pgfpathlineto{\pgfqpoint{1.285578in}{1.520489in}}%
\pgfpathlineto{\pgfqpoint{1.299669in}{1.534105in}}%
\pgfpathlineto{\pgfqpoint{1.313760in}{1.547721in}}%
\pgfpathlineto{\pgfqpoint{1.327851in}{1.561338in}}%
\pgfpathlineto{\pgfqpoint{1.341942in}{1.574954in}}%
\pgfpathlineto{\pgfqpoint{1.356033in}{1.588570in}}%
\pgfpathlineto{\pgfqpoint{1.370124in}{1.602186in}}%
\pgfpathlineto{\pgfqpoint{1.384215in}{1.615802in}}%
\pgfpathlineto{\pgfqpoint{1.398306in}{1.629418in}}%
\pgfpathlineto{\pgfqpoint{1.412396in}{1.643034in}}%
\pgfpathlineto{\pgfqpoint{1.426487in}{1.656650in}}%
\pgfpathlineto{\pgfqpoint{1.440578in}{1.670267in}}%
\pgfpathlineto{\pgfqpoint{1.454669in}{1.683883in}}%
\pgfpathlineto{\pgfqpoint{1.468760in}{1.697499in}}%
\pgfpathlineto{\pgfqpoint{1.482851in}{1.711115in}}%
\pgfpathlineto{\pgfqpoint{1.496942in}{1.724731in}}%
\pgfpathlineto{\pgfqpoint{1.511033in}{1.738347in}}%
\pgfpathlineto{\pgfqpoint{1.525124in}{1.751963in}}%
\pgfpathlineto{\pgfqpoint{1.539215in}{1.765579in}}%
\pgfpathlineto{\pgfqpoint{1.553306in}{1.779196in}}%
\pgfpathlineto{\pgfqpoint{1.567396in}{1.792812in}}%
\pgfpathlineto{\pgfqpoint{1.581487in}{1.806428in}}%
\pgfpathlineto{\pgfqpoint{1.595578in}{1.820044in}}%
\pgfpathlineto{\pgfqpoint{1.609669in}{1.833660in}}%
\pgfpathlineto{\pgfqpoint{1.623760in}{1.847276in}}%
\pgfpathlineto{\pgfqpoint{1.637851in}{1.860892in}}%
\pgfpathlineto{\pgfqpoint{1.651942in}{1.874508in}}%
\pgfpathlineto{\pgfqpoint{1.666033in}{1.888125in}}%
\pgfpathlineto{\pgfqpoint{1.680124in}{1.901741in}}%
\pgfpathlineto{\pgfqpoint{1.694215in}{1.915357in}}%
\pgfpathlineto{\pgfqpoint{1.708306in}{1.928973in}}%
\pgfpathlineto{\pgfqpoint{1.722396in}{1.942589in}}%
\pgfpathlineto{\pgfqpoint{1.736487in}{1.956205in}}%
\pgfpathlineto{\pgfqpoint{1.750578in}{1.969821in}}%
\pgfpathlineto{\pgfqpoint{1.764669in}{1.983437in}}%
\pgfpathlineto{\pgfqpoint{1.778760in}{1.997054in}}%
\pgfpathlineto{\pgfqpoint{1.792851in}{2.010670in}}%
\pgfpathlineto{\pgfqpoint{1.806942in}{2.024286in}}%
\pgfpathlineto{\pgfqpoint{1.821033in}{2.037902in}}%
\pgfpathlineto{\pgfqpoint{1.835124in}{2.051518in}}%
\pgfpathlineto{\pgfqpoint{1.849215in}{2.065134in}}%
\pgfpathlineto{\pgfqpoint{1.863306in}{2.078750in}}%
\pgfpathlineto{\pgfqpoint{1.877396in}{2.092366in}}%
\pgfpathlineto{\pgfqpoint{1.891487in}{2.105983in}}%
\pgfpathlineto{\pgfqpoint{1.905578in}{2.119599in}}%
\pgfpathlineto{\pgfqpoint{1.919669in}{2.133215in}}%
\pgfpathlineto{\pgfqpoint{1.933760in}{2.146831in}}%
\pgfpathlineto{\pgfqpoint{1.947851in}{2.160447in}}%
\pgfpathlineto{\pgfqpoint{1.961942in}{2.174063in}}%
\pgfpathlineto{\pgfqpoint{1.976033in}{2.187679in}}%
\pgfpathlineto{\pgfqpoint{1.989785in}{2.200968in}}%
\pgfusepath{stroke}%
\end{pgfscope}%
\begin{pgfscope}%
\pgfsetrectcap%
\pgfsetmiterjoin%
\pgfsetlinewidth{0.803000pt}%
\definecolor{currentstroke}{rgb}{0.000000,0.000000,0.000000}%
\pgfsetstrokecolor{currentstroke}%
\pgfsetdash{}{0pt}%
\pgfpathmoveto{\pgfqpoint{0.679669in}{0.526079in}}%
\pgfpathlineto{\pgfqpoint{0.679669in}{2.187079in}}%
\pgfusepath{stroke}%
\end{pgfscope}%
\begin{pgfscope}%
\pgfsetrectcap%
\pgfsetmiterjoin%
\pgfsetlinewidth{0.803000pt}%
\definecolor{currentstroke}{rgb}{0.000000,0.000000,0.000000}%
\pgfsetstrokecolor{currentstroke}%
\pgfsetdash{}{0pt}%
\pgfpathmoveto{\pgfqpoint{3.004669in}{0.526079in}}%
\pgfpathlineto{\pgfqpoint{3.004669in}{2.187079in}}%
\pgfusepath{stroke}%
\end{pgfscope}%
\begin{pgfscope}%
\pgfsetrectcap%
\pgfsetmiterjoin%
\pgfsetlinewidth{0.803000pt}%
\definecolor{currentstroke}{rgb}{0.000000,0.000000,0.000000}%
\pgfsetstrokecolor{currentstroke}%
\pgfsetdash{}{0pt}%
\pgfpathmoveto{\pgfqpoint{0.679669in}{0.526079in}}%
\pgfpathlineto{\pgfqpoint{3.004669in}{0.526079in}}%
\pgfusepath{stroke}%
\end{pgfscope}%
\begin{pgfscope}%
\pgfsetrectcap%
\pgfsetmiterjoin%
\pgfsetlinewidth{0.803000pt}%
\definecolor{currentstroke}{rgb}{0.000000,0.000000,0.000000}%
\pgfsetstrokecolor{currentstroke}%
\pgfsetdash{}{0pt}%
\pgfpathmoveto{\pgfqpoint{0.679669in}{2.187079in}}%
\pgfpathlineto{\pgfqpoint{3.004669in}{2.187079in}}%
\pgfusepath{stroke}%
\end{pgfscope}%
\begin{pgfscope}%
\pgfsetbuttcap%
\pgfsetmiterjoin%
\definecolor{currentfill}{rgb}{1.000000,1.000000,1.000000}%
\pgfsetfillcolor{currentfill}%
\pgfsetfillopacity{0.800000}%
\pgfsetlinewidth{1.003750pt}%
\definecolor{currentstroke}{rgb}{0.800000,0.800000,0.800000}%
\pgfsetstrokecolor{currentstroke}%
\pgfsetstrokeopacity{0.800000}%
\pgfsetdash{}{0pt}%
\pgfpathmoveto{\pgfqpoint{1.124371in}{0.595524in}}%
\pgfpathlineto{\pgfqpoint{2.907447in}{0.595524in}}%
\pgfpathquadraticcurveto{\pgfqpoint{2.935225in}{0.595524in}}{\pgfqpoint{2.935225in}{0.623302in}}%
\pgfpathlineto{\pgfqpoint{2.935225in}{1.019094in}}%
\pgfpathquadraticcurveto{\pgfqpoint{2.935225in}{1.046872in}}{\pgfqpoint{2.907447in}{1.046872in}}%
\pgfpathlineto{\pgfqpoint{1.124371in}{1.046872in}}%
\pgfpathquadraticcurveto{\pgfqpoint{1.096593in}{1.046872in}}{\pgfqpoint{1.096593in}{1.019094in}}%
\pgfpathlineto{\pgfqpoint{1.096593in}{0.623302in}}%
\pgfpathquadraticcurveto{\pgfqpoint{1.096593in}{0.595524in}}{\pgfqpoint{1.124371in}{0.595524in}}%
\pgfpathclose%
\pgfusepath{stroke,fill}%
\end{pgfscope}%
\begin{pgfscope}%
\pgfsetrectcap%
\pgfsetroundjoin%
\pgfsetlinewidth{1.003750pt}%
\definecolor{currentstroke}{rgb}{1.000000,0.549020,0.000000}%
\pgfsetstrokecolor{currentstroke}%
\pgfsetdash{}{0pt}%
\pgfpathmoveto{\pgfqpoint{1.152148in}{0.934404in}}%
\pgfpathlineto{\pgfqpoint{1.429926in}{0.934404in}}%
\pgfusepath{stroke}%
\end{pgfscope}%
\begin{pgfscope}%
\pgftext[x=1.541037in,y=0.885793in,left,base]{\rmfamily\fontsize{10.000000}{12.000000}\selectfont \(\displaystyle \mathcal{E}_{B}\)}%
\end{pgfscope}%
\begin{pgfscope}%
\pgfsetrectcap%
\pgfsetroundjoin%
\pgfsetlinewidth{1.003750pt}%
\definecolor{currentstroke}{rgb}{0.501961,0.000000,0.501961}%
\pgfsetstrokecolor{currentstroke}%
\pgfsetdash{}{0pt}%
\pgfpathmoveto{\pgfqpoint{1.152148in}{0.730547in}}%
\pgfpathlineto{\pgfqpoint{1.429926in}{0.730547in}}%
\pgfusepath{stroke}%
\end{pgfscope}%
\begin{pgfscope}%
\pgftext[x=1.541037in,y=0.681936in,left,base]{\rmfamily\fontsize{10.000000}{12.000000}\selectfont \(\displaystyle \mathcal{E}_{E}\)}%
\end{pgfscope}%
\begin{pgfscope}%
\pgfsetbuttcap%
\pgfsetroundjoin%
\pgfsetlinewidth{1.003750pt}%
\definecolor{currentstroke}{rgb}{0.627451,0.321569,0.176471}%
\pgfsetstrokecolor{currentstroke}%
\pgfsetdash{{3.700000pt}{1.600000pt}}{0.000000pt}%
\pgfpathmoveto{\pgfqpoint{1.987715in}{0.934404in}}%
\pgfpathlineto{\pgfqpoint{2.265493in}{0.934404in}}%
\pgfusepath{stroke}%
\end{pgfscope}%
\begin{pgfscope}%
\pgftext[x=2.376604in,y=0.885793in,left,base]{\rmfamily\fontsize{10.000000}{12.000000}\selectfont \(\displaystyle \mathcal{E}_\mathrm{c}\)}%
\end{pgfscope}%
\begin{pgfscope}%
\pgfsetbuttcap%
\pgfsetroundjoin%
\pgfsetlinewidth{1.003750pt}%
\definecolor{currentstroke}{rgb}{0.000000,0.000000,0.000000}%
\pgfsetstrokecolor{currentstroke}%
\pgfsetdash{{3.700000pt}{1.600000pt}}{0.000000pt}%
\pgfpathmoveto{\pgfqpoint{1.987715in}{0.730547in}}%
\pgfpathlineto{\pgfqpoint{2.265493in}{0.730547in}}%
\pgfusepath{stroke}%
\end{pgfscope}%
\begin{pgfscope}%
\pgftext[x=2.376604in,y=0.681936in,left,base]{\rmfamily\fontsize{10.000000}{12.000000}\selectfont growth}%
\end{pgfscope}%
\end{pgfpicture}%
\makeatother%
\endgroup%
}
\subfigure[]{%% Creator: Matplotlib, PGF backend
%%
%% To include the figure in your LaTeX document, write
%%   \input{<filename>.pgf}
%%
%% Make sure the required packages are loaded in your preamble
%%   \usepackage{pgf}
%%
%% Figures using additional raster images can only be included by \input if
%% they are in the same directory as the main LaTeX file. For loading figures
%% from other directories you can use the `import` package
%%   \usepackage{import}
%% and then include the figures with
%%   \import{<path to file>}{<filename>.pgf}
%%
%% Matplotlib used the following preamble
%%   \usepackage{fontspec}
%%   \setmainfont{DejaVu Serif}
%%   \setsansfont{DejaVu Sans}
%%   \setmonofont{DejaVu Sans Mono}
%%
\begingroup%
\makeatletter%
\begin{pgfpicture}%
\pgfpathrectangle{\pgfpointorigin}{\pgfqpoint{3.208836in}{2.339841in}}%
\pgfusepath{use as bounding box, clip}%
\begin{pgfscope}%
\pgfsetbuttcap%
\pgfsetmiterjoin%
\definecolor{currentfill}{rgb}{1.000000,1.000000,1.000000}%
\pgfsetfillcolor{currentfill}%
\pgfsetlinewidth{0.000000pt}%
\definecolor{currentstroke}{rgb}{1.000000,1.000000,1.000000}%
\pgfsetstrokecolor{currentstroke}%
\pgfsetdash{}{0pt}%
\pgfpathmoveto{\pgfqpoint{0.000000in}{0.000000in}}%
\pgfpathlineto{\pgfqpoint{3.208836in}{0.000000in}}%
\pgfpathlineto{\pgfqpoint{3.208836in}{2.339841in}}%
\pgfpathlineto{\pgfqpoint{0.000000in}{2.339841in}}%
\pgfpathclose%
\pgfusepath{fill}%
\end{pgfscope}%
\begin{pgfscope}%
\pgfsetbuttcap%
\pgfsetmiterjoin%
\definecolor{currentfill}{rgb}{1.000000,1.000000,1.000000}%
\pgfsetfillcolor{currentfill}%
\pgfsetlinewidth{0.000000pt}%
\definecolor{currentstroke}{rgb}{0.000000,0.000000,0.000000}%
\pgfsetstrokecolor{currentstroke}%
\pgfsetstrokeopacity{0.000000}%
\pgfsetdash{}{0pt}%
\pgfpathmoveto{\pgfqpoint{0.679669in}{0.526079in}}%
\pgfpathlineto{\pgfqpoint{3.004669in}{0.526079in}}%
\pgfpathlineto{\pgfqpoint{3.004669in}{2.187079in}}%
\pgfpathlineto{\pgfqpoint{0.679669in}{2.187079in}}%
\pgfpathclose%
\pgfusepath{fill}%
\end{pgfscope}%
\begin{pgfscope}%
\pgfsetbuttcap%
\pgfsetroundjoin%
\definecolor{currentfill}{rgb}{0.000000,0.000000,0.000000}%
\pgfsetfillcolor{currentfill}%
\pgfsetlinewidth{0.803000pt}%
\definecolor{currentstroke}{rgb}{0.000000,0.000000,0.000000}%
\pgfsetstrokecolor{currentstroke}%
\pgfsetdash{}{0pt}%
\pgfsys@defobject{currentmarker}{\pgfqpoint{0.000000in}{-0.048611in}}{\pgfqpoint{0.000000in}{0.000000in}}{%
\pgfpathmoveto{\pgfqpoint{0.000000in}{0.000000in}}%
\pgfpathlineto{\pgfqpoint{0.000000in}{-0.048611in}}%
\pgfusepath{stroke,fill}%
}%
\begin{pgfscope}%
\pgfsys@transformshift{0.679669in}{0.526079in}%
\pgfsys@useobject{currentmarker}{}%
\end{pgfscope}%
\end{pgfscope}%
\begin{pgfscope}%
\pgftext[x=0.679669in,y=0.428857in,,top]{\rmfamily\fontsize{10.000000}{12.000000}\selectfont \(\displaystyle 0\)}%
\end{pgfscope}%
\begin{pgfscope}%
\pgfsetbuttcap%
\pgfsetroundjoin%
\definecolor{currentfill}{rgb}{0.000000,0.000000,0.000000}%
\pgfsetfillcolor{currentfill}%
\pgfsetlinewidth{0.803000pt}%
\definecolor{currentstroke}{rgb}{0.000000,0.000000,0.000000}%
\pgfsetstrokecolor{currentstroke}%
\pgfsetdash{}{0pt}%
\pgfsys@defobject{currentmarker}{\pgfqpoint{0.000000in}{-0.048611in}}{\pgfqpoint{0.000000in}{0.000000in}}{%
\pgfpathmoveto{\pgfqpoint{0.000000in}{0.000000in}}%
\pgfpathlineto{\pgfqpoint{0.000000in}{-0.048611in}}%
\pgfusepath{stroke,fill}%
}%
\begin{pgfscope}%
\pgfsys@transformshift{1.260919in}{0.526079in}%
\pgfsys@useobject{currentmarker}{}%
\end{pgfscope}%
\end{pgfscope}%
\begin{pgfscope}%
\pgftext[x=1.260919in,y=0.428857in,,top]{\rmfamily\fontsize{10.000000}{12.000000}\selectfont \(\displaystyle 50\)}%
\end{pgfscope}%
\begin{pgfscope}%
\pgfsetbuttcap%
\pgfsetroundjoin%
\definecolor{currentfill}{rgb}{0.000000,0.000000,0.000000}%
\pgfsetfillcolor{currentfill}%
\pgfsetlinewidth{0.803000pt}%
\definecolor{currentstroke}{rgb}{0.000000,0.000000,0.000000}%
\pgfsetstrokecolor{currentstroke}%
\pgfsetdash{}{0pt}%
\pgfsys@defobject{currentmarker}{\pgfqpoint{0.000000in}{-0.048611in}}{\pgfqpoint{0.000000in}{0.000000in}}{%
\pgfpathmoveto{\pgfqpoint{0.000000in}{0.000000in}}%
\pgfpathlineto{\pgfqpoint{0.000000in}{-0.048611in}}%
\pgfusepath{stroke,fill}%
}%
\begin{pgfscope}%
\pgfsys@transformshift{1.842169in}{0.526079in}%
\pgfsys@useobject{currentmarker}{}%
\end{pgfscope}%
\end{pgfscope}%
\begin{pgfscope}%
\pgftext[x=1.842169in,y=0.428857in,,top]{\rmfamily\fontsize{10.000000}{12.000000}\selectfont \(\displaystyle 100\)}%
\end{pgfscope}%
\begin{pgfscope}%
\pgfsetbuttcap%
\pgfsetroundjoin%
\definecolor{currentfill}{rgb}{0.000000,0.000000,0.000000}%
\pgfsetfillcolor{currentfill}%
\pgfsetlinewidth{0.803000pt}%
\definecolor{currentstroke}{rgb}{0.000000,0.000000,0.000000}%
\pgfsetstrokecolor{currentstroke}%
\pgfsetdash{}{0pt}%
\pgfsys@defobject{currentmarker}{\pgfqpoint{0.000000in}{-0.048611in}}{\pgfqpoint{0.000000in}{0.000000in}}{%
\pgfpathmoveto{\pgfqpoint{0.000000in}{0.000000in}}%
\pgfpathlineto{\pgfqpoint{0.000000in}{-0.048611in}}%
\pgfusepath{stroke,fill}%
}%
\begin{pgfscope}%
\pgfsys@transformshift{2.423419in}{0.526079in}%
\pgfsys@useobject{currentmarker}{}%
\end{pgfscope}%
\end{pgfscope}%
\begin{pgfscope}%
\pgftext[x=2.423419in,y=0.428857in,,top]{\rmfamily\fontsize{10.000000}{12.000000}\selectfont \(\displaystyle 150\)}%
\end{pgfscope}%
\begin{pgfscope}%
\pgfsetbuttcap%
\pgfsetroundjoin%
\definecolor{currentfill}{rgb}{0.000000,0.000000,0.000000}%
\pgfsetfillcolor{currentfill}%
\pgfsetlinewidth{0.803000pt}%
\definecolor{currentstroke}{rgb}{0.000000,0.000000,0.000000}%
\pgfsetstrokecolor{currentstroke}%
\pgfsetdash{}{0pt}%
\pgfsys@defobject{currentmarker}{\pgfqpoint{0.000000in}{-0.048611in}}{\pgfqpoint{0.000000in}{0.000000in}}{%
\pgfpathmoveto{\pgfqpoint{0.000000in}{0.000000in}}%
\pgfpathlineto{\pgfqpoint{0.000000in}{-0.048611in}}%
\pgfusepath{stroke,fill}%
}%
\begin{pgfscope}%
\pgfsys@transformshift{3.004669in}{0.526079in}%
\pgfsys@useobject{currentmarker}{}%
\end{pgfscope}%
\end{pgfscope}%
\begin{pgfscope}%
\pgftext[x=3.004669in,y=0.428857in,,top]{\rmfamily\fontsize{10.000000}{12.000000}\selectfont \(\displaystyle 200\)}%
\end{pgfscope}%
\begin{pgfscope}%
\pgftext[x=1.842169in,y=0.238889in,,top]{\rmfamily\fontsize{10.000000}{12.000000}\selectfont \(\displaystyle t|\Omega_\mathrm{ce}|\)}%
\end{pgfscope}%
\begin{pgfscope}%
\pgfsetbuttcap%
\pgfsetroundjoin%
\definecolor{currentfill}{rgb}{0.000000,0.000000,0.000000}%
\pgfsetfillcolor{currentfill}%
\pgfsetlinewidth{0.803000pt}%
\definecolor{currentstroke}{rgb}{0.000000,0.000000,0.000000}%
\pgfsetstrokecolor{currentstroke}%
\pgfsetdash{}{0pt}%
\pgfsys@defobject{currentmarker}{\pgfqpoint{-0.048611in}{0.000000in}}{\pgfqpoint{0.000000in}{0.000000in}}{%
\pgfpathmoveto{\pgfqpoint{0.000000in}{0.000000in}}%
\pgfpathlineto{\pgfqpoint{-0.048611in}{0.000000in}}%
\pgfusepath{stroke,fill}%
}%
\begin{pgfscope}%
\pgfsys@transformshift{0.679669in}{0.526079in}%
\pgfsys@useobject{currentmarker}{}%
\end{pgfscope}%
\end{pgfscope}%
\begin{pgfscope}%
\pgftext[x=0.294444in,y=0.473318in,left,base]{\rmfamily\fontsize{10.000000}{12.000000}\selectfont \(\displaystyle 10^{-8}\)}%
\end{pgfscope}%
\begin{pgfscope}%
\pgfsetbuttcap%
\pgfsetroundjoin%
\definecolor{currentfill}{rgb}{0.000000,0.000000,0.000000}%
\pgfsetfillcolor{currentfill}%
\pgfsetlinewidth{0.803000pt}%
\definecolor{currentstroke}{rgb}{0.000000,0.000000,0.000000}%
\pgfsetstrokecolor{currentstroke}%
\pgfsetdash{}{0pt}%
\pgfsys@defobject{currentmarker}{\pgfqpoint{-0.048611in}{0.000000in}}{\pgfqpoint{0.000000in}{0.000000in}}{%
\pgfpathmoveto{\pgfqpoint{0.000000in}{0.000000in}}%
\pgfpathlineto{\pgfqpoint{-0.048611in}{0.000000in}}%
\pgfusepath{stroke,fill}%
}%
\begin{pgfscope}%
\pgfsys@transformshift{0.679669in}{1.079746in}%
\pgfsys@useobject{currentmarker}{}%
\end{pgfscope}%
\end{pgfscope}%
\begin{pgfscope}%
\pgftext[x=0.294444in,y=1.026985in,left,base]{\rmfamily\fontsize{10.000000}{12.000000}\selectfont \(\displaystyle 10^{-6}\)}%
\end{pgfscope}%
\begin{pgfscope}%
\pgfsetbuttcap%
\pgfsetroundjoin%
\definecolor{currentfill}{rgb}{0.000000,0.000000,0.000000}%
\pgfsetfillcolor{currentfill}%
\pgfsetlinewidth{0.803000pt}%
\definecolor{currentstroke}{rgb}{0.000000,0.000000,0.000000}%
\pgfsetstrokecolor{currentstroke}%
\pgfsetdash{}{0pt}%
\pgfsys@defobject{currentmarker}{\pgfqpoint{-0.048611in}{0.000000in}}{\pgfqpoint{0.000000in}{0.000000in}}{%
\pgfpathmoveto{\pgfqpoint{0.000000in}{0.000000in}}%
\pgfpathlineto{\pgfqpoint{-0.048611in}{0.000000in}}%
\pgfusepath{stroke,fill}%
}%
\begin{pgfscope}%
\pgfsys@transformshift{0.679669in}{1.633413in}%
\pgfsys@useobject{currentmarker}{}%
\end{pgfscope}%
\end{pgfscope}%
\begin{pgfscope}%
\pgftext[x=0.294444in,y=1.580651in,left,base]{\rmfamily\fontsize{10.000000}{12.000000}\selectfont \(\displaystyle 10^{-4}\)}%
\end{pgfscope}%
\begin{pgfscope}%
\pgfsetbuttcap%
\pgfsetroundjoin%
\definecolor{currentfill}{rgb}{0.000000,0.000000,0.000000}%
\pgfsetfillcolor{currentfill}%
\pgfsetlinewidth{0.803000pt}%
\definecolor{currentstroke}{rgb}{0.000000,0.000000,0.000000}%
\pgfsetstrokecolor{currentstroke}%
\pgfsetdash{}{0pt}%
\pgfsys@defobject{currentmarker}{\pgfqpoint{-0.048611in}{0.000000in}}{\pgfqpoint{0.000000in}{0.000000in}}{%
\pgfpathmoveto{\pgfqpoint{0.000000in}{0.000000in}}%
\pgfpathlineto{\pgfqpoint{-0.048611in}{0.000000in}}%
\pgfusepath{stroke,fill}%
}%
\begin{pgfscope}%
\pgfsys@transformshift{0.679669in}{2.187079in}%
\pgfsys@useobject{currentmarker}{}%
\end{pgfscope}%
\end{pgfscope}%
\begin{pgfscope}%
\pgftext[x=0.294444in,y=2.134318in,left,base]{\rmfamily\fontsize{10.000000}{12.000000}\selectfont \(\displaystyle 10^{-2}\)}%
\end{pgfscope}%
\begin{pgfscope}%
\pgftext[x=0.238889in,y=1.356579in,,bottom,rotate=90.000000]{\rmfamily\fontsize{10.000000}{12.000000}\selectfont \(\displaystyle \mathcal{E} / \mathcal{E}(0)\)}%
\end{pgfscope}%
\begin{pgfscope}%
\pgfpathrectangle{\pgfqpoint{0.679669in}{0.526079in}}{\pgfqpoint{2.325000in}{1.661000in}} %
\pgfusepath{clip}%
\pgfsetrectcap%
\pgfsetroundjoin%
\pgfsetlinewidth{1.003750pt}%
\definecolor{currentstroke}{rgb}{1.000000,0.549020,0.000000}%
\pgfsetstrokecolor{currentstroke}%
\pgfsetdash{}{0pt}%
\pgfpathmoveto{\pgfqpoint{0.679669in}{0.674098in}}%
\pgfpathlineto{\pgfqpoint{0.680105in}{0.675227in}}%
\pgfpathlineto{\pgfqpoint{0.680977in}{0.737943in}}%
\pgfpathlineto{\pgfqpoint{0.683302in}{0.812370in}}%
\pgfpathlineto{\pgfqpoint{0.683447in}{0.812212in}}%
\pgfpathlineto{\pgfqpoint{0.684319in}{0.804560in}}%
\pgfpathlineto{\pgfqpoint{0.684610in}{0.802102in}}%
\pgfpathlineto{\pgfqpoint{0.685191in}{0.807980in}}%
\pgfpathlineto{\pgfqpoint{0.690277in}{0.877301in}}%
\pgfpathlineto{\pgfqpoint{0.693764in}{0.895841in}}%
\pgfpathlineto{\pgfqpoint{0.694055in}{0.893477in}}%
\pgfpathlineto{\pgfqpoint{0.699141in}{0.788801in}}%
\pgfpathlineto{\pgfqpoint{0.700594in}{0.803337in}}%
\pgfpathlineto{\pgfqpoint{0.701466in}{0.815769in}}%
\pgfpathlineto{\pgfqpoint{0.703500in}{0.851559in}}%
\pgfpathlineto{\pgfqpoint{0.703791in}{0.849864in}}%
\pgfpathlineto{\pgfqpoint{0.709604in}{0.768838in}}%
\pgfpathlineto{\pgfqpoint{0.710185in}{0.774889in}}%
\pgfpathlineto{\pgfqpoint{0.712800in}{0.814712in}}%
\pgfpathlineto{\pgfqpoint{0.714689in}{0.859235in}}%
\pgfpathlineto{\pgfqpoint{0.715125in}{0.858553in}}%
\pgfpathlineto{\pgfqpoint{0.715997in}{0.876943in}}%
\pgfpathlineto{\pgfqpoint{0.717741in}{0.909027in}}%
\pgfpathlineto{\pgfqpoint{0.718613in}{0.899476in}}%
\pgfpathlineto{\pgfqpoint{0.719921in}{0.875721in}}%
\pgfpathlineto{\pgfqpoint{0.722536in}{0.817544in}}%
\pgfpathlineto{\pgfqpoint{0.723554in}{0.792547in}}%
\pgfpathlineto{\pgfqpoint{0.724571in}{0.803945in}}%
\pgfpathlineto{\pgfqpoint{0.725007in}{0.805161in}}%
\pgfpathlineto{\pgfqpoint{0.725443in}{0.803126in}}%
\pgfpathlineto{\pgfqpoint{0.726314in}{0.784486in}}%
\pgfpathlineto{\pgfqpoint{0.726896in}{0.802579in}}%
\pgfpathlineto{\pgfqpoint{0.727477in}{0.818673in}}%
\pgfpathlineto{\pgfqpoint{0.728494in}{0.809567in}}%
\pgfpathlineto{\pgfqpoint{0.732708in}{0.752087in}}%
\pgfpathlineto{\pgfqpoint{0.733725in}{0.764292in}}%
\pgfpathlineto{\pgfqpoint{0.738666in}{0.872608in}}%
\pgfpathlineto{\pgfqpoint{0.742299in}{0.935992in}}%
\pgfpathlineto{\pgfqpoint{0.743171in}{0.926548in}}%
\pgfpathlineto{\pgfqpoint{0.745932in}{0.901126in}}%
\pgfpathlineto{\pgfqpoint{0.750291in}{0.824330in}}%
\pgfpathlineto{\pgfqpoint{0.750582in}{0.826312in}}%
\pgfpathlineto{\pgfqpoint{0.750727in}{0.827292in}}%
\pgfpathlineto{\pgfqpoint{0.751163in}{0.821708in}}%
\pgfpathlineto{\pgfqpoint{0.751454in}{0.818732in}}%
\pgfpathlineto{\pgfqpoint{0.752035in}{0.829532in}}%
\pgfpathlineto{\pgfqpoint{0.752325in}{0.831705in}}%
\pgfpathlineto{\pgfqpoint{0.753197in}{0.825245in}}%
\pgfpathlineto{\pgfqpoint{0.754214in}{0.815705in}}%
\pgfpathlineto{\pgfqpoint{0.754650in}{0.807334in}}%
\pgfpathlineto{\pgfqpoint{0.755522in}{0.819125in}}%
\pgfpathlineto{\pgfqpoint{0.755813in}{0.819847in}}%
\pgfpathlineto{\pgfqpoint{0.756249in}{0.816948in}}%
\pgfpathlineto{\pgfqpoint{0.756830in}{0.809855in}}%
\pgfpathlineto{\pgfqpoint{0.757266in}{0.820213in}}%
\pgfpathlineto{\pgfqpoint{0.759010in}{0.878694in}}%
\pgfpathlineto{\pgfqpoint{0.759882in}{0.870984in}}%
\pgfpathlineto{\pgfqpoint{0.761189in}{0.864050in}}%
\pgfpathlineto{\pgfqpoint{0.761771in}{0.867039in}}%
\pgfpathlineto{\pgfqpoint{0.764532in}{0.903472in}}%
\pgfpathlineto{\pgfqpoint{0.765113in}{0.913734in}}%
\pgfpathlineto{\pgfqpoint{0.766275in}{0.911533in}}%
\pgfpathlineto{\pgfqpoint{0.771216in}{0.861304in}}%
\pgfpathlineto{\pgfqpoint{0.773832in}{0.825941in}}%
\pgfpathlineto{\pgfqpoint{0.773977in}{0.824291in}}%
\pgfpathlineto{\pgfqpoint{0.774704in}{0.830572in}}%
\pgfpathlineto{\pgfqpoint{0.776302in}{0.868286in}}%
\pgfpathlineto{\pgfqpoint{0.777755in}{0.856740in}}%
\pgfpathlineto{\pgfqpoint{0.778046in}{0.853450in}}%
\pgfpathlineto{\pgfqpoint{0.778772in}{0.861388in}}%
\pgfpathlineto{\pgfqpoint{0.778918in}{0.862224in}}%
\pgfpathlineto{\pgfqpoint{0.779499in}{0.857593in}}%
\pgfpathlineto{\pgfqpoint{0.785166in}{0.737276in}}%
\pgfpathlineto{\pgfqpoint{0.786038in}{0.753841in}}%
\pgfpathlineto{\pgfqpoint{0.789671in}{0.862924in}}%
\pgfpathlineto{\pgfqpoint{0.790252in}{0.870856in}}%
\pgfpathlineto{\pgfqpoint{0.791124in}{0.864857in}}%
\pgfpathlineto{\pgfqpoint{0.791414in}{0.863063in}}%
\pgfpathlineto{\pgfqpoint{0.792141in}{0.869014in}}%
\pgfpathlineto{\pgfqpoint{0.793013in}{0.876354in}}%
\pgfpathlineto{\pgfqpoint{0.793449in}{0.870356in}}%
\pgfpathlineto{\pgfqpoint{0.794902in}{0.840298in}}%
\pgfpathlineto{\pgfqpoint{0.795919in}{0.851602in}}%
\pgfpathlineto{\pgfqpoint{0.799261in}{0.875472in}}%
\pgfpathlineto{\pgfqpoint{0.799988in}{0.869456in}}%
\pgfpathlineto{\pgfqpoint{0.800569in}{0.865471in}}%
\pgfpathlineto{\pgfqpoint{0.801150in}{0.871688in}}%
\pgfpathlineto{\pgfqpoint{0.804057in}{0.903168in}}%
\pgfpathlineto{\pgfqpoint{0.802022in}{0.871362in}}%
\pgfpathlineto{\pgfqpoint{0.804493in}{0.899505in}}%
\pgfpathlineto{\pgfqpoint{0.808707in}{0.842530in}}%
\pgfpathlineto{\pgfqpoint{0.809143in}{0.847656in}}%
\pgfpathlineto{\pgfqpoint{0.809579in}{0.851435in}}%
\pgfpathlineto{\pgfqpoint{0.810450in}{0.845018in}}%
\pgfpathlineto{\pgfqpoint{0.811758in}{0.824625in}}%
\pgfpathlineto{\pgfqpoint{0.813066in}{0.833876in}}%
\pgfpathlineto{\pgfqpoint{0.815536in}{0.860851in}}%
\pgfpathlineto{\pgfqpoint{0.815682in}{0.859514in}}%
\pgfpathlineto{\pgfqpoint{0.817571in}{0.843858in}}%
\pgfpathlineto{\pgfqpoint{0.818152in}{0.844265in}}%
\pgfpathlineto{\pgfqpoint{0.819750in}{0.816930in}}%
\pgfpathlineto{\pgfqpoint{0.820477in}{0.835868in}}%
\pgfpathlineto{\pgfqpoint{0.822366in}{0.855150in}}%
\pgfpathlineto{\pgfqpoint{0.822802in}{0.851202in}}%
\pgfpathlineto{\pgfqpoint{0.823093in}{0.853791in}}%
\pgfpathlineto{\pgfqpoint{0.826289in}{0.904136in}}%
\pgfpathlineto{\pgfqpoint{0.830068in}{0.942482in}}%
\pgfpathlineto{\pgfqpoint{0.830649in}{0.944977in}}%
\pgfpathlineto{\pgfqpoint{0.831085in}{0.941558in}}%
\pgfpathlineto{\pgfqpoint{0.833555in}{0.925949in}}%
\pgfpathlineto{\pgfqpoint{0.833991in}{0.927445in}}%
\pgfpathlineto{\pgfqpoint{0.834427in}{0.924167in}}%
\pgfpathlineto{\pgfqpoint{0.839658in}{0.866700in}}%
\pgfpathlineto{\pgfqpoint{0.840094in}{0.872324in}}%
\pgfpathlineto{\pgfqpoint{0.840385in}{0.874457in}}%
\pgfpathlineto{\pgfqpoint{0.841111in}{0.867000in}}%
\pgfpathlineto{\pgfqpoint{0.842710in}{0.854647in}}%
\pgfpathlineto{\pgfqpoint{0.841838in}{0.870579in}}%
\pgfpathlineto{\pgfqpoint{0.843291in}{0.863529in}}%
\pgfpathlineto{\pgfqpoint{0.852882in}{1.003785in}}%
\pgfpathlineto{\pgfqpoint{0.854480in}{1.015323in}}%
\pgfpathlineto{\pgfqpoint{0.855207in}{1.010951in}}%
\pgfpathlineto{\pgfqpoint{0.856369in}{1.002983in}}%
\pgfpathlineto{\pgfqpoint{0.861891in}{0.915999in}}%
\pgfpathlineto{\pgfqpoint{0.865088in}{0.871406in}}%
\pgfpathlineto{\pgfqpoint{0.865669in}{0.882914in}}%
\pgfpathlineto{\pgfqpoint{0.865960in}{0.880202in}}%
\pgfpathlineto{\pgfqpoint{0.866250in}{0.875090in}}%
\pgfpathlineto{\pgfqpoint{0.866977in}{0.883993in}}%
\pgfpathlineto{\pgfqpoint{0.867268in}{0.883100in}}%
\pgfpathlineto{\pgfqpoint{0.868285in}{0.898098in}}%
\pgfpathlineto{\pgfqpoint{0.870029in}{0.911102in}}%
\pgfpathlineto{\pgfqpoint{0.870319in}{0.910762in}}%
\pgfpathlineto{\pgfqpoint{0.871482in}{0.925674in}}%
\pgfpathlineto{\pgfqpoint{0.874969in}{0.987295in}}%
\pgfpathlineto{\pgfqpoint{0.877730in}{1.013425in}}%
\pgfpathlineto{\pgfqpoint{0.879619in}{1.017104in}}%
\pgfpathlineto{\pgfqpoint{0.879764in}{1.017464in}}%
\pgfpathlineto{\pgfqpoint{0.880200in}{1.014851in}}%
\pgfpathlineto{\pgfqpoint{0.882380in}{1.006111in}}%
\pgfpathlineto{\pgfqpoint{0.882525in}{1.006205in}}%
\pgfpathlineto{\pgfqpoint{0.882961in}{1.007873in}}%
\pgfpathlineto{\pgfqpoint{0.883252in}{1.004997in}}%
\pgfpathlineto{\pgfqpoint{0.887757in}{0.940332in}}%
\pgfpathlineto{\pgfqpoint{0.888629in}{0.934342in}}%
\pgfpathlineto{\pgfqpoint{0.889064in}{0.939444in}}%
\pgfpathlineto{\pgfqpoint{0.889355in}{0.942025in}}%
\pgfpathlineto{\pgfqpoint{0.889936in}{0.930893in}}%
\pgfpathlineto{\pgfqpoint{0.891680in}{0.915711in}}%
\pgfpathlineto{\pgfqpoint{0.891971in}{0.918653in}}%
\pgfpathlineto{\pgfqpoint{0.894586in}{0.944552in}}%
\pgfpathlineto{\pgfqpoint{0.895749in}{0.969980in}}%
\pgfpathlineto{\pgfqpoint{0.903741in}{1.066303in}}%
\pgfpathlineto{\pgfqpoint{0.904904in}{1.065117in}}%
\pgfpathlineto{\pgfqpoint{0.907229in}{1.054400in}}%
\pgfpathlineto{\pgfqpoint{0.914204in}{1.005328in}}%
\pgfpathlineto{\pgfqpoint{0.914639in}{1.006194in}}%
\pgfpathlineto{\pgfqpoint{0.914930in}{1.005894in}}%
\pgfpathlineto{\pgfqpoint{0.915075in}{1.005221in}}%
\pgfpathlineto{\pgfqpoint{0.915366in}{1.004484in}}%
\pgfpathlineto{\pgfqpoint{0.915657in}{1.006967in}}%
\pgfpathlineto{\pgfqpoint{0.917836in}{1.025736in}}%
\pgfpathlineto{\pgfqpoint{0.918272in}{1.024310in}}%
\pgfpathlineto{\pgfqpoint{0.918418in}{1.023863in}}%
\pgfpathlineto{\pgfqpoint{0.918999in}{1.026797in}}%
\pgfpathlineto{\pgfqpoint{0.922341in}{1.043571in}}%
\pgfpathlineto{\pgfqpoint{0.928444in}{1.087626in}}%
\pgfpathlineto{\pgfqpoint{0.929752in}{1.083162in}}%
\pgfpathlineto{\pgfqpoint{0.930333in}{1.080580in}}%
\pgfpathlineto{\pgfqpoint{0.936000in}{1.039205in}}%
\pgfpathlineto{\pgfqpoint{0.936436in}{1.040348in}}%
\pgfpathlineto{\pgfqpoint{0.938907in}{1.038805in}}%
\pgfpathlineto{\pgfqpoint{0.939197in}{1.039751in}}%
\pgfpathlineto{\pgfqpoint{0.941232in}{1.046393in}}%
\pgfpathlineto{\pgfqpoint{0.941668in}{1.044131in}}%
\pgfpathlineto{\pgfqpoint{0.941958in}{1.042393in}}%
\pgfpathlineto{\pgfqpoint{0.942539in}{1.047400in}}%
\pgfpathlineto{\pgfqpoint{0.944574in}{1.053352in}}%
\pgfpathlineto{\pgfqpoint{0.944719in}{1.053190in}}%
\pgfpathlineto{\pgfqpoint{0.945010in}{1.053328in}}%
\pgfpathlineto{\pgfqpoint{0.945155in}{1.053914in}}%
\pgfpathlineto{\pgfqpoint{0.949079in}{1.067204in}}%
\pgfpathlineto{\pgfqpoint{0.950241in}{1.073662in}}%
\pgfpathlineto{\pgfqpoint{0.952421in}{1.082387in}}%
\pgfpathlineto{\pgfqpoint{0.953874in}{1.087250in}}%
\pgfpathlineto{\pgfqpoint{0.954891in}{1.086904in}}%
\pgfpathlineto{\pgfqpoint{0.955182in}{1.086807in}}%
\pgfpathlineto{\pgfqpoint{0.955472in}{1.085649in}}%
\pgfpathlineto{\pgfqpoint{0.957797in}{1.080629in}}%
\pgfpathlineto{\pgfqpoint{0.960122in}{1.076499in}}%
\pgfpathlineto{\pgfqpoint{0.960558in}{1.075006in}}%
\pgfpathlineto{\pgfqpoint{0.961575in}{1.075818in}}%
\pgfpathlineto{\pgfqpoint{0.965208in}{1.086698in}}%
\pgfpathlineto{\pgfqpoint{0.965499in}{1.086391in}}%
\pgfpathlineto{\pgfqpoint{0.967533in}{1.086943in}}%
\pgfpathlineto{\pgfqpoint{0.967969in}{1.088083in}}%
\pgfpathlineto{\pgfqpoint{0.968550in}{1.086105in}}%
\pgfpathlineto{\pgfqpoint{0.969132in}{1.087752in}}%
\pgfpathlineto{\pgfqpoint{0.972910in}{1.078079in}}%
\pgfpathlineto{\pgfqpoint{0.976397in}{1.074487in}}%
\pgfpathlineto{\pgfqpoint{0.977124in}{1.074910in}}%
\pgfpathlineto{\pgfqpoint{0.977560in}{1.073583in}}%
\pgfpathlineto{\pgfqpoint{0.977705in}{1.073405in}}%
\pgfpathlineto{\pgfqpoint{0.978141in}{1.074947in}}%
\pgfpathlineto{\pgfqpoint{0.980466in}{1.079304in}}%
\pgfpathlineto{\pgfqpoint{0.981047in}{1.077266in}}%
\pgfpathlineto{\pgfqpoint{0.981483in}{1.080577in}}%
\pgfpathlineto{\pgfqpoint{0.983663in}{1.086817in}}%
\pgfpathlineto{\pgfqpoint{0.984680in}{1.096706in}}%
\pgfpathlineto{\pgfqpoint{0.988168in}{1.114574in}}%
\pgfpathlineto{\pgfqpoint{0.992963in}{1.128395in}}%
\pgfpathlineto{\pgfqpoint{0.997758in}{1.116872in}}%
\pgfpathlineto{\pgfqpoint{1.000664in}{1.107940in}}%
\pgfpathlineto{\pgfqpoint{1.001246in}{1.105903in}}%
\pgfpathlineto{\pgfqpoint{1.002554in}{1.106891in}}%
\pgfpathlineto{\pgfqpoint{1.004879in}{1.108958in}}%
\pgfpathlineto{\pgfqpoint{1.012725in}{1.157437in}}%
\pgfpathlineto{\pgfqpoint{1.015777in}{1.172852in}}%
\pgfpathlineto{\pgfqpoint{1.016358in}{1.174507in}}%
\pgfpathlineto{\pgfqpoint{1.017230in}{1.172667in}}%
\pgfpathlineto{\pgfqpoint{1.028129in}{1.125634in}}%
\pgfpathlineto{\pgfqpoint{1.028855in}{1.126350in}}%
\pgfpathlineto{\pgfqpoint{1.029291in}{1.126465in}}%
\pgfpathlineto{\pgfqpoint{1.029727in}{1.127473in}}%
\pgfpathlineto{\pgfqpoint{1.031907in}{1.139313in}}%
\pgfpathlineto{\pgfqpoint{1.039754in}{1.198593in}}%
\pgfpathlineto{\pgfqpoint{1.041207in}{1.200492in}}%
\pgfpathlineto{\pgfqpoint{1.041788in}{1.199738in}}%
\pgfpathlineto{\pgfqpoint{1.044839in}{1.186373in}}%
\pgfpathlineto{\pgfqpoint{1.051524in}{1.139817in}}%
\pgfpathlineto{\pgfqpoint{1.052105in}{1.138765in}}%
\pgfpathlineto{\pgfqpoint{1.053122in}{1.139236in}}%
\pgfpathlineto{\pgfqpoint{1.054575in}{1.144149in}}%
\pgfpathlineto{\pgfqpoint{1.057336in}{1.164334in}}%
\pgfpathlineto{\pgfqpoint{1.063730in}{1.212214in}}%
\pgfpathlineto{\pgfqpoint{1.066491in}{1.218991in}}%
\pgfpathlineto{\pgfqpoint{1.068089in}{1.216455in}}%
\pgfpathlineto{\pgfqpoint{1.071286in}{1.202557in}}%
\pgfpathlineto{\pgfqpoint{1.077099in}{1.177064in}}%
\pgfpathlineto{\pgfqpoint{1.077244in}{1.177185in}}%
\pgfpathlineto{\pgfqpoint{1.080441in}{1.186185in}}%
\pgfpathlineto{\pgfqpoint{1.089596in}{1.235750in}}%
\pgfpathlineto{\pgfqpoint{1.090177in}{1.235735in}}%
\pgfpathlineto{\pgfqpoint{1.090758in}{1.234932in}}%
\pgfpathlineto{\pgfqpoint{1.094827in}{1.222737in}}%
\pgfpathlineto{\pgfqpoint{1.098605in}{1.200817in}}%
\pgfpathlineto{\pgfqpoint{1.101947in}{1.188220in}}%
\pgfpathlineto{\pgfqpoint{1.102529in}{1.188370in}}%
\pgfpathlineto{\pgfqpoint{1.102964in}{1.187861in}}%
\pgfpathlineto{\pgfqpoint{1.103836in}{1.188751in}}%
\pgfpathlineto{\pgfqpoint{1.105725in}{1.195882in}}%
\pgfpathlineto{\pgfqpoint{1.116769in}{1.243181in}}%
\pgfpathlineto{\pgfqpoint{1.118804in}{1.239397in}}%
\pgfpathlineto{\pgfqpoint{1.124325in}{1.230673in}}%
\pgfpathlineto{\pgfqpoint{1.126941in}{1.230300in}}%
\pgfpathlineto{\pgfqpoint{1.130574in}{1.238265in}}%
\pgfpathlineto{\pgfqpoint{1.137404in}{1.257779in}}%
\pgfpathlineto{\pgfqpoint{1.140455in}{1.260552in}}%
\pgfpathlineto{\pgfqpoint{1.140600in}{1.260439in}}%
\pgfpathlineto{\pgfqpoint{1.143797in}{1.259887in}}%
\pgfpathlineto{\pgfqpoint{1.146994in}{1.259490in}}%
\pgfpathlineto{\pgfqpoint{1.150772in}{1.264868in}}%
\pgfpathlineto{\pgfqpoint{1.153679in}{1.268446in}}%
\pgfpathlineto{\pgfqpoint{1.157747in}{1.271462in}}%
\pgfpathlineto{\pgfqpoint{1.157893in}{1.271284in}}%
\pgfpathlineto{\pgfqpoint{1.159346in}{1.271258in}}%
\pgfpathlineto{\pgfqpoint{1.159491in}{1.271401in}}%
\pgfpathlineto{\pgfqpoint{1.168500in}{1.282139in}}%
\pgfpathlineto{\pgfqpoint{1.172860in}{1.290368in}}%
\pgfpathlineto{\pgfqpoint{1.180416in}{1.301585in}}%
\pgfpathlineto{\pgfqpoint{1.180707in}{1.301331in}}%
\pgfpathlineto{\pgfqpoint{1.184194in}{1.296628in}}%
\pgfpathlineto{\pgfqpoint{1.188554in}{1.291407in}}%
\pgfpathlineto{\pgfqpoint{1.188699in}{1.291502in}}%
\pgfpathlineto{\pgfqpoint{1.194075in}{1.298648in}}%
\pgfpathlineto{\pgfqpoint{1.203375in}{1.329456in}}%
\pgfpathlineto{\pgfqpoint{1.204102in}{1.328975in}}%
\pgfpathlineto{\pgfqpoint{1.205700in}{1.328062in}}%
\pgfpathlineto{\pgfqpoint{1.214710in}{1.310400in}}%
\pgfpathlineto{\pgfqpoint{1.215582in}{1.312143in}}%
\pgfpathlineto{\pgfqpoint{1.218779in}{1.321588in}}%
\pgfpathlineto{\pgfqpoint{1.227352in}{1.353648in}}%
\pgfpathlineto{\pgfqpoint{1.230258in}{1.351702in}}%
\pgfpathlineto{\pgfqpoint{1.234618in}{1.337873in}}%
\pgfpathlineto{\pgfqpoint{1.238250in}{1.328718in}}%
\pgfpathlineto{\pgfqpoint{1.238832in}{1.328156in}}%
\pgfpathlineto{\pgfqpoint{1.239704in}{1.329225in}}%
\pgfpathlineto{\pgfqpoint{1.242174in}{1.334483in}}%
\pgfpathlineto{\pgfqpoint{1.253508in}{1.375437in}}%
\pgfpathlineto{\pgfqpoint{1.254089in}{1.374986in}}%
\pgfpathlineto{\pgfqpoint{1.256850in}{1.371021in}}%
\pgfpathlineto{\pgfqpoint{1.263535in}{1.358792in}}%
\pgfpathlineto{\pgfqpoint{1.263971in}{1.359288in}}%
\pgfpathlineto{\pgfqpoint{1.267022in}{1.365357in}}%
\pgfpathlineto{\pgfqpoint{1.276613in}{1.398698in}}%
\pgfpathlineto{\pgfqpoint{1.279519in}{1.400330in}}%
\pgfpathlineto{\pgfqpoint{1.282571in}{1.396822in}}%
\pgfpathlineto{\pgfqpoint{1.287221in}{1.391133in}}%
\pgfpathlineto{\pgfqpoint{1.289836in}{1.392706in}}%
\pgfpathlineto{\pgfqpoint{1.291435in}{1.395229in}}%
\pgfpathlineto{\pgfqpoint{1.297102in}{1.409648in}}%
\pgfpathlineto{\pgfqpoint{1.301607in}{1.417108in}}%
\pgfpathlineto{\pgfqpoint{1.304077in}{1.417625in}}%
\pgfpathlineto{\pgfqpoint{1.310325in}{1.415612in}}%
\pgfpathlineto{\pgfqpoint{1.313958in}{1.416989in}}%
\pgfpathlineto{\pgfqpoint{1.344474in}{1.448361in}}%
\pgfpathlineto{\pgfqpoint{1.352030in}{1.450094in}}%
\pgfpathlineto{\pgfqpoint{1.356825in}{1.456419in}}%
\pgfpathlineto{\pgfqpoint{1.364818in}{1.471132in}}%
\pgfpathlineto{\pgfqpoint{1.368305in}{1.471818in}}%
\pgfpathlineto{\pgfqpoint{1.377024in}{1.469113in}}%
\pgfpathlineto{\pgfqpoint{1.380657in}{1.475108in}}%
\pgfpathlineto{\pgfqpoint{1.389957in}{1.494276in}}%
\pgfpathlineto{\pgfqpoint{1.392282in}{1.493795in}}%
\pgfpathlineto{\pgfqpoint{1.396350in}{1.488721in}}%
\pgfpathlineto{\pgfqpoint{1.400274in}{1.485436in}}%
\pgfpathlineto{\pgfqpoint{1.402599in}{1.487074in}}%
\pgfpathlineto{\pgfqpoint{1.406377in}{1.495911in}}%
\pgfpathlineto{\pgfqpoint{1.415386in}{1.516401in}}%
\pgfpathlineto{\pgfqpoint{1.418002in}{1.516127in}}%
\pgfpathlineto{\pgfqpoint{1.425558in}{1.512342in}}%
\pgfpathlineto{\pgfqpoint{1.428755in}{1.517435in}}%
\pgfpathlineto{\pgfqpoint{1.440671in}{1.543478in}}%
\pgfpathlineto{\pgfqpoint{1.444449in}{1.542167in}}%
\pgfpathlineto{\pgfqpoint{1.449389in}{1.540516in}}%
\pgfpathlineto{\pgfqpoint{1.453022in}{1.543752in}}%
\pgfpathlineto{\pgfqpoint{1.457527in}{1.552377in}}%
\pgfpathlineto{\pgfqpoint{1.463775in}{1.561993in}}%
\pgfpathlineto{\pgfqpoint{1.468280in}{1.562982in}}%
\pgfpathlineto{\pgfqpoint{1.475255in}{1.563548in}}%
\pgfpathlineto{\pgfqpoint{1.479905in}{1.568777in}}%
\pgfpathlineto{\pgfqpoint{1.492111in}{1.582932in}}%
\pgfpathlineto{\pgfqpoint{1.502138in}{1.591706in}}%
\pgfpathlineto{\pgfqpoint{1.513472in}{1.601391in}}%
\pgfpathlineto{\pgfqpoint{1.519285in}{1.607347in}}%
\pgfpathlineto{\pgfqpoint{1.531200in}{1.618182in}}%
\pgfpathlineto{\pgfqpoint{1.539629in}{1.621712in}}%
\pgfpathlineto{\pgfqpoint{1.544424in}{1.629247in}}%
\pgfpathlineto{\pgfqpoint{1.553143in}{1.641393in}}%
\pgfpathlineto{\pgfqpoint{1.557647in}{1.642077in}}%
\pgfpathlineto{\pgfqpoint{1.563605in}{1.644041in}}%
\pgfpathlineto{\pgfqpoint{1.567819in}{1.650117in}}%
\pgfpathlineto{\pgfqpoint{1.577410in}{1.664520in}}%
\pgfpathlineto{\pgfqpoint{1.581479in}{1.664751in}}%
\pgfpathlineto{\pgfqpoint{1.588018in}{1.665508in}}%
\pgfpathlineto{\pgfqpoint{1.592522in}{1.671954in}}%
\pgfpathlineto{\pgfqpoint{1.602113in}{1.686920in}}%
\pgfpathlineto{\pgfqpoint{1.605891in}{1.687464in}}%
\pgfpathlineto{\pgfqpoint{1.611704in}{1.687820in}}%
\pgfpathlineto{\pgfqpoint{1.615772in}{1.691837in}}%
\pgfpathlineto{\pgfqpoint{1.628560in}{1.707967in}}%
\pgfpathlineto{\pgfqpoint{1.639458in}{1.713861in}}%
\pgfpathlineto{\pgfqpoint{1.656750in}{1.732791in}}%
\pgfpathlineto{\pgfqpoint{1.664161in}{1.738532in}}%
\pgfpathlineto{\pgfqpoint{1.694241in}{1.763849in}}%
\pgfpathlineto{\pgfqpoint{1.704268in}{1.771240in}}%
\pgfpathlineto{\pgfqpoint{1.718218in}{1.784404in}}%
\pgfpathlineto{\pgfqpoint{1.727518in}{1.790512in}}%
\pgfpathlineto{\pgfqpoint{1.741468in}{1.804580in}}%
\pgfpathlineto{\pgfqpoint{1.752366in}{1.809286in}}%
\pgfpathlineto{\pgfqpoint{1.766897in}{1.822977in}}%
\pgfpathlineto{\pgfqpoint{1.776343in}{1.826799in}}%
\pgfpathlineto{\pgfqpoint{1.793780in}{1.841608in}}%
\pgfpathlineto{\pgfqpoint{1.801191in}{1.844918in}}%
\pgfpathlineto{\pgfqpoint{1.819210in}{1.858099in}}%
\pgfpathlineto{\pgfqpoint{1.826621in}{1.862329in}}%
\pgfpathlineto{\pgfqpoint{1.846238in}{1.876437in}}%
\pgfpathlineto{\pgfqpoint{1.930955in}{1.925897in}}%
\pgfpathlineto{\pgfqpoint{1.938511in}{1.928639in}}%
\pgfpathlineto{\pgfqpoint{1.955658in}{1.938229in}}%
\pgfpathlineto{\pgfqpoint{1.963650in}{1.940847in}}%
\pgfpathlineto{\pgfqpoint{1.980507in}{1.949509in}}%
\pgfpathlineto{\pgfqpoint{1.989516in}{1.952544in}}%
\pgfpathlineto{\pgfqpoint{2.003466in}{1.959033in}}%
\pgfpathlineto{\pgfqpoint{2.016399in}{1.963398in}}%
\pgfpathlineto{\pgfqpoint{2.034854in}{1.970700in}}%
\pgfpathlineto{\pgfqpoint{2.067258in}{1.981081in}}%
\pgfpathlineto{\pgfqpoint{2.077866in}{1.984039in}}%
\pgfpathlineto{\pgfqpoint{2.090218in}{1.988320in}}%
\pgfpathlineto{\pgfqpoint{2.102714in}{1.990288in}}%
\pgfpathlineto{\pgfqpoint{2.114630in}{1.994088in}}%
\pgfpathlineto{\pgfqpoint{2.126982in}{1.995446in}}%
\pgfpathlineto{\pgfqpoint{2.139333in}{1.999328in}}%
\pgfpathlineto{\pgfqpoint{2.153283in}{2.000802in}}%
\pgfpathlineto{\pgfqpoint{2.164182in}{2.003256in}}%
\pgfpathlineto{\pgfqpoint{2.178713in}{2.004920in}}%
\pgfpathlineto{\pgfqpoint{2.190774in}{2.007013in}}%
\pgfpathlineto{\pgfqpoint{2.219255in}{2.010367in}}%
\pgfpathlineto{\pgfqpoint{2.234077in}{2.010944in}}%
\pgfpathlineto{\pgfqpoint{2.266918in}{2.013173in}}%
\pgfpathlineto{\pgfqpoint{2.276218in}{2.014143in}}%
\pgfpathlineto{\pgfqpoint{2.290604in}{2.013586in}}%
\pgfpathlineto{\pgfqpoint{2.300630in}{2.014360in}}%
\pgfpathlineto{\pgfqpoint{2.314725in}{2.013096in}}%
\pgfpathlineto{\pgfqpoint{2.326060in}{2.013825in}}%
\pgfpathlineto{\pgfqpoint{2.339719in}{2.012747in}}%
\pgfpathlineto{\pgfqpoint{2.351780in}{2.012927in}}%
\pgfpathlineto{\pgfqpoint{2.367183in}{2.011758in}}%
\pgfpathlineto{\pgfqpoint{2.385057in}{2.010438in}}%
\pgfpathlineto{\pgfqpoint{2.446669in}{2.005175in}}%
\pgfpathlineto{\pgfqpoint{2.455679in}{2.005779in}}%
\pgfpathlineto{\pgfqpoint{2.463380in}{2.004887in}}%
\pgfpathlineto{\pgfqpoint{2.475732in}{2.002486in}}%
\pgfpathlineto{\pgfqpoint{2.488083in}{2.002517in}}%
\pgfpathlineto{\pgfqpoint{2.500580in}{2.000456in}}%
\pgfpathlineto{\pgfqpoint{2.513077in}{2.000460in}}%
\pgfpathlineto{\pgfqpoint{2.526446in}{1.998557in}}%
\pgfpathlineto{\pgfqpoint{2.539814in}{1.997771in}}%
\pgfpathlineto{\pgfqpoint{2.555363in}{1.996559in}}%
\pgfpathlineto{\pgfqpoint{2.605786in}{1.992541in}}%
\pgfpathlineto{\pgfqpoint{2.614069in}{1.992403in}}%
\pgfpathlineto{\pgfqpoint{2.624968in}{1.992519in}}%
\pgfpathlineto{\pgfqpoint{2.637755in}{1.990934in}}%
\pgfpathlineto{\pgfqpoint{2.649816in}{1.991595in}}%
\pgfpathlineto{\pgfqpoint{2.662894in}{1.989701in}}%
\pgfpathlineto{\pgfqpoint{2.675246in}{1.990286in}}%
\pgfpathlineto{\pgfqpoint{2.686435in}{1.989288in}}%
\pgfpathlineto{\pgfqpoint{2.701111in}{1.990327in}}%
\pgfpathlineto{\pgfqpoint{2.713608in}{1.989762in}}%
\pgfpathlineto{\pgfqpoint{2.729011in}{1.989567in}}%
\pgfpathlineto{\pgfqpoint{2.751099in}{1.989908in}}%
\pgfpathlineto{\pgfqpoint{2.768972in}{1.990191in}}%
\pgfpathlineto{\pgfqpoint{2.779289in}{1.991499in}}%
\pgfpathlineto{\pgfqpoint{2.788154in}{1.991422in}}%
\pgfpathlineto{\pgfqpoint{2.799343in}{1.990938in}}%
\pgfpathlineto{\pgfqpoint{2.812130in}{1.992490in}}%
\pgfpathlineto{\pgfqpoint{2.824046in}{1.991998in}}%
\pgfpathlineto{\pgfqpoint{2.836543in}{1.993953in}}%
\pgfpathlineto{\pgfqpoint{2.849330in}{1.993289in}}%
\pgfpathlineto{\pgfqpoint{2.860810in}{1.994688in}}%
\pgfpathlineto{\pgfqpoint{2.875341in}{1.994046in}}%
\pgfpathlineto{\pgfqpoint{2.888274in}{1.994795in}}%
\pgfpathlineto{\pgfqpoint{2.901207in}{1.995400in}}%
\pgfpathlineto{\pgfqpoint{2.922277in}{1.996760in}}%
\pgfpathlineto{\pgfqpoint{2.978949in}{1.996439in}}%
\pgfpathlineto{\pgfqpoint{2.986214in}{1.996936in}}%
\pgfpathlineto{\pgfqpoint{2.998275in}{1.998527in}}%
\pgfpathlineto{\pgfqpoint{3.004814in}{1.997076in}}%
\pgfpathlineto{\pgfqpoint{3.004814in}{1.997076in}}%
\pgfusepath{stroke}%
\end{pgfscope}%
\begin{pgfscope}%
\pgfpathrectangle{\pgfqpoint{0.679669in}{0.526079in}}{\pgfqpoint{2.325000in}{1.661000in}} %
\pgfusepath{clip}%
\pgfsetrectcap%
\pgfsetroundjoin%
\pgfsetlinewidth{1.003750pt}%
\definecolor{currentstroke}{rgb}{0.501961,0.000000,0.501961}%
\pgfsetstrokecolor{currentstroke}%
\pgfsetdash{}{0pt}%
\pgfpathmoveto{\pgfqpoint{0.680081in}{0.512191in}}%
\pgfpathlineto{\pgfqpoint{0.681994in}{0.794329in}}%
\pgfpathlineto{\pgfqpoint{0.685772in}{0.920868in}}%
\pgfpathlineto{\pgfqpoint{0.687225in}{0.929334in}}%
\pgfpathlineto{\pgfqpoint{0.687952in}{0.926182in}}%
\pgfpathlineto{\pgfqpoint{0.689405in}{0.907986in}}%
\pgfpathlineto{\pgfqpoint{0.693764in}{0.791547in}}%
\pgfpathlineto{\pgfqpoint{0.694782in}{0.821138in}}%
\pgfpathlineto{\pgfqpoint{0.699432in}{0.889794in}}%
\pgfpathlineto{\pgfqpoint{0.699577in}{0.889762in}}%
\pgfpathlineto{\pgfqpoint{0.701466in}{0.885398in}}%
\pgfpathlineto{\pgfqpoint{0.703500in}{0.869764in}}%
\pgfpathlineto{\pgfqpoint{0.703791in}{0.871933in}}%
\pgfpathlineto{\pgfqpoint{0.707133in}{0.896003in}}%
\pgfpathlineto{\pgfqpoint{0.707860in}{0.893858in}}%
\pgfpathlineto{\pgfqpoint{0.708296in}{0.896233in}}%
\pgfpathlineto{\pgfqpoint{0.709168in}{0.901162in}}%
\pgfpathlineto{\pgfqpoint{0.709749in}{0.896917in}}%
\pgfpathlineto{\pgfqpoint{0.713091in}{0.836116in}}%
\pgfpathlineto{\pgfqpoint{0.714544in}{0.789586in}}%
\pgfpathlineto{\pgfqpoint{0.715561in}{0.792161in}}%
\pgfpathlineto{\pgfqpoint{0.717160in}{0.778504in}}%
\pgfpathlineto{\pgfqpoint{0.717450in}{0.780415in}}%
\pgfpathlineto{\pgfqpoint{0.719485in}{0.857725in}}%
\pgfpathlineto{\pgfqpoint{0.722827in}{0.935890in}}%
\pgfpathlineto{\pgfqpoint{0.723263in}{0.938244in}}%
\pgfpathlineto{\pgfqpoint{0.723844in}{0.932486in}}%
\pgfpathlineto{\pgfqpoint{0.727041in}{0.846096in}}%
\pgfpathlineto{\pgfqpoint{0.730674in}{0.697592in}}%
\pgfpathlineto{\pgfqpoint{0.730819in}{0.700006in}}%
\pgfpathlineto{\pgfqpoint{0.737504in}{0.930144in}}%
\pgfpathlineto{\pgfqpoint{0.737939in}{0.928136in}}%
\pgfpathlineto{\pgfqpoint{0.740991in}{0.892149in}}%
\pgfpathlineto{\pgfqpoint{0.742444in}{0.861388in}}%
\pgfpathlineto{\pgfqpoint{0.743171in}{0.865795in}}%
\pgfpathlineto{\pgfqpoint{0.744624in}{0.852924in}}%
\pgfpathlineto{\pgfqpoint{0.745350in}{0.861316in}}%
\pgfpathlineto{\pgfqpoint{0.746513in}{0.874329in}}%
\pgfpathlineto{\pgfqpoint{0.747239in}{0.869551in}}%
\pgfpathlineto{\pgfqpoint{0.747966in}{0.864542in}}%
\pgfpathlineto{\pgfqpoint{0.748693in}{0.868392in}}%
\pgfpathlineto{\pgfqpoint{0.750000in}{0.872430in}}%
\pgfpathlineto{\pgfqpoint{0.750291in}{0.869204in}}%
\pgfpathlineto{\pgfqpoint{0.752616in}{0.841760in}}%
\pgfpathlineto{\pgfqpoint{0.752907in}{0.843125in}}%
\pgfpathlineto{\pgfqpoint{0.755668in}{0.874698in}}%
\pgfpathlineto{\pgfqpoint{0.757557in}{0.907067in}}%
\pgfpathlineto{\pgfqpoint{0.758574in}{0.900085in}}%
\pgfpathlineto{\pgfqpoint{0.758719in}{0.899817in}}%
\pgfpathlineto{\pgfqpoint{0.759155in}{0.901400in}}%
\pgfpathlineto{\pgfqpoint{0.760027in}{0.906959in}}%
\pgfpathlineto{\pgfqpoint{0.760754in}{0.902139in}}%
\pgfpathlineto{\pgfqpoint{0.763224in}{0.840763in}}%
\pgfpathlineto{\pgfqpoint{0.765985in}{0.674733in}}%
\pgfpathlineto{\pgfqpoint{0.767002in}{0.722398in}}%
\pgfpathlineto{\pgfqpoint{0.771216in}{0.925598in}}%
\pgfpathlineto{\pgfqpoint{0.773686in}{0.952144in}}%
\pgfpathlineto{\pgfqpoint{0.773832in}{0.952443in}}%
\pgfpathlineto{\pgfqpoint{0.774122in}{0.950758in}}%
\pgfpathlineto{\pgfqpoint{0.775721in}{0.921604in}}%
\pgfpathlineto{\pgfqpoint{0.781824in}{0.775056in}}%
\pgfpathlineto{\pgfqpoint{0.782114in}{0.775961in}}%
\pgfpathlineto{\pgfqpoint{0.782986in}{0.799909in}}%
\pgfpathlineto{\pgfqpoint{0.786038in}{0.886663in}}%
\pgfpathlineto{\pgfqpoint{0.786474in}{0.885138in}}%
\pgfpathlineto{\pgfqpoint{0.786619in}{0.884995in}}%
\pgfpathlineto{\pgfqpoint{0.786910in}{0.886119in}}%
\pgfpathlineto{\pgfqpoint{0.788072in}{0.897471in}}%
\pgfpathlineto{\pgfqpoint{0.789235in}{0.890002in}}%
\pgfpathlineto{\pgfqpoint{0.789961in}{0.878911in}}%
\pgfpathlineto{\pgfqpoint{0.790979in}{0.885353in}}%
\pgfpathlineto{\pgfqpoint{0.791269in}{0.887548in}}%
\pgfpathlineto{\pgfqpoint{0.792286in}{0.884804in}}%
\pgfpathlineto{\pgfqpoint{0.792868in}{0.880494in}}%
\pgfpathlineto{\pgfqpoint{0.793304in}{0.885591in}}%
\pgfpathlineto{\pgfqpoint{0.796355in}{0.912835in}}%
\pgfpathlineto{\pgfqpoint{0.796500in}{0.913136in}}%
\pgfpathlineto{\pgfqpoint{0.797082in}{0.910951in}}%
\pgfpathlineto{\pgfqpoint{0.797663in}{0.911952in}}%
\pgfpathlineto{\pgfqpoint{0.797808in}{0.911044in}}%
\pgfpathlineto{\pgfqpoint{0.800569in}{0.859361in}}%
\pgfpathlineto{\pgfqpoint{0.803911in}{0.794731in}}%
\pgfpathlineto{\pgfqpoint{0.804202in}{0.800610in}}%
\pgfpathlineto{\pgfqpoint{0.808997in}{0.923857in}}%
\pgfpathlineto{\pgfqpoint{0.809288in}{0.923609in}}%
\pgfpathlineto{\pgfqpoint{0.811758in}{0.917398in}}%
\pgfpathlineto{\pgfqpoint{0.814229in}{0.834488in}}%
\pgfpathlineto{\pgfqpoint{0.815391in}{0.757847in}}%
\pgfpathlineto{\pgfqpoint{0.816699in}{0.700096in}}%
\pgfpathlineto{\pgfqpoint{0.817425in}{0.722768in}}%
\pgfpathlineto{\pgfqpoint{0.824982in}{0.944679in}}%
\pgfpathlineto{\pgfqpoint{0.825272in}{0.943969in}}%
\pgfpathlineto{\pgfqpoint{0.826871in}{0.919192in}}%
\pgfpathlineto{\pgfqpoint{0.830358in}{0.828980in}}%
\pgfpathlineto{\pgfqpoint{0.831811in}{0.804767in}}%
\pgfpathlineto{\pgfqpoint{0.832247in}{0.810246in}}%
\pgfpathlineto{\pgfqpoint{0.832393in}{0.810696in}}%
\pgfpathlineto{\pgfqpoint{0.832538in}{0.808879in}}%
\pgfpathlineto{\pgfqpoint{0.832829in}{0.804531in}}%
\pgfpathlineto{\pgfqpoint{0.833555in}{0.816818in}}%
\pgfpathlineto{\pgfqpoint{0.833991in}{0.810204in}}%
\pgfpathlineto{\pgfqpoint{0.834572in}{0.818188in}}%
\pgfpathlineto{\pgfqpoint{0.836607in}{0.853434in}}%
\pgfpathlineto{\pgfqpoint{0.837479in}{0.837446in}}%
\pgfpathlineto{\pgfqpoint{0.838205in}{0.828792in}}%
\pgfpathlineto{\pgfqpoint{0.838932in}{0.834318in}}%
\pgfpathlineto{\pgfqpoint{0.843146in}{0.916991in}}%
\pgfpathlineto{\pgfqpoint{0.845907in}{0.945333in}}%
\pgfpathlineto{\pgfqpoint{0.846779in}{0.943692in}}%
\pgfpathlineto{\pgfqpoint{0.849394in}{0.905432in}}%
\pgfpathlineto{\pgfqpoint{0.851138in}{0.857678in}}%
\pgfpathlineto{\pgfqpoint{0.853318in}{0.799812in}}%
\pgfpathlineto{\pgfqpoint{0.853463in}{0.801201in}}%
\pgfpathlineto{\pgfqpoint{0.853754in}{0.803834in}}%
\pgfpathlineto{\pgfqpoint{0.854044in}{0.795399in}}%
\pgfpathlineto{\pgfqpoint{0.854480in}{0.779970in}}%
\pgfpathlineto{\pgfqpoint{0.855207in}{0.813993in}}%
\pgfpathlineto{\pgfqpoint{0.860002in}{0.953138in}}%
\pgfpathlineto{\pgfqpoint{0.860729in}{0.947705in}}%
\pgfpathlineto{\pgfqpoint{0.862182in}{0.932830in}}%
\pgfpathlineto{\pgfqpoint{0.868721in}{0.784667in}}%
\pgfpathlineto{\pgfqpoint{0.869011in}{0.798216in}}%
\pgfpathlineto{\pgfqpoint{0.871336in}{0.882980in}}%
\pgfpathlineto{\pgfqpoint{0.871627in}{0.880468in}}%
\pgfpathlineto{\pgfqpoint{0.871772in}{0.879069in}}%
\pgfpathlineto{\pgfqpoint{0.872208in}{0.886321in}}%
\pgfpathlineto{\pgfqpoint{0.874388in}{0.923851in}}%
\pgfpathlineto{\pgfqpoint{0.874824in}{0.919089in}}%
\pgfpathlineto{\pgfqpoint{0.876132in}{0.919810in}}%
\pgfpathlineto{\pgfqpoint{0.877149in}{0.909562in}}%
\pgfpathlineto{\pgfqpoint{0.877294in}{0.909886in}}%
\pgfpathlineto{\pgfqpoint{0.877585in}{0.907876in}}%
\pgfpathlineto{\pgfqpoint{0.879764in}{0.877324in}}%
\pgfpathlineto{\pgfqpoint{0.880200in}{0.884870in}}%
\pgfpathlineto{\pgfqpoint{0.880491in}{0.887314in}}%
\pgfpathlineto{\pgfqpoint{0.881072in}{0.880935in}}%
\pgfpathlineto{\pgfqpoint{0.881508in}{0.883633in}}%
\pgfpathlineto{\pgfqpoint{0.882525in}{0.877273in}}%
\pgfpathlineto{\pgfqpoint{0.883107in}{0.863872in}}%
\pgfpathlineto{\pgfqpoint{0.884124in}{0.875184in}}%
\pgfpathlineto{\pgfqpoint{0.884414in}{0.875683in}}%
\pgfpathlineto{\pgfqpoint{0.884705in}{0.873767in}}%
\pgfpathlineto{\pgfqpoint{0.889355in}{0.800778in}}%
\pgfpathlineto{\pgfqpoint{0.889646in}{0.811458in}}%
\pgfpathlineto{\pgfqpoint{0.892988in}{0.909974in}}%
\pgfpathlineto{\pgfqpoint{0.896621in}{0.947260in}}%
\pgfpathlineto{\pgfqpoint{0.897057in}{0.944021in}}%
\pgfpathlineto{\pgfqpoint{0.897783in}{0.947189in}}%
\pgfpathlineto{\pgfqpoint{0.899236in}{0.923144in}}%
\pgfpathlineto{\pgfqpoint{0.903450in}{0.778471in}}%
\pgfpathlineto{\pgfqpoint{0.904032in}{0.811116in}}%
\pgfpathlineto{\pgfqpoint{0.907519in}{0.929643in}}%
\pgfpathlineto{\pgfqpoint{0.910571in}{0.976064in}}%
\pgfpathlineto{\pgfqpoint{0.911297in}{0.974951in}}%
\pgfpathlineto{\pgfqpoint{0.911733in}{0.973522in}}%
\pgfpathlineto{\pgfqpoint{0.913913in}{0.942090in}}%
\pgfpathlineto{\pgfqpoint{0.916819in}{0.861129in}}%
\pgfpathlineto{\pgfqpoint{0.919144in}{0.807669in}}%
\pgfpathlineto{\pgfqpoint{0.919289in}{0.809563in}}%
\pgfpathlineto{\pgfqpoint{0.919580in}{0.811978in}}%
\pgfpathlineto{\pgfqpoint{0.920016in}{0.797278in}}%
\pgfpathlineto{\pgfqpoint{0.920161in}{0.794021in}}%
\pgfpathlineto{\pgfqpoint{0.920743in}{0.808814in}}%
\pgfpathlineto{\pgfqpoint{0.923213in}{0.855608in}}%
\pgfpathlineto{\pgfqpoint{0.923649in}{0.855459in}}%
\pgfpathlineto{\pgfqpoint{0.924230in}{0.865112in}}%
\pgfpathlineto{\pgfqpoint{0.926555in}{0.891379in}}%
\pgfpathlineto{\pgfqpoint{0.927572in}{0.898980in}}%
\pgfpathlineto{\pgfqpoint{0.931350in}{0.946936in}}%
\pgfpathlineto{\pgfqpoint{0.932658in}{0.963696in}}%
\pgfpathlineto{\pgfqpoint{0.933966in}{0.954709in}}%
\pgfpathlineto{\pgfqpoint{0.938616in}{0.877730in}}%
\pgfpathlineto{\pgfqpoint{0.941232in}{0.810921in}}%
\pgfpathlineto{\pgfqpoint{0.941958in}{0.846854in}}%
\pgfpathlineto{\pgfqpoint{0.942394in}{0.843588in}}%
\pgfpathlineto{\pgfqpoint{0.942830in}{0.854982in}}%
\pgfpathlineto{\pgfqpoint{0.945300in}{0.907805in}}%
\pgfpathlineto{\pgfqpoint{0.947044in}{0.930094in}}%
\pgfpathlineto{\pgfqpoint{0.948352in}{0.922642in}}%
\pgfpathlineto{\pgfqpoint{0.951113in}{0.863243in}}%
\pgfpathlineto{\pgfqpoint{0.952566in}{0.835834in}}%
\pgfpathlineto{\pgfqpoint{0.953002in}{0.843373in}}%
\pgfpathlineto{\pgfqpoint{0.953147in}{0.844839in}}%
\pgfpathlineto{\pgfqpoint{0.953438in}{0.837606in}}%
\pgfpathlineto{\pgfqpoint{0.953729in}{0.827165in}}%
\pgfpathlineto{\pgfqpoint{0.954600in}{0.848522in}}%
\pgfpathlineto{\pgfqpoint{0.955036in}{0.854844in}}%
\pgfpathlineto{\pgfqpoint{0.959686in}{0.978486in}}%
\pgfpathlineto{\pgfqpoint{0.961285in}{0.989056in}}%
\pgfpathlineto{\pgfqpoint{0.962011in}{0.984735in}}%
\pgfpathlineto{\pgfqpoint{0.962447in}{0.983916in}}%
\pgfpathlineto{\pgfqpoint{0.964772in}{0.939716in}}%
\pgfpathlineto{\pgfqpoint{0.969132in}{0.816000in}}%
\pgfpathlineto{\pgfqpoint{0.969568in}{0.834842in}}%
\pgfpathlineto{\pgfqpoint{0.969713in}{0.838119in}}%
\pgfpathlineto{\pgfqpoint{0.970439in}{0.822027in}}%
\pgfpathlineto{\pgfqpoint{0.972183in}{0.802524in}}%
\pgfpathlineto{\pgfqpoint{0.972329in}{0.802545in}}%
\pgfpathlineto{\pgfqpoint{0.972910in}{0.808681in}}%
\pgfpathlineto{\pgfqpoint{0.973636in}{0.804870in}}%
\pgfpathlineto{\pgfqpoint{0.974072in}{0.794482in}}%
\pgfpathlineto{\pgfqpoint{0.974799in}{0.805256in}}%
\pgfpathlineto{\pgfqpoint{0.975235in}{0.800943in}}%
\pgfpathlineto{\pgfqpoint{0.976252in}{0.840507in}}%
\pgfpathlineto{\pgfqpoint{0.981193in}{0.972765in}}%
\pgfpathlineto{\pgfqpoint{0.981629in}{0.971426in}}%
\pgfpathlineto{\pgfqpoint{0.982355in}{0.982555in}}%
\pgfpathlineto{\pgfqpoint{0.983954in}{0.994719in}}%
\pgfpathlineto{\pgfqpoint{0.984244in}{0.992064in}}%
\pgfpathlineto{\pgfqpoint{0.988749in}{0.903341in}}%
\pgfpathlineto{\pgfqpoint{0.991655in}{0.819757in}}%
\pgfpathlineto{\pgfqpoint{0.992818in}{0.820797in}}%
\pgfpathlineto{\pgfqpoint{0.998049in}{0.921427in}}%
\pgfpathlineto{\pgfqpoint{0.998339in}{0.919158in}}%
\pgfpathlineto{\pgfqpoint{0.999502in}{0.900261in}}%
\pgfpathlineto{\pgfqpoint{1.000519in}{0.903354in}}%
\pgfpathlineto{\pgfqpoint{1.000664in}{0.903312in}}%
\pgfpathlineto{\pgfqpoint{1.000810in}{0.904233in}}%
\pgfpathlineto{\pgfqpoint{1.001100in}{0.905530in}}%
\pgfpathlineto{\pgfqpoint{1.001391in}{0.899693in}}%
\pgfpathlineto{\pgfqpoint{1.002699in}{0.879522in}}%
\pgfpathlineto{\pgfqpoint{1.003135in}{0.887402in}}%
\pgfpathlineto{\pgfqpoint{1.009383in}{0.980781in}}%
\pgfpathlineto{\pgfqpoint{1.009964in}{0.979594in}}%
\pgfpathlineto{\pgfqpoint{1.011272in}{0.987567in}}%
\pgfpathlineto{\pgfqpoint{1.012144in}{0.985081in}}%
\pgfpathlineto{\pgfqpoint{1.012580in}{0.981666in}}%
\pgfpathlineto{\pgfqpoint{1.016358in}{0.927188in}}%
\pgfpathlineto{\pgfqpoint{1.016504in}{0.927207in}}%
\pgfpathlineto{\pgfqpoint{1.016939in}{0.926538in}}%
\pgfpathlineto{\pgfqpoint{1.017230in}{0.928082in}}%
\pgfpathlineto{\pgfqpoint{1.017521in}{0.929874in}}%
\pgfpathlineto{\pgfqpoint{1.018102in}{0.922517in}}%
\pgfpathlineto{\pgfqpoint{1.018247in}{0.922287in}}%
\pgfpathlineto{\pgfqpoint{1.018538in}{0.924586in}}%
\pgfpathlineto{\pgfqpoint{1.018683in}{0.924778in}}%
\pgfpathlineto{\pgfqpoint{1.018829in}{0.923216in}}%
\pgfpathlineto{\pgfqpoint{1.021154in}{0.899804in}}%
\pgfpathlineto{\pgfqpoint{1.021299in}{0.901439in}}%
\pgfpathlineto{\pgfqpoint{1.021735in}{0.909568in}}%
\pgfpathlineto{\pgfqpoint{1.022316in}{0.891328in}}%
\pgfpathlineto{\pgfqpoint{1.025513in}{0.835327in}}%
\pgfpathlineto{\pgfqpoint{1.026094in}{0.843203in}}%
\pgfpathlineto{\pgfqpoint{1.027402in}{0.861783in}}%
\pgfpathlineto{\pgfqpoint{1.031471in}{0.975357in}}%
\pgfpathlineto{\pgfqpoint{1.034377in}{1.003960in}}%
\pgfpathlineto{\pgfqpoint{1.034522in}{1.003268in}}%
\pgfpathlineto{\pgfqpoint{1.036557in}{0.977743in}}%
\pgfpathlineto{\pgfqpoint{1.041643in}{0.860677in}}%
\pgfpathlineto{\pgfqpoint{1.042950in}{0.876829in}}%
\pgfpathlineto{\pgfqpoint{1.045566in}{0.942504in}}%
\pgfpathlineto{\pgfqpoint{1.046002in}{0.940044in}}%
\pgfpathlineto{\pgfqpoint{1.046729in}{0.950751in}}%
\pgfpathlineto{\pgfqpoint{1.047600in}{0.941567in}}%
\pgfpathlineto{\pgfqpoint{1.047891in}{0.943149in}}%
\pgfpathlineto{\pgfqpoint{1.048472in}{0.952798in}}%
\pgfpathlineto{\pgfqpoint{1.049489in}{0.945164in}}%
\pgfpathlineto{\pgfqpoint{1.049780in}{0.944044in}}%
\pgfpathlineto{\pgfqpoint{1.053994in}{0.906942in}}%
\pgfpathlineto{\pgfqpoint{1.060824in}{0.959876in}}%
\pgfpathlineto{\pgfqpoint{1.061405in}{0.951294in}}%
\pgfpathlineto{\pgfqpoint{1.061550in}{0.950970in}}%
\pgfpathlineto{\pgfqpoint{1.061841in}{0.953913in}}%
\pgfpathlineto{\pgfqpoint{1.062713in}{0.960352in}}%
\pgfpathlineto{\pgfqpoint{1.063294in}{0.955229in}}%
\pgfpathlineto{\pgfqpoint{1.064602in}{0.946886in}}%
\pgfpathlineto{\pgfqpoint{1.065474in}{0.949298in}}%
\pgfpathlineto{\pgfqpoint{1.068380in}{0.970188in}}%
\pgfpathlineto{\pgfqpoint{1.070269in}{0.974596in}}%
\pgfpathlineto{\pgfqpoint{1.070705in}{0.972167in}}%
\pgfpathlineto{\pgfqpoint{1.072739in}{0.957723in}}%
\pgfpathlineto{\pgfqpoint{1.073030in}{0.958685in}}%
\pgfpathlineto{\pgfqpoint{1.073321in}{0.960600in}}%
\pgfpathlineto{\pgfqpoint{1.073757in}{0.956323in}}%
\pgfpathlineto{\pgfqpoint{1.075646in}{0.913360in}}%
\pgfpathlineto{\pgfqpoint{1.076808in}{0.915169in}}%
\pgfpathlineto{\pgfqpoint{1.076954in}{0.915221in}}%
\pgfpathlineto{\pgfqpoint{1.077099in}{0.914291in}}%
\pgfpathlineto{\pgfqpoint{1.077535in}{0.909367in}}%
\pgfpathlineto{\pgfqpoint{1.078552in}{0.912102in}}%
\pgfpathlineto{\pgfqpoint{1.084510in}{0.986664in}}%
\pgfpathlineto{\pgfqpoint{1.085382in}{0.979760in}}%
\pgfpathlineto{\pgfqpoint{1.086108in}{0.978386in}}%
\pgfpathlineto{\pgfqpoint{1.089741in}{0.931316in}}%
\pgfpathlineto{\pgfqpoint{1.090177in}{0.936384in}}%
\pgfpathlineto{\pgfqpoint{1.095554in}{0.997342in}}%
\pgfpathlineto{\pgfqpoint{1.096280in}{1.000813in}}%
\pgfpathlineto{\pgfqpoint{1.097443in}{1.008549in}}%
\pgfpathlineto{\pgfqpoint{1.097879in}{1.004509in}}%
\pgfpathlineto{\pgfqpoint{1.105289in}{0.910190in}}%
\pgfpathlineto{\pgfqpoint{1.106743in}{0.895845in}}%
\pgfpathlineto{\pgfqpoint{1.107033in}{0.899918in}}%
\pgfpathlineto{\pgfqpoint{1.107905in}{0.910060in}}%
\pgfpathlineto{\pgfqpoint{1.108486in}{0.903072in}}%
\pgfpathlineto{\pgfqpoint{1.110230in}{0.879501in}}%
\pgfpathlineto{\pgfqpoint{1.110666in}{0.887015in}}%
\pgfpathlineto{\pgfqpoint{1.112846in}{0.903359in}}%
\pgfpathlineto{\pgfqpoint{1.113282in}{0.905694in}}%
\pgfpathlineto{\pgfqpoint{1.119385in}{1.035324in}}%
\pgfpathlineto{\pgfqpoint{1.119966in}{1.037611in}}%
\pgfpathlineto{\pgfqpoint{1.121419in}{1.042655in}}%
\pgfpathlineto{\pgfqpoint{1.121855in}{1.041269in}}%
\pgfpathlineto{\pgfqpoint{1.122582in}{1.036419in}}%
\pgfpathlineto{\pgfqpoint{1.129557in}{0.896429in}}%
\pgfpathlineto{\pgfqpoint{1.131300in}{0.913802in}}%
\pgfpathlineto{\pgfqpoint{1.136386in}{0.953902in}}%
\pgfpathlineto{\pgfqpoint{1.136822in}{0.946736in}}%
\pgfpathlineto{\pgfqpoint{1.137113in}{0.942460in}}%
\pgfpathlineto{\pgfqpoint{1.138130in}{0.947873in}}%
\pgfpathlineto{\pgfqpoint{1.141472in}{0.990810in}}%
\pgfpathlineto{\pgfqpoint{1.144524in}{1.029210in}}%
\pgfpathlineto{\pgfqpoint{1.147575in}{1.046340in}}%
\pgfpathlineto{\pgfqpoint{1.147721in}{1.046271in}}%
\pgfpathlineto{\pgfqpoint{1.149319in}{1.035551in}}%
\pgfpathlineto{\pgfqpoint{1.154260in}{0.955510in}}%
\pgfpathlineto{\pgfqpoint{1.157602in}{0.885196in}}%
\pgfpathlineto{\pgfqpoint{1.158038in}{0.892242in}}%
\pgfpathlineto{\pgfqpoint{1.158183in}{0.892143in}}%
\pgfpathlineto{\pgfqpoint{1.160799in}{0.845341in}}%
\pgfpathlineto{\pgfqpoint{1.161525in}{0.859888in}}%
\pgfpathlineto{\pgfqpoint{1.162543in}{0.886115in}}%
\pgfpathlineto{\pgfqpoint{1.167483in}{1.037373in}}%
\pgfpathlineto{\pgfqpoint{1.170971in}{1.081250in}}%
\pgfpathlineto{\pgfqpoint{1.171116in}{1.081073in}}%
\pgfpathlineto{\pgfqpoint{1.173441in}{1.069263in}}%
\pgfpathlineto{\pgfqpoint{1.177364in}{0.982355in}}%
\pgfpathlineto{\pgfqpoint{1.180852in}{0.913943in}}%
\pgfpathlineto{\pgfqpoint{1.181288in}{0.906988in}}%
\pgfpathlineto{\pgfqpoint{1.182014in}{0.918344in}}%
\pgfpathlineto{\pgfqpoint{1.182450in}{0.911728in}}%
\pgfpathlineto{\pgfqpoint{1.182596in}{0.911629in}}%
\pgfpathlineto{\pgfqpoint{1.188263in}{0.985872in}}%
\pgfpathlineto{\pgfqpoint{1.190297in}{1.008457in}}%
\pgfpathlineto{\pgfqpoint{1.194657in}{1.053346in}}%
\pgfpathlineto{\pgfqpoint{1.195383in}{1.052955in}}%
\pgfpathlineto{\pgfqpoint{1.197272in}{1.059241in}}%
\pgfpathlineto{\pgfqpoint{1.197563in}{1.058768in}}%
\pgfpathlineto{\pgfqpoint{1.199161in}{1.049498in}}%
\pgfpathlineto{\pgfqpoint{1.201050in}{1.032249in}}%
\pgfpathlineto{\pgfqpoint{1.203230in}{1.011330in}}%
\pgfpathlineto{\pgfqpoint{1.203521in}{1.013585in}}%
\pgfpathlineto{\pgfqpoint{1.203811in}{1.016054in}}%
\pgfpathlineto{\pgfqpoint{1.204538in}{1.008557in}}%
\pgfpathlineto{\pgfqpoint{1.212821in}{0.967235in}}%
\pgfpathlineto{\pgfqpoint{1.212966in}{0.969422in}}%
\pgfpathlineto{\pgfqpoint{1.216744in}{1.032715in}}%
\pgfpathlineto{\pgfqpoint{1.217616in}{1.047607in}}%
\pgfpathlineto{\pgfqpoint{1.221104in}{1.087272in}}%
\pgfpathlineto{\pgfqpoint{1.221249in}{1.087592in}}%
\pgfpathlineto{\pgfqpoint{1.221685in}{1.084976in}}%
\pgfpathlineto{\pgfqpoint{1.230258in}{1.010632in}}%
\pgfpathlineto{\pgfqpoint{1.231711in}{1.014682in}}%
\pgfpathlineto{\pgfqpoint{1.232147in}{1.019206in}}%
\pgfpathlineto{\pgfqpoint{1.233455in}{1.018057in}}%
\pgfpathlineto{\pgfqpoint{1.238541in}{1.037046in}}%
\pgfpathlineto{\pgfqpoint{1.238832in}{1.035423in}}%
\pgfpathlineto{\pgfqpoint{1.239413in}{1.032318in}}%
\pgfpathlineto{\pgfqpoint{1.240575in}{1.033216in}}%
\pgfpathlineto{\pgfqpoint{1.240721in}{1.032876in}}%
\pgfpathlineto{\pgfqpoint{1.241011in}{1.035416in}}%
\pgfpathlineto{\pgfqpoint{1.243336in}{1.050049in}}%
\pgfpathlineto{\pgfqpoint{1.244063in}{1.051333in}}%
\pgfpathlineto{\pgfqpoint{1.245080in}{1.055809in}}%
\pgfpathlineto{\pgfqpoint{1.245661in}{1.051341in}}%
\pgfpathlineto{\pgfqpoint{1.248713in}{1.042006in}}%
\pgfpathlineto{\pgfqpoint{1.249149in}{1.044606in}}%
\pgfpathlineto{\pgfqpoint{1.249294in}{1.045237in}}%
\pgfpathlineto{\pgfqpoint{1.249730in}{1.042501in}}%
\pgfpathlineto{\pgfqpoint{1.250166in}{1.039006in}}%
\pgfpathlineto{\pgfqpoint{1.251183in}{1.042748in}}%
\pgfpathlineto{\pgfqpoint{1.258013in}{1.083625in}}%
\pgfpathlineto{\pgfqpoint{1.258304in}{1.082836in}}%
\pgfpathlineto{\pgfqpoint{1.258449in}{1.082594in}}%
\pgfpathlineto{\pgfqpoint{1.258885in}{1.084747in}}%
\pgfpathlineto{\pgfqpoint{1.259030in}{1.085428in}}%
\pgfpathlineto{\pgfqpoint{1.259611in}{1.081716in}}%
\pgfpathlineto{\pgfqpoint{1.263535in}{1.049951in}}%
\pgfpathlineto{\pgfqpoint{1.266005in}{1.032463in}}%
\pgfpathlineto{\pgfqpoint{1.270074in}{1.053845in}}%
\pgfpathlineto{\pgfqpoint{1.270655in}{1.053484in}}%
\pgfpathlineto{\pgfqpoint{1.271382in}{1.059288in}}%
\pgfpathlineto{\pgfqpoint{1.273125in}{1.067647in}}%
\pgfpathlineto{\pgfqpoint{1.273416in}{1.067048in}}%
\pgfpathlineto{\pgfqpoint{1.273852in}{1.065918in}}%
\pgfpathlineto{\pgfqpoint{1.274579in}{1.067707in}}%
\pgfpathlineto{\pgfqpoint{1.274869in}{1.067099in}}%
\pgfpathlineto{\pgfqpoint{1.275160in}{1.067877in}}%
\pgfpathlineto{\pgfqpoint{1.279229in}{1.094020in}}%
\pgfpathlineto{\pgfqpoint{1.284024in}{1.117501in}}%
\pgfpathlineto{\pgfqpoint{1.284460in}{1.114392in}}%
\pgfpathlineto{\pgfqpoint{1.288238in}{1.093445in}}%
\pgfpathlineto{\pgfqpoint{1.290999in}{1.066486in}}%
\pgfpathlineto{\pgfqpoint{1.294196in}{1.046099in}}%
\pgfpathlineto{\pgfqpoint{1.294632in}{1.047613in}}%
\pgfpathlineto{\pgfqpoint{1.295068in}{1.045022in}}%
\pgfpathlineto{\pgfqpoint{1.297393in}{1.034222in}}%
\pgfpathlineto{\pgfqpoint{1.297538in}{1.033866in}}%
\pgfpathlineto{\pgfqpoint{1.297974in}{1.036189in}}%
\pgfpathlineto{\pgfqpoint{1.304368in}{1.121877in}}%
\pgfpathlineto{\pgfqpoint{1.307274in}{1.148148in}}%
\pgfpathlineto{\pgfqpoint{1.307710in}{1.149528in}}%
\pgfpathlineto{\pgfqpoint{1.308291in}{1.147503in}}%
\pgfpathlineto{\pgfqpoint{1.309018in}{1.149044in}}%
\pgfpathlineto{\pgfqpoint{1.311924in}{1.126809in}}%
\pgfpathlineto{\pgfqpoint{1.313522in}{1.101581in}}%
\pgfpathlineto{\pgfqpoint{1.319335in}{1.023833in}}%
\pgfpathlineto{\pgfqpoint{1.319771in}{1.026948in}}%
\pgfpathlineto{\pgfqpoint{1.323839in}{1.067731in}}%
\pgfpathlineto{\pgfqpoint{1.332704in}{1.161865in}}%
\pgfpathlineto{\pgfqpoint{1.334593in}{1.159792in}}%
\pgfpathlineto{\pgfqpoint{1.336336in}{1.147380in}}%
\pgfpathlineto{\pgfqpoint{1.342004in}{1.093079in}}%
\pgfpathlineto{\pgfqpoint{1.344474in}{1.070593in}}%
\pgfpathlineto{\pgfqpoint{1.345055in}{1.074675in}}%
\pgfpathlineto{\pgfqpoint{1.345200in}{1.075034in}}%
\pgfpathlineto{\pgfqpoint{1.345491in}{1.073019in}}%
\pgfpathlineto{\pgfqpoint{1.345927in}{1.069228in}}%
\pgfpathlineto{\pgfqpoint{1.347235in}{1.070093in}}%
\pgfpathlineto{\pgfqpoint{1.347525in}{1.070211in}}%
\pgfpathlineto{\pgfqpoint{1.347671in}{1.070721in}}%
\pgfpathlineto{\pgfqpoint{1.350722in}{1.100761in}}%
\pgfpathlineto{\pgfqpoint{1.357843in}{1.185635in}}%
\pgfpathlineto{\pgfqpoint{1.359005in}{1.185071in}}%
\pgfpathlineto{\pgfqpoint{1.359732in}{1.182238in}}%
\pgfpathlineto{\pgfqpoint{1.363074in}{1.157169in}}%
\pgfpathlineto{\pgfqpoint{1.368014in}{1.100705in}}%
\pgfpathlineto{\pgfqpoint{1.369322in}{1.097905in}}%
\pgfpathlineto{\pgfqpoint{1.370921in}{1.096693in}}%
\pgfpathlineto{\pgfqpoint{1.371066in}{1.096920in}}%
\pgfpathlineto{\pgfqpoint{1.373246in}{1.116454in}}%
\pgfpathlineto{\pgfqpoint{1.382546in}{1.186963in}}%
\pgfpathlineto{\pgfqpoint{1.382836in}{1.187611in}}%
\pgfpathlineto{\pgfqpoint{1.383563in}{1.185263in}}%
\pgfpathlineto{\pgfqpoint{1.386324in}{1.174648in}}%
\pgfpathlineto{\pgfqpoint{1.390538in}{1.152581in}}%
\pgfpathlineto{\pgfqpoint{1.396205in}{1.146751in}}%
\pgfpathlineto{\pgfqpoint{1.396786in}{1.148736in}}%
\pgfpathlineto{\pgfqpoint{1.396932in}{1.149039in}}%
\pgfpathlineto{\pgfqpoint{1.397513in}{1.146919in}}%
\pgfpathlineto{\pgfqpoint{1.399257in}{1.144382in}}%
\pgfpathlineto{\pgfqpoint{1.399402in}{1.144478in}}%
\pgfpathlineto{\pgfqpoint{1.402889in}{1.156019in}}%
\pgfpathlineto{\pgfqpoint{1.406668in}{1.178103in}}%
\pgfpathlineto{\pgfqpoint{1.409574in}{1.185197in}}%
\pgfpathlineto{\pgfqpoint{1.409719in}{1.185142in}}%
\pgfpathlineto{\pgfqpoint{1.412335in}{1.180907in}}%
\pgfpathlineto{\pgfqpoint{1.412771in}{1.181577in}}%
\pgfpathlineto{\pgfqpoint{1.412916in}{1.181723in}}%
\pgfpathlineto{\pgfqpoint{1.413352in}{1.180413in}}%
\pgfpathlineto{\pgfqpoint{1.415532in}{1.176090in}}%
\pgfpathlineto{\pgfqpoint{1.415968in}{1.176994in}}%
\pgfpathlineto{\pgfqpoint{1.416113in}{1.177080in}}%
\pgfpathlineto{\pgfqpoint{1.416549in}{1.175586in}}%
\pgfpathlineto{\pgfqpoint{1.416694in}{1.175297in}}%
\pgfpathlineto{\pgfqpoint{1.417421in}{1.176909in}}%
\pgfpathlineto{\pgfqpoint{1.417566in}{1.176867in}}%
\pgfpathlineto{\pgfqpoint{1.418874in}{1.176800in}}%
\pgfpathlineto{\pgfqpoint{1.419164in}{1.177636in}}%
\pgfpathlineto{\pgfqpoint{1.420763in}{1.179741in}}%
\pgfpathlineto{\pgfqpoint{1.421199in}{1.179254in}}%
\pgfpathlineto{\pgfqpoint{1.422797in}{1.179501in}}%
\pgfpathlineto{\pgfqpoint{1.425413in}{1.184629in}}%
\pgfpathlineto{\pgfqpoint{1.425704in}{1.184393in}}%
\pgfpathlineto{\pgfqpoint{1.425994in}{1.184304in}}%
\pgfpathlineto{\pgfqpoint{1.426721in}{1.185473in}}%
\pgfpathlineto{\pgfqpoint{1.430644in}{1.193917in}}%
\pgfpathlineto{\pgfqpoint{1.430789in}{1.193826in}}%
\pgfpathlineto{\pgfqpoint{1.431371in}{1.195175in}}%
\pgfpathlineto{\pgfqpoint{1.431661in}{1.195695in}}%
\pgfpathlineto{\pgfqpoint{1.432243in}{1.194644in}}%
\pgfpathlineto{\pgfqpoint{1.432679in}{1.194694in}}%
\pgfpathlineto{\pgfqpoint{1.433114in}{1.194008in}}%
\pgfpathlineto{\pgfqpoint{1.434277in}{1.194918in}}%
\pgfpathlineto{\pgfqpoint{1.436021in}{1.197314in}}%
\pgfpathlineto{\pgfqpoint{1.435294in}{1.194478in}}%
\pgfpathlineto{\pgfqpoint{1.436311in}{1.196883in}}%
\pgfpathlineto{\pgfqpoint{1.436602in}{1.196981in}}%
\pgfpathlineto{\pgfqpoint{1.436893in}{1.197815in}}%
\pgfpathlineto{\pgfqpoint{1.440816in}{1.214036in}}%
\pgfpathlineto{\pgfqpoint{1.444304in}{1.229919in}}%
\pgfpathlineto{\pgfqpoint{1.446047in}{1.231801in}}%
\pgfpathlineto{\pgfqpoint{1.446338in}{1.231226in}}%
\pgfpathlineto{\pgfqpoint{1.450116in}{1.216547in}}%
\pgfpathlineto{\pgfqpoint{1.456074in}{1.189533in}}%
\pgfpathlineto{\pgfqpoint{1.458399in}{1.193071in}}%
\pgfpathlineto{\pgfqpoint{1.469588in}{1.252424in}}%
\pgfpathlineto{\pgfqpoint{1.470024in}{1.252096in}}%
\pgfpathlineto{\pgfqpoint{1.472349in}{1.247156in}}%
\pgfpathlineto{\pgfqpoint{1.477144in}{1.218203in}}%
\pgfpathlineto{\pgfqpoint{1.481068in}{1.199255in}}%
\pgfpathlineto{\pgfqpoint{1.484846in}{1.206539in}}%
\pgfpathlineto{\pgfqpoint{1.487607in}{1.227458in}}%
\pgfpathlineto{\pgfqpoint{1.492983in}{1.274352in}}%
\pgfpathlineto{\pgfqpoint{1.495599in}{1.280750in}}%
\pgfpathlineto{\pgfqpoint{1.496325in}{1.280726in}}%
\pgfpathlineto{\pgfqpoint{1.496616in}{1.280037in}}%
\pgfpathlineto{\pgfqpoint{1.499668in}{1.264914in}}%
\pgfpathlineto{\pgfqpoint{1.506643in}{1.219374in}}%
\pgfpathlineto{\pgfqpoint{1.507514in}{1.220915in}}%
\pgfpathlineto{\pgfqpoint{1.507950in}{1.220643in}}%
\pgfpathlineto{\pgfqpoint{1.508241in}{1.221548in}}%
\pgfpathlineto{\pgfqpoint{1.512310in}{1.252101in}}%
\pgfpathlineto{\pgfqpoint{1.518122in}{1.296914in}}%
\pgfpathlineto{\pgfqpoint{1.519866in}{1.299529in}}%
\pgfpathlineto{\pgfqpoint{1.520447in}{1.298925in}}%
\pgfpathlineto{\pgfqpoint{1.522918in}{1.291655in}}%
\pgfpathlineto{\pgfqpoint{1.531636in}{1.246852in}}%
\pgfpathlineto{\pgfqpoint{1.532218in}{1.248851in}}%
\pgfpathlineto{\pgfqpoint{1.532944in}{1.248119in}}%
\pgfpathlineto{\pgfqpoint{1.535560in}{1.258871in}}%
\pgfpathlineto{\pgfqpoint{1.545005in}{1.313007in}}%
\pgfpathlineto{\pgfqpoint{1.546749in}{1.312118in}}%
\pgfpathlineto{\pgfqpoint{1.550236in}{1.301280in}}%
\pgfpathlineto{\pgfqpoint{1.556775in}{1.281972in}}%
\pgfpathlineto{\pgfqpoint{1.556921in}{1.282186in}}%
\pgfpathlineto{\pgfqpoint{1.563314in}{1.305241in}}%
\pgfpathlineto{\pgfqpoint{1.567529in}{1.324405in}}%
\pgfpathlineto{\pgfqpoint{1.570144in}{1.327964in}}%
\pgfpathlineto{\pgfqpoint{1.570289in}{1.327792in}}%
\pgfpathlineto{\pgfqpoint{1.578572in}{1.316196in}}%
\pgfpathlineto{\pgfqpoint{1.578718in}{1.316391in}}%
\pgfpathlineto{\pgfqpoint{1.580461in}{1.316342in}}%
\pgfpathlineto{\pgfqpoint{1.601386in}{1.344486in}}%
\pgfpathlineto{\pgfqpoint{1.605164in}{1.349983in}}%
\pgfpathlineto{\pgfqpoint{1.607780in}{1.351794in}}%
\pgfpathlineto{\pgfqpoint{1.607925in}{1.351668in}}%
\pgfpathlineto{\pgfqpoint{1.614174in}{1.346041in}}%
\pgfpathlineto{\pgfqpoint{1.619550in}{1.345183in}}%
\pgfpathlineto{\pgfqpoint{1.634372in}{1.377852in}}%
\pgfpathlineto{\pgfqpoint{1.637714in}{1.370394in}}%
\pgfpathlineto{\pgfqpoint{1.641638in}{1.360525in}}%
\pgfpathlineto{\pgfqpoint{1.643963in}{1.359761in}}%
\pgfpathlineto{\pgfqpoint{1.645125in}{1.361989in}}%
\pgfpathlineto{\pgfqpoint{1.646724in}{1.367070in}}%
\pgfpathlineto{\pgfqpoint{1.656169in}{1.407729in}}%
\pgfpathlineto{\pgfqpoint{1.658639in}{1.407090in}}%
\pgfpathlineto{\pgfqpoint{1.661400in}{1.400064in}}%
\pgfpathlineto{\pgfqpoint{1.666486in}{1.382600in}}%
\pgfpathlineto{\pgfqpoint{1.668375in}{1.381174in}}%
\pgfpathlineto{\pgfqpoint{1.668666in}{1.381395in}}%
\pgfpathlineto{\pgfqpoint{1.670846in}{1.384973in}}%
\pgfpathlineto{\pgfqpoint{1.676222in}{1.410134in}}%
\pgfpathlineto{\pgfqpoint{1.680872in}{1.427576in}}%
\pgfpathlineto{\pgfqpoint{1.682616in}{1.428820in}}%
\pgfpathlineto{\pgfqpoint{1.683052in}{1.428411in}}%
\pgfpathlineto{\pgfqpoint{1.685522in}{1.423689in}}%
\pgfpathlineto{\pgfqpoint{1.688429in}{1.412904in}}%
\pgfpathlineto{\pgfqpoint{1.692933in}{1.400000in}}%
\pgfpathlineto{\pgfqpoint{1.693514in}{1.399306in}}%
\pgfpathlineto{\pgfqpoint{1.694532in}{1.399957in}}%
\pgfpathlineto{\pgfqpoint{1.697147in}{1.406574in}}%
\pgfpathlineto{\pgfqpoint{1.701216in}{1.425917in}}%
\pgfpathlineto{\pgfqpoint{1.705575in}{1.442149in}}%
\pgfpathlineto{\pgfqpoint{1.708191in}{1.443289in}}%
\pgfpathlineto{\pgfqpoint{1.710080in}{1.440476in}}%
\pgfpathlineto{\pgfqpoint{1.713422in}{1.431518in}}%
\pgfpathlineto{\pgfqpoint{1.716764in}{1.425557in}}%
\pgfpathlineto{\pgfqpoint{1.718363in}{1.425387in}}%
\pgfpathlineto{\pgfqpoint{1.718508in}{1.425561in}}%
\pgfpathlineto{\pgfqpoint{1.722577in}{1.434943in}}%
\pgfpathlineto{\pgfqpoint{1.730133in}{1.460185in}}%
\pgfpathlineto{\pgfqpoint{1.733039in}{1.462534in}}%
\pgfpathlineto{\pgfqpoint{1.735655in}{1.460526in}}%
\pgfpathlineto{\pgfqpoint{1.742485in}{1.453372in}}%
\pgfpathlineto{\pgfqpoint{1.748733in}{1.459844in}}%
\pgfpathlineto{\pgfqpoint{1.755563in}{1.472695in}}%
\pgfpathlineto{\pgfqpoint{1.760068in}{1.474510in}}%
\pgfpathlineto{\pgfqpoint{1.769368in}{1.477914in}}%
\pgfpathlineto{\pgfqpoint{1.772710in}{1.480085in}}%
\pgfpathlineto{\pgfqpoint{1.777796in}{1.482816in}}%
\pgfpathlineto{\pgfqpoint{1.782010in}{1.486192in}}%
\pgfpathlineto{\pgfqpoint{1.794071in}{1.502653in}}%
\pgfpathlineto{\pgfqpoint{1.796396in}{1.502341in}}%
\pgfpathlineto{\pgfqpoint{1.801046in}{1.497659in}}%
\pgfpathlineto{\pgfqpoint{1.804824in}{1.495818in}}%
\pgfpathlineto{\pgfqpoint{1.808021in}{1.498622in}}%
\pgfpathlineto{\pgfqpoint{1.811218in}{1.505588in}}%
\pgfpathlineto{\pgfqpoint{1.819500in}{1.520742in}}%
\pgfpathlineto{\pgfqpoint{1.821389in}{1.519827in}}%
\pgfpathlineto{\pgfqpoint{1.824150in}{1.515743in}}%
\pgfpathlineto{\pgfqpoint{1.829382in}{1.508605in}}%
\pgfpathlineto{\pgfqpoint{1.831271in}{1.509142in}}%
\pgfpathlineto{\pgfqpoint{1.833305in}{1.512429in}}%
\pgfpathlineto{\pgfqpoint{1.844785in}{1.540684in}}%
\pgfpathlineto{\pgfqpoint{1.845075in}{1.540518in}}%
\pgfpathlineto{\pgfqpoint{1.848272in}{1.536022in}}%
\pgfpathlineto{\pgfqpoint{1.854085in}{1.526179in}}%
\pgfpathlineto{\pgfqpoint{1.856410in}{1.526933in}}%
\pgfpathlineto{\pgfqpoint{1.859752in}{1.534018in}}%
\pgfpathlineto{\pgfqpoint{1.867599in}{1.554005in}}%
\pgfpathlineto{\pgfqpoint{1.869924in}{1.554206in}}%
\pgfpathlineto{\pgfqpoint{1.872394in}{1.551101in}}%
\pgfpathlineto{\pgfqpoint{1.881258in}{1.540530in}}%
\pgfpathlineto{\pgfqpoint{1.881549in}{1.540803in}}%
\pgfpathlineto{\pgfqpoint{1.884891in}{1.547375in}}%
\pgfpathlineto{\pgfqpoint{1.893610in}{1.566890in}}%
\pgfpathlineto{\pgfqpoint{1.896807in}{1.565865in}}%
\pgfpathlineto{\pgfqpoint{1.904944in}{1.558266in}}%
\pgfpathlineto{\pgfqpoint{1.907560in}{1.560244in}}%
\pgfpathlineto{\pgfqpoint{1.910757in}{1.565798in}}%
\pgfpathlineto{\pgfqpoint{1.917877in}{1.577256in}}%
\pgfpathlineto{\pgfqpoint{1.921800in}{1.577890in}}%
\pgfpathlineto{\pgfqpoint{1.928339in}{1.576208in}}%
\pgfpathlineto{\pgfqpoint{1.933135in}{1.577841in}}%
\pgfpathlineto{\pgfqpoint{1.948683in}{1.589169in}}%
\pgfpathlineto{\pgfqpoint{1.971788in}{1.597459in}}%
\pgfpathlineto{\pgfqpoint{1.982250in}{1.607278in}}%
\pgfpathlineto{\pgfqpoint{1.987772in}{1.602977in}}%
\pgfpathlineto{\pgfqpoint{1.991260in}{1.601398in}}%
\pgfpathlineto{\pgfqpoint{1.993730in}{1.602536in}}%
\pgfpathlineto{\pgfqpoint{1.999397in}{1.612132in}}%
\pgfpathlineto{\pgfqpoint{2.004629in}{1.619125in}}%
\pgfpathlineto{\pgfqpoint{2.007244in}{1.618941in}}%
\pgfpathlineto{\pgfqpoint{2.018724in}{1.609929in}}%
\pgfpathlineto{\pgfqpoint{2.022502in}{1.616586in}}%
\pgfpathlineto{\pgfqpoint{2.029913in}{1.629835in}}%
\pgfpathlineto{\pgfqpoint{2.032238in}{1.629649in}}%
\pgfpathlineto{\pgfqpoint{2.037033in}{1.623153in}}%
\pgfpathlineto{\pgfqpoint{2.041829in}{1.618402in}}%
\pgfpathlineto{\pgfqpoint{2.044589in}{1.620393in}}%
\pgfpathlineto{\pgfqpoint{2.047932in}{1.626385in}}%
\pgfpathlineto{\pgfqpoint{2.054761in}{1.636483in}}%
\pgfpathlineto{\pgfqpoint{2.057377in}{1.635833in}}%
\pgfpathlineto{\pgfqpoint{2.067404in}{1.626694in}}%
\pgfpathlineto{\pgfqpoint{2.070455in}{1.629198in}}%
\pgfpathlineto{\pgfqpoint{2.081499in}{1.642822in}}%
\pgfpathlineto{\pgfqpoint{2.085132in}{1.640854in}}%
\pgfpathlineto{\pgfqpoint{2.093705in}{1.637149in}}%
\pgfpathlineto{\pgfqpoint{2.100535in}{1.642887in}}%
\pgfpathlineto{\pgfqpoint{2.105766in}{1.645655in}}%
\pgfpathlineto{\pgfqpoint{2.112741in}{1.644947in}}%
\pgfpathlineto{\pgfqpoint{2.117100in}{1.645131in}}%
\pgfpathlineto{\pgfqpoint{2.132794in}{1.649781in}}%
\pgfpathlineto{\pgfqpoint{2.142239in}{1.654322in}}%
\pgfpathlineto{\pgfqpoint{2.146454in}{1.653075in}}%
\pgfpathlineto{\pgfqpoint{2.154010in}{1.650172in}}%
\pgfpathlineto{\pgfqpoint{2.158224in}{1.652575in}}%
\pgfpathlineto{\pgfqpoint{2.168105in}{1.659802in}}%
\pgfpathlineto{\pgfqpoint{2.172029in}{1.656750in}}%
\pgfpathlineto{\pgfqpoint{2.178277in}{1.651713in}}%
\pgfpathlineto{\pgfqpoint{2.181183in}{1.652966in}}%
\pgfpathlineto{\pgfqpoint{2.186850in}{1.661250in}}%
\pgfpathlineto{\pgfqpoint{2.191355in}{1.665586in}}%
\pgfpathlineto{\pgfqpoint{2.194552in}{1.664393in}}%
\pgfpathlineto{\pgfqpoint{2.205450in}{1.655245in}}%
\pgfpathlineto{\pgfqpoint{2.208211in}{1.658325in}}%
\pgfpathlineto{\pgfqpoint{2.216930in}{1.669019in}}%
\pgfpathlineto{\pgfqpoint{2.219691in}{1.667744in}}%
\pgfpathlineto{\pgfqpoint{2.230444in}{1.658600in}}%
\pgfpathlineto{\pgfqpoint{2.233786in}{1.661967in}}%
\pgfpathlineto{\pgfqpoint{2.241633in}{1.669442in}}%
\pgfpathlineto{\pgfqpoint{2.244685in}{1.667717in}}%
\pgfpathlineto{\pgfqpoint{2.254275in}{1.659533in}}%
\pgfpathlineto{\pgfqpoint{2.257763in}{1.661468in}}%
\pgfpathlineto{\pgfqpoint{2.267063in}{1.669082in}}%
\pgfpathlineto{\pgfqpoint{2.270841in}{1.668331in}}%
\pgfpathlineto{\pgfqpoint{2.279560in}{1.664912in}}%
\pgfpathlineto{\pgfqpoint{2.287407in}{1.666728in}}%
\pgfpathlineto{\pgfqpoint{2.299613in}{1.666889in}}%
\pgfpathlineto{\pgfqpoint{2.319085in}{1.664551in}}%
\pgfpathlineto{\pgfqpoint{2.330564in}{1.668410in}}%
\pgfpathlineto{\pgfqpoint{2.342625in}{1.662203in}}%
\pgfpathlineto{\pgfqpoint{2.346258in}{1.665408in}}%
\pgfpathlineto{\pgfqpoint{2.353379in}{1.671046in}}%
\pgfpathlineto{\pgfqpoint{2.356721in}{1.669149in}}%
\pgfpathlineto{\pgfqpoint{2.366747in}{1.659189in}}%
\pgfpathlineto{\pgfqpoint{2.369944in}{1.661633in}}%
\pgfpathlineto{\pgfqpoint{2.378518in}{1.670181in}}%
\pgfpathlineto{\pgfqpoint{2.381714in}{1.667892in}}%
\pgfpathlineto{\pgfqpoint{2.391450in}{1.656992in}}%
\pgfpathlineto{\pgfqpoint{2.395083in}{1.660001in}}%
\pgfpathlineto{\pgfqpoint{2.403221in}{1.668708in}}%
\pgfpathlineto{\pgfqpoint{2.406563in}{1.666885in}}%
\pgfpathlineto{\pgfqpoint{2.417461in}{1.656933in}}%
\pgfpathlineto{\pgfqpoint{2.420804in}{1.659623in}}%
\pgfpathlineto{\pgfqpoint{2.428214in}{1.665694in}}%
\pgfpathlineto{\pgfqpoint{2.432719in}{1.663001in}}%
\pgfpathlineto{\pgfqpoint{2.440275in}{1.656929in}}%
\pgfpathlineto{\pgfqpoint{2.445361in}{1.658156in}}%
\pgfpathlineto{\pgfqpoint{2.454225in}{1.660938in}}%
\pgfpathlineto{\pgfqpoint{2.459021in}{1.658954in}}%
\pgfpathlineto{\pgfqpoint{2.470646in}{1.655418in}}%
\pgfpathlineto{\pgfqpoint{2.480672in}{1.655328in}}%
\pgfpathlineto{\pgfqpoint{2.491861in}{1.656587in}}%
\pgfpathlineto{\pgfqpoint{2.496947in}{1.653808in}}%
\pgfpathlineto{\pgfqpoint{2.503632in}{1.651374in}}%
\pgfpathlineto{\pgfqpoint{2.507991in}{1.653872in}}%
\pgfpathlineto{\pgfqpoint{2.514385in}{1.658181in}}%
\pgfpathlineto{\pgfqpoint{2.517436in}{1.657064in}}%
\pgfpathlineto{\pgfqpoint{2.522668in}{1.650690in}}%
\pgfpathlineto{\pgfqpoint{2.527172in}{1.646653in}}%
\pgfpathlineto{\pgfqpoint{2.530369in}{1.647640in}}%
\pgfpathlineto{\pgfqpoint{2.535019in}{1.653212in}}%
\pgfpathlineto{\pgfqpoint{2.539960in}{1.656809in}}%
\pgfpathlineto{\pgfqpoint{2.543011in}{1.655058in}}%
\pgfpathlineto{\pgfqpoint{2.554346in}{1.643769in}}%
\pgfpathlineto{\pgfqpoint{2.557833in}{1.647832in}}%
\pgfpathlineto{\pgfqpoint{2.563936in}{1.655586in}}%
\pgfpathlineto{\pgfqpoint{2.567279in}{1.654785in}}%
\pgfpathlineto{\pgfqpoint{2.570911in}{1.649791in}}%
\pgfpathlineto{\pgfqpoint{2.577014in}{1.642098in}}%
\pgfpathlineto{\pgfqpoint{2.579921in}{1.643029in}}%
\pgfpathlineto{\pgfqpoint{2.583844in}{1.647983in}}%
\pgfpathlineto{\pgfqpoint{2.589511in}{1.653572in}}%
\pgfpathlineto{\pgfqpoint{2.592418in}{1.652788in}}%
\pgfpathlineto{\pgfqpoint{2.596777in}{1.647616in}}%
\pgfpathlineto{\pgfqpoint{2.602444in}{1.642731in}}%
\pgfpathlineto{\pgfqpoint{2.606658in}{1.644353in}}%
\pgfpathlineto{\pgfqpoint{2.615522in}{1.650370in}}%
\pgfpathlineto{\pgfqpoint{2.621189in}{1.646830in}}%
\pgfpathlineto{\pgfqpoint{2.627147in}{1.643260in}}%
\pgfpathlineto{\pgfqpoint{2.633541in}{1.644923in}}%
\pgfpathlineto{\pgfqpoint{2.640661in}{1.646949in}}%
\pgfpathlineto{\pgfqpoint{2.670450in}{1.643763in}}%
\pgfpathlineto{\pgfqpoint{2.677861in}{1.646199in}}%
\pgfpathlineto{\pgfqpoint{2.681785in}{1.644119in}}%
\pgfpathlineto{\pgfqpoint{2.689486in}{1.639861in}}%
\pgfpathlineto{\pgfqpoint{2.693555in}{1.642488in}}%
\pgfpathlineto{\pgfqpoint{2.702710in}{1.650120in}}%
\pgfpathlineto{\pgfqpoint{2.706343in}{1.646945in}}%
\pgfpathlineto{\pgfqpoint{2.714771in}{1.637658in}}%
\pgfpathlineto{\pgfqpoint{2.718694in}{1.640848in}}%
\pgfpathlineto{\pgfqpoint{2.726541in}{1.649684in}}%
\pgfpathlineto{\pgfqpoint{2.730029in}{1.647591in}}%
\pgfpathlineto{\pgfqpoint{2.740346in}{1.636382in}}%
\pgfpathlineto{\pgfqpoint{2.742816in}{1.638449in}}%
\pgfpathlineto{\pgfqpoint{2.753714in}{1.649701in}}%
\pgfpathlineto{\pgfqpoint{2.756766in}{1.646637in}}%
\pgfpathlineto{\pgfqpoint{2.762869in}{1.639280in}}%
\pgfpathlineto{\pgfqpoint{2.766066in}{1.639413in}}%
\pgfpathlineto{\pgfqpoint{2.770425in}{1.644724in}}%
\pgfpathlineto{\pgfqpoint{2.775511in}{1.650178in}}%
\pgfpathlineto{\pgfqpoint{2.778999in}{1.649712in}}%
\pgfpathlineto{\pgfqpoint{2.781760in}{1.647110in}}%
\pgfpathlineto{\pgfqpoint{2.787863in}{1.641330in}}%
\pgfpathlineto{\pgfqpoint{2.791350in}{1.641245in}}%
\pgfpathlineto{\pgfqpoint{2.797018in}{1.645246in}}%
\pgfpathlineto{\pgfqpoint{2.801813in}{1.647575in}}%
\pgfpathlineto{\pgfqpoint{2.806318in}{1.646519in}}%
\pgfpathlineto{\pgfqpoint{2.813874in}{1.643852in}}%
\pgfpathlineto{\pgfqpoint{2.822302in}{1.645469in}}%
\pgfpathlineto{\pgfqpoint{2.838868in}{1.648874in}}%
\pgfpathlineto{\pgfqpoint{2.844680in}{1.647118in}}%
\pgfpathlineto{\pgfqpoint{2.851800in}{1.644928in}}%
\pgfpathlineto{\pgfqpoint{2.857758in}{1.648676in}}%
\pgfpathlineto{\pgfqpoint{2.863280in}{1.651783in}}%
\pgfpathlineto{\pgfqpoint{2.867494in}{1.649494in}}%
\pgfpathlineto{\pgfqpoint{2.876358in}{1.641915in}}%
\pgfpathlineto{\pgfqpoint{2.878974in}{1.643237in}}%
\pgfpathlineto{\pgfqpoint{2.890454in}{1.652720in}}%
\pgfpathlineto{\pgfqpoint{2.895249in}{1.647191in}}%
\pgfpathlineto{\pgfqpoint{2.900916in}{1.641820in}}%
\pgfpathlineto{\pgfqpoint{2.903386in}{1.642938in}}%
\pgfpathlineto{\pgfqpoint{2.915157in}{1.655871in}}%
\pgfpathlineto{\pgfqpoint{2.915447in}{1.655611in}}%
\pgfpathlineto{\pgfqpoint{2.920097in}{1.649736in}}%
\pgfpathlineto{\pgfqpoint{2.926200in}{1.643978in}}%
\pgfpathlineto{\pgfqpoint{2.928671in}{1.645340in}}%
\pgfpathlineto{\pgfqpoint{2.938552in}{1.656430in}}%
\pgfpathlineto{\pgfqpoint{2.941749in}{1.654712in}}%
\pgfpathlineto{\pgfqpoint{2.952502in}{1.644771in}}%
\pgfpathlineto{\pgfqpoint{2.956716in}{1.648253in}}%
\pgfpathlineto{\pgfqpoint{2.962529in}{1.653437in}}%
\pgfpathlineto{\pgfqpoint{2.965725in}{1.653018in}}%
\pgfpathlineto{\pgfqpoint{2.977932in}{1.647149in}}%
\pgfpathlineto{\pgfqpoint{2.982000in}{1.649551in}}%
\pgfpathlineto{\pgfqpoint{2.987232in}{1.652643in}}%
\pgfpathlineto{\pgfqpoint{2.992318in}{1.653028in}}%
\pgfpathlineto{\pgfqpoint{3.000455in}{1.651985in}}%
\pgfpathlineto{\pgfqpoint{3.004814in}{1.651433in}}%
\pgfpathlineto{\pgfqpoint{3.004814in}{1.651433in}}%
\pgfusepath{stroke}%
\end{pgfscope}%
\begin{pgfscope}%
\pgfpathrectangle{\pgfqpoint{0.679669in}{0.526079in}}{\pgfqpoint{2.325000in}{1.661000in}} %
\pgfusepath{clip}%
\pgfsetbuttcap%
\pgfsetroundjoin%
\pgfsetlinewidth{1.003750pt}%
\definecolor{currentstroke}{rgb}{0.627451,0.321569,0.176471}%
\pgfsetstrokecolor{currentstroke}%
\pgfsetdash{{3.700000pt}{1.600000pt}}{0.000000pt}%
\pgfpathmoveto{\pgfqpoint{0.682558in}{0.512191in}}%
\pgfpathlineto{\pgfqpoint{0.686499in}{0.845595in}}%
\pgfpathlineto{\pgfqpoint{0.690858in}{0.998743in}}%
\pgfpathlineto{\pgfqpoint{0.694491in}{1.041427in}}%
\pgfpathlineto{\pgfqpoint{0.696380in}{1.044921in}}%
\pgfpathlineto{\pgfqpoint{0.696816in}{1.044546in}}%
\pgfpathlineto{\pgfqpoint{0.698850in}{1.039139in}}%
\pgfpathlineto{\pgfqpoint{0.704227in}{1.015819in}}%
\pgfpathlineto{\pgfqpoint{0.711493in}{0.977799in}}%
\pgfpathlineto{\pgfqpoint{0.713382in}{0.978683in}}%
\pgfpathlineto{\pgfqpoint{0.714980in}{0.977787in}}%
\pgfpathlineto{\pgfqpoint{0.717160in}{0.970526in}}%
\pgfpathlineto{\pgfqpoint{0.721810in}{0.945722in}}%
\pgfpathlineto{\pgfqpoint{0.722827in}{0.948514in}}%
\pgfpathlineto{\pgfqpoint{0.724861in}{0.969404in}}%
\pgfpathlineto{\pgfqpoint{0.729947in}{1.016808in}}%
\pgfpathlineto{\pgfqpoint{0.730529in}{1.016783in}}%
\pgfpathlineto{\pgfqpoint{0.730964in}{1.015916in}}%
\pgfpathlineto{\pgfqpoint{0.732708in}{1.005896in}}%
\pgfpathlineto{\pgfqpoint{0.737213in}{0.973963in}}%
\pgfpathlineto{\pgfqpoint{0.737794in}{0.975140in}}%
\pgfpathlineto{\pgfqpoint{0.739538in}{0.988213in}}%
\pgfpathlineto{\pgfqpoint{0.746513in}{1.046384in}}%
\pgfpathlineto{\pgfqpoint{0.748402in}{1.048434in}}%
\pgfpathlineto{\pgfqpoint{0.748838in}{1.048145in}}%
\pgfpathlineto{\pgfqpoint{0.750582in}{1.044079in}}%
\pgfpathlineto{\pgfqpoint{0.752761in}{1.029778in}}%
\pgfpathlineto{\pgfqpoint{0.755668in}{0.985833in}}%
\pgfpathlineto{\pgfqpoint{0.763079in}{0.825499in}}%
\pgfpathlineto{\pgfqpoint{0.764677in}{0.832157in}}%
\pgfpathlineto{\pgfqpoint{0.765839in}{0.835399in}}%
\pgfpathlineto{\pgfqpoint{0.766566in}{0.833841in}}%
\pgfpathlineto{\pgfqpoint{0.768455in}{0.818697in}}%
\pgfpathlineto{\pgfqpoint{0.770780in}{0.800622in}}%
\pgfpathlineto{\pgfqpoint{0.771361in}{0.803598in}}%
\pgfpathlineto{\pgfqpoint{0.772814in}{0.835072in}}%
\pgfpathlineto{\pgfqpoint{0.780080in}{1.018711in}}%
\pgfpathlineto{\pgfqpoint{0.781969in}{1.026310in}}%
\pgfpathlineto{\pgfqpoint{0.782405in}{1.025952in}}%
\pgfpathlineto{\pgfqpoint{0.783713in}{1.020502in}}%
\pgfpathlineto{\pgfqpoint{0.791850in}{0.970287in}}%
\pgfpathlineto{\pgfqpoint{0.792722in}{0.971347in}}%
\pgfpathlineto{\pgfqpoint{0.794902in}{0.979470in}}%
\pgfpathlineto{\pgfqpoint{0.802313in}{1.018976in}}%
\pgfpathlineto{\pgfqpoint{0.803185in}{1.017449in}}%
\pgfpathlineto{\pgfqpoint{0.804929in}{1.006075in}}%
\pgfpathlineto{\pgfqpoint{0.808271in}{0.957109in}}%
\pgfpathlineto{\pgfqpoint{0.811904in}{0.919200in}}%
\pgfpathlineto{\pgfqpoint{0.812630in}{0.920411in}}%
\pgfpathlineto{\pgfqpoint{0.816554in}{0.935765in}}%
\pgfpathlineto{\pgfqpoint{0.817571in}{0.933475in}}%
\pgfpathlineto{\pgfqpoint{0.819750in}{0.917569in}}%
\pgfpathlineto{\pgfqpoint{0.822947in}{0.894407in}}%
\pgfpathlineto{\pgfqpoint{0.823529in}{0.896335in}}%
\pgfpathlineto{\pgfqpoint{0.825127in}{0.915642in}}%
\pgfpathlineto{\pgfqpoint{0.833555in}{1.037307in}}%
\pgfpathlineto{\pgfqpoint{0.836025in}{1.043771in}}%
\pgfpathlineto{\pgfqpoint{0.836171in}{1.043724in}}%
\pgfpathlineto{\pgfqpoint{0.837624in}{1.041239in}}%
\pgfpathlineto{\pgfqpoint{0.839658in}{1.030271in}}%
\pgfpathlineto{\pgfqpoint{0.842855in}{0.993346in}}%
\pgfpathlineto{\pgfqpoint{0.847360in}{0.944668in}}%
\pgfpathlineto{\pgfqpoint{0.847650in}{0.944899in}}%
\pgfpathlineto{\pgfqpoint{0.848813in}{0.950473in}}%
\pgfpathlineto{\pgfqpoint{0.856950in}{1.001040in}}%
\pgfpathlineto{\pgfqpoint{0.858985in}{1.009877in}}%
\pgfpathlineto{\pgfqpoint{0.862036in}{1.042599in}}%
\pgfpathlineto{\pgfqpoint{0.866541in}{1.081812in}}%
\pgfpathlineto{\pgfqpoint{0.867268in}{1.082188in}}%
\pgfpathlineto{\pgfqpoint{0.867849in}{1.081286in}}%
\pgfpathlineto{\pgfqpoint{0.869593in}{1.073052in}}%
\pgfpathlineto{\pgfqpoint{0.874097in}{1.030752in}}%
\pgfpathlineto{\pgfqpoint{0.876568in}{1.018054in}}%
\pgfpathlineto{\pgfqpoint{0.876858in}{1.018310in}}%
\pgfpathlineto{\pgfqpoint{0.878021in}{1.023206in}}%
\pgfpathlineto{\pgfqpoint{0.881072in}{1.054378in}}%
\pgfpathlineto{\pgfqpoint{0.887321in}{1.113066in}}%
\pgfpathlineto{\pgfqpoint{0.889210in}{1.116598in}}%
\pgfpathlineto{\pgfqpoint{0.889646in}{1.116139in}}%
\pgfpathlineto{\pgfqpoint{0.891244in}{1.110061in}}%
\pgfpathlineto{\pgfqpoint{0.893860in}{1.085516in}}%
\pgfpathlineto{\pgfqpoint{0.898655in}{1.002611in}}%
\pgfpathlineto{\pgfqpoint{0.902724in}{0.956514in}}%
\pgfpathlineto{\pgfqpoint{0.906938in}{0.935228in}}%
\pgfpathlineto{\pgfqpoint{0.907519in}{0.936484in}}%
\pgfpathlineto{\pgfqpoint{0.908972in}{0.950964in}}%
\pgfpathlineto{\pgfqpoint{0.919144in}{1.098861in}}%
\pgfpathlineto{\pgfqpoint{0.919725in}{1.098392in}}%
\pgfpathlineto{\pgfqpoint{0.921179in}{1.092940in}}%
\pgfpathlineto{\pgfqpoint{0.924085in}{1.066796in}}%
\pgfpathlineto{\pgfqpoint{0.929316in}{1.018538in}}%
\pgfpathlineto{\pgfqpoint{0.929607in}{1.018803in}}%
\pgfpathlineto{\pgfqpoint{0.930769in}{1.024868in}}%
\pgfpathlineto{\pgfqpoint{0.933675in}{1.064457in}}%
\pgfpathlineto{\pgfqpoint{0.939052in}{1.125479in}}%
\pgfpathlineto{\pgfqpoint{0.941377in}{1.131740in}}%
\pgfpathlineto{\pgfqpoint{0.941668in}{1.131539in}}%
\pgfpathlineto{\pgfqpoint{0.943121in}{1.127735in}}%
\pgfpathlineto{\pgfqpoint{0.946463in}{1.107139in}}%
\pgfpathlineto{\pgfqpoint{0.950968in}{1.086086in}}%
\pgfpathlineto{\pgfqpoint{0.959105in}{1.070028in}}%
\pgfpathlineto{\pgfqpoint{0.960122in}{1.071978in}}%
\pgfpathlineto{\pgfqpoint{0.962302in}{1.084600in}}%
\pgfpathlineto{\pgfqpoint{0.972910in}{1.159098in}}%
\pgfpathlineto{\pgfqpoint{0.974508in}{1.160261in}}%
\pgfpathlineto{\pgfqpoint{0.974944in}{1.159853in}}%
\pgfpathlineto{\pgfqpoint{0.976688in}{1.154958in}}%
\pgfpathlineto{\pgfqpoint{0.979739in}{1.134665in}}%
\pgfpathlineto{\pgfqpoint{0.986714in}{1.086103in}}%
\pgfpathlineto{\pgfqpoint{0.988022in}{1.085944in}}%
\pgfpathlineto{\pgfqpoint{0.988458in}{1.086413in}}%
\pgfpathlineto{\pgfqpoint{0.998194in}{1.102043in}}%
\pgfpathlineto{\pgfqpoint{1.004152in}{1.121708in}}%
\pgfpathlineto{\pgfqpoint{1.004588in}{1.121448in}}%
\pgfpathlineto{\pgfqpoint{1.006477in}{1.117485in}}%
\pgfpathlineto{\pgfqpoint{1.011708in}{1.103860in}}%
\pgfpathlineto{\pgfqpoint{1.012144in}{1.104177in}}%
\pgfpathlineto{\pgfqpoint{1.014033in}{1.108867in}}%
\pgfpathlineto{\pgfqpoint{1.018247in}{1.128780in}}%
\pgfpathlineto{\pgfqpoint{1.026530in}{1.172178in}}%
\pgfpathlineto{\pgfqpoint{1.026821in}{1.172033in}}%
\pgfpathlineto{\pgfqpoint{1.028274in}{1.168614in}}%
\pgfpathlineto{\pgfqpoint{1.031035in}{1.151423in}}%
\pgfpathlineto{\pgfqpoint{1.038591in}{1.096954in}}%
\pgfpathlineto{\pgfqpoint{1.038882in}{1.097109in}}%
\pgfpathlineto{\pgfqpoint{1.040189in}{1.100502in}}%
\pgfpathlineto{\pgfqpoint{1.043096in}{1.119214in}}%
\pgfpathlineto{\pgfqpoint{1.054575in}{1.206264in}}%
\pgfpathlineto{\pgfqpoint{1.054866in}{1.206099in}}%
\pgfpathlineto{\pgfqpoint{1.056319in}{1.202695in}}%
\pgfpathlineto{\pgfqpoint{1.059371in}{1.185232in}}%
\pgfpathlineto{\pgfqpoint{1.066636in}{1.142533in}}%
\pgfpathlineto{\pgfqpoint{1.067508in}{1.142503in}}%
\pgfpathlineto{\pgfqpoint{1.067944in}{1.143236in}}%
\pgfpathlineto{\pgfqpoint{1.069688in}{1.151115in}}%
\pgfpathlineto{\pgfqpoint{1.078261in}{1.205575in}}%
\pgfpathlineto{\pgfqpoint{1.079133in}{1.204723in}}%
\pgfpathlineto{\pgfqpoint{1.081168in}{1.198078in}}%
\pgfpathlineto{\pgfqpoint{1.088724in}{1.161488in}}%
\pgfpathlineto{\pgfqpoint{1.089886in}{1.162911in}}%
\pgfpathlineto{\pgfqpoint{1.092066in}{1.172377in}}%
\pgfpathlineto{\pgfqpoint{1.097007in}{1.212478in}}%
\pgfpathlineto{\pgfqpoint{1.103400in}{1.256679in}}%
\pgfpathlineto{\pgfqpoint{1.105580in}{1.259703in}}%
\pgfpathlineto{\pgfqpoint{1.105871in}{1.259526in}}%
\pgfpathlineto{\pgfqpoint{1.107614in}{1.255892in}}%
\pgfpathlineto{\pgfqpoint{1.112264in}{1.235601in}}%
\pgfpathlineto{\pgfqpoint{1.118513in}{1.213580in}}%
\pgfpathlineto{\pgfqpoint{1.119966in}{1.213857in}}%
\pgfpathlineto{\pgfqpoint{1.120111in}{1.214055in}}%
\pgfpathlineto{\pgfqpoint{1.122436in}{1.220488in}}%
\pgfpathlineto{\pgfqpoint{1.128830in}{1.238773in}}%
\pgfpathlineto{\pgfqpoint{1.130864in}{1.238233in}}%
\pgfpathlineto{\pgfqpoint{1.135079in}{1.231740in}}%
\pgfpathlineto{\pgfqpoint{1.138421in}{1.229492in}}%
\pgfpathlineto{\pgfqpoint{1.140746in}{1.231422in}}%
\pgfpathlineto{\pgfqpoint{1.150772in}{1.248643in}}%
\pgfpathlineto{\pgfqpoint{1.158183in}{1.262095in}}%
\pgfpathlineto{\pgfqpoint{1.162107in}{1.264380in}}%
\pgfpathlineto{\pgfqpoint{1.164432in}{1.261808in}}%
\pgfpathlineto{\pgfqpoint{1.171988in}{1.247645in}}%
\pgfpathlineto{\pgfqpoint{1.173005in}{1.248615in}}%
\pgfpathlineto{\pgfqpoint{1.176638in}{1.256526in}}%
\pgfpathlineto{\pgfqpoint{1.191314in}{1.289690in}}%
\pgfpathlineto{\pgfqpoint{1.193494in}{1.288894in}}%
\pgfpathlineto{\pgfqpoint{1.196836in}{1.283015in}}%
\pgfpathlineto{\pgfqpoint{1.201777in}{1.275977in}}%
\pgfpathlineto{\pgfqpoint{1.203811in}{1.277465in}}%
\pgfpathlineto{\pgfqpoint{1.206427in}{1.284446in}}%
\pgfpathlineto{\pgfqpoint{1.215291in}{1.313582in}}%
\pgfpathlineto{\pgfqpoint{1.215872in}{1.313272in}}%
\pgfpathlineto{\pgfqpoint{1.218052in}{1.309149in}}%
\pgfpathlineto{\pgfqpoint{1.225318in}{1.293547in}}%
\pgfpathlineto{\pgfqpoint{1.227061in}{1.296056in}}%
\pgfpathlineto{\pgfqpoint{1.229968in}{1.307855in}}%
\pgfpathlineto{\pgfqpoint{1.239994in}{1.353032in}}%
\pgfpathlineto{\pgfqpoint{1.242174in}{1.353268in}}%
\pgfpathlineto{\pgfqpoint{1.244499in}{1.349317in}}%
\pgfpathlineto{\pgfqpoint{1.252636in}{1.332079in}}%
\pgfpathlineto{\pgfqpoint{1.252927in}{1.332234in}}%
\pgfpathlineto{\pgfqpoint{1.254961in}{1.335463in}}%
\pgfpathlineto{\pgfqpoint{1.258449in}{1.348524in}}%
\pgfpathlineto{\pgfqpoint{1.264697in}{1.369975in}}%
\pgfpathlineto{\pgfqpoint{1.266732in}{1.369861in}}%
\pgfpathlineto{\pgfqpoint{1.269347in}{1.365081in}}%
\pgfpathlineto{\pgfqpoint{1.276032in}{1.352590in}}%
\pgfpathlineto{\pgfqpoint{1.277921in}{1.354217in}}%
\pgfpathlineto{\pgfqpoint{1.280827in}{1.362545in}}%
\pgfpathlineto{\pgfqpoint{1.290418in}{1.393055in}}%
\pgfpathlineto{\pgfqpoint{1.293033in}{1.393340in}}%
\pgfpathlineto{\pgfqpoint{1.296811in}{1.389349in}}%
\pgfpathlineto{\pgfqpoint{1.301171in}{1.386342in}}%
\pgfpathlineto{\pgfqpoint{1.303932in}{1.388283in}}%
\pgfpathlineto{\pgfqpoint{1.307274in}{1.395124in}}%
\pgfpathlineto{\pgfqpoint{1.317155in}{1.417646in}}%
\pgfpathlineto{\pgfqpoint{1.320497in}{1.417182in}}%
\pgfpathlineto{\pgfqpoint{1.328925in}{1.414995in}}%
\pgfpathlineto{\pgfqpoint{1.333285in}{1.417704in}}%
\pgfpathlineto{\pgfqpoint{1.353047in}{1.432986in}}%
\pgfpathlineto{\pgfqpoint{1.361185in}{1.432406in}}%
\pgfpathlineto{\pgfqpoint{1.364963in}{1.438644in}}%
\pgfpathlineto{\pgfqpoint{1.377896in}{1.462743in}}%
\pgfpathlineto{\pgfqpoint{1.382546in}{1.462909in}}%
\pgfpathlineto{\pgfqpoint{1.386614in}{1.464087in}}%
\pgfpathlineto{\pgfqpoint{1.389811in}{1.468716in}}%
\pgfpathlineto{\pgfqpoint{1.394607in}{1.482586in}}%
\pgfpathlineto{\pgfqpoint{1.401146in}{1.499038in}}%
\pgfpathlineto{\pgfqpoint{1.403907in}{1.499827in}}%
\pgfpathlineto{\pgfqpoint{1.406813in}{1.496634in}}%
\pgfpathlineto{\pgfqpoint{1.412916in}{1.489840in}}%
\pgfpathlineto{\pgfqpoint{1.415241in}{1.491507in}}%
\pgfpathlineto{\pgfqpoint{1.419164in}{1.499416in}}%
\pgfpathlineto{\pgfqpoint{1.425413in}{1.510672in}}%
\pgfpathlineto{\pgfqpoint{1.427883in}{1.510253in}}%
\pgfpathlineto{\pgfqpoint{1.431225in}{1.505489in}}%
\pgfpathlineto{\pgfqpoint{1.438200in}{1.495647in}}%
\pgfpathlineto{\pgfqpoint{1.440525in}{1.497203in}}%
\pgfpathlineto{\pgfqpoint{1.443868in}{1.504339in}}%
\pgfpathlineto{\pgfqpoint{1.453313in}{1.526024in}}%
\pgfpathlineto{\pgfqpoint{1.456655in}{1.526532in}}%
\pgfpathlineto{\pgfqpoint{1.462758in}{1.526162in}}%
\pgfpathlineto{\pgfqpoint{1.465810in}{1.530547in}}%
\pgfpathlineto{\pgfqpoint{1.471477in}{1.545823in}}%
\pgfpathlineto{\pgfqpoint{1.477289in}{1.557856in}}%
\pgfpathlineto{\pgfqpoint{1.481213in}{1.559961in}}%
\pgfpathlineto{\pgfqpoint{1.492402in}{1.562324in}}%
\pgfpathlineto{\pgfqpoint{1.499522in}{1.571852in}}%
\pgfpathlineto{\pgfqpoint{1.506788in}{1.579178in}}%
\pgfpathlineto{\pgfqpoint{1.515216in}{1.582498in}}%
\pgfpathlineto{\pgfqpoint{1.521610in}{1.586217in}}%
\pgfpathlineto{\pgfqpoint{1.529021in}{1.595189in}}%
\pgfpathlineto{\pgfqpoint{1.537739in}{1.604876in}}%
\pgfpathlineto{\pgfqpoint{1.541808in}{1.604883in}}%
\pgfpathlineto{\pgfqpoint{1.549364in}{1.604016in}}%
\pgfpathlineto{\pgfqpoint{1.553143in}{1.608103in}}%
\pgfpathlineto{\pgfqpoint{1.564913in}{1.623409in}}%
\pgfpathlineto{\pgfqpoint{1.569999in}{1.622684in}}%
\pgfpathlineto{\pgfqpoint{1.573922in}{1.623545in}}%
\pgfpathlineto{\pgfqpoint{1.577410in}{1.627937in}}%
\pgfpathlineto{\pgfqpoint{1.584239in}{1.643340in}}%
\pgfpathlineto{\pgfqpoint{1.589471in}{1.651122in}}%
\pgfpathlineto{\pgfqpoint{1.593394in}{1.651922in}}%
\pgfpathlineto{\pgfqpoint{1.599643in}{1.652451in}}%
\pgfpathlineto{\pgfqpoint{1.603130in}{1.657121in}}%
\pgfpathlineto{\pgfqpoint{1.616789in}{1.679771in}}%
\pgfpathlineto{\pgfqpoint{1.628124in}{1.683181in}}%
\pgfpathlineto{\pgfqpoint{1.642219in}{1.697930in}}%
\pgfpathlineto{\pgfqpoint{1.653989in}{1.699129in}}%
\pgfpathlineto{\pgfqpoint{1.660674in}{1.707233in}}%
\pgfpathlineto{\pgfqpoint{1.668085in}{1.714227in}}%
\pgfpathlineto{\pgfqpoint{1.680146in}{1.723279in}}%
\pgfpathlineto{\pgfqpoint{1.691335in}{1.738519in}}%
\pgfpathlineto{\pgfqpoint{1.700199in}{1.747908in}}%
\pgfpathlineto{\pgfqpoint{1.732604in}{1.768982in}}%
\pgfpathlineto{\pgfqpoint{1.737544in}{1.771673in}}%
\pgfpathlineto{\pgfqpoint{1.743357in}{1.779220in}}%
\pgfpathlineto{\pgfqpoint{1.751058in}{1.788120in}}%
\pgfpathlineto{\pgfqpoint{1.756725in}{1.789648in}}%
\pgfpathlineto{\pgfqpoint{1.762393in}{1.791901in}}%
\pgfpathlineto{\pgfqpoint{1.767769in}{1.798199in}}%
\pgfpathlineto{\pgfqpoint{1.776197in}{1.807499in}}%
\pgfpathlineto{\pgfqpoint{1.781574in}{1.808522in}}%
\pgfpathlineto{\pgfqpoint{1.787677in}{1.810172in}}%
\pgfpathlineto{\pgfqpoint{1.792763in}{1.815557in}}%
\pgfpathlineto{\pgfqpoint{1.802208in}{1.825676in}}%
\pgfpathlineto{\pgfqpoint{1.810055in}{1.827948in}}%
\pgfpathlineto{\pgfqpoint{1.815722in}{1.831547in}}%
\pgfpathlineto{\pgfqpoint{1.829382in}{1.842036in}}%
\pgfpathlineto{\pgfqpoint{1.844930in}{1.847752in}}%
\pgfpathlineto{\pgfqpoint{1.862949in}{1.858922in}}%
\pgfpathlineto{\pgfqpoint{1.873993in}{1.866677in}}%
\pgfpathlineto{\pgfqpoint{1.888669in}{1.877395in}}%
\pgfpathlineto{\pgfqpoint{1.901893in}{1.883920in}}%
\pgfpathlineto{\pgfqpoint{1.914244in}{1.892515in}}%
\pgfpathlineto{\pgfqpoint{1.927177in}{1.897238in}}%
\pgfpathlineto{\pgfqpoint{1.939819in}{1.906033in}}%
\pgfpathlineto{\pgfqpoint{1.952025in}{1.909695in}}%
\pgfpathlineto{\pgfqpoint{1.965394in}{1.918378in}}%
\pgfpathlineto{\pgfqpoint{1.977746in}{1.921301in}}%
\pgfpathlineto{\pgfqpoint{1.991841in}{1.928971in}}%
\pgfpathlineto{\pgfqpoint{2.002885in}{1.932067in}}%
\pgfpathlineto{\pgfqpoint{2.019305in}{1.939676in}}%
\pgfpathlineto{\pgfqpoint{2.034127in}{1.943924in}}%
\pgfpathlineto{\pgfqpoint{2.052582in}{1.950471in}}%
\pgfpathlineto{\pgfqpoint{2.065224in}{1.954725in}}%
\pgfpathlineto{\pgfqpoint{2.075250in}{1.957705in}}%
\pgfpathlineto{\pgfqpoint{2.089636in}{1.959719in}}%
\pgfpathlineto{\pgfqpoint{2.101116in}{1.964017in}}%
\pgfpathlineto{\pgfqpoint{2.115066in}{1.965645in}}%
\pgfpathlineto{\pgfqpoint{2.125819in}{1.969372in}}%
\pgfpathlineto{\pgfqpoint{2.140932in}{1.970375in}}%
\pgfpathlineto{\pgfqpoint{2.151975in}{1.974230in}}%
\pgfpathlineto{\pgfqpoint{2.166216in}{1.976281in}}%
\pgfpathlineto{\pgfqpoint{2.177405in}{1.979500in}}%
\pgfpathlineto{\pgfqpoint{2.196732in}{1.981533in}}%
\pgfpathlineto{\pgfqpoint{2.209374in}{1.982998in}}%
\pgfpathlineto{\pgfqpoint{2.226085in}{1.984848in}}%
\pgfpathlineto{\pgfqpoint{2.239163in}{1.986564in}}%
\pgfpathlineto{\pgfqpoint{2.252096in}{1.987699in}}%
\pgfpathlineto{\pgfqpoint{2.261396in}{1.988387in}}%
\pgfpathlineto{\pgfqpoint{2.279269in}{1.986323in}}%
\pgfpathlineto{\pgfqpoint{2.287843in}{1.986849in}}%
\pgfpathlineto{\pgfqpoint{2.301647in}{1.985704in}}%
\pgfpathlineto{\pgfqpoint{2.313272in}{1.987561in}}%
\pgfpathlineto{\pgfqpoint{2.328385in}{1.986290in}}%
\pgfpathlineto{\pgfqpoint{2.338557in}{1.987116in}}%
\pgfpathlineto{\pgfqpoint{2.354396in}{1.985860in}}%
\pgfpathlineto{\pgfqpoint{2.367183in}{1.986763in}}%
\pgfpathlineto{\pgfqpoint{2.386510in}{1.986526in}}%
\pgfpathlineto{\pgfqpoint{2.398861in}{1.985394in}}%
\pgfpathlineto{\pgfqpoint{2.413974in}{1.983522in}}%
\pgfpathlineto{\pgfqpoint{2.424582in}{1.982580in}}%
\pgfpathlineto{\pgfqpoint{2.437950in}{1.980010in}}%
\pgfpathlineto{\pgfqpoint{2.450447in}{1.979911in}}%
\pgfpathlineto{\pgfqpoint{2.462363in}{1.977811in}}%
\pgfpathlineto{\pgfqpoint{2.475441in}{1.978497in}}%
\pgfpathlineto{\pgfqpoint{2.489100in}{1.975286in}}%
\pgfpathlineto{\pgfqpoint{2.499854in}{1.974977in}}%
\pgfpathlineto{\pgfqpoint{2.514821in}{1.971661in}}%
\pgfpathlineto{\pgfqpoint{2.527754in}{1.971886in}}%
\pgfpathlineto{\pgfqpoint{2.540832in}{1.970742in}}%
\pgfpathlineto{\pgfqpoint{2.556671in}{1.970739in}}%
\pgfpathlineto{\pgfqpoint{2.574835in}{1.969719in}}%
\pgfpathlineto{\pgfqpoint{2.586896in}{1.968960in}}%
\pgfpathlineto{\pgfqpoint{2.599247in}{1.967684in}}%
\pgfpathlineto{\pgfqpoint{2.612035in}{1.968039in}}%
\pgfpathlineto{\pgfqpoint{2.624241in}{1.966341in}}%
\pgfpathlineto{\pgfqpoint{2.636883in}{1.967101in}}%
\pgfpathlineto{\pgfqpoint{2.649380in}{1.964887in}}%
\pgfpathlineto{\pgfqpoint{2.663039in}{1.966442in}}%
\pgfpathlineto{\pgfqpoint{2.675536in}{1.964447in}}%
\pgfpathlineto{\pgfqpoint{2.686725in}{1.964535in}}%
\pgfpathlineto{\pgfqpoint{2.701547in}{1.961693in}}%
\pgfpathlineto{\pgfqpoint{2.719275in}{1.963114in}}%
\pgfpathlineto{\pgfqpoint{2.730755in}{1.964502in}}%
\pgfpathlineto{\pgfqpoint{2.744269in}{1.965700in}}%
\pgfpathlineto{\pgfqpoint{2.767664in}{1.965612in}}%
\pgfpathlineto{\pgfqpoint{2.777836in}{1.964531in}}%
\pgfpathlineto{\pgfqpoint{2.785683in}{1.965052in}}%
\pgfpathlineto{\pgfqpoint{2.798761in}{1.967087in}}%
\pgfpathlineto{\pgfqpoint{2.811694in}{1.965970in}}%
\pgfpathlineto{\pgfqpoint{2.822883in}{1.967253in}}%
\pgfpathlineto{\pgfqpoint{2.837560in}{1.965265in}}%
\pgfpathlineto{\pgfqpoint{2.849330in}{1.967178in}}%
\pgfpathlineto{\pgfqpoint{2.861246in}{1.966968in}}%
\pgfpathlineto{\pgfqpoint{2.876213in}{1.970258in}}%
\pgfpathlineto{\pgfqpoint{2.890744in}{1.970196in}}%
\pgfpathlineto{\pgfqpoint{2.901788in}{1.970017in}}%
\pgfpathlineto{\pgfqpoint{2.921405in}{1.968108in}}%
\pgfpathlineto{\pgfqpoint{2.953955in}{1.972246in}}%
\pgfpathlineto{\pgfqpoint{2.961511in}{1.972609in}}%
\pgfpathlineto{\pgfqpoint{2.973718in}{1.972554in}}%
\pgfpathlineto{\pgfqpoint{2.983744in}{1.973336in}}%
\pgfpathlineto{\pgfqpoint{3.000164in}{1.970826in}}%
\pgfpathlineto{\pgfqpoint{3.004814in}{1.972082in}}%
\pgfpathlineto{\pgfqpoint{3.004814in}{1.972082in}}%
\pgfusepath{stroke}%
\end{pgfscope}%
\begin{pgfscope}%
\pgfpathrectangle{\pgfqpoint{0.679669in}{0.526079in}}{\pgfqpoint{2.325000in}{1.661000in}} %
\pgfusepath{clip}%
\pgfsetbuttcap%
\pgfsetroundjoin%
\pgfsetlinewidth{1.003750pt}%
\definecolor{currentstroke}{rgb}{0.000000,0.000000,0.000000}%
\pgfsetstrokecolor{currentstroke}%
\pgfsetdash{{3.700000pt}{1.600000pt}}{0.000000pt}%
\pgfpathmoveto{\pgfqpoint{0.679669in}{0.934996in}}%
\pgfpathlineto{\pgfqpoint{0.693760in}{0.948612in}}%
\pgfpathlineto{\pgfqpoint{0.707851in}{0.962228in}}%
\pgfpathlineto{\pgfqpoint{0.721942in}{0.975844in}}%
\pgfpathlineto{\pgfqpoint{0.736033in}{0.989460in}}%
\pgfpathlineto{\pgfqpoint{0.750124in}{1.003076in}}%
\pgfpathlineto{\pgfqpoint{0.764215in}{1.016693in}}%
\pgfpathlineto{\pgfqpoint{0.778306in}{1.030309in}}%
\pgfpathlineto{\pgfqpoint{0.792396in}{1.043925in}}%
\pgfpathlineto{\pgfqpoint{0.806487in}{1.057541in}}%
\pgfpathlineto{\pgfqpoint{0.820578in}{1.071157in}}%
\pgfpathlineto{\pgfqpoint{0.834669in}{1.084773in}}%
\pgfpathlineto{\pgfqpoint{0.848760in}{1.098389in}}%
\pgfpathlineto{\pgfqpoint{0.862851in}{1.112005in}}%
\pgfpathlineto{\pgfqpoint{0.876942in}{1.125622in}}%
\pgfpathlineto{\pgfqpoint{0.891033in}{1.139238in}}%
\pgfpathlineto{\pgfqpoint{0.905124in}{1.152854in}}%
\pgfpathlineto{\pgfqpoint{0.919215in}{1.166470in}}%
\pgfpathlineto{\pgfqpoint{0.933306in}{1.180086in}}%
\pgfpathlineto{\pgfqpoint{0.947396in}{1.193702in}}%
\pgfpathlineto{\pgfqpoint{0.961487in}{1.207318in}}%
\pgfpathlineto{\pgfqpoint{0.975578in}{1.220934in}}%
\pgfpathlineto{\pgfqpoint{0.989669in}{1.234551in}}%
\pgfpathlineto{\pgfqpoint{1.003760in}{1.248167in}}%
\pgfpathlineto{\pgfqpoint{1.017851in}{1.261783in}}%
\pgfpathlineto{\pgfqpoint{1.031942in}{1.275399in}}%
\pgfpathlineto{\pgfqpoint{1.046033in}{1.289015in}}%
\pgfpathlineto{\pgfqpoint{1.060124in}{1.302631in}}%
\pgfpathlineto{\pgfqpoint{1.074215in}{1.316247in}}%
\pgfpathlineto{\pgfqpoint{1.088306in}{1.329863in}}%
\pgfpathlineto{\pgfqpoint{1.102396in}{1.343480in}}%
\pgfpathlineto{\pgfqpoint{1.116487in}{1.357096in}}%
\pgfpathlineto{\pgfqpoint{1.130578in}{1.370712in}}%
\pgfpathlineto{\pgfqpoint{1.144669in}{1.384328in}}%
\pgfpathlineto{\pgfqpoint{1.158760in}{1.397944in}}%
\pgfpathlineto{\pgfqpoint{1.172851in}{1.411560in}}%
\pgfpathlineto{\pgfqpoint{1.186942in}{1.425176in}}%
\pgfpathlineto{\pgfqpoint{1.201033in}{1.438792in}}%
\pgfpathlineto{\pgfqpoint{1.215124in}{1.452409in}}%
\pgfpathlineto{\pgfqpoint{1.229215in}{1.466025in}}%
\pgfpathlineto{\pgfqpoint{1.243306in}{1.479641in}}%
\pgfpathlineto{\pgfqpoint{1.257396in}{1.493257in}}%
\pgfpathlineto{\pgfqpoint{1.271487in}{1.506873in}}%
\pgfpathlineto{\pgfqpoint{1.285578in}{1.520489in}}%
\pgfpathlineto{\pgfqpoint{1.299669in}{1.534105in}}%
\pgfpathlineto{\pgfqpoint{1.313760in}{1.547721in}}%
\pgfpathlineto{\pgfqpoint{1.327851in}{1.561338in}}%
\pgfpathlineto{\pgfqpoint{1.341942in}{1.574954in}}%
\pgfpathlineto{\pgfqpoint{1.356033in}{1.588570in}}%
\pgfpathlineto{\pgfqpoint{1.370124in}{1.602186in}}%
\pgfpathlineto{\pgfqpoint{1.384215in}{1.615802in}}%
\pgfpathlineto{\pgfqpoint{1.398306in}{1.629418in}}%
\pgfpathlineto{\pgfqpoint{1.412396in}{1.643034in}}%
\pgfpathlineto{\pgfqpoint{1.426487in}{1.656650in}}%
\pgfpathlineto{\pgfqpoint{1.440578in}{1.670267in}}%
\pgfpathlineto{\pgfqpoint{1.454669in}{1.683883in}}%
\pgfpathlineto{\pgfqpoint{1.468760in}{1.697499in}}%
\pgfpathlineto{\pgfqpoint{1.482851in}{1.711115in}}%
\pgfpathlineto{\pgfqpoint{1.496942in}{1.724731in}}%
\pgfpathlineto{\pgfqpoint{1.511033in}{1.738347in}}%
\pgfpathlineto{\pgfqpoint{1.525124in}{1.751963in}}%
\pgfpathlineto{\pgfqpoint{1.539215in}{1.765579in}}%
\pgfpathlineto{\pgfqpoint{1.553306in}{1.779196in}}%
\pgfpathlineto{\pgfqpoint{1.567396in}{1.792812in}}%
\pgfpathlineto{\pgfqpoint{1.581487in}{1.806428in}}%
\pgfpathlineto{\pgfqpoint{1.595578in}{1.820044in}}%
\pgfpathlineto{\pgfqpoint{1.609669in}{1.833660in}}%
\pgfpathlineto{\pgfqpoint{1.623760in}{1.847276in}}%
\pgfpathlineto{\pgfqpoint{1.637851in}{1.860892in}}%
\pgfpathlineto{\pgfqpoint{1.651942in}{1.874508in}}%
\pgfpathlineto{\pgfqpoint{1.666033in}{1.888125in}}%
\pgfpathlineto{\pgfqpoint{1.680124in}{1.901741in}}%
\pgfpathlineto{\pgfqpoint{1.694215in}{1.915357in}}%
\pgfpathlineto{\pgfqpoint{1.708306in}{1.928973in}}%
\pgfpathlineto{\pgfqpoint{1.722396in}{1.942589in}}%
\pgfpathlineto{\pgfqpoint{1.736487in}{1.956205in}}%
\pgfpathlineto{\pgfqpoint{1.750578in}{1.969821in}}%
\pgfpathlineto{\pgfqpoint{1.764669in}{1.983437in}}%
\pgfpathlineto{\pgfqpoint{1.778760in}{1.997054in}}%
\pgfpathlineto{\pgfqpoint{1.792851in}{2.010670in}}%
\pgfpathlineto{\pgfqpoint{1.806942in}{2.024286in}}%
\pgfpathlineto{\pgfqpoint{1.821033in}{2.037902in}}%
\pgfpathlineto{\pgfqpoint{1.835124in}{2.051518in}}%
\pgfpathlineto{\pgfqpoint{1.849215in}{2.065134in}}%
\pgfpathlineto{\pgfqpoint{1.863306in}{2.078750in}}%
\pgfpathlineto{\pgfqpoint{1.877396in}{2.092366in}}%
\pgfpathlineto{\pgfqpoint{1.891487in}{2.105983in}}%
\pgfpathlineto{\pgfqpoint{1.905578in}{2.119599in}}%
\pgfpathlineto{\pgfqpoint{1.919669in}{2.133215in}}%
\pgfpathlineto{\pgfqpoint{1.933760in}{2.146831in}}%
\pgfpathlineto{\pgfqpoint{1.947851in}{2.160447in}}%
\pgfpathlineto{\pgfqpoint{1.961942in}{2.174063in}}%
\pgfpathlineto{\pgfqpoint{1.976033in}{2.187679in}}%
\pgfpathlineto{\pgfqpoint{1.989785in}{2.200968in}}%
\pgfusepath{stroke}%
\end{pgfscope}%
\begin{pgfscope}%
\pgfsetrectcap%
\pgfsetmiterjoin%
\pgfsetlinewidth{0.803000pt}%
\definecolor{currentstroke}{rgb}{0.000000,0.000000,0.000000}%
\pgfsetstrokecolor{currentstroke}%
\pgfsetdash{}{0pt}%
\pgfpathmoveto{\pgfqpoint{0.679669in}{0.526079in}}%
\pgfpathlineto{\pgfqpoint{0.679669in}{2.187079in}}%
\pgfusepath{stroke}%
\end{pgfscope}%
\begin{pgfscope}%
\pgfsetrectcap%
\pgfsetmiterjoin%
\pgfsetlinewidth{0.803000pt}%
\definecolor{currentstroke}{rgb}{0.000000,0.000000,0.000000}%
\pgfsetstrokecolor{currentstroke}%
\pgfsetdash{}{0pt}%
\pgfpathmoveto{\pgfqpoint{3.004669in}{0.526079in}}%
\pgfpathlineto{\pgfqpoint{3.004669in}{2.187079in}}%
\pgfusepath{stroke}%
\end{pgfscope}%
\begin{pgfscope}%
\pgfsetrectcap%
\pgfsetmiterjoin%
\pgfsetlinewidth{0.803000pt}%
\definecolor{currentstroke}{rgb}{0.000000,0.000000,0.000000}%
\pgfsetstrokecolor{currentstroke}%
\pgfsetdash{}{0pt}%
\pgfpathmoveto{\pgfqpoint{0.679669in}{0.526079in}}%
\pgfpathlineto{\pgfqpoint{3.004669in}{0.526079in}}%
\pgfusepath{stroke}%
\end{pgfscope}%
\begin{pgfscope}%
\pgfsetrectcap%
\pgfsetmiterjoin%
\pgfsetlinewidth{0.803000pt}%
\definecolor{currentstroke}{rgb}{0.000000,0.000000,0.000000}%
\pgfsetstrokecolor{currentstroke}%
\pgfsetdash{}{0pt}%
\pgfpathmoveto{\pgfqpoint{0.679669in}{2.187079in}}%
\pgfpathlineto{\pgfqpoint{3.004669in}{2.187079in}}%
\pgfusepath{stroke}%
\end{pgfscope}%
\begin{pgfscope}%
\pgfsetbuttcap%
\pgfsetmiterjoin%
\definecolor{currentfill}{rgb}{1.000000,1.000000,1.000000}%
\pgfsetfillcolor{currentfill}%
\pgfsetfillopacity{0.800000}%
\pgfsetlinewidth{1.003750pt}%
\definecolor{currentstroke}{rgb}{0.800000,0.800000,0.800000}%
\pgfsetstrokecolor{currentstroke}%
\pgfsetstrokeopacity{0.800000}%
\pgfsetdash{}{0pt}%
\pgfpathmoveto{\pgfqpoint{1.124371in}{0.595524in}}%
\pgfpathlineto{\pgfqpoint{2.907447in}{0.595524in}}%
\pgfpathquadraticcurveto{\pgfqpoint{2.935225in}{0.595524in}}{\pgfqpoint{2.935225in}{0.623302in}}%
\pgfpathlineto{\pgfqpoint{2.935225in}{1.019094in}}%
\pgfpathquadraticcurveto{\pgfqpoint{2.935225in}{1.046872in}}{\pgfqpoint{2.907447in}{1.046872in}}%
\pgfpathlineto{\pgfqpoint{1.124371in}{1.046872in}}%
\pgfpathquadraticcurveto{\pgfqpoint{1.096593in}{1.046872in}}{\pgfqpoint{1.096593in}{1.019094in}}%
\pgfpathlineto{\pgfqpoint{1.096593in}{0.623302in}}%
\pgfpathquadraticcurveto{\pgfqpoint{1.096593in}{0.595524in}}{\pgfqpoint{1.124371in}{0.595524in}}%
\pgfpathclose%
\pgfusepath{stroke,fill}%
\end{pgfscope}%
\begin{pgfscope}%
\pgfsetrectcap%
\pgfsetroundjoin%
\pgfsetlinewidth{1.003750pt}%
\definecolor{currentstroke}{rgb}{1.000000,0.549020,0.000000}%
\pgfsetstrokecolor{currentstroke}%
\pgfsetdash{}{0pt}%
\pgfpathmoveto{\pgfqpoint{1.152148in}{0.934404in}}%
\pgfpathlineto{\pgfqpoint{1.429926in}{0.934404in}}%
\pgfusepath{stroke}%
\end{pgfscope}%
\begin{pgfscope}%
\pgftext[x=1.541037in,y=0.885793in,left,base]{\rmfamily\fontsize{10.000000}{12.000000}\selectfont \(\displaystyle \mathcal{E}_{B}\)}%
\end{pgfscope}%
\begin{pgfscope}%
\pgfsetrectcap%
\pgfsetroundjoin%
\pgfsetlinewidth{1.003750pt}%
\definecolor{currentstroke}{rgb}{0.501961,0.000000,0.501961}%
\pgfsetstrokecolor{currentstroke}%
\pgfsetdash{}{0pt}%
\pgfpathmoveto{\pgfqpoint{1.152148in}{0.730547in}}%
\pgfpathlineto{\pgfqpoint{1.429926in}{0.730547in}}%
\pgfusepath{stroke}%
\end{pgfscope}%
\begin{pgfscope}%
\pgftext[x=1.541037in,y=0.681936in,left,base]{\rmfamily\fontsize{10.000000}{12.000000}\selectfont \(\displaystyle \mathcal{E}_{E}\)}%
\end{pgfscope}%
\begin{pgfscope}%
\pgfsetbuttcap%
\pgfsetroundjoin%
\pgfsetlinewidth{1.003750pt}%
\definecolor{currentstroke}{rgb}{0.627451,0.321569,0.176471}%
\pgfsetstrokecolor{currentstroke}%
\pgfsetdash{{3.700000pt}{1.600000pt}}{0.000000pt}%
\pgfpathmoveto{\pgfqpoint{1.987715in}{0.934404in}}%
\pgfpathlineto{\pgfqpoint{2.265493in}{0.934404in}}%
\pgfusepath{stroke}%
\end{pgfscope}%
\begin{pgfscope}%
\pgftext[x=2.376604in,y=0.885793in,left,base]{\rmfamily\fontsize{10.000000}{12.000000}\selectfont \(\displaystyle \mathcal{E}_\mathrm{c}\)}%
\end{pgfscope}%
\begin{pgfscope}%
\pgfsetbuttcap%
\pgfsetroundjoin%
\pgfsetlinewidth{1.003750pt}%
\definecolor{currentstroke}{rgb}{0.000000,0.000000,0.000000}%
\pgfsetstrokecolor{currentstroke}%
\pgfsetdash{{3.700000pt}{1.600000pt}}{0.000000pt}%
\pgfpathmoveto{\pgfqpoint{1.987715in}{0.730547in}}%
\pgfpathlineto{\pgfqpoint{2.265493in}{0.730547in}}%
\pgfusepath{stroke}%
\end{pgfscope}%
\begin{pgfscope}%
\pgftext[x=2.376604in,y=0.681936in,left,base]{\rmfamily\fontsize{10.000000}{12.000000}\selectfont growth}%
\end{pgfscope}%
\end{pgfpicture}%
\makeatother%
\endgroup%
}
\caption{(a) Time evolution of energies for parameters given in tab. \ref{tab_parameters} obtained with standard finite element particle-in-cell methods explained in section \ref{sec_standard} together with the analytical growth rate. (b) Same for structrue-preserving finite element particle-in-cell methods explained in section \ref{sec_geometric} with the Lie-Trotter splitting (\ref{eq_LieTrotter}).\label{fig_energies}}
\end{figure}

With the choice of parameters in tab. \ref{tab_parameters}, the numerical solution of the dispersion relation (\ref{eq_dispersion_relation}) yields an expected growth rate of $\gamma\approx0.0447|\Omega_\mr{ce}|$. In fig. \ref{fig_energies}, we plot the resulting time evolution of the magnetic field energy $\mathcal{E}_B$, the electric field energy $\mathcal{E}_E$ and the cold plasma energy $\mathcal{E}_\mr{c}$ (see (\ref{eq_discrete_Hamiltonian})) normalized to the total energy $\mathcal{E}=\mathcal{E}_B+\mathcal{E}_E+\mathcal{E}_\mr{c}+\mathcal{E}_\mr{h}$ together with the expected growth rate (which is $2\gamma$ in the case of energies). Note that most of the energy is carried by the energetic electrons which is why $\mathcal{E}_\mr{h}$ would be orders of magnitude above the other curves in fig. \ref{fig_energies}. Qualitatively, we observe a similar behavior for the two codes: First, as expected, all quantities grow linearly, i.e. energy is transfered from the fast electrons to the electromagnetic field and the cold plasma. After this, the wave fields saturate, when nonlinear terms start to play a role and the linear theory thus breaks down. In both cases, the numerical growth matches the analytical one very well and the curves and up at the same saturation level. However, the standard FEM code seems to be more sensitive to the noise induced by the random particle initialization, since it takes some time in the beginning until linear growth phase is reached (obvious for the electric field energy). 

Finally, we check the conservation of the total energy in the system and plot in fig. \ref{fig_comparison} its relative error with respect to time for the standard and structure-preserving code with Lie-Trotter (\ref{eq_LieTrotter}) and Strang splitting (\ref{eq_Strang}), respectively. We find that there is an increase of the error of about three orders of magnitude for the first case already in the liner phase at $t\approx40|\Omega_\mr{ce}|$, whereas the error is bounded in case of the Lie-Trotter splitting for the whole simulation time, even in the nonlinear phase. The usage of a second-order method (Strang) instead of a first-order method (Lie-Trotter) leads to an error reduction of about three orders of magnitude and to a similar behavior up to $t\approx120|\Omega_\mr{ce}|$, i.e. the error does not increase. Subsequently, we observe a slight increase of the error, however, it remains two orders of magnitude below the error of the standard methods.


\begin{figure}[!t]
\centering
%% Creator: Matplotlib, PGF backend
%%
%% To include the figure in your LaTeX document, write
%%   \input{<filename>.pgf}
%%
%% Make sure the required packages are loaded in your preamble
%%   \usepackage{pgf}
%%
%% Figures using additional raster images can only be included by \input if
%% they are in the same directory as the main LaTeX file. For loading figures
%% from other directories you can use the `import` package
%%   \usepackage{import}
%% and then include the figures with
%%   \import{<path to file>}{<filename>.pgf}
%%
%% Matplotlib used the following preamble
%%   \usepackage{fontspec}
%%   \setmainfont{DejaVu Serif}
%%   \setsansfont{DejaVu Sans}
%%   \setmonofont{DejaVu Sans Mono}
%%
\begingroup%
\makeatletter%
\begin{pgfpicture}%
\pgfpathrectangle{\pgfpointorigin}{\pgfqpoint{5.274168in}{2.339841in}}%
\pgfusepath{use as bounding box, clip}%
\begin{pgfscope}%
\pgfsetbuttcap%
\pgfsetmiterjoin%
\definecolor{currentfill}{rgb}{1.000000,1.000000,1.000000}%
\pgfsetfillcolor{currentfill}%
\pgfsetlinewidth{0.000000pt}%
\definecolor{currentstroke}{rgb}{1.000000,1.000000,1.000000}%
\pgfsetstrokecolor{currentstroke}%
\pgfsetdash{}{0pt}%
\pgfpathmoveto{\pgfqpoint{0.000000in}{0.000000in}}%
\pgfpathlineto{\pgfqpoint{5.274168in}{0.000000in}}%
\pgfpathlineto{\pgfqpoint{5.274168in}{2.339841in}}%
\pgfpathlineto{\pgfqpoint{0.000000in}{2.339841in}}%
\pgfpathclose%
\pgfusepath{fill}%
\end{pgfscope}%
\begin{pgfscope}%
\pgfsetbuttcap%
\pgfsetmiterjoin%
\definecolor{currentfill}{rgb}{1.000000,1.000000,1.000000}%
\pgfsetfillcolor{currentfill}%
\pgfsetlinewidth{0.000000pt}%
\definecolor{currentstroke}{rgb}{0.000000,0.000000,0.000000}%
\pgfsetstrokecolor{currentstroke}%
\pgfsetstrokeopacity{0.000000}%
\pgfsetdash{}{0pt}%
\pgfpathmoveto{\pgfqpoint{0.735032in}{0.526079in}}%
\pgfpathlineto{\pgfqpoint{2.905032in}{0.526079in}}%
\pgfpathlineto{\pgfqpoint{2.905032in}{2.187079in}}%
\pgfpathlineto{\pgfqpoint{0.735032in}{2.187079in}}%
\pgfpathclose%
\pgfusepath{fill}%
\end{pgfscope}%
\begin{pgfscope}%
\pgfsetbuttcap%
\pgfsetroundjoin%
\definecolor{currentfill}{rgb}{0.000000,0.000000,0.000000}%
\pgfsetfillcolor{currentfill}%
\pgfsetlinewidth{0.803000pt}%
\definecolor{currentstroke}{rgb}{0.000000,0.000000,0.000000}%
\pgfsetstrokecolor{currentstroke}%
\pgfsetdash{}{0pt}%
\pgfsys@defobject{currentmarker}{\pgfqpoint{0.000000in}{-0.048611in}}{\pgfqpoint{0.000000in}{0.000000in}}{%
\pgfpathmoveto{\pgfqpoint{0.000000in}{0.000000in}}%
\pgfpathlineto{\pgfqpoint{0.000000in}{-0.048611in}}%
\pgfusepath{stroke,fill}%
}%
\begin{pgfscope}%
\pgfsys@transformshift{0.735032in}{0.526079in}%
\pgfsys@useobject{currentmarker}{}%
\end{pgfscope}%
\end{pgfscope}%
\begin{pgfscope}%
\pgftext[x=0.735032in,y=0.428857in,,top]{\rmfamily\fontsize{10.000000}{12.000000}\selectfont \(\displaystyle 0\)}%
\end{pgfscope}%
\begin{pgfscope}%
\pgfsetbuttcap%
\pgfsetroundjoin%
\definecolor{currentfill}{rgb}{0.000000,0.000000,0.000000}%
\pgfsetfillcolor{currentfill}%
\pgfsetlinewidth{0.803000pt}%
\definecolor{currentstroke}{rgb}{0.000000,0.000000,0.000000}%
\pgfsetstrokecolor{currentstroke}%
\pgfsetdash{}{0pt}%
\pgfsys@defobject{currentmarker}{\pgfqpoint{0.000000in}{-0.048611in}}{\pgfqpoint{0.000000in}{0.000000in}}{%
\pgfpathmoveto{\pgfqpoint{0.000000in}{0.000000in}}%
\pgfpathlineto{\pgfqpoint{0.000000in}{-0.048611in}}%
\pgfusepath{stroke,fill}%
}%
\begin{pgfscope}%
\pgfsys@transformshift{1.277532in}{0.526079in}%
\pgfsys@useobject{currentmarker}{}%
\end{pgfscope}%
\end{pgfscope}%
\begin{pgfscope}%
\pgftext[x=1.277532in,y=0.428857in,,top]{\rmfamily\fontsize{10.000000}{12.000000}\selectfont \(\displaystyle 50\)}%
\end{pgfscope}%
\begin{pgfscope}%
\pgfsetbuttcap%
\pgfsetroundjoin%
\definecolor{currentfill}{rgb}{0.000000,0.000000,0.000000}%
\pgfsetfillcolor{currentfill}%
\pgfsetlinewidth{0.803000pt}%
\definecolor{currentstroke}{rgb}{0.000000,0.000000,0.000000}%
\pgfsetstrokecolor{currentstroke}%
\pgfsetdash{}{0pt}%
\pgfsys@defobject{currentmarker}{\pgfqpoint{0.000000in}{-0.048611in}}{\pgfqpoint{0.000000in}{0.000000in}}{%
\pgfpathmoveto{\pgfqpoint{0.000000in}{0.000000in}}%
\pgfpathlineto{\pgfqpoint{0.000000in}{-0.048611in}}%
\pgfusepath{stroke,fill}%
}%
\begin{pgfscope}%
\pgfsys@transformshift{1.820032in}{0.526079in}%
\pgfsys@useobject{currentmarker}{}%
\end{pgfscope}%
\end{pgfscope}%
\begin{pgfscope}%
\pgftext[x=1.820032in,y=0.428857in,,top]{\rmfamily\fontsize{10.000000}{12.000000}\selectfont \(\displaystyle 100\)}%
\end{pgfscope}%
\begin{pgfscope}%
\pgfsetbuttcap%
\pgfsetroundjoin%
\definecolor{currentfill}{rgb}{0.000000,0.000000,0.000000}%
\pgfsetfillcolor{currentfill}%
\pgfsetlinewidth{0.803000pt}%
\definecolor{currentstroke}{rgb}{0.000000,0.000000,0.000000}%
\pgfsetstrokecolor{currentstroke}%
\pgfsetdash{}{0pt}%
\pgfsys@defobject{currentmarker}{\pgfqpoint{0.000000in}{-0.048611in}}{\pgfqpoint{0.000000in}{0.000000in}}{%
\pgfpathmoveto{\pgfqpoint{0.000000in}{0.000000in}}%
\pgfpathlineto{\pgfqpoint{0.000000in}{-0.048611in}}%
\pgfusepath{stroke,fill}%
}%
\begin{pgfscope}%
\pgfsys@transformshift{2.362532in}{0.526079in}%
\pgfsys@useobject{currentmarker}{}%
\end{pgfscope}%
\end{pgfscope}%
\begin{pgfscope}%
\pgftext[x=2.362532in,y=0.428857in,,top]{\rmfamily\fontsize{10.000000}{12.000000}\selectfont \(\displaystyle 150\)}%
\end{pgfscope}%
\begin{pgfscope}%
\pgfsetbuttcap%
\pgfsetroundjoin%
\definecolor{currentfill}{rgb}{0.000000,0.000000,0.000000}%
\pgfsetfillcolor{currentfill}%
\pgfsetlinewidth{0.803000pt}%
\definecolor{currentstroke}{rgb}{0.000000,0.000000,0.000000}%
\pgfsetstrokecolor{currentstroke}%
\pgfsetdash{}{0pt}%
\pgfsys@defobject{currentmarker}{\pgfqpoint{0.000000in}{-0.048611in}}{\pgfqpoint{0.000000in}{0.000000in}}{%
\pgfpathmoveto{\pgfqpoint{0.000000in}{0.000000in}}%
\pgfpathlineto{\pgfqpoint{0.000000in}{-0.048611in}}%
\pgfusepath{stroke,fill}%
}%
\begin{pgfscope}%
\pgfsys@transformshift{2.905032in}{0.526079in}%
\pgfsys@useobject{currentmarker}{}%
\end{pgfscope}%
\end{pgfscope}%
\begin{pgfscope}%
\pgftext[x=2.905032in,y=0.428857in,,top]{\rmfamily\fontsize{10.000000}{12.000000}\selectfont \(\displaystyle 200\)}%
\end{pgfscope}%
\begin{pgfscope}%
\pgftext[x=1.820032in,y=0.238889in,,top]{\rmfamily\fontsize{10.000000}{12.000000}\selectfont \(\displaystyle t|\Omega_\mathrm{ce}|\)}%
\end{pgfscope}%
\begin{pgfscope}%
\pgfsetbuttcap%
\pgfsetroundjoin%
\definecolor{currentfill}{rgb}{0.000000,0.000000,0.000000}%
\pgfsetfillcolor{currentfill}%
\pgfsetlinewidth{0.803000pt}%
\definecolor{currentstroke}{rgb}{0.000000,0.000000,0.000000}%
\pgfsetstrokecolor{currentstroke}%
\pgfsetdash{}{0pt}%
\pgfsys@defobject{currentmarker}{\pgfqpoint{-0.048611in}{0.000000in}}{\pgfqpoint{0.000000in}{0.000000in}}{%
\pgfpathmoveto{\pgfqpoint{0.000000in}{0.000000in}}%
\pgfpathlineto{\pgfqpoint{-0.048611in}{0.000000in}}%
\pgfusepath{stroke,fill}%
}%
\begin{pgfscope}%
\pgfsys@transformshift{0.735032in}{0.526079in}%
\pgfsys@useobject{currentmarker}{}%
\end{pgfscope}%
\end{pgfscope}%
\begin{pgfscope}%
\pgftext[x=0.294444in,y=0.473318in,left,base]{\rmfamily\fontsize{10.000000}{12.000000}\selectfont \(\displaystyle 10^{-12}\)}%
\end{pgfscope}%
\begin{pgfscope}%
\pgfsetbuttcap%
\pgfsetroundjoin%
\definecolor{currentfill}{rgb}{0.000000,0.000000,0.000000}%
\pgfsetfillcolor{currentfill}%
\pgfsetlinewidth{0.803000pt}%
\definecolor{currentstroke}{rgb}{0.000000,0.000000,0.000000}%
\pgfsetstrokecolor{currentstroke}%
\pgfsetdash{}{0pt}%
\pgfsys@defobject{currentmarker}{\pgfqpoint{-0.048611in}{0.000000in}}{\pgfqpoint{0.000000in}{0.000000in}}{%
\pgfpathmoveto{\pgfqpoint{0.000000in}{0.000000in}}%
\pgfpathlineto{\pgfqpoint{-0.048611in}{0.000000in}}%
\pgfusepath{stroke,fill}%
}%
\begin{pgfscope}%
\pgfsys@transformshift{0.735032in}{0.858279in}%
\pgfsys@useobject{currentmarker}{}%
\end{pgfscope}%
\end{pgfscope}%
\begin{pgfscope}%
\pgftext[x=0.294444in,y=0.805518in,left,base]{\rmfamily\fontsize{10.000000}{12.000000}\selectfont \(\displaystyle 10^{-10}\)}%
\end{pgfscope}%
\begin{pgfscope}%
\pgfsetbuttcap%
\pgfsetroundjoin%
\definecolor{currentfill}{rgb}{0.000000,0.000000,0.000000}%
\pgfsetfillcolor{currentfill}%
\pgfsetlinewidth{0.803000pt}%
\definecolor{currentstroke}{rgb}{0.000000,0.000000,0.000000}%
\pgfsetstrokecolor{currentstroke}%
\pgfsetdash{}{0pt}%
\pgfsys@defobject{currentmarker}{\pgfqpoint{-0.048611in}{0.000000in}}{\pgfqpoint{0.000000in}{0.000000in}}{%
\pgfpathmoveto{\pgfqpoint{0.000000in}{0.000000in}}%
\pgfpathlineto{\pgfqpoint{-0.048611in}{0.000000in}}%
\pgfusepath{stroke,fill}%
}%
\begin{pgfscope}%
\pgfsys@transformshift{0.735032in}{1.190479in}%
\pgfsys@useobject{currentmarker}{}%
\end{pgfscope}%
\end{pgfscope}%
\begin{pgfscope}%
\pgftext[x=0.349807in,y=1.137718in,left,base]{\rmfamily\fontsize{10.000000}{12.000000}\selectfont \(\displaystyle 10^{-8}\)}%
\end{pgfscope}%
\begin{pgfscope}%
\pgfsetbuttcap%
\pgfsetroundjoin%
\definecolor{currentfill}{rgb}{0.000000,0.000000,0.000000}%
\pgfsetfillcolor{currentfill}%
\pgfsetlinewidth{0.803000pt}%
\definecolor{currentstroke}{rgb}{0.000000,0.000000,0.000000}%
\pgfsetstrokecolor{currentstroke}%
\pgfsetdash{}{0pt}%
\pgfsys@defobject{currentmarker}{\pgfqpoint{-0.048611in}{0.000000in}}{\pgfqpoint{0.000000in}{0.000000in}}{%
\pgfpathmoveto{\pgfqpoint{0.000000in}{0.000000in}}%
\pgfpathlineto{\pgfqpoint{-0.048611in}{0.000000in}}%
\pgfusepath{stroke,fill}%
}%
\begin{pgfscope}%
\pgfsys@transformshift{0.735032in}{1.522679in}%
\pgfsys@useobject{currentmarker}{}%
\end{pgfscope}%
\end{pgfscope}%
\begin{pgfscope}%
\pgftext[x=0.349807in,y=1.469918in,left,base]{\rmfamily\fontsize{10.000000}{12.000000}\selectfont \(\displaystyle 10^{-6}\)}%
\end{pgfscope}%
\begin{pgfscope}%
\pgfsetbuttcap%
\pgfsetroundjoin%
\definecolor{currentfill}{rgb}{0.000000,0.000000,0.000000}%
\pgfsetfillcolor{currentfill}%
\pgfsetlinewidth{0.803000pt}%
\definecolor{currentstroke}{rgb}{0.000000,0.000000,0.000000}%
\pgfsetstrokecolor{currentstroke}%
\pgfsetdash{}{0pt}%
\pgfsys@defobject{currentmarker}{\pgfqpoint{-0.048611in}{0.000000in}}{\pgfqpoint{0.000000in}{0.000000in}}{%
\pgfpathmoveto{\pgfqpoint{0.000000in}{0.000000in}}%
\pgfpathlineto{\pgfqpoint{-0.048611in}{0.000000in}}%
\pgfusepath{stroke,fill}%
}%
\begin{pgfscope}%
\pgfsys@transformshift{0.735032in}{1.854879in}%
\pgfsys@useobject{currentmarker}{}%
\end{pgfscope}%
\end{pgfscope}%
\begin{pgfscope}%
\pgftext[x=0.349807in,y=1.802118in,left,base]{\rmfamily\fontsize{10.000000}{12.000000}\selectfont \(\displaystyle 10^{-4}\)}%
\end{pgfscope}%
\begin{pgfscope}%
\pgfsetbuttcap%
\pgfsetroundjoin%
\definecolor{currentfill}{rgb}{0.000000,0.000000,0.000000}%
\pgfsetfillcolor{currentfill}%
\pgfsetlinewidth{0.803000pt}%
\definecolor{currentstroke}{rgb}{0.000000,0.000000,0.000000}%
\pgfsetstrokecolor{currentstroke}%
\pgfsetdash{}{0pt}%
\pgfsys@defobject{currentmarker}{\pgfqpoint{-0.048611in}{0.000000in}}{\pgfqpoint{0.000000in}{0.000000in}}{%
\pgfpathmoveto{\pgfqpoint{0.000000in}{0.000000in}}%
\pgfpathlineto{\pgfqpoint{-0.048611in}{0.000000in}}%
\pgfusepath{stroke,fill}%
}%
\begin{pgfscope}%
\pgfsys@transformshift{0.735032in}{2.187079in}%
\pgfsys@useobject{currentmarker}{}%
\end{pgfscope}%
\end{pgfscope}%
\begin{pgfscope}%
\pgftext[x=0.349807in,y=2.134318in,left,base]{\rmfamily\fontsize{10.000000}{12.000000}\selectfont \(\displaystyle 10^{-2}\)}%
\end{pgfscope}%
\begin{pgfscope}%
\pgftext[x=0.238889in,y=1.356579in,,bottom,rotate=90.000000]{\rmfamily\fontsize{10.000000}{12.000000}\selectfont \(\displaystyle |\mathcal{E}(t) - \mathcal{E}(0)|/\mathcal{E}(0)\)}%
\end{pgfscope}%
\begin{pgfscope}%
\pgfpathrectangle{\pgfqpoint{0.735032in}{0.526079in}}{\pgfqpoint{2.170000in}{1.661000in}} %
\pgfusepath{clip}%
\pgfsetrectcap%
\pgfsetroundjoin%
\pgfsetlinewidth{1.003750pt}%
\definecolor{currentstroke}{rgb}{0.627451,0.321569,0.176471}%
\pgfsetstrokecolor{currentstroke}%
\pgfsetdash{}{0pt}%
\pgfpathmoveto{\pgfqpoint{0.735167in}{0.512191in}}%
\pgfpathlineto{\pgfqpoint{0.736659in}{1.466220in}}%
\pgfpathlineto{\pgfqpoint{0.739236in}{1.674852in}}%
\pgfpathlineto{\pgfqpoint{0.743305in}{1.776080in}}%
\pgfpathlineto{\pgfqpoint{0.747916in}{1.824291in}}%
\pgfpathlineto{\pgfqpoint{0.751714in}{1.838166in}}%
\pgfpathlineto{\pgfqpoint{0.753477in}{1.838355in}}%
\pgfpathlineto{\pgfqpoint{0.753613in}{1.838211in}}%
\pgfpathlineto{\pgfqpoint{0.755647in}{1.833274in}}%
\pgfpathlineto{\pgfqpoint{0.758631in}{1.815808in}}%
\pgfpathlineto{\pgfqpoint{0.762157in}{1.774587in}}%
\pgfpathlineto{\pgfqpoint{0.765548in}{1.696117in}}%
\pgfpathlineto{\pgfqpoint{0.767989in}{1.564798in}}%
\pgfpathlineto{\pgfqpoint{0.768938in}{1.396353in}}%
\pgfpathlineto{\pgfqpoint{0.770294in}{1.268274in}}%
\pgfpathlineto{\pgfqpoint{0.769752in}{1.422385in}}%
\pgfpathlineto{\pgfqpoint{0.770430in}{1.347722in}}%
\pgfpathlineto{\pgfqpoint{0.772464in}{1.635120in}}%
\pgfpathlineto{\pgfqpoint{0.776262in}{1.756277in}}%
\pgfpathlineto{\pgfqpoint{0.780873in}{1.816211in}}%
\pgfpathlineto{\pgfqpoint{0.784942in}{1.836682in}}%
\pgfpathlineto{\pgfqpoint{0.787112in}{1.838674in}}%
\pgfpathlineto{\pgfqpoint{0.787383in}{1.838515in}}%
\pgfpathlineto{\pgfqpoint{0.789146in}{1.835242in}}%
\pgfpathlineto{\pgfqpoint{0.791859in}{1.822191in}}%
\pgfpathlineto{\pgfqpoint{0.795385in}{1.787245in}}%
\pgfpathlineto{\pgfqpoint{0.798911in}{1.718192in}}%
\pgfpathlineto{\pgfqpoint{0.801624in}{1.601235in}}%
\pgfpathlineto{\pgfqpoint{0.802980in}{1.414016in}}%
\pgfpathlineto{\pgfqpoint{0.803251in}{1.279953in}}%
\pgfpathlineto{\pgfqpoint{0.803794in}{1.422368in}}%
\pgfpathlineto{\pgfqpoint{0.804472in}{1.304187in}}%
\pgfpathlineto{\pgfqpoint{0.806371in}{1.624950in}}%
\pgfpathlineto{\pgfqpoint{0.810033in}{1.749923in}}%
\pgfpathlineto{\pgfqpoint{0.814644in}{1.813577in}}%
\pgfpathlineto{\pgfqpoint{0.818848in}{1.836251in}}%
\pgfpathlineto{\pgfqpoint{0.821154in}{1.838687in}}%
\pgfpathlineto{\pgfqpoint{0.821425in}{1.838543in}}%
\pgfpathlineto{\pgfqpoint{0.823188in}{1.835370in}}%
\pgfpathlineto{\pgfqpoint{0.825901in}{1.822491in}}%
\pgfpathlineto{\pgfqpoint{0.829291in}{1.789658in}}%
\pgfpathlineto{\pgfqpoint{0.832818in}{1.723114in}}%
\pgfpathlineto{\pgfqpoint{0.835666in}{1.604315in}}%
\pgfpathlineto{\pgfqpoint{0.837022in}{1.428917in}}%
\pgfpathlineto{\pgfqpoint{0.837293in}{1.063954in}}%
\pgfpathlineto{\pgfqpoint{0.838649in}{1.377468in}}%
\pgfpathlineto{\pgfqpoint{0.840819in}{1.644631in}}%
\pgfpathlineto{\pgfqpoint{0.844753in}{1.762316in}}%
\pgfpathlineto{\pgfqpoint{0.849364in}{1.818691in}}%
\pgfpathlineto{\pgfqpoint{0.853297in}{1.837071in}}%
\pgfpathlineto{\pgfqpoint{0.855331in}{1.838649in}}%
\pgfpathlineto{\pgfqpoint{0.855603in}{1.838473in}}%
\pgfpathlineto{\pgfqpoint{0.857366in}{1.835100in}}%
\pgfpathlineto{\pgfqpoint{0.860078in}{1.821873in}}%
\pgfpathlineto{\pgfqpoint{0.863604in}{1.786611in}}%
\pgfpathlineto{\pgfqpoint{0.867131in}{1.716901in}}%
\pgfpathlineto{\pgfqpoint{0.869843in}{1.597879in}}%
\pgfpathlineto{\pgfqpoint{0.871199in}{1.394440in}}%
\pgfpathlineto{\pgfqpoint{0.872556in}{1.272233in}}%
\pgfpathlineto{\pgfqpoint{0.872013in}{1.423258in}}%
\pgfpathlineto{\pgfqpoint{0.872691in}{1.344952in}}%
\pgfpathlineto{\pgfqpoint{0.874726in}{1.635171in}}%
\pgfpathlineto{\pgfqpoint{0.878523in}{1.756352in}}%
\pgfpathlineto{\pgfqpoint{0.883134in}{1.816237in}}%
\pgfpathlineto{\pgfqpoint{0.887203in}{1.836685in}}%
\pgfpathlineto{\pgfqpoint{0.889373in}{1.838667in}}%
\pgfpathlineto{\pgfqpoint{0.889644in}{1.838509in}}%
\pgfpathlineto{\pgfqpoint{0.891408in}{1.835230in}}%
\pgfpathlineto{\pgfqpoint{0.894120in}{1.822169in}}%
\pgfpathlineto{\pgfqpoint{0.897511in}{1.789038in}}%
\pgfpathlineto{\pgfqpoint{0.901037in}{1.721872in}}%
\pgfpathlineto{\pgfqpoint{0.903749in}{1.610330in}}%
\pgfpathlineto{\pgfqpoint{0.905241in}{1.413428in}}%
\pgfpathlineto{\pgfqpoint{0.905513in}{1.284777in}}%
\pgfpathlineto{\pgfqpoint{0.906055in}{1.421737in}}%
\pgfpathlineto{\pgfqpoint{0.906733in}{1.304357in}}%
\pgfpathlineto{\pgfqpoint{0.908632in}{1.625073in}}%
\pgfpathlineto{\pgfqpoint{0.912294in}{1.749995in}}%
\pgfpathlineto{\pgfqpoint{0.916905in}{1.813603in}}%
\pgfpathlineto{\pgfqpoint{0.921109in}{1.836251in}}%
\pgfpathlineto{\pgfqpoint{0.923415in}{1.838681in}}%
\pgfpathlineto{\pgfqpoint{0.923686in}{1.838535in}}%
\pgfpathlineto{\pgfqpoint{0.925449in}{1.835360in}}%
\pgfpathlineto{\pgfqpoint{0.928162in}{1.822472in}}%
\pgfpathlineto{\pgfqpoint{0.931688in}{1.787820in}}%
\pgfpathlineto{\pgfqpoint{0.935214in}{1.719392in}}%
\pgfpathlineto{\pgfqpoint{0.937927in}{1.604219in}}%
\pgfpathlineto{\pgfqpoint{0.939283in}{1.428999in}}%
\pgfpathlineto{\pgfqpoint{0.939554in}{1.041060in}}%
\pgfpathlineto{\pgfqpoint{0.940911in}{1.378532in}}%
\pgfpathlineto{\pgfqpoint{0.943081in}{1.644763in}}%
\pgfpathlineto{\pgfqpoint{0.947014in}{1.762380in}}%
\pgfpathlineto{\pgfqpoint{0.951625in}{1.818707in}}%
\pgfpathlineto{\pgfqpoint{0.955558in}{1.837062in}}%
\pgfpathlineto{\pgfqpoint{0.957593in}{1.838640in}}%
\pgfpathlineto{\pgfqpoint{0.957864in}{1.838463in}}%
\pgfpathlineto{\pgfqpoint{0.959763in}{1.834663in}}%
\pgfpathlineto{\pgfqpoint{0.962475in}{1.820909in}}%
\pgfpathlineto{\pgfqpoint{0.966001in}{1.784688in}}%
\pgfpathlineto{\pgfqpoint{0.969528in}{1.712982in}}%
\pgfpathlineto{\pgfqpoint{0.972240in}{1.587872in}}%
\pgfpathlineto{\pgfqpoint{0.973461in}{1.401285in}}%
\pgfpathlineto{\pgfqpoint{0.974817in}{1.192461in}}%
\pgfpathlineto{\pgfqpoint{0.974274in}{1.418349in}}%
\pgfpathlineto{\pgfqpoint{0.974953in}{1.362325in}}%
\pgfpathlineto{\pgfqpoint{0.976987in}{1.635581in}}%
\pgfpathlineto{\pgfqpoint{0.980784in}{1.756429in}}%
\pgfpathlineto{\pgfqpoint{0.985396in}{1.816245in}}%
\pgfpathlineto{\pgfqpoint{0.989464in}{1.836675in}}%
\pgfpathlineto{\pgfqpoint{0.991634in}{1.838653in}}%
\pgfpathlineto{\pgfqpoint{0.991906in}{1.838493in}}%
\pgfpathlineto{\pgfqpoint{0.993669in}{1.835215in}}%
\pgfpathlineto{\pgfqpoint{0.996381in}{1.822150in}}%
\pgfpathlineto{\pgfqpoint{0.999772in}{1.788999in}}%
\pgfpathlineto{\pgfqpoint{1.003027in}{1.725408in}}%
\pgfpathlineto{\pgfqpoint{1.005875in}{1.610499in}}%
\pgfpathlineto{\pgfqpoint{1.007367in}{1.419678in}}%
\pgfpathlineto{\pgfqpoint{1.007638in}{1.231780in}}%
\pgfpathlineto{\pgfqpoint{1.008859in}{1.335114in}}%
\pgfpathlineto{\pgfqpoint{1.010893in}{1.633292in}}%
\pgfpathlineto{\pgfqpoint{1.014691in}{1.755584in}}%
\pgfpathlineto{\pgfqpoint{1.019302in}{1.815886in}}%
\pgfpathlineto{\pgfqpoint{1.023371in}{1.836567in}}%
\pgfpathlineto{\pgfqpoint{1.025541in}{1.838664in}}%
\pgfpathlineto{\pgfqpoint{1.025812in}{1.838518in}}%
\pgfpathlineto{\pgfqpoint{1.027575in}{1.835341in}}%
\pgfpathlineto{\pgfqpoint{1.030288in}{1.822443in}}%
\pgfpathlineto{\pgfqpoint{1.033678in}{1.789578in}}%
\pgfpathlineto{\pgfqpoint{1.037204in}{1.722984in}}%
\pgfpathlineto{\pgfqpoint{1.040053in}{1.604220in}}%
\pgfpathlineto{\pgfqpoint{1.041409in}{1.431028in}}%
\pgfpathlineto{\pgfqpoint{1.041680in}{1.193972in}}%
\pgfpathlineto{\pgfqpoint{1.043036in}{1.387259in}}%
\pgfpathlineto{\pgfqpoint{1.045206in}{1.645113in}}%
\pgfpathlineto{\pgfqpoint{1.049139in}{1.762449in}}%
\pgfpathlineto{\pgfqpoint{1.053751in}{1.818741in}}%
\pgfpathlineto{\pgfqpoint{1.057684in}{1.837073in}}%
\pgfpathlineto{\pgfqpoint{1.059718in}{1.838638in}}%
\pgfpathlineto{\pgfqpoint{1.059989in}{1.838461in}}%
\pgfpathlineto{\pgfqpoint{1.061888in}{1.834648in}}%
\pgfpathlineto{\pgfqpoint{1.064736in}{1.819912in}}%
\pgfpathlineto{\pgfqpoint{1.068263in}{1.782690in}}%
\pgfpathlineto{\pgfqpoint{1.071789in}{1.708816in}}%
\pgfpathlineto{\pgfqpoint{1.074366in}{1.587682in}}%
\pgfpathlineto{\pgfqpoint{1.075586in}{1.398775in}}%
\pgfpathlineto{\pgfqpoint{1.076943in}{1.188558in}}%
\pgfpathlineto{\pgfqpoint{1.076400in}{1.416734in}}%
\pgfpathlineto{\pgfqpoint{1.077078in}{1.364441in}}%
\pgfpathlineto{\pgfqpoint{1.079248in}{1.642862in}}%
\pgfpathlineto{\pgfqpoint{1.083181in}{1.761630in}}%
\pgfpathlineto{\pgfqpoint{1.087793in}{1.818409in}}%
\pgfpathlineto{\pgfqpoint{1.091861in}{1.837258in}}%
\pgfpathlineto{\pgfqpoint{1.093896in}{1.838584in}}%
\pgfpathlineto{\pgfqpoint{1.094167in}{1.838378in}}%
\pgfpathlineto{\pgfqpoint{1.096066in}{1.834349in}}%
\pgfpathlineto{\pgfqpoint{1.098914in}{1.819237in}}%
\pgfpathlineto{\pgfqpoint{1.102440in}{1.781364in}}%
\pgfpathlineto{\pgfqpoint{1.105966in}{1.706027in}}%
\pgfpathlineto{\pgfqpoint{1.108543in}{1.580232in}}%
\pgfpathlineto{\pgfqpoint{1.109764in}{1.344654in}}%
\pgfpathlineto{\pgfqpoint{1.109899in}{1.273687in}}%
\pgfpathlineto{\pgfqpoint{1.110442in}{1.418491in}}%
\pgfpathlineto{\pgfqpoint{1.111120in}{1.329518in}}%
\pgfpathlineto{\pgfqpoint{1.113154in}{1.633244in}}%
\pgfpathlineto{\pgfqpoint{1.116952in}{1.755617in}}%
\pgfpathlineto{\pgfqpoint{1.121563in}{1.815938in}}%
\pgfpathlineto{\pgfqpoint{1.125632in}{1.836601in}}%
\pgfpathlineto{\pgfqpoint{1.127802in}{1.838682in}}%
\pgfpathlineto{\pgfqpoint{1.128073in}{1.838534in}}%
\pgfpathlineto{\pgfqpoint{1.129836in}{1.835343in}}%
\pgfpathlineto{\pgfqpoint{1.132549in}{1.822425in}}%
\pgfpathlineto{\pgfqpoint{1.135939in}{1.789525in}}%
\pgfpathlineto{\pgfqpoint{1.139466in}{1.722843in}}%
\pgfpathlineto{\pgfqpoint{1.142314in}{1.603773in}}%
\pgfpathlineto{\pgfqpoint{1.143670in}{1.427848in}}%
\pgfpathlineto{\pgfqpoint{1.143941in}{1.052219in}}%
\pgfpathlineto{\pgfqpoint{1.145298in}{1.384367in}}%
\pgfpathlineto{\pgfqpoint{1.147468in}{1.645107in}}%
\pgfpathlineto{\pgfqpoint{1.151401in}{1.762471in}}%
\pgfpathlineto{\pgfqpoint{1.156012in}{1.818772in}}%
\pgfpathlineto{\pgfqpoint{1.159945in}{1.837094in}}%
\pgfpathlineto{\pgfqpoint{1.161979in}{1.838644in}}%
\pgfpathlineto{\pgfqpoint{1.162251in}{1.838468in}}%
\pgfpathlineto{\pgfqpoint{1.164149in}{1.834648in}}%
\pgfpathlineto{\pgfqpoint{1.166998in}{1.819894in}}%
\pgfpathlineto{\pgfqpoint{1.170524in}{1.782631in}}%
\pgfpathlineto{\pgfqpoint{1.174050in}{1.708614in}}%
\pgfpathlineto{\pgfqpoint{1.176627in}{1.586622in}}%
\pgfpathlineto{\pgfqpoint{1.177848in}{1.384333in}}%
\pgfpathlineto{\pgfqpoint{1.177983in}{1.230393in}}%
\pgfpathlineto{\pgfqpoint{1.178661in}{1.423870in}}%
\pgfpathlineto{\pgfqpoint{1.179339in}{1.352122in}}%
\pgfpathlineto{\pgfqpoint{1.181374in}{1.635788in}}%
\pgfpathlineto{\pgfqpoint{1.185171in}{1.756545in}}%
\pgfpathlineto{\pgfqpoint{1.189783in}{1.816323in}}%
\pgfpathlineto{\pgfqpoint{1.193851in}{1.836699in}}%
\pgfpathlineto{\pgfqpoint{1.196021in}{1.838650in}}%
\pgfpathlineto{\pgfqpoint{1.196293in}{1.838487in}}%
\pgfpathlineto{\pgfqpoint{1.198056in}{1.835190in}}%
\pgfpathlineto{\pgfqpoint{1.200768in}{1.822104in}}%
\pgfpathlineto{\pgfqpoint{1.204294in}{1.787078in}}%
\pgfpathlineto{\pgfqpoint{1.207821in}{1.717893in}}%
\pgfpathlineto{\pgfqpoint{1.210533in}{1.600589in}}%
\pgfpathlineto{\pgfqpoint{1.211889in}{1.412925in}}%
\pgfpathlineto{\pgfqpoint{1.212161in}{1.281177in}}%
\pgfpathlineto{\pgfqpoint{1.212703in}{1.419422in}}%
\pgfpathlineto{\pgfqpoint{1.213381in}{1.328833in}}%
\pgfpathlineto{\pgfqpoint{1.215416in}{1.633459in}}%
\pgfpathlineto{\pgfqpoint{1.219213in}{1.755670in}}%
\pgfpathlineto{\pgfqpoint{1.223824in}{1.815950in}}%
\pgfpathlineto{\pgfqpoint{1.227893in}{1.836587in}}%
\pgfpathlineto{\pgfqpoint{1.230063in}{1.838667in}}%
\pgfpathlineto{\pgfqpoint{1.230334in}{1.838513in}}%
\pgfpathlineto{\pgfqpoint{1.232098in}{1.835317in}}%
\pgfpathlineto{\pgfqpoint{1.234810in}{1.822392in}}%
\pgfpathlineto{\pgfqpoint{1.238201in}{1.789485in}}%
\pgfpathlineto{\pgfqpoint{1.241727in}{1.722730in}}%
\pgfpathlineto{\pgfqpoint{1.244575in}{1.603274in}}%
\pgfpathlineto{\pgfqpoint{1.245931in}{1.424105in}}%
\pgfpathlineto{\pgfqpoint{1.246203in}{1.194772in}}%
\pgfpathlineto{\pgfqpoint{1.246745in}{1.425581in}}%
\pgfpathlineto{\pgfqpoint{1.247559in}{1.383854in}}%
\pgfpathlineto{\pgfqpoint{1.249729in}{1.645252in}}%
\pgfpathlineto{\pgfqpoint{1.253662in}{1.762555in}}%
\pgfpathlineto{\pgfqpoint{1.258273in}{1.818792in}}%
\pgfpathlineto{\pgfqpoint{1.262206in}{1.837092in}}%
\pgfpathlineto{\pgfqpoint{1.264241in}{1.838626in}}%
\pgfpathlineto{\pgfqpoint{1.264512in}{1.838447in}}%
\pgfpathlineto{\pgfqpoint{1.266275in}{1.835047in}}%
\pgfpathlineto{\pgfqpoint{1.268988in}{1.821767in}}%
\pgfpathlineto{\pgfqpoint{1.272514in}{1.786414in}}%
\pgfpathlineto{\pgfqpoint{1.276040in}{1.716454in}}%
\pgfpathlineto{\pgfqpoint{1.278753in}{1.596702in}}%
\pgfpathlineto{\pgfqpoint{1.280109in}{1.384294in}}%
\pgfpathlineto{\pgfqpoint{1.281465in}{1.225378in}}%
\pgfpathlineto{\pgfqpoint{1.280923in}{1.423530in}}%
\pgfpathlineto{\pgfqpoint{1.281601in}{1.359049in}}%
\pgfpathlineto{\pgfqpoint{1.283635in}{1.635965in}}%
\pgfpathlineto{\pgfqpoint{1.287433in}{1.756624in}}%
\pgfpathlineto{\pgfqpoint{1.292044in}{1.816346in}}%
\pgfpathlineto{\pgfqpoint{1.296113in}{1.836714in}}%
\pgfpathlineto{\pgfqpoint{1.298283in}{1.838656in}}%
\pgfpathlineto{\pgfqpoint{1.298554in}{1.838492in}}%
\pgfpathlineto{\pgfqpoint{1.300317in}{1.835180in}}%
\pgfpathlineto{\pgfqpoint{1.303029in}{1.822063in}}%
\pgfpathlineto{\pgfqpoint{1.306556in}{1.786981in}}%
\pgfpathlineto{\pgfqpoint{1.310082in}{1.717635in}}%
\pgfpathlineto{\pgfqpoint{1.312794in}{1.599502in}}%
\pgfpathlineto{\pgfqpoint{1.314151in}{1.403718in}}%
\pgfpathlineto{\pgfqpoint{1.314964in}{1.425528in}}%
\pgfpathlineto{\pgfqpoint{1.315643in}{1.299759in}}%
\pgfpathlineto{\pgfqpoint{1.317541in}{1.625778in}}%
\pgfpathlineto{\pgfqpoint{1.321203in}{1.750231in}}%
\pgfpathlineto{\pgfqpoint{1.325814in}{1.813698in}}%
\pgfpathlineto{\pgfqpoint{1.330019in}{1.836288in}}%
\pgfpathlineto{\pgfqpoint{1.332324in}{1.838686in}}%
\pgfpathlineto{\pgfqpoint{1.332596in}{1.838535in}}%
\pgfpathlineto{\pgfqpoint{1.334359in}{1.835324in}}%
\pgfpathlineto{\pgfqpoint{1.337071in}{1.822385in}}%
\pgfpathlineto{\pgfqpoint{1.340598in}{1.787615in}}%
\pgfpathlineto{\pgfqpoint{1.344124in}{1.718890in}}%
\pgfpathlineto{\pgfqpoint{1.346836in}{1.602526in}}%
\pgfpathlineto{\pgfqpoint{1.348193in}{1.414894in}}%
\pgfpathlineto{\pgfqpoint{1.349006in}{1.430190in}}%
\pgfpathlineto{\pgfqpoint{1.349684in}{1.146464in}}%
\pgfpathlineto{\pgfqpoint{1.351448in}{1.614616in}}%
\pgfpathlineto{\pgfqpoint{1.355109in}{1.746415in}}%
\pgfpathlineto{\pgfqpoint{1.359721in}{1.812154in}}%
\pgfpathlineto{\pgfqpoint{1.363925in}{1.835836in}}%
\pgfpathlineto{\pgfqpoint{1.366366in}{1.838700in}}%
\pgfpathlineto{\pgfqpoint{1.366502in}{1.838644in}}%
\pgfpathlineto{\pgfqpoint{1.368129in}{1.836177in}}%
\pgfpathlineto{\pgfqpoint{1.370706in}{1.825240in}}%
\pgfpathlineto{\pgfqpoint{1.374097in}{1.795086in}}%
\pgfpathlineto{\pgfqpoint{1.377623in}{1.733990in}}%
\pgfpathlineto{\pgfqpoint{1.380607in}{1.622706in}}%
\pgfpathlineto{\pgfqpoint{1.382234in}{1.428223in}}%
\pgfpathlineto{\pgfqpoint{1.382506in}{1.232770in}}%
\pgfpathlineto{\pgfqpoint{1.383184in}{1.432341in}}%
\pgfpathlineto{\pgfqpoint{1.383862in}{1.332077in}}%
\pgfpathlineto{\pgfqpoint{1.385896in}{1.635541in}}%
\pgfpathlineto{\pgfqpoint{1.389694in}{1.756554in}}%
\pgfpathlineto{\pgfqpoint{1.394305in}{1.816312in}}%
\pgfpathlineto{\pgfqpoint{1.398374in}{1.836703in}}%
\pgfpathlineto{\pgfqpoint{1.400544in}{1.838659in}}%
\pgfpathlineto{\pgfqpoint{1.400815in}{1.838495in}}%
\pgfpathlineto{\pgfqpoint{1.402578in}{1.835199in}}%
\pgfpathlineto{\pgfqpoint{1.405291in}{1.822090in}}%
\pgfpathlineto{\pgfqpoint{1.408681in}{1.788866in}}%
\pgfpathlineto{\pgfqpoint{1.412208in}{1.721434in}}%
\pgfpathlineto{\pgfqpoint{1.414920in}{1.608554in}}%
\pgfpathlineto{\pgfqpoint{1.416276in}{1.441191in}}%
\pgfpathlineto{\pgfqpoint{1.417904in}{1.085429in}}%
\pgfpathlineto{\pgfqpoint{1.419667in}{1.616452in}}%
\pgfpathlineto{\pgfqpoint{1.423329in}{1.747219in}}%
\pgfpathlineto{\pgfqpoint{1.427940in}{1.812478in}}%
\pgfpathlineto{\pgfqpoint{1.432144in}{1.835922in}}%
\pgfpathlineto{\pgfqpoint{1.434586in}{1.838680in}}%
\pgfpathlineto{\pgfqpoint{1.434721in}{1.838618in}}%
\pgfpathlineto{\pgfqpoint{1.436484in}{1.835722in}}%
\pgfpathlineto{\pgfqpoint{1.439061in}{1.824166in}}%
\pgfpathlineto{\pgfqpoint{1.442452in}{1.792975in}}%
\pgfpathlineto{\pgfqpoint{1.445978in}{1.729674in}}%
\pgfpathlineto{\pgfqpoint{1.448826in}{1.619805in}}%
\pgfpathlineto{\pgfqpoint{1.450318in}{1.445299in}}%
\pgfpathlineto{\pgfqpoint{1.450589in}{1.230293in}}%
\pgfpathlineto{\pgfqpoint{1.451268in}{1.446345in}}%
\pgfpathlineto{\pgfqpoint{1.452081in}{1.282164in}}%
\pgfpathlineto{\pgfqpoint{1.453980in}{1.628798in}}%
\pgfpathlineto{\pgfqpoint{1.457778in}{1.754388in}}%
\pgfpathlineto{\pgfqpoint{1.462389in}{1.815456in}}%
\pgfpathlineto{\pgfqpoint{1.466458in}{1.836444in}}%
\pgfpathlineto{\pgfqpoint{1.468763in}{1.838629in}}%
\pgfpathlineto{\pgfqpoint{1.468899in}{1.838552in}}%
\pgfpathlineto{\pgfqpoint{1.470662in}{1.835466in}}%
\pgfpathlineto{\pgfqpoint{1.473374in}{1.822738in}}%
\pgfpathlineto{\pgfqpoint{1.476765in}{1.790132in}}%
\pgfpathlineto{\pgfqpoint{1.480291in}{1.724080in}}%
\pgfpathlineto{\pgfqpoint{1.483139in}{1.606386in}}%
\pgfpathlineto{\pgfqpoint{1.484496in}{1.429817in}}%
\pgfpathlineto{\pgfqpoint{1.484767in}{1.210203in}}%
\pgfpathlineto{\pgfqpoint{1.485445in}{1.435653in}}%
\pgfpathlineto{\pgfqpoint{1.486123in}{1.302641in}}%
\pgfpathlineto{\pgfqpoint{1.488022in}{1.626922in}}%
\pgfpathlineto{\pgfqpoint{1.491684in}{1.750758in}}%
\pgfpathlineto{\pgfqpoint{1.496295in}{1.813905in}}%
\pgfpathlineto{\pgfqpoint{1.500499in}{1.836299in}}%
\pgfpathlineto{\pgfqpoint{1.502805in}{1.838607in}}%
\pgfpathlineto{\pgfqpoint{1.503076in}{1.838446in}}%
\pgfpathlineto{\pgfqpoint{1.504839in}{1.835176in}}%
\pgfpathlineto{\pgfqpoint{1.507552in}{1.822137in}}%
\pgfpathlineto{\pgfqpoint{1.511078in}{1.787167in}}%
\pgfpathlineto{\pgfqpoint{1.514604in}{1.717949in}}%
\pgfpathlineto{\pgfqpoint{1.517317in}{1.599532in}}%
\pgfpathlineto{\pgfqpoint{1.518538in}{1.435816in}}%
\pgfpathlineto{\pgfqpoint{1.518809in}{1.182769in}}%
\pgfpathlineto{\pgfqpoint{1.519487in}{1.448641in}}%
\pgfpathlineto{\pgfqpoint{1.520301in}{1.325253in}}%
\pgfpathlineto{\pgfqpoint{1.522199in}{1.630369in}}%
\pgfpathlineto{\pgfqpoint{1.525997in}{1.754970in}}%
\pgfpathlineto{\pgfqpoint{1.530608in}{1.815701in}}%
\pgfpathlineto{\pgfqpoint{1.534677in}{1.836493in}}%
\pgfpathlineto{\pgfqpoint{1.536847in}{1.838612in}}%
\pgfpathlineto{\pgfqpoint{1.537118in}{1.838465in}}%
\pgfpathlineto{\pgfqpoint{1.538881in}{1.835300in}}%
\pgfpathlineto{\pgfqpoint{1.541594in}{1.822389in}}%
\pgfpathlineto{\pgfqpoint{1.544984in}{1.789428in}}%
\pgfpathlineto{\pgfqpoint{1.548511in}{1.722383in}}%
\pgfpathlineto{\pgfqpoint{1.551223in}{1.610221in}}%
\pgfpathlineto{\pgfqpoint{1.552579in}{1.436822in}}%
\pgfpathlineto{\pgfqpoint{1.553529in}{1.454858in}}%
\pgfpathlineto{\pgfqpoint{1.554343in}{1.024065in}}%
\pgfpathlineto{\pgfqpoint{1.556106in}{1.620051in}}%
\pgfpathlineto{\pgfqpoint{1.559768in}{1.748815in}}%
\pgfpathlineto{\pgfqpoint{1.564379in}{1.813177in}}%
\pgfpathlineto{\pgfqpoint{1.568583in}{1.836142in}}%
\pgfpathlineto{\pgfqpoint{1.571024in}{1.838642in}}%
\pgfpathlineto{\pgfqpoint{1.571160in}{1.838567in}}%
\pgfpathlineto{\pgfqpoint{1.572923in}{1.835490in}}%
\pgfpathlineto{\pgfqpoint{1.575636in}{1.822735in}}%
\pgfpathlineto{\pgfqpoint{1.579026in}{1.790035in}}%
\pgfpathlineto{\pgfqpoint{1.582553in}{1.723332in}}%
\pgfpathlineto{\pgfqpoint{1.585265in}{1.610900in}}%
\pgfpathlineto{\pgfqpoint{1.586621in}{1.418167in}}%
\pgfpathlineto{\pgfqpoint{1.586757in}{1.310483in}}%
\pgfpathlineto{\pgfqpoint{1.587706in}{1.475929in}}%
\pgfpathlineto{\pgfqpoint{1.588113in}{1.456489in}}%
\pgfpathlineto{\pgfqpoint{1.588520in}{1.317585in}}%
\pgfpathlineto{\pgfqpoint{1.589063in}{1.504633in}}%
\pgfpathlineto{\pgfqpoint{1.591775in}{1.691480in}}%
\pgfpathlineto{\pgfqpoint{1.595979in}{1.785077in}}%
\pgfpathlineto{\pgfqpoint{1.600591in}{1.827818in}}%
\pgfpathlineto{\pgfqpoint{1.604117in}{1.838388in}}%
\pgfpathlineto{\pgfqpoint{1.605880in}{1.837785in}}%
\pgfpathlineto{\pgfqpoint{1.608050in}{1.831677in}}%
\pgfpathlineto{\pgfqpoint{1.611034in}{1.812506in}}%
\pgfpathlineto{\pgfqpoint{1.614560in}{1.768052in}}%
\pgfpathlineto{\pgfqpoint{1.617951in}{1.681324in}}%
\pgfpathlineto{\pgfqpoint{1.620121in}{1.543258in}}%
\pgfpathlineto{\pgfqpoint{1.620934in}{1.284963in}}%
\pgfpathlineto{\pgfqpoint{1.622698in}{1.338104in}}%
\pgfpathlineto{\pgfqpoint{1.624596in}{1.636009in}}%
\pgfpathlineto{\pgfqpoint{1.628394in}{1.757340in}}%
\pgfpathlineto{\pgfqpoint{1.633005in}{1.816705in}}%
\pgfpathlineto{\pgfqpoint{1.637074in}{1.836765in}}%
\pgfpathlineto{\pgfqpoint{1.639244in}{1.838551in}}%
\pgfpathlineto{\pgfqpoint{1.639515in}{1.838358in}}%
\pgfpathlineto{\pgfqpoint{1.641414in}{1.834479in}}%
\pgfpathlineto{\pgfqpoint{1.644126in}{1.820565in}}%
\pgfpathlineto{\pgfqpoint{1.647653in}{1.783933in}}%
\pgfpathlineto{\pgfqpoint{1.651179in}{1.710890in}}%
\pgfpathlineto{\pgfqpoint{1.653756in}{1.589014in}}%
\pgfpathlineto{\pgfqpoint{1.654841in}{1.411344in}}%
\pgfpathlineto{\pgfqpoint{1.654976in}{1.293825in}}%
\pgfpathlineto{\pgfqpoint{1.655926in}{1.474272in}}%
\pgfpathlineto{\pgfqpoint{1.656333in}{1.449699in}}%
\pgfpathlineto{\pgfqpoint{1.656739in}{1.202736in}}%
\pgfpathlineto{\pgfqpoint{1.657282in}{1.510395in}}%
\pgfpathlineto{\pgfqpoint{1.659994in}{1.692473in}}%
\pgfpathlineto{\pgfqpoint{1.664199in}{1.785310in}}%
\pgfpathlineto{\pgfqpoint{1.668810in}{1.827680in}}%
\pgfpathlineto{\pgfqpoint{1.672336in}{1.838067in}}%
\pgfpathlineto{\pgfqpoint{1.674099in}{1.837387in}}%
\pgfpathlineto{\pgfqpoint{1.676269in}{1.831164in}}%
\pgfpathlineto{\pgfqpoint{1.679389in}{1.810650in}}%
\pgfpathlineto{\pgfqpoint{1.682915in}{1.764937in}}%
\pgfpathlineto{\pgfqpoint{1.686170in}{1.680860in}}%
\pgfpathlineto{\pgfqpoint{1.688476in}{1.531480in}}%
\pgfpathlineto{\pgfqpoint{1.690646in}{1.254830in}}%
\pgfpathlineto{\pgfqpoint{1.690781in}{1.361444in}}%
\pgfpathlineto{\pgfqpoint{1.692816in}{1.639499in}}%
\pgfpathlineto{\pgfqpoint{1.696613in}{1.757897in}}%
\pgfpathlineto{\pgfqpoint{1.701224in}{1.816649in}}%
\pgfpathlineto{\pgfqpoint{1.705293in}{1.836504in}}%
\pgfpathlineto{\pgfqpoint{1.707463in}{1.838189in}}%
\pgfpathlineto{\pgfqpoint{1.707599in}{1.838102in}}%
\pgfpathlineto{\pgfqpoint{1.709362in}{1.834883in}}%
\pgfpathlineto{\pgfqpoint{1.712074in}{1.821929in}}%
\pgfpathlineto{\pgfqpoint{1.715465in}{1.788922in}}%
\pgfpathlineto{\pgfqpoint{1.718991in}{1.721956in}}%
\pgfpathlineto{\pgfqpoint{1.721839in}{1.601624in}}%
\pgfpathlineto{\pgfqpoint{1.723196in}{1.403825in}}%
\pgfpathlineto{\pgfqpoint{1.724145in}{1.430473in}}%
\pgfpathlineto{\pgfqpoint{1.724688in}{1.284000in}}%
\pgfpathlineto{\pgfqpoint{1.726858in}{1.637287in}}%
\pgfpathlineto{\pgfqpoint{1.730655in}{1.756750in}}%
\pgfpathlineto{\pgfqpoint{1.735266in}{1.815967in}}%
\pgfpathlineto{\pgfqpoint{1.739335in}{1.836013in}}%
\pgfpathlineto{\pgfqpoint{1.741505in}{1.837798in}}%
\pgfpathlineto{\pgfqpoint{1.741776in}{1.837610in}}%
\pgfpathlineto{\pgfqpoint{1.743675in}{1.833736in}}%
\pgfpathlineto{\pgfqpoint{1.746523in}{1.818874in}}%
\pgfpathlineto{\pgfqpoint{1.750049in}{1.781397in}}%
\pgfpathlineto{\pgfqpoint{1.753440in}{1.710757in}}%
\pgfpathlineto{\pgfqpoint{1.756017in}{1.591059in}}%
\pgfpathlineto{\pgfqpoint{1.757238in}{1.373659in}}%
\pgfpathlineto{\pgfqpoint{1.757373in}{1.256617in}}%
\pgfpathlineto{\pgfqpoint{1.758187in}{1.456610in}}%
\pgfpathlineto{\pgfqpoint{1.758729in}{1.396231in}}%
\pgfpathlineto{\pgfqpoint{1.758865in}{1.301260in}}%
\pgfpathlineto{\pgfqpoint{1.759950in}{1.565038in}}%
\pgfpathlineto{\pgfqpoint{1.763069in}{1.717705in}}%
\pgfpathlineto{\pgfqpoint{1.767545in}{1.798632in}}%
\pgfpathlineto{\pgfqpoint{1.772021in}{1.831832in}}%
\pgfpathlineto{\pgfqpoint{1.775140in}{1.838126in}}%
\pgfpathlineto{\pgfqpoint{1.776632in}{1.836852in}}%
\pgfpathlineto{\pgfqpoint{1.778802in}{1.829982in}}%
\pgfpathlineto{\pgfqpoint{1.781921in}{1.808336in}}%
\pgfpathlineto{\pgfqpoint{1.785583in}{1.758173in}}%
\pgfpathlineto{\pgfqpoint{1.788838in}{1.666969in}}%
\pgfpathlineto{\pgfqpoint{1.791008in}{1.507827in}}%
\pgfpathlineto{\pgfqpoint{1.791686in}{1.145853in}}%
\pgfpathlineto{\pgfqpoint{1.793043in}{1.408113in}}%
\pgfpathlineto{\pgfqpoint{1.795348in}{1.652354in}}%
\pgfpathlineto{\pgfqpoint{1.799281in}{1.764499in}}%
\pgfpathlineto{\pgfqpoint{1.803893in}{1.819014in}}%
\pgfpathlineto{\pgfqpoint{1.807826in}{1.836421in}}%
\pgfpathlineto{\pgfqpoint{1.809724in}{1.837629in}}%
\pgfpathlineto{\pgfqpoint{1.810131in}{1.837306in}}%
\pgfpathlineto{\pgfqpoint{1.812030in}{1.833080in}}%
\pgfpathlineto{\pgfqpoint{1.814878in}{1.817528in}}%
\pgfpathlineto{\pgfqpoint{1.818404in}{1.778572in}}%
\pgfpathlineto{\pgfqpoint{1.822066in}{1.698540in}}%
\pgfpathlineto{\pgfqpoint{1.824779in}{1.563232in}}%
\pgfpathlineto{\pgfqpoint{1.826135in}{1.451517in}}%
\pgfpathlineto{\pgfqpoint{1.826949in}{1.504187in}}%
\pgfpathlineto{\pgfqpoint{1.830611in}{1.707159in}}%
\pgfpathlineto{\pgfqpoint{1.835086in}{1.794223in}}%
\pgfpathlineto{\pgfqpoint{1.839562in}{1.830858in}}%
\pgfpathlineto{\pgfqpoint{1.842953in}{1.839316in}}%
\pgfpathlineto{\pgfqpoint{1.844580in}{1.838312in}}%
\pgfpathlineto{\pgfqpoint{1.846750in}{1.831885in}}%
\pgfpathlineto{\pgfqpoint{1.849869in}{1.811218in}}%
\pgfpathlineto{\pgfqpoint{1.853531in}{1.763581in}}%
\pgfpathlineto{\pgfqpoint{1.857058in}{1.671289in}}%
\pgfpathlineto{\pgfqpoint{1.860313in}{1.510615in}}%
\pgfpathlineto{\pgfqpoint{1.861398in}{1.556944in}}%
\pgfpathlineto{\pgfqpoint{1.865466in}{1.728556in}}%
\pgfpathlineto{\pgfqpoint{1.870078in}{1.804052in}}%
\pgfpathlineto{\pgfqpoint{1.874418in}{1.833587in}}%
\pgfpathlineto{\pgfqpoint{1.877401in}{1.838895in}}%
\pgfpathlineto{\pgfqpoint{1.879029in}{1.837194in}}%
\pgfpathlineto{\pgfqpoint{1.881334in}{1.829019in}}%
\pgfpathlineto{\pgfqpoint{1.884589in}{1.804286in}}%
\pgfpathlineto{\pgfqpoint{1.888251in}{1.749617in}}%
\pgfpathlineto{\pgfqpoint{1.891371in}{1.654383in}}%
\pgfpathlineto{\pgfqpoint{1.893541in}{1.485859in}}%
\pgfpathlineto{\pgfqpoint{1.894354in}{1.337795in}}%
\pgfpathlineto{\pgfqpoint{1.895168in}{1.470523in}}%
\pgfpathlineto{\pgfqpoint{1.898152in}{1.683399in}}%
\pgfpathlineto{\pgfqpoint{1.902356in}{1.781163in}}%
\pgfpathlineto{\pgfqpoint{1.906968in}{1.826493in}}%
\pgfpathlineto{\pgfqpoint{1.910629in}{1.838792in}}%
\pgfpathlineto{\pgfqpoint{1.912257in}{1.838869in}}%
\pgfpathlineto{\pgfqpoint{1.912528in}{1.838549in}}%
\pgfpathlineto{\pgfqpoint{1.914563in}{1.833352in}}%
\pgfpathlineto{\pgfqpoint{1.917546in}{1.815577in}}%
\pgfpathlineto{\pgfqpoint{1.921208in}{1.772119in}}%
\pgfpathlineto{\pgfqpoint{1.924734in}{1.689171in}}%
\pgfpathlineto{\pgfqpoint{1.927583in}{1.544272in}}%
\pgfpathlineto{\pgfqpoint{1.928532in}{1.503940in}}%
\pgfpathlineto{\pgfqpoint{1.929210in}{1.528902in}}%
\pgfpathlineto{\pgfqpoint{1.933821in}{1.732711in}}%
\pgfpathlineto{\pgfqpoint{1.938568in}{1.807339in}}%
\pgfpathlineto{\pgfqpoint{1.942908in}{1.834825in}}%
\pgfpathlineto{\pgfqpoint{1.945756in}{1.838999in}}%
\pgfpathlineto{\pgfqpoint{1.947384in}{1.836896in}}%
\pgfpathlineto{\pgfqpoint{1.949825in}{1.827399in}}%
\pgfpathlineto{\pgfqpoint{1.953080in}{1.800812in}}%
\pgfpathlineto{\pgfqpoint{1.956742in}{1.742493in}}%
\pgfpathlineto{\pgfqpoint{1.959861in}{1.638034in}}%
\pgfpathlineto{\pgfqpoint{1.961896in}{1.448196in}}%
\pgfpathlineto{\pgfqpoint{1.962574in}{1.224098in}}%
\pgfpathlineto{\pgfqpoint{1.963252in}{1.445453in}}%
\pgfpathlineto{\pgfqpoint{1.966100in}{1.674024in}}%
\pgfpathlineto{\pgfqpoint{1.970304in}{1.777431in}}%
\pgfpathlineto{\pgfqpoint{1.974916in}{1.824902in}}%
\pgfpathlineto{\pgfqpoint{1.978713in}{1.838576in}}%
\pgfpathlineto{\pgfqpoint{1.980341in}{1.838885in}}%
\pgfpathlineto{\pgfqpoint{1.980612in}{1.838612in}}%
\pgfpathlineto{\pgfqpoint{1.982646in}{1.833707in}}%
\pgfpathlineto{\pgfqpoint{1.985630in}{1.816469in}}%
\pgfpathlineto{\pgfqpoint{1.989292in}{1.773947in}}%
\pgfpathlineto{\pgfqpoint{1.992818in}{1.692636in}}%
\pgfpathlineto{\pgfqpoint{1.995666in}{1.550725in}}%
\pgfpathlineto{\pgfqpoint{1.996616in}{1.503197in}}%
\pgfpathlineto{\pgfqpoint{1.997429in}{1.529396in}}%
\pgfpathlineto{\pgfqpoint{2.002176in}{1.737013in}}%
\pgfpathlineto{\pgfqpoint{2.006788in}{1.808091in}}%
\pgfpathlineto{\pgfqpoint{2.011128in}{1.835281in}}%
\pgfpathlineto{\pgfqpoint{2.013976in}{1.839255in}}%
\pgfpathlineto{\pgfqpoint{2.015603in}{1.837027in}}%
\pgfpathlineto{\pgfqpoint{2.018044in}{1.827284in}}%
\pgfpathlineto{\pgfqpoint{2.021299in}{1.800227in}}%
\pgfpathlineto{\pgfqpoint{2.024826in}{1.743865in}}%
\pgfpathlineto{\pgfqpoint{2.027945in}{1.639775in}}%
\pgfpathlineto{\pgfqpoint{2.029844in}{1.461062in}}%
\pgfpathlineto{\pgfqpoint{2.031200in}{1.256858in}}%
\pgfpathlineto{\pgfqpoint{2.031471in}{1.424510in}}%
\pgfpathlineto{\pgfqpoint{2.033913in}{1.658756in}}%
\pgfpathlineto{\pgfqpoint{2.037981in}{1.769643in}}%
\pgfpathlineto{\pgfqpoint{2.042593in}{1.821689in}}%
\pgfpathlineto{\pgfqpoint{2.046526in}{1.837976in}}%
\pgfpathlineto{\pgfqpoint{2.048424in}{1.838769in}}%
\pgfpathlineto{\pgfqpoint{2.048696in}{1.838515in}}%
\pgfpathlineto{\pgfqpoint{2.050730in}{1.833771in}}%
\pgfpathlineto{\pgfqpoint{2.053578in}{1.817852in}}%
\pgfpathlineto{\pgfqpoint{2.057104in}{1.779119in}}%
\pgfpathlineto{\pgfqpoint{2.060766in}{1.700314in}}%
\pgfpathlineto{\pgfqpoint{2.064835in}{1.533942in}}%
\pgfpathlineto{\pgfqpoint{2.066056in}{1.578959in}}%
\pgfpathlineto{\pgfqpoint{2.070667in}{1.745986in}}%
\pgfpathlineto{\pgfqpoint{2.075414in}{1.812752in}}%
\pgfpathlineto{\pgfqpoint{2.079618in}{1.836296in}}%
\pgfpathlineto{\pgfqpoint{2.082059in}{1.839287in}}%
\pgfpathlineto{\pgfqpoint{2.082331in}{1.839149in}}%
\pgfpathlineto{\pgfqpoint{2.084229in}{1.835813in}}%
\pgfpathlineto{\pgfqpoint{2.086942in}{1.823199in}}%
\pgfpathlineto{\pgfqpoint{2.090468in}{1.788616in}}%
\pgfpathlineto{\pgfqpoint{2.094130in}{1.719052in}}%
\pgfpathlineto{\pgfqpoint{2.097385in}{1.592759in}}%
\pgfpathlineto{\pgfqpoint{2.098877in}{1.539532in}}%
\pgfpathlineto{\pgfqpoint{2.099555in}{1.558795in}}%
\pgfpathlineto{\pgfqpoint{2.105387in}{1.760870in}}%
\pgfpathlineto{\pgfqpoint{2.110134in}{1.819170in}}%
\pgfpathlineto{\pgfqpoint{2.114203in}{1.838303in}}%
\pgfpathlineto{\pgfqpoint{2.116373in}{1.839903in}}%
\pgfpathlineto{\pgfqpoint{2.116508in}{1.839815in}}%
\pgfpathlineto{\pgfqpoint{2.118407in}{1.836238in}}%
\pgfpathlineto{\pgfqpoint{2.121255in}{1.822151in}}%
\pgfpathlineto{\pgfqpoint{2.124781in}{1.787005in}}%
\pgfpathlineto{\pgfqpoint{2.128579in}{1.714438in}}%
\pgfpathlineto{\pgfqpoint{2.132919in}{1.587185in}}%
\pgfpathlineto{\pgfqpoint{2.134004in}{1.608409in}}%
\pgfpathlineto{\pgfqpoint{2.140649in}{1.782573in}}%
\pgfpathlineto{\pgfqpoint{2.145396in}{1.827913in}}%
\pgfpathlineto{\pgfqpoint{2.149058in}{1.839954in}}%
\pgfpathlineto{\pgfqpoint{2.150821in}{1.839921in}}%
\pgfpathlineto{\pgfqpoint{2.150957in}{1.839769in}}%
\pgfpathlineto{\pgfqpoint{2.152991in}{1.834729in}}%
\pgfpathlineto{\pgfqpoint{2.155975in}{1.817483in}}%
\pgfpathlineto{\pgfqpoint{2.159637in}{1.775649in}}%
\pgfpathlineto{\pgfqpoint{2.163299in}{1.694200in}}%
\pgfpathlineto{\pgfqpoint{2.167096in}{1.559784in}}%
\pgfpathlineto{\pgfqpoint{2.168046in}{1.584157in}}%
\pgfpathlineto{\pgfqpoint{2.173742in}{1.763799in}}%
\pgfpathlineto{\pgfqpoint{2.178489in}{1.820244in}}%
\pgfpathlineto{\pgfqpoint{2.182558in}{1.838545in}}%
\pgfpathlineto{\pgfqpoint{2.184592in}{1.839864in}}%
\pgfpathlineto{\pgfqpoint{2.184863in}{1.839663in}}%
\pgfpathlineto{\pgfqpoint{2.186762in}{1.835750in}}%
\pgfpathlineto{\pgfqpoint{2.189610in}{1.821138in}}%
\pgfpathlineto{\pgfqpoint{2.193136in}{1.785239in}}%
\pgfpathlineto{\pgfqpoint{2.196934in}{1.711767in}}%
\pgfpathlineto{\pgfqpoint{2.201138in}{1.596814in}}%
\pgfpathlineto{\pgfqpoint{2.202088in}{1.612555in}}%
\pgfpathlineto{\pgfqpoint{2.209954in}{1.796994in}}%
\pgfpathlineto{\pgfqpoint{2.214565in}{1.832707in}}%
\pgfpathlineto{\pgfqpoint{2.217820in}{1.840329in}}%
\pgfpathlineto{\pgfqpoint{2.219448in}{1.839411in}}%
\pgfpathlineto{\pgfqpoint{2.221618in}{1.833218in}}%
\pgfpathlineto{\pgfqpoint{2.224737in}{1.813363in}}%
\pgfpathlineto{\pgfqpoint{2.228399in}{1.768300in}}%
\pgfpathlineto{\pgfqpoint{2.232196in}{1.678581in}}%
\pgfpathlineto{\pgfqpoint{2.235316in}{1.588907in}}%
\pgfpathlineto{\pgfqpoint{2.235994in}{1.598233in}}%
\pgfpathlineto{\pgfqpoint{2.245759in}{1.812273in}}%
\pgfpathlineto{\pgfqpoint{2.249963in}{1.836076in}}%
\pgfpathlineto{\pgfqpoint{2.252676in}{1.839329in}}%
\pgfpathlineto{\pgfqpoint{2.254303in}{1.837093in}}%
\pgfpathlineto{\pgfqpoint{2.256880in}{1.826860in}}%
\pgfpathlineto{\pgfqpoint{2.260271in}{1.798904in}}%
\pgfpathlineto{\pgfqpoint{2.264204in}{1.737004in}}%
\pgfpathlineto{\pgfqpoint{2.269358in}{1.613364in}}%
\pgfpathlineto{\pgfqpoint{2.270578in}{1.631805in}}%
\pgfpathlineto{\pgfqpoint{2.278173in}{1.797101in}}%
\pgfpathlineto{\pgfqpoint{2.282784in}{1.832714in}}%
\pgfpathlineto{\pgfqpoint{2.286039in}{1.840435in}}%
\pgfpathlineto{\pgfqpoint{2.287667in}{1.839578in}}%
\pgfpathlineto{\pgfqpoint{2.289837in}{1.833516in}}%
\pgfpathlineto{\pgfqpoint{2.292956in}{1.813884in}}%
\pgfpathlineto{\pgfqpoint{2.296618in}{1.769314in}}%
\pgfpathlineto{\pgfqpoint{2.300416in}{1.681658in}}%
\pgfpathlineto{\pgfqpoint{2.303535in}{1.599669in}}%
\pgfpathlineto{\pgfqpoint{2.304213in}{1.606829in}}%
\pgfpathlineto{\pgfqpoint{2.314521in}{1.817251in}}%
\pgfpathlineto{\pgfqpoint{2.318589in}{1.837290in}}%
\pgfpathlineto{\pgfqpoint{2.320895in}{1.839400in}}%
\pgfpathlineto{\pgfqpoint{2.321166in}{1.839238in}}%
\pgfpathlineto{\pgfqpoint{2.323065in}{1.835616in}}%
\pgfpathlineto{\pgfqpoint{2.325913in}{1.821668in}}%
\pgfpathlineto{\pgfqpoint{2.329439in}{1.787310in}}%
\pgfpathlineto{\pgfqpoint{2.333237in}{1.718652in}}%
\pgfpathlineto{\pgfqpoint{2.337577in}{1.615808in}}%
\pgfpathlineto{\pgfqpoint{2.338391in}{1.624729in}}%
\pgfpathlineto{\pgfqpoint{2.349648in}{1.825431in}}%
\pgfpathlineto{\pgfqpoint{2.353445in}{1.839634in}}%
\pgfpathlineto{\pgfqpoint{2.355344in}{1.840189in}}%
\pgfpathlineto{\pgfqpoint{2.355615in}{1.839916in}}%
\pgfpathlineto{\pgfqpoint{2.357649in}{1.835166in}}%
\pgfpathlineto{\pgfqpoint{2.360633in}{1.818577in}}%
\pgfpathlineto{\pgfqpoint{2.364295in}{1.778667in}}%
\pgfpathlineto{\pgfqpoint{2.368093in}{1.701571in}}%
\pgfpathlineto{\pgfqpoint{2.371754in}{1.613356in}}%
\pgfpathlineto{\pgfqpoint{2.372433in}{1.619959in}}%
\pgfpathlineto{\pgfqpoint{2.375823in}{1.713667in}}%
\pgfpathlineto{\pgfqpoint{2.380841in}{1.799739in}}%
\pgfpathlineto{\pgfqpoint{2.385317in}{1.832405in}}%
\pgfpathlineto{\pgfqpoint{2.388572in}{1.839234in}}%
\pgfpathlineto{\pgfqpoint{2.390199in}{1.837917in}}%
\pgfpathlineto{\pgfqpoint{2.392641in}{1.829947in}}%
\pgfpathlineto{\pgfqpoint{2.395896in}{1.806710in}}%
\pgfpathlineto{\pgfqpoint{2.399829in}{1.752545in}}%
\pgfpathlineto{\pgfqpoint{2.404169in}{1.642450in}}%
\pgfpathlineto{\pgfqpoint{2.405661in}{1.617559in}}%
\pgfpathlineto{\pgfqpoint{2.406339in}{1.622893in}}%
\pgfpathlineto{\pgfqpoint{2.408644in}{1.683715in}}%
\pgfpathlineto{\pgfqpoint{2.413798in}{1.787925in}}%
\pgfpathlineto{\pgfqpoint{2.418545in}{1.829556in}}%
\pgfpathlineto{\pgfqpoint{2.422071in}{1.839933in}}%
\pgfpathlineto{\pgfqpoint{2.423834in}{1.839562in}}%
\pgfpathlineto{\pgfqpoint{2.425869in}{1.834574in}}%
\pgfpathlineto{\pgfqpoint{2.428853in}{1.817688in}}%
\pgfpathlineto{\pgfqpoint{2.432514in}{1.777648in}}%
\pgfpathlineto{\pgfqpoint{2.436583in}{1.695136in}}%
\pgfpathlineto{\pgfqpoint{2.439838in}{1.626800in}}%
\pgfpathlineto{\pgfqpoint{2.440381in}{1.629814in}}%
\pgfpathlineto{\pgfqpoint{2.442279in}{1.673195in}}%
\pgfpathlineto{\pgfqpoint{2.447840in}{1.786553in}}%
\pgfpathlineto{\pgfqpoint{2.452587in}{1.828183in}}%
\pgfpathlineto{\pgfqpoint{2.456113in}{1.838585in}}%
\pgfpathlineto{\pgfqpoint{2.457876in}{1.838234in}}%
\pgfpathlineto{\pgfqpoint{2.460046in}{1.832729in}}%
\pgfpathlineto{\pgfqpoint{2.463030in}{1.815245in}}%
\pgfpathlineto{\pgfqpoint{2.466828in}{1.771716in}}%
\pgfpathlineto{\pgfqpoint{2.470761in}{1.687018in}}%
\pgfpathlineto{\pgfqpoint{2.473880in}{1.617412in}}%
\pgfpathlineto{\pgfqpoint{2.474423in}{1.621569in}}%
\pgfpathlineto{\pgfqpoint{2.476728in}{1.681743in}}%
\pgfpathlineto{\pgfqpoint{2.482018in}{1.788764in}}%
\pgfpathlineto{\pgfqpoint{2.486764in}{1.829820in}}%
\pgfpathlineto{\pgfqpoint{2.490291in}{1.839803in}}%
\pgfpathlineto{\pgfqpoint{2.492054in}{1.839277in}}%
\pgfpathlineto{\pgfqpoint{2.494224in}{1.833545in}}%
\pgfpathlineto{\pgfqpoint{2.497343in}{1.814653in}}%
\pgfpathlineto{\pgfqpoint{2.501141in}{1.770096in}}%
\pgfpathlineto{\pgfqpoint{2.505345in}{1.678883in}}%
\pgfpathlineto{\pgfqpoint{2.507922in}{1.629231in}}%
\pgfpathlineto{\pgfqpoint{2.508464in}{1.632112in}}%
\pgfpathlineto{\pgfqpoint{2.510092in}{1.666378in}}%
\pgfpathlineto{\pgfqpoint{2.516195in}{1.789705in}}%
\pgfpathlineto{\pgfqpoint{2.520942in}{1.829421in}}%
\pgfpathlineto{\pgfqpoint{2.524468in}{1.838871in}}%
\pgfpathlineto{\pgfqpoint{2.526231in}{1.838064in}}%
\pgfpathlineto{\pgfqpoint{2.528401in}{1.831982in}}%
\pgfpathlineto{\pgfqpoint{2.531521in}{1.812420in}}%
\pgfpathlineto{\pgfqpoint{2.535318in}{1.766053in}}%
\pgfpathlineto{\pgfqpoint{2.539251in}{1.675588in}}%
\pgfpathlineto{\pgfqpoint{2.541964in}{1.612085in}}%
\pgfpathlineto{\pgfqpoint{2.542642in}{1.617959in}}%
\pgfpathlineto{\pgfqpoint{2.545219in}{1.689098in}}%
\pgfpathlineto{\pgfqpoint{2.550237in}{1.788972in}}%
\pgfpathlineto{\pgfqpoint{2.554984in}{1.829805in}}%
\pgfpathlineto{\pgfqpoint{2.558510in}{1.839657in}}%
\pgfpathlineto{\pgfqpoint{2.560273in}{1.839038in}}%
\pgfpathlineto{\pgfqpoint{2.562443in}{1.833222in}}%
\pgfpathlineto{\pgfqpoint{2.565427in}{1.815299in}}%
\pgfpathlineto{\pgfqpoint{2.569089in}{1.773638in}}%
\pgfpathlineto{\pgfqpoint{2.573293in}{1.685835in}}%
\pgfpathlineto{\pgfqpoint{2.576141in}{1.631351in}}%
\pgfpathlineto{\pgfqpoint{2.576684in}{1.634295in}}%
\pgfpathlineto{\pgfqpoint{2.578583in}{1.675692in}}%
\pgfpathlineto{\pgfqpoint{2.584279in}{1.788881in}}%
\pgfpathlineto{\pgfqpoint{2.589026in}{1.829185in}}%
\pgfpathlineto{\pgfqpoint{2.592552in}{1.838927in}}%
\pgfpathlineto{\pgfqpoint{2.594315in}{1.838259in}}%
\pgfpathlineto{\pgfqpoint{2.596485in}{1.832358in}}%
\pgfpathlineto{\pgfqpoint{2.599604in}{1.813067in}}%
\pgfpathlineto{\pgfqpoint{2.603402in}{1.767276in}}%
\pgfpathlineto{\pgfqpoint{2.607335in}{1.678202in}}%
\pgfpathlineto{\pgfqpoint{2.610183in}{1.612590in}}%
\pgfpathlineto{\pgfqpoint{2.610861in}{1.618875in}}%
\pgfpathlineto{\pgfqpoint{2.613981in}{1.704694in}}%
\pgfpathlineto{\pgfqpoint{2.618863in}{1.793723in}}%
\pgfpathlineto{\pgfqpoint{2.623474in}{1.830426in}}%
\pgfpathlineto{\pgfqpoint{2.626865in}{1.838972in}}%
\pgfpathlineto{\pgfqpoint{2.628628in}{1.838114in}}%
\pgfpathlineto{\pgfqpoint{2.630798in}{1.832045in}}%
\pgfpathlineto{\pgfqpoint{2.633918in}{1.812723in}}%
\pgfpathlineto{\pgfqpoint{2.637715in}{1.768021in}}%
\pgfpathlineto{\pgfqpoint{2.642191in}{1.673627in}}%
\pgfpathlineto{\pgfqpoint{2.644361in}{1.640322in}}%
\pgfpathlineto{\pgfqpoint{2.644903in}{1.643065in}}%
\pgfpathlineto{\pgfqpoint{2.646531in}{1.673088in}}%
\pgfpathlineto{\pgfqpoint{2.653312in}{1.798078in}}%
\pgfpathlineto{\pgfqpoint{2.657923in}{1.831713in}}%
\pgfpathlineto{\pgfqpoint{2.661178in}{1.838600in}}%
\pgfpathlineto{\pgfqpoint{2.662806in}{1.837396in}}%
\pgfpathlineto{\pgfqpoint{2.665111in}{1.830284in}}%
\pgfpathlineto{\pgfqpoint{2.668231in}{1.809465in}}%
\pgfpathlineto{\pgfqpoint{2.672028in}{1.761197in}}%
\pgfpathlineto{\pgfqpoint{2.676233in}{1.662835in}}%
\pgfpathlineto{\pgfqpoint{2.678403in}{1.621524in}}%
\pgfpathlineto{\pgfqpoint{2.678945in}{1.624674in}}%
\pgfpathlineto{\pgfqpoint{2.680979in}{1.673047in}}%
\pgfpathlineto{\pgfqpoint{2.686404in}{1.784873in}}%
\pgfpathlineto{\pgfqpoint{2.691151in}{1.827057in}}%
\pgfpathlineto{\pgfqpoint{2.694813in}{1.838039in}}%
\pgfpathlineto{\pgfqpoint{2.696712in}{1.837524in}}%
\pgfpathlineto{\pgfqpoint{2.698882in}{1.831780in}}%
\pgfpathlineto{\pgfqpoint{2.702001in}{1.813115in}}%
\pgfpathlineto{\pgfqpoint{2.705799in}{1.769571in}}%
\pgfpathlineto{\pgfqpoint{2.710410in}{1.674957in}}%
\pgfpathlineto{\pgfqpoint{2.712444in}{1.644624in}}%
\pgfpathlineto{\pgfqpoint{2.713123in}{1.647067in}}%
\pgfpathlineto{\pgfqpoint{2.714750in}{1.676233in}}%
\pgfpathlineto{\pgfqpoint{2.721396in}{1.796808in}}%
\pgfpathlineto{\pgfqpoint{2.726007in}{1.830986in}}%
\pgfpathlineto{\pgfqpoint{2.729398in}{1.838353in}}%
\pgfpathlineto{\pgfqpoint{2.731025in}{1.837162in}}%
\pgfpathlineto{\pgfqpoint{2.733331in}{1.830099in}}%
\pgfpathlineto{\pgfqpoint{2.736586in}{1.808276in}}%
\pgfpathlineto{\pgfqpoint{2.740383in}{1.759817in}}%
\pgfpathlineto{\pgfqpoint{2.744994in}{1.653243in}}%
\pgfpathlineto{\pgfqpoint{2.746758in}{1.628169in}}%
\pgfpathlineto{\pgfqpoint{2.747300in}{1.632191in}}%
\pgfpathlineto{\pgfqpoint{2.749877in}{1.692901in}}%
\pgfpathlineto{\pgfqpoint{2.755166in}{1.791092in}}%
\pgfpathlineto{\pgfqpoint{2.759913in}{1.829196in}}%
\pgfpathlineto{\pgfqpoint{2.763439in}{1.837886in}}%
\pgfpathlineto{\pgfqpoint{2.765067in}{1.836990in}}%
\pgfpathlineto{\pgfqpoint{2.767373in}{1.830434in}}%
\pgfpathlineto{\pgfqpoint{2.770492in}{1.810734in}}%
\pgfpathlineto{\pgfqpoint{2.774425in}{1.763503in}}%
\pgfpathlineto{\pgfqpoint{2.780799in}{1.649544in}}%
\pgfpathlineto{\pgfqpoint{2.782020in}{1.660948in}}%
\pgfpathlineto{\pgfqpoint{2.792870in}{1.823935in}}%
\pgfpathlineto{\pgfqpoint{2.796668in}{1.837219in}}%
\pgfpathlineto{\pgfqpoint{2.798566in}{1.837539in}}%
\pgfpathlineto{\pgfqpoint{2.798702in}{1.837415in}}%
\pgfpathlineto{\pgfqpoint{2.800601in}{1.833292in}}%
\pgfpathlineto{\pgfqpoint{2.803449in}{1.818835in}}%
\pgfpathlineto{\pgfqpoint{2.807111in}{1.782778in}}%
\pgfpathlineto{\pgfqpoint{2.811315in}{1.707155in}}%
\pgfpathlineto{\pgfqpoint{2.814977in}{1.643413in}}%
\pgfpathlineto{\pgfqpoint{2.815384in}{1.645176in}}%
\pgfpathlineto{\pgfqpoint{2.817011in}{1.673339in}}%
\pgfpathlineto{\pgfqpoint{2.823928in}{1.797406in}}%
\pgfpathlineto{\pgfqpoint{2.828539in}{1.830220in}}%
\pgfpathlineto{\pgfqpoint{2.831794in}{1.836889in}}%
\pgfpathlineto{\pgfqpoint{2.833422in}{1.835658in}}%
\pgfpathlineto{\pgfqpoint{2.835728in}{1.828660in}}%
\pgfpathlineto{\pgfqpoint{2.838983in}{1.807131in}}%
\pgfpathlineto{\pgfqpoint{2.842916in}{1.758331in}}%
\pgfpathlineto{\pgfqpoint{2.848883in}{1.658220in}}%
\pgfpathlineto{\pgfqpoint{2.849968in}{1.665881in}}%
\pgfpathlineto{\pgfqpoint{2.862853in}{1.832408in}}%
\pgfpathlineto{\pgfqpoint{2.865972in}{1.837541in}}%
\pgfpathlineto{\pgfqpoint{2.867599in}{1.835831in}}%
\pgfpathlineto{\pgfqpoint{2.870041in}{1.827434in}}%
\pgfpathlineto{\pgfqpoint{2.873431in}{1.802969in}}%
\pgfpathlineto{\pgfqpoint{2.877364in}{1.750144in}}%
\pgfpathlineto{\pgfqpoint{2.883061in}{1.650798in}}%
\pgfpathlineto{\pgfqpoint{2.884010in}{1.658959in}}%
\pgfpathlineto{\pgfqpoint{2.887401in}{1.728669in}}%
\pgfpathlineto{\pgfqpoint{2.892690in}{1.804737in}}%
\pgfpathlineto{\pgfqpoint{2.897166in}{1.832086in}}%
\pgfpathlineto{\pgfqpoint{2.900149in}{1.836496in}}%
\pgfpathlineto{\pgfqpoint{2.901777in}{1.834604in}}%
\pgfpathlineto{\pgfqpoint{2.904218in}{1.826000in}}%
\pgfpathlineto{\pgfqpoint{2.905168in}{1.820610in}}%
\pgfpathlineto{\pgfqpoint{2.905168in}{1.820610in}}%
\pgfusepath{stroke}%
\end{pgfscope}%
\begin{pgfscope}%
\pgfpathrectangle{\pgfqpoint{0.735032in}{0.526079in}}{\pgfqpoint{2.170000in}{1.661000in}} %
\pgfusepath{clip}%
\pgfsetrectcap%
\pgfsetroundjoin%
\pgfsetlinewidth{1.003750pt}%
\definecolor{currentstroke}{rgb}{1.000000,0.549020,0.000000}%
\pgfsetstrokecolor{currentstroke}%
\pgfsetdash{}{0pt}%
\pgfpathmoveto{\pgfqpoint{0.735167in}{0.512191in}}%
\pgfpathlineto{\pgfqpoint{0.736659in}{1.136871in}}%
\pgfpathlineto{\pgfqpoint{0.739372in}{1.189683in}}%
\pgfpathlineto{\pgfqpoint{0.740999in}{1.195399in}}%
\pgfpathlineto{\pgfqpoint{0.741542in}{1.193452in}}%
\pgfpathlineto{\pgfqpoint{0.742627in}{1.185982in}}%
\pgfpathlineto{\pgfqpoint{0.744526in}{1.149630in}}%
\pgfpathlineto{\pgfqpoint{0.746018in}{1.047164in}}%
\pgfpathlineto{\pgfqpoint{0.746424in}{0.868593in}}%
\pgfpathlineto{\pgfqpoint{0.747238in}{1.098553in}}%
\pgfpathlineto{\pgfqpoint{0.749951in}{1.210327in}}%
\pgfpathlineto{\pgfqpoint{0.754291in}{1.263713in}}%
\pgfpathlineto{\pgfqpoint{0.757546in}{1.274219in}}%
\pgfpathlineto{\pgfqpoint{0.758902in}{1.273824in}}%
\pgfpathlineto{\pgfqpoint{0.759038in}{1.273606in}}%
\pgfpathlineto{\pgfqpoint{0.761072in}{1.265590in}}%
\pgfpathlineto{\pgfqpoint{0.764056in}{1.238635in}}%
\pgfpathlineto{\pgfqpoint{0.767446in}{1.161991in}}%
\pgfpathlineto{\pgfqpoint{0.768938in}{1.033356in}}%
\pgfpathlineto{\pgfqpoint{0.769209in}{0.897785in}}%
\pgfpathlineto{\pgfqpoint{0.770159in}{1.100727in}}%
\pgfpathlineto{\pgfqpoint{0.772600in}{1.178737in}}%
\pgfpathlineto{\pgfqpoint{0.774906in}{1.192026in}}%
\pgfpathlineto{\pgfqpoint{0.776533in}{1.183068in}}%
\pgfpathlineto{\pgfqpoint{0.778296in}{1.153059in}}%
\pgfpathlineto{\pgfqpoint{0.779788in}{1.070163in}}%
\pgfpathlineto{\pgfqpoint{0.780331in}{0.806571in}}%
\pgfpathlineto{\pgfqpoint{0.781144in}{1.096237in}}%
\pgfpathlineto{\pgfqpoint{0.783857in}{1.208658in}}%
\pgfpathlineto{\pgfqpoint{0.787519in}{1.258337in}}%
\pgfpathlineto{\pgfqpoint{0.791452in}{1.274253in}}%
\pgfpathlineto{\pgfqpoint{0.791994in}{1.274832in}}%
\pgfpathlineto{\pgfqpoint{0.792944in}{1.273941in}}%
\pgfpathlineto{\pgfqpoint{0.794571in}{1.269162in}}%
\pgfpathlineto{\pgfqpoint{0.797555in}{1.247173in}}%
\pgfpathlineto{\pgfqpoint{0.800403in}{1.198754in}}%
\pgfpathlineto{\pgfqpoint{0.802709in}{1.089346in}}%
\pgfpathlineto{\pgfqpoint{0.803251in}{0.971504in}}%
\pgfpathlineto{\pgfqpoint{0.803387in}{0.718627in}}%
\pgfpathlineto{\pgfqpoint{0.804608in}{1.117786in}}%
\pgfpathlineto{\pgfqpoint{0.807591in}{1.185034in}}%
\pgfpathlineto{\pgfqpoint{0.809083in}{1.190805in}}%
\pgfpathlineto{\pgfqpoint{0.809490in}{1.189751in}}%
\pgfpathlineto{\pgfqpoint{0.809626in}{1.189888in}}%
\pgfpathlineto{\pgfqpoint{0.809897in}{1.188928in}}%
\pgfpathlineto{\pgfqpoint{0.811796in}{1.167328in}}%
\pgfpathlineto{\pgfqpoint{0.813288in}{1.117184in}}%
\pgfpathlineto{\pgfqpoint{0.814237in}{1.020760in}}%
\pgfpathlineto{\pgfqpoint{0.814508in}{0.805789in}}%
\pgfpathlineto{\pgfqpoint{0.815458in}{1.110824in}}%
\pgfpathlineto{\pgfqpoint{0.818306in}{1.215905in}}%
\pgfpathlineto{\pgfqpoint{0.822103in}{1.261053in}}%
\pgfpathlineto{\pgfqpoint{0.825494in}{1.273741in}}%
\pgfpathlineto{\pgfqpoint{0.826308in}{1.274198in}}%
\pgfpathlineto{\pgfqpoint{0.826986in}{1.273133in}}%
\pgfpathlineto{\pgfqpoint{0.828206in}{1.269892in}}%
\pgfpathlineto{\pgfqpoint{0.829969in}{1.260214in}}%
\pgfpathlineto{\pgfqpoint{0.833631in}{1.214604in}}%
\pgfpathlineto{\pgfqpoint{0.835801in}{1.145764in}}%
\pgfpathlineto{\pgfqpoint{0.837022in}{1.006085in}}%
\pgfpathlineto{\pgfqpoint{0.837158in}{0.907454in}}%
\pgfpathlineto{\pgfqpoint{0.838378in}{1.121147in}}%
\pgfpathlineto{\pgfqpoint{0.840955in}{1.187353in}}%
\pgfpathlineto{\pgfqpoint{0.842718in}{1.197828in}}%
\pgfpathlineto{\pgfqpoint{0.843396in}{1.196725in}}%
\pgfpathlineto{\pgfqpoint{0.844753in}{1.191254in}}%
\pgfpathlineto{\pgfqpoint{0.846923in}{1.156626in}}%
\pgfpathlineto{\pgfqpoint{0.848686in}{1.024882in}}%
\pgfpathlineto{\pgfqpoint{0.848957in}{0.910956in}}%
\pgfpathlineto{\pgfqpoint{0.849906in}{1.112173in}}%
\pgfpathlineto{\pgfqpoint{0.852890in}{1.219761in}}%
\pgfpathlineto{\pgfqpoint{0.857637in}{1.266781in}}%
\pgfpathlineto{\pgfqpoint{0.860485in}{1.272328in}}%
\pgfpathlineto{\pgfqpoint{0.862384in}{1.267275in}}%
\pgfpathlineto{\pgfqpoint{0.864554in}{1.253848in}}%
\pgfpathlineto{\pgfqpoint{0.867402in}{1.217072in}}%
\pgfpathlineto{\pgfqpoint{0.869708in}{1.147400in}}%
\pgfpathlineto{\pgfqpoint{0.870928in}{1.022552in}}%
\pgfpathlineto{\pgfqpoint{0.871199in}{0.827038in}}%
\pgfpathlineto{\pgfqpoint{0.872149in}{1.099907in}}%
\pgfpathlineto{\pgfqpoint{0.874861in}{1.187541in}}%
\pgfpathlineto{\pgfqpoint{0.876896in}{1.200765in}}%
\pgfpathlineto{\pgfqpoint{0.877167in}{1.200089in}}%
\pgfpathlineto{\pgfqpoint{0.877845in}{1.200144in}}%
\pgfpathlineto{\pgfqpoint{0.877981in}{1.199340in}}%
\pgfpathlineto{\pgfqpoint{0.880964in}{1.155868in}}%
\pgfpathlineto{\pgfqpoint{0.882728in}{1.033480in}}%
\pgfpathlineto{\pgfqpoint{0.883134in}{0.891915in}}%
\pgfpathlineto{\pgfqpoint{0.883948in}{1.109070in}}%
\pgfpathlineto{\pgfqpoint{0.887203in}{1.222508in}}%
\pgfpathlineto{\pgfqpoint{0.890865in}{1.261692in}}%
\pgfpathlineto{\pgfqpoint{0.894120in}{1.271585in}}%
\pgfpathlineto{\pgfqpoint{0.894663in}{1.271490in}}%
\pgfpathlineto{\pgfqpoint{0.894934in}{1.270729in}}%
\pgfpathlineto{\pgfqpoint{0.898189in}{1.256291in}}%
\pgfpathlineto{\pgfqpoint{0.900766in}{1.227868in}}%
\pgfpathlineto{\pgfqpoint{0.903207in}{1.168110in}}%
\pgfpathlineto{\pgfqpoint{0.904699in}{1.072131in}}%
\pgfpathlineto{\pgfqpoint{0.905241in}{0.885298in}}%
\pgfpathlineto{\pgfqpoint{0.906055in}{1.100135in}}%
\pgfpathlineto{\pgfqpoint{0.908903in}{1.189427in}}%
\pgfpathlineto{\pgfqpoint{0.910802in}{1.201419in}}%
\pgfpathlineto{\pgfqpoint{0.910938in}{1.201374in}}%
\pgfpathlineto{\pgfqpoint{0.911751in}{1.201526in}}%
\pgfpathlineto{\pgfqpoint{0.912023in}{1.200270in}}%
\pgfpathlineto{\pgfqpoint{0.914871in}{1.164396in}}%
\pgfpathlineto{\pgfqpoint{0.916634in}{1.069236in}}%
\pgfpathlineto{\pgfqpoint{0.917176in}{0.886321in}}%
\pgfpathlineto{\pgfqpoint{0.917854in}{1.081818in}}%
\pgfpathlineto{\pgfqpoint{0.920431in}{1.203019in}}%
\pgfpathlineto{\pgfqpoint{0.923958in}{1.254525in}}%
\pgfpathlineto{\pgfqpoint{0.927755in}{1.270984in}}%
\pgfpathlineto{\pgfqpoint{0.928162in}{1.271199in}}%
\pgfpathlineto{\pgfqpoint{0.928704in}{1.270234in}}%
\pgfpathlineto{\pgfqpoint{0.928976in}{1.270191in}}%
\pgfpathlineto{\pgfqpoint{0.930196in}{1.267564in}}%
\pgfpathlineto{\pgfqpoint{0.933316in}{1.248099in}}%
\pgfpathlineto{\pgfqpoint{0.936842in}{1.184881in}}%
\pgfpathlineto{\pgfqpoint{0.938741in}{1.074660in}}%
\pgfpathlineto{\pgfqpoint{0.939283in}{0.859992in}}%
\pgfpathlineto{\pgfqpoint{0.940097in}{1.098286in}}%
\pgfpathlineto{\pgfqpoint{0.942674in}{1.187143in}}%
\pgfpathlineto{\pgfqpoint{0.944844in}{1.203408in}}%
\pgfpathlineto{\pgfqpoint{0.945386in}{1.202869in}}%
\pgfpathlineto{\pgfqpoint{0.945658in}{1.203374in}}%
\pgfpathlineto{\pgfqpoint{0.946471in}{1.200335in}}%
\pgfpathlineto{\pgfqpoint{0.947285in}{1.194578in}}%
\pgfpathlineto{\pgfqpoint{0.949862in}{1.145802in}}%
\pgfpathlineto{\pgfqpoint{0.951218in}{1.017790in}}%
\pgfpathlineto{\pgfqpoint{0.951489in}{0.749190in}}%
\pgfpathlineto{\pgfqpoint{0.952439in}{1.113144in}}%
\pgfpathlineto{\pgfqpoint{0.955287in}{1.217648in}}%
\pgfpathlineto{\pgfqpoint{0.959084in}{1.260587in}}%
\pgfpathlineto{\pgfqpoint{0.961933in}{1.270646in}}%
\pgfpathlineto{\pgfqpoint{0.962339in}{1.270551in}}%
\pgfpathlineto{\pgfqpoint{0.962746in}{1.270482in}}%
\pgfpathlineto{\pgfqpoint{0.963424in}{1.269447in}}%
\pgfpathlineto{\pgfqpoint{0.963967in}{1.267830in}}%
\pgfpathlineto{\pgfqpoint{0.966815in}{1.251934in}}%
\pgfpathlineto{\pgfqpoint{0.970477in}{1.197082in}}%
\pgfpathlineto{\pgfqpoint{0.972511in}{1.111659in}}%
\pgfpathlineto{\pgfqpoint{0.973325in}{0.871693in}}%
\pgfpathlineto{\pgfqpoint{0.974139in}{1.085996in}}%
\pgfpathlineto{\pgfqpoint{0.976851in}{1.182677in}}%
\pgfpathlineto{\pgfqpoint{0.978614in}{1.198335in}}%
\pgfpathlineto{\pgfqpoint{0.979428in}{1.198174in}}%
\pgfpathlineto{\pgfqpoint{0.979699in}{1.199215in}}%
\pgfpathlineto{\pgfqpoint{0.980242in}{1.196268in}}%
\pgfpathlineto{\pgfqpoint{0.983090in}{1.161295in}}%
\pgfpathlineto{\pgfqpoint{0.984853in}{1.062648in}}%
\pgfpathlineto{\pgfqpoint{0.985396in}{0.889244in}}%
\pgfpathlineto{\pgfqpoint{0.986074in}{1.097021in}}%
\pgfpathlineto{\pgfqpoint{0.989058in}{1.211878in}}%
\pgfpathlineto{\pgfqpoint{0.992719in}{1.258821in}}%
\pgfpathlineto{\pgfqpoint{0.996381in}{1.271060in}}%
\pgfpathlineto{\pgfqpoint{0.996924in}{1.271274in}}%
\pgfpathlineto{\pgfqpoint{0.997602in}{1.270501in}}%
\pgfpathlineto{\pgfqpoint{1.000043in}{1.260415in}}%
\pgfpathlineto{\pgfqpoint{1.003163in}{1.226520in}}%
\pgfpathlineto{\pgfqpoint{1.006011in}{1.148187in}}%
\pgfpathlineto{\pgfqpoint{1.007231in}{1.016916in}}%
\pgfpathlineto{\pgfqpoint{1.007503in}{0.777696in}}%
\pgfpathlineto{\pgfqpoint{1.008452in}{1.103389in}}%
\pgfpathlineto{\pgfqpoint{1.011029in}{1.186769in}}%
\pgfpathlineto{\pgfqpoint{1.013063in}{1.198835in}}%
\pgfpathlineto{\pgfqpoint{1.013199in}{1.198758in}}%
\pgfpathlineto{\pgfqpoint{1.013334in}{1.198455in}}%
\pgfpathlineto{\pgfqpoint{1.013741in}{1.200969in}}%
\pgfpathlineto{\pgfqpoint{1.014013in}{1.202037in}}%
\pgfpathlineto{\pgfqpoint{1.014691in}{1.197445in}}%
\pgfpathlineto{\pgfqpoint{1.015776in}{1.188340in}}%
\pgfpathlineto{\pgfqpoint{1.017810in}{1.150161in}}%
\pgfpathlineto{\pgfqpoint{1.019031in}{1.069124in}}%
\pgfpathlineto{\pgfqpoint{1.019573in}{0.772580in}}%
\pgfpathlineto{\pgfqpoint{1.020251in}{1.090490in}}%
\pgfpathlineto{\pgfqpoint{1.023235in}{1.215751in}}%
\pgfpathlineto{\pgfqpoint{1.027033in}{1.260408in}}%
\pgfpathlineto{\pgfqpoint{1.030288in}{1.271895in}}%
\pgfpathlineto{\pgfqpoint{1.030694in}{1.271539in}}%
\pgfpathlineto{\pgfqpoint{1.031915in}{1.270317in}}%
\pgfpathlineto{\pgfqpoint{1.035713in}{1.246355in}}%
\pgfpathlineto{\pgfqpoint{1.039646in}{1.168524in}}%
\pgfpathlineto{\pgfqpoint{1.041273in}{1.011653in}}%
\pgfpathlineto{\pgfqpoint{1.041544in}{0.814726in}}%
\pgfpathlineto{\pgfqpoint{1.042494in}{1.102229in}}%
\pgfpathlineto{\pgfqpoint{1.045071in}{1.183381in}}%
\pgfpathlineto{\pgfqpoint{1.047783in}{1.199685in}}%
\pgfpathlineto{\pgfqpoint{1.048054in}{1.200113in}}%
\pgfpathlineto{\pgfqpoint{1.048326in}{1.197347in}}%
\pgfpathlineto{\pgfqpoint{1.051038in}{1.166547in}}%
\pgfpathlineto{\pgfqpoint{1.052937in}{1.071966in}}%
\pgfpathlineto{\pgfqpoint{1.053479in}{0.902234in}}%
\pgfpathlineto{\pgfqpoint{1.054293in}{1.099744in}}%
\pgfpathlineto{\pgfqpoint{1.057277in}{1.218348in}}%
\pgfpathlineto{\pgfqpoint{1.061753in}{1.266689in}}%
\pgfpathlineto{\pgfqpoint{1.064058in}{1.272937in}}%
\pgfpathlineto{\pgfqpoint{1.064329in}{1.272429in}}%
\pgfpathlineto{\pgfqpoint{1.064736in}{1.272709in}}%
\pgfpathlineto{\pgfqpoint{1.068127in}{1.262102in}}%
\pgfpathlineto{\pgfqpoint{1.071246in}{1.229107in}}%
\pgfpathlineto{\pgfqpoint{1.073959in}{1.161699in}}%
\pgfpathlineto{\pgfqpoint{1.075586in}{1.038230in}}%
\pgfpathlineto{\pgfqpoint{1.075858in}{0.893978in}}%
\pgfpathlineto{\pgfqpoint{1.076807in}{1.104612in}}%
\pgfpathlineto{\pgfqpoint{1.079248in}{1.184481in}}%
\pgfpathlineto{\pgfqpoint{1.079926in}{1.189876in}}%
\pgfpathlineto{\pgfqpoint{1.081689in}{1.198071in}}%
\pgfpathlineto{\pgfqpoint{1.081961in}{1.197135in}}%
\pgfpathlineto{\pgfqpoint{1.084809in}{1.168466in}}%
\pgfpathlineto{\pgfqpoint{1.086979in}{1.059472in}}%
\pgfpathlineto{\pgfqpoint{1.087521in}{0.884090in}}%
\pgfpathlineto{\pgfqpoint{1.088199in}{1.096652in}}%
\pgfpathlineto{\pgfqpoint{1.091183in}{1.215425in}}%
\pgfpathlineto{\pgfqpoint{1.096744in}{1.270156in}}%
\pgfpathlineto{\pgfqpoint{1.098914in}{1.272785in}}%
\pgfpathlineto{\pgfqpoint{1.099185in}{1.273078in}}%
\pgfpathlineto{\pgfqpoint{1.099863in}{1.271390in}}%
\pgfpathlineto{\pgfqpoint{1.102711in}{1.258155in}}%
\pgfpathlineto{\pgfqpoint{1.106102in}{1.214754in}}%
\pgfpathlineto{\pgfqpoint{1.108814in}{1.115154in}}%
\pgfpathlineto{\pgfqpoint{1.109764in}{0.913604in}}%
\pgfpathlineto{\pgfqpoint{1.110578in}{1.088552in}}%
\pgfpathlineto{\pgfqpoint{1.113154in}{1.182189in}}%
\pgfpathlineto{\pgfqpoint{1.116003in}{1.199489in}}%
\pgfpathlineto{\pgfqpoint{1.116409in}{1.198361in}}%
\pgfpathlineto{\pgfqpoint{1.117494in}{1.191272in}}%
\pgfpathlineto{\pgfqpoint{1.120071in}{1.130871in}}%
\pgfpathlineto{\pgfqpoint{1.121292in}{0.998813in}}%
\pgfpathlineto{\pgfqpoint{1.121428in}{0.899295in}}%
\pgfpathlineto{\pgfqpoint{1.122648in}{1.127980in}}%
\pgfpathlineto{\pgfqpoint{1.126039in}{1.231112in}}%
\pgfpathlineto{\pgfqpoint{1.129294in}{1.262247in}}%
\pgfpathlineto{\pgfqpoint{1.132006in}{1.271779in}}%
\pgfpathlineto{\pgfqpoint{1.132549in}{1.271025in}}%
\pgfpathlineto{\pgfqpoint{1.133498in}{1.272447in}}%
\pgfpathlineto{\pgfqpoint{1.134176in}{1.270644in}}%
\pgfpathlineto{\pgfqpoint{1.136346in}{1.261004in}}%
\pgfpathlineto{\pgfqpoint{1.139059in}{1.234001in}}%
\pgfpathlineto{\pgfqpoint{1.141907in}{1.165419in}}%
\pgfpathlineto{\pgfqpoint{1.143670in}{1.016984in}}%
\pgfpathlineto{\pgfqpoint{1.143941in}{0.919178in}}%
\pgfpathlineto{\pgfqpoint{1.144891in}{1.118136in}}%
\pgfpathlineto{\pgfqpoint{1.147468in}{1.189052in}}%
\pgfpathlineto{\pgfqpoint{1.149773in}{1.203756in}}%
\pgfpathlineto{\pgfqpoint{1.149909in}{1.203740in}}%
\pgfpathlineto{\pgfqpoint{1.152893in}{1.181848in}}%
\pgfpathlineto{\pgfqpoint{1.155063in}{1.106505in}}%
\pgfpathlineto{\pgfqpoint{1.155741in}{0.970100in}}%
\pgfpathlineto{\pgfqpoint{1.155876in}{0.664419in}}%
\pgfpathlineto{\pgfqpoint{1.157097in}{1.133303in}}%
\pgfpathlineto{\pgfqpoint{1.162793in}{1.256705in}}%
\pgfpathlineto{\pgfqpoint{1.166048in}{1.269913in}}%
\pgfpathlineto{\pgfqpoint{1.167676in}{1.270030in}}%
\pgfpathlineto{\pgfqpoint{1.167947in}{1.268824in}}%
\pgfpathlineto{\pgfqpoint{1.172558in}{1.241579in}}%
\pgfpathlineto{\pgfqpoint{1.175813in}{1.176222in}}%
\pgfpathlineto{\pgfqpoint{1.177576in}{1.065989in}}%
\pgfpathlineto{\pgfqpoint{1.178119in}{0.815747in}}%
\pgfpathlineto{\pgfqpoint{1.178797in}{1.089783in}}%
\pgfpathlineto{\pgfqpoint{1.182052in}{1.188877in}}%
\pgfpathlineto{\pgfqpoint{1.183679in}{1.197765in}}%
\pgfpathlineto{\pgfqpoint{1.184086in}{1.197026in}}%
\pgfpathlineto{\pgfqpoint{1.184222in}{1.197510in}}%
\pgfpathlineto{\pgfqpoint{1.184900in}{1.194517in}}%
\pgfpathlineto{\pgfqpoint{1.187613in}{1.160640in}}%
\pgfpathlineto{\pgfqpoint{1.189104in}{1.077329in}}%
\pgfpathlineto{\pgfqpoint{1.189783in}{0.874510in}}%
\pgfpathlineto{\pgfqpoint{1.190461in}{1.087366in}}%
\pgfpathlineto{\pgfqpoint{1.193173in}{1.210889in}}%
\pgfpathlineto{\pgfqpoint{1.197106in}{1.261395in}}%
\pgfpathlineto{\pgfqpoint{1.200226in}{1.273400in}}%
\pgfpathlineto{\pgfqpoint{1.201039in}{1.273202in}}%
\pgfpathlineto{\pgfqpoint{1.202938in}{1.269351in}}%
\pgfpathlineto{\pgfqpoint{1.206058in}{1.247421in}}%
\pgfpathlineto{\pgfqpoint{1.208363in}{1.214211in}}%
\pgfpathlineto{\pgfqpoint{1.210533in}{1.146799in}}%
\pgfpathlineto{\pgfqpoint{1.211754in}{1.017929in}}%
\pgfpathlineto{\pgfqpoint{1.212025in}{0.819260in}}%
\pgfpathlineto{\pgfqpoint{1.212974in}{1.121128in}}%
\pgfpathlineto{\pgfqpoint{1.216772in}{1.202983in}}%
\pgfpathlineto{\pgfqpoint{1.217179in}{1.202417in}}%
\pgfpathlineto{\pgfqpoint{1.217450in}{1.204746in}}%
\pgfpathlineto{\pgfqpoint{1.217857in}{1.207943in}}%
\pgfpathlineto{\pgfqpoint{1.218264in}{1.204140in}}%
\pgfpathlineto{\pgfqpoint{1.218942in}{1.205367in}}%
\pgfpathlineto{\pgfqpoint{1.222875in}{1.138524in}}%
\pgfpathlineto{\pgfqpoint{1.224231in}{0.967343in}}%
\pgfpathlineto{\pgfqpoint{1.224367in}{0.853790in}}%
\pgfpathlineto{\pgfqpoint{1.225452in}{1.124141in}}%
\pgfpathlineto{\pgfqpoint{1.228029in}{1.212831in}}%
\pgfpathlineto{\pgfqpoint{1.231691in}{1.256737in}}%
\pgfpathlineto{\pgfqpoint{1.234539in}{1.267405in}}%
\pgfpathlineto{\pgfqpoint{1.235488in}{1.267039in}}%
\pgfpathlineto{\pgfqpoint{1.235895in}{1.265363in}}%
\pgfpathlineto{\pgfqpoint{1.236438in}{1.265590in}}%
\pgfpathlineto{\pgfqpoint{1.238879in}{1.253337in}}%
\pgfpathlineto{\pgfqpoint{1.241998in}{1.213988in}}%
\pgfpathlineto{\pgfqpoint{1.244439in}{1.130575in}}%
\pgfpathlineto{\pgfqpoint{1.245660in}{0.899615in}}%
\pgfpathlineto{\pgfqpoint{1.246203in}{1.099180in}}%
\pgfpathlineto{\pgfqpoint{1.250814in}{1.209756in}}%
\pgfpathlineto{\pgfqpoint{1.251356in}{1.216703in}}%
\pgfpathlineto{\pgfqpoint{1.252577in}{1.214122in}}%
\pgfpathlineto{\pgfqpoint{1.253255in}{1.209951in}}%
\pgfpathlineto{\pgfqpoint{1.257459in}{1.125973in}}%
\pgfpathlineto{\pgfqpoint{1.258409in}{1.003313in}}%
\pgfpathlineto{\pgfqpoint{1.258544in}{0.814791in}}%
\pgfpathlineto{\pgfqpoint{1.259765in}{1.129207in}}%
\pgfpathlineto{\pgfqpoint{1.263427in}{1.232437in}}%
\pgfpathlineto{\pgfqpoint{1.265868in}{1.256763in}}%
\pgfpathlineto{\pgfqpoint{1.267903in}{1.265802in}}%
\pgfpathlineto{\pgfqpoint{1.268309in}{1.264015in}}%
\pgfpathlineto{\pgfqpoint{1.268445in}{1.263962in}}%
\pgfpathlineto{\pgfqpoint{1.268581in}{1.264970in}}%
\pgfpathlineto{\pgfqpoint{1.268852in}{1.266358in}}%
\pgfpathlineto{\pgfqpoint{1.270073in}{1.265278in}}%
\pgfpathlineto{\pgfqpoint{1.272921in}{1.251823in}}%
\pgfpathlineto{\pgfqpoint{1.277532in}{1.163720in}}%
\pgfpathlineto{\pgfqpoint{1.279295in}{1.032909in}}%
\pgfpathlineto{\pgfqpoint{1.279566in}{0.882226in}}%
\pgfpathlineto{\pgfqpoint{1.280516in}{1.118243in}}%
\pgfpathlineto{\pgfqpoint{1.283771in}{1.203730in}}%
\pgfpathlineto{\pgfqpoint{1.284313in}{1.205039in}}%
\pgfpathlineto{\pgfqpoint{1.286619in}{1.215848in}}%
\pgfpathlineto{\pgfqpoint{1.291773in}{1.129207in}}%
\pgfpathlineto{\pgfqpoint{1.292858in}{0.963643in}}%
\pgfpathlineto{\pgfqpoint{1.292993in}{0.860932in}}%
\pgfpathlineto{\pgfqpoint{1.294078in}{1.139090in}}%
\pgfpathlineto{\pgfqpoint{1.298418in}{1.243735in}}%
\pgfpathlineto{\pgfqpoint{1.301944in}{1.263465in}}%
\pgfpathlineto{\pgfqpoint{1.302758in}{1.262779in}}%
\pgfpathlineto{\pgfqpoint{1.302894in}{1.263434in}}%
\pgfpathlineto{\pgfqpoint{1.303301in}{1.265442in}}%
\pgfpathlineto{\pgfqpoint{1.303708in}{1.261653in}}%
\pgfpathlineto{\pgfqpoint{1.304386in}{1.264097in}}%
\pgfpathlineto{\pgfqpoint{1.310489in}{1.195267in}}%
\pgfpathlineto{\pgfqpoint{1.312523in}{1.092782in}}%
\pgfpathlineto{\pgfqpoint{1.312930in}{0.881528in}}%
\pgfpathlineto{\pgfqpoint{1.313879in}{1.107581in}}%
\pgfpathlineto{\pgfqpoint{1.316049in}{1.190353in}}%
\pgfpathlineto{\pgfqpoint{1.319169in}{1.224138in}}%
\pgfpathlineto{\pgfqpoint{1.319440in}{1.223180in}}%
\pgfpathlineto{\pgfqpoint{1.319983in}{1.224480in}}%
\pgfpathlineto{\pgfqpoint{1.322153in}{1.219555in}}%
\pgfpathlineto{\pgfqpoint{1.326493in}{1.110415in}}%
\pgfpathlineto{\pgfqpoint{1.327442in}{0.883782in}}%
\pgfpathlineto{\pgfqpoint{1.328120in}{1.090155in}}%
\pgfpathlineto{\pgfqpoint{1.331918in}{1.229035in}}%
\pgfpathlineto{\pgfqpoint{1.332189in}{1.228141in}}%
\pgfpathlineto{\pgfqpoint{1.332324in}{1.229372in}}%
\pgfpathlineto{\pgfqpoint{1.334766in}{1.253745in}}%
\pgfpathlineto{\pgfqpoint{1.335037in}{1.253563in}}%
\pgfpathlineto{\pgfqpoint{1.337614in}{1.260404in}}%
\pgfpathlineto{\pgfqpoint{1.338699in}{1.259093in}}%
\pgfpathlineto{\pgfqpoint{1.345209in}{1.172893in}}%
\pgfpathlineto{\pgfqpoint{1.347108in}{1.042129in}}%
\pgfpathlineto{\pgfqpoint{1.347379in}{0.832222in}}%
\pgfpathlineto{\pgfqpoint{1.348328in}{1.124479in}}%
\pgfpathlineto{\pgfqpoint{1.351041in}{1.203283in}}%
\pgfpathlineto{\pgfqpoint{1.353482in}{1.224861in}}%
\pgfpathlineto{\pgfqpoint{1.353618in}{1.222175in}}%
\pgfpathlineto{\pgfqpoint{1.354024in}{1.217115in}}%
\pgfpathlineto{\pgfqpoint{1.354567in}{1.225306in}}%
\pgfpathlineto{\pgfqpoint{1.355245in}{1.218792in}}%
\pgfpathlineto{\pgfqpoint{1.355516in}{1.220869in}}%
\pgfpathlineto{\pgfqpoint{1.356059in}{1.214898in}}%
\pgfpathlineto{\pgfqpoint{1.356330in}{1.215184in}}%
\pgfpathlineto{\pgfqpoint{1.358636in}{1.187029in}}%
\pgfpathlineto{\pgfqpoint{1.361348in}{1.001556in}}%
\pgfpathlineto{\pgfqpoint{1.361619in}{0.938130in}}%
\pgfpathlineto{\pgfqpoint{1.362569in}{1.109052in}}%
\pgfpathlineto{\pgfqpoint{1.365281in}{1.214211in}}%
\pgfpathlineto{\pgfqpoint{1.368401in}{1.252836in}}%
\pgfpathlineto{\pgfqpoint{1.368536in}{1.252426in}}%
\pgfpathlineto{\pgfqpoint{1.368672in}{1.252306in}}%
\pgfpathlineto{\pgfqpoint{1.368808in}{1.253234in}}%
\pgfpathlineto{\pgfqpoint{1.371113in}{1.263579in}}%
\pgfpathlineto{\pgfqpoint{1.375453in}{1.247681in}}%
\pgfpathlineto{\pgfqpoint{1.380336in}{1.150915in}}%
\pgfpathlineto{\pgfqpoint{1.381556in}{1.015220in}}%
\pgfpathlineto{\pgfqpoint{1.381692in}{0.851150in}}%
\pgfpathlineto{\pgfqpoint{1.382913in}{1.123482in}}%
\pgfpathlineto{\pgfqpoint{1.385761in}{1.193810in}}%
\pgfpathlineto{\pgfqpoint{1.387795in}{1.206234in}}%
\pgfpathlineto{\pgfqpoint{1.386303in}{1.192987in}}%
\pgfpathlineto{\pgfqpoint{1.387931in}{1.205207in}}%
\pgfpathlineto{\pgfqpoint{1.389423in}{1.199436in}}%
\pgfpathlineto{\pgfqpoint{1.388744in}{1.208451in}}%
\pgfpathlineto{\pgfqpoint{1.389558in}{1.201852in}}%
\pgfpathlineto{\pgfqpoint{1.389829in}{1.205136in}}%
\pgfpathlineto{\pgfqpoint{1.390236in}{1.197012in}}%
\pgfpathlineto{\pgfqpoint{1.394169in}{1.073341in}}%
\pgfpathlineto{\pgfqpoint{1.394983in}{0.864306in}}%
\pgfpathlineto{\pgfqpoint{1.395526in}{1.111652in}}%
\pgfpathlineto{\pgfqpoint{1.397967in}{1.201574in}}%
\pgfpathlineto{\pgfqpoint{1.401764in}{1.255183in}}%
\pgfpathlineto{\pgfqpoint{1.402036in}{1.256618in}}%
\pgfpathlineto{\pgfqpoint{1.404613in}{1.267848in}}%
\pgfpathlineto{\pgfqpoint{1.407054in}{1.262149in}}%
\pgfpathlineto{\pgfqpoint{1.407189in}{1.262745in}}%
\pgfpathlineto{\pgfqpoint{1.407461in}{1.263890in}}%
\pgfpathlineto{\pgfqpoint{1.407868in}{1.259829in}}%
\pgfpathlineto{\pgfqpoint{1.411529in}{1.230368in}}%
\pgfpathlineto{\pgfqpoint{1.412886in}{1.205498in}}%
\pgfpathlineto{\pgfqpoint{1.415463in}{1.071342in}}%
\pgfpathlineto{\pgfqpoint{1.415869in}{0.939016in}}%
\pgfpathlineto{\pgfqpoint{1.416683in}{1.084533in}}%
\pgfpathlineto{\pgfqpoint{1.419124in}{1.188593in}}%
\pgfpathlineto{\pgfqpoint{1.421430in}{1.212458in}}%
\pgfpathlineto{\pgfqpoint{1.421837in}{1.207140in}}%
\pgfpathlineto{\pgfqpoint{1.422379in}{1.215976in}}%
\pgfpathlineto{\pgfqpoint{1.422922in}{1.209943in}}%
\pgfpathlineto{\pgfqpoint{1.423329in}{1.212680in}}%
\pgfpathlineto{\pgfqpoint{1.424007in}{1.207000in}}%
\pgfpathlineto{\pgfqpoint{1.424278in}{1.209153in}}%
\pgfpathlineto{\pgfqpoint{1.428347in}{1.111513in}}%
\pgfpathlineto{\pgfqpoint{1.429161in}{0.977203in}}%
\pgfpathlineto{\pgfqpoint{1.429296in}{0.755672in}}%
\pgfpathlineto{\pgfqpoint{1.430517in}{1.128003in}}%
\pgfpathlineto{\pgfqpoint{1.432687in}{1.211603in}}%
\pgfpathlineto{\pgfqpoint{1.435942in}{1.251653in}}%
\pgfpathlineto{\pgfqpoint{1.438926in}{1.265138in}}%
\pgfpathlineto{\pgfqpoint{1.440011in}{1.266125in}}%
\pgfpathlineto{\pgfqpoint{1.443401in}{1.247800in}}%
\pgfpathlineto{\pgfqpoint{1.447470in}{1.173939in}}%
\pgfpathlineto{\pgfqpoint{1.449369in}{1.035025in}}%
\pgfpathlineto{\pgfqpoint{1.449504in}{0.856296in}}%
\pgfpathlineto{\pgfqpoint{1.450725in}{1.145744in}}%
\pgfpathlineto{\pgfqpoint{1.450861in}{1.145040in}}%
\pgfpathlineto{\pgfqpoint{1.450996in}{1.148539in}}%
\pgfpathlineto{\pgfqpoint{1.454658in}{1.217912in}}%
\pgfpathlineto{\pgfqpoint{1.456557in}{1.223259in}}%
\pgfpathlineto{\pgfqpoint{1.456964in}{1.219487in}}%
\pgfpathlineto{\pgfqpoint{1.457642in}{1.221426in}}%
\pgfpathlineto{\pgfqpoint{1.461982in}{1.156636in}}%
\pgfpathlineto{\pgfqpoint{1.463609in}{0.991442in}}%
\pgfpathlineto{\pgfqpoint{1.463881in}{0.929488in}}%
\pgfpathlineto{\pgfqpoint{1.464830in}{1.118806in}}%
\pgfpathlineto{\pgfqpoint{1.467543in}{1.213703in}}%
\pgfpathlineto{\pgfqpoint{1.473510in}{1.260010in}}%
\pgfpathlineto{\pgfqpoint{1.467814in}{1.213621in}}%
\pgfpathlineto{\pgfqpoint{1.474188in}{1.258250in}}%
\pgfpathlineto{\pgfqpoint{1.474324in}{1.258961in}}%
\pgfpathlineto{\pgfqpoint{1.474731in}{1.254440in}}%
\pgfpathlineto{\pgfqpoint{1.475273in}{1.256185in}}%
\pgfpathlineto{\pgfqpoint{1.477172in}{1.244540in}}%
\pgfpathlineto{\pgfqpoint{1.478528in}{1.232175in}}%
\pgfpathlineto{\pgfqpoint{1.483411in}{1.058397in}}%
\pgfpathlineto{\pgfqpoint{1.483818in}{0.810194in}}%
\pgfpathlineto{\pgfqpoint{1.484631in}{1.092851in}}%
\pgfpathlineto{\pgfqpoint{1.487073in}{1.182396in}}%
\pgfpathlineto{\pgfqpoint{1.487479in}{1.184496in}}%
\pgfpathlineto{\pgfqpoint{1.487615in}{1.184315in}}%
\pgfpathlineto{\pgfqpoint{1.490328in}{1.217607in}}%
\pgfpathlineto{\pgfqpoint{1.491684in}{1.211366in}}%
\pgfpathlineto{\pgfqpoint{1.491955in}{1.215608in}}%
\pgfpathlineto{\pgfqpoint{1.492091in}{1.216332in}}%
\pgfpathlineto{\pgfqpoint{1.492498in}{1.210259in}}%
\pgfpathlineto{\pgfqpoint{1.493176in}{1.214663in}}%
\pgfpathlineto{\pgfqpoint{1.496295in}{1.168449in}}%
\pgfpathlineto{\pgfqpoint{1.498465in}{0.930550in}}%
\pgfpathlineto{\pgfqpoint{1.499008in}{1.068341in}}%
\pgfpathlineto{\pgfqpoint{1.503212in}{1.227469in}}%
\pgfpathlineto{\pgfqpoint{1.503348in}{1.227278in}}%
\pgfpathlineto{\pgfqpoint{1.507009in}{1.250843in}}%
\pgfpathlineto{\pgfqpoint{1.507281in}{1.249589in}}%
\pgfpathlineto{\pgfqpoint{1.512434in}{1.219228in}}%
\pgfpathlineto{\pgfqpoint{1.515147in}{1.170536in}}%
\pgfpathlineto{\pgfqpoint{1.517181in}{0.977668in}}%
\pgfpathlineto{\pgfqpoint{1.517724in}{1.096670in}}%
\pgfpathlineto{\pgfqpoint{1.520708in}{1.209948in}}%
\pgfpathlineto{\pgfqpoint{1.520979in}{1.208742in}}%
\pgfpathlineto{\pgfqpoint{1.521521in}{1.218635in}}%
\pgfpathlineto{\pgfqpoint{1.524505in}{1.235137in}}%
\pgfpathlineto{\pgfqpoint{1.521928in}{1.217826in}}%
\pgfpathlineto{\pgfqpoint{1.524912in}{1.231659in}}%
\pgfpathlineto{\pgfqpoint{1.526133in}{1.229351in}}%
\pgfpathlineto{\pgfqpoint{1.525590in}{1.235602in}}%
\pgfpathlineto{\pgfqpoint{1.526539in}{1.230873in}}%
\pgfpathlineto{\pgfqpoint{1.526675in}{1.231956in}}%
\pgfpathlineto{\pgfqpoint{1.527353in}{1.227379in}}%
\pgfpathlineto{\pgfqpoint{1.530337in}{1.181092in}}%
\pgfpathlineto{\pgfqpoint{1.532236in}{1.095694in}}%
\pgfpathlineto{\pgfqpoint{1.532914in}{0.924192in}}%
\pgfpathlineto{\pgfqpoint{1.533728in}{1.074159in}}%
\pgfpathlineto{\pgfqpoint{1.535898in}{1.185129in}}%
\pgfpathlineto{\pgfqpoint{1.540509in}{1.237474in}}%
\pgfpathlineto{\pgfqpoint{1.540780in}{1.235047in}}%
\pgfpathlineto{\pgfqpoint{1.540916in}{1.234880in}}%
\pgfpathlineto{\pgfqpoint{1.541051in}{1.236410in}}%
\pgfpathlineto{\pgfqpoint{1.542272in}{1.239731in}}%
\pgfpathlineto{\pgfqpoint{1.542543in}{1.236450in}}%
\pgfpathlineto{\pgfqpoint{1.543899in}{1.230817in}}%
\pgfpathlineto{\pgfqpoint{1.543357in}{1.238936in}}%
\pgfpathlineto{\pgfqpoint{1.544306in}{1.232158in}}%
\pgfpathlineto{\pgfqpoint{1.546341in}{1.210208in}}%
\pgfpathlineto{\pgfqpoint{1.549731in}{1.033826in}}%
\pgfpathlineto{\pgfqpoint{1.550003in}{0.640056in}}%
\pgfpathlineto{\pgfqpoint{1.551088in}{1.128266in}}%
\pgfpathlineto{\pgfqpoint{1.554478in}{1.225679in}}%
\pgfpathlineto{\pgfqpoint{1.554885in}{1.231068in}}%
\pgfpathlineto{\pgfqpoint{1.557191in}{1.250203in}}%
\pgfpathlineto{\pgfqpoint{1.557733in}{1.248949in}}%
\pgfpathlineto{\pgfqpoint{1.558140in}{1.250879in}}%
\pgfpathlineto{\pgfqpoint{1.558818in}{1.247606in}}%
\pgfpathlineto{\pgfqpoint{1.559089in}{1.248918in}}%
\pgfpathlineto{\pgfqpoint{1.561666in}{1.238783in}}%
\pgfpathlineto{\pgfqpoint{1.561938in}{1.240172in}}%
\pgfpathlineto{\pgfqpoint{1.562073in}{1.241250in}}%
\pgfpathlineto{\pgfqpoint{1.562480in}{1.232804in}}%
\pgfpathlineto{\pgfqpoint{1.566278in}{1.148473in}}%
\pgfpathlineto{\pgfqpoint{1.567905in}{0.855501in}}%
\pgfpathlineto{\pgfqpoint{1.568312in}{1.081680in}}%
\pgfpathlineto{\pgfqpoint{1.570618in}{1.193767in}}%
\pgfpathlineto{\pgfqpoint{1.574686in}{1.246167in}}%
\pgfpathlineto{\pgfqpoint{1.575093in}{1.242803in}}%
\pgfpathlineto{\pgfqpoint{1.576585in}{1.239856in}}%
\pgfpathlineto{\pgfqpoint{1.575907in}{1.245086in}}%
\pgfpathlineto{\pgfqpoint{1.576992in}{1.240890in}}%
\pgfpathlineto{\pgfqpoint{1.577263in}{1.239997in}}%
\pgfpathlineto{\pgfqpoint{1.582146in}{1.171759in}}%
\pgfpathlineto{\pgfqpoint{1.584044in}{1.034766in}}%
\pgfpathlineto{\pgfqpoint{1.584316in}{0.928788in}}%
\pgfpathlineto{\pgfqpoint{1.585401in}{1.128561in}}%
\pgfpathlineto{\pgfqpoint{1.585943in}{1.150474in}}%
\pgfpathlineto{\pgfqpoint{1.588520in}{1.220485in}}%
\pgfpathlineto{\pgfqpoint{1.588927in}{1.216903in}}%
\pgfpathlineto{\pgfqpoint{1.589469in}{1.228008in}}%
\pgfpathlineto{\pgfqpoint{1.589605in}{1.227891in}}%
\pgfpathlineto{\pgfqpoint{1.590148in}{1.225488in}}%
\pgfpathlineto{\pgfqpoint{1.590283in}{1.228104in}}%
\pgfpathlineto{\pgfqpoint{1.592453in}{1.240788in}}%
\pgfpathlineto{\pgfqpoint{1.592724in}{1.238580in}}%
\pgfpathlineto{\pgfqpoint{1.594623in}{1.229666in}}%
\pgfpathlineto{\pgfqpoint{1.599234in}{1.158641in}}%
\pgfpathlineto{\pgfqpoint{1.600455in}{1.039383in}}%
\pgfpathlineto{\pgfqpoint{1.600591in}{0.941980in}}%
\pgfpathlineto{\pgfqpoint{1.601811in}{1.128125in}}%
\pgfpathlineto{\pgfqpoint{1.604388in}{1.215768in}}%
\pgfpathlineto{\pgfqpoint{1.608999in}{1.256677in}}%
\pgfpathlineto{\pgfqpoint{1.610084in}{1.261863in}}%
\pgfpathlineto{\pgfqpoint{1.610627in}{1.257475in}}%
\pgfpathlineto{\pgfqpoint{1.611169in}{1.259265in}}%
\pgfpathlineto{\pgfqpoint{1.613068in}{1.248745in}}%
\pgfpathlineto{\pgfqpoint{1.618358in}{1.144968in}}%
\pgfpathlineto{\pgfqpoint{1.619578in}{0.975143in}}%
\pgfpathlineto{\pgfqpoint{1.619714in}{0.765672in}}%
\pgfpathlineto{\pgfqpoint{1.620934in}{1.149472in}}%
\pgfpathlineto{\pgfqpoint{1.624325in}{1.213191in}}%
\pgfpathlineto{\pgfqpoint{1.624596in}{1.215977in}}%
\pgfpathlineto{\pgfqpoint{1.626766in}{1.232547in}}%
\pgfpathlineto{\pgfqpoint{1.627716in}{1.237161in}}%
\pgfpathlineto{\pgfqpoint{1.628123in}{1.234003in}}%
\pgfpathlineto{\pgfqpoint{1.629886in}{1.225839in}}%
\pgfpathlineto{\pgfqpoint{1.631920in}{1.206802in}}%
\pgfpathlineto{\pgfqpoint{1.635718in}{1.020607in}}%
\pgfpathlineto{\pgfqpoint{1.635853in}{0.673382in}}%
\pgfpathlineto{\pgfqpoint{1.637209in}{1.142933in}}%
\pgfpathlineto{\pgfqpoint{1.639922in}{1.214707in}}%
\pgfpathlineto{\pgfqpoint{1.640329in}{1.213227in}}%
\pgfpathlineto{\pgfqpoint{1.640871in}{1.217879in}}%
\pgfpathlineto{\pgfqpoint{1.642634in}{1.228700in}}%
\pgfpathlineto{\pgfqpoint{1.642770in}{1.227099in}}%
\pgfpathlineto{\pgfqpoint{1.643991in}{1.228830in}}%
\pgfpathlineto{\pgfqpoint{1.643177in}{1.226598in}}%
\pgfpathlineto{\pgfqpoint{1.644398in}{1.228053in}}%
\pgfpathlineto{\pgfqpoint{1.646161in}{1.216537in}}%
\pgfpathlineto{\pgfqpoint{1.646296in}{1.220824in}}%
\pgfpathlineto{\pgfqpoint{1.646432in}{1.221281in}}%
\pgfpathlineto{\pgfqpoint{1.646703in}{1.218441in}}%
\pgfpathlineto{\pgfqpoint{1.650365in}{1.121019in}}%
\pgfpathlineto{\pgfqpoint{1.650908in}{1.052125in}}%
\pgfpathlineto{\pgfqpoint{1.651314in}{0.818994in}}%
\pgfpathlineto{\pgfqpoint{1.652264in}{1.130182in}}%
\pgfpathlineto{\pgfqpoint{1.652535in}{1.127746in}}%
\pgfpathlineto{\pgfqpoint{1.652671in}{1.137416in}}%
\pgfpathlineto{\pgfqpoint{1.655248in}{1.212658in}}%
\pgfpathlineto{\pgfqpoint{1.655383in}{1.212040in}}%
\pgfpathlineto{\pgfqpoint{1.655654in}{1.217725in}}%
\pgfpathlineto{\pgfqpoint{1.660266in}{1.259062in}}%
\pgfpathlineto{\pgfqpoint{1.660673in}{1.257519in}}%
\pgfpathlineto{\pgfqpoint{1.661351in}{1.262906in}}%
\pgfpathlineto{\pgfqpoint{1.662436in}{1.261331in}}%
\pgfpathlineto{\pgfqpoint{1.664741in}{1.246235in}}%
\pgfpathlineto{\pgfqpoint{1.667589in}{1.214245in}}%
\pgfpathlineto{\pgfqpoint{1.670166in}{1.112319in}}%
\pgfpathlineto{\pgfqpoint{1.670980in}{0.946779in}}%
\pgfpathlineto{\pgfqpoint{1.671794in}{1.105064in}}%
\pgfpathlineto{\pgfqpoint{1.672201in}{1.130618in}}%
\pgfpathlineto{\pgfqpoint{1.675320in}{1.222485in}}%
\pgfpathlineto{\pgfqpoint{1.675456in}{1.222156in}}%
\pgfpathlineto{\pgfqpoint{1.675998in}{1.224095in}}%
\pgfpathlineto{\pgfqpoint{1.677897in}{1.241304in}}%
\pgfpathlineto{\pgfqpoint{1.678711in}{1.238345in}}%
\pgfpathlineto{\pgfqpoint{1.678846in}{1.239049in}}%
\pgfpathlineto{\pgfqpoint{1.679253in}{1.235056in}}%
\pgfpathlineto{\pgfqpoint{1.679524in}{1.235275in}}%
\pgfpathlineto{\pgfqpoint{1.683186in}{1.201394in}}%
\pgfpathlineto{\pgfqpoint{1.686713in}{1.042899in}}%
\pgfpathlineto{\pgfqpoint{1.686848in}{1.043025in}}%
\pgfpathlineto{\pgfqpoint{1.687119in}{0.946912in}}%
\pgfpathlineto{\pgfqpoint{1.687798in}{1.111923in}}%
\pgfpathlineto{\pgfqpoint{1.688069in}{1.098416in}}%
\pgfpathlineto{\pgfqpoint{1.690917in}{1.206312in}}%
\pgfpathlineto{\pgfqpoint{1.693358in}{1.230202in}}%
\pgfpathlineto{\pgfqpoint{1.693765in}{1.228155in}}%
\pgfpathlineto{\pgfqpoint{1.694579in}{1.234109in}}%
\pgfpathlineto{\pgfqpoint{1.695121in}{1.233140in}}%
\pgfpathlineto{\pgfqpoint{1.695257in}{1.235275in}}%
\pgfpathlineto{\pgfqpoint{1.695528in}{1.238732in}}%
\pgfpathlineto{\pgfqpoint{1.696206in}{1.230630in}}%
\pgfpathlineto{\pgfqpoint{1.696749in}{1.235901in}}%
\pgfpathlineto{\pgfqpoint{1.699868in}{1.210709in}}%
\pgfpathlineto{\pgfqpoint{1.703801in}{1.016949in}}%
\pgfpathlineto{\pgfqpoint{1.704073in}{0.861682in}}%
\pgfpathlineto{\pgfqpoint{1.705022in}{1.135388in}}%
\pgfpathlineto{\pgfqpoint{1.707056in}{1.196819in}}%
\pgfpathlineto{\pgfqpoint{1.710311in}{1.240425in}}%
\pgfpathlineto{\pgfqpoint{1.710718in}{1.237655in}}%
\pgfpathlineto{\pgfqpoint{1.711261in}{1.241057in}}%
\pgfpathlineto{\pgfqpoint{1.711803in}{1.239025in}}%
\pgfpathlineto{\pgfqpoint{1.713295in}{1.250459in}}%
\pgfpathlineto{\pgfqpoint{1.713702in}{1.245628in}}%
\pgfpathlineto{\pgfqpoint{1.714787in}{1.242833in}}%
\pgfpathlineto{\pgfqpoint{1.714380in}{1.250148in}}%
\pgfpathlineto{\pgfqpoint{1.715194in}{1.243365in}}%
\pgfpathlineto{\pgfqpoint{1.715329in}{1.244660in}}%
\pgfpathlineto{\pgfqpoint{1.716008in}{1.238860in}}%
\pgfpathlineto{\pgfqpoint{1.716279in}{1.241074in}}%
\pgfpathlineto{\pgfqpoint{1.718856in}{1.200317in}}%
\pgfpathlineto{\pgfqpoint{1.721568in}{0.988304in}}%
\pgfpathlineto{\pgfqpoint{1.722111in}{1.088961in}}%
\pgfpathlineto{\pgfqpoint{1.724281in}{1.212469in}}%
\pgfpathlineto{\pgfqpoint{1.727129in}{1.239589in}}%
\pgfpathlineto{\pgfqpoint{1.727536in}{1.238267in}}%
\pgfpathlineto{\pgfqpoint{1.727671in}{1.240221in}}%
\pgfpathlineto{\pgfqpoint{1.729434in}{1.254591in}}%
\pgfpathlineto{\pgfqpoint{1.729706in}{1.252374in}}%
\pgfpathlineto{\pgfqpoint{1.731604in}{1.245517in}}%
\pgfpathlineto{\pgfqpoint{1.729977in}{1.252558in}}%
\pgfpathlineto{\pgfqpoint{1.731876in}{1.247827in}}%
\pgfpathlineto{\pgfqpoint{1.732011in}{1.249443in}}%
\pgfpathlineto{\pgfqpoint{1.732554in}{1.237693in}}%
\pgfpathlineto{\pgfqpoint{1.737301in}{1.126237in}}%
\pgfpathlineto{\pgfqpoint{1.738386in}{0.924609in}}%
\pgfpathlineto{\pgfqpoint{1.739064in}{1.010546in}}%
\pgfpathlineto{\pgfqpoint{1.740691in}{1.160508in}}%
\pgfpathlineto{\pgfqpoint{1.740827in}{1.158349in}}%
\pgfpathlineto{\pgfqpoint{1.741098in}{1.154810in}}%
\pgfpathlineto{\pgfqpoint{1.741234in}{1.163891in}}%
\pgfpathlineto{\pgfqpoint{1.743133in}{1.204172in}}%
\pgfpathlineto{\pgfqpoint{1.743539in}{1.205737in}}%
\pgfpathlineto{\pgfqpoint{1.747879in}{1.237816in}}%
\pgfpathlineto{\pgfqpoint{1.743811in}{1.205581in}}%
\pgfpathlineto{\pgfqpoint{1.748964in}{1.233691in}}%
\pgfpathlineto{\pgfqpoint{1.750863in}{1.204779in}}%
\pgfpathlineto{\pgfqpoint{1.753847in}{1.107341in}}%
\pgfpathlineto{\pgfqpoint{1.754389in}{0.765975in}}%
\pgfpathlineto{\pgfqpoint{1.755339in}{1.098958in}}%
\pgfpathlineto{\pgfqpoint{1.758458in}{1.208311in}}%
\pgfpathlineto{\pgfqpoint{1.761578in}{1.243227in}}%
\pgfpathlineto{\pgfqpoint{1.761849in}{1.242193in}}%
\pgfpathlineto{\pgfqpoint{1.762663in}{1.237218in}}%
\pgfpathlineto{\pgfqpoint{1.763341in}{1.241732in}}%
\pgfpathlineto{\pgfqpoint{1.763476in}{1.243404in}}%
\pgfpathlineto{\pgfqpoint{1.764426in}{1.237264in}}%
\pgfpathlineto{\pgfqpoint{1.767681in}{1.207565in}}%
\pgfpathlineto{\pgfqpoint{1.769851in}{1.171588in}}%
\pgfpathlineto{\pgfqpoint{1.771614in}{1.013514in}}%
\pgfpathlineto{\pgfqpoint{1.771885in}{0.742441in}}%
\pgfpathlineto{\pgfqpoint{1.772970in}{1.123651in}}%
\pgfpathlineto{\pgfqpoint{1.775004in}{1.188570in}}%
\pgfpathlineto{\pgfqpoint{1.777174in}{1.220465in}}%
\pgfpathlineto{\pgfqpoint{1.777581in}{1.217991in}}%
\pgfpathlineto{\pgfqpoint{1.778124in}{1.226473in}}%
\pgfpathlineto{\pgfqpoint{1.781379in}{1.241783in}}%
\pgfpathlineto{\pgfqpoint{1.781650in}{1.240447in}}%
\pgfpathlineto{\pgfqpoint{1.781786in}{1.237290in}}%
\pgfpathlineto{\pgfqpoint{1.782871in}{1.245116in}}%
\pgfpathlineto{\pgfqpoint{1.783142in}{1.246724in}}%
\pgfpathlineto{\pgfqpoint{1.783820in}{1.243804in}}%
\pgfpathlineto{\pgfqpoint{1.783956in}{1.244092in}}%
\pgfpathlineto{\pgfqpoint{1.787618in}{1.176665in}}%
\pgfpathlineto{\pgfqpoint{1.789381in}{1.012814in}}%
\pgfpathlineto{\pgfqpoint{1.789516in}{0.730842in}}%
\pgfpathlineto{\pgfqpoint{1.790737in}{1.169354in}}%
\pgfpathlineto{\pgfqpoint{1.794128in}{1.237948in}}%
\pgfpathlineto{\pgfqpoint{1.794263in}{1.233958in}}%
\pgfpathlineto{\pgfqpoint{1.795077in}{1.244829in}}%
\pgfpathlineto{\pgfqpoint{1.795348in}{1.243831in}}%
\pgfpathlineto{\pgfqpoint{1.796840in}{1.261492in}}%
\pgfpathlineto{\pgfqpoint{1.799146in}{1.258558in}}%
\pgfpathlineto{\pgfqpoint{1.801723in}{1.241256in}}%
\pgfpathlineto{\pgfqpoint{1.801994in}{1.240163in}}%
\pgfpathlineto{\pgfqpoint{1.806605in}{1.113823in}}%
\pgfpathlineto{\pgfqpoint{1.807148in}{0.902534in}}%
\pgfpathlineto{\pgfqpoint{1.808233in}{1.085832in}}%
\pgfpathlineto{\pgfqpoint{1.808911in}{1.080841in}}%
\pgfpathlineto{\pgfqpoint{1.810131in}{1.157662in}}%
\pgfpathlineto{\pgfqpoint{1.810267in}{1.157613in}}%
\pgfpathlineto{\pgfqpoint{1.810945in}{1.188223in}}%
\pgfpathlineto{\pgfqpoint{1.812844in}{1.218582in}}%
\pgfpathlineto{\pgfqpoint{1.813658in}{1.209611in}}%
\pgfpathlineto{\pgfqpoint{1.813115in}{1.219072in}}%
\pgfpathlineto{\pgfqpoint{1.814607in}{1.214753in}}%
\pgfpathlineto{\pgfqpoint{1.814743in}{1.219695in}}%
\pgfpathlineto{\pgfqpoint{1.815421in}{1.202166in}}%
\pgfpathlineto{\pgfqpoint{1.815963in}{1.208722in}}%
\pgfpathlineto{\pgfqpoint{1.816234in}{1.210431in}}%
\pgfpathlineto{\pgfqpoint{1.816641in}{1.204956in}}%
\pgfpathlineto{\pgfqpoint{1.816777in}{1.206738in}}%
\pgfpathlineto{\pgfqpoint{1.816913in}{1.202357in}}%
\pgfpathlineto{\pgfqpoint{1.817862in}{1.209946in}}%
\pgfpathlineto{\pgfqpoint{1.818269in}{1.205885in}}%
\pgfpathlineto{\pgfqpoint{1.818404in}{1.206872in}}%
\pgfpathlineto{\pgfqpoint{1.818811in}{1.201474in}}%
\pgfpathlineto{\pgfqpoint{1.821524in}{1.063096in}}%
\pgfpathlineto{\pgfqpoint{1.821795in}{0.954045in}}%
\pgfpathlineto{\pgfqpoint{1.822880in}{1.118292in}}%
\pgfpathlineto{\pgfqpoint{1.826135in}{1.235623in}}%
\pgfpathlineto{\pgfqpoint{1.829119in}{1.262351in}}%
\pgfpathlineto{\pgfqpoint{1.829661in}{1.260218in}}%
\pgfpathlineto{\pgfqpoint{1.830068in}{1.265607in}}%
\pgfpathlineto{\pgfqpoint{1.830475in}{1.263252in}}%
\pgfpathlineto{\pgfqpoint{1.831560in}{1.267644in}}%
\pgfpathlineto{\pgfqpoint{1.831967in}{1.263759in}}%
\pgfpathlineto{\pgfqpoint{1.832509in}{1.262553in}}%
\pgfpathlineto{\pgfqpoint{1.832916in}{1.264947in}}%
\pgfpathlineto{\pgfqpoint{1.833323in}{1.263645in}}%
\pgfpathlineto{\pgfqpoint{1.834001in}{1.266661in}}%
\pgfpathlineto{\pgfqpoint{1.834408in}{1.261557in}}%
\pgfpathlineto{\pgfqpoint{1.835086in}{1.262528in}}%
\pgfpathlineto{\pgfqpoint{1.836578in}{1.252335in}}%
\pgfpathlineto{\pgfqpoint{1.836985in}{1.250671in}}%
\pgfpathlineto{\pgfqpoint{1.839969in}{1.206661in}}%
\pgfpathlineto{\pgfqpoint{1.840376in}{1.213958in}}%
\pgfpathlineto{\pgfqpoint{1.840783in}{1.199062in}}%
\pgfpathlineto{\pgfqpoint{1.843631in}{1.028178in}}%
\pgfpathlineto{\pgfqpoint{1.843766in}{0.811569in}}%
\pgfpathlineto{\pgfqpoint{1.844851in}{1.152608in}}%
\pgfpathlineto{\pgfqpoint{1.844987in}{1.151636in}}%
\pgfpathlineto{\pgfqpoint{1.845258in}{1.153524in}}%
\pgfpathlineto{\pgfqpoint{1.845394in}{1.150419in}}%
\pgfpathlineto{\pgfqpoint{1.847021in}{1.105396in}}%
\pgfpathlineto{\pgfqpoint{1.847564in}{1.120155in}}%
\pgfpathlineto{\pgfqpoint{1.849056in}{1.148609in}}%
\pgfpathlineto{\pgfqpoint{1.848378in}{1.106988in}}%
\pgfpathlineto{\pgfqpoint{1.849191in}{1.146409in}}%
\pgfpathlineto{\pgfqpoint{1.851226in}{1.108955in}}%
\pgfpathlineto{\pgfqpoint{1.849598in}{1.150470in}}%
\pgfpathlineto{\pgfqpoint{1.851361in}{1.112285in}}%
\pgfpathlineto{\pgfqpoint{1.851497in}{1.124191in}}%
\pgfpathlineto{\pgfqpoint{1.852039in}{1.071476in}}%
\pgfpathlineto{\pgfqpoint{1.853124in}{0.996412in}}%
\pgfpathlineto{\pgfqpoint{1.853396in}{1.106698in}}%
\pgfpathlineto{\pgfqpoint{1.855294in}{1.176187in}}%
\pgfpathlineto{\pgfqpoint{1.858278in}{1.228243in}}%
\pgfpathlineto{\pgfqpoint{1.858549in}{1.229892in}}%
\pgfpathlineto{\pgfqpoint{1.860313in}{1.258778in}}%
\pgfpathlineto{\pgfqpoint{1.860584in}{1.258314in}}%
\pgfpathlineto{\pgfqpoint{1.860719in}{1.257893in}}%
\pgfpathlineto{\pgfqpoint{1.860991in}{1.260391in}}%
\pgfpathlineto{\pgfqpoint{1.863161in}{1.271799in}}%
\pgfpathlineto{\pgfqpoint{1.863703in}{1.268227in}}%
\pgfpathlineto{\pgfqpoint{1.863974in}{1.271930in}}%
\pgfpathlineto{\pgfqpoint{1.864788in}{1.268296in}}%
\pgfpathlineto{\pgfqpoint{1.864924in}{1.268083in}}%
\pgfpathlineto{\pgfqpoint{1.865059in}{1.270374in}}%
\pgfpathlineto{\pgfqpoint{1.865466in}{1.273293in}}%
\pgfpathlineto{\pgfqpoint{1.866144in}{1.266653in}}%
\pgfpathlineto{\pgfqpoint{1.866551in}{1.272114in}}%
\pgfpathlineto{\pgfqpoint{1.874553in}{1.194308in}}%
\pgfpathlineto{\pgfqpoint{1.876316in}{1.074469in}}%
\pgfpathlineto{\pgfqpoint{1.876588in}{0.887335in}}%
\pgfpathlineto{\pgfqpoint{1.877808in}{1.091435in}}%
\pgfpathlineto{\pgfqpoint{1.880114in}{1.189447in}}%
\pgfpathlineto{\pgfqpoint{1.878215in}{1.061932in}}%
\pgfpathlineto{\pgfqpoint{1.880385in}{1.181556in}}%
\pgfpathlineto{\pgfqpoint{1.880656in}{1.167025in}}%
\pgfpathlineto{\pgfqpoint{1.881606in}{1.195310in}}%
\pgfpathlineto{\pgfqpoint{1.883098in}{1.202037in}}%
\pgfpathlineto{\pgfqpoint{1.882419in}{1.187921in}}%
\pgfpathlineto{\pgfqpoint{1.883233in}{1.197842in}}%
\pgfpathlineto{\pgfqpoint{1.883369in}{1.192035in}}%
\pgfpathlineto{\pgfqpoint{1.884318in}{1.213738in}}%
\pgfpathlineto{\pgfqpoint{1.884454in}{1.215701in}}%
\pgfpathlineto{\pgfqpoint{1.884861in}{1.200943in}}%
\pgfpathlineto{\pgfqpoint{1.887166in}{1.098129in}}%
\pgfpathlineto{\pgfqpoint{1.887573in}{1.017327in}}%
\pgfpathlineto{\pgfqpoint{1.888658in}{1.018783in}}%
\pgfpathlineto{\pgfqpoint{1.890286in}{1.175910in}}%
\pgfpathlineto{\pgfqpoint{1.890421in}{1.164699in}}%
\pgfpathlineto{\pgfqpoint{1.890557in}{1.155874in}}%
\pgfpathlineto{\pgfqpoint{1.891099in}{1.183623in}}%
\pgfpathlineto{\pgfqpoint{1.891642in}{1.180363in}}%
\pgfpathlineto{\pgfqpoint{1.898830in}{1.259969in}}%
\pgfpathlineto{\pgfqpoint{1.898966in}{1.260358in}}%
\pgfpathlineto{\pgfqpoint{1.899237in}{1.257437in}}%
\pgfpathlineto{\pgfqpoint{1.899373in}{1.257803in}}%
\pgfpathlineto{\pgfqpoint{1.901271in}{1.248118in}}%
\pgfpathlineto{\pgfqpoint{1.901949in}{1.251127in}}%
\pgfpathlineto{\pgfqpoint{1.901543in}{1.246942in}}%
\pgfpathlineto{\pgfqpoint{1.902221in}{1.247698in}}%
\pgfpathlineto{\pgfqpoint{1.907646in}{1.110604in}}%
\pgfpathlineto{\pgfqpoint{1.909544in}{0.882952in}}%
\pgfpathlineto{\pgfqpoint{1.909680in}{1.022139in}}%
\pgfpathlineto{\pgfqpoint{1.911443in}{1.161407in}}%
\pgfpathlineto{\pgfqpoint{1.913206in}{1.189957in}}%
\pgfpathlineto{\pgfqpoint{1.913342in}{1.185026in}}%
\pgfpathlineto{\pgfqpoint{1.914427in}{1.161332in}}%
\pgfpathlineto{\pgfqpoint{1.914969in}{1.181090in}}%
\pgfpathlineto{\pgfqpoint{1.915241in}{1.175647in}}%
\pgfpathlineto{\pgfqpoint{1.917411in}{1.099709in}}%
\pgfpathlineto{\pgfqpoint{1.917953in}{1.122668in}}%
\pgfpathlineto{\pgfqpoint{1.918089in}{1.094862in}}%
\pgfpathlineto{\pgfqpoint{1.918631in}{0.857373in}}%
\pgfpathlineto{\pgfqpoint{1.919716in}{1.058974in}}%
\pgfpathlineto{\pgfqpoint{1.919852in}{1.079921in}}%
\pgfpathlineto{\pgfqpoint{1.920123in}{0.993866in}}%
\pgfpathlineto{\pgfqpoint{1.920530in}{0.763283in}}%
\pgfpathlineto{\pgfqpoint{1.920937in}{1.077898in}}%
\pgfpathlineto{\pgfqpoint{1.921615in}{1.040675in}}%
\pgfpathlineto{\pgfqpoint{1.921886in}{0.963668in}}%
\pgfpathlineto{\pgfqpoint{1.922158in}{1.086667in}}%
\pgfpathlineto{\pgfqpoint{1.922564in}{1.064947in}}%
\pgfpathlineto{\pgfqpoint{1.924463in}{1.150163in}}%
\pgfpathlineto{\pgfqpoint{1.925006in}{1.150060in}}%
\pgfpathlineto{\pgfqpoint{1.926904in}{1.200718in}}%
\pgfpathlineto{\pgfqpoint{1.929074in}{1.245120in}}%
\pgfpathlineto{\pgfqpoint{1.929481in}{1.239478in}}%
\pgfpathlineto{\pgfqpoint{1.933143in}{1.267552in}}%
\pgfpathlineto{\pgfqpoint{1.929888in}{1.238195in}}%
\pgfpathlineto{\pgfqpoint{1.933550in}{1.263919in}}%
\pgfpathlineto{\pgfqpoint{1.933957in}{1.259365in}}%
\pgfpathlineto{\pgfqpoint{1.934771in}{1.266609in}}%
\pgfpathlineto{\pgfqpoint{1.935042in}{1.264985in}}%
\pgfpathlineto{\pgfqpoint{1.935449in}{1.266260in}}%
\pgfpathlineto{\pgfqpoint{1.935856in}{1.263334in}}%
\pgfpathlineto{\pgfqpoint{1.936398in}{1.251302in}}%
\pgfpathlineto{\pgfqpoint{1.937619in}{1.255430in}}%
\pgfpathlineto{\pgfqpoint{1.938297in}{1.258740in}}%
\pgfpathlineto{\pgfqpoint{1.937890in}{1.254084in}}%
\pgfpathlineto{\pgfqpoint{1.938433in}{1.255115in}}%
\pgfpathlineto{\pgfqpoint{1.943993in}{1.149104in}}%
\pgfpathlineto{\pgfqpoint{1.946028in}{1.016819in}}%
\pgfpathlineto{\pgfqpoint{1.946163in}{1.071232in}}%
\pgfpathlineto{\pgfqpoint{1.946570in}{1.108357in}}%
\pgfpathlineto{\pgfqpoint{1.946977in}{1.023926in}}%
\pgfpathlineto{\pgfqpoint{1.948062in}{0.915384in}}%
\pgfpathlineto{\pgfqpoint{1.947926in}{1.040459in}}%
\pgfpathlineto{\pgfqpoint{1.948469in}{0.982422in}}%
\pgfpathlineto{\pgfqpoint{1.950503in}{1.183409in}}%
\pgfpathlineto{\pgfqpoint{1.951181in}{1.175716in}}%
\pgfpathlineto{\pgfqpoint{1.951317in}{1.160877in}}%
\pgfpathlineto{\pgfqpoint{1.952538in}{1.176391in}}%
\pgfpathlineto{\pgfqpoint{1.952673in}{1.173118in}}%
\pgfpathlineto{\pgfqpoint{1.953623in}{1.199039in}}%
\pgfpathlineto{\pgfqpoint{1.954979in}{1.193443in}}%
\pgfpathlineto{\pgfqpoint{1.956064in}{1.195965in}}%
\pgfpathlineto{\pgfqpoint{1.958098in}{1.021770in}}%
\pgfpathlineto{\pgfqpoint{1.958234in}{0.898708in}}%
\pgfpathlineto{\pgfqpoint{1.959454in}{1.171037in}}%
\pgfpathlineto{\pgfqpoint{1.959997in}{1.165936in}}%
\pgfpathlineto{\pgfqpoint{1.961624in}{1.224093in}}%
\pgfpathlineto{\pgfqpoint{1.961760in}{1.223433in}}%
\pgfpathlineto{\pgfqpoint{1.961896in}{1.229247in}}%
\pgfpathlineto{\pgfqpoint{1.962031in}{1.227913in}}%
\pgfpathlineto{\pgfqpoint{1.964337in}{1.261283in}}%
\pgfpathlineto{\pgfqpoint{1.966100in}{1.272611in}}%
\pgfpathlineto{\pgfqpoint{1.966236in}{1.269380in}}%
\pgfpathlineto{\pgfqpoint{1.966643in}{1.267644in}}%
\pgfpathlineto{\pgfqpoint{1.966914in}{1.274317in}}%
\pgfpathlineto{\pgfqpoint{1.967456in}{1.271571in}}%
\pgfpathlineto{\pgfqpoint{1.970033in}{1.284943in}}%
\pgfpathlineto{\pgfqpoint{1.970440in}{1.284533in}}%
\pgfpathlineto{\pgfqpoint{1.980883in}{1.158619in}}%
\pgfpathlineto{\pgfqpoint{1.981290in}{1.181081in}}%
\pgfpathlineto{\pgfqpoint{1.981426in}{1.180166in}}%
\pgfpathlineto{\pgfqpoint{1.984003in}{1.002320in}}%
\pgfpathlineto{\pgfqpoint{1.985494in}{1.147749in}}%
\pgfpathlineto{\pgfqpoint{1.984545in}{0.992353in}}%
\pgfpathlineto{\pgfqpoint{1.985630in}{1.122758in}}%
\pgfpathlineto{\pgfqpoint{1.988207in}{1.165957in}}%
\pgfpathlineto{\pgfqpoint{1.986037in}{1.117284in}}%
\pgfpathlineto{\pgfqpoint{1.988478in}{1.159880in}}%
\pgfpathlineto{\pgfqpoint{1.988614in}{1.151208in}}%
\pgfpathlineto{\pgfqpoint{1.989428in}{1.188296in}}%
\pgfpathlineto{\pgfqpoint{1.990513in}{1.212975in}}%
\pgfpathlineto{\pgfqpoint{1.989699in}{1.187536in}}%
\pgfpathlineto{\pgfqpoint{1.990919in}{1.194199in}}%
\pgfpathlineto{\pgfqpoint{1.991191in}{1.181161in}}%
\pgfpathlineto{\pgfqpoint{1.991733in}{1.200871in}}%
\pgfpathlineto{\pgfqpoint{1.991869in}{1.197834in}}%
\pgfpathlineto{\pgfqpoint{1.994039in}{1.239845in}}%
\pgfpathlineto{\pgfqpoint{1.996209in}{1.273248in}}%
\pgfpathlineto{\pgfqpoint{1.996344in}{1.271806in}}%
\pgfpathlineto{\pgfqpoint{1.996616in}{1.268641in}}%
\pgfpathlineto{\pgfqpoint{1.997429in}{1.275666in}}%
\pgfpathlineto{\pgfqpoint{1.997972in}{1.269823in}}%
\pgfpathlineto{\pgfqpoint{1.999599in}{1.284462in}}%
\pgfpathlineto{\pgfqpoint{2.000142in}{1.281514in}}%
\pgfpathlineto{\pgfqpoint{2.001905in}{1.269502in}}%
\pgfpathlineto{\pgfqpoint{2.001091in}{1.282355in}}%
\pgfpathlineto{\pgfqpoint{2.002041in}{1.274341in}}%
\pgfpathlineto{\pgfqpoint{2.002312in}{1.272810in}}%
\pgfpathlineto{\pgfqpoint{2.002719in}{1.274709in}}%
\pgfpathlineto{\pgfqpoint{2.003397in}{1.269933in}}%
\pgfpathlineto{\pgfqpoint{2.003804in}{1.276136in}}%
\pgfpathlineto{\pgfqpoint{2.004211in}{1.274558in}}%
\pgfpathlineto{\pgfqpoint{2.005024in}{1.278293in}}%
\pgfpathlineto{\pgfqpoint{2.005567in}{1.273320in}}%
\pgfpathlineto{\pgfqpoint{2.008144in}{1.253769in}}%
\pgfpathlineto{\pgfqpoint{2.005974in}{1.275920in}}%
\pgfpathlineto{\pgfqpoint{2.008279in}{1.253994in}}%
\pgfpathlineto{\pgfqpoint{2.010992in}{1.202550in}}%
\pgfpathlineto{\pgfqpoint{2.011534in}{1.215335in}}%
\pgfpathlineto{\pgfqpoint{2.011941in}{1.186695in}}%
\pgfpathlineto{\pgfqpoint{2.013840in}{1.136056in}}%
\pgfpathlineto{\pgfqpoint{2.014111in}{1.142694in}}%
\pgfpathlineto{\pgfqpoint{2.014383in}{1.130094in}}%
\pgfpathlineto{\pgfqpoint{2.014925in}{0.911659in}}%
\pgfpathlineto{\pgfqpoint{2.015061in}{0.668767in}}%
\pgfpathlineto{\pgfqpoint{2.015332in}{1.092209in}}%
\pgfpathlineto{\pgfqpoint{2.016417in}{0.795777in}}%
\pgfpathlineto{\pgfqpoint{2.017366in}{1.093600in}}%
\pgfpathlineto{\pgfqpoint{2.018044in}{0.897501in}}%
\pgfpathlineto{\pgfqpoint{2.018180in}{0.724024in}}%
\pgfpathlineto{\pgfqpoint{2.018858in}{1.069700in}}%
\pgfpathlineto{\pgfqpoint{2.019401in}{1.010222in}}%
\pgfpathlineto{\pgfqpoint{2.020350in}{1.076365in}}%
\pgfpathlineto{\pgfqpoint{2.020621in}{1.002839in}}%
\pgfpathlineto{\pgfqpoint{2.021028in}{0.913046in}}%
\pgfpathlineto{\pgfqpoint{2.021978in}{1.111115in}}%
\pgfpathlineto{\pgfqpoint{2.023605in}{1.202255in}}%
\pgfpathlineto{\pgfqpoint{2.023876in}{1.196411in}}%
\pgfpathlineto{\pgfqpoint{2.027267in}{1.249861in}}%
\pgfpathlineto{\pgfqpoint{2.027538in}{1.246631in}}%
\pgfpathlineto{\pgfqpoint{2.027809in}{1.253275in}}%
\pgfpathlineto{\pgfqpoint{2.027945in}{1.252036in}}%
\pgfpathlineto{\pgfqpoint{2.028623in}{1.265226in}}%
\pgfpathlineto{\pgfqpoint{2.029708in}{1.263713in}}%
\pgfpathlineto{\pgfqpoint{2.032014in}{1.285427in}}%
\pgfpathlineto{\pgfqpoint{2.030115in}{1.262387in}}%
\pgfpathlineto{\pgfqpoint{2.032421in}{1.284901in}}%
\pgfpathlineto{\pgfqpoint{2.032556in}{1.284643in}}%
\pgfpathlineto{\pgfqpoint{2.032963in}{1.286658in}}%
\pgfpathlineto{\pgfqpoint{2.033099in}{1.284946in}}%
\pgfpathlineto{\pgfqpoint{2.034862in}{1.291965in}}%
\pgfpathlineto{\pgfqpoint{2.033777in}{1.284718in}}%
\pgfpathlineto{\pgfqpoint{2.034998in}{1.291875in}}%
\pgfpathlineto{\pgfqpoint{2.035676in}{1.291339in}}%
\pgfpathlineto{\pgfqpoint{2.035540in}{1.292991in}}%
\pgfpathlineto{\pgfqpoint{2.035947in}{1.292938in}}%
\pgfpathlineto{\pgfqpoint{2.036354in}{1.297980in}}%
\pgfpathlineto{\pgfqpoint{2.036896in}{1.288412in}}%
\pgfpathlineto{\pgfqpoint{2.039066in}{1.280220in}}%
\pgfpathlineto{\pgfqpoint{2.037168in}{1.289132in}}%
\pgfpathlineto{\pgfqpoint{2.040151in}{1.282475in}}%
\pgfpathlineto{\pgfqpoint{2.040287in}{1.284789in}}%
\pgfpathlineto{\pgfqpoint{2.040829in}{1.275470in}}%
\pgfpathlineto{\pgfqpoint{2.041101in}{1.276371in}}%
\pgfpathlineto{\pgfqpoint{2.044084in}{1.241219in}}%
\pgfpathlineto{\pgfqpoint{2.044220in}{1.245961in}}%
\pgfpathlineto{\pgfqpoint{2.044491in}{1.252059in}}%
\pgfpathlineto{\pgfqpoint{2.045169in}{1.235777in}}%
\pgfpathlineto{\pgfqpoint{2.045441in}{1.236468in}}%
\pgfpathlineto{\pgfqpoint{2.045576in}{1.237411in}}%
\pgfpathlineto{\pgfqpoint{2.045848in}{1.230666in}}%
\pgfpathlineto{\pgfqpoint{2.049781in}{1.115532in}}%
\pgfpathlineto{\pgfqpoint{2.051137in}{0.929674in}}%
\pgfpathlineto{\pgfqpoint{2.051544in}{0.998000in}}%
\pgfpathlineto{\pgfqpoint{2.051951in}{1.110473in}}%
\pgfpathlineto{\pgfqpoint{2.053171in}{1.106476in}}%
\pgfpathlineto{\pgfqpoint{2.053849in}{1.142898in}}%
\pgfpathlineto{\pgfqpoint{2.053443in}{1.099612in}}%
\pgfpathlineto{\pgfqpoint{2.054934in}{1.139165in}}%
\pgfpathlineto{\pgfqpoint{2.055206in}{1.108331in}}%
\pgfpathlineto{\pgfqpoint{2.056155in}{1.171460in}}%
\pgfpathlineto{\pgfqpoint{2.059953in}{1.260216in}}%
\pgfpathlineto{\pgfqpoint{2.060088in}{1.260804in}}%
\pgfpathlineto{\pgfqpoint{2.060224in}{1.256185in}}%
\pgfpathlineto{\pgfqpoint{2.060359in}{1.252866in}}%
\pgfpathlineto{\pgfqpoint{2.060902in}{1.266363in}}%
\pgfpathlineto{\pgfqpoint{2.061444in}{1.262049in}}%
\pgfpathlineto{\pgfqpoint{2.061580in}{1.262061in}}%
\pgfpathlineto{\pgfqpoint{2.061716in}{1.259618in}}%
\pgfpathlineto{\pgfqpoint{2.062258in}{1.273577in}}%
\pgfpathlineto{\pgfqpoint{2.062936in}{1.261350in}}%
\pgfpathlineto{\pgfqpoint{2.065784in}{1.301136in}}%
\pgfpathlineto{\pgfqpoint{2.066463in}{1.300179in}}%
\pgfpathlineto{\pgfqpoint{2.066598in}{1.298673in}}%
\pgfpathlineto{\pgfqpoint{2.067412in}{1.303182in}}%
\pgfpathlineto{\pgfqpoint{2.069039in}{1.315106in}}%
\pgfpathlineto{\pgfqpoint{2.069311in}{1.312156in}}%
\pgfpathlineto{\pgfqpoint{2.070531in}{1.313460in}}%
\pgfpathlineto{\pgfqpoint{2.070667in}{1.315282in}}%
\pgfpathlineto{\pgfqpoint{2.071616in}{1.312491in}}%
\pgfpathlineto{\pgfqpoint{2.072159in}{1.315086in}}%
\pgfpathlineto{\pgfqpoint{2.073922in}{1.307829in}}%
\pgfpathlineto{\pgfqpoint{2.074058in}{1.310258in}}%
\pgfpathlineto{\pgfqpoint{2.074329in}{1.311459in}}%
\pgfpathlineto{\pgfqpoint{2.074464in}{1.308370in}}%
\pgfpathlineto{\pgfqpoint{2.075414in}{1.309138in}}%
\pgfpathlineto{\pgfqpoint{2.075685in}{1.307157in}}%
\pgfpathlineto{\pgfqpoint{2.075821in}{1.307492in}}%
\pgfpathlineto{\pgfqpoint{2.077855in}{1.298673in}}%
\pgfpathlineto{\pgfqpoint{2.078940in}{1.295084in}}%
\pgfpathlineto{\pgfqpoint{2.078262in}{1.299155in}}%
\pgfpathlineto{\pgfqpoint{2.079483in}{1.298272in}}%
\pgfpathlineto{\pgfqpoint{2.079889in}{1.294651in}}%
\pgfpathlineto{\pgfqpoint{2.081110in}{1.290992in}}%
\pgfpathlineto{\pgfqpoint{2.080432in}{1.295763in}}%
\pgfpathlineto{\pgfqpoint{2.081381in}{1.295227in}}%
\pgfpathlineto{\pgfqpoint{2.081517in}{1.295024in}}%
\pgfpathlineto{\pgfqpoint{2.083416in}{1.282313in}}%
\pgfpathlineto{\pgfqpoint{2.084908in}{1.286899in}}%
\pgfpathlineto{\pgfqpoint{2.084094in}{1.281417in}}%
\pgfpathlineto{\pgfqpoint{2.085179in}{1.284043in}}%
\pgfpathlineto{\pgfqpoint{2.087349in}{1.271002in}}%
\pgfpathlineto{\pgfqpoint{2.085586in}{1.286408in}}%
\pgfpathlineto{\pgfqpoint{2.087484in}{1.272648in}}%
\pgfpathlineto{\pgfqpoint{2.088163in}{1.267447in}}%
\pgfpathlineto{\pgfqpoint{2.088569in}{1.273091in}}%
\pgfpathlineto{\pgfqpoint{2.089112in}{1.271576in}}%
\pgfpathlineto{\pgfqpoint{2.093045in}{1.303719in}}%
\pgfpathlineto{\pgfqpoint{2.093316in}{1.300772in}}%
\pgfpathlineto{\pgfqpoint{2.093859in}{1.308155in}}%
\pgfpathlineto{\pgfqpoint{2.094401in}{1.306301in}}%
\pgfpathlineto{\pgfqpoint{2.102810in}{1.342899in}}%
\pgfpathlineto{\pgfqpoint{2.103217in}{1.341302in}}%
\pgfpathlineto{\pgfqpoint{2.104573in}{1.342459in}}%
\pgfpathlineto{\pgfqpoint{2.104709in}{1.343564in}}%
\pgfpathlineto{\pgfqpoint{2.105929in}{1.341273in}}%
\pgfpathlineto{\pgfqpoint{2.107693in}{1.336772in}}%
\pgfpathlineto{\pgfqpoint{2.106608in}{1.342088in}}%
\pgfpathlineto{\pgfqpoint{2.107964in}{1.337378in}}%
\pgfpathlineto{\pgfqpoint{2.108778in}{1.334561in}}%
\pgfpathlineto{\pgfqpoint{2.108913in}{1.334789in}}%
\pgfpathlineto{\pgfqpoint{2.111219in}{1.326878in}}%
\pgfpathlineto{\pgfqpoint{2.111354in}{1.327561in}}%
\pgfpathlineto{\pgfqpoint{2.112168in}{1.328509in}}%
\pgfpathlineto{\pgfqpoint{2.112439in}{1.325716in}}%
\pgfpathlineto{\pgfqpoint{2.113253in}{1.328614in}}%
\pgfpathlineto{\pgfqpoint{2.114474in}{1.321525in}}%
\pgfpathlineto{\pgfqpoint{2.119221in}{1.305135in}}%
\pgfpathlineto{\pgfqpoint{2.119763in}{1.307240in}}%
\pgfpathlineto{\pgfqpoint{2.119899in}{1.308049in}}%
\pgfpathlineto{\pgfqpoint{2.120577in}{1.303290in}}%
\pgfpathlineto{\pgfqpoint{2.120713in}{1.303763in}}%
\pgfpathlineto{\pgfqpoint{2.122476in}{1.298300in}}%
\pgfpathlineto{\pgfqpoint{2.121526in}{1.304564in}}%
\pgfpathlineto{\pgfqpoint{2.122747in}{1.300929in}}%
\pgfpathlineto{\pgfqpoint{2.124103in}{1.311095in}}%
\pgfpathlineto{\pgfqpoint{2.124781in}{1.310559in}}%
\pgfpathlineto{\pgfqpoint{2.127765in}{1.325215in}}%
\pgfpathlineto{\pgfqpoint{2.128172in}{1.324751in}}%
\pgfpathlineto{\pgfqpoint{2.128308in}{1.324491in}}%
\pgfpathlineto{\pgfqpoint{2.128579in}{1.327096in}}%
\pgfpathlineto{\pgfqpoint{2.130206in}{1.337238in}}%
\pgfpathlineto{\pgfqpoint{2.131156in}{1.335774in}}%
\pgfpathlineto{\pgfqpoint{2.132376in}{1.338711in}}%
\pgfpathlineto{\pgfqpoint{2.133190in}{1.338283in}}%
\pgfpathlineto{\pgfqpoint{2.133597in}{1.337488in}}%
\pgfpathlineto{\pgfqpoint{2.133868in}{1.340694in}}%
\pgfpathlineto{\pgfqpoint{2.134139in}{1.340627in}}%
\pgfpathlineto{\pgfqpoint{2.136852in}{1.348218in}}%
\pgfpathlineto{\pgfqpoint{2.137123in}{1.345607in}}%
\pgfpathlineto{\pgfqpoint{2.137259in}{1.344098in}}%
\pgfpathlineto{\pgfqpoint{2.138208in}{1.347957in}}%
\pgfpathlineto{\pgfqpoint{2.138479in}{1.347521in}}%
\pgfpathlineto{\pgfqpoint{2.138886in}{1.350440in}}%
\pgfpathlineto{\pgfqpoint{2.139971in}{1.347643in}}%
\pgfpathlineto{\pgfqpoint{2.140378in}{1.348693in}}%
\pgfpathlineto{\pgfqpoint{2.140649in}{1.346681in}}%
\pgfpathlineto{\pgfqpoint{2.140921in}{1.347025in}}%
\pgfpathlineto{\pgfqpoint{2.143904in}{1.339094in}}%
\pgfpathlineto{\pgfqpoint{2.144040in}{1.339182in}}%
\pgfpathlineto{\pgfqpoint{2.144176in}{1.338921in}}%
\pgfpathlineto{\pgfqpoint{2.144583in}{1.340823in}}%
\pgfpathlineto{\pgfqpoint{2.144854in}{1.342166in}}%
\pgfpathlineto{\pgfqpoint{2.145396in}{1.339241in}}%
\pgfpathlineto{\pgfqpoint{2.145939in}{1.339629in}}%
\pgfpathlineto{\pgfqpoint{2.146346in}{1.338886in}}%
\pgfpathlineto{\pgfqpoint{2.148923in}{1.329153in}}%
\pgfpathlineto{\pgfqpoint{2.149058in}{1.329810in}}%
\pgfpathlineto{\pgfqpoint{2.149872in}{1.327096in}}%
\pgfpathlineto{\pgfqpoint{2.150008in}{1.327611in}}%
\pgfpathlineto{\pgfqpoint{2.151364in}{1.322271in}}%
\pgfpathlineto{\pgfqpoint{2.151499in}{1.321415in}}%
\pgfpathlineto{\pgfqpoint{2.152584in}{1.324215in}}%
\pgfpathlineto{\pgfqpoint{2.152856in}{1.325220in}}%
\pgfpathlineto{\pgfqpoint{2.154076in}{1.324445in}}%
\pgfpathlineto{\pgfqpoint{2.156518in}{1.334550in}}%
\pgfpathlineto{\pgfqpoint{2.156653in}{1.334154in}}%
\pgfpathlineto{\pgfqpoint{2.156924in}{1.334623in}}%
\pgfpathlineto{\pgfqpoint{2.157060in}{1.332654in}}%
\pgfpathlineto{\pgfqpoint{2.158145in}{1.337658in}}%
\pgfpathlineto{\pgfqpoint{2.163028in}{1.355975in}}%
\pgfpathlineto{\pgfqpoint{2.163706in}{1.354906in}}%
\pgfpathlineto{\pgfqpoint{2.164519in}{1.356777in}}%
\pgfpathlineto{\pgfqpoint{2.169402in}{1.370176in}}%
\pgfpathlineto{\pgfqpoint{2.169809in}{1.369427in}}%
\pgfpathlineto{\pgfqpoint{2.169944in}{1.368570in}}%
\pgfpathlineto{\pgfqpoint{2.170894in}{1.372269in}}%
\pgfpathlineto{\pgfqpoint{2.172386in}{1.374894in}}%
\pgfpathlineto{\pgfqpoint{2.171165in}{1.371466in}}%
\pgfpathlineto{\pgfqpoint{2.172928in}{1.373042in}}%
\pgfpathlineto{\pgfqpoint{2.175369in}{1.368400in}}%
\pgfpathlineto{\pgfqpoint{2.173471in}{1.374531in}}%
\pgfpathlineto{\pgfqpoint{2.175776in}{1.368699in}}%
\pgfpathlineto{\pgfqpoint{2.175912in}{1.369395in}}%
\pgfpathlineto{\pgfqpoint{2.176319in}{1.367842in}}%
\pgfpathlineto{\pgfqpoint{2.177133in}{1.368496in}}%
\pgfpathlineto{\pgfqpoint{2.177539in}{1.368810in}}%
\pgfpathlineto{\pgfqpoint{2.183371in}{1.352631in}}%
\pgfpathlineto{\pgfqpoint{2.184049in}{1.356168in}}%
\pgfpathlineto{\pgfqpoint{2.184999in}{1.353622in}}%
\pgfpathlineto{\pgfqpoint{2.185406in}{1.353090in}}%
\pgfpathlineto{\pgfqpoint{2.187304in}{1.347455in}}%
\pgfpathlineto{\pgfqpoint{2.187576in}{1.348196in}}%
\pgfpathlineto{\pgfqpoint{2.187847in}{1.346314in}}%
\pgfpathlineto{\pgfqpoint{2.188932in}{1.347615in}}%
\pgfpathlineto{\pgfqpoint{2.189610in}{1.343753in}}%
\pgfpathlineto{\pgfqpoint{2.190966in}{1.349694in}}%
\pgfpathlineto{\pgfqpoint{2.191509in}{1.348514in}}%
\pgfpathlineto{\pgfqpoint{2.193272in}{1.352983in}}%
\pgfpathlineto{\pgfqpoint{2.193679in}{1.351981in}}%
\pgfpathlineto{\pgfqpoint{2.194628in}{1.350258in}}%
\pgfpathlineto{\pgfqpoint{2.195171in}{1.352170in}}%
\pgfpathlineto{\pgfqpoint{2.197476in}{1.359317in}}%
\pgfpathlineto{\pgfqpoint{2.198019in}{1.358167in}}%
\pgfpathlineto{\pgfqpoint{2.198290in}{1.360189in}}%
\pgfpathlineto{\pgfqpoint{2.200867in}{1.373722in}}%
\pgfpathlineto{\pgfqpoint{2.201003in}{1.373587in}}%
\pgfpathlineto{\pgfqpoint{2.201274in}{1.372673in}}%
\pgfpathlineto{\pgfqpoint{2.201681in}{1.374166in}}%
\pgfpathlineto{\pgfqpoint{2.203851in}{1.378929in}}%
\pgfpathlineto{\pgfqpoint{2.203986in}{1.377942in}}%
\pgfpathlineto{\pgfqpoint{2.205071in}{1.380239in}}%
\pgfpathlineto{\pgfqpoint{2.205343in}{1.378656in}}%
\pgfpathlineto{\pgfqpoint{2.206292in}{1.381729in}}%
\pgfpathlineto{\pgfqpoint{2.207106in}{1.379755in}}%
\pgfpathlineto{\pgfqpoint{2.209140in}{1.375552in}}%
\pgfpathlineto{\pgfqpoint{2.209547in}{1.376365in}}%
\pgfpathlineto{\pgfqpoint{2.220804in}{1.351751in}}%
\pgfpathlineto{\pgfqpoint{2.221211in}{1.353976in}}%
\pgfpathlineto{\pgfqpoint{2.221482in}{1.355877in}}%
\pgfpathlineto{\pgfqpoint{2.222431in}{1.351215in}}%
\pgfpathlineto{\pgfqpoint{2.224059in}{1.348849in}}%
\pgfpathlineto{\pgfqpoint{2.223245in}{1.352395in}}%
\pgfpathlineto{\pgfqpoint{2.224330in}{1.348954in}}%
\pgfpathlineto{\pgfqpoint{2.234773in}{1.372371in}}%
\pgfpathlineto{\pgfqpoint{2.235044in}{1.372154in}}%
\pgfpathlineto{\pgfqpoint{2.235587in}{1.370769in}}%
\pgfpathlineto{\pgfqpoint{2.236401in}{1.372558in}}%
\pgfpathlineto{\pgfqpoint{2.237350in}{1.376460in}}%
\pgfpathlineto{\pgfqpoint{2.238028in}{1.374412in}}%
\pgfpathlineto{\pgfqpoint{2.240605in}{1.375275in}}%
\pgfpathlineto{\pgfqpoint{2.241012in}{1.373635in}}%
\pgfpathlineto{\pgfqpoint{2.241148in}{1.373183in}}%
\pgfpathlineto{\pgfqpoint{2.241961in}{1.375691in}}%
\pgfpathlineto{\pgfqpoint{2.242097in}{1.375103in}}%
\pgfpathlineto{\pgfqpoint{2.242504in}{1.376468in}}%
\pgfpathlineto{\pgfqpoint{2.242775in}{1.374782in}}%
\pgfpathlineto{\pgfqpoint{2.245216in}{1.366679in}}%
\pgfpathlineto{\pgfqpoint{2.246166in}{1.367411in}}%
\pgfpathlineto{\pgfqpoint{2.245759in}{1.365666in}}%
\pgfpathlineto{\pgfqpoint{2.246437in}{1.366347in}}%
\pgfpathlineto{\pgfqpoint{2.250641in}{1.350450in}}%
\pgfpathlineto{\pgfqpoint{2.250777in}{1.351012in}}%
\pgfpathlineto{\pgfqpoint{2.251726in}{1.351706in}}%
\pgfpathlineto{\pgfqpoint{2.252540in}{1.342797in}}%
\pgfpathlineto{\pgfqpoint{2.254981in}{1.331227in}}%
\pgfpathlineto{\pgfqpoint{2.256338in}{1.329860in}}%
\pgfpathlineto{\pgfqpoint{2.255931in}{1.333243in}}%
\pgfpathlineto{\pgfqpoint{2.256609in}{1.330897in}}%
\pgfpathlineto{\pgfqpoint{2.258236in}{1.334714in}}%
\pgfpathlineto{\pgfqpoint{2.257423in}{1.330740in}}%
\pgfpathlineto{\pgfqpoint{2.258372in}{1.334050in}}%
\pgfpathlineto{\pgfqpoint{2.258643in}{1.333286in}}%
\pgfpathlineto{\pgfqpoint{2.258779in}{1.333657in}}%
\pgfpathlineto{\pgfqpoint{2.260406in}{1.336621in}}%
\pgfpathlineto{\pgfqpoint{2.260542in}{1.335375in}}%
\pgfpathlineto{\pgfqpoint{2.261220in}{1.339558in}}%
\pgfpathlineto{\pgfqpoint{2.261491in}{1.339253in}}%
\pgfpathlineto{\pgfqpoint{2.265424in}{1.355989in}}%
\pgfpathlineto{\pgfqpoint{2.265560in}{1.355014in}}%
\pgfpathlineto{\pgfqpoint{2.266238in}{1.359544in}}%
\pgfpathlineto{\pgfqpoint{2.266509in}{1.358852in}}%
\pgfpathlineto{\pgfqpoint{2.268951in}{1.368539in}}%
\pgfpathlineto{\pgfqpoint{2.269086in}{1.367918in}}%
\pgfpathlineto{\pgfqpoint{2.269358in}{1.367698in}}%
\pgfpathlineto{\pgfqpoint{2.269629in}{1.368834in}}%
\pgfpathlineto{\pgfqpoint{2.269900in}{1.368810in}}%
\pgfpathlineto{\pgfqpoint{2.270985in}{1.369314in}}%
\pgfpathlineto{\pgfqpoint{2.270578in}{1.367325in}}%
\pgfpathlineto{\pgfqpoint{2.271528in}{1.369102in}}%
\pgfpathlineto{\pgfqpoint{2.272748in}{1.370136in}}%
\pgfpathlineto{\pgfqpoint{2.273426in}{1.372233in}}%
\pgfpathlineto{\pgfqpoint{2.274376in}{1.371679in}}%
\pgfpathlineto{\pgfqpoint{2.275189in}{1.369544in}}%
\pgfpathlineto{\pgfqpoint{2.276139in}{1.371217in}}%
\pgfpathlineto{\pgfqpoint{2.276546in}{1.372260in}}%
\pgfpathlineto{\pgfqpoint{2.277088in}{1.369793in}}%
\pgfpathlineto{\pgfqpoint{2.279123in}{1.362300in}}%
\pgfpathlineto{\pgfqpoint{2.279258in}{1.363098in}}%
\pgfpathlineto{\pgfqpoint{2.279394in}{1.364193in}}%
\pgfpathlineto{\pgfqpoint{2.280208in}{1.359583in}}%
\pgfpathlineto{\pgfqpoint{2.280479in}{1.361136in}}%
\pgfpathlineto{\pgfqpoint{2.284276in}{1.351890in}}%
\pgfpathlineto{\pgfqpoint{2.284412in}{1.353329in}}%
\pgfpathlineto{\pgfqpoint{2.285090in}{1.349912in}}%
\pgfpathlineto{\pgfqpoint{2.285497in}{1.349992in}}%
\pgfpathlineto{\pgfqpoint{2.286718in}{1.348777in}}%
\pgfpathlineto{\pgfqpoint{2.286039in}{1.351385in}}%
\pgfpathlineto{\pgfqpoint{2.286853in}{1.349332in}}%
\pgfpathlineto{\pgfqpoint{2.288209in}{1.353434in}}%
\pgfpathlineto{\pgfqpoint{2.287531in}{1.349147in}}%
\pgfpathlineto{\pgfqpoint{2.288616in}{1.351498in}}%
\pgfpathlineto{\pgfqpoint{2.289023in}{1.347719in}}%
\pgfpathlineto{\pgfqpoint{2.290379in}{1.349613in}}%
\pgfpathlineto{\pgfqpoint{2.293906in}{1.359064in}}%
\pgfpathlineto{\pgfqpoint{2.290651in}{1.349030in}}%
\pgfpathlineto{\pgfqpoint{2.294177in}{1.358415in}}%
\pgfpathlineto{\pgfqpoint{2.294448in}{1.359125in}}%
\pgfpathlineto{\pgfqpoint{2.294855in}{1.361723in}}%
\pgfpathlineto{\pgfqpoint{2.295398in}{1.358879in}}%
\pgfpathlineto{\pgfqpoint{2.296211in}{1.361335in}}%
\pgfpathlineto{\pgfqpoint{2.301365in}{1.372316in}}%
\pgfpathlineto{\pgfqpoint{2.301908in}{1.370875in}}%
\pgfpathlineto{\pgfqpoint{2.302043in}{1.370424in}}%
\pgfpathlineto{\pgfqpoint{2.302586in}{1.373002in}}%
\pgfpathlineto{\pgfqpoint{2.302857in}{1.372826in}}%
\pgfpathlineto{\pgfqpoint{2.302993in}{1.372900in}}%
\pgfpathlineto{\pgfqpoint{2.303264in}{1.371887in}}%
\pgfpathlineto{\pgfqpoint{2.303399in}{1.371043in}}%
\pgfpathlineto{\pgfqpoint{2.304213in}{1.373517in}}%
\pgfpathlineto{\pgfqpoint{2.304484in}{1.373240in}}%
\pgfpathlineto{\pgfqpoint{2.306519in}{1.377226in}}%
\pgfpathlineto{\pgfqpoint{2.306654in}{1.376196in}}%
\pgfpathlineto{\pgfqpoint{2.307875in}{1.377994in}}%
\pgfpathlineto{\pgfqpoint{2.307197in}{1.375357in}}%
\pgfpathlineto{\pgfqpoint{2.308282in}{1.376389in}}%
\pgfpathlineto{\pgfqpoint{2.308418in}{1.376824in}}%
\pgfpathlineto{\pgfqpoint{2.309096in}{1.374751in}}%
\pgfpathlineto{\pgfqpoint{2.309367in}{1.374733in}}%
\pgfpathlineto{\pgfqpoint{2.310181in}{1.375717in}}%
\pgfpathlineto{\pgfqpoint{2.311401in}{1.372141in}}%
\pgfpathlineto{\pgfqpoint{2.312215in}{1.373190in}}%
\pgfpathlineto{\pgfqpoint{2.312486in}{1.372361in}}%
\pgfpathlineto{\pgfqpoint{2.312622in}{1.370935in}}%
\pgfpathlineto{\pgfqpoint{2.313436in}{1.372487in}}%
\pgfpathlineto{\pgfqpoint{2.313978in}{1.371396in}}%
\pgfpathlineto{\pgfqpoint{2.314656in}{1.373595in}}%
\pgfpathlineto{\pgfqpoint{2.315470in}{1.371261in}}%
\pgfpathlineto{\pgfqpoint{2.315741in}{1.371792in}}%
\pgfpathlineto{\pgfqpoint{2.315877in}{1.371060in}}%
\pgfpathlineto{\pgfqpoint{2.317911in}{1.362861in}}%
\pgfpathlineto{\pgfqpoint{2.318589in}{1.363298in}}%
\pgfpathlineto{\pgfqpoint{2.320217in}{1.359178in}}%
\pgfpathlineto{\pgfqpoint{2.320624in}{1.359576in}}%
\pgfpathlineto{\pgfqpoint{2.321166in}{1.357905in}}%
\pgfpathlineto{\pgfqpoint{2.321438in}{1.356041in}}%
\pgfpathlineto{\pgfqpoint{2.322794in}{1.356781in}}%
\pgfpathlineto{\pgfqpoint{2.323472in}{1.357072in}}%
\pgfpathlineto{\pgfqpoint{2.323608in}{1.356176in}}%
\pgfpathlineto{\pgfqpoint{2.324286in}{1.354355in}}%
\pgfpathlineto{\pgfqpoint{2.324828in}{1.356465in}}%
\pgfpathlineto{\pgfqpoint{2.324964in}{1.356380in}}%
\pgfpathlineto{\pgfqpoint{2.326456in}{1.358489in}}%
\pgfpathlineto{\pgfqpoint{2.325506in}{1.354798in}}%
\pgfpathlineto{\pgfqpoint{2.326591in}{1.357010in}}%
\pgfpathlineto{\pgfqpoint{2.328219in}{1.356148in}}%
\pgfpathlineto{\pgfqpoint{2.327541in}{1.358184in}}%
\pgfpathlineto{\pgfqpoint{2.328354in}{1.356395in}}%
\pgfpathlineto{\pgfqpoint{2.334593in}{1.372237in}}%
\pgfpathlineto{\pgfqpoint{2.336763in}{1.376960in}}%
\pgfpathlineto{\pgfqpoint{2.337713in}{1.378083in}}%
\pgfpathlineto{\pgfqpoint{2.338526in}{1.376600in}}%
\pgfpathlineto{\pgfqpoint{2.339476in}{1.379906in}}%
\pgfpathlineto{\pgfqpoint{2.340696in}{1.381177in}}%
\pgfpathlineto{\pgfqpoint{2.341103in}{1.380263in}}%
\pgfpathlineto{\pgfqpoint{2.341510in}{1.379637in}}%
\pgfpathlineto{\pgfqpoint{2.342053in}{1.380671in}}%
\pgfpathlineto{\pgfqpoint{2.342459in}{1.380702in}}%
\pgfpathlineto{\pgfqpoint{2.344494in}{1.382114in}}%
\pgfpathlineto{\pgfqpoint{2.344765in}{1.382288in}}%
\pgfpathlineto{\pgfqpoint{2.345036in}{1.380915in}}%
\pgfpathlineto{\pgfqpoint{2.351818in}{1.367462in}}%
\pgfpathlineto{\pgfqpoint{2.346121in}{1.382076in}}%
\pgfpathlineto{\pgfqpoint{2.352089in}{1.368345in}}%
\pgfpathlineto{\pgfqpoint{2.353174in}{1.369180in}}%
\pgfpathlineto{\pgfqpoint{2.352767in}{1.366753in}}%
\pgfpathlineto{\pgfqpoint{2.353581in}{1.368093in}}%
\pgfpathlineto{\pgfqpoint{2.354666in}{1.365211in}}%
\pgfpathlineto{\pgfqpoint{2.354937in}{1.367386in}}%
\pgfpathlineto{\pgfqpoint{2.355344in}{1.369586in}}%
\pgfpathlineto{\pgfqpoint{2.355751in}{1.367114in}}%
\pgfpathlineto{\pgfqpoint{2.356429in}{1.367331in}}%
\pgfpathlineto{\pgfqpoint{2.356564in}{1.367478in}}%
\pgfpathlineto{\pgfqpoint{2.356971in}{1.365733in}}%
\pgfpathlineto{\pgfqpoint{2.359141in}{1.363435in}}%
\pgfpathlineto{\pgfqpoint{2.357243in}{1.366172in}}%
\pgfpathlineto{\pgfqpoint{2.359548in}{1.363935in}}%
\pgfpathlineto{\pgfqpoint{2.359955in}{1.366059in}}%
\pgfpathlineto{\pgfqpoint{2.361311in}{1.365456in}}%
\pgfpathlineto{\pgfqpoint{2.361718in}{1.364279in}}%
\pgfpathlineto{\pgfqpoint{2.361989in}{1.366065in}}%
\pgfpathlineto{\pgfqpoint{2.362668in}{1.365396in}}%
\pgfpathlineto{\pgfqpoint{2.364024in}{1.369291in}}%
\pgfpathlineto{\pgfqpoint{2.364566in}{1.368239in}}%
\pgfpathlineto{\pgfqpoint{2.364702in}{1.368211in}}%
\pgfpathlineto{\pgfqpoint{2.364838in}{1.369239in}}%
\pgfpathlineto{\pgfqpoint{2.367821in}{1.378554in}}%
\pgfpathlineto{\pgfqpoint{2.369042in}{1.375980in}}%
\pgfpathlineto{\pgfqpoint{2.369313in}{1.377605in}}%
\pgfpathlineto{\pgfqpoint{2.370669in}{1.379993in}}%
\pgfpathlineto{\pgfqpoint{2.371076in}{1.379465in}}%
\pgfpathlineto{\pgfqpoint{2.372704in}{1.382866in}}%
\pgfpathlineto{\pgfqpoint{2.373382in}{1.381441in}}%
\pgfpathlineto{\pgfqpoint{2.375552in}{1.377888in}}%
\pgfpathlineto{\pgfqpoint{2.375959in}{1.380360in}}%
\pgfpathlineto{\pgfqpoint{2.376094in}{1.381035in}}%
\pgfpathlineto{\pgfqpoint{2.376501in}{1.377946in}}%
\pgfpathlineto{\pgfqpoint{2.377179in}{1.380022in}}%
\pgfpathlineto{\pgfqpoint{2.377722in}{1.377913in}}%
\pgfpathlineto{\pgfqpoint{2.378536in}{1.380477in}}%
\pgfpathlineto{\pgfqpoint{2.378671in}{1.380420in}}%
\pgfpathlineto{\pgfqpoint{2.380570in}{1.374752in}}%
\pgfpathlineto{\pgfqpoint{2.381113in}{1.375684in}}%
\pgfpathlineto{\pgfqpoint{2.381248in}{1.376025in}}%
\pgfpathlineto{\pgfqpoint{2.381655in}{1.374353in}}%
\pgfpathlineto{\pgfqpoint{2.381791in}{1.374404in}}%
\pgfpathlineto{\pgfqpoint{2.383961in}{1.367559in}}%
\pgfpathlineto{\pgfqpoint{2.384774in}{1.364688in}}%
\pgfpathlineto{\pgfqpoint{2.385724in}{1.365287in}}%
\pgfpathlineto{\pgfqpoint{2.386131in}{1.363645in}}%
\pgfpathlineto{\pgfqpoint{2.386944in}{1.358598in}}%
\pgfpathlineto{\pgfqpoint{2.388436in}{1.352158in}}%
\pgfpathlineto{\pgfqpoint{2.388708in}{1.355478in}}%
\pgfpathlineto{\pgfqpoint{2.390199in}{1.357667in}}%
\pgfpathlineto{\pgfqpoint{2.389521in}{1.353823in}}%
\pgfpathlineto{\pgfqpoint{2.390471in}{1.355797in}}%
\pgfpathlineto{\pgfqpoint{2.390606in}{1.355325in}}%
\pgfpathlineto{\pgfqpoint{2.391284in}{1.357716in}}%
\pgfpathlineto{\pgfqpoint{2.391556in}{1.356871in}}%
\pgfpathlineto{\pgfqpoint{2.391691in}{1.357142in}}%
\pgfpathlineto{\pgfqpoint{2.392098in}{1.354763in}}%
\pgfpathlineto{\pgfqpoint{2.392912in}{1.354722in}}%
\pgfpathlineto{\pgfqpoint{2.392369in}{1.355783in}}%
\pgfpathlineto{\pgfqpoint{2.393048in}{1.355148in}}%
\pgfpathlineto{\pgfqpoint{2.395353in}{1.361975in}}%
\pgfpathlineto{\pgfqpoint{2.395489in}{1.361684in}}%
\pgfpathlineto{\pgfqpoint{2.396167in}{1.361476in}}%
\pgfpathlineto{\pgfqpoint{2.396438in}{1.362592in}}%
\pgfpathlineto{\pgfqpoint{2.398337in}{1.368944in}}%
\pgfpathlineto{\pgfqpoint{2.399829in}{1.368460in}}%
\pgfpathlineto{\pgfqpoint{2.400100in}{1.367679in}}%
\pgfpathlineto{\pgfqpoint{2.400643in}{1.369653in}}%
\pgfpathlineto{\pgfqpoint{2.401999in}{1.371995in}}%
\pgfpathlineto{\pgfqpoint{2.401592in}{1.368668in}}%
\pgfpathlineto{\pgfqpoint{2.402270in}{1.370905in}}%
\pgfpathlineto{\pgfqpoint{2.402677in}{1.371996in}}%
\pgfpathlineto{\pgfqpoint{2.405118in}{1.381838in}}%
\pgfpathlineto{\pgfqpoint{2.405389in}{1.380094in}}%
\pgfpathlineto{\pgfqpoint{2.405932in}{1.382798in}}%
\pgfpathlineto{\pgfqpoint{2.406610in}{1.381851in}}%
\pgfpathlineto{\pgfqpoint{2.407559in}{1.383595in}}%
\pgfpathlineto{\pgfqpoint{2.408238in}{1.382561in}}%
\pgfpathlineto{\pgfqpoint{2.409865in}{1.382343in}}%
\pgfpathlineto{\pgfqpoint{2.411221in}{1.380740in}}%
\pgfpathlineto{\pgfqpoint{2.410408in}{1.382667in}}%
\pgfpathlineto{\pgfqpoint{2.411493in}{1.381003in}}%
\pgfpathlineto{\pgfqpoint{2.411628in}{1.381715in}}%
\pgfpathlineto{\pgfqpoint{2.412171in}{1.379171in}}%
\pgfpathlineto{\pgfqpoint{2.412713in}{1.380476in}}%
\pgfpathlineto{\pgfqpoint{2.414205in}{1.374677in}}%
\pgfpathlineto{\pgfqpoint{2.414748in}{1.375112in}}%
\pgfpathlineto{\pgfqpoint{2.417189in}{1.368798in}}%
\pgfpathlineto{\pgfqpoint{2.417324in}{1.369005in}}%
\pgfpathlineto{\pgfqpoint{2.417867in}{1.367117in}}%
\pgfpathlineto{\pgfqpoint{2.418003in}{1.367581in}}%
\pgfpathlineto{\pgfqpoint{2.422614in}{1.360715in}}%
\pgfpathlineto{\pgfqpoint{2.422885in}{1.362043in}}%
\pgfpathlineto{\pgfqpoint{2.423156in}{1.363703in}}%
\pgfpathlineto{\pgfqpoint{2.423699in}{1.361219in}}%
\pgfpathlineto{\pgfqpoint{2.424377in}{1.362623in}}%
\pgfpathlineto{\pgfqpoint{2.425733in}{1.363486in}}%
\pgfpathlineto{\pgfqpoint{2.426276in}{1.360101in}}%
\pgfpathlineto{\pgfqpoint{2.427903in}{1.362127in}}%
\pgfpathlineto{\pgfqpoint{2.428039in}{1.361904in}}%
\pgfpathlineto{\pgfqpoint{2.429124in}{1.362259in}}%
\pgfpathlineto{\pgfqpoint{2.428581in}{1.360823in}}%
\pgfpathlineto{\pgfqpoint{2.429259in}{1.361713in}}%
\pgfpathlineto{\pgfqpoint{2.429395in}{1.360784in}}%
\pgfpathlineto{\pgfqpoint{2.430480in}{1.363489in}}%
\pgfpathlineto{\pgfqpoint{2.430751in}{1.365073in}}%
\pgfpathlineto{\pgfqpoint{2.433464in}{1.374263in}}%
\pgfpathlineto{\pgfqpoint{2.433599in}{1.374055in}}%
\pgfpathlineto{\pgfqpoint{2.441059in}{1.391113in}}%
\pgfpathlineto{\pgfqpoint{2.434142in}{1.373037in}}%
\pgfpathlineto{\pgfqpoint{2.442279in}{1.389721in}}%
\pgfpathlineto{\pgfqpoint{2.443907in}{1.389231in}}%
\pgfpathlineto{\pgfqpoint{2.443229in}{1.390309in}}%
\pgfpathlineto{\pgfqpoint{2.444043in}{1.389641in}}%
\pgfpathlineto{\pgfqpoint{2.445263in}{1.390851in}}%
\pgfpathlineto{\pgfqpoint{2.444721in}{1.388128in}}%
\pgfpathlineto{\pgfqpoint{2.445806in}{1.389583in}}%
\pgfpathlineto{\pgfqpoint{2.447433in}{1.388748in}}%
\pgfpathlineto{\pgfqpoint{2.447704in}{1.389696in}}%
\pgfpathlineto{\pgfqpoint{2.449196in}{1.390945in}}%
\pgfpathlineto{\pgfqpoint{2.449332in}{1.390450in}}%
\pgfpathlineto{\pgfqpoint{2.451095in}{1.383701in}}%
\pgfpathlineto{\pgfqpoint{2.451502in}{1.385065in}}%
\pgfpathlineto{\pgfqpoint{2.452451in}{1.386765in}}%
\pgfpathlineto{\pgfqpoint{2.452994in}{1.385165in}}%
\pgfpathlineto{\pgfqpoint{2.455571in}{1.380939in}}%
\pgfpathlineto{\pgfqpoint{2.455842in}{1.381704in}}%
\pgfpathlineto{\pgfqpoint{2.456791in}{1.379634in}}%
\pgfpathlineto{\pgfqpoint{2.456927in}{1.379709in}}%
\pgfpathlineto{\pgfqpoint{2.457334in}{1.378400in}}%
\pgfpathlineto{\pgfqpoint{2.458961in}{1.376692in}}%
\pgfpathlineto{\pgfqpoint{2.459233in}{1.377173in}}%
\pgfpathlineto{\pgfqpoint{2.460860in}{1.378219in}}%
\pgfpathlineto{\pgfqpoint{2.459504in}{1.377055in}}%
\pgfpathlineto{\pgfqpoint{2.460996in}{1.378153in}}%
\pgfpathlineto{\pgfqpoint{2.461809in}{1.375679in}}%
\pgfpathlineto{\pgfqpoint{2.462216in}{1.378864in}}%
\pgfpathlineto{\pgfqpoint{2.464929in}{1.381050in}}%
\pgfpathlineto{\pgfqpoint{2.462623in}{1.378353in}}%
\pgfpathlineto{\pgfqpoint{2.465200in}{1.379952in}}%
\pgfpathlineto{\pgfqpoint{2.465336in}{1.379557in}}%
\pgfpathlineto{\pgfqpoint{2.465878in}{1.381718in}}%
\pgfpathlineto{\pgfqpoint{2.466421in}{1.380693in}}%
\pgfpathlineto{\pgfqpoint{2.472795in}{1.389704in}}%
\pgfpathlineto{\pgfqpoint{2.473744in}{1.387949in}}%
\pgfpathlineto{\pgfqpoint{2.474287in}{1.389786in}}%
\pgfpathlineto{\pgfqpoint{2.474829in}{1.388292in}}%
\pgfpathlineto{\pgfqpoint{2.475372in}{1.390672in}}%
\pgfpathlineto{\pgfqpoint{2.476050in}{1.389521in}}%
\pgfpathlineto{\pgfqpoint{2.477813in}{1.390980in}}%
\pgfpathlineto{\pgfqpoint{2.477135in}{1.388809in}}%
\pgfpathlineto{\pgfqpoint{2.477949in}{1.390467in}}%
\pgfpathlineto{\pgfqpoint{2.478084in}{1.389880in}}%
\pgfpathlineto{\pgfqpoint{2.478491in}{1.390938in}}%
\pgfpathlineto{\pgfqpoint{2.479441in}{1.390834in}}%
\pgfpathlineto{\pgfqpoint{2.482696in}{1.381767in}}%
\pgfpathlineto{\pgfqpoint{2.483103in}{1.383230in}}%
\pgfpathlineto{\pgfqpoint{2.483916in}{1.380007in}}%
\pgfpathlineto{\pgfqpoint{2.484052in}{1.380125in}}%
\pgfpathlineto{\pgfqpoint{2.486493in}{1.374464in}}%
\pgfpathlineto{\pgfqpoint{2.487171in}{1.373191in}}%
\pgfpathlineto{\pgfqpoint{2.489477in}{1.366374in}}%
\pgfpathlineto{\pgfqpoint{2.489884in}{1.364254in}}%
\pgfpathlineto{\pgfqpoint{2.490969in}{1.360820in}}%
\pgfpathlineto{\pgfqpoint{2.491511in}{1.362738in}}%
\pgfpathlineto{\pgfqpoint{2.494088in}{1.361871in}}%
\pgfpathlineto{\pgfqpoint{2.495444in}{1.360149in}}%
\pgfpathlineto{\pgfqpoint{2.494631in}{1.361949in}}%
\pgfpathlineto{\pgfqpoint{2.495716in}{1.360921in}}%
\pgfpathlineto{\pgfqpoint{2.497072in}{1.357732in}}%
\pgfpathlineto{\pgfqpoint{2.497750in}{1.360648in}}%
\pgfpathlineto{\pgfqpoint{2.498835in}{1.362442in}}%
\pgfpathlineto{\pgfqpoint{2.498021in}{1.360018in}}%
\pgfpathlineto{\pgfqpoint{2.499242in}{1.360518in}}%
\pgfpathlineto{\pgfqpoint{2.499378in}{1.360401in}}%
\pgfpathlineto{\pgfqpoint{2.499513in}{1.361094in}}%
\pgfpathlineto{\pgfqpoint{2.501548in}{1.369992in}}%
\pgfpathlineto{\pgfqpoint{2.501683in}{1.368552in}}%
\pgfpathlineto{\pgfqpoint{2.502768in}{1.373215in}}%
\pgfpathlineto{\pgfqpoint{2.504260in}{1.370993in}}%
\pgfpathlineto{\pgfqpoint{2.503582in}{1.373429in}}%
\pgfpathlineto{\pgfqpoint{2.504396in}{1.371868in}}%
\pgfpathlineto{\pgfqpoint{2.505481in}{1.371845in}}%
\pgfpathlineto{\pgfqpoint{2.504803in}{1.373205in}}%
\pgfpathlineto{\pgfqpoint{2.505752in}{1.372985in}}%
\pgfpathlineto{\pgfqpoint{2.506159in}{1.372419in}}%
\pgfpathlineto{\pgfqpoint{2.507651in}{1.375313in}}%
\pgfpathlineto{\pgfqpoint{2.509414in}{1.379206in}}%
\pgfpathlineto{\pgfqpoint{2.510906in}{1.377743in}}%
\pgfpathlineto{\pgfqpoint{2.511041in}{1.378196in}}%
\pgfpathlineto{\pgfqpoint{2.511855in}{1.382950in}}%
\pgfpathlineto{\pgfqpoint{2.513076in}{1.381189in}}%
\pgfpathlineto{\pgfqpoint{2.515110in}{1.378539in}}%
\pgfpathlineto{\pgfqpoint{2.515517in}{1.379299in}}%
\pgfpathlineto{\pgfqpoint{2.516466in}{1.377833in}}%
\pgfpathlineto{\pgfqpoint{2.519043in}{1.372746in}}%
\pgfpathlineto{\pgfqpoint{2.516738in}{1.378795in}}%
\pgfpathlineto{\pgfqpoint{2.519314in}{1.373700in}}%
\pgfpathlineto{\pgfqpoint{2.521078in}{1.377118in}}%
\pgfpathlineto{\pgfqpoint{2.521349in}{1.376755in}}%
\pgfpathlineto{\pgfqpoint{2.522027in}{1.377112in}}%
\pgfpathlineto{\pgfqpoint{2.523112in}{1.374619in}}%
\pgfpathlineto{\pgfqpoint{2.524197in}{1.372516in}}%
\pgfpathlineto{\pgfqpoint{2.525418in}{1.370369in}}%
\pgfpathlineto{\pgfqpoint{2.525824in}{1.371810in}}%
\pgfpathlineto{\pgfqpoint{2.526231in}{1.373522in}}%
\pgfpathlineto{\pgfqpoint{2.527316in}{1.372597in}}%
\pgfpathlineto{\pgfqpoint{2.529079in}{1.370900in}}%
\pgfpathlineto{\pgfqpoint{2.529215in}{1.371052in}}%
\pgfpathlineto{\pgfqpoint{2.530843in}{1.368828in}}%
\pgfpathlineto{\pgfqpoint{2.531114in}{1.369062in}}%
\pgfpathlineto{\pgfqpoint{2.533691in}{1.374148in}}%
\pgfpathlineto{\pgfqpoint{2.539387in}{1.385573in}}%
\pgfpathlineto{\pgfqpoint{2.539523in}{1.385453in}}%
\pgfpathlineto{\pgfqpoint{2.539929in}{1.386426in}}%
\pgfpathlineto{\pgfqpoint{2.541557in}{1.389070in}}%
\pgfpathlineto{\pgfqpoint{2.541828in}{1.388011in}}%
\pgfpathlineto{\pgfqpoint{2.541964in}{1.387432in}}%
\pgfpathlineto{\pgfqpoint{2.542913in}{1.389811in}}%
\pgfpathlineto{\pgfqpoint{2.544812in}{1.392106in}}%
\pgfpathlineto{\pgfqpoint{2.545626in}{1.391198in}}%
\pgfpathlineto{\pgfqpoint{2.546575in}{1.387656in}}%
\pgfpathlineto{\pgfqpoint{2.548067in}{1.388661in}}%
\pgfpathlineto{\pgfqpoint{2.549966in}{1.386552in}}%
\pgfpathlineto{\pgfqpoint{2.550779in}{1.387392in}}%
\pgfpathlineto{\pgfqpoint{2.551051in}{1.389101in}}%
\pgfpathlineto{\pgfqpoint{2.551729in}{1.385918in}}%
\pgfpathlineto{\pgfqpoint{2.552000in}{1.385892in}}%
\pgfpathlineto{\pgfqpoint{2.552271in}{1.386300in}}%
\pgfpathlineto{\pgfqpoint{2.553899in}{1.382673in}}%
\pgfpathlineto{\pgfqpoint{2.557018in}{1.377225in}}%
\pgfpathlineto{\pgfqpoint{2.554577in}{1.383999in}}%
\pgfpathlineto{\pgfqpoint{2.557289in}{1.377989in}}%
\pgfpathlineto{\pgfqpoint{2.558510in}{1.376886in}}%
\pgfpathlineto{\pgfqpoint{2.558917in}{1.379106in}}%
\pgfpathlineto{\pgfqpoint{2.561765in}{1.370651in}}%
\pgfpathlineto{\pgfqpoint{2.562986in}{1.368983in}}%
\pgfpathlineto{\pgfqpoint{2.564071in}{1.366788in}}%
\pgfpathlineto{\pgfqpoint{2.564613in}{1.367525in}}%
\pgfpathlineto{\pgfqpoint{2.566648in}{1.371825in}}%
\pgfpathlineto{\pgfqpoint{2.566919in}{1.371169in}}%
\pgfpathlineto{\pgfqpoint{2.568004in}{1.372465in}}%
\pgfpathlineto{\pgfqpoint{2.573429in}{1.381306in}}%
\pgfpathlineto{\pgfqpoint{2.574514in}{1.382214in}}%
\pgfpathlineto{\pgfqpoint{2.575056in}{1.381143in}}%
\pgfpathlineto{\pgfqpoint{2.575328in}{1.380721in}}%
\pgfpathlineto{\pgfqpoint{2.575870in}{1.381814in}}%
\pgfpathlineto{\pgfqpoint{2.576141in}{1.382860in}}%
\pgfpathlineto{\pgfqpoint{2.576819in}{1.380850in}}%
\pgfpathlineto{\pgfqpoint{2.577226in}{1.380858in}}%
\pgfpathlineto{\pgfqpoint{2.578718in}{1.379552in}}%
\pgfpathlineto{\pgfqpoint{2.578989in}{1.380463in}}%
\pgfpathlineto{\pgfqpoint{2.579261in}{1.380779in}}%
\pgfpathlineto{\pgfqpoint{2.579803in}{1.379193in}}%
\pgfpathlineto{\pgfqpoint{2.580481in}{1.377826in}}%
\pgfpathlineto{\pgfqpoint{2.581159in}{1.379557in}}%
\pgfpathlineto{\pgfqpoint{2.581295in}{1.380123in}}%
\pgfpathlineto{\pgfqpoint{2.581838in}{1.377874in}}%
\pgfpathlineto{\pgfqpoint{2.582516in}{1.379201in}}%
\pgfpathlineto{\pgfqpoint{2.584279in}{1.376774in}}%
\pgfpathlineto{\pgfqpoint{2.584686in}{1.377489in}}%
\pgfpathlineto{\pgfqpoint{2.584821in}{1.378186in}}%
\pgfpathlineto{\pgfqpoint{2.585499in}{1.376417in}}%
\pgfpathlineto{\pgfqpoint{2.586042in}{1.376730in}}%
\pgfpathlineto{\pgfqpoint{2.590789in}{1.371535in}}%
\pgfpathlineto{\pgfqpoint{2.591060in}{1.371947in}}%
\pgfpathlineto{\pgfqpoint{2.591331in}{1.373262in}}%
\pgfpathlineto{\pgfqpoint{2.592416in}{1.370675in}}%
\pgfpathlineto{\pgfqpoint{2.594858in}{1.364055in}}%
\pgfpathlineto{\pgfqpoint{2.596078in}{1.365949in}}%
\pgfpathlineto{\pgfqpoint{2.597163in}{1.370210in}}%
\pgfpathlineto{\pgfqpoint{2.597841in}{1.369215in}}%
\pgfpathlineto{\pgfqpoint{2.598384in}{1.365532in}}%
\pgfpathlineto{\pgfqpoint{2.599604in}{1.366537in}}%
\pgfpathlineto{\pgfqpoint{2.603944in}{1.372487in}}%
\pgfpathlineto{\pgfqpoint{2.604487in}{1.373610in}}%
\pgfpathlineto{\pgfqpoint{2.607606in}{1.381107in}}%
\pgfpathlineto{\pgfqpoint{2.607742in}{1.381058in}}%
\pgfpathlineto{\pgfqpoint{2.609641in}{1.387237in}}%
\pgfpathlineto{\pgfqpoint{2.610454in}{1.386418in}}%
\pgfpathlineto{\pgfqpoint{2.610590in}{1.386093in}}%
\pgfpathlineto{\pgfqpoint{2.610861in}{1.387164in}}%
\pgfpathlineto{\pgfqpoint{2.611539in}{1.387036in}}%
\pgfpathlineto{\pgfqpoint{2.613574in}{1.389566in}}%
\pgfpathlineto{\pgfqpoint{2.613709in}{1.389446in}}%
\pgfpathlineto{\pgfqpoint{2.615879in}{1.388391in}}%
\pgfpathlineto{\pgfqpoint{2.616015in}{1.389318in}}%
\pgfpathlineto{\pgfqpoint{2.616286in}{1.390078in}}%
\pgfpathlineto{\pgfqpoint{2.616964in}{1.388764in}}%
\pgfpathlineto{\pgfqpoint{2.617371in}{1.389097in}}%
\pgfpathlineto{\pgfqpoint{2.618728in}{1.386036in}}%
\pgfpathlineto{\pgfqpoint{2.619134in}{1.387937in}}%
\pgfpathlineto{\pgfqpoint{2.620491in}{1.390205in}}%
\pgfpathlineto{\pgfqpoint{2.619541in}{1.387457in}}%
\pgfpathlineto{\pgfqpoint{2.620898in}{1.389094in}}%
\pgfpathlineto{\pgfqpoint{2.621983in}{1.388683in}}%
\pgfpathlineto{\pgfqpoint{2.621440in}{1.390957in}}%
\pgfpathlineto{\pgfqpoint{2.622254in}{1.389242in}}%
\pgfpathlineto{\pgfqpoint{2.622389in}{1.389745in}}%
\pgfpathlineto{\pgfqpoint{2.623068in}{1.387718in}}%
\pgfpathlineto{\pgfqpoint{2.624831in}{1.384448in}}%
\pgfpathlineto{\pgfqpoint{2.628764in}{1.378227in}}%
\pgfpathlineto{\pgfqpoint{2.629306in}{1.380114in}}%
\pgfpathlineto{\pgfqpoint{2.629442in}{1.380525in}}%
\pgfpathlineto{\pgfqpoint{2.629984in}{1.379061in}}%
\pgfpathlineto{\pgfqpoint{2.630527in}{1.379091in}}%
\pgfpathlineto{\pgfqpoint{2.634189in}{1.378373in}}%
\pgfpathlineto{\pgfqpoint{2.636223in}{1.383313in}}%
\pgfpathlineto{\pgfqpoint{2.638529in}{1.388060in}}%
\pgfpathlineto{\pgfqpoint{2.641377in}{1.392698in}}%
\pgfpathlineto{\pgfqpoint{2.639207in}{1.386576in}}%
\pgfpathlineto{\pgfqpoint{2.641513in}{1.392382in}}%
\pgfpathlineto{\pgfqpoint{2.642055in}{1.390510in}}%
\pgfpathlineto{\pgfqpoint{2.642869in}{1.392842in}}%
\pgfpathlineto{\pgfqpoint{2.643818in}{1.392817in}}%
\pgfpathlineto{\pgfqpoint{2.643140in}{1.392502in}}%
\pgfpathlineto{\pgfqpoint{2.644225in}{1.392147in}}%
\pgfpathlineto{\pgfqpoint{2.644361in}{1.391678in}}%
\pgfpathlineto{\pgfqpoint{2.644903in}{1.394122in}}%
\pgfpathlineto{\pgfqpoint{2.646259in}{1.396381in}}%
\pgfpathlineto{\pgfqpoint{2.646531in}{1.395460in}}%
\pgfpathlineto{\pgfqpoint{2.646666in}{1.394739in}}%
\pgfpathlineto{\pgfqpoint{2.647480in}{1.398289in}}%
\pgfpathlineto{\pgfqpoint{2.648429in}{1.398223in}}%
\pgfpathlineto{\pgfqpoint{2.648023in}{1.397614in}}%
\pgfpathlineto{\pgfqpoint{2.648701in}{1.397882in}}%
\pgfpathlineto{\pgfqpoint{2.651278in}{1.393822in}}%
\pgfpathlineto{\pgfqpoint{2.651549in}{1.394301in}}%
\pgfpathlineto{\pgfqpoint{2.652769in}{1.395230in}}%
\pgfpathlineto{\pgfqpoint{2.652363in}{1.393102in}}%
\pgfpathlineto{\pgfqpoint{2.653312in}{1.394402in}}%
\pgfpathlineto{\pgfqpoint{2.655618in}{1.389869in}}%
\pgfpathlineto{\pgfqpoint{2.655889in}{1.390417in}}%
\pgfpathlineto{\pgfqpoint{2.656838in}{1.390476in}}%
\pgfpathlineto{\pgfqpoint{2.656296in}{1.389883in}}%
\pgfpathlineto{\pgfqpoint{2.656974in}{1.390381in}}%
\pgfpathlineto{\pgfqpoint{2.658737in}{1.386894in}}%
\pgfpathlineto{\pgfqpoint{2.659144in}{1.388451in}}%
\pgfpathlineto{\pgfqpoint{2.660093in}{1.386430in}}%
\pgfpathlineto{\pgfqpoint{2.660229in}{1.386628in}}%
\pgfpathlineto{\pgfqpoint{2.660636in}{1.386762in}}%
\pgfpathlineto{\pgfqpoint{2.662941in}{1.382324in}}%
\pgfpathlineto{\pgfqpoint{2.664162in}{1.379992in}}%
\pgfpathlineto{\pgfqpoint{2.663213in}{1.382493in}}%
\pgfpathlineto{\pgfqpoint{2.664569in}{1.381937in}}%
\pgfpathlineto{\pgfqpoint{2.667010in}{1.386989in}}%
\pgfpathlineto{\pgfqpoint{2.668366in}{1.388773in}}%
\pgfpathlineto{\pgfqpoint{2.670536in}{1.393013in}}%
\pgfpathlineto{\pgfqpoint{2.671757in}{1.395493in}}%
\pgfpathlineto{\pgfqpoint{2.672435in}{1.397282in}}%
\pgfpathlineto{\pgfqpoint{2.673520in}{1.396616in}}%
\pgfpathlineto{\pgfqpoint{2.683285in}{1.403702in}}%
\pgfpathlineto{\pgfqpoint{2.683556in}{1.402759in}}%
\pgfpathlineto{\pgfqpoint{2.684099in}{1.405044in}}%
\pgfpathlineto{\pgfqpoint{2.684777in}{1.403149in}}%
\pgfpathlineto{\pgfqpoint{2.685048in}{1.403880in}}%
\pgfpathlineto{\pgfqpoint{2.685862in}{1.402076in}}%
\pgfpathlineto{\pgfqpoint{2.688032in}{1.400126in}}%
\pgfpathlineto{\pgfqpoint{2.691558in}{1.394398in}}%
\pgfpathlineto{\pgfqpoint{2.691965in}{1.395085in}}%
\pgfpathlineto{\pgfqpoint{2.692508in}{1.393790in}}%
\pgfpathlineto{\pgfqpoint{2.692914in}{1.393848in}}%
\pgfpathlineto{\pgfqpoint{2.693999in}{1.392028in}}%
\pgfpathlineto{\pgfqpoint{2.695763in}{1.389586in}}%
\pgfpathlineto{\pgfqpoint{2.696441in}{1.388498in}}%
\pgfpathlineto{\pgfqpoint{2.697933in}{1.384194in}}%
\pgfpathlineto{\pgfqpoint{2.698339in}{1.385183in}}%
\pgfpathlineto{\pgfqpoint{2.699424in}{1.386314in}}%
\pgfpathlineto{\pgfqpoint{2.699018in}{1.384699in}}%
\pgfpathlineto{\pgfqpoint{2.699967in}{1.385824in}}%
\pgfpathlineto{\pgfqpoint{2.700645in}{1.384662in}}%
\pgfpathlineto{\pgfqpoint{2.701323in}{1.386535in}}%
\pgfpathlineto{\pgfqpoint{2.702001in}{1.385284in}}%
\pgfpathlineto{\pgfqpoint{2.702544in}{1.387794in}}%
\pgfpathlineto{\pgfqpoint{2.704171in}{1.390192in}}%
\pgfpathlineto{\pgfqpoint{2.704307in}{1.390034in}}%
\pgfpathlineto{\pgfqpoint{2.709868in}{1.395175in}}%
\pgfpathlineto{\pgfqpoint{2.711359in}{1.397526in}}%
\pgfpathlineto{\pgfqpoint{2.711631in}{1.397128in}}%
\pgfpathlineto{\pgfqpoint{2.712716in}{1.397792in}}%
\pgfpathlineto{\pgfqpoint{2.712851in}{1.398122in}}%
\pgfpathlineto{\pgfqpoint{2.714479in}{1.402138in}}%
\pgfpathlineto{\pgfqpoint{2.715021in}{1.400698in}}%
\pgfpathlineto{\pgfqpoint{2.715293in}{1.401766in}}%
\pgfpathlineto{\pgfqpoint{2.721260in}{1.395651in}}%
\pgfpathlineto{\pgfqpoint{2.726414in}{1.383269in}}%
\pgfpathlineto{\pgfqpoint{2.726549in}{1.383912in}}%
\pgfpathlineto{\pgfqpoint{2.726956in}{1.382123in}}%
\pgfpathlineto{\pgfqpoint{2.727634in}{1.382644in}}%
\pgfpathlineto{\pgfqpoint{2.728041in}{1.379576in}}%
\pgfpathlineto{\pgfqpoint{2.729669in}{1.380225in}}%
\pgfpathlineto{\pgfqpoint{2.730483in}{1.379040in}}%
\pgfpathlineto{\pgfqpoint{2.731296in}{1.381335in}}%
\pgfpathlineto{\pgfqpoint{2.733331in}{1.377819in}}%
\pgfpathlineto{\pgfqpoint{2.734416in}{1.376319in}}%
\pgfpathlineto{\pgfqpoint{2.735094in}{1.379159in}}%
\pgfpathlineto{\pgfqpoint{2.735229in}{1.379350in}}%
\pgfpathlineto{\pgfqpoint{2.735501in}{1.377999in}}%
\pgfpathlineto{\pgfqpoint{2.735908in}{1.377912in}}%
\pgfpathlineto{\pgfqpoint{2.736043in}{1.377667in}}%
\pgfpathlineto{\pgfqpoint{2.736450in}{1.379929in}}%
\pgfpathlineto{\pgfqpoint{2.739027in}{1.384725in}}%
\pgfpathlineto{\pgfqpoint{2.740654in}{1.387437in}}%
\pgfpathlineto{\pgfqpoint{2.740790in}{1.387092in}}%
\pgfpathlineto{\pgfqpoint{2.741197in}{1.386986in}}%
\pgfpathlineto{\pgfqpoint{2.741468in}{1.387988in}}%
\pgfpathlineto{\pgfqpoint{2.743096in}{1.390543in}}%
\pgfpathlineto{\pgfqpoint{2.743231in}{1.389929in}}%
\pgfpathlineto{\pgfqpoint{2.744181in}{1.392653in}}%
\pgfpathlineto{\pgfqpoint{2.747164in}{1.396053in}}%
\pgfpathlineto{\pgfqpoint{2.747300in}{1.395964in}}%
\pgfpathlineto{\pgfqpoint{2.747436in}{1.395248in}}%
\pgfpathlineto{\pgfqpoint{2.748521in}{1.397809in}}%
\pgfpathlineto{\pgfqpoint{2.749470in}{1.396148in}}%
\pgfpathlineto{\pgfqpoint{2.750013in}{1.397709in}}%
\pgfpathlineto{\pgfqpoint{2.750555in}{1.397066in}}%
\pgfpathlineto{\pgfqpoint{2.752183in}{1.398558in}}%
\pgfpathlineto{\pgfqpoint{2.762219in}{1.385150in}}%
\pgfpathlineto{\pgfqpoint{2.762490in}{1.385722in}}%
\pgfpathlineto{\pgfqpoint{2.762761in}{1.385964in}}%
\pgfpathlineto{\pgfqpoint{2.763168in}{1.384648in}}%
\pgfpathlineto{\pgfqpoint{2.764796in}{1.381247in}}%
\pgfpathlineto{\pgfqpoint{2.764931in}{1.382114in}}%
\pgfpathlineto{\pgfqpoint{2.766559in}{1.385169in}}%
\pgfpathlineto{\pgfqpoint{2.765745in}{1.381784in}}%
\pgfpathlineto{\pgfqpoint{2.766694in}{1.383225in}}%
\pgfpathlineto{\pgfqpoint{2.767237in}{1.379934in}}%
\pgfpathlineto{\pgfqpoint{2.768458in}{1.381467in}}%
\pgfpathlineto{\pgfqpoint{2.770763in}{1.384701in}}%
\pgfpathlineto{\pgfqpoint{2.771034in}{1.384335in}}%
\pgfpathlineto{\pgfqpoint{2.782698in}{1.399677in}}%
\pgfpathlineto{\pgfqpoint{2.771306in}{1.383999in}}%
\pgfpathlineto{\pgfqpoint{2.782834in}{1.399151in}}%
\pgfpathlineto{\pgfqpoint{2.784597in}{1.397072in}}%
\pgfpathlineto{\pgfqpoint{2.786496in}{1.398612in}}%
\pgfpathlineto{\pgfqpoint{2.786903in}{1.397207in}}%
\pgfpathlineto{\pgfqpoint{2.787988in}{1.398328in}}%
\pgfpathlineto{\pgfqpoint{2.788530in}{1.398588in}}%
\pgfpathlineto{\pgfqpoint{2.789208in}{1.397647in}}%
\pgfpathlineto{\pgfqpoint{2.790836in}{1.393360in}}%
\pgfpathlineto{\pgfqpoint{2.791243in}{1.395272in}}%
\pgfpathlineto{\pgfqpoint{2.791649in}{1.393849in}}%
\pgfpathlineto{\pgfqpoint{2.794091in}{1.388143in}}%
\pgfpathlineto{\pgfqpoint{2.796396in}{1.387282in}}%
\pgfpathlineto{\pgfqpoint{2.796532in}{1.387977in}}%
\pgfpathlineto{\pgfqpoint{2.796803in}{1.388538in}}%
\pgfpathlineto{\pgfqpoint{2.797346in}{1.386317in}}%
\pgfpathlineto{\pgfqpoint{2.797888in}{1.387412in}}%
\pgfpathlineto{\pgfqpoint{2.798295in}{1.386219in}}%
\pgfpathlineto{\pgfqpoint{2.798431in}{1.384647in}}%
\pgfpathlineto{\pgfqpoint{2.799380in}{1.387205in}}%
\pgfpathlineto{\pgfqpoint{2.799787in}{1.386649in}}%
\pgfpathlineto{\pgfqpoint{2.801279in}{1.384552in}}%
\pgfpathlineto{\pgfqpoint{2.800736in}{1.387422in}}%
\pgfpathlineto{\pgfqpoint{2.801414in}{1.384963in}}%
\pgfpathlineto{\pgfqpoint{2.801686in}{1.385435in}}%
\pgfpathlineto{\pgfqpoint{2.802364in}{1.383775in}}%
\pgfpathlineto{\pgfqpoint{2.802635in}{1.384190in}}%
\pgfpathlineto{\pgfqpoint{2.803720in}{1.384291in}}%
\pgfpathlineto{\pgfqpoint{2.803178in}{1.385505in}}%
\pgfpathlineto{\pgfqpoint{2.803856in}{1.385152in}}%
\pgfpathlineto{\pgfqpoint{2.806704in}{1.389538in}}%
\pgfpathlineto{\pgfqpoint{2.806975in}{1.389283in}}%
\pgfpathlineto{\pgfqpoint{2.807382in}{1.390927in}}%
\pgfpathlineto{\pgfqpoint{2.809688in}{1.396256in}}%
\pgfpathlineto{\pgfqpoint{2.813485in}{1.402926in}}%
\pgfpathlineto{\pgfqpoint{2.815655in}{1.402759in}}%
\pgfpathlineto{\pgfqpoint{2.815791in}{1.403082in}}%
\pgfpathlineto{\pgfqpoint{2.816198in}{1.405155in}}%
\pgfpathlineto{\pgfqpoint{2.817418in}{1.404718in}}%
\pgfpathlineto{\pgfqpoint{2.819588in}{1.401914in}}%
\pgfpathlineto{\pgfqpoint{2.822301in}{1.400840in}}%
\pgfpathlineto{\pgfqpoint{2.822572in}{1.402080in}}%
\pgfpathlineto{\pgfqpoint{2.822979in}{1.403600in}}%
\pgfpathlineto{\pgfqpoint{2.823521in}{1.401535in}}%
\pgfpathlineto{\pgfqpoint{2.824199in}{1.402936in}}%
\pgfpathlineto{\pgfqpoint{2.828268in}{1.393090in}}%
\pgfpathlineto{\pgfqpoint{2.829760in}{1.390057in}}%
\pgfpathlineto{\pgfqpoint{2.830167in}{1.390666in}}%
\pgfpathlineto{\pgfqpoint{2.830438in}{1.391028in}}%
\pgfpathlineto{\pgfqpoint{2.830845in}{1.389467in}}%
\pgfpathlineto{\pgfqpoint{2.832066in}{1.387734in}}%
\pgfpathlineto{\pgfqpoint{2.832608in}{1.388307in}}%
\pgfpathlineto{\pgfqpoint{2.833829in}{1.388043in}}%
\pgfpathlineto{\pgfqpoint{2.833964in}{1.387434in}}%
\pgfpathlineto{\pgfqpoint{2.835321in}{1.386045in}}%
\pgfpathlineto{\pgfqpoint{2.834507in}{1.389075in}}%
\pgfpathlineto{\pgfqpoint{2.835728in}{1.386936in}}%
\pgfpathlineto{\pgfqpoint{2.835863in}{1.387450in}}%
\pgfpathlineto{\pgfqpoint{2.836406in}{1.383768in}}%
\pgfpathlineto{\pgfqpoint{2.836541in}{1.383688in}}%
\pgfpathlineto{\pgfqpoint{2.836677in}{1.384906in}}%
\pgfpathlineto{\pgfqpoint{2.836813in}{1.384807in}}%
\pgfpathlineto{\pgfqpoint{2.838169in}{1.386990in}}%
\pgfpathlineto{\pgfqpoint{2.838440in}{1.386033in}}%
\pgfpathlineto{\pgfqpoint{2.838983in}{1.385040in}}%
\pgfpathlineto{\pgfqpoint{2.846306in}{1.396910in}}%
\pgfpathlineto{\pgfqpoint{2.847934in}{1.399058in}}%
\pgfpathlineto{\pgfqpoint{2.846984in}{1.396712in}}%
\pgfpathlineto{\pgfqpoint{2.848069in}{1.398965in}}%
\pgfpathlineto{\pgfqpoint{2.849561in}{1.398738in}}%
\pgfpathlineto{\pgfqpoint{2.848748in}{1.400124in}}%
\pgfpathlineto{\pgfqpoint{2.849833in}{1.399091in}}%
\pgfpathlineto{\pgfqpoint{2.851460in}{1.399902in}}%
\pgfpathlineto{\pgfqpoint{2.850918in}{1.398717in}}%
\pgfpathlineto{\pgfqpoint{2.851596in}{1.399588in}}%
\pgfpathlineto{\pgfqpoint{2.852681in}{1.400184in}}%
\pgfpathlineto{\pgfqpoint{2.854579in}{1.396836in}}%
\pgfpathlineto{\pgfqpoint{2.855258in}{1.398232in}}%
\pgfpathlineto{\pgfqpoint{2.855936in}{1.396098in}}%
\pgfpathlineto{\pgfqpoint{2.856071in}{1.396612in}}%
\pgfpathlineto{\pgfqpoint{2.857699in}{1.393826in}}%
\pgfpathlineto{\pgfqpoint{2.860004in}{1.387526in}}%
\pgfpathlineto{\pgfqpoint{2.860140in}{1.387858in}}%
\pgfpathlineto{\pgfqpoint{2.860954in}{1.388850in}}%
\pgfpathlineto{\pgfqpoint{2.861225in}{1.387302in}}%
\pgfpathlineto{\pgfqpoint{2.862581in}{1.384849in}}%
\pgfpathlineto{\pgfqpoint{2.861768in}{1.388552in}}%
\pgfpathlineto{\pgfqpoint{2.863124in}{1.385805in}}%
\pgfpathlineto{\pgfqpoint{2.863259in}{1.386313in}}%
\pgfpathlineto{\pgfqpoint{2.863666in}{1.383651in}}%
\pgfpathlineto{\pgfqpoint{2.864344in}{1.385104in}}%
\pgfpathlineto{\pgfqpoint{2.866379in}{1.380971in}}%
\pgfpathlineto{\pgfqpoint{2.866514in}{1.381147in}}%
\pgfpathlineto{\pgfqpoint{2.868142in}{1.384029in}}%
\pgfpathlineto{\pgfqpoint{2.868413in}{1.383553in}}%
\pgfpathlineto{\pgfqpoint{2.870448in}{1.383623in}}%
\pgfpathlineto{\pgfqpoint{2.870583in}{1.382936in}}%
\pgfpathlineto{\pgfqpoint{2.871804in}{1.382768in}}%
\pgfpathlineto{\pgfqpoint{2.871261in}{1.384466in}}%
\pgfpathlineto{\pgfqpoint{2.871939in}{1.383135in}}%
\pgfpathlineto{\pgfqpoint{2.873703in}{1.386389in}}%
\pgfpathlineto{\pgfqpoint{2.874245in}{1.384822in}}%
\pgfpathlineto{\pgfqpoint{2.874923in}{1.387332in}}%
\pgfpathlineto{\pgfqpoint{2.876008in}{1.390046in}}%
\pgfpathlineto{\pgfqpoint{2.876686in}{1.389087in}}%
\pgfpathlineto{\pgfqpoint{2.877364in}{1.390092in}}%
\pgfpathlineto{\pgfqpoint{2.878449in}{1.389097in}}%
\pgfpathlineto{\pgfqpoint{2.880077in}{1.391469in}}%
\pgfpathlineto{\pgfqpoint{2.880348in}{1.390672in}}%
\pgfpathlineto{\pgfqpoint{2.880755in}{1.389782in}}%
\pgfpathlineto{\pgfqpoint{2.881162in}{1.391184in}}%
\pgfpathlineto{\pgfqpoint{2.881569in}{1.391141in}}%
\pgfpathlineto{\pgfqpoint{2.883196in}{1.394116in}}%
\pgfpathlineto{\pgfqpoint{2.883468in}{1.392685in}}%
\pgfpathlineto{\pgfqpoint{2.883739in}{1.391449in}}%
\pgfpathlineto{\pgfqpoint{2.884281in}{1.394222in}}%
\pgfpathlineto{\pgfqpoint{2.884959in}{1.393052in}}%
\pgfpathlineto{\pgfqpoint{2.885502in}{1.395497in}}%
\pgfpathlineto{\pgfqpoint{2.886723in}{1.394815in}}%
\pgfpathlineto{\pgfqpoint{2.887129in}{1.393664in}}%
\pgfpathlineto{\pgfqpoint{2.887808in}{1.396072in}}%
\pgfpathlineto{\pgfqpoint{2.887943in}{1.396661in}}%
\pgfpathlineto{\pgfqpoint{2.888486in}{1.394134in}}%
\pgfpathlineto{\pgfqpoint{2.889028in}{1.395688in}}%
\pgfpathlineto{\pgfqpoint{2.891334in}{1.390252in}}%
\pgfpathlineto{\pgfqpoint{2.891741in}{1.390645in}}%
\pgfpathlineto{\pgfqpoint{2.894182in}{1.387295in}}%
\pgfpathlineto{\pgfqpoint{2.894724in}{1.388238in}}%
\pgfpathlineto{\pgfqpoint{2.895538in}{1.386079in}}%
\pgfpathlineto{\pgfqpoint{2.897979in}{1.378394in}}%
\pgfpathlineto{\pgfqpoint{2.898522in}{1.381470in}}%
\pgfpathlineto{\pgfqpoint{2.899064in}{1.377377in}}%
\pgfpathlineto{\pgfqpoint{2.899607in}{1.379742in}}%
\pgfpathlineto{\pgfqpoint{2.902455in}{1.374484in}}%
\pgfpathlineto{\pgfqpoint{2.903947in}{1.376501in}}%
\pgfpathlineto{\pgfqpoint{2.904354in}{1.375149in}}%
\pgfpathlineto{\pgfqpoint{2.904896in}{1.372259in}}%
\pgfpathlineto{\pgfqpoint{2.905168in}{1.373336in}}%
\pgfusepath{stroke}%
\end{pgfscope}%
\begin{pgfscope}%
\pgfpathrectangle{\pgfqpoint{0.735032in}{0.526079in}}{\pgfqpoint{2.170000in}{1.661000in}} %
\pgfusepath{clip}%
\pgfsetrectcap%
\pgfsetroundjoin%
\pgfsetlinewidth{1.003750pt}%
\definecolor{currentstroke}{rgb}{0.501961,0.000000,0.501961}%
\pgfsetstrokecolor{currentstroke}%
\pgfsetdash{}{0pt}%
\pgfpathmoveto{\pgfqpoint{0.735167in}{0.512191in}}%
\pgfpathlineto{\pgfqpoint{0.736659in}{1.178350in}}%
\pgfpathlineto{\pgfqpoint{0.736931in}{1.180453in}}%
\pgfpathlineto{\pgfqpoint{0.737609in}{1.172144in}}%
\pgfpathlineto{\pgfqpoint{0.739372in}{1.138553in}}%
\pgfpathlineto{\pgfqpoint{0.740186in}{1.153709in}}%
\pgfpathlineto{\pgfqpoint{0.740457in}{1.157317in}}%
\pgfpathlineto{\pgfqpoint{0.740999in}{1.150547in}}%
\pgfpathlineto{\pgfqpoint{0.743305in}{1.029142in}}%
\pgfpathlineto{\pgfqpoint{0.744119in}{1.083305in}}%
\pgfpathlineto{\pgfqpoint{0.744254in}{1.086764in}}%
\pgfpathlineto{\pgfqpoint{0.744526in}{1.066282in}}%
\pgfpathlineto{\pgfqpoint{0.744797in}{1.069132in}}%
\pgfpathlineto{\pgfqpoint{0.745475in}{1.083624in}}%
\pgfpathlineto{\pgfqpoint{0.746289in}{1.046499in}}%
\pgfpathlineto{\pgfqpoint{0.748052in}{1.110230in}}%
\pgfpathlineto{\pgfqpoint{0.747103in}{1.041279in}}%
\pgfpathlineto{\pgfqpoint{0.748188in}{1.105981in}}%
\pgfpathlineto{\pgfqpoint{0.748730in}{1.107522in}}%
\pgfpathlineto{\pgfqpoint{0.749001in}{1.101623in}}%
\pgfpathlineto{\pgfqpoint{0.749815in}{1.082442in}}%
\pgfpathlineto{\pgfqpoint{0.750358in}{1.104263in}}%
\pgfpathlineto{\pgfqpoint{0.750493in}{1.103350in}}%
\pgfpathlineto{\pgfqpoint{0.751171in}{1.119449in}}%
\pgfpathlineto{\pgfqpoint{0.751849in}{1.097008in}}%
\pgfpathlineto{\pgfqpoint{0.752663in}{1.015140in}}%
\pgfpathlineto{\pgfqpoint{0.753477in}{1.078299in}}%
\pgfpathlineto{\pgfqpoint{0.754155in}{1.120757in}}%
\pgfpathlineto{\pgfqpoint{0.754291in}{1.127010in}}%
\pgfpathlineto{\pgfqpoint{0.754969in}{1.099182in}}%
\pgfpathlineto{\pgfqpoint{0.755647in}{0.907199in}}%
\pgfpathlineto{\pgfqpoint{0.757003in}{1.042425in}}%
\pgfpathlineto{\pgfqpoint{0.757274in}{1.064752in}}%
\pgfpathlineto{\pgfqpoint{0.758224in}{1.000460in}}%
\pgfpathlineto{\pgfqpoint{0.759716in}{1.029299in}}%
\pgfpathlineto{\pgfqpoint{0.758631in}{0.980187in}}%
\pgfpathlineto{\pgfqpoint{0.759987in}{1.016572in}}%
\pgfpathlineto{\pgfqpoint{0.760801in}{0.855657in}}%
\pgfpathlineto{\pgfqpoint{0.761072in}{1.009100in}}%
\pgfpathlineto{\pgfqpoint{0.761750in}{1.081593in}}%
\pgfpathlineto{\pgfqpoint{0.762157in}{0.970650in}}%
\pgfpathlineto{\pgfqpoint{0.762428in}{1.009160in}}%
\pgfpathlineto{\pgfqpoint{0.762699in}{0.812517in}}%
\pgfpathlineto{\pgfqpoint{0.763649in}{1.080839in}}%
\pgfpathlineto{\pgfqpoint{0.764327in}{1.067508in}}%
\pgfpathlineto{\pgfqpoint{0.764869in}{1.081689in}}%
\pgfpathlineto{\pgfqpoint{0.765954in}{1.116320in}}%
\pgfpathlineto{\pgfqpoint{0.766497in}{1.099539in}}%
\pgfpathlineto{\pgfqpoint{0.767446in}{0.860528in}}%
\pgfpathlineto{\pgfqpoint{0.768667in}{0.972157in}}%
\pgfpathlineto{\pgfqpoint{0.769888in}{1.098064in}}%
\pgfpathlineto{\pgfqpoint{0.770430in}{1.084455in}}%
\pgfpathlineto{\pgfqpoint{0.771108in}{1.127608in}}%
\pgfpathlineto{\pgfqpoint{0.772058in}{1.102784in}}%
\pgfpathlineto{\pgfqpoint{0.772329in}{1.096510in}}%
\pgfpathlineto{\pgfqpoint{0.772464in}{1.112435in}}%
\pgfpathlineto{\pgfqpoint{0.773685in}{1.175882in}}%
\pgfpathlineto{\pgfqpoint{0.774228in}{1.159866in}}%
\pgfpathlineto{\pgfqpoint{0.774363in}{1.162779in}}%
\pgfpathlineto{\pgfqpoint{0.774906in}{1.156950in}}%
\pgfpathlineto{\pgfqpoint{0.775719in}{1.158988in}}%
\pgfpathlineto{\pgfqpoint{0.776126in}{1.158761in}}%
\pgfpathlineto{\pgfqpoint{0.778703in}{1.131115in}}%
\pgfpathlineto{\pgfqpoint{0.778839in}{1.140202in}}%
\pgfpathlineto{\pgfqpoint{0.778974in}{1.142725in}}%
\pgfpathlineto{\pgfqpoint{0.779517in}{1.126952in}}%
\pgfpathlineto{\pgfqpoint{0.779924in}{1.133385in}}%
\pgfpathlineto{\pgfqpoint{0.781009in}{1.115461in}}%
\pgfpathlineto{\pgfqpoint{0.781823in}{1.124697in}}%
\pgfpathlineto{\pgfqpoint{0.782229in}{1.129660in}}%
\pgfpathlineto{\pgfqpoint{0.782908in}{1.164845in}}%
\pgfpathlineto{\pgfqpoint{0.784806in}{1.204976in}}%
\pgfpathlineto{\pgfqpoint{0.785484in}{1.201537in}}%
\pgfpathlineto{\pgfqpoint{0.787112in}{1.164472in}}%
\pgfpathlineto{\pgfqpoint{0.787654in}{1.182153in}}%
\pgfpathlineto{\pgfqpoint{0.789011in}{1.229698in}}%
\pgfpathlineto{\pgfqpoint{0.789553in}{1.219683in}}%
\pgfpathlineto{\pgfqpoint{0.789824in}{1.221775in}}%
\pgfpathlineto{\pgfqpoint{0.790096in}{1.206994in}}%
\pgfpathlineto{\pgfqpoint{0.791588in}{1.172093in}}%
\pgfpathlineto{\pgfqpoint{0.792537in}{1.194119in}}%
\pgfpathlineto{\pgfqpoint{0.794571in}{1.232828in}}%
\pgfpathlineto{\pgfqpoint{0.794978in}{1.228210in}}%
\pgfpathlineto{\pgfqpoint{0.795385in}{1.218925in}}%
\pgfpathlineto{\pgfqpoint{0.796334in}{1.228696in}}%
\pgfpathlineto{\pgfqpoint{0.796470in}{1.227567in}}%
\pgfpathlineto{\pgfqpoint{0.796877in}{1.233131in}}%
\pgfpathlineto{\pgfqpoint{0.797555in}{1.225311in}}%
\pgfpathlineto{\pgfqpoint{0.799318in}{1.137028in}}%
\pgfpathlineto{\pgfqpoint{0.800810in}{1.164750in}}%
\pgfpathlineto{\pgfqpoint{0.803658in}{1.211598in}}%
\pgfpathlineto{\pgfqpoint{0.804065in}{1.220839in}}%
\pgfpathlineto{\pgfqpoint{0.805150in}{1.213033in}}%
\pgfpathlineto{\pgfqpoint{0.806506in}{1.092648in}}%
\pgfpathlineto{\pgfqpoint{0.807591in}{1.161300in}}%
\pgfpathlineto{\pgfqpoint{0.808676in}{1.196277in}}%
\pgfpathlineto{\pgfqpoint{0.809354in}{1.179936in}}%
\pgfpathlineto{\pgfqpoint{0.812474in}{1.118723in}}%
\pgfpathlineto{\pgfqpoint{0.810439in}{1.181449in}}%
\pgfpathlineto{\pgfqpoint{0.812609in}{1.132545in}}%
\pgfpathlineto{\pgfqpoint{0.814373in}{1.195441in}}%
\pgfpathlineto{\pgfqpoint{0.814644in}{1.194429in}}%
\pgfpathlineto{\pgfqpoint{0.815593in}{1.201956in}}%
\pgfpathlineto{\pgfqpoint{0.815864in}{1.177570in}}%
\pgfpathlineto{\pgfqpoint{0.816000in}{1.188545in}}%
\pgfpathlineto{\pgfqpoint{0.817899in}{1.221202in}}%
\pgfpathlineto{\pgfqpoint{0.818577in}{1.215989in}}%
\pgfpathlineto{\pgfqpoint{0.819526in}{1.195844in}}%
\pgfpathlineto{\pgfqpoint{0.820340in}{1.203598in}}%
\pgfpathlineto{\pgfqpoint{0.822374in}{1.224021in}}%
\pgfpathlineto{\pgfqpoint{0.822646in}{1.223021in}}%
\pgfpathlineto{\pgfqpoint{0.822781in}{1.219122in}}%
\pgfpathlineto{\pgfqpoint{0.823595in}{1.226636in}}%
\pgfpathlineto{\pgfqpoint{0.824002in}{1.225973in}}%
\pgfpathlineto{\pgfqpoint{0.824138in}{1.225585in}}%
\pgfpathlineto{\pgfqpoint{0.825494in}{1.184213in}}%
\pgfpathlineto{\pgfqpoint{0.826172in}{1.205428in}}%
\pgfpathlineto{\pgfqpoint{0.827257in}{1.222219in}}%
\pgfpathlineto{\pgfqpoint{0.827664in}{1.212460in}}%
\pgfpathlineto{\pgfqpoint{0.828206in}{1.193851in}}%
\pgfpathlineto{\pgfqpoint{0.828884in}{1.226507in}}%
\pgfpathlineto{\pgfqpoint{0.829969in}{1.250214in}}%
\pgfpathlineto{\pgfqpoint{0.830648in}{1.239993in}}%
\pgfpathlineto{\pgfqpoint{0.831868in}{1.218063in}}%
\pgfpathlineto{\pgfqpoint{0.832682in}{1.224194in}}%
\pgfpathlineto{\pgfqpoint{0.835123in}{1.253266in}}%
\pgfpathlineto{\pgfqpoint{0.835394in}{1.251206in}}%
\pgfpathlineto{\pgfqpoint{0.837836in}{1.206143in}}%
\pgfpathlineto{\pgfqpoint{0.838649in}{1.227675in}}%
\pgfpathlineto{\pgfqpoint{0.840277in}{1.255342in}}%
\pgfpathlineto{\pgfqpoint{0.840819in}{1.249690in}}%
\pgfpathlineto{\pgfqpoint{0.842583in}{1.217303in}}%
\pgfpathlineto{\pgfqpoint{0.843261in}{1.234565in}}%
\pgfpathlineto{\pgfqpoint{0.845159in}{1.273322in}}%
\pgfpathlineto{\pgfqpoint{0.845295in}{1.274112in}}%
\pgfpathlineto{\pgfqpoint{0.845702in}{1.269336in}}%
\pgfpathlineto{\pgfqpoint{0.848686in}{1.225600in}}%
\pgfpathlineto{\pgfqpoint{0.850584in}{1.232945in}}%
\pgfpathlineto{\pgfqpoint{0.852212in}{1.240259in}}%
\pgfpathlineto{\pgfqpoint{0.852348in}{1.238745in}}%
\pgfpathlineto{\pgfqpoint{0.852754in}{1.239319in}}%
\pgfpathlineto{\pgfqpoint{0.853568in}{1.217854in}}%
\pgfpathlineto{\pgfqpoint{0.855060in}{1.194543in}}%
\pgfpathlineto{\pgfqpoint{0.855196in}{1.197277in}}%
\pgfpathlineto{\pgfqpoint{0.856552in}{1.231027in}}%
\pgfpathlineto{\pgfqpoint{0.857094in}{1.225345in}}%
\pgfpathlineto{\pgfqpoint{0.858315in}{1.186856in}}%
\pgfpathlineto{\pgfqpoint{0.858993in}{1.209016in}}%
\pgfpathlineto{\pgfqpoint{0.860892in}{1.260882in}}%
\pgfpathlineto{\pgfqpoint{0.861570in}{1.252944in}}%
\pgfpathlineto{\pgfqpoint{0.861706in}{1.252967in}}%
\pgfpathlineto{\pgfqpoint{0.862926in}{1.230656in}}%
\pgfpathlineto{\pgfqpoint{0.863469in}{1.239825in}}%
\pgfpathlineto{\pgfqpoint{0.863876in}{1.244026in}}%
\pgfpathlineto{\pgfqpoint{0.864283in}{1.240211in}}%
\pgfpathlineto{\pgfqpoint{0.865368in}{1.205322in}}%
\pgfpathlineto{\pgfqpoint{0.866453in}{1.218517in}}%
\pgfpathlineto{\pgfqpoint{0.868216in}{1.250299in}}%
\pgfpathlineto{\pgfqpoint{0.868351in}{1.250037in}}%
\pgfpathlineto{\pgfqpoint{0.868758in}{1.247158in}}%
\pgfpathlineto{\pgfqpoint{0.870114in}{1.215844in}}%
\pgfpathlineto{\pgfqpoint{0.870657in}{1.233315in}}%
\pgfpathlineto{\pgfqpoint{0.870793in}{1.241213in}}%
\pgfpathlineto{\pgfqpoint{0.871742in}{1.208267in}}%
\pgfpathlineto{\pgfqpoint{0.872827in}{1.046894in}}%
\pgfpathlineto{\pgfqpoint{0.872963in}{0.918754in}}%
\pgfpathlineto{\pgfqpoint{0.874048in}{1.094001in}}%
\pgfpathlineto{\pgfqpoint{0.874319in}{1.038968in}}%
\pgfpathlineto{\pgfqpoint{0.874997in}{0.939873in}}%
\pgfpathlineto{\pgfqpoint{0.875404in}{1.088831in}}%
\pgfpathlineto{\pgfqpoint{0.877167in}{1.244952in}}%
\pgfpathlineto{\pgfqpoint{0.877709in}{1.235477in}}%
\pgfpathlineto{\pgfqpoint{0.878659in}{1.207630in}}%
\pgfpathlineto{\pgfqpoint{0.879473in}{1.169280in}}%
\pgfpathlineto{\pgfqpoint{0.880286in}{1.201824in}}%
\pgfpathlineto{\pgfqpoint{0.880693in}{1.217751in}}%
\pgfpathlineto{\pgfqpoint{0.881643in}{1.207416in}}%
\pgfpathlineto{\pgfqpoint{0.882999in}{1.067002in}}%
\pgfpathlineto{\pgfqpoint{0.884219in}{0.842587in}}%
\pgfpathlineto{\pgfqpoint{0.884491in}{1.068601in}}%
\pgfpathlineto{\pgfqpoint{0.886389in}{1.234182in}}%
\pgfpathlineto{\pgfqpoint{0.886932in}{1.228906in}}%
\pgfpathlineto{\pgfqpoint{0.887339in}{1.215907in}}%
\pgfpathlineto{\pgfqpoint{0.888559in}{1.217364in}}%
\pgfpathlineto{\pgfqpoint{0.889102in}{1.225243in}}%
\pgfpathlineto{\pgfqpoint{0.890323in}{1.221360in}}%
\pgfpathlineto{\pgfqpoint{0.891136in}{1.223592in}}%
\pgfpathlineto{\pgfqpoint{0.891408in}{1.220424in}}%
\pgfpathlineto{\pgfqpoint{0.892493in}{1.169985in}}%
\pgfpathlineto{\pgfqpoint{0.893713in}{1.199126in}}%
\pgfpathlineto{\pgfqpoint{0.894391in}{1.214753in}}%
\pgfpathlineto{\pgfqpoint{0.895341in}{1.205179in}}%
\pgfpathlineto{\pgfqpoint{0.895883in}{1.197075in}}%
\pgfpathlineto{\pgfqpoint{0.896426in}{1.207803in}}%
\pgfpathlineto{\pgfqpoint{0.896968in}{1.201141in}}%
\pgfpathlineto{\pgfqpoint{0.899409in}{1.245840in}}%
\pgfpathlineto{\pgfqpoint{0.900088in}{1.232354in}}%
\pgfpathlineto{\pgfqpoint{0.901037in}{1.212624in}}%
\pgfpathlineto{\pgfqpoint{0.902393in}{1.219197in}}%
\pgfpathlineto{\pgfqpoint{0.902664in}{1.227428in}}%
\pgfpathlineto{\pgfqpoint{0.903749in}{1.209952in}}%
\pgfpathlineto{\pgfqpoint{0.903885in}{1.203173in}}%
\pgfpathlineto{\pgfqpoint{0.904834in}{1.226168in}}%
\pgfpathlineto{\pgfqpoint{0.905241in}{1.230798in}}%
\pgfpathlineto{\pgfqpoint{0.905784in}{1.217301in}}%
\pgfpathlineto{\pgfqpoint{0.905919in}{1.217323in}}%
\pgfpathlineto{\pgfqpoint{0.906462in}{1.228571in}}%
\pgfpathlineto{\pgfqpoint{0.906733in}{1.231162in}}%
\pgfpathlineto{\pgfqpoint{0.907547in}{1.223213in}}%
\pgfpathlineto{\pgfqpoint{0.908768in}{1.196351in}}%
\pgfpathlineto{\pgfqpoint{0.910802in}{1.146687in}}%
\pgfpathlineto{\pgfqpoint{0.910938in}{1.145414in}}%
\pgfpathlineto{\pgfqpoint{0.911073in}{1.152555in}}%
\pgfpathlineto{\pgfqpoint{0.911480in}{1.150825in}}%
\pgfpathlineto{\pgfqpoint{0.912565in}{1.209341in}}%
\pgfpathlineto{\pgfqpoint{0.913514in}{1.254463in}}%
\pgfpathlineto{\pgfqpoint{0.914328in}{1.231800in}}%
\pgfpathlineto{\pgfqpoint{0.915684in}{1.187214in}}%
\pgfpathlineto{\pgfqpoint{0.916227in}{1.202970in}}%
\pgfpathlineto{\pgfqpoint{0.918126in}{1.223660in}}%
\pgfpathlineto{\pgfqpoint{0.918668in}{1.209429in}}%
\pgfpathlineto{\pgfqpoint{0.918804in}{1.217439in}}%
\pgfpathlineto{\pgfqpoint{0.920567in}{0.918550in}}%
\pgfpathlineto{\pgfqpoint{0.920838in}{0.921787in}}%
\pgfpathlineto{\pgfqpoint{0.920974in}{0.802750in}}%
\pgfpathlineto{\pgfqpoint{0.922059in}{1.134241in}}%
\pgfpathlineto{\pgfqpoint{0.922601in}{1.144419in}}%
\pgfpathlineto{\pgfqpoint{0.923279in}{1.124666in}}%
\pgfpathlineto{\pgfqpoint{0.923415in}{1.124058in}}%
\pgfpathlineto{\pgfqpoint{0.926670in}{1.247570in}}%
\pgfpathlineto{\pgfqpoint{0.926941in}{1.237771in}}%
\pgfpathlineto{\pgfqpoint{0.928433in}{1.158397in}}%
\pgfpathlineto{\pgfqpoint{0.929383in}{1.199250in}}%
\pgfpathlineto{\pgfqpoint{0.930468in}{1.240621in}}%
\pgfpathlineto{\pgfqpoint{0.931146in}{1.220754in}}%
\pgfpathlineto{\pgfqpoint{0.934265in}{1.052519in}}%
\pgfpathlineto{\pgfqpoint{0.934401in}{0.842053in}}%
\pgfpathlineto{\pgfqpoint{0.935621in}{1.183034in}}%
\pgfpathlineto{\pgfqpoint{0.935893in}{1.185401in}}%
\pgfpathlineto{\pgfqpoint{0.938469in}{0.989268in}}%
\pgfpathlineto{\pgfqpoint{0.936978in}{1.193493in}}%
\pgfpathlineto{\pgfqpoint{0.938741in}{1.073930in}}%
\pgfpathlineto{\pgfqpoint{0.939826in}{0.991894in}}%
\pgfpathlineto{\pgfqpoint{0.940639in}{1.161413in}}%
\pgfpathlineto{\pgfqpoint{0.941318in}{1.185479in}}%
\pgfpathlineto{\pgfqpoint{0.941724in}{1.147689in}}%
\pgfpathlineto{\pgfqpoint{0.942403in}{1.020894in}}%
\pgfpathlineto{\pgfqpoint{0.943216in}{1.141551in}}%
\pgfpathlineto{\pgfqpoint{0.943352in}{1.143421in}}%
\pgfpathlineto{\pgfqpoint{0.943623in}{1.127164in}}%
\pgfpathlineto{\pgfqpoint{0.944166in}{1.072972in}}%
\pgfpathlineto{\pgfqpoint{0.944573in}{1.030223in}}%
\pgfpathlineto{\pgfqpoint{0.945386in}{1.136278in}}%
\pgfpathlineto{\pgfqpoint{0.946064in}{1.174256in}}%
\pgfpathlineto{\pgfqpoint{0.946878in}{1.148795in}}%
\pgfpathlineto{\pgfqpoint{0.948641in}{0.917709in}}%
\pgfpathlineto{\pgfqpoint{0.948777in}{0.926083in}}%
\pgfpathlineto{\pgfqpoint{0.949998in}{1.163848in}}%
\pgfpathlineto{\pgfqpoint{0.950404in}{1.116293in}}%
\pgfpathlineto{\pgfqpoint{0.950811in}{0.998532in}}%
\pgfpathlineto{\pgfqpoint{0.951489in}{1.209425in}}%
\pgfpathlineto{\pgfqpoint{0.952168in}{1.229912in}}%
\pgfpathlineto{\pgfqpoint{0.953253in}{1.219547in}}%
\pgfpathlineto{\pgfqpoint{0.953795in}{1.201009in}}%
\pgfpathlineto{\pgfqpoint{0.954338in}{1.228451in}}%
\pgfpathlineto{\pgfqpoint{0.955016in}{1.234194in}}%
\pgfpathlineto{\pgfqpoint{0.955558in}{1.227428in}}%
\pgfpathlineto{\pgfqpoint{0.957186in}{1.213695in}}%
\pgfpathlineto{\pgfqpoint{0.957321in}{1.218057in}}%
\pgfpathlineto{\pgfqpoint{0.959356in}{1.248435in}}%
\pgfpathlineto{\pgfqpoint{0.960305in}{1.242735in}}%
\pgfpathlineto{\pgfqpoint{0.961119in}{1.233733in}}%
\pgfpathlineto{\pgfqpoint{0.961933in}{1.241058in}}%
\pgfpathlineto{\pgfqpoint{0.962339in}{1.241671in}}%
\pgfpathlineto{\pgfqpoint{0.964238in}{1.167301in}}%
\pgfpathlineto{\pgfqpoint{0.965323in}{1.070019in}}%
\pgfpathlineto{\pgfqpoint{0.965866in}{1.120289in}}%
\pgfpathlineto{\pgfqpoint{0.968036in}{1.236822in}}%
\pgfpathlineto{\pgfqpoint{0.968171in}{1.235718in}}%
\pgfpathlineto{\pgfqpoint{0.970613in}{1.171961in}}%
\pgfpathlineto{\pgfqpoint{0.971426in}{1.192172in}}%
\pgfpathlineto{\pgfqpoint{0.974003in}{1.278168in}}%
\pgfpathlineto{\pgfqpoint{0.974410in}{1.272601in}}%
\pgfpathlineto{\pgfqpoint{0.974681in}{1.272450in}}%
\pgfpathlineto{\pgfqpoint{0.975631in}{1.275699in}}%
\pgfpathlineto{\pgfqpoint{0.981327in}{1.203076in}}%
\pgfpathlineto{\pgfqpoint{0.983090in}{1.241558in}}%
\pgfpathlineto{\pgfqpoint{0.984582in}{1.280091in}}%
\pgfpathlineto{\pgfqpoint{0.985531in}{1.262028in}}%
\pgfpathlineto{\pgfqpoint{0.986481in}{1.209268in}}%
\pgfpathlineto{\pgfqpoint{0.987566in}{1.248087in}}%
\pgfpathlineto{\pgfqpoint{0.987701in}{1.249590in}}%
\pgfpathlineto{\pgfqpoint{0.988108in}{1.239342in}}%
\pgfpathlineto{\pgfqpoint{0.989058in}{1.220300in}}%
\pgfpathlineto{\pgfqpoint{0.989871in}{1.228997in}}%
\pgfpathlineto{\pgfqpoint{0.990278in}{1.235601in}}%
\pgfpathlineto{\pgfqpoint{0.990956in}{1.224917in}}%
\pgfpathlineto{\pgfqpoint{0.992177in}{1.193043in}}%
\pgfpathlineto{\pgfqpoint{0.992584in}{1.219848in}}%
\pgfpathlineto{\pgfqpoint{0.993262in}{1.253720in}}%
\pgfpathlineto{\pgfqpoint{0.994483in}{1.233205in}}%
\pgfpathlineto{\pgfqpoint{0.996381in}{1.149173in}}%
\pgfpathlineto{\pgfqpoint{0.997195in}{0.951193in}}%
\pgfpathlineto{\pgfqpoint{0.998280in}{0.985426in}}%
\pgfpathlineto{\pgfqpoint{0.999908in}{1.141802in}}%
\pgfpathlineto{\pgfqpoint{1.000043in}{1.136546in}}%
\pgfpathlineto{\pgfqpoint{1.000179in}{1.140816in}}%
\pgfpathlineto{\pgfqpoint{1.000857in}{1.121882in}}%
\pgfpathlineto{\pgfqpoint{1.001399in}{0.796750in}}%
\pgfpathlineto{\pgfqpoint{1.002484in}{1.078010in}}%
\pgfpathlineto{\pgfqpoint{1.002891in}{1.083592in}}%
\pgfpathlineto{\pgfqpoint{1.002756in}{1.064638in}}%
\pgfpathlineto{\pgfqpoint{1.003027in}{1.075309in}}%
\pgfpathlineto{\pgfqpoint{1.003976in}{0.841997in}}%
\pgfpathlineto{\pgfqpoint{1.004654in}{1.031675in}}%
\pgfpathlineto{\pgfqpoint{1.005604in}{1.140649in}}%
\pgfpathlineto{\pgfqpoint{1.006418in}{1.096272in}}%
\pgfpathlineto{\pgfqpoint{1.007367in}{0.953936in}}%
\pgfpathlineto{\pgfqpoint{1.007503in}{1.126935in}}%
\pgfpathlineto{\pgfqpoint{1.007909in}{1.082977in}}%
\pgfpathlineto{\pgfqpoint{1.008045in}{1.081934in}}%
\pgfpathlineto{\pgfqpoint{1.008316in}{0.864471in}}%
\pgfpathlineto{\pgfqpoint{1.009266in}{1.093568in}}%
\pgfpathlineto{\pgfqpoint{1.009401in}{1.075271in}}%
\pgfpathlineto{\pgfqpoint{1.010215in}{1.184350in}}%
\pgfpathlineto{\pgfqpoint{1.011029in}{1.150306in}}%
\pgfpathlineto{\pgfqpoint{1.011843in}{0.994351in}}%
\pgfpathlineto{\pgfqpoint{1.012385in}{1.143391in}}%
\pgfpathlineto{\pgfqpoint{1.013470in}{1.177907in}}%
\pgfpathlineto{\pgfqpoint{1.013741in}{1.146761in}}%
\pgfpathlineto{\pgfqpoint{1.014013in}{0.905433in}}%
\pgfpathlineto{\pgfqpoint{1.014962in}{1.167844in}}%
\pgfpathlineto{\pgfqpoint{1.015233in}{1.141458in}}%
\pgfpathlineto{\pgfqpoint{1.017132in}{1.011031in}}%
\pgfpathlineto{\pgfqpoint{1.017403in}{1.039803in}}%
\pgfpathlineto{\pgfqpoint{1.017674in}{1.078500in}}%
\pgfpathlineto{\pgfqpoint{1.017946in}{1.036849in}}%
\pgfpathlineto{\pgfqpoint{1.019031in}{0.721670in}}%
\pgfpathlineto{\pgfqpoint{1.018624in}{1.085287in}}%
\pgfpathlineto{\pgfqpoint{1.019438in}{1.015813in}}%
\pgfpathlineto{\pgfqpoint{1.019709in}{0.949872in}}%
\pgfpathlineto{\pgfqpoint{1.019980in}{1.133706in}}%
\pgfpathlineto{\pgfqpoint{1.020116in}{1.122032in}}%
\pgfpathlineto{\pgfqpoint{1.020251in}{1.143806in}}%
\pgfpathlineto{\pgfqpoint{1.021065in}{1.072850in}}%
\pgfpathlineto{\pgfqpoint{1.021201in}{1.102492in}}%
\pgfpathlineto{\pgfqpoint{1.021608in}{0.958165in}}%
\pgfpathlineto{\pgfqpoint{1.022150in}{1.109085in}}%
\pgfpathlineto{\pgfqpoint{1.022693in}{1.097891in}}%
\pgfpathlineto{\pgfqpoint{1.024320in}{0.983259in}}%
\pgfpathlineto{\pgfqpoint{1.023235in}{1.102163in}}%
\pgfpathlineto{\pgfqpoint{1.024727in}{1.030300in}}%
\pgfpathlineto{\pgfqpoint{1.026354in}{1.138522in}}%
\pgfpathlineto{\pgfqpoint{1.026490in}{1.137628in}}%
\pgfpathlineto{\pgfqpoint{1.026626in}{1.142069in}}%
\pgfpathlineto{\pgfqpoint{1.027033in}{1.160686in}}%
\pgfpathlineto{\pgfqpoint{1.027982in}{1.134320in}}%
\pgfpathlineto{\pgfqpoint{1.028118in}{1.137269in}}%
\pgfpathlineto{\pgfqpoint{1.028660in}{1.118414in}}%
\pgfpathlineto{\pgfqpoint{1.028796in}{1.118834in}}%
\pgfpathlineto{\pgfqpoint{1.028931in}{1.116918in}}%
\pgfpathlineto{\pgfqpoint{1.029067in}{1.128329in}}%
\pgfpathlineto{\pgfqpoint{1.029474in}{1.127916in}}%
\pgfpathlineto{\pgfqpoint{1.030830in}{0.796447in}}%
\pgfpathlineto{\pgfqpoint{1.031101in}{1.042885in}}%
\pgfpathlineto{\pgfqpoint{1.032593in}{1.204156in}}%
\pgfpathlineto{\pgfqpoint{1.033000in}{1.200220in}}%
\pgfpathlineto{\pgfqpoint{1.034628in}{1.174220in}}%
\pgfpathlineto{\pgfqpoint{1.034899in}{1.182026in}}%
\pgfpathlineto{\pgfqpoint{1.036255in}{1.209664in}}%
\pgfpathlineto{\pgfqpoint{1.036391in}{1.205966in}}%
\pgfpathlineto{\pgfqpoint{1.037340in}{1.038807in}}%
\pgfpathlineto{\pgfqpoint{1.038154in}{1.140010in}}%
\pgfpathlineto{\pgfqpoint{1.038289in}{1.151032in}}%
\pgfpathlineto{\pgfqpoint{1.038968in}{1.082917in}}%
\pgfpathlineto{\pgfqpoint{1.039374in}{0.625195in}}%
\pgfpathlineto{\pgfqpoint{1.040324in}{1.096327in}}%
\pgfpathlineto{\pgfqpoint{1.040595in}{1.000713in}}%
\pgfpathlineto{\pgfqpoint{1.041680in}{1.165977in}}%
\pgfpathlineto{\pgfqpoint{1.042087in}{1.118027in}}%
\pgfpathlineto{\pgfqpoint{1.042223in}{0.903373in}}%
\pgfpathlineto{\pgfqpoint{1.043579in}{1.073860in}}%
\pgfpathlineto{\pgfqpoint{1.044393in}{1.031610in}}%
\pgfpathlineto{\pgfqpoint{1.043986in}{1.074560in}}%
\pgfpathlineto{\pgfqpoint{1.044935in}{1.056378in}}%
\pgfpathlineto{\pgfqpoint{1.045206in}{0.943092in}}%
\pgfpathlineto{\pgfqpoint{1.046156in}{1.108706in}}%
\pgfpathlineto{\pgfqpoint{1.046427in}{1.068929in}}%
\pgfpathlineto{\pgfqpoint{1.046563in}{1.077951in}}%
\pgfpathlineto{\pgfqpoint{1.046698in}{1.007771in}}%
\pgfpathlineto{\pgfqpoint{1.046969in}{0.799399in}}%
\pgfpathlineto{\pgfqpoint{1.047512in}{1.097696in}}%
\pgfpathlineto{\pgfqpoint{1.048054in}{1.053560in}}%
\pgfpathlineto{\pgfqpoint{1.049953in}{1.171222in}}%
\pgfpathlineto{\pgfqpoint{1.050089in}{1.168991in}}%
\pgfpathlineto{\pgfqpoint{1.050903in}{1.054303in}}%
\pgfpathlineto{\pgfqpoint{1.051174in}{0.944769in}}%
\pgfpathlineto{\pgfqpoint{1.052123in}{1.104815in}}%
\pgfpathlineto{\pgfqpoint{1.052394in}{1.044924in}}%
\pgfpathlineto{\pgfqpoint{1.053073in}{0.938131in}}%
\pgfpathlineto{\pgfqpoint{1.052666in}{1.071174in}}%
\pgfpathlineto{\pgfqpoint{1.053344in}{0.940934in}}%
\pgfpathlineto{\pgfqpoint{1.054836in}{1.131867in}}%
\pgfpathlineto{\pgfqpoint{1.054971in}{1.129015in}}%
\pgfpathlineto{\pgfqpoint{1.056056in}{1.170662in}}%
\pgfpathlineto{\pgfqpoint{1.056734in}{1.167777in}}%
\pgfpathlineto{\pgfqpoint{1.057277in}{0.747934in}}%
\pgfpathlineto{\pgfqpoint{1.058633in}{0.995197in}}%
\pgfpathlineto{\pgfqpoint{1.060396in}{1.255877in}}%
\pgfpathlineto{\pgfqpoint{1.060668in}{1.252336in}}%
\pgfpathlineto{\pgfqpoint{1.061481in}{1.219796in}}%
\pgfpathlineto{\pgfqpoint{1.062973in}{1.177429in}}%
\pgfpathlineto{\pgfqpoint{1.063244in}{1.180974in}}%
\pgfpathlineto{\pgfqpoint{1.065279in}{1.271202in}}%
\pgfpathlineto{\pgfqpoint{1.065686in}{1.274946in}}%
\pgfpathlineto{\pgfqpoint{1.066228in}{1.263144in}}%
\pgfpathlineto{\pgfqpoint{1.067720in}{1.204776in}}%
\pgfpathlineto{\pgfqpoint{1.068805in}{1.232356in}}%
\pgfpathlineto{\pgfqpoint{1.069212in}{1.240334in}}%
\pgfpathlineto{\pgfqpoint{1.069890in}{1.221315in}}%
\pgfpathlineto{\pgfqpoint{1.071111in}{1.158964in}}%
\pgfpathlineto{\pgfqpoint{1.071518in}{1.198465in}}%
\pgfpathlineto{\pgfqpoint{1.072196in}{1.216345in}}%
\pgfpathlineto{\pgfqpoint{1.073145in}{1.204895in}}%
\pgfpathlineto{\pgfqpoint{1.073959in}{1.189293in}}%
\pgfpathlineto{\pgfqpoint{1.074637in}{1.150983in}}%
\pgfpathlineto{\pgfqpoint{1.075179in}{1.204914in}}%
\pgfpathlineto{\pgfqpoint{1.075315in}{1.202062in}}%
\pgfpathlineto{\pgfqpoint{1.076129in}{1.218821in}}%
\pgfpathlineto{\pgfqpoint{1.076671in}{1.204174in}}%
\pgfpathlineto{\pgfqpoint{1.077485in}{1.144603in}}%
\pgfpathlineto{\pgfqpoint{1.078434in}{1.187061in}}%
\pgfpathlineto{\pgfqpoint{1.080062in}{1.219693in}}%
\pgfpathlineto{\pgfqpoint{1.079791in}{1.186920in}}%
\pgfpathlineto{\pgfqpoint{1.080333in}{1.215305in}}%
\pgfpathlineto{\pgfqpoint{1.081283in}{1.090639in}}%
\pgfpathlineto{\pgfqpoint{1.082096in}{1.172349in}}%
\pgfpathlineto{\pgfqpoint{1.083453in}{1.222002in}}%
\pgfpathlineto{\pgfqpoint{1.084538in}{1.202252in}}%
\pgfpathlineto{\pgfqpoint{1.086572in}{1.081224in}}%
\pgfpathlineto{\pgfqpoint{1.086843in}{1.090477in}}%
\pgfpathlineto{\pgfqpoint{1.088064in}{0.870875in}}%
\pgfpathlineto{\pgfqpoint{1.088335in}{1.056856in}}%
\pgfpathlineto{\pgfqpoint{1.088878in}{1.097240in}}%
\pgfpathlineto{\pgfqpoint{1.089284in}{0.972015in}}%
\pgfpathlineto{\pgfqpoint{1.090098in}{0.831558in}}%
\pgfpathlineto{\pgfqpoint{1.090369in}{1.071832in}}%
\pgfpathlineto{\pgfqpoint{1.090776in}{1.011091in}}%
\pgfpathlineto{\pgfqpoint{1.091319in}{0.965859in}}%
\pgfpathlineto{\pgfqpoint{1.092404in}{1.094646in}}%
\pgfpathlineto{\pgfqpoint{1.092675in}{0.997334in}}%
\pgfpathlineto{\pgfqpoint{1.093760in}{1.189700in}}%
\pgfpathlineto{\pgfqpoint{1.095116in}{1.223358in}}%
\pgfpathlineto{\pgfqpoint{1.095523in}{1.220888in}}%
\pgfpathlineto{\pgfqpoint{1.095930in}{1.195511in}}%
\pgfpathlineto{\pgfqpoint{1.098643in}{0.935655in}}%
\pgfpathlineto{\pgfqpoint{1.098914in}{0.999229in}}%
\pgfpathlineto{\pgfqpoint{1.099728in}{0.954523in}}%
\pgfpathlineto{\pgfqpoint{1.101219in}{1.223940in}}%
\pgfpathlineto{\pgfqpoint{1.101626in}{1.235652in}}%
\pgfpathlineto{\pgfqpoint{1.102033in}{1.214439in}}%
\pgfpathlineto{\pgfqpoint{1.102440in}{1.214648in}}%
\pgfpathlineto{\pgfqpoint{1.103254in}{1.183719in}}%
\pgfpathlineto{\pgfqpoint{1.104203in}{1.190050in}}%
\pgfpathlineto{\pgfqpoint{1.104474in}{1.201665in}}%
\pgfpathlineto{\pgfqpoint{1.105017in}{1.181821in}}%
\pgfpathlineto{\pgfqpoint{1.105695in}{0.979233in}}%
\pgfpathlineto{\pgfqpoint{1.106509in}{1.186718in}}%
\pgfpathlineto{\pgfqpoint{1.107729in}{1.234332in}}%
\pgfpathlineto{\pgfqpoint{1.108408in}{1.220559in}}%
\pgfpathlineto{\pgfqpoint{1.109357in}{1.046059in}}%
\pgfpathlineto{\pgfqpoint{1.109493in}{0.911987in}}%
\pgfpathlineto{\pgfqpoint{1.110035in}{1.145370in}}%
\pgfpathlineto{\pgfqpoint{1.110713in}{1.124446in}}%
\pgfpathlineto{\pgfqpoint{1.111120in}{1.027577in}}%
\pgfpathlineto{\pgfqpoint{1.111256in}{0.620527in}}%
\pgfpathlineto{\pgfqpoint{1.111934in}{1.158077in}}%
\pgfpathlineto{\pgfqpoint{1.112612in}{1.064213in}}%
\pgfpathlineto{\pgfqpoint{1.114375in}{1.243846in}}%
\pgfpathlineto{\pgfqpoint{1.115189in}{1.167399in}}%
\pgfpathlineto{\pgfqpoint{1.115731in}{0.901110in}}%
\pgfpathlineto{\pgfqpoint{1.117359in}{1.057993in}}%
\pgfpathlineto{\pgfqpoint{1.118173in}{1.214498in}}%
\pgfpathlineto{\pgfqpoint{1.118986in}{1.162841in}}%
\pgfpathlineto{\pgfqpoint{1.119664in}{0.976122in}}%
\pgfpathlineto{\pgfqpoint{1.120749in}{1.119417in}}%
\pgfpathlineto{\pgfqpoint{1.121970in}{0.818645in}}%
\pgfpathlineto{\pgfqpoint{1.122648in}{1.017940in}}%
\pgfpathlineto{\pgfqpoint{1.123598in}{1.126488in}}%
\pgfpathlineto{\pgfqpoint{1.124276in}{1.101320in}}%
\pgfpathlineto{\pgfqpoint{1.124547in}{1.097339in}}%
\pgfpathlineto{\pgfqpoint{1.124683in}{1.110094in}}%
\pgfpathlineto{\pgfqpoint{1.125225in}{1.028206in}}%
\pgfpathlineto{\pgfqpoint{1.125361in}{1.062822in}}%
\pgfpathlineto{\pgfqpoint{1.125903in}{0.954879in}}%
\pgfpathlineto{\pgfqpoint{1.126717in}{1.115019in}}%
\pgfpathlineto{\pgfqpoint{1.127666in}{1.177380in}}%
\pgfpathlineto{\pgfqpoint{1.128616in}{1.174654in}}%
\pgfpathlineto{\pgfqpoint{1.130243in}{1.040273in}}%
\pgfpathlineto{\pgfqpoint{1.130379in}{0.789159in}}%
\pgfpathlineto{\pgfqpoint{1.131599in}{1.132206in}}%
\pgfpathlineto{\pgfqpoint{1.131735in}{1.141004in}}%
\pgfpathlineto{\pgfqpoint{1.132549in}{1.109101in}}%
\pgfpathlineto{\pgfqpoint{1.133091in}{0.981502in}}%
\pgfpathlineto{\pgfqpoint{1.133227in}{0.850938in}}%
\pgfpathlineto{\pgfqpoint{1.134448in}{1.109829in}}%
\pgfpathlineto{\pgfqpoint{1.134583in}{1.109649in}}%
\pgfpathlineto{\pgfqpoint{1.134719in}{1.110485in}}%
\pgfpathlineto{\pgfqpoint{1.135397in}{1.170307in}}%
\pgfpathlineto{\pgfqpoint{1.136346in}{1.143007in}}%
\pgfpathlineto{\pgfqpoint{1.137567in}{1.144963in}}%
\pgfpathlineto{\pgfqpoint{1.138516in}{1.012805in}}%
\pgfpathlineto{\pgfqpoint{1.138652in}{0.815945in}}%
\pgfpathlineto{\pgfqpoint{1.139873in}{1.178648in}}%
\pgfpathlineto{\pgfqpoint{1.140551in}{1.209391in}}%
\pgfpathlineto{\pgfqpoint{1.141364in}{1.176061in}}%
\pgfpathlineto{\pgfqpoint{1.143806in}{0.903105in}}%
\pgfpathlineto{\pgfqpoint{1.144348in}{1.046200in}}%
\pgfpathlineto{\pgfqpoint{1.146111in}{1.147079in}}%
\pgfpathlineto{\pgfqpoint{1.145162in}{0.942854in}}%
\pgfpathlineto{\pgfqpoint{1.146383in}{1.138313in}}%
\pgfpathlineto{\pgfqpoint{1.147874in}{1.001942in}}%
\pgfpathlineto{\pgfqpoint{1.148959in}{0.940393in}}%
\pgfpathlineto{\pgfqpoint{1.148553in}{1.055521in}}%
\pgfpathlineto{\pgfqpoint{1.149231in}{1.024358in}}%
\pgfpathlineto{\pgfqpoint{1.150587in}{1.010089in}}%
\pgfpathlineto{\pgfqpoint{1.151265in}{1.220290in}}%
\pgfpathlineto{\pgfqpoint{1.153164in}{1.283168in}}%
\pgfpathlineto{\pgfqpoint{1.153299in}{1.280801in}}%
\pgfpathlineto{\pgfqpoint{1.154656in}{1.255299in}}%
\pgfpathlineto{\pgfqpoint{1.155605in}{1.262726in}}%
\pgfpathlineto{\pgfqpoint{1.156148in}{1.265258in}}%
\pgfpathlineto{\pgfqpoint{1.156690in}{1.260262in}}%
\pgfpathlineto{\pgfqpoint{1.156826in}{1.260264in}}%
\pgfpathlineto{\pgfqpoint{1.156961in}{1.259927in}}%
\pgfpathlineto{\pgfqpoint{1.157504in}{1.262305in}}%
\pgfpathlineto{\pgfqpoint{1.157911in}{1.263240in}}%
\pgfpathlineto{\pgfqpoint{1.157775in}{1.261514in}}%
\pgfpathlineto{\pgfqpoint{1.158046in}{1.262336in}}%
\pgfpathlineto{\pgfqpoint{1.159131in}{1.247927in}}%
\pgfpathlineto{\pgfqpoint{1.159674in}{1.254006in}}%
\pgfpathlineto{\pgfqpoint{1.161301in}{1.278622in}}%
\pgfpathlineto{\pgfqpoint{1.161979in}{1.265414in}}%
\pgfpathlineto{\pgfqpoint{1.162658in}{1.238674in}}%
\pgfpathlineto{\pgfqpoint{1.163607in}{1.253381in}}%
\pgfpathlineto{\pgfqpoint{1.164556in}{1.264262in}}%
\pgfpathlineto{\pgfqpoint{1.165506in}{1.261003in}}%
\pgfpathlineto{\pgfqpoint{1.165777in}{1.258253in}}%
\pgfpathlineto{\pgfqpoint{1.166184in}{1.262838in}}%
\pgfpathlineto{\pgfqpoint{1.166998in}{1.260704in}}%
\pgfpathlineto{\pgfqpoint{1.167540in}{1.262083in}}%
\pgfpathlineto{\pgfqpoint{1.167947in}{1.259323in}}%
\pgfpathlineto{\pgfqpoint{1.169032in}{1.241798in}}%
\pgfpathlineto{\pgfqpoint{1.169439in}{1.253488in}}%
\pgfpathlineto{\pgfqpoint{1.171066in}{1.269424in}}%
\pgfpathlineto{\pgfqpoint{1.171338in}{1.265719in}}%
\pgfpathlineto{\pgfqpoint{1.171744in}{1.270390in}}%
\pgfpathlineto{\pgfqpoint{1.172287in}{1.263437in}}%
\pgfpathlineto{\pgfqpoint{1.172558in}{1.264442in}}%
\pgfpathlineto{\pgfqpoint{1.172694in}{1.261595in}}%
\pgfpathlineto{\pgfqpoint{1.173508in}{1.273337in}}%
\pgfpathlineto{\pgfqpoint{1.175271in}{1.281108in}}%
\pgfpathlineto{\pgfqpoint{1.175678in}{1.267528in}}%
\pgfpathlineto{\pgfqpoint{1.176763in}{1.278314in}}%
\pgfpathlineto{\pgfqpoint{1.177712in}{1.292194in}}%
\pgfpathlineto{\pgfqpoint{1.178526in}{1.285657in}}%
\pgfpathlineto{\pgfqpoint{1.180424in}{1.254231in}}%
\pgfpathlineto{\pgfqpoint{1.181509in}{1.260665in}}%
\pgfpathlineto{\pgfqpoint{1.183408in}{1.286271in}}%
\pgfpathlineto{\pgfqpoint{1.184222in}{1.283254in}}%
\pgfpathlineto{\pgfqpoint{1.184900in}{1.262228in}}%
\pgfpathlineto{\pgfqpoint{1.185849in}{1.272957in}}%
\pgfpathlineto{\pgfqpoint{1.187748in}{1.300492in}}%
\pgfpathlineto{\pgfqpoint{1.188019in}{1.298679in}}%
\pgfpathlineto{\pgfqpoint{1.189511in}{1.276580in}}%
\pgfpathlineto{\pgfqpoint{1.190732in}{1.279800in}}%
\pgfpathlineto{\pgfqpoint{1.190868in}{1.282421in}}%
\pgfpathlineto{\pgfqpoint{1.191681in}{1.273349in}}%
\pgfpathlineto{\pgfqpoint{1.193309in}{1.256303in}}%
\pgfpathlineto{\pgfqpoint{1.193851in}{1.258790in}}%
\pgfpathlineto{\pgfqpoint{1.194123in}{1.263940in}}%
\pgfpathlineto{\pgfqpoint{1.194665in}{1.246995in}}%
\pgfpathlineto{\pgfqpoint{1.195208in}{1.255008in}}%
\pgfpathlineto{\pgfqpoint{1.195479in}{1.250694in}}%
\pgfpathlineto{\pgfqpoint{1.196293in}{1.263384in}}%
\pgfpathlineto{\pgfqpoint{1.196428in}{1.262893in}}%
\pgfpathlineto{\pgfqpoint{1.197649in}{1.274532in}}%
\pgfpathlineto{\pgfqpoint{1.198327in}{1.270520in}}%
\pgfpathlineto{\pgfqpoint{1.199276in}{1.260465in}}%
\pgfpathlineto{\pgfqpoint{1.200090in}{1.265260in}}%
\pgfpathlineto{\pgfqpoint{1.200226in}{1.268001in}}%
\pgfpathlineto{\pgfqpoint{1.201175in}{1.259008in}}%
\pgfpathlineto{\pgfqpoint{1.203074in}{1.244004in}}%
\pgfpathlineto{\pgfqpoint{1.203616in}{1.248634in}}%
\pgfpathlineto{\pgfqpoint{1.207007in}{1.291112in}}%
\pgfpathlineto{\pgfqpoint{1.207821in}{1.282798in}}%
\pgfpathlineto{\pgfqpoint{1.208499in}{1.292075in}}%
\pgfpathlineto{\pgfqpoint{1.210262in}{1.299642in}}%
\pgfpathlineto{\pgfqpoint{1.210398in}{1.299450in}}%
\pgfpathlineto{\pgfqpoint{1.212974in}{1.259843in}}%
\pgfpathlineto{\pgfqpoint{1.210940in}{1.300989in}}%
\pgfpathlineto{\pgfqpoint{1.213381in}{1.268683in}}%
\pgfpathlineto{\pgfqpoint{1.213924in}{1.267958in}}%
\pgfpathlineto{\pgfqpoint{1.216229in}{1.293890in}}%
\pgfpathlineto{\pgfqpoint{1.217993in}{1.281138in}}%
\pgfpathlineto{\pgfqpoint{1.218264in}{1.284678in}}%
\pgfpathlineto{\pgfqpoint{1.220163in}{1.301668in}}%
\pgfpathlineto{\pgfqpoint{1.220298in}{1.301237in}}%
\pgfpathlineto{\pgfqpoint{1.223689in}{1.267563in}}%
\pgfpathlineto{\pgfqpoint{1.225045in}{1.283870in}}%
\pgfpathlineto{\pgfqpoint{1.223960in}{1.266074in}}%
\pgfpathlineto{\pgfqpoint{1.225859in}{1.283210in}}%
\pgfpathlineto{\pgfqpoint{1.227079in}{1.277740in}}%
\pgfpathlineto{\pgfqpoint{1.227351in}{1.282547in}}%
\pgfpathlineto{\pgfqpoint{1.229792in}{1.293172in}}%
\pgfpathlineto{\pgfqpoint{1.230063in}{1.293992in}}%
\pgfpathlineto{\pgfqpoint{1.230741in}{1.292247in}}%
\pgfpathlineto{\pgfqpoint{1.232369in}{1.282550in}}%
\pgfpathlineto{\pgfqpoint{1.233183in}{1.285474in}}%
\pgfpathlineto{\pgfqpoint{1.235353in}{1.308866in}}%
\pgfpathlineto{\pgfqpoint{1.236844in}{1.320438in}}%
\pgfpathlineto{\pgfqpoint{1.237523in}{1.312742in}}%
\pgfpathlineto{\pgfqpoint{1.238065in}{1.308002in}}%
\pgfpathlineto{\pgfqpoint{1.238743in}{1.314773in}}%
\pgfpathlineto{\pgfqpoint{1.239286in}{1.320408in}}%
\pgfpathlineto{\pgfqpoint{1.240099in}{1.312702in}}%
\pgfpathlineto{\pgfqpoint{1.240913in}{1.306237in}}%
\pgfpathlineto{\pgfqpoint{1.241591in}{1.311484in}}%
\pgfpathlineto{\pgfqpoint{1.244575in}{1.329472in}}%
\pgfpathlineto{\pgfqpoint{1.245118in}{1.322047in}}%
\pgfpathlineto{\pgfqpoint{1.246203in}{1.325144in}}%
\pgfpathlineto{\pgfqpoint{1.247830in}{1.335189in}}%
\pgfpathlineto{\pgfqpoint{1.248101in}{1.332799in}}%
\pgfpathlineto{\pgfqpoint{1.248644in}{1.328699in}}%
\pgfpathlineto{\pgfqpoint{1.249458in}{1.333046in}}%
\pgfpathlineto{\pgfqpoint{1.250271in}{1.342831in}}%
\pgfpathlineto{\pgfqpoint{1.251221in}{1.338668in}}%
\pgfpathlineto{\pgfqpoint{1.252577in}{1.326538in}}%
\pgfpathlineto{\pgfqpoint{1.253391in}{1.331094in}}%
\pgfpathlineto{\pgfqpoint{1.253662in}{1.330182in}}%
\pgfpathlineto{\pgfqpoint{1.253933in}{1.331903in}}%
\pgfpathlineto{\pgfqpoint{1.255832in}{1.338853in}}%
\pgfpathlineto{\pgfqpoint{1.256103in}{1.340648in}}%
\pgfpathlineto{\pgfqpoint{1.257866in}{1.332491in}}%
\pgfpathlineto{\pgfqpoint{1.259358in}{1.321136in}}%
\pgfpathlineto{\pgfqpoint{1.260036in}{1.324178in}}%
\pgfpathlineto{\pgfqpoint{1.261664in}{1.335905in}}%
\pgfpathlineto{\pgfqpoint{1.262342in}{1.330915in}}%
\pgfpathlineto{\pgfqpoint{1.264783in}{1.302969in}}%
\pgfpathlineto{\pgfqpoint{1.265054in}{1.305855in}}%
\pgfpathlineto{\pgfqpoint{1.266139in}{1.319171in}}%
\pgfpathlineto{\pgfqpoint{1.266818in}{1.311551in}}%
\pgfpathlineto{\pgfqpoint{1.267089in}{1.301514in}}%
\pgfpathlineto{\pgfqpoint{1.268174in}{1.310921in}}%
\pgfpathlineto{\pgfqpoint{1.269394in}{1.317155in}}%
\pgfpathlineto{\pgfqpoint{1.269937in}{1.316659in}}%
\pgfpathlineto{\pgfqpoint{1.274141in}{1.302308in}}%
\pgfpathlineto{\pgfqpoint{1.275226in}{1.306120in}}%
\pgfpathlineto{\pgfqpoint{1.275362in}{1.306977in}}%
\pgfpathlineto{\pgfqpoint{1.276040in}{1.303092in}}%
\pgfpathlineto{\pgfqpoint{1.276718in}{1.288874in}}%
\pgfpathlineto{\pgfqpoint{1.277396in}{1.304756in}}%
\pgfpathlineto{\pgfqpoint{1.277532in}{1.304808in}}%
\pgfpathlineto{\pgfqpoint{1.277668in}{1.303497in}}%
\pgfpathlineto{\pgfqpoint{1.277803in}{1.303335in}}%
\pgfpathlineto{\pgfqpoint{1.277939in}{1.302967in}}%
\pgfpathlineto{\pgfqpoint{1.278074in}{1.304945in}}%
\pgfpathlineto{\pgfqpoint{1.279159in}{1.304710in}}%
\pgfpathlineto{\pgfqpoint{1.280923in}{1.324220in}}%
\pgfpathlineto{\pgfqpoint{1.281465in}{1.324372in}}%
\pgfpathlineto{\pgfqpoint{1.282550in}{1.323725in}}%
\pgfpathlineto{\pgfqpoint{1.283364in}{1.331176in}}%
\pgfpathlineto{\pgfqpoint{1.285534in}{1.322865in}}%
\pgfpathlineto{\pgfqpoint{1.284313in}{1.333911in}}%
\pgfpathlineto{\pgfqpoint{1.286076in}{1.325216in}}%
\pgfpathlineto{\pgfqpoint{1.287975in}{1.331489in}}%
\pgfpathlineto{\pgfqpoint{1.286890in}{1.324403in}}%
\pgfpathlineto{\pgfqpoint{1.288246in}{1.329049in}}%
\pgfpathlineto{\pgfqpoint{1.288382in}{1.326197in}}%
\pgfpathlineto{\pgfqpoint{1.288789in}{1.334052in}}%
\pgfpathlineto{\pgfqpoint{1.289603in}{1.331455in}}%
\pgfpathlineto{\pgfqpoint{1.289738in}{1.333957in}}%
\pgfpathlineto{\pgfqpoint{1.290688in}{1.324801in}}%
\pgfpathlineto{\pgfqpoint{1.291094in}{1.320027in}}%
\pgfpathlineto{\pgfqpoint{1.292179in}{1.325249in}}%
\pgfpathlineto{\pgfqpoint{1.292451in}{1.326793in}}%
\pgfpathlineto{\pgfqpoint{1.293536in}{1.333964in}}%
\pgfpathlineto{\pgfqpoint{1.294078in}{1.328449in}}%
\pgfpathlineto{\pgfqpoint{1.294214in}{1.329390in}}%
\pgfpathlineto{\pgfqpoint{1.294621in}{1.321436in}}%
\pgfpathlineto{\pgfqpoint{1.295299in}{1.312997in}}%
\pgfpathlineto{\pgfqpoint{1.296113in}{1.320978in}}%
\pgfpathlineto{\pgfqpoint{1.297469in}{1.328797in}}%
\pgfpathlineto{\pgfqpoint{1.297740in}{1.324066in}}%
\pgfpathlineto{\pgfqpoint{1.298825in}{1.313625in}}%
\pgfpathlineto{\pgfqpoint{1.299503in}{1.319718in}}%
\pgfpathlineto{\pgfqpoint{1.300724in}{1.330591in}}%
\pgfpathlineto{\pgfqpoint{1.301538in}{1.328556in}}%
\pgfpathlineto{\pgfqpoint{1.301809in}{1.327419in}}%
\pgfpathlineto{\pgfqpoint{1.302487in}{1.331359in}}%
\pgfpathlineto{\pgfqpoint{1.304657in}{1.339679in}}%
\pgfpathlineto{\pgfqpoint{1.304793in}{1.339696in}}%
\pgfpathlineto{\pgfqpoint{1.305064in}{1.336663in}}%
\pgfpathlineto{\pgfqpoint{1.305878in}{1.343773in}}%
\pgfpathlineto{\pgfqpoint{1.308319in}{1.356855in}}%
\pgfpathlineto{\pgfqpoint{1.310760in}{1.365817in}}%
\pgfpathlineto{\pgfqpoint{1.310896in}{1.365333in}}%
\pgfpathlineto{\pgfqpoint{1.311709in}{1.364204in}}%
\pgfpathlineto{\pgfqpoint{1.313608in}{1.358096in}}%
\pgfpathlineto{\pgfqpoint{1.313744in}{1.358134in}}%
\pgfpathlineto{\pgfqpoint{1.316592in}{1.369662in}}%
\pgfpathlineto{\pgfqpoint{1.318219in}{1.373830in}}%
\pgfpathlineto{\pgfqpoint{1.318355in}{1.373732in}}%
\pgfpathlineto{\pgfqpoint{1.320118in}{1.375102in}}%
\pgfpathlineto{\pgfqpoint{1.318626in}{1.373159in}}%
\pgfpathlineto{\pgfqpoint{1.320661in}{1.373878in}}%
\pgfpathlineto{\pgfqpoint{1.322017in}{1.369150in}}%
\pgfpathlineto{\pgfqpoint{1.324729in}{1.365627in}}%
\pgfpathlineto{\pgfqpoint{1.325543in}{1.368577in}}%
\pgfpathlineto{\pgfqpoint{1.326899in}{1.371364in}}%
\pgfpathlineto{\pgfqpoint{1.327035in}{1.370187in}}%
\pgfpathlineto{\pgfqpoint{1.329205in}{1.355797in}}%
\pgfpathlineto{\pgfqpoint{1.329748in}{1.360418in}}%
\pgfpathlineto{\pgfqpoint{1.331239in}{1.366628in}}%
\pgfpathlineto{\pgfqpoint{1.331646in}{1.365541in}}%
\pgfpathlineto{\pgfqpoint{1.333003in}{1.359022in}}%
\pgfpathlineto{\pgfqpoint{1.334088in}{1.361094in}}%
\pgfpathlineto{\pgfqpoint{1.335715in}{1.365638in}}%
\pgfpathlineto{\pgfqpoint{1.336664in}{1.365201in}}%
\pgfpathlineto{\pgfqpoint{1.337885in}{1.362713in}}%
\pgfpathlineto{\pgfqpoint{1.338428in}{1.363673in}}%
\pgfpathlineto{\pgfqpoint{1.340598in}{1.376526in}}%
\pgfpathlineto{\pgfqpoint{1.340869in}{1.375590in}}%
\pgfpathlineto{\pgfqpoint{1.341411in}{1.377507in}}%
\pgfpathlineto{\pgfqpoint{1.343988in}{1.381106in}}%
\pgfpathlineto{\pgfqpoint{1.344938in}{1.380173in}}%
\pgfpathlineto{\pgfqpoint{1.347108in}{1.374333in}}%
\pgfpathlineto{\pgfqpoint{1.347379in}{1.373428in}}%
\pgfpathlineto{\pgfqpoint{1.352261in}{1.387851in}}%
\pgfpathlineto{\pgfqpoint{1.353346in}{1.388915in}}%
\pgfpathlineto{\pgfqpoint{1.353889in}{1.387861in}}%
\pgfpathlineto{\pgfqpoint{1.357415in}{1.393140in}}%
\pgfpathlineto{\pgfqpoint{1.354296in}{1.387006in}}%
\pgfpathlineto{\pgfqpoint{1.357551in}{1.392365in}}%
\pgfpathlineto{\pgfqpoint{1.359314in}{1.385127in}}%
\pgfpathlineto{\pgfqpoint{1.359992in}{1.386149in}}%
\pgfpathlineto{\pgfqpoint{1.362162in}{1.395059in}}%
\pgfpathlineto{\pgfqpoint{1.365010in}{1.397238in}}%
\pgfpathlineto{\pgfqpoint{1.366638in}{1.404162in}}%
\pgfpathlineto{\pgfqpoint{1.367451in}{1.402886in}}%
\pgfpathlineto{\pgfqpoint{1.368129in}{1.402122in}}%
\pgfpathlineto{\pgfqpoint{1.368672in}{1.403325in}}%
\pgfpathlineto{\pgfqpoint{1.368943in}{1.402701in}}%
\pgfpathlineto{\pgfqpoint{1.371113in}{1.406728in}}%
\pgfpathlineto{\pgfqpoint{1.371656in}{1.405842in}}%
\pgfpathlineto{\pgfqpoint{1.372605in}{1.406839in}}%
\pgfpathlineto{\pgfqpoint{1.375724in}{1.411651in}}%
\pgfpathlineto{\pgfqpoint{1.375996in}{1.410833in}}%
\pgfpathlineto{\pgfqpoint{1.377216in}{1.408487in}}%
\pgfpathlineto{\pgfqpoint{1.377759in}{1.410114in}}%
\pgfpathlineto{\pgfqpoint{1.379658in}{1.417028in}}%
\pgfpathlineto{\pgfqpoint{1.380471in}{1.415274in}}%
\pgfpathlineto{\pgfqpoint{1.381421in}{1.412935in}}%
\pgfpathlineto{\pgfqpoint{1.382099in}{1.414701in}}%
\pgfpathlineto{\pgfqpoint{1.384540in}{1.418629in}}%
\pgfpathlineto{\pgfqpoint{1.384676in}{1.418521in}}%
\pgfpathlineto{\pgfqpoint{1.385896in}{1.416940in}}%
\pgfpathlineto{\pgfqpoint{1.386439in}{1.417752in}}%
\pgfpathlineto{\pgfqpoint{1.387931in}{1.419812in}}%
\pgfpathlineto{\pgfqpoint{1.388202in}{1.418248in}}%
\pgfpathlineto{\pgfqpoint{1.390508in}{1.414525in}}%
\pgfpathlineto{\pgfqpoint{1.391050in}{1.415176in}}%
\pgfpathlineto{\pgfqpoint{1.391593in}{1.416645in}}%
\pgfpathlineto{\pgfqpoint{1.392542in}{1.415670in}}%
\pgfpathlineto{\pgfqpoint{1.393627in}{1.413219in}}%
\pgfpathlineto{\pgfqpoint{1.394305in}{1.414045in}}%
\pgfpathlineto{\pgfqpoint{1.395661in}{1.417175in}}%
\pgfpathlineto{\pgfqpoint{1.396339in}{1.415204in}}%
\pgfpathlineto{\pgfqpoint{1.396475in}{1.414749in}}%
\pgfpathlineto{\pgfqpoint{1.397424in}{1.416648in}}%
\pgfpathlineto{\pgfqpoint{1.402307in}{1.426413in}}%
\pgfpathlineto{\pgfqpoint{1.403121in}{1.426059in}}%
\pgfpathlineto{\pgfqpoint{1.404613in}{1.427622in}}%
\pgfpathlineto{\pgfqpoint{1.405291in}{1.425957in}}%
\pgfpathlineto{\pgfqpoint{1.406783in}{1.424870in}}%
\pgfpathlineto{\pgfqpoint{1.406918in}{1.425698in}}%
\pgfpathlineto{\pgfqpoint{1.407461in}{1.426216in}}%
\pgfpathlineto{\pgfqpoint{1.408274in}{1.424751in}}%
\pgfpathlineto{\pgfqpoint{1.408410in}{1.425402in}}%
\pgfpathlineto{\pgfqpoint{1.408546in}{1.425873in}}%
\pgfpathlineto{\pgfqpoint{1.409088in}{1.424509in}}%
\pgfpathlineto{\pgfqpoint{1.409766in}{1.425185in}}%
\pgfpathlineto{\pgfqpoint{1.410038in}{1.424343in}}%
\pgfpathlineto{\pgfqpoint{1.410716in}{1.425580in}}%
\pgfpathlineto{\pgfqpoint{1.411801in}{1.428343in}}%
\pgfpathlineto{\pgfqpoint{1.412343in}{1.427229in}}%
\pgfpathlineto{\pgfqpoint{1.412750in}{1.427263in}}%
\pgfpathlineto{\pgfqpoint{1.413157in}{1.426336in}}%
\pgfpathlineto{\pgfqpoint{1.413428in}{1.426144in}}%
\pgfpathlineto{\pgfqpoint{1.415056in}{1.427484in}}%
\pgfpathlineto{\pgfqpoint{1.416141in}{1.428035in}}%
\pgfpathlineto{\pgfqpoint{1.416548in}{1.427279in}}%
\pgfpathlineto{\pgfqpoint{1.416683in}{1.426660in}}%
\pgfpathlineto{\pgfqpoint{1.417633in}{1.428243in}}%
\pgfpathlineto{\pgfqpoint{1.417768in}{1.428233in}}%
\pgfpathlineto{\pgfqpoint{1.420345in}{1.433040in}}%
\pgfpathlineto{\pgfqpoint{1.421023in}{1.431588in}}%
\pgfpathlineto{\pgfqpoint{1.421566in}{1.433850in}}%
\pgfpathlineto{\pgfqpoint{1.424007in}{1.436936in}}%
\pgfpathlineto{\pgfqpoint{1.425363in}{1.437873in}}%
\pgfpathlineto{\pgfqpoint{1.425770in}{1.437970in}}%
\pgfpathlineto{\pgfqpoint{1.426584in}{1.436919in}}%
\pgfpathlineto{\pgfqpoint{1.426855in}{1.436315in}}%
\pgfpathlineto{\pgfqpoint{1.427804in}{1.437762in}}%
\pgfpathlineto{\pgfqpoint{1.429974in}{1.441422in}}%
\pgfpathlineto{\pgfqpoint{1.430246in}{1.441275in}}%
\pgfpathlineto{\pgfqpoint{1.430517in}{1.439808in}}%
\pgfpathlineto{\pgfqpoint{1.431602in}{1.441869in}}%
\pgfpathlineto{\pgfqpoint{1.431738in}{1.441677in}}%
\pgfpathlineto{\pgfqpoint{1.435399in}{1.448647in}}%
\pgfpathlineto{\pgfqpoint{1.435671in}{1.447909in}}%
\pgfpathlineto{\pgfqpoint{1.437298in}{1.444785in}}%
\pgfpathlineto{\pgfqpoint{1.437569in}{1.445212in}}%
\pgfpathlineto{\pgfqpoint{1.440418in}{1.447301in}}%
\pgfpathlineto{\pgfqpoint{1.442045in}{1.449652in}}%
\pgfpathlineto{\pgfqpoint{1.442452in}{1.449151in}}%
\pgfpathlineto{\pgfqpoint{1.444758in}{1.452222in}}%
\pgfpathlineto{\pgfqpoint{1.444893in}{1.452764in}}%
\pgfpathlineto{\pgfqpoint{1.446249in}{1.451900in}}%
\pgfpathlineto{\pgfqpoint{1.447606in}{1.453454in}}%
\pgfpathlineto{\pgfqpoint{1.448962in}{1.456895in}}%
\pgfpathlineto{\pgfqpoint{1.449504in}{1.455545in}}%
\pgfpathlineto{\pgfqpoint{1.451674in}{1.454068in}}%
\pgfpathlineto{\pgfqpoint{1.452217in}{1.455057in}}%
\pgfpathlineto{\pgfqpoint{1.454116in}{1.456236in}}%
\pgfpathlineto{\pgfqpoint{1.454251in}{1.455905in}}%
\pgfpathlineto{\pgfqpoint{1.454523in}{1.455385in}}%
\pgfpathlineto{\pgfqpoint{1.455065in}{1.456541in}}%
\pgfpathlineto{\pgfqpoint{1.455608in}{1.456640in}}%
\pgfpathlineto{\pgfqpoint{1.459134in}{1.454417in}}%
\pgfpathlineto{\pgfqpoint{1.459405in}{1.455542in}}%
\pgfpathlineto{\pgfqpoint{1.460761in}{1.460272in}}%
\pgfpathlineto{\pgfqpoint{1.461304in}{1.459068in}}%
\pgfpathlineto{\pgfqpoint{1.463203in}{1.457457in}}%
\pgfpathlineto{\pgfqpoint{1.463338in}{1.457620in}}%
\pgfpathlineto{\pgfqpoint{1.465915in}{1.461357in}}%
\pgfpathlineto{\pgfqpoint{1.467678in}{1.460425in}}%
\pgfpathlineto{\pgfqpoint{1.468221in}{1.459878in}}%
\pgfpathlineto{\pgfqpoint{1.468763in}{1.461002in}}%
\pgfpathlineto{\pgfqpoint{1.471069in}{1.463950in}}%
\pgfpathlineto{\pgfqpoint{1.474324in}{1.464528in}}%
\pgfpathlineto{\pgfqpoint{1.474595in}{1.463664in}}%
\pgfpathlineto{\pgfqpoint{1.475816in}{1.463901in}}%
\pgfpathlineto{\pgfqpoint{1.475951in}{1.464214in}}%
\pgfpathlineto{\pgfqpoint{1.478393in}{1.466513in}}%
\pgfpathlineto{\pgfqpoint{1.478528in}{1.466410in}}%
\pgfpathlineto{\pgfqpoint{1.478664in}{1.467078in}}%
\pgfpathlineto{\pgfqpoint{1.479071in}{1.467307in}}%
\pgfpathlineto{\pgfqpoint{1.481783in}{1.473763in}}%
\pgfpathlineto{\pgfqpoint{1.483411in}{1.473020in}}%
\pgfpathlineto{\pgfqpoint{1.483546in}{1.473431in}}%
\pgfpathlineto{\pgfqpoint{1.484360in}{1.474460in}}%
\pgfpathlineto{\pgfqpoint{1.485174in}{1.473793in}}%
\pgfpathlineto{\pgfqpoint{1.486123in}{1.473878in}}%
\pgfpathlineto{\pgfqpoint{1.486394in}{1.474514in}}%
\pgfpathlineto{\pgfqpoint{1.489243in}{1.476935in}}%
\pgfpathlineto{\pgfqpoint{1.490463in}{1.476730in}}%
\pgfpathlineto{\pgfqpoint{1.490734in}{1.476420in}}%
\pgfpathlineto{\pgfqpoint{1.492091in}{1.474160in}}%
\pgfpathlineto{\pgfqpoint{1.492633in}{1.475991in}}%
\pgfpathlineto{\pgfqpoint{1.496431in}{1.482490in}}%
\pgfpathlineto{\pgfqpoint{1.496566in}{1.482409in}}%
\pgfpathlineto{\pgfqpoint{1.497244in}{1.483605in}}%
\pgfpathlineto{\pgfqpoint{1.498058in}{1.483674in}}%
\pgfpathlineto{\pgfqpoint{1.498465in}{1.482994in}}%
\pgfpathlineto{\pgfqpoint{1.500499in}{1.482914in}}%
\pgfpathlineto{\pgfqpoint{1.501856in}{1.485868in}}%
\pgfpathlineto{\pgfqpoint{1.502805in}{1.485154in}}%
\pgfpathlineto{\pgfqpoint{1.502941in}{1.485000in}}%
\pgfpathlineto{\pgfqpoint{1.503483in}{1.486464in}}%
\pgfpathlineto{\pgfqpoint{1.503890in}{1.486110in}}%
\pgfpathlineto{\pgfqpoint{1.508230in}{1.492145in}}%
\pgfpathlineto{\pgfqpoint{1.512570in}{1.493102in}}%
\pgfpathlineto{\pgfqpoint{1.513791in}{1.494332in}}%
\pgfpathlineto{\pgfqpoint{1.514333in}{1.493888in}}%
\pgfpathlineto{\pgfqpoint{1.515825in}{1.496068in}}%
\pgfpathlineto{\pgfqpoint{1.517724in}{1.497296in}}%
\pgfpathlineto{\pgfqpoint{1.517859in}{1.497240in}}%
\pgfpathlineto{\pgfqpoint{1.521250in}{1.498118in}}%
\pgfpathlineto{\pgfqpoint{1.524234in}{1.502131in}}%
\pgfpathlineto{\pgfqpoint{1.524505in}{1.501971in}}%
\pgfpathlineto{\pgfqpoint{1.526404in}{1.502930in}}%
\pgfpathlineto{\pgfqpoint{1.532371in}{1.505535in}}%
\pgfpathlineto{\pgfqpoint{1.533321in}{1.504538in}}%
\pgfpathlineto{\pgfqpoint{1.534134in}{1.505116in}}%
\pgfpathlineto{\pgfqpoint{1.537389in}{1.508010in}}%
\pgfpathlineto{\pgfqpoint{1.556513in}{1.519056in}}%
\pgfpathlineto{\pgfqpoint{1.558411in}{1.519971in}}%
\pgfpathlineto{\pgfqpoint{1.561802in}{1.522919in}}%
\pgfpathlineto{\pgfqpoint{1.562073in}{1.522711in}}%
\pgfpathlineto{\pgfqpoint{1.565193in}{1.522731in}}%
\pgfpathlineto{\pgfqpoint{1.568041in}{1.523900in}}%
\pgfpathlineto{\pgfqpoint{1.571024in}{1.524649in}}%
\pgfpathlineto{\pgfqpoint{1.572923in}{1.524599in}}%
\pgfpathlineto{\pgfqpoint{1.574822in}{1.525637in}}%
\pgfpathlineto{\pgfqpoint{1.577941in}{1.529756in}}%
\pgfpathlineto{\pgfqpoint{1.578077in}{1.529676in}}%
\pgfpathlineto{\pgfqpoint{1.579976in}{1.529979in}}%
\pgfpathlineto{\pgfqpoint{1.581739in}{1.530516in}}%
\pgfpathlineto{\pgfqpoint{1.583909in}{1.530698in}}%
\pgfpathlineto{\pgfqpoint{1.585672in}{1.531062in}}%
\pgfpathlineto{\pgfqpoint{1.585808in}{1.531408in}}%
\pgfpathlineto{\pgfqpoint{1.590012in}{1.534605in}}%
\pgfpathlineto{\pgfqpoint{1.590148in}{1.534389in}}%
\pgfpathlineto{\pgfqpoint{1.593131in}{1.535830in}}%
\pgfpathlineto{\pgfqpoint{1.594623in}{1.536582in}}%
\pgfpathlineto{\pgfqpoint{1.594894in}{1.536281in}}%
\pgfpathlineto{\pgfqpoint{1.638837in}{1.554596in}}%
\pgfpathlineto{\pgfqpoint{1.642228in}{1.556798in}}%
\pgfpathlineto{\pgfqpoint{1.652399in}{1.560333in}}%
\pgfpathlineto{\pgfqpoint{1.656061in}{1.562522in}}%
\pgfpathlineto{\pgfqpoint{1.658638in}{1.563802in}}%
\pgfpathlineto{\pgfqpoint{1.709633in}{1.581041in}}%
\pgfpathlineto{\pgfqpoint{1.712210in}{1.581720in}}%
\pgfpathlineto{\pgfqpoint{1.712346in}{1.581550in}}%
\pgfpathlineto{\pgfqpoint{1.714109in}{1.582246in}}%
\pgfpathlineto{\pgfqpoint{1.721839in}{1.584794in}}%
\pgfpathlineto{\pgfqpoint{1.723331in}{1.585542in}}%
\pgfpathlineto{\pgfqpoint{1.726179in}{1.586475in}}%
\pgfpathlineto{\pgfqpoint{1.729163in}{1.587342in}}%
\pgfpathlineto{\pgfqpoint{1.730655in}{1.587234in}}%
\pgfpathlineto{\pgfqpoint{1.730791in}{1.587360in}}%
\pgfpathlineto{\pgfqpoint{1.734317in}{1.589200in}}%
\pgfpathlineto{\pgfqpoint{1.738114in}{1.590455in}}%
\pgfpathlineto{\pgfqpoint{1.741369in}{1.591615in}}%
\pgfpathlineto{\pgfqpoint{1.782871in}{1.608030in}}%
\pgfpathlineto{\pgfqpoint{1.785719in}{1.608952in}}%
\pgfpathlineto{\pgfqpoint{1.793178in}{1.611260in}}%
\pgfpathlineto{\pgfqpoint{1.799417in}{1.613195in}}%
\pgfpathlineto{\pgfqpoint{1.802265in}{1.613881in}}%
\pgfpathlineto{\pgfqpoint{1.804571in}{1.614597in}}%
\pgfpathlineto{\pgfqpoint{1.888929in}{1.636705in}}%
\pgfpathlineto{\pgfqpoint{1.893676in}{1.637468in}}%
\pgfpathlineto{\pgfqpoint{1.899644in}{1.638685in}}%
\pgfpathlineto{\pgfqpoint{1.989428in}{1.655395in}}%
\pgfpathlineto{\pgfqpoint{1.992954in}{1.655755in}}%
\pgfpathlineto{\pgfqpoint{1.996344in}{1.656293in}}%
\pgfpathlineto{\pgfqpoint{2.004075in}{1.657364in}}%
\pgfpathlineto{\pgfqpoint{2.008279in}{1.657597in}}%
\pgfpathlineto{\pgfqpoint{2.040016in}{1.660878in}}%
\pgfpathlineto{\pgfqpoint{2.061851in}{1.664814in}}%
\pgfpathlineto{\pgfqpoint{2.065649in}{1.665332in}}%
\pgfpathlineto{\pgfqpoint{2.069039in}{1.665715in}}%
\pgfpathlineto{\pgfqpoint{2.074193in}{1.666302in}}%
\pgfpathlineto{\pgfqpoint{2.078669in}{1.666947in}}%
\pgfpathlineto{\pgfqpoint{2.082195in}{1.667796in}}%
\pgfpathlineto{\pgfqpoint{2.087349in}{1.667664in}}%
\pgfpathlineto{\pgfqpoint{2.089519in}{1.668317in}}%
\pgfpathlineto{\pgfqpoint{2.092774in}{1.668210in}}%
\pgfpathlineto{\pgfqpoint{2.098877in}{1.669087in}}%
\pgfpathlineto{\pgfqpoint{2.102810in}{1.669707in}}%
\pgfpathlineto{\pgfqpoint{2.108913in}{1.670049in}}%
\pgfpathlineto{\pgfqpoint{2.140514in}{1.671691in}}%
\pgfpathlineto{\pgfqpoint{2.148244in}{1.672731in}}%
\pgfpathlineto{\pgfqpoint{2.165469in}{1.673835in}}%
\pgfpathlineto{\pgfqpoint{2.171165in}{1.673994in}}%
\pgfpathlineto{\pgfqpoint{2.173606in}{1.674127in}}%
\pgfpathlineto{\pgfqpoint{2.181201in}{1.674388in}}%
\pgfpathlineto{\pgfqpoint{2.184049in}{1.674228in}}%
\pgfpathlineto{\pgfqpoint{2.187169in}{1.674178in}}%
\pgfpathlineto{\pgfqpoint{2.190153in}{1.674648in}}%
\pgfpathlineto{\pgfqpoint{2.194899in}{1.674827in}}%
\pgfpathlineto{\pgfqpoint{2.219854in}{1.676322in}}%
\pgfpathlineto{\pgfqpoint{2.222974in}{1.676497in}}%
\pgfpathlineto{\pgfqpoint{2.226093in}{1.676766in}}%
\pgfpathlineto{\pgfqpoint{2.228941in}{1.676399in}}%
\pgfpathlineto{\pgfqpoint{2.238435in}{1.677188in}}%
\pgfpathlineto{\pgfqpoint{2.243589in}{1.676943in}}%
\pgfpathlineto{\pgfqpoint{2.250777in}{1.677175in}}%
\pgfpathlineto{\pgfqpoint{2.254303in}{1.677865in}}%
\pgfpathlineto{\pgfqpoint{2.266103in}{1.678462in}}%
\pgfpathlineto{\pgfqpoint{2.271799in}{1.679528in}}%
\pgfpathlineto{\pgfqpoint{2.275732in}{1.679180in}}%
\pgfpathlineto{\pgfqpoint{2.284005in}{1.679591in}}%
\pgfpathlineto{\pgfqpoint{2.288209in}{1.679471in}}%
\pgfpathlineto{\pgfqpoint{2.295940in}{1.679238in}}%
\pgfpathlineto{\pgfqpoint{2.298517in}{1.679257in}}%
\pgfpathlineto{\pgfqpoint{2.387351in}{1.678773in}}%
\pgfpathlineto{\pgfqpoint{2.390742in}{1.678899in}}%
\pgfpathlineto{\pgfqpoint{2.398608in}{1.678608in}}%
\pgfpathlineto{\pgfqpoint{2.403084in}{1.678900in}}%
\pgfpathlineto{\pgfqpoint{2.408373in}{1.679476in}}%
\pgfpathlineto{\pgfqpoint{2.413256in}{1.679001in}}%
\pgfpathlineto{\pgfqpoint{2.420308in}{1.679037in}}%
\pgfpathlineto{\pgfqpoint{2.423970in}{1.678922in}}%
\pgfpathlineto{\pgfqpoint{2.539794in}{1.677562in}}%
\pgfpathlineto{\pgfqpoint{2.544812in}{1.677199in}}%
\pgfpathlineto{\pgfqpoint{2.548067in}{1.676909in}}%
\pgfpathlineto{\pgfqpoint{2.552949in}{1.676467in}}%
\pgfpathlineto{\pgfqpoint{2.561629in}{1.677076in}}%
\pgfpathlineto{\pgfqpoint{2.566919in}{1.676781in}}%
\pgfpathlineto{\pgfqpoint{2.571394in}{1.676858in}}%
\pgfpathlineto{\pgfqpoint{2.576277in}{1.676321in}}%
\pgfpathlineto{\pgfqpoint{2.702951in}{1.677558in}}%
\pgfpathlineto{\pgfqpoint{2.708240in}{1.676888in}}%
\pgfpathlineto{\pgfqpoint{2.711766in}{1.676481in}}%
\pgfpathlineto{\pgfqpoint{2.714750in}{1.676310in}}%
\pgfpathlineto{\pgfqpoint{2.719497in}{1.676192in}}%
\pgfpathlineto{\pgfqpoint{2.726414in}{1.676117in}}%
\pgfpathlineto{\pgfqpoint{2.728991in}{1.675944in}}%
\pgfpathlineto{\pgfqpoint{2.763304in}{1.675051in}}%
\pgfpathlineto{\pgfqpoint{2.766288in}{1.675104in}}%
\pgfpathlineto{\pgfqpoint{2.778087in}{1.674400in}}%
\pgfpathlineto{\pgfqpoint{2.781613in}{1.674665in}}%
\pgfpathlineto{\pgfqpoint{2.785411in}{1.673915in}}%
\pgfpathlineto{\pgfqpoint{2.790429in}{1.673140in}}%
\pgfpathlineto{\pgfqpoint{2.797074in}{1.672870in}}%
\pgfpathlineto{\pgfqpoint{2.802771in}{1.672308in}}%
\pgfpathlineto{\pgfqpoint{2.831252in}{1.670408in}}%
\pgfpathlineto{\pgfqpoint{2.834507in}{1.670275in}}%
\pgfpathlineto{\pgfqpoint{2.838983in}{1.669960in}}%
\pgfpathlineto{\pgfqpoint{2.851324in}{1.669142in}}%
\pgfpathlineto{\pgfqpoint{2.855258in}{1.669561in}}%
\pgfpathlineto{\pgfqpoint{2.859869in}{1.668954in}}%
\pgfpathlineto{\pgfqpoint{2.863802in}{1.668570in}}%
\pgfpathlineto{\pgfqpoint{2.868820in}{1.667933in}}%
\pgfpathlineto{\pgfqpoint{2.871397in}{1.668038in}}%
\pgfpathlineto{\pgfqpoint{2.878043in}{1.667543in}}%
\pgfpathlineto{\pgfqpoint{2.885909in}{1.666731in}}%
\pgfpathlineto{\pgfqpoint{2.893639in}{1.666625in}}%
\pgfpathlineto{\pgfqpoint{2.899607in}{1.666486in}}%
\pgfpathlineto{\pgfqpoint{2.905168in}{1.665906in}}%
\pgfpathlineto{\pgfqpoint{2.905168in}{1.665906in}}%
\pgfusepath{stroke}%
\end{pgfscope}%
\begin{pgfscope}%
\pgfsetrectcap%
\pgfsetmiterjoin%
\pgfsetlinewidth{0.803000pt}%
\definecolor{currentstroke}{rgb}{0.000000,0.000000,0.000000}%
\pgfsetstrokecolor{currentstroke}%
\pgfsetdash{}{0pt}%
\pgfpathmoveto{\pgfqpoint{0.735032in}{0.526079in}}%
\pgfpathlineto{\pgfqpoint{0.735032in}{2.187079in}}%
\pgfusepath{stroke}%
\end{pgfscope}%
\begin{pgfscope}%
\pgfsetrectcap%
\pgfsetmiterjoin%
\pgfsetlinewidth{0.803000pt}%
\definecolor{currentstroke}{rgb}{0.000000,0.000000,0.000000}%
\pgfsetstrokecolor{currentstroke}%
\pgfsetdash{}{0pt}%
\pgfpathmoveto{\pgfqpoint{2.905032in}{0.526079in}}%
\pgfpathlineto{\pgfqpoint{2.905032in}{2.187079in}}%
\pgfusepath{stroke}%
\end{pgfscope}%
\begin{pgfscope}%
\pgfsetrectcap%
\pgfsetmiterjoin%
\pgfsetlinewidth{0.803000pt}%
\definecolor{currentstroke}{rgb}{0.000000,0.000000,0.000000}%
\pgfsetstrokecolor{currentstroke}%
\pgfsetdash{}{0pt}%
\pgfpathmoveto{\pgfqpoint{0.735032in}{0.526079in}}%
\pgfpathlineto{\pgfqpoint{2.905032in}{0.526079in}}%
\pgfusepath{stroke}%
\end{pgfscope}%
\begin{pgfscope}%
\pgfsetrectcap%
\pgfsetmiterjoin%
\pgfsetlinewidth{0.803000pt}%
\definecolor{currentstroke}{rgb}{0.000000,0.000000,0.000000}%
\pgfsetstrokecolor{currentstroke}%
\pgfsetdash{}{0pt}%
\pgfpathmoveto{\pgfqpoint{0.735032in}{2.187079in}}%
\pgfpathlineto{\pgfqpoint{2.905032in}{2.187079in}}%
\pgfusepath{stroke}%
\end{pgfscope}%
\begin{pgfscope}%
\pgfsetbuttcap%
\pgfsetmiterjoin%
\definecolor{currentfill}{rgb}{1.000000,1.000000,1.000000}%
\pgfsetfillcolor{currentfill}%
\pgfsetfillopacity{0.800000}%
\pgfsetlinewidth{1.003750pt}%
\definecolor{currentstroke}{rgb}{0.800000,0.800000,0.800000}%
\pgfsetstrokecolor{currentstroke}%
\pgfsetstrokeopacity{0.800000}%
\pgfsetdash{}{0pt}%
\pgfpathmoveto{\pgfqpoint{3.050674in}{1.462430in}}%
\pgfpathlineto{\pgfqpoint{5.146390in}{1.462430in}}%
\pgfpathquadraticcurveto{\pgfqpoint{5.174168in}{1.462430in}}{\pgfqpoint{5.174168in}{1.490208in}}%
\pgfpathlineto{\pgfqpoint{5.174168in}{2.089857in}}%
\pgfpathquadraticcurveto{\pgfqpoint{5.174168in}{2.117635in}}{\pgfqpoint{5.146390in}{2.117635in}}%
\pgfpathlineto{\pgfqpoint{3.050674in}{2.117635in}}%
\pgfpathquadraticcurveto{\pgfqpoint{3.022896in}{2.117635in}}{\pgfqpoint{3.022896in}{2.089857in}}%
\pgfpathlineto{\pgfqpoint{3.022896in}{1.490208in}}%
\pgfpathquadraticcurveto{\pgfqpoint{3.022896in}{1.462430in}}{\pgfqpoint{3.050674in}{1.462430in}}%
\pgfpathclose%
\pgfusepath{stroke,fill}%
\end{pgfscope}%
\begin{pgfscope}%
\pgfsetrectcap%
\pgfsetroundjoin%
\pgfsetlinewidth{1.003750pt}%
\definecolor{currentstroke}{rgb}{0.627451,0.321569,0.176471}%
\pgfsetstrokecolor{currentstroke}%
\pgfsetdash{}{0pt}%
\pgfpathmoveto{\pgfqpoint{3.078451in}{2.005167in}}%
\pgfpathlineto{\pgfqpoint{3.356229in}{2.005167in}}%
\pgfusepath{stroke}%
\end{pgfscope}%
\begin{pgfscope}%
\pgftext[x=3.467340in,y=1.956556in,left,base]{\rmfamily\fontsize{10.000000}{12.000000}\selectfont Geometric (Lie-Trotter)}%
\end{pgfscope}%
\begin{pgfscope}%
\pgfsetrectcap%
\pgfsetroundjoin%
\pgfsetlinewidth{1.003750pt}%
\definecolor{currentstroke}{rgb}{1.000000,0.549020,0.000000}%
\pgfsetstrokecolor{currentstroke}%
\pgfsetdash{}{0pt}%
\pgfpathmoveto{\pgfqpoint{3.078451in}{1.801310in}}%
\pgfpathlineto{\pgfqpoint{3.356229in}{1.801310in}}%
\pgfusepath{stroke}%
\end{pgfscope}%
\begin{pgfscope}%
\pgftext[x=3.467340in,y=1.752699in,left,base]{\rmfamily\fontsize{10.000000}{12.000000}\selectfont Geometric (Strang)}%
\end{pgfscope}%
\begin{pgfscope}%
\pgfsetrectcap%
\pgfsetroundjoin%
\pgfsetlinewidth{1.003750pt}%
\definecolor{currentstroke}{rgb}{0.501961,0.000000,0.501961}%
\pgfsetstrokecolor{currentstroke}%
\pgfsetdash{}{0pt}%
\pgfpathmoveto{\pgfqpoint{3.078451in}{1.595486in}}%
\pgfpathlineto{\pgfqpoint{3.356229in}{1.595486in}}%
\pgfusepath{stroke}%
\end{pgfscope}%
\begin{pgfscope}%
\pgftext[x=3.467340in,y=1.546875in,left,base]{\rmfamily\fontsize{10.000000}{12.000000}\selectfont Standard}%
\end{pgfscope}%
\end{pgfpicture}%
\makeatother%
\endgroup%

\caption{Time evoltuion of the relative error in the conservation of energy for the different numerical algorithms.\label{fig_comparison}}
\end{figure}
\section{Summary}
In this work, we have presented two different finite element particle-in-cell algorithms for a four-dimensional hybrid plasma model and compared the results for a single test run. The considered hybrid plasma model is a combined kinetic/fluid description for a magnetized plasma, which consists of cold (fluid) electrons and energetic (kinetic) electrons that move in a stationary, neutralizing background of ions. The model's key physics content for wave propagation parallel to a uniform background magnetic field is that it predicts the existence of growing/damped modes due to energy exchange between the energetic electrons which propagate in the cold plasma.

For this case, first, a combination of one-dimensional B-spline finite elements for Maxwell's equations and the momentum balance equation for the cold electrons and the standard particle-in-cell method with a Boris particle pusher for the Vlasov equation (one dimension in real space and three dimensions in velocity space) has been applied in an intuitive way without taking into account the geometric structure of the equations. Second, geometric finite element particle-in-cell methods \citep{Krausetal2017} which use tools from the \textit{finite element exterior calculus} have been applied on the same model. By choosing finite elements spaces and projectors on these spaces satisfying a commuting diagram with the continuous spaces, a semi-discrete system (discrete in space and continuous in time) with a non-canonical Hamiltonian structure for the time evolution of all finite element coefficients and particle configurations has been derived. Consequently, this system exhibits exact energy conservation. The subsequent construction of Poisson time integrators by splitting the Hamiltonian and analytically solving the resulting subsystems has led qualitatively to a bounded error in the conservation of energy for the presented numerical experiment.  

\appendix
\section{Time integrators for Hamiltonian splitting}
\label{sec_appendix}
\noindent \textbf{Problem 1}. For $t\in[0,\Delta t]$ and $\textbf{u}(t=0)=\textbf{u}^0$ we have
\begin{align}
\frac{\mathrm{d} \textbf{u}}{\mathrm{d}t}=\mathbb{J}(\textbf{u})\nabla_{\textbf{u}} H_E(\textbf{u})=\mathbb{J}(\mb{u})\nabla_{\textbf{u}}\left[\frac{\epsilon_0}{2}(\textbf{e}_x^\top\mathbb{M}^0\textbf{e}_x+\textbf{e}_y^\top\mathbb{M}^0\textbf{e}_y)\right].
\end{align}
This can be solved analytically as
\begin{subequations}
\begin{alignat}{3}
    &\frac{\mathrm{d} \textbf{e}_x}{\mathrm{d}t}=0 &&\Longrightarrow\quad \textbf{e}_x(\Delta t) = \textbf{e}_x^0,\\
    &\frac{\mathrm{d} \textbf{e}_y}{\mathrm{d}t}=0 &&\Longrightarrow\quad \textbf{e}_y(\Delta t) = \textbf{e}_y^0,\\
    &\frac{\mathrm{d} \textbf{b}_x}{\mathrm{d}t}=\frac{1}{\epsilon_0}\mathbb{G}(\mathbb{M}^0)^{-1}\epsilon_0\mathbb{M}^0\textbf{e}_y &&\Longrightarrow\quad \textbf{b}_x(\Delta t) = \textbf{b}_x^0 + \Delta t\mathbb{G}\textbf{e}_y^0,\\
    &\frac{\mathrm{d} \textbf{b}_y}{\mathrm{d}t}=-\frac{1}{\epsilon_0}\mathbb{G}(\mathbb{M}^0)^{-1}\epsilon_0\mathbb{M}^0\textbf{e}_x &&\Longrightarrow\quad \textbf{b}_y(\Delta t) = \textbf{b}_y^0 - \Delta t\mathbb{G}\textbf{e}_x^0,\\
     &\frac{\mathrm{d} \textbf{y}_x}{\mathrm{d}t}=\Omega_\mathrm{pe}^2(\mathbb{M}^0)^{-1}\epsilon_0\mathbb{M}^0\textbf{e}_x &&\Longrightarrow\quad \textbf{y}_x(\Delta t) = \textbf{y}_x^0 + \Delta t\epsilon_0\Omega_\mathrm{pe}^2\textbf{e}_x^0,\\
     &\frac{\mathrm{d} \textbf{y}_y}{\mathrm{d}t}=\Omega_\mathrm{pe}^2(\mathbb{M}^0)^{-1}\epsilon_0\mathbb{M}^0\textbf{e}_y &&\Longrightarrow\quad \textbf{y}_y(\Delta t) = \textbf{y}_y^0 + \Delta t\epsilon_0\Omega_\mathrm{pe}^2\textbf{e}_y^0,\\
     &\frac{\mathrm{d} \textbf{Z}}{\mathrm{d}t}=0 &&\Longrightarrow\quad \textbf{Z}(\Delta t) = \textbf{Z}^0,\\
     &\frac{\mathrm{d} \textbf{V}_x}{\mathrm{d}t}=\frac{q_\text{e}}{\epsilon_0m_\text{e}}(\mathbb{Q}^0)^\top(\mathbb{M}^0)^{-1}\epsilon_0\mathbb{M}^0\textbf{e}_x\quad &&\Longrightarrow\quad \textbf{V}_x(\Delta t) = \textbf{V}_x^0 + \Delta t\frac{q_\text{e}}{m_\text{e}}(\mathbb{Q}^0)^\top(\textbf{Z}^0)\textbf{e}_x^0,\\ 
     &\frac{\mathrm{d} \textbf{V}_y}{\mathrm{d}t}=\frac{q_\text{e}}{\epsilon_0m_\text{e}}(\mathbb{Q}^0)^\top(\mathbb{M}^0)^{-1}\epsilon_0\mathbb{M}^0\textbf{e}_y &&\Longrightarrow\quad \textbf{V}_y(\Delta t) = \textbf{V}_x^0 + \Delta t\frac{q_\text{e}}{m_\text{e}}(\mathbb{Q}^0)^\top(\textbf{Z}^0)\textbf{e}_y^0,\\ 
     &\frac{\mathrm{d} \textbf{V}_z}{\mathrm{d}t}=0 &&\Longrightarrow\quad \textbf{V}_z(\Delta t) = \textbf{V}_z^0.
\end{alignat}
\end{subequations}
The corresponding integrator is denoted by $\textbf{u}(\Delta t)=\Phi_{\Delta t}^E(\textbf{u}^0)$.\\ \\
\textbf{Problem 2}. For $t\in[0,\Delta t]$ and $\textbf{u}(t=0)=\textbf{u}^0$ we have
\begin{align}
\frac{\mathrm{d} \textbf{u}}{\mathrm{d}t}=\mathbb{J}(\textbf{u})\nabla_{\textbf{u}} H_B(\textbf{u})=\mathbb{J}(\mb{u})\nabla_{\textbf{u}}\left[\frac{1}{2\mu_0}(\textbf{b}_x^\top\mathbb{M}^1\textbf{b}_x+\textbf{b}_y^\top\mathbb{M}^1\textbf{b}_y)\right].
\end{align}
This can be solved analytically as
\begin{subequations}
\begin{alignat}{3}
    &\frac{\mathrm{d} \textbf{e}_x}{\mathrm{d}t}=\frac{1}{\epsilon_0}(\mathbb{M}^0)^{-1}\mathbb{G}^\top\frac{1}{\mu_0}\mathbb{M}^1\textbf{b}_y &&\Longrightarrow\quad \textbf{e}_x(\Delta t) = \textbf{e}_x^0 + \Delta tc^2(\mathbb{M}^0)^{-1}\mathbb{G}^\top\mathbb{M}^1\textbf{b}_y^0,\\
    &\frac{\mathrm{d} \textbf{e}_y}{\mathrm{d}t}=-\frac{1}{\epsilon_0}(\mathbb{M}^0)^{-1}\mathbb{G}^\top\frac{1}{\mu_0}\mathbb{M}^1\textbf{b}_x \quad &&\Longrightarrow\quad \textbf{e}_y(\Delta t) = \textbf{e}_y^0 - \Delta tc^2(\mathbb{M}^0)^{-1}\mathbb{G}^\top\mathbb{M}^1\textbf{b}_x^0,\\
    &\frac{\mathrm{d} \textbf{b}_x}{\mathrm{d}t}=0 &&\Longrightarrow\quad \textbf{b}_x(\Delta t) = \textbf{b}_x^0,\\
    &\frac{\mathrm{d} \textbf{b}_y}{\mathrm{d}t}=0 &&\Longrightarrow\quad \textbf{b}_y(\Delta t) = \textbf{b}_y^0,\\
     &\frac{\mathrm{d} \textbf{y}_x}{\mathrm{d}t}=0 &&\Longrightarrow\quad \textbf{y}_x(\Delta t) = \textbf{y}_x^0,\\
     &\frac{\mathrm{d} \textbf{y}_y}{\mathrm{d}t}=0 &&\Longrightarrow\quad \textbf{y}_y(\Delta t) = \textbf{y}_y^0,\\
     &\frac{\mathrm{d} \textbf{Z}}{\mathrm{d}t}=0 &&\Longrightarrow\quad \textbf{Z}(\Delta t) = \textbf{Z}^0,\\
     &\frac{\mathrm{d} \textbf{V}_x}{\mathrm{d}t}=0 &&\Longrightarrow\quad \textbf{V}_x(\Delta t) = \textbf{V}_x^0,\\
     &\frac{\mathrm{d} \textbf{V}_y}{\mathrm{d}t}=0 &&\Longrightarrow\quad \textbf{V}_y(\Delta t) = \textbf{V}_y^0,\\
     &\frac{\mathrm{d} \textbf{V}_z}{\mathrm{d}t}=0 &&\Longrightarrow\quad \textbf{V}_z(\Delta t) = \textbf{V}_z^0.
\end{alignat}
\end{subequations}
The corresponding integrator is denoted by $\textbf{u}(\Delta t)=\Phi_{\Delta t}^B(\textbf{u}^0)$.\\ \\
\textbf{Problem 3}. For $t\in[0,\Delta t]$ and $\textbf{u}(t=0)=\textbf{u}^0$, we have
\begin{align}
\frac{\mathrm{d} \textbf{u}}{\mathrm{d}t}=\mathbb{J}(\textbf{u})\nabla_{\textbf{u}} H_Y(\textbf{u})=\mathbb{J}(\mb{u})\nabla_{\textbf{u}}\left[\frac{1}{2\epsilon_0\Omega_\mr{pe}^2}(\textbf{y}_x^\top\mathbb{M}^0\textbf{y}_x+\textbf{y}_y^\top\mathbb{M}^0\textbf{y}_y)\right].
\end{align}
This can be solved analytically as
\begin{subequations}
\begin{alignat}{3}
\begin{split}
    &\frac{\mathrm{d} \textbf{e}_x}{\mathrm{d}t}=-\Omega_\mathrm{pe}^2(\mathbb{M}^0)^{-1}\frac{1}{\epsilon_0\Omega_\mathrm{pe}^2}\mathbb{M}^0\textbf{y}_x \\
    &\Longrightarrow \textbf{e}_x(\Delta t)=\textbf{e}_x^0 - \frac{1}{\epsilon_0}\int_{0}^{\Delta t} y_x(t^\prime)\mathrm{d}t^\prime =\textbf{e}_x^0 - \frac{1}{\epsilon_0\Omega_\mathrm{ce}}[\textbf{y}_x^0\sin(\Omega_\textbf{ce}t)-\textbf{y}_y^0\cos(\Omega_\textbf{ce}t)+\textbf{y}_y^0],
\end{split}\\
\begin{split}
    &\frac{\mathrm{d} \textbf{e}_y}{\mathrm{d}t}=-\Omega_\mathrm{pe}^2(\mathbb{M}^0)^{-1}\frac{1}{\epsilon_0\Omega_\mathrm{pe}^2}\mathbb{M}^0\textbf{y}_y \\
    &\Longrightarrow \textbf{e}_y(\Delta t)= \textbf{e}_y^0 - \frac{1}{\epsilon_0}\int_{0}^{\Delta t} y_y(t^\prime)\mathrm{d}t^\prime =\textbf{e}_y^0 - \frac{1}{\epsilon_0\Omega_\mathrm{ce}}[\textbf{y}_y^0\sin(\Omega_\textbf{ce}t)+\textbf{y}_x^0\cos(\Omega_\textbf{ce}t)-\textbf{y}_x^0],
\end{split}
\end{alignat}
\begin{alignat}{3}
    &\frac{\mathrm{d} \textbf{b}_x}{\mathrm{d}t}=0 &&\Longrightarrow\quad \textbf{b}_x(\Delta t) = \textbf{b}_x^0,\\
    &\frac{\mathrm{d} \textbf{b}_y}{\mathrm{d}t}=0 &&\Longrightarrow\quad \textbf{b}_y(\Delta t) = \textbf{b}_y^0,\\
     &\frac{\mathrm{d} \textbf{y}_x}{\mathrm{d}t}=\epsilon_0\Omega_\mathrm{pe}^2\Omega_\mathrm{ce}(\mathbb{M}^0)^{-1}\frac{1}{\epsilon_0\Omega_\mathrm{pe}^2}\mathbb{M}^0\textbf{y}_y &&\Longrightarrow\quad \textbf{y}_x(\Delta t) = \textbf{y}_x^0\cos(\Omega_\textbf{ce}\Delta t)+\textbf{y}_y^0\sin(\Omega_\textbf{ce}\Delta t),\\
     &\frac{\mathrm{d} \textbf{y}_y}{\mathrm{d}t}=-\epsilon_0\Omega_\mathrm{pe}^2\Omega_\mathrm{ce}(\mathbb{M}^0)^{-1}\frac{1}{\epsilon_0\Omega_\mathrm{pe}^2}\mathbb{M}^0\textbf{y}_x \quad &&\Longrightarrow\quad\textbf{y}_y(\Delta t) = \textbf{y}_y^0\cos(\Omega_\textbf{ce}\Delta t)-\textbf{y}_x^0\sin(\Omega_\textbf{ce}\Delta t),\\
     &\frac{\mathrm{d} \textbf{Z}}{\mathrm{d}t}=0 &&\Longrightarrow\quad \textbf{Z}(\Delta t) = \textbf{Z}^0,\\
     &\frac{\mathrm{d} \textbf{V}_x}{\mathrm{d}t}=0 &&\Longrightarrow\quad\textbf{V}_x(\Delta t) = \textbf{V}_x^0,\\
     &\frac{\mathrm{d} \textbf{V}_y}{\mathrm{d}t}=0 &&\Longrightarrow\quad \textbf{V}_y(\Delta t) = \textbf{V}_y^0,\\
     &\frac{\mathrm{d} \textbf{V}_z}{\mathrm{d}t}=0 &&\Longrightarrow\quad\textbf{V}_z(\Delta t) = \textbf{V}_z^0.
\end{alignat}
\end{subequations}
The corresponding integrator is denoted by $\textbf{u}(\Delta t)=\Phi_{\Delta t}^Y(\textbf{u}^0)$.\\ \\
\textbf{Problem 4.} For $t\in[0,\Delta t]$ and $\textbf{u}(t=0)=\textbf{u}^0$, we have
\begin{align}
\frac{\mathrm{d} \textbf{u}}{\mathrm{d}t}=\mathbb{J}(\textbf{u})\nabla_{\textbf{u}} H_x(\textbf{u})=\mathbb{J}(\mb{u})\nabla_{\textbf{u}}\left(\frac{m_\mr{e}}{2}\textbf{V}_x^\top\mathbb{W}\textbf{V}_x\right).
\end{align}
This can be solved analytically as
\begin{subequations}
\begin{alignat}{3}
    &\frac{\mathrm{d} \textbf{e}_x}{\mathrm{d}t}=-\frac{q_\text{e}}{\epsilon_0m_\text{e}}(\mathbb{M}^0)^{-1}\mathbb{Q}^0m_\text{e}\mathbb{W}\textbf{V}_x\quad &&\Longrightarrow\quad\textbf{e}_x(\Delta t) =\textbf{e}_y^0 - \Delta t\frac{q_\text{e}}{\epsilon_0}(\mathbb{M}^0)^{-1}\mathbb{Q}^0(\textbf{Z}^0)\mathbb{W}\textbf{V}_x^0,\\
    &\frac{\mathrm{d} \textbf{e}_y}{\mathrm{d}t}=0 &&\Longrightarrow\quad \textbf{e}_y(\Delta t) = \textbf{e}_y^0,\\
    &\frac{\mathrm{d} \textbf{b}_x}{\mathrm{d}t}=0 &&\Longrightarrow\quad \textbf{b}_x(\Delta t) = \textbf{b}_x^0,\\
    &\frac{\mathrm{d} \textbf{b}_y}{\mathrm{d}t}=0 &&\Longrightarrow\quad \textbf{b}_y(\Delta t) = \textbf{b}_y^0,
\end{alignat}
\begin{alignat}{3}
     &\frac{\mathrm{d} \textbf{y}_x}{\mathrm{d}t}=0 &&\Longrightarrow\quad \textbf{y}_x(\Delta t) = \textbf{y}_x^0,\\
     &\frac{\mathrm{d} \textbf{y}_y}{\mathrm{d}t}=0 &&\Longrightarrow\quad \textbf{y}_y(\Delta t) = \textbf{y}_y^0,\\
     &\frac{\mathrm{d} \textbf{Z}}{\mathrm{d}t}=0 &&\Longrightarrow\quad \textbf{Z}(\Delta t) = \textbf{Z}^0,\\
     &\frac{\mathrm{d} \textbf{V}_x}{\mathrm{d}t}=0 &&\Longrightarrow\quad \textbf{V}_x(\Delta t) = \textbf{V}_x^0,\\
     &\frac{\mathrm{d} \textbf{V}_y}{\mathrm{d}t}=-\frac{\Omega_\mathrm{ce}}{m}\mathbb{W}^{-1}m_\text{e}\mathbb{W}\textbf{V}_x &&\Longrightarrow\quad \textbf{V}_y(\Delta t) = \textbf{V}_y^0-\Delta t\Omega_\text{ce}\textbf{V}_x^0,\\
     &\frac{\mathrm{d} \textbf{V}_z}{\mathrm{d}t}=\frac{q_\text{e}}{m_\text{e}^2}\mathbb{B}_y\mathbb{W}^{-1}m_\text{e}\mathbb{W}\textbf{V}_x \quad &&\Longrightarrow\quad \textbf{V}_z(\Delta t) = \textbf{V}_z^0+\Delta t\frac{q_\text{e}}{m_\text{e}}\mathbb{B}_y(\textbf{Z}^0,\textbf{b}_y^0)\textbf{V}_x^0.
\end{alignat}
\end{subequations}
The corresponding integrator is denoted by $\textbf{u}(\Delta t)=\Phi_{\Delta t}^y(\textbf{u}^0)$.\\ \\
\textbf{Problem 5.} For $t\in[0,\Delta t]$ and $\textbf{u}(t=0)=\textbf{u}^0$, we have
\begin{align}
\frac{\mathrm{d}\textbf{u}}{\mathrm{d}t}=\mathbb{J}(\textbf{u})\nabla_{\textbf{u}} H_y(\textbf{u})=\mathbb{J}(\mb{u})\nabla_{\textbf{u}}\left(\frac{m_\mr{e}}{2}\textbf{V}_y^\top\mathbb{W}\textbf{V}_y\right).
\end{align}
This can be solved analytically as
\begin{subequations}
\begin{alignat}{3}
    &\frac{\mathrm{d} \textbf{e}_x}{\mathrm{d}t}=0 &&\Longrightarrow\quad \textbf{e}_x(\Delta t) = \textbf{e}_x^0,\\
    &\frac{\mathrm{d} \textbf{e}_y}{\mathrm{d}t}=-\frac{q_\text{e}}{\epsilon_0m_\text{e}}(\mathbb{M}^0)^{-1}\mathbb{Q}^0m_\text{e}\mathbb{W}\textbf{V}_y\quad &&\Longrightarrow\quad \textbf{e}_y(\Delta t) = \textbf{e}_y^0 - \Delta t\frac{q_\text{e}}{\epsilon_0}(\mathbb{M}^0)^{-1}\mathbb{Q}^0(\textbf{Z}^0)\mathbb{W}\textbf{V}_y^0,\\
    &\frac{\mathrm{d} \textbf{b}_x}{\mathrm{d}t}=0 &&\Longrightarrow\quad \textbf{b}_x(\Delta t) = \textbf{b}_x^0,\\
    &\frac{\mathrm{d} \textbf{b}_y}{\mathrm{d}t}=0 &&\Longrightarrow\quad \textbf{b}_y(\Delta t) = \textbf{b}_y^0,\\
     &\frac{\mathrm{d} \textbf{y}_x}{\mathrm{d}t}=0 &&\Longrightarrow\quad \textbf{y}_x(\Delta t) = \textbf{y}_x^0,\\
     &\frac{\mathrm{d} \textbf{y}_y}{\mathrm{d}t}=0 &&\Longrightarrow\quad \textbf{y}_y(\Delta t) = \textbf{y}_y^0,\\
     &\frac{\mathrm{d} \textbf{Z}}{\mathrm{d}t}=0 &&\Longrightarrow\quad \textbf{Z}(\Delta t) = \textbf{Z}^0,\\
     &\frac{\mathrm{d} \textbf{V}_x}{\mathrm{d}t}=\frac{\Omega_\mathrm{ce}}{m}\mathbb{W}^{-1}m_\text{e}\mathbb{W}\textbf{V}_y &&\Longrightarrow\quad \textbf{V}_x(\Delta t) = \textbf{V}_x^0+\Delta t\Omega_\mathrm{ce}\textbf{V}_y^0,\\
     &\frac{\mathrm{d} \textbf{V}_y}{\mathrm{d}t}=0 &&\Longrightarrow\quad \textbf{V}_y(\Delta t) = \textbf{V}_y^0,\\
     &\frac{\mathrm{d} \textbf{V}_z}{\mathrm{d}t}=-\frac{q_\text{e}}{m_\text{e}^2}\mathbb{B}_x\mathbb{W}^{-1}m_\text{e}\mathbb{W}\textbf{V}_y &&\Longrightarrow\quad \textbf{V}_z(\Delta t) = \textbf{V}_z^0-\Delta t\frac{q_\text{e}}{m_\text{e}}\mathbb{B}_x(\textbf{Z}^0,\textbf{b}_x^0)\textbf{V}_y^0.
\end{alignat}
\end{subequations}
The corresponding integrator is denoted by $\textbf{u}(\Delta t)=\Phi_{\Delta t}^y(\textbf{u}^0)$.\\ \\
\textbf{Problem 6.} For $t\in[0,\Delta t]$ and $\textbf{u}(t=0)=\textbf{u}^0$, we have
\begin{align}
\frac{\mathrm{d} \textbf{u}}{\mathrm{d}t}=\mathbb{J}(\textbf{u})\nabla_{\textbf{u}} H_z(\textbf{u})=\mathbb{J}(\mb{u})\nabla_{\textbf{u}}\left(\frac{m_\mr{e}}{2}\textbf{V}_z^\top\mathbb{W}\textbf{V}_z\right).
\end{align}
This can be solved analytically as
\begin{subequations}
\begin{alignat}{3}
    &\frac{\mathrm{d} \textbf{e}_x}{\mathrm{d}t}=0\quad &&\Longrightarrow\quad \textbf{e}_x(\Delta t) = \textbf{e}_x^0,\\
    &\frac{\mathrm{d} \textbf{e}_y}{\mathrm{d}t}=0 &&\Longrightarrow\quad \textbf{e}_y(\Delta t) = \textbf{e}_y^0,
\end{alignat}
\begin{alignat}{3}
    &\frac{\mathrm{d} \textbf{b}_x}{\mathrm{d}t}=0 &&\Longrightarrow\quad \textbf{b}_x(\Delta t) = \textbf{b}_x^0,\\
    &\frac{\mathrm{d} \textbf{b}_y}{\mathrm{d}t}=0 &&\Longrightarrow\quad \textbf{b}_y(\Delta t) = \textbf{b}_y^0,\\
     &\frac{\mathrm{d} \textbf{y}_x}{\mathrm{d}t}=0 &&\Longrightarrow\quad \textbf{y}_x(\Delta t) = \textbf{y}_x^0,\\
     &\frac{\mathrm{d} \textbf{y}_y}{\mathrm{d}t}=0 &&\Longrightarrow\quad \textbf{y}_y(\Delta t) = \textbf{y}_y^0,\\
     &\frac{\mathrm{d} \textbf{Z}}{\mathrm{d}t}=\frac{1}{m}\mathbb{W}^{-1}m_\text{e}\mathbb{W}\textbf{V}_z &&\Longrightarrow\quad \textbf{Z}(\Delta t) = \textbf{Z}^0+\Delta t\textbf{V}_z^0,\\
     &\frac{\mathrm{d} \textbf{V}_x}{\mathrm{d}t}=-\frac{q_\text{e}}{m_\text{e}^2}\mathbb{B}_y\mathbb{W}^{-1}m_\text{e}\mathbb{W}\textbf{V}_z \quad &&\Longrightarrow\quad \textbf{V}_x(\Delta t) = \textbf{V}_x^0-\frac{q_\text{e}}{m_\text{e}}\int_{0}^{\Delta t}\mathbb{B}_y(\textbf{Z}(s),\textbf{b}_y^0)\mathrm{d}s\textbf{V}_z^0\label{eq_lineIntegral_1}\\
     &\frac{\mathrm{d} \textbf{V}_y}{\mathrm{d}t}=\frac{q_\text{e}}{m_\text{e}^2}\mathbb{B}_x\mathbb{W}^{-1}m_\text{e}\mathbb{W}\textbf{V}_z &&\Longrightarrow\quad \textbf{V}_y(\Delta t) = \textbf{V}_y^0+\frac{q_\text{e}}{m_\text{e}}\int_{0}^{\Delta t}\mathbb{B}_x(\textbf{Z}(s),\textbf{b}_x^0)\mathrm{d}s\textbf{V}_z^0\label{eq_lineIntegral_2}\\
     &\frac{\mathrm{d} \textbf{V}_z}{\mathrm{d}t}=0 &&\Longrightarrow\quad \textbf{V}_z(\Delta t) = \textbf{V}_z^0.
\end{alignat}
\end{subequations}
The corresponding integrator is denoted by $\textbf{u}(\Delta t)=\Phi_{\Delta t}^z(\textbf{u}^0)$. Note that the integrals can be computed exactly along each particle trajectories as the basis functions are piecewise polynomials.

%%Vancouver style references.
\bibliographystyle{model1-num-names}
\bibliography{refs}
\end{document}

%%
