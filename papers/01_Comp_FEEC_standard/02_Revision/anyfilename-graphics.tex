%% This is file `jcomp-template.tex',
%% 
%% Copyright 2017 Elsevier Ltd
%% 
%% This file is part of the 'Elsarticle Bundle'.
%% ---------------------------------------------
%% 
%% It may be distributed under the conditions of the LaTeX Project Public
%% License, either version 1.2 of this license or (at your option) any
%% later version.  The latest version of this license is in
%%    http://www.latex-project.org/lppl.txt
%% and version 1.2 or later is part of all distributions of LaTeX
%% version 1999/12/01 or later.
%% 
%% The list of all files belonging to the 'Elsarticle Bundle' is
%% given in the file `manifest.txt'.
%% 
%% Template article for Elsevier's document class `elsarticle'
%% with harvard style bibliographic references
%%
%% $Id: jcomp-template.tex 100 2017-07-14 13:15:12Z rishi $
%%
%% Use the option review to obtain double line spacing
%\documentclass[times,review,preprint,authoryear]{elsarticle}

%% Use the options `twocolumn,final' to obtain the final layout
%% Use longtitle option to break abstract to multiple pages if overfull.
%% For Review pdf (With double line spacing)
%%\documentclass[times,twocolumn,review]{elsarticle}
%% For abstracts longer than one page.
%\documentclass[times,twocolumn,review,longtitle]{elsarticle}
%% For Review pdf without preprint line
%\documentclass[times,twocolumn,review,nopreprintline]{elsarticle}
%% Final pdf
\documentclass[fleqn,times,final]{elsarticle}
%%
%\documentclass[times,twocolumn,final,longtitle]{elsarticle}
%%


%% Stylefile to load JCOMP template
\usepackage{jcomp}
\usepackage{framed,multirow}

%% The amssymb package provides various useful mathematical symbols
\usepackage{amsthm}
\usepackage{amssymb}
\usepackage{latexsym}
\usepackage{subfigure}
\usepackage{amsmath}
\usepackage[overload]{empheq}
\usepackage{calc}
\usepackage{wrapfig}
\usepackage{enumitem}
\usepackage[toc,page]{appendix}
\usepackage{cite}
\usepackage{import}
\usepackage[active,tightpage,graphics]{preview}
\PreviewBorder=12pt\relax



% Commands
\newcommand{\pa}{\partial}
\newcommand{\mr}[1]{\mathrm{#1}}
\newcommand{\mb}[1]{\mathbf{#1}}
\newcommand{\vpar}{v_{\mr{th}\parallel}}
\newcommand{\vperp}{v_{\mr{th}\perp}}
\newcommand{\uproman}[1]{\uppercase\expandafter{\romannumeral#1}}
\newcommand{\blo}[9]{\frac{\pa(\mathbb{J}_{#1,#2})_{#3#4}}{\pa #5}(\mathbb{J}_{#6,#7})_{#8#9}}
\newcommand{\block}[5]{\frac{\pa\hat{\mathbb{J}}_{#1,#2}}{\pa #3}\hat{\mathbb{J}}_{#4,#5}}
\newtheorem*{theorem}{Theorem}
\newtheorem*{lemma}{Lemma}


% Following three lines are needed for this document.
% If you are not loading colors or url, then these are
% not required.
\usepackage{url}
\usepackage{xcolor}
\usepackage{pgf}
\definecolor{newcolor}{rgb}{.8,.349,.1}

\journal{Journal of Computational Physics}

\begin{document}

\verso{\textit{F. Holderied et al.}}

\begin{frontmatter}

%\title{Comparison of standard and structure-preserving finite element particle-in-cell methods for a four-dimensional electron hybrid plasma model}%

\title{Structure-preserving vs. standard particle-in-cell methods: the case of an electron hybrid model}

\author[1]{Florian \snm{Holderied}\corref{cor1}}
\cortext[cor1]{Corresponding author:}
\ead{florian.holderied@ipp.mpg.de}
\author[1,2]{Stefan \snm{Possanner}}
\author[1,2]{Ahmed \snm{Ratnani}}
\author[1]{Xin \snm{Wang}}


\address[1]{Max-Planck-Institut f\"{u}r Plasmaphysik, Boltzmannstraße 2, 85748 Garching, Deutschland}
\address[2]{Technische Universit\"{a}t M\"{u}nchen, Zentrum Mathematik, Boltzmannstraße 3,
85748 Garching, Deutschland}

%\received{1 May 2013}
%\finalform{10 May 2013}
%\accepted{13 May 2013}
%\availableonline{15 May 2013}
%\communicated{S. Sarkar}


\begin{abstract}
%%%
Two numerical methods both belonging to the class of finite element particle-in-cell methods have been applied to a four-dimensional (one dimension in real space and three dimensions in velocity space) hybrid plasma model for electrons in a stationary, neutralizing background of ions. Here, the term \textit{hybrid} means that (energetic) electrons with velocities close to the phase velocities of the model's characteristic waves are treated kinetically, whereas electrons that are much slower than the phase velocity are treated with fluid equations. The two developed numerical schemes are based on standard finite elements on the one hand and on structure-preserving geometric finite elements on the other hand. They have been tested and compared in the linear and in the nonlinear regime. We show that the structure-preserving algorithm leads to better results in both regimes. This can be related to the fact that the spatial discretization results in a large system of ordinary differential equations that exhibits a noncanonical Hamiltonian structure. For such systems special time integration schemes with good conservation properties can be applied.
%%%%
\end{abstract}

\begin{keyword}
%% MSC codes here, in the form: \MSC code \sep code
%% or \MSC[2008] code \sep code (2000 is the default)
%\MSC 41A05\sep 41A10\sep 65D05\sep 65D17
%% Keywords
\KWD Particle-in-cell method\sep Plasma simulation\sep ??
\end{keyword}

\end{frontmatter}

%\linenumbers

%% main text
 
\section{Introduction}
\label{sec_intro}
We present two numerical algorithms for a hybrid plasma model in order to demonstrate similarities and differences of standard finite element particle-in-cell (PIC) methods compared to structure-preserving finite element PIC methods. The latter use techniques from  \textit{finite element exterior calculus} (FEEC) \citep{Arnoldetal2006} and were applied by Kraus et al. \citep{Krausetal2017} on the full six-dimensional Vlasov-Maxwell model. By taking into account the geometric structure of the system of equations, FEEC methods exactly preserve conservation laws like energy or the two divergence constraints arising in electrodynamics, $\nabla\cdot\mb{E}=\rho/\epsilon_0$ and $\nabla\cdot\mb{B}=0$, on the semi-discrete level (discrete in space and continuous in time). Here, $\mb{E}=\mb{E}(\mb{x},t)$ and $\mb{B}=\mb{B}(\mb{x},t)$ denote the electric field and the magnetic flux density (or induction), which we will simply refer to as magnetic field. Furthermore,  $\rho=\rho(\mb{x},t)$ and $\epsilon_0$ are the charge density and the vacuum permittivity, respectively. As shown by Arnold, Falk \& Winther \citep{Arnoldetal2010}, the preservation of such invariants goes hand in hand with numerical stability. In this work, we shall apply these methods as well as classical finite element PIC methods on a hybrid plasma model which makes use of a combined fluid/kinetic description for different particle species to get a good balance between accuracy (kinetic models) and computational costs (fluid models). By comparing numerical results to the analytical theory in the linear stage and the conservation of energy in the nonlinear stage, our aim is to investigate whether there is a visible difference in the performances of the two algorithms. 

There are several plasma configurations which involve the interaction of an energetic plasma species with a lower temperature bulk plasma, e.g. fusion born alpha-particles interacting with the ambient plasma in nuclear fusion devices \citep{Chenetal2016} or the interaction of energetic electrons in the solar wind with planetary magnetoshperes. The model which is used in this work corresponds to the latter case and is thus applicable to plasma dynamics in the Earth's magnetospheres, for instance. It has been used intensively for the simulation \citep{Katohetal2007, Tao2014} of a special type of electromagnetic waves called \textit{Chorus waves} \citep{Tsurutaniatal1974, Burtisetal1976}, which are electromagnetic emissions whose frequency-time-spectrograms show a series of discrete elements with rising frequencies with respect to time. This phenomenon is also known as \textit{frequency chirping} \citep{Santoliketal2004}. An important condition for the excitation of Chorus waves is the injection of energetic electrons with an anisotropic velocity distribution with respect to the Earth's magnetic field into the magnetosphere, which then interact with Whistler mode waves propagating in the background plasma therein \citep{Thorne2010}.

This article is structured as follows. In Sec. \ref{sec_theory}, we introduce and discuss the considered electron hybrid model by starting with nonlinear fluid equations and subsequently performing a model reduction until we arrive at the simplified model which will be treated numerically. Besides this, we review and study the dispersion relation for waves with transverse disturbances propagating parallel to the external magnetic field in order to have a test case for the developed numerical algorithms. In Sec. \ref{sec_numerical_methods}, we successively apply the two above mentioned finite element PIC methods. For the case of structure-preserving geometric finite element PIC methods, we show, after having done the spatial discretization, that we end up with a noncanonical Hamiltonian system in time by proving the anti-symmetry and the Jacobi identity of the resulting system matrix. In Sec. \ref{sec_numerical_results}, we compare results obtained with the two developed algorithms before we summarize and conclude in Sec. \ref{sec_summary}. For completeness and clarity in the main text, the article contains three appendices. In \ref{sec_appendix1}, the system matrix of the noncanonical Hamiltonian system is displayed, while \ref{sec_appendix2} contains a table which is helpful for the understanding of the proof of the Jacobi identity. Finally, \ref{sec_appendix3} lists the time integrators for the geometric algorithm.

 
\section{Theoretical background}
\label{sec_theory}

\subsection{The full model}
\label{sec_model}
The considered model is a high-frequency plasma model, which means that wave frequencies $\omega$ are of the order of the electron cyclotron frequency $\Omega_\mr{ce}=q_\mr{e}|\mb{B}|/m_\mr{e}$, where $q_\mr{e}=-e$ and $m_\mr{e}$ are the electron charge and mass, respectively ($e$ is the elementary charge). Therefore, the plasma ions (denoted by the subscript i) cannot react on the fast fluctuations of the electromagnetic fields and are treated as a stationary, neutralizing background. Furthermore, we assume that the electron population consists mainly of cold electrons (denoted by the subscript c for ``cold''), which are in local thermal equilibrium and have negligible thermal effects (temperature $T_\mr{c}\approx0$). In this case, fluid equations without thermal forces are applicable. Moreover, we assume that there is a small amount of energetic electrons (denoted by the subscript h for ``hot'') for which we shall use a kinetic description with negligible collisionality, assuming that the average collision times are much larger than the considered time scales $\omega^{-1}$. Using the mass and momentum balance equation for the cold electrons, the Vlasov equation for the energetic electrons and Maxwell's equations for the self-consistent dynamics of the electromagnetic fields, the full set of equations in SI-units reads 
\begin{subequations}
\label{eq_model_full}
\begin{align}[left ={\text{cold fluid electrons}\hspace{2.4mm}\empheqlbrace}]
&\frac{\pa n_\mr{c}}{\pa t}+\nabla\cdot(n_\mr{c}\mb{u}_\mr{c})=0\,,\label{eq_model_full_cold_1}\\
&\frac{\partial \mb{u}_\mr{c}}{\partial t}+(\mb{u}_\mr{c}\cdot\nabla)\mb{u}_\mr{c}=\frac{q_\mr{e}}{m_\mr{e}}(\mb{E}+\mb{u}_\mr{c}\times\mb{B})\,,\label{eq_model_full_cold_2}\\
&\mb{j}_\mr{c}=q_\mr{e}n_\mr{c}\mb{u}_\mr{c},\label{eq_model_full_cold_3}
\end{align}
\begin{align}[left ={\text{hot kinetic electrons}\hspace{0.7mm}\empheqlbrace}]
&\frac{\pa f_\mr{h}}{\pa t}+\mb{v}\cdot\nabla f_\mr{h}+\frac{q_\mr{e}}{m_\mr{e}}(\mb{E}+\mb{v}\times\mb{B})\cdot\nabla_\mb{v}f_\mr{h}=0,\label{eq_model_full_hot_1}\\
&n_\mr{h}=\int f_\mr{h}\,\mr{d}^3\mb{v},\label{eq_model_full_hot_2}\\
&\mb{j}_\mr{h}=q_\mr{e}\int\mb{v}f_\mr{h}\,\mr{d}^3\mb{v}=q_\mr{e}n_\mr{h}\mb{u}_\mr{h},\label{eq_model_full_hot_3}
\end{align}
\begin{align}[left ={\text{Maxwell's equations}\empheqlbrace}]
&\frac{\pa \mb{B}}{\pa t}=-\nabla\times\mb{E},\label{eq_model_full_maxwell_1}\\
&\frac{1}{c^2}\frac{\pa \mb{E}}{\pa t}=\nabla\times\mb{B}-\mu_0(\mb{j}_\mr{c}+\mb{j}_\mr{h}),\label{eq_model_full_maxwell_2}\\
&\nabla\cdot\mb{E}=\frac{1}{\epsilon_0}[q_\mr{i}n_\mr{i}+q_\mr{e}(n_\mr{c}+n_\mr{h})],\label{eq_model_full_maxwell_3}\\
&\nabla\cdot\mb{B}=0,\label{eq_model_full_maxwell_4}
\end{align}
\end{subequations}
where, as stated above, the ions shall form a stationary background. This implies a constant number density $n_\mr{i}=n_\mr{i}(\mb{x})$ in time, i.e. $\pa n_\mr{i}/\pa t=0$, and a vanishing ion current $\mb{j}_\mr{i}=0$ for all times. Furthermore, $n_{\mr{c}/\mr{h}}=n_{\mr{c}/\mr{h}}(\mb{x},t)$ denote the number densities of the cold/hot electrons, $\mb{j}_{\mr{c}/\mr{h}}$ the current densities, $\mb{u}_{\mr{c}/\mr{h}}=\mb{u}_{\mr{c}/\mr{h}}(\mb{x},t)$ the mean velocities, respectively and $f_\mr{h}=f_\mr{h}(\mb{x},\mb{v},t)$ denotes the distribution function of the energetic electrons. Moreover, $c$ is the speed of light and $\mu_0$ the vaccuum permeability with $c^2\mu_0\epsilon_0=1$. Roughly speaking, the cold plasma approximation is valid as long as the thermal velocity of a particle species is much smaller than the phase velocity of the considered wave \citep{Brambilla1998}.

The model (\ref{eq_model_full}) possesses a noncanonical Hamiltonian structure which means that the dynamical equations can be derived from a Poisson bracket and a Hamiltonian representing the total energy of the system \citep{Tronci2010}. Thus, when we talk about structure-preserving numerical methods, we aim to perform a discretization that preserves this noncanonical Hamiltonian structure (see \citep{Krausetal2017}).

\subsection{Model reduction}
The model (\ref{eq_model_full}) can be reduced to an equivalent set of equations for the time evolution of the unknowns ($\mb{u}_\mr{c}$, $f_\mr{h}$, $\mb{E}$, $\mb{B}$) with the constraint that Gauss' law (\ref{eq_model_full_maxwell_3}) and the divergence constraint (\ref{eq_model_full_maxwell_4}) must be satisfied at the initial time $t=0$. The reduced model then takes the form
\begin{subequations}
\label{eq_model_reduced}
\begin{align}
&\frac{\partial \mb{u}_\mr{c}}{\partial t}+(\mb{u}_\mr{c}\cdot\nabla)\mb{u}_\mr{c}=\frac{q_\mr{e}}{m_\mr{e}}(\mb{E}+\mb{v}\times\mb{B}),\label{eq_model_reduced_1}\\
&\frac{\pa f_\mr{h}}{\pa t}+\mb{v}\cdot\nabla f_\mr{h}+\frac{q_\mr{e}}{m_\mr{e}}(\mb{E}+\mb{v}\times\mb{B})\cdot\nabla_\mb{v}f_\mr{h}=0,\label{eq_model_reduced_2}\\
&\frac{\pa \mb{B}}{\pa t}=-\nabla\times\mb{E},\label{eq_model_reduced_3}\\
&\frac{1}{c^2}\frac{\pa \mb{E}}{\pa t}=\nabla\times\mb{B}-\mu_0q_\mr{e}n_\mr{c}\mb{u}_\mr{c}-\mu_0q_\mr{e}\int\mb{v}f_\mr{h}\,\mr{d}^3\mb{v},\label{eq_model_reduced_4}
\end{align}
\end{subequations}
combined with the aforementioned constraints at $t=0$. The proof that the model (\ref{eq_model_reduced}) is indeed equivalent to the full model (\ref{eq_model_full}) consists of two steps: First, we note that the dynamics given by Faraday's law (\ref{eq_model_full_maxwell_1}) conserves the divergence constraint for the magnetic field,
\begin{align}
0=\nabla\cdot\left(\frac{\pa \mb{B}}{\pa t}+\nabla\times\mb{E}\right)=\frac{\pa}{\pa t}(\nabla\cdot\mb{B}),
\end{align}
i.e. the divergence constraint remains satisfied at later times $t>0$ provided that it was satisfied at the initial time $t=0$. Likewise, the mass continuity equation for the cold fluid electrons (\ref{eq_model_full_cold_1}) is automatically satisfied by Amp\'{e}re's law (\ref{eq_model_full_maxwell_2}) by assuming that the cold electron number density $n_\mr{c}$ can be reconstructed from the divergence of the electric field (\ref{eq_model_full_maxwell_3}) at any time $t\geq0$:
\begin{align}
\begin{split}
0&=\nabla\cdot\left(\frac{1}{c^2}\frac{\pa\mb{E}}{\pa t}-\nabla\times\mb{B}+\mu_0q_\mr{e}n_\mr{c}\mb{u}_\mr{c}+\mu_0q_\mr{e}\int\mb{v}f_\mr{h}\,\mr{d}^3\mb{v}\right)\\
&\underset{\underset{\text{(\ref{eq_model_full_maxwell_3})}}{\uparrow}}{=}\mu_0q_\mr{e}\frac{\pa}{\pa t}(n_\mr{c}+n_\mr{h})+\mu_0q_\mr{e}\nabla\cdot(n_\mr{c}\mb{u}_\mr{c})+\mu_0q_\mr{e}\int\mb{v}\cdot\nabla f_\mr{h}\,\mr{d}^3\mb{v}\\
&\underset{\underset{\text{(\ref{eq_model_full_hot_1}), (\ref{eq_model_full_hot_2})}}{\uparrow}}{=}\mu_0q_\mr{e}\underbrace{\left[\frac{\pa n_\mr{c}}{\pa t}+\nabla\cdot(n_\mr{c}\mb{u}_\mr{c})\right]}_{\text{cont. eq. (\ref{eq_model_full_cold_1})}}-\frac{\mu_0q_\mr{e}^2}{m_\mr{e}}\underbrace{\int(\mb{E}+\mb{v}\times\mb{B})\cdot\nabla_\mb{v}f_\mr{h}\,\mr{d}^3\mb{v}}_{=0}.
\end{split}
\end{align}
From the second to the third line we first used the Vlasov equation (\ref{eq_model_full_hot_1}) to replace the $\mb{v}\cdot\nabla f_\mr{h}$ term in the integral and subsequently used the definition of the hot electron number density (\ref{eq_model_full_hot_2}). The disappearance of the integral in the third line can easily be verified by partial integration in $\mb{v}$ and noting that $f_\mr{h}\rightarrow0$ for $v\rightarrow\infty$. Consequently, the divergence of Amp\'{e}re's law reduces to the the mass continuity equation for the fluid electrons (\ref{eq_model_full_cold_1}) which is therefore satisfied automatically. In summary, we showed that solutions ($\mb{u}_\mr{c}$, $f_\mr{h}$, $\mb{E}$, $\mb{B}$) of the reduced model (\ref{eq_model_reduced}) with compatible initial conditions are indeed solutions ($n_\mr{c}$, $\mb{u}_\mr{c}$, $f_\mr{h}$, $\mb{E}$, $\mb{B}$) of the full model (\ref{eq_model_full}).

The model can further be simplified by considering waves as small-amplitude perturbations (denoted by tildes) about a given time-independent equilibrium state (denoted by the subscript ``0''). In this case, we can write
\begin{subequations}
\label{eq_linearization}
\begin{align}
&n_\mr{c}(\mb{x},t)=n_{\mr{c}0}(\mb{x})+\tilde{n}_\mr{c}(\mb{x},t),\label{eq_linearization_1}\\
&\mb{u}_\mr{c}(\mb{x},t)=\tilde{\mb{u}}_\mr{c}(\mb{x},t),\label{eq_linearization_2}\\
&\mb{B}(\mb{x},t)=\mb{B}_0(\mb{x})+\tilde{\mb{B}}(\mb{x},t),\label{eq_linearization_3}\\
&\mb{E}(\mb{x},t)=\tilde{\mb{E}}(\mb{x},t),\label{eq_linearization_4}\\
&f_\mr{h}(\mb{x},\mb{v},t)=f_\mr{h}^0(\mb{x},\mb{v})+\tilde{f}_\mr{h}(\mb{x},\mb{v},t),\label{eq_linearization_5}
\end{align}  
\end{subequations}
where we assumed that there is no background electric field and no equilibrium plasma flow (which also means that there is no cold equilibrium current $\mb{j}_{\mr{c}0}$ and thus $\nabla\times\mb{B}_0=-\mu_0\mb{j}_{\mr{h}0}$ must be satisfied). In what follows, we neglect nonlinear terms for the fluid quantities, e.g. the perturbed cold current density $\tilde{\mb{j}}_{\mr{c}}=q_\mr{e}n_{\mr{c}0}\tilde{\mb{u}}_\mr{c}$. This leads to a modified momentum balance equation by first linearizing (\ref{eq_model_reduced_1}) and subsequently expressing $\tilde{\mb{u}}_\mr{c}$ in terms of $\tilde{\mb{j}}_\mr{c}$ according to $\tilde{\mb{u}}_\mr{c}=\tilde{\mb{j}}_\mb{c}/q_\mr{e}n_{\mr{c}0}$. However, we keep all nonlinearities in the Vlasov equation for the full distribution function $f_\mr{h}$ in order to apply classical particle-in-cell methods which exploit the fact that the distribution function is constant along its characteristics in a Lagrangian frame, i.e. $\mr{d}/\mr{d}t f_\mr{h}(\mb{x}(t),\mb{v}(t),t)=0$. Finally, this leads to the model
\begin{subequations}
\label{eq_model_linearized}
\begin{align}
&\frac{\partial\tilde{\mb{j}}_\mr{c}}{\pa t}=\epsilon_0\Omega_\mr{pe}^2\tilde{\mb{E}}+\tilde{\mb{j}}_\mr{c}\times\mb{\Omega}_\mr{ce},\label{eq_model_linearized_1}\\
&\frac{\pa f_\mr{h}}{\pa t}+\mb{v}\cdot\nabla f_\mr{h}+\frac{q_\mr{e}}{m_\mr{e}}(\mb{E}+\mb{v}\times\mb{B})\cdot\nabla_\mb{v}f_\mr{h}=0,\label{eq_model_linearized_2}\\
&\frac{\pa \tilde{\mb{B}}}{\pa t}=-\nabla\times\tilde{\mb{E}},\label{eq_model_linearized_3}\\
&\frac{1}{c^2}\frac{\pa \tilde{\mb{E}}}{\pa t}=\nabla\times\tilde{\mb{B}}-\mu_0\tilde{\mb{j}}_\mr{c}-\mu_0q_\mr{e}\int\mb{v}\tilde{f}_\mr{h}\,\mr{d}^3\mb{v},\label{eq_model_linearized_4}
\end{align}
\end{subequations}
where we introduced the spatially dependent cold electron plasma frequency $\Omega_\mr{pe}^2(\mb{x})=e^2n_{\mr{c}0}(\mb{x})/\epsilon_0m_\mr{e}$, the oriented electron cyclotron frequency $\mb{\Omega}_\mr{ce}(\mb{x})=q_\mr{e}\mb{B}_0(\mb{x})/m_\mr{e}$ corresponding to the background magnetic field $\mb{B}_0$. An important property of the linearized model (\ref{eq_model_linearized}) is that its dynamics conserves the total energy 
\begin{align}
\epsilon:=\underbrace{\frac{\epsilon_0}{2}\int_\Omega\tilde{\mb{E}}^2\,\mr{d}^3\mb{x}}_{=:\epsilon_E}+\underbrace{\frac{1}{2\mu_0}\int
_\Omega\tilde{\mb{B}}^2\,\mr{d}^3\mb{x}}_{=:\epsilon_B}+\underbrace{\frac{1}{2\epsilon_0}\int_\Omega\frac{1}{\Omega_\mr{pe}^2}\tilde{\mb{j}}_\mr{c}^2\,\mr{d}^3\mb{x}}_{:=\epsilon_\mr{c}}+\underbrace{\frac{m_\mr{e}}{2}\int_\Omega\int |\mb{v}|^2f_\mr{h}\,\mr{d}^3\mb{v}\mr{d}^3\mb{x}}_{\epsilon_\mr{h}}\label{eq_total_energy}
\end{align}
in the domain $\Omega=\mathbb{R}^3$, which is the sum of the electric field energy $\epsilon_E$, the magnetic field energy $\epsilon_B$, the kinetic energy of the cold electrons $\epsilon_\mr{c}$ and the kinetic energy of the hot electrons $\epsilon_\mr{h}$, respectively. It is relatively straightforward to proof this property by computing $\mr{d}\epsilon/\mr{d}t$, using the dynamical equations (\ref{eq_model_linearized}) to replace the occurring partial time derivatives, noting that all quantities vanish at infinity (or assuming a periodic domain) and then summing everything up to show that $\mr{d}\epsilon/\mr{d}t=0$. We will use this energy conservation property later as a criterion for the performances of the developed numerical schemes.

\subsection{Linear dispersion relation}
\label{sec_dispersion}
We study the linear dispersion relation of the model (\ref{eq_model_linearized}) for the case of wave propagation parallel to a uniform magnetic field $\mb{B}_0=B_0\mb{e}_z$ ($\Rightarrow\Omega_\mr{ce}(\mb{x})=\Omega_\mr{ce}=const.$), i.e. the wave vector $\mb{k}=k\mb{e}_z$, and a spatially uniform plasma in the equilibrium state. The latter implies a constant cold electron plasma frequency $\Omega_\mr{pe}(\mb{x})=\Omega_\mr{pe}=const.$ and a uniform hot electron equilibrium distribution function $f_\mr{h}^0=f_\mr{h}^0(\mb{v})$. In order to obtain a linear dispersion relation, we now linearize the Vlasov equation as well to get the fully linearized model
\begin{subequations}
\label{eq_model_fullylinearized}
\begin{align}
&\frac{\partial\mb{j}_\mr{c}}{\pa t}=\epsilon_0\Omega_\mr{pe}^2\mb{E}+\Omega_\mr{ce}\mb{j}_\mr{c}\times\mb{e}_z,\label{eq_model_fullylinearized_1}\\
&\frac{\pa f_\mr{h}}{\pa t}+\mb{v}\cdot\nabla f_\mr{h}+\Omega_\mr{ce}(\mb{v}\times\mb{e}_z)\cdot\nabla_\mb{v}f_\mr{h}=-\frac{q_\mr{e}}{m_\mr{e}}(\mb{E}+\mb{v}\times\mb{B})\cdot\nabla_\mb{v}f_\mr{h}^0,\label{eq_model_fullylinearized_2}\\
&\frac{\pa \mb{B}}{\pa t}=-\nabla\times\mb{E},\label{eq_model_fullylinearized_3}\\
&\frac{1}{c^2}\frac{\pa \mb{E}}{\pa t}=\nabla\times\mb{B}-\mu_0\mb{j}_\mr{c}-\mu_0q_\mr{e}\int\mb{v}f_\mr{h}\,\mr{d}^3\mb{v},\label{eq_model_fullylinearized_4}
\end{align}
\end{subequations}
where we performed a relabeling ($\tilde{\mb{B}}\rightarrow\mb{B}$, $\tilde{f}_\mr{h}\rightarrow f_\mr{h}$, $\ldots$) for reasons of clarity. Note that $\Omega_\mr{ce}<0$ for electrons. In the above stated case of parallel wave propagation, the problem becomes effectively one-dimensional in space, which is why we can set $\nabla=\mb{e}_z\pa/\pa z$ in (\ref{eq_model_fullylinearized}). By looking for plane wave solutions $\sim \exp[i(kz-\omega t)]$ for all quantities and solving the linearized Vlasov equation in velocity space with the method of characteristics (see \citep{Brambilla1998}, pp. 93ff.), one ends up with three linear independent solutions: One of these solutions corresponds to electrostatic waves (longitudinal waves with perturbations parallel to the background magnetic field) which we do not consider further. 
\begin{figure}[!t]
\centering
\includegraphics[scale=1]{01_Figures/Real_freq.pdf}
\includegraphics[scale=1]{01_Figures/Growth_rates.pdf}
%%% Creator: Matplotlib, PGF backend
%%
%% To include the figure in your LaTeX document, write
%%   \input{<filename>.pgf}
%%
%% Make sure the required packages are loaded in your preamble
%%   \usepackage{pgf}
%%
%% Figures using additional raster images can only be included by \input if
%% they are in the same directory as the main LaTeX file. For loading figures
%% from other directories you can use the `import` package
%%   \usepackage{import}
%% and then include the figures with
%%   \import{<path to file>}{<filename>.pgf}
%%
%% Matplotlib used the following preamble
%%   \usepackage{fontspec}
%%   \setmainfont{DejaVu Serif}
%%   \setsansfont{DejaVu Sans}
%%   \setmonofont{DejaVu Sans Mono}
%%
\begingroup%
\makeatletter%
\begin{pgfpicture}%
\pgfpathrectangle{\pgfpointorigin}{\pgfqpoint{2.934722in}{2.188841in}}%
\pgfusepath{use as bounding box, clip}%
\begin{pgfscope}%
\pgfsetbuttcap%
\pgfsetmiterjoin%
\definecolor{currentfill}{rgb}{1.000000,1.000000,1.000000}%
\pgfsetfillcolor{currentfill}%
\pgfsetlinewidth{0.000000pt}%
\definecolor{currentstroke}{rgb}{1.000000,1.000000,1.000000}%
\pgfsetstrokecolor{currentstroke}%
\pgfsetdash{}{0pt}%
\pgfpathmoveto{\pgfqpoint{0.000000in}{0.000000in}}%
\pgfpathlineto{\pgfqpoint{2.934722in}{0.000000in}}%
\pgfpathlineto{\pgfqpoint{2.934722in}{2.188841in}}%
\pgfpathlineto{\pgfqpoint{0.000000in}{2.188841in}}%
\pgfpathclose%
\pgfusepath{fill}%
\end{pgfscope}%
\begin{pgfscope}%
\pgfsetbuttcap%
\pgfsetmiterjoin%
\definecolor{currentfill}{rgb}{1.000000,1.000000,1.000000}%
\pgfsetfillcolor{currentfill}%
\pgfsetlinewidth{0.000000pt}%
\definecolor{currentstroke}{rgb}{0.000000,0.000000,0.000000}%
\pgfsetstrokecolor{currentstroke}%
\pgfsetstrokeopacity{0.000000}%
\pgfsetdash{}{0pt}%
\pgfpathmoveto{\pgfqpoint{0.461111in}{0.526079in}}%
\pgfpathlineto{\pgfqpoint{2.786111in}{0.526079in}}%
\pgfpathlineto{\pgfqpoint{2.786111in}{2.036079in}}%
\pgfpathlineto{\pgfqpoint{0.461111in}{2.036079in}}%
\pgfpathclose%
\pgfusepath{fill}%
\end{pgfscope}%
\begin{pgfscope}%
\pgfsetbuttcap%
\pgfsetroundjoin%
\definecolor{currentfill}{rgb}{0.000000,0.000000,0.000000}%
\pgfsetfillcolor{currentfill}%
\pgfsetlinewidth{0.803000pt}%
\definecolor{currentstroke}{rgb}{0.000000,0.000000,0.000000}%
\pgfsetstrokecolor{currentstroke}%
\pgfsetdash{}{0pt}%
\pgfsys@defobject{currentmarker}{\pgfqpoint{0.000000in}{-0.048611in}}{\pgfqpoint{0.000000in}{0.000000in}}{%
\pgfpathmoveto{\pgfqpoint{0.000000in}{0.000000in}}%
\pgfpathlineto{\pgfqpoint{0.000000in}{-0.048611in}}%
\pgfusepath{stroke,fill}%
}%
\begin{pgfscope}%
\pgfsys@transformshift{0.461111in}{0.526079in}%
\pgfsys@useobject{currentmarker}{}%
\end{pgfscope}%
\end{pgfscope}%
\begin{pgfscope}%
\pgftext[x=0.461111in,y=0.428857in,,top]{\rmfamily\fontsize{10.000000}{12.000000}\selectfont \(\displaystyle 0\)}%
\end{pgfscope}%
\begin{pgfscope}%
\pgfsetbuttcap%
\pgfsetroundjoin%
\definecolor{currentfill}{rgb}{0.000000,0.000000,0.000000}%
\pgfsetfillcolor{currentfill}%
\pgfsetlinewidth{0.803000pt}%
\definecolor{currentstroke}{rgb}{0.000000,0.000000,0.000000}%
\pgfsetstrokecolor{currentstroke}%
\pgfsetdash{}{0pt}%
\pgfsys@defobject{currentmarker}{\pgfqpoint{0.000000in}{-0.048611in}}{\pgfqpoint{0.000000in}{0.000000in}}{%
\pgfpathmoveto{\pgfqpoint{0.000000in}{0.000000in}}%
\pgfpathlineto{\pgfqpoint{0.000000in}{-0.048611in}}%
\pgfusepath{stroke,fill}%
}%
\begin{pgfscope}%
\pgfsys@transformshift{1.042361in}{0.526079in}%
\pgfsys@useobject{currentmarker}{}%
\end{pgfscope}%
\end{pgfscope}%
\begin{pgfscope}%
\pgftext[x=1.042361in,y=0.428857in,,top]{\rmfamily\fontsize{10.000000}{12.000000}\selectfont \(\displaystyle 2\)}%
\end{pgfscope}%
\begin{pgfscope}%
\pgfsetbuttcap%
\pgfsetroundjoin%
\definecolor{currentfill}{rgb}{0.000000,0.000000,0.000000}%
\pgfsetfillcolor{currentfill}%
\pgfsetlinewidth{0.803000pt}%
\definecolor{currentstroke}{rgb}{0.000000,0.000000,0.000000}%
\pgfsetstrokecolor{currentstroke}%
\pgfsetdash{}{0pt}%
\pgfsys@defobject{currentmarker}{\pgfqpoint{0.000000in}{-0.048611in}}{\pgfqpoint{0.000000in}{0.000000in}}{%
\pgfpathmoveto{\pgfqpoint{0.000000in}{0.000000in}}%
\pgfpathlineto{\pgfqpoint{0.000000in}{-0.048611in}}%
\pgfusepath{stroke,fill}%
}%
\begin{pgfscope}%
\pgfsys@transformshift{1.623611in}{0.526079in}%
\pgfsys@useobject{currentmarker}{}%
\end{pgfscope}%
\end{pgfscope}%
\begin{pgfscope}%
\pgftext[x=1.623611in,y=0.428857in,,top]{\rmfamily\fontsize{10.000000}{12.000000}\selectfont \(\displaystyle 4\)}%
\end{pgfscope}%
\begin{pgfscope}%
\pgfsetbuttcap%
\pgfsetroundjoin%
\definecolor{currentfill}{rgb}{0.000000,0.000000,0.000000}%
\pgfsetfillcolor{currentfill}%
\pgfsetlinewidth{0.803000pt}%
\definecolor{currentstroke}{rgb}{0.000000,0.000000,0.000000}%
\pgfsetstrokecolor{currentstroke}%
\pgfsetdash{}{0pt}%
\pgfsys@defobject{currentmarker}{\pgfqpoint{0.000000in}{-0.048611in}}{\pgfqpoint{0.000000in}{0.000000in}}{%
\pgfpathmoveto{\pgfqpoint{0.000000in}{0.000000in}}%
\pgfpathlineto{\pgfqpoint{0.000000in}{-0.048611in}}%
\pgfusepath{stroke,fill}%
}%
\begin{pgfscope}%
\pgfsys@transformshift{2.204861in}{0.526079in}%
\pgfsys@useobject{currentmarker}{}%
\end{pgfscope}%
\end{pgfscope}%
\begin{pgfscope}%
\pgftext[x=2.204861in,y=0.428857in,,top]{\rmfamily\fontsize{10.000000}{12.000000}\selectfont \(\displaystyle 6\)}%
\end{pgfscope}%
\begin{pgfscope}%
\pgfsetbuttcap%
\pgfsetroundjoin%
\definecolor{currentfill}{rgb}{0.000000,0.000000,0.000000}%
\pgfsetfillcolor{currentfill}%
\pgfsetlinewidth{0.803000pt}%
\definecolor{currentstroke}{rgb}{0.000000,0.000000,0.000000}%
\pgfsetstrokecolor{currentstroke}%
\pgfsetdash{}{0pt}%
\pgfsys@defobject{currentmarker}{\pgfqpoint{0.000000in}{-0.048611in}}{\pgfqpoint{0.000000in}{0.000000in}}{%
\pgfpathmoveto{\pgfqpoint{0.000000in}{0.000000in}}%
\pgfpathlineto{\pgfqpoint{0.000000in}{-0.048611in}}%
\pgfusepath{stroke,fill}%
}%
\begin{pgfscope}%
\pgfsys@transformshift{2.786111in}{0.526079in}%
\pgfsys@useobject{currentmarker}{}%
\end{pgfscope}%
\end{pgfscope}%
\begin{pgfscope}%
\pgftext[x=2.786111in,y=0.428857in,,top]{\rmfamily\fontsize{10.000000}{12.000000}\selectfont \(\displaystyle 8\)}%
\end{pgfscope}%
\begin{pgfscope}%
\pgftext[x=1.623611in,y=0.238889in,,top]{\rmfamily\fontsize{10.000000}{12.000000}\selectfont \(\displaystyle kc / |\Omega_\mathrm{ce}|\)}%
\end{pgfscope}%
\begin{pgfscope}%
\pgfsetbuttcap%
\pgfsetroundjoin%
\definecolor{currentfill}{rgb}{0.000000,0.000000,0.000000}%
\pgfsetfillcolor{currentfill}%
\pgfsetlinewidth{0.803000pt}%
\definecolor{currentstroke}{rgb}{0.000000,0.000000,0.000000}%
\pgfsetstrokecolor{currentstroke}%
\pgfsetdash{}{0pt}%
\pgfsys@defobject{currentmarker}{\pgfqpoint{-0.048611in}{0.000000in}}{\pgfqpoint{0.000000in}{0.000000in}}{%
\pgfpathmoveto{\pgfqpoint{0.000000in}{0.000000in}}%
\pgfpathlineto{\pgfqpoint{-0.048611in}{0.000000in}}%
\pgfusepath{stroke,fill}%
}%
\begin{pgfscope}%
\pgfsys@transformshift{0.461111in}{0.526079in}%
\pgfsys@useobject{currentmarker}{}%
\end{pgfscope}%
\end{pgfscope}%
\begin{pgfscope}%
\pgftext[x=0.294444in,y=0.473318in,left,base]{\rmfamily\fontsize{10.000000}{12.000000}\selectfont \(\displaystyle 0\)}%
\end{pgfscope}%
\begin{pgfscope}%
\pgfsetbuttcap%
\pgfsetroundjoin%
\definecolor{currentfill}{rgb}{0.000000,0.000000,0.000000}%
\pgfsetfillcolor{currentfill}%
\pgfsetlinewidth{0.803000pt}%
\definecolor{currentstroke}{rgb}{0.000000,0.000000,0.000000}%
\pgfsetstrokecolor{currentstroke}%
\pgfsetdash{}{0pt}%
\pgfsys@defobject{currentmarker}{\pgfqpoint{-0.048611in}{0.000000in}}{\pgfqpoint{0.000000in}{0.000000in}}{%
\pgfpathmoveto{\pgfqpoint{0.000000in}{0.000000in}}%
\pgfpathlineto{\pgfqpoint{-0.048611in}{0.000000in}}%
\pgfusepath{stroke,fill}%
}%
\begin{pgfscope}%
\pgfsys@transformshift{0.461111in}{0.903579in}%
\pgfsys@useobject{currentmarker}{}%
\end{pgfscope}%
\end{pgfscope}%
\begin{pgfscope}%
\pgftext[x=0.294444in,y=0.850818in,left,base]{\rmfamily\fontsize{10.000000}{12.000000}\selectfont \(\displaystyle 1\)}%
\end{pgfscope}%
\begin{pgfscope}%
\pgfsetbuttcap%
\pgfsetroundjoin%
\definecolor{currentfill}{rgb}{0.000000,0.000000,0.000000}%
\pgfsetfillcolor{currentfill}%
\pgfsetlinewidth{0.803000pt}%
\definecolor{currentstroke}{rgb}{0.000000,0.000000,0.000000}%
\pgfsetstrokecolor{currentstroke}%
\pgfsetdash{}{0pt}%
\pgfsys@defobject{currentmarker}{\pgfqpoint{-0.048611in}{0.000000in}}{\pgfqpoint{0.000000in}{0.000000in}}{%
\pgfpathmoveto{\pgfqpoint{0.000000in}{0.000000in}}%
\pgfpathlineto{\pgfqpoint{-0.048611in}{0.000000in}}%
\pgfusepath{stroke,fill}%
}%
\begin{pgfscope}%
\pgfsys@transformshift{0.461111in}{1.281079in}%
\pgfsys@useobject{currentmarker}{}%
\end{pgfscope}%
\end{pgfscope}%
\begin{pgfscope}%
\pgftext[x=0.294444in,y=1.228318in,left,base]{\rmfamily\fontsize{10.000000}{12.000000}\selectfont \(\displaystyle 2\)}%
\end{pgfscope}%
\begin{pgfscope}%
\pgfsetbuttcap%
\pgfsetroundjoin%
\definecolor{currentfill}{rgb}{0.000000,0.000000,0.000000}%
\pgfsetfillcolor{currentfill}%
\pgfsetlinewidth{0.803000pt}%
\definecolor{currentstroke}{rgb}{0.000000,0.000000,0.000000}%
\pgfsetstrokecolor{currentstroke}%
\pgfsetdash{}{0pt}%
\pgfsys@defobject{currentmarker}{\pgfqpoint{-0.048611in}{0.000000in}}{\pgfqpoint{0.000000in}{0.000000in}}{%
\pgfpathmoveto{\pgfqpoint{0.000000in}{0.000000in}}%
\pgfpathlineto{\pgfqpoint{-0.048611in}{0.000000in}}%
\pgfusepath{stroke,fill}%
}%
\begin{pgfscope}%
\pgfsys@transformshift{0.461111in}{1.658579in}%
\pgfsys@useobject{currentmarker}{}%
\end{pgfscope}%
\end{pgfscope}%
\begin{pgfscope}%
\pgftext[x=0.294444in,y=1.605818in,left,base]{\rmfamily\fontsize{10.000000}{12.000000}\selectfont \(\displaystyle 3\)}%
\end{pgfscope}%
\begin{pgfscope}%
\pgfsetbuttcap%
\pgfsetroundjoin%
\definecolor{currentfill}{rgb}{0.000000,0.000000,0.000000}%
\pgfsetfillcolor{currentfill}%
\pgfsetlinewidth{0.803000pt}%
\definecolor{currentstroke}{rgb}{0.000000,0.000000,0.000000}%
\pgfsetstrokecolor{currentstroke}%
\pgfsetdash{}{0pt}%
\pgfsys@defobject{currentmarker}{\pgfqpoint{-0.048611in}{0.000000in}}{\pgfqpoint{0.000000in}{0.000000in}}{%
\pgfpathmoveto{\pgfqpoint{0.000000in}{0.000000in}}%
\pgfpathlineto{\pgfqpoint{-0.048611in}{0.000000in}}%
\pgfusepath{stroke,fill}%
}%
\begin{pgfscope}%
\pgfsys@transformshift{0.461111in}{2.036079in}%
\pgfsys@useobject{currentmarker}{}%
\end{pgfscope}%
\end{pgfscope}%
\begin{pgfscope}%
\pgftext[x=0.294444in,y=1.983318in,left,base]{\rmfamily\fontsize{10.000000}{12.000000}\selectfont \(\displaystyle 4\)}%
\end{pgfscope}%
\begin{pgfscope}%
\pgftext[x=0.238889in,y=1.281079in,,bottom,rotate=90.000000]{\rmfamily\fontsize{10.000000}{12.000000}\selectfont \(\displaystyle \omega_\mathrm{r} / |\Omega_\mathrm{ce}|\)}%
\end{pgfscope}%
\begin{pgfscope}%
\pgfpathrectangle{\pgfqpoint{0.461111in}{0.526079in}}{\pgfqpoint{2.325000in}{1.510000in}} %
\pgfusepath{clip}%
\pgfsetrectcap%
\pgfsetroundjoin%
\pgfsetlinewidth{1.003750pt}%
\definecolor{currentstroke}{rgb}{1.000000,0.549020,0.000000}%
\pgfsetstrokecolor{currentstroke}%
\pgfsetdash{}{0pt}%
\pgfpathmoveto{\pgfqpoint{0.577361in}{1.503843in}}%
\pgfpathlineto{\pgfqpoint{0.599672in}{1.507607in}}%
\pgfpathlineto{\pgfqpoint{0.621983in}{1.512030in}}%
\pgfpathlineto{\pgfqpoint{0.644293in}{1.517111in}}%
\pgfpathlineto{\pgfqpoint{0.666604in}{1.522848in}}%
\pgfpathlineto{\pgfqpoint{0.688914in}{1.529241in}}%
\pgfpathlineto{\pgfqpoint{0.711225in}{1.536288in}}%
\pgfpathlineto{\pgfqpoint{0.733536in}{1.543985in}}%
\pgfpathlineto{\pgfqpoint{0.755846in}{1.552329in}}%
\pgfpathlineto{\pgfqpoint{0.778157in}{1.561316in}}%
\pgfpathlineto{\pgfqpoint{0.800467in}{1.570940in}}%
\pgfpathlineto{\pgfqpoint{0.822778in}{1.581196in}}%
\pgfpathlineto{\pgfqpoint{0.845089in}{1.592077in}}%
\pgfpathlineto{\pgfqpoint{0.867399in}{1.603574in}}%
\pgfpathlineto{\pgfqpoint{0.889710in}{1.615677in}}%
\pgfpathlineto{\pgfqpoint{0.912020in}{1.628378in}}%
\pgfpathlineto{\pgfqpoint{0.934331in}{1.641665in}}%
\pgfpathlineto{\pgfqpoint{0.956642in}{1.655525in}}%
\pgfpathlineto{\pgfqpoint{0.978952in}{1.669946in}}%
\pgfpathlineto{\pgfqpoint{1.001263in}{1.684914in}}%
\pgfpathlineto{\pgfqpoint{1.023574in}{1.700415in}}%
\pgfpathlineto{\pgfqpoint{1.045884in}{1.716435in}}%
\pgfpathlineto{\pgfqpoint{1.068195in}{1.732957in}}%
\pgfpathlineto{\pgfqpoint{1.090505in}{1.749965in}}%
\pgfpathlineto{\pgfqpoint{1.112816in}{1.767445in}}%
\pgfpathlineto{\pgfqpoint{1.135127in}{1.785379in}}%
\pgfpathlineto{\pgfqpoint{1.157437in}{1.803752in}}%
\pgfpathlineto{\pgfqpoint{1.179748in}{1.822547in}}%
\pgfpathlineto{\pgfqpoint{1.202058in}{1.841748in}}%
\pgfpathlineto{\pgfqpoint{1.224369in}{1.861340in}}%
\pgfpathlineto{\pgfqpoint{1.246680in}{1.881305in}}%
\pgfpathlineto{\pgfqpoint{1.268990in}{1.901629in}}%
\pgfpathlineto{\pgfqpoint{1.291301in}{1.922297in}}%
\pgfpathlineto{\pgfqpoint{1.313611in}{1.943294in}}%
\pgfpathlineto{\pgfqpoint{1.335922in}{1.964606in}}%
\pgfpathlineto{\pgfqpoint{1.358233in}{1.986219in}}%
\pgfpathlineto{\pgfqpoint{1.380543in}{2.008119in}}%
\pgfpathlineto{\pgfqpoint{1.402854in}{2.030294in}}%
\pgfpathlineto{\pgfqpoint{1.422417in}{2.049968in}}%
\pgfusepath{stroke}%
\end{pgfscope}%
\begin{pgfscope}%
\pgfpathrectangle{\pgfqpoint{0.461111in}{0.526079in}}{\pgfqpoint{2.325000in}{1.510000in}} %
\pgfusepath{clip}%
\pgfsetrectcap%
\pgfsetroundjoin%
\pgfsetlinewidth{1.003750pt}%
\definecolor{currentstroke}{rgb}{0.501961,0.000000,0.501961}%
\pgfsetstrokecolor{currentstroke}%
\pgfsetdash{}{0pt}%
\pgfpathmoveto{\pgfqpoint{0.577361in}{1.140663in}}%
\pgfpathlineto{\pgfqpoint{0.599672in}{1.150067in}}%
\pgfpathlineto{\pgfqpoint{0.621983in}{1.160858in}}%
\pgfpathlineto{\pgfqpoint{0.644293in}{1.172938in}}%
\pgfpathlineto{\pgfqpoint{0.666604in}{1.186207in}}%
\pgfpathlineto{\pgfqpoint{0.688914in}{1.200570in}}%
\pgfpathlineto{\pgfqpoint{0.711225in}{1.215935in}}%
\pgfpathlineto{\pgfqpoint{0.733536in}{1.232216in}}%
\pgfpathlineto{\pgfqpoint{0.755846in}{1.249333in}}%
\pgfpathlineto{\pgfqpoint{0.778157in}{1.267214in}}%
\pgfpathlineto{\pgfqpoint{0.800467in}{1.285791in}}%
\pgfpathlineto{\pgfqpoint{0.822778in}{1.305005in}}%
\pgfpathlineto{\pgfqpoint{0.845089in}{1.324801in}}%
\pgfpathlineto{\pgfqpoint{0.867399in}{1.345129in}}%
\pgfpathlineto{\pgfqpoint{0.889710in}{1.365946in}}%
\pgfpathlineto{\pgfqpoint{0.912020in}{1.387211in}}%
\pgfpathlineto{\pgfqpoint{0.934331in}{1.408889in}}%
\pgfpathlineto{\pgfqpoint{0.956642in}{1.430947in}}%
\pgfpathlineto{\pgfqpoint{0.978952in}{1.453356in}}%
\pgfpathlineto{\pgfqpoint{1.001263in}{1.476091in}}%
\pgfpathlineto{\pgfqpoint{1.023574in}{1.499126in}}%
\pgfpathlineto{\pgfqpoint{1.045884in}{1.522441in}}%
\pgfpathlineto{\pgfqpoint{1.068195in}{1.546016in}}%
\pgfpathlineto{\pgfqpoint{1.090505in}{1.569833in}}%
\pgfpathlineto{\pgfqpoint{1.112816in}{1.593876in}}%
\pgfpathlineto{\pgfqpoint{1.135127in}{1.618131in}}%
\pgfpathlineto{\pgfqpoint{1.157437in}{1.642583in}}%
\pgfpathlineto{\pgfqpoint{1.179748in}{1.667221in}}%
\pgfpathlineto{\pgfqpoint{1.202058in}{1.692033in}}%
\pgfpathlineto{\pgfqpoint{1.224369in}{1.717009in}}%
\pgfpathlineto{\pgfqpoint{1.246680in}{1.742139in}}%
\pgfpathlineto{\pgfqpoint{1.268990in}{1.767415in}}%
\pgfpathlineto{\pgfqpoint{1.291301in}{1.792827in}}%
\pgfpathlineto{\pgfqpoint{1.313611in}{1.818370in}}%
\pgfpathlineto{\pgfqpoint{1.335922in}{1.844035in}}%
\pgfpathlineto{\pgfqpoint{1.358233in}{1.869816in}}%
\pgfpathlineto{\pgfqpoint{1.380543in}{1.895707in}}%
\pgfpathlineto{\pgfqpoint{1.402854in}{1.921702in}}%
\pgfpathlineto{\pgfqpoint{1.425164in}{1.947797in}}%
\pgfpathlineto{\pgfqpoint{1.447475in}{1.973986in}}%
\pgfpathlineto{\pgfqpoint{1.469786in}{2.000265in}}%
\pgfpathlineto{\pgfqpoint{1.492096in}{2.026629in}}%
\pgfpathlineto{\pgfqpoint{1.511786in}{2.049968in}}%
\pgfusepath{stroke}%
\end{pgfscope}%
\begin{pgfscope}%
\pgfpathrectangle{\pgfqpoint{0.461111in}{0.526079in}}{\pgfqpoint{2.325000in}{1.510000in}} %
\pgfusepath{clip}%
\pgfsetrectcap%
\pgfsetroundjoin%
\pgfsetlinewidth{1.003750pt}%
\definecolor{currentstroke}{rgb}{0.627451,0.321569,0.176471}%
\pgfsetstrokecolor{currentstroke}%
\pgfsetdash{}{0pt}%
\pgfpathmoveto{\pgfqpoint{0.577361in}{0.540500in}}%
\pgfpathlineto{\pgfqpoint{0.599672in}{0.546184in}}%
\pgfpathlineto{\pgfqpoint{0.621983in}{0.552605in}}%
\pgfpathlineto{\pgfqpoint{0.644293in}{0.559665in}}%
\pgfpathlineto{\pgfqpoint{0.666604in}{0.567270in}}%
\pgfpathlineto{\pgfqpoint{0.688914in}{0.575325in}}%
\pgfpathlineto{\pgfqpoint{0.711225in}{0.583746in}}%
\pgfpathlineto{\pgfqpoint{0.733536in}{0.592455in}}%
\pgfpathlineto{\pgfqpoint{0.755846in}{0.601388in}}%
\pgfpathlineto{\pgfqpoint{0.778157in}{0.610479in}}%
\pgfpathlineto{\pgfqpoint{0.800467in}{0.619657in}}%
\pgfpathlineto{\pgfqpoint{0.822778in}{0.628839in}}%
\pgfpathlineto{\pgfqpoint{0.845089in}{0.637932in}}%
\pgfpathlineto{\pgfqpoint{0.867399in}{0.646860in}}%
\pgfpathlineto{\pgfqpoint{0.889710in}{0.655568in}}%
\pgfpathlineto{\pgfqpoint{0.912020in}{0.664030in}}%
\pgfpathlineto{\pgfqpoint{0.934331in}{0.672239in}}%
\pgfpathlineto{\pgfqpoint{0.956642in}{0.680201in}}%
\pgfpathlineto{\pgfqpoint{0.978952in}{0.687923in}}%
\pgfpathlineto{\pgfqpoint{1.001263in}{0.695417in}}%
\pgfpathlineto{\pgfqpoint{1.023574in}{0.702690in}}%
\pgfpathlineto{\pgfqpoint{1.045884in}{0.709748in}}%
\pgfpathlineto{\pgfqpoint{1.068195in}{0.716593in}}%
\pgfpathlineto{\pgfqpoint{1.090505in}{0.723228in}}%
\pgfpathlineto{\pgfqpoint{1.112816in}{0.729652in}}%
\pgfpathlineto{\pgfqpoint{1.135127in}{0.735865in}}%
\pgfpathlineto{\pgfqpoint{1.157437in}{0.741868in}}%
\pgfpathlineto{\pgfqpoint{1.179748in}{0.747660in}}%
\pgfpathlineto{\pgfqpoint{1.202058in}{0.753243in}}%
\pgfpathlineto{\pgfqpoint{1.224369in}{0.758618in}}%
\pgfpathlineto{\pgfqpoint{1.246680in}{0.763788in}}%
\pgfpathlineto{\pgfqpoint{1.268990in}{0.768756in}}%
\pgfpathlineto{\pgfqpoint{1.291301in}{0.773526in}}%
\pgfpathlineto{\pgfqpoint{1.313611in}{0.778104in}}%
\pgfpathlineto{\pgfqpoint{1.335922in}{0.782494in}}%
\pgfpathlineto{\pgfqpoint{1.358233in}{0.786702in}}%
\pgfpathlineto{\pgfqpoint{1.380543in}{0.790734in}}%
\pgfpathlineto{\pgfqpoint{1.402854in}{0.794596in}}%
\pgfpathlineto{\pgfqpoint{1.425164in}{0.798295in}}%
\pgfpathlineto{\pgfqpoint{1.447475in}{0.801837in}}%
\pgfpathlineto{\pgfqpoint{1.469786in}{0.805229in}}%
\pgfpathlineto{\pgfqpoint{1.492096in}{0.808476in}}%
\pgfpathlineto{\pgfqpoint{1.514407in}{0.811585in}}%
\pgfpathlineto{\pgfqpoint{1.536717in}{0.814562in}}%
\pgfpathlineto{\pgfqpoint{1.559028in}{0.817414in}}%
\pgfpathlineto{\pgfqpoint{1.581339in}{0.820145in}}%
\pgfpathlineto{\pgfqpoint{1.603649in}{0.822761in}}%
\pgfpathlineto{\pgfqpoint{1.625960in}{0.825269in}}%
\pgfpathlineto{\pgfqpoint{1.648270in}{0.827672in}}%
\pgfpathlineto{\pgfqpoint{1.670581in}{0.829976in}}%
\pgfpathlineto{\pgfqpoint{1.692892in}{0.832185in}}%
\pgfpathlineto{\pgfqpoint{1.715202in}{0.834304in}}%
\pgfpathlineto{\pgfqpoint{1.737513in}{0.836338in}}%
\pgfpathlineto{\pgfqpoint{1.759824in}{0.838289in}}%
\pgfpathlineto{\pgfqpoint{1.782134in}{0.840163in}}%
\pgfpathlineto{\pgfqpoint{1.804445in}{0.841963in}}%
\pgfpathlineto{\pgfqpoint{1.826755in}{0.843692in}}%
\pgfpathlineto{\pgfqpoint{1.849066in}{0.845354in}}%
\pgfpathlineto{\pgfqpoint{1.871377in}{0.846952in}}%
\pgfpathlineto{\pgfqpoint{1.893687in}{0.848489in}}%
\pgfpathlineto{\pgfqpoint{1.915998in}{0.849967in}}%
\pgfpathlineto{\pgfqpoint{1.938308in}{0.851390in}}%
\pgfpathlineto{\pgfqpoint{1.960619in}{0.852760in}}%
\pgfpathlineto{\pgfqpoint{1.982930in}{0.854079in}}%
\pgfpathlineto{\pgfqpoint{2.005240in}{0.855350in}}%
\pgfpathlineto{\pgfqpoint{2.027551in}{0.856575in}}%
\pgfpathlineto{\pgfqpoint{2.049861in}{0.857756in}}%
\pgfpathlineto{\pgfqpoint{2.072172in}{0.858895in}}%
\pgfpathlineto{\pgfqpoint{2.094483in}{0.859994in}}%
\pgfpathlineto{\pgfqpoint{2.116793in}{0.861054in}}%
\pgfpathlineto{\pgfqpoint{2.139104in}{0.862077in}}%
\pgfpathlineto{\pgfqpoint{2.161414in}{0.863066in}}%
\pgfpathlineto{\pgfqpoint{2.183725in}{0.864021in}}%
\pgfpathlineto{\pgfqpoint{2.206036in}{0.864943in}}%
\pgfpathlineto{\pgfqpoint{2.228346in}{0.865835in}}%
\pgfpathlineto{\pgfqpoint{2.250657in}{0.866697in}}%
\pgfpathlineto{\pgfqpoint{2.272967in}{0.867532in}}%
\pgfpathlineto{\pgfqpoint{2.295278in}{0.868339in}}%
\pgfpathlineto{\pgfqpoint{2.317589in}{0.869120in}}%
\pgfpathlineto{\pgfqpoint{2.339899in}{0.869876in}}%
\pgfpathlineto{\pgfqpoint{2.362210in}{0.870608in}}%
\pgfpathlineto{\pgfqpoint{2.384520in}{0.871317in}}%
\pgfpathlineto{\pgfqpoint{2.406831in}{0.872004in}}%
\pgfpathlineto{\pgfqpoint{2.429142in}{0.872670in}}%
\pgfpathlineto{\pgfqpoint{2.451452in}{0.873316in}}%
\pgfpathlineto{\pgfqpoint{2.473763in}{0.873942in}}%
\pgfpathlineto{\pgfqpoint{2.496074in}{0.874550in}}%
\pgfpathlineto{\pgfqpoint{2.518384in}{0.875139in}}%
\pgfpathlineto{\pgfqpoint{2.540695in}{0.875711in}}%
\pgfpathlineto{\pgfqpoint{2.563005in}{0.876267in}}%
\pgfpathlineto{\pgfqpoint{2.585316in}{0.876806in}}%
\pgfpathlineto{\pgfqpoint{2.607627in}{0.877330in}}%
\pgfpathlineto{\pgfqpoint{2.629937in}{0.877839in}}%
\pgfpathlineto{\pgfqpoint{2.652248in}{0.878333in}}%
\pgfpathlineto{\pgfqpoint{2.674558in}{0.878814in}}%
\pgfpathlineto{\pgfqpoint{2.696869in}{0.879282in}}%
\pgfpathlineto{\pgfqpoint{2.719180in}{0.879736in}}%
\pgfpathlineto{\pgfqpoint{2.741490in}{0.880178in}}%
\pgfpathlineto{\pgfqpoint{2.763801in}{0.880609in}}%
\pgfpathlineto{\pgfqpoint{2.786111in}{0.881027in}}%
\pgfusepath{stroke}%
\end{pgfscope}%
\begin{pgfscope}%
\pgfpathrectangle{\pgfqpoint{0.461111in}{0.526079in}}{\pgfqpoint{2.325000in}{1.510000in}} %
\pgfusepath{clip}%
\pgfsetbuttcap%
\pgfsetroundjoin%
\pgfsetlinewidth{0.501875pt}%
\definecolor{currentstroke}{rgb}{0.000000,0.000000,0.000000}%
\pgfsetstrokecolor{currentstroke}%
\pgfsetdash{{1.850000pt}{0.800000pt}}{0.000000pt}%
\pgfpathmoveto{\pgfqpoint{0.490174in}{0.903579in}}%
\pgfpathlineto{\pgfqpoint{0.563948in}{0.903579in}}%
\pgfpathlineto{\pgfqpoint{0.637722in}{0.903579in}}%
\pgfpathlineto{\pgfqpoint{0.711496in}{0.903579in}}%
\pgfpathlineto{\pgfqpoint{0.785270in}{0.903579in}}%
\pgfpathlineto{\pgfqpoint{0.859044in}{0.903579in}}%
\pgfpathlineto{\pgfqpoint{0.932818in}{0.903579in}}%
\pgfpathlineto{\pgfqpoint{1.006592in}{0.903579in}}%
\pgfpathlineto{\pgfqpoint{1.080366in}{0.903579in}}%
\pgfpathlineto{\pgfqpoint{1.154140in}{0.903579in}}%
\pgfpathlineto{\pgfqpoint{1.227914in}{0.903579in}}%
\pgfpathlineto{\pgfqpoint{1.301688in}{0.903579in}}%
\pgfpathlineto{\pgfqpoint{1.375462in}{0.903579in}}%
\pgfpathlineto{\pgfqpoint{1.449236in}{0.903579in}}%
\pgfpathlineto{\pgfqpoint{1.523010in}{0.903579in}}%
\pgfpathlineto{\pgfqpoint{1.596784in}{0.903579in}}%
\pgfpathlineto{\pgfqpoint{1.670559in}{0.903579in}}%
\pgfpathlineto{\pgfqpoint{1.744333in}{0.903579in}}%
\pgfpathlineto{\pgfqpoint{1.818107in}{0.903579in}}%
\pgfpathlineto{\pgfqpoint{1.891881in}{0.903579in}}%
\pgfpathlineto{\pgfqpoint{1.965655in}{0.903579in}}%
\pgfpathlineto{\pgfqpoint{2.039429in}{0.903579in}}%
\pgfpathlineto{\pgfqpoint{2.113203in}{0.903579in}}%
\pgfpathlineto{\pgfqpoint{2.186977in}{0.903579in}}%
\pgfpathlineto{\pgfqpoint{2.260751in}{0.903579in}}%
\pgfpathlineto{\pgfqpoint{2.334525in}{0.903579in}}%
\pgfpathlineto{\pgfqpoint{2.408299in}{0.903579in}}%
\pgfpathlineto{\pgfqpoint{2.482073in}{0.903579in}}%
\pgfpathlineto{\pgfqpoint{2.555847in}{0.903579in}}%
\pgfpathlineto{\pgfqpoint{2.629621in}{0.903579in}}%
\pgfpathlineto{\pgfqpoint{2.703395in}{0.903579in}}%
\pgfpathlineto{\pgfqpoint{2.777169in}{0.903579in}}%
\pgfpathlineto{\pgfqpoint{2.800000in}{0.903579in}}%
\pgfusepath{stroke}%
\end{pgfscope}%
\begin{pgfscope}%
\pgfsetrectcap%
\pgfsetmiterjoin%
\pgfsetlinewidth{0.803000pt}%
\definecolor{currentstroke}{rgb}{0.000000,0.000000,0.000000}%
\pgfsetstrokecolor{currentstroke}%
\pgfsetdash{}{0pt}%
\pgfpathmoveto{\pgfqpoint{0.461111in}{0.526079in}}%
\pgfpathlineto{\pgfqpoint{0.461111in}{2.036079in}}%
\pgfusepath{stroke}%
\end{pgfscope}%
\begin{pgfscope}%
\pgfsetrectcap%
\pgfsetmiterjoin%
\pgfsetlinewidth{0.803000pt}%
\definecolor{currentstroke}{rgb}{0.000000,0.000000,0.000000}%
\pgfsetstrokecolor{currentstroke}%
\pgfsetdash{}{0pt}%
\pgfpathmoveto{\pgfqpoint{2.786111in}{0.526079in}}%
\pgfpathlineto{\pgfqpoint{2.786111in}{2.036079in}}%
\pgfusepath{stroke}%
\end{pgfscope}%
\begin{pgfscope}%
\pgfsetrectcap%
\pgfsetmiterjoin%
\pgfsetlinewidth{0.803000pt}%
\definecolor{currentstroke}{rgb}{0.000000,0.000000,0.000000}%
\pgfsetstrokecolor{currentstroke}%
\pgfsetdash{}{0pt}%
\pgfpathmoveto{\pgfqpoint{0.461111in}{0.526079in}}%
\pgfpathlineto{\pgfqpoint{2.786111in}{0.526079in}}%
\pgfusepath{stroke}%
\end{pgfscope}%
\begin{pgfscope}%
\pgfsetrectcap%
\pgfsetmiterjoin%
\pgfsetlinewidth{0.803000pt}%
\definecolor{currentstroke}{rgb}{0.000000,0.000000,0.000000}%
\pgfsetstrokecolor{currentstroke}%
\pgfsetdash{}{0pt}%
\pgfpathmoveto{\pgfqpoint{0.461111in}{2.036079in}}%
\pgfpathlineto{\pgfqpoint{2.786111in}{2.036079in}}%
\pgfusepath{stroke}%
\end{pgfscope}%
\begin{pgfscope}%
\pgfsetbuttcap%
\pgfsetmiterjoin%
\definecolor{currentfill}{rgb}{1.000000,1.000000,1.000000}%
\pgfsetfillcolor{currentfill}%
\pgfsetfillopacity{0.800000}%
\pgfsetlinewidth{1.003750pt}%
\definecolor{currentstroke}{rgb}{0.800000,0.800000,0.800000}%
\pgfsetstrokecolor{currentstroke}%
\pgfsetstrokeopacity{0.800000}%
\pgfsetdash{}{0pt}%
\pgfpathmoveto{\pgfqpoint{1.642299in}{1.103707in}}%
\pgfpathlineto{\pgfqpoint{2.688889in}{1.103707in}}%
\pgfpathquadraticcurveto{\pgfqpoint{2.716667in}{1.103707in}}{\pgfqpoint{2.716667in}{1.131485in}}%
\pgfpathlineto{\pgfqpoint{2.716667in}{1.938857in}}%
\pgfpathquadraticcurveto{\pgfqpoint{2.716667in}{1.966635in}}{\pgfqpoint{2.688889in}{1.966635in}}%
\pgfpathlineto{\pgfqpoint{1.642299in}{1.966635in}}%
\pgfpathquadraticcurveto{\pgfqpoint{1.614521in}{1.966635in}}{\pgfqpoint{1.614521in}{1.938857in}}%
\pgfpathlineto{\pgfqpoint{1.614521in}{1.131485in}}%
\pgfpathquadraticcurveto{\pgfqpoint{1.614521in}{1.103707in}}{\pgfqpoint{1.642299in}{1.103707in}}%
\pgfpathclose%
\pgfusepath{stroke,fill}%
\end{pgfscope}%
\begin{pgfscope}%
\pgfsetrectcap%
\pgfsetroundjoin%
\pgfsetlinewidth{1.003750pt}%
\definecolor{currentstroke}{rgb}{1.000000,0.549020,0.000000}%
\pgfsetstrokecolor{currentstroke}%
\pgfsetdash{}{0pt}%
\pgfpathmoveto{\pgfqpoint{1.670077in}{1.854168in}}%
\pgfpathlineto{\pgfqpoint{1.947855in}{1.854168in}}%
\pgfusepath{stroke}%
\end{pgfscope}%
\begin{pgfscope}%
\pgftext[x=2.058966in,y=1.805556in,left,base]{\rmfamily\fontsize{10.000000}{12.000000}\selectfont \(\displaystyle \mathrm{R}\) - wave}%
\end{pgfscope}%
\begin{pgfscope}%
\pgfsetrectcap%
\pgfsetroundjoin%
\pgfsetlinewidth{1.003750pt}%
\definecolor{currentstroke}{rgb}{0.501961,0.000000,0.501961}%
\pgfsetstrokecolor{currentstroke}%
\pgfsetdash{}{0pt}%
\pgfpathmoveto{\pgfqpoint{1.670077in}{1.650310in}}%
\pgfpathlineto{\pgfqpoint{1.947855in}{1.650310in}}%
\pgfusepath{stroke}%
\end{pgfscope}%
\begin{pgfscope}%
\pgftext[x=2.058966in,y=1.601699in,left,base]{\rmfamily\fontsize{10.000000}{12.000000}\selectfont L - wave}%
\end{pgfscope}%
\begin{pgfscope}%
\pgfsetrectcap%
\pgfsetroundjoin%
\pgfsetlinewidth{1.003750pt}%
\definecolor{currentstroke}{rgb}{0.627451,0.321569,0.176471}%
\pgfsetstrokecolor{currentstroke}%
\pgfsetdash{}{0pt}%
\pgfpathmoveto{\pgfqpoint{1.670077in}{1.446453in}}%
\pgfpathlineto{\pgfqpoint{1.947855in}{1.446453in}}%
\pgfusepath{stroke}%
\end{pgfscope}%
\begin{pgfscope}%
\pgftext[x=2.058966in,y=1.397842in,left,base]{\rmfamily\fontsize{10.000000}{12.000000}\selectfont \(\displaystyle \mathrm{R}\) - wave}%
\end{pgfscope}%
\begin{pgfscope}%
\pgfsetbuttcap%
\pgfsetroundjoin%
\pgfsetlinewidth{0.501875pt}%
\definecolor{currentstroke}{rgb}{0.000000,0.000000,0.000000}%
\pgfsetstrokecolor{currentstroke}%
\pgfsetdash{{1.850000pt}{0.800000pt}}{0.000000pt}%
\pgfpathmoveto{\pgfqpoint{1.670077in}{1.242596in}}%
\pgfpathlineto{\pgfqpoint{1.947855in}{1.242596in}}%
\pgfusepath{stroke}%
\end{pgfscope}%
\begin{pgfscope}%
\pgftext[x=2.058966in,y=1.193985in,left,base]{\rmfamily\fontsize{10.000000}{12.000000}\selectfont \(\displaystyle |\Omega_\mathrm{ce}|\)}%
\end{pgfscope}%
\end{pgfpicture}%
\makeatother%
\endgroup%

%%% Creator: Matplotlib, PGF backend
%%
%% To include the figure in your LaTeX document, write
%%   \input{<filename>.pgf}
%%
%% Make sure the required packages are loaded in your preamble
%%   \usepackage{pgf}
%%
%% Figures using additional raster images can only be included by \input if
%% they are in the same directory as the main LaTeX file. For loading figures
%% from other directories you can use the `import` package
%%   \usepackage{import}
%% and then include the figures with
%%   \import{<path to file>}{<filename>.pgf}
%%
%% Matplotlib used the following preamble
%%   \usepackage{fontspec}
%%   \setmainfont{DejaVu Serif}
%%   \setsansfont{DejaVu Sans}
%%   \setmonofont{DejaVu Sans Mono}
%%
\begingroup%
\makeatletter%
\begin{pgfpicture}%
\pgfpathrectangle{\pgfpointorigin}{\pgfqpoint{3.491637in}{2.184691in}}%
\pgfusepath{use as bounding box, clip}%
\begin{pgfscope}%
\pgfsetbuttcap%
\pgfsetmiterjoin%
\definecolor{currentfill}{rgb}{1.000000,1.000000,1.000000}%
\pgfsetfillcolor{currentfill}%
\pgfsetlinewidth{0.000000pt}%
\definecolor{currentstroke}{rgb}{1.000000,1.000000,1.000000}%
\pgfsetstrokecolor{currentstroke}%
\pgfsetdash{}{0pt}%
\pgfpathmoveto{\pgfqpoint{0.000000in}{0.000000in}}%
\pgfpathlineto{\pgfqpoint{3.491637in}{0.000000in}}%
\pgfpathlineto{\pgfqpoint{3.491637in}{2.184691in}}%
\pgfpathlineto{\pgfqpoint{0.000000in}{2.184691in}}%
\pgfpathclose%
\pgfusepath{fill}%
\end{pgfscope}%
\begin{pgfscope}%
\pgfsetbuttcap%
\pgfsetmiterjoin%
\definecolor{currentfill}{rgb}{1.000000,1.000000,1.000000}%
\pgfsetfillcolor{currentfill}%
\pgfsetlinewidth{0.000000pt}%
\definecolor{currentstroke}{rgb}{0.000000,0.000000,0.000000}%
\pgfsetstrokecolor{currentstroke}%
\pgfsetstrokeopacity{0.000000}%
\pgfsetdash{}{0pt}%
\pgfpathmoveto{\pgfqpoint{0.708026in}{0.526079in}}%
\pgfpathlineto{\pgfqpoint{3.343026in}{0.526079in}}%
\pgfpathlineto{\pgfqpoint{3.343026in}{2.036079in}}%
\pgfpathlineto{\pgfqpoint{0.708026in}{2.036079in}}%
\pgfpathclose%
\pgfusepath{fill}%
\end{pgfscope}%
\begin{pgfscope}%
\pgfsetbuttcap%
\pgfsetroundjoin%
\definecolor{currentfill}{rgb}{0.000000,0.000000,0.000000}%
\pgfsetfillcolor{currentfill}%
\pgfsetlinewidth{0.803000pt}%
\definecolor{currentstroke}{rgb}{0.000000,0.000000,0.000000}%
\pgfsetstrokecolor{currentstroke}%
\pgfsetdash{}{0pt}%
\pgfsys@defobject{currentmarker}{\pgfqpoint{0.000000in}{-0.048611in}}{\pgfqpoint{0.000000in}{0.000000in}}{%
\pgfpathmoveto{\pgfqpoint{0.000000in}{0.000000in}}%
\pgfpathlineto{\pgfqpoint{0.000000in}{-0.048611in}}%
\pgfusepath{stroke,fill}%
}%
\begin{pgfscope}%
\pgfsys@transformshift{0.708026in}{0.526079in}%
\pgfsys@useobject{currentmarker}{}%
\end{pgfscope}%
\end{pgfscope}%
\begin{pgfscope}%
\pgftext[x=0.708026in,y=0.428857in,,top]{\rmfamily\fontsize{10.000000}{12.000000}\selectfont \(\displaystyle 0\)}%
\end{pgfscope}%
\begin{pgfscope}%
\pgfsetbuttcap%
\pgfsetroundjoin%
\definecolor{currentfill}{rgb}{0.000000,0.000000,0.000000}%
\pgfsetfillcolor{currentfill}%
\pgfsetlinewidth{0.803000pt}%
\definecolor{currentstroke}{rgb}{0.000000,0.000000,0.000000}%
\pgfsetstrokecolor{currentstroke}%
\pgfsetdash{}{0pt}%
\pgfsys@defobject{currentmarker}{\pgfqpoint{0.000000in}{-0.048611in}}{\pgfqpoint{0.000000in}{0.000000in}}{%
\pgfpathmoveto{\pgfqpoint{0.000000in}{0.000000in}}%
\pgfpathlineto{\pgfqpoint{0.000000in}{-0.048611in}}%
\pgfusepath{stroke,fill}%
}%
\begin{pgfscope}%
\pgfsys@transformshift{1.366776in}{0.526079in}%
\pgfsys@useobject{currentmarker}{}%
\end{pgfscope}%
\end{pgfscope}%
\begin{pgfscope}%
\pgftext[x=1.366776in,y=0.428857in,,top]{\rmfamily\fontsize{10.000000}{12.000000}\selectfont \(\displaystyle 2\)}%
\end{pgfscope}%
\begin{pgfscope}%
\pgfsetbuttcap%
\pgfsetroundjoin%
\definecolor{currentfill}{rgb}{0.000000,0.000000,0.000000}%
\pgfsetfillcolor{currentfill}%
\pgfsetlinewidth{0.803000pt}%
\definecolor{currentstroke}{rgb}{0.000000,0.000000,0.000000}%
\pgfsetstrokecolor{currentstroke}%
\pgfsetdash{}{0pt}%
\pgfsys@defobject{currentmarker}{\pgfqpoint{0.000000in}{-0.048611in}}{\pgfqpoint{0.000000in}{0.000000in}}{%
\pgfpathmoveto{\pgfqpoint{0.000000in}{0.000000in}}%
\pgfpathlineto{\pgfqpoint{0.000000in}{-0.048611in}}%
\pgfusepath{stroke,fill}%
}%
\begin{pgfscope}%
\pgfsys@transformshift{2.025526in}{0.526079in}%
\pgfsys@useobject{currentmarker}{}%
\end{pgfscope}%
\end{pgfscope}%
\begin{pgfscope}%
\pgftext[x=2.025526in,y=0.428857in,,top]{\rmfamily\fontsize{10.000000}{12.000000}\selectfont \(\displaystyle 4\)}%
\end{pgfscope}%
\begin{pgfscope}%
\pgfsetbuttcap%
\pgfsetroundjoin%
\definecolor{currentfill}{rgb}{0.000000,0.000000,0.000000}%
\pgfsetfillcolor{currentfill}%
\pgfsetlinewidth{0.803000pt}%
\definecolor{currentstroke}{rgb}{0.000000,0.000000,0.000000}%
\pgfsetstrokecolor{currentstroke}%
\pgfsetdash{}{0pt}%
\pgfsys@defobject{currentmarker}{\pgfqpoint{0.000000in}{-0.048611in}}{\pgfqpoint{0.000000in}{0.000000in}}{%
\pgfpathmoveto{\pgfqpoint{0.000000in}{0.000000in}}%
\pgfpathlineto{\pgfqpoint{0.000000in}{-0.048611in}}%
\pgfusepath{stroke,fill}%
}%
\begin{pgfscope}%
\pgfsys@transformshift{2.684276in}{0.526079in}%
\pgfsys@useobject{currentmarker}{}%
\end{pgfscope}%
\end{pgfscope}%
\begin{pgfscope}%
\pgftext[x=2.684276in,y=0.428857in,,top]{\rmfamily\fontsize{10.000000}{12.000000}\selectfont \(\displaystyle 6\)}%
\end{pgfscope}%
\begin{pgfscope}%
\pgfsetbuttcap%
\pgfsetroundjoin%
\definecolor{currentfill}{rgb}{0.000000,0.000000,0.000000}%
\pgfsetfillcolor{currentfill}%
\pgfsetlinewidth{0.803000pt}%
\definecolor{currentstroke}{rgb}{0.000000,0.000000,0.000000}%
\pgfsetstrokecolor{currentstroke}%
\pgfsetdash{}{0pt}%
\pgfsys@defobject{currentmarker}{\pgfqpoint{0.000000in}{-0.048611in}}{\pgfqpoint{0.000000in}{0.000000in}}{%
\pgfpathmoveto{\pgfqpoint{0.000000in}{0.000000in}}%
\pgfpathlineto{\pgfqpoint{0.000000in}{-0.048611in}}%
\pgfusepath{stroke,fill}%
}%
\begin{pgfscope}%
\pgfsys@transformshift{3.343026in}{0.526079in}%
\pgfsys@useobject{currentmarker}{}%
\end{pgfscope}%
\end{pgfscope}%
\begin{pgfscope}%
\pgftext[x=3.343026in,y=0.428857in,,top]{\rmfamily\fontsize{10.000000}{12.000000}\selectfont \(\displaystyle 8\)}%
\end{pgfscope}%
\begin{pgfscope}%
\pgftext[x=2.025526in,y=0.238889in,,top]{\rmfamily\fontsize{10.000000}{12.000000}\selectfont \(\displaystyle kc / |\Omega_\mathrm{ce}|\)}%
\end{pgfscope}%
\begin{pgfscope}%
\pgfsetbuttcap%
\pgfsetroundjoin%
\definecolor{currentfill}{rgb}{0.000000,0.000000,0.000000}%
\pgfsetfillcolor{currentfill}%
\pgfsetlinewidth{0.803000pt}%
\definecolor{currentstroke}{rgb}{0.000000,0.000000,0.000000}%
\pgfsetstrokecolor{currentstroke}%
\pgfsetdash{}{0pt}%
\pgfsys@defobject{currentmarker}{\pgfqpoint{-0.048611in}{0.000000in}}{\pgfqpoint{0.000000in}{0.000000in}}{%
\pgfpathmoveto{\pgfqpoint{0.000000in}{0.000000in}}%
\pgfpathlineto{\pgfqpoint{-0.048611in}{0.000000in}}%
\pgfusepath{stroke,fill}%
}%
\begin{pgfscope}%
\pgfsys@transformshift{0.708026in}{0.596518in}%
\pgfsys@useobject{currentmarker}{}%
\end{pgfscope}%
\end{pgfscope}%
\begin{pgfscope}%
\pgftext[x=0.294444in,y=0.543757in,left,base]{\rmfamily\fontsize{10.000000}{12.000000}\selectfont \(\displaystyle 0.000\)}%
\end{pgfscope}%
\begin{pgfscope}%
\pgfsetbuttcap%
\pgfsetroundjoin%
\definecolor{currentfill}{rgb}{0.000000,0.000000,0.000000}%
\pgfsetfillcolor{currentfill}%
\pgfsetlinewidth{0.803000pt}%
\definecolor{currentstroke}{rgb}{0.000000,0.000000,0.000000}%
\pgfsetstrokecolor{currentstroke}%
\pgfsetdash{}{0pt}%
\pgfsys@defobject{currentmarker}{\pgfqpoint{-0.048611in}{0.000000in}}{\pgfqpoint{0.000000in}{0.000000in}}{%
\pgfpathmoveto{\pgfqpoint{0.000000in}{0.000000in}}%
\pgfpathlineto{\pgfqpoint{-0.048611in}{0.000000in}}%
\pgfusepath{stroke,fill}%
}%
\begin{pgfscope}%
\pgfsys@transformshift{0.708026in}{1.038821in}%
\pgfsys@useobject{currentmarker}{}%
\end{pgfscope}%
\end{pgfscope}%
\begin{pgfscope}%
\pgftext[x=0.294444in,y=0.986059in,left,base]{\rmfamily\fontsize{10.000000}{12.000000}\selectfont \(\displaystyle 0.002\)}%
\end{pgfscope}%
\begin{pgfscope}%
\pgfsetbuttcap%
\pgfsetroundjoin%
\definecolor{currentfill}{rgb}{0.000000,0.000000,0.000000}%
\pgfsetfillcolor{currentfill}%
\pgfsetlinewidth{0.803000pt}%
\definecolor{currentstroke}{rgb}{0.000000,0.000000,0.000000}%
\pgfsetstrokecolor{currentstroke}%
\pgfsetdash{}{0pt}%
\pgfsys@defobject{currentmarker}{\pgfqpoint{-0.048611in}{0.000000in}}{\pgfqpoint{0.000000in}{0.000000in}}{%
\pgfpathmoveto{\pgfqpoint{0.000000in}{0.000000in}}%
\pgfpathlineto{\pgfqpoint{-0.048611in}{0.000000in}}%
\pgfusepath{stroke,fill}%
}%
\begin{pgfscope}%
\pgfsys@transformshift{0.708026in}{1.481123in}%
\pgfsys@useobject{currentmarker}{}%
\end{pgfscope}%
\end{pgfscope}%
\begin{pgfscope}%
\pgftext[x=0.294444in,y=1.428362in,left,base]{\rmfamily\fontsize{10.000000}{12.000000}\selectfont \(\displaystyle 0.004\)}%
\end{pgfscope}%
\begin{pgfscope}%
\pgfsetbuttcap%
\pgfsetroundjoin%
\definecolor{currentfill}{rgb}{0.000000,0.000000,0.000000}%
\pgfsetfillcolor{currentfill}%
\pgfsetlinewidth{0.803000pt}%
\definecolor{currentstroke}{rgb}{0.000000,0.000000,0.000000}%
\pgfsetstrokecolor{currentstroke}%
\pgfsetdash{}{0pt}%
\pgfsys@defobject{currentmarker}{\pgfqpoint{-0.048611in}{0.000000in}}{\pgfqpoint{0.000000in}{0.000000in}}{%
\pgfpathmoveto{\pgfqpoint{0.000000in}{0.000000in}}%
\pgfpathlineto{\pgfqpoint{-0.048611in}{0.000000in}}%
\pgfusepath{stroke,fill}%
}%
\begin{pgfscope}%
\pgfsys@transformshift{0.708026in}{1.923426in}%
\pgfsys@useobject{currentmarker}{}%
\end{pgfscope}%
\end{pgfscope}%
\begin{pgfscope}%
\pgftext[x=0.294444in,y=1.870664in,left,base]{\rmfamily\fontsize{10.000000}{12.000000}\selectfont \(\displaystyle 0.006\)}%
\end{pgfscope}%
\begin{pgfscope}%
\pgftext[x=0.238889in,y=1.281079in,,bottom,rotate=90.000000]{\rmfamily\fontsize{10.000000}{12.000000}\selectfont \(\displaystyle \gamma / |\Omega_\mathrm{ce}|\)}%
\end{pgfscope}%
\begin{pgfscope}%
\pgfpathrectangle{\pgfqpoint{0.708026in}{0.526079in}}{\pgfqpoint{2.635000in}{1.510000in}} %
\pgfusepath{clip}%
\pgfsetrectcap%
\pgfsetroundjoin%
\pgfsetlinewidth{1.003750pt}%
\definecolor{currentstroke}{rgb}{0.627451,0.321569,0.176471}%
\pgfsetstrokecolor{currentstroke}%
\pgfsetdash{}{0pt}%
\pgfpathmoveto{\pgfqpoint{0.839776in}{0.596518in}}%
\pgfpathlineto{\pgfqpoint{0.865061in}{0.596518in}}%
\pgfpathlineto{\pgfqpoint{0.890346in}{0.596518in}}%
\pgfpathlineto{\pgfqpoint{0.915632in}{0.596518in}}%
\pgfpathlineto{\pgfqpoint{0.940917in}{0.596518in}}%
\pgfpathlineto{\pgfqpoint{0.966202in}{0.596527in}}%
\pgfpathlineto{\pgfqpoint{0.991488in}{0.596717in}}%
\pgfpathlineto{\pgfqpoint{1.016773in}{0.598452in}}%
\pgfpathlineto{\pgfqpoint{1.042059in}{0.607122in}}%
\pgfpathlineto{\pgfqpoint{1.067344in}{0.634871in}}%
\pgfpathlineto{\pgfqpoint{1.092629in}{0.698318in}}%
\pgfpathlineto{\pgfqpoint{1.117915in}{0.810099in}}%
\pgfpathlineto{\pgfqpoint{1.143200in}{0.970285in}}%
\pgfpathlineto{\pgfqpoint{1.168485in}{1.164607in}}%
\pgfpathlineto{\pgfqpoint{1.193771in}{1.370191in}}%
\pgfpathlineto{\pgfqpoint{1.219056in}{1.563739in}}%
\pgfpathlineto{\pgfqpoint{1.244341in}{1.727360in}}%
\pgfpathlineto{\pgfqpoint{1.269627in}{1.850659in}}%
\pgfpathlineto{\pgfqpoint{1.294912in}{1.930138in}}%
\pgfpathlineto{\pgfqpoint{1.320197in}{1.967441in}}%
\pgfpathlineto{\pgfqpoint{1.345483in}{1.967443in}}%
\pgfpathlineto{\pgfqpoint{1.370768in}{1.936656in}}%
\pgfpathlineto{\pgfqpoint{1.396053in}{1.882067in}}%
\pgfpathlineto{\pgfqpoint{1.421339in}{1.810376in}}%
\pgfpathlineto{\pgfqpoint{1.446624in}{1.727552in}}%
\pgfpathlineto{\pgfqpoint{1.471910in}{1.638620in}}%
\pgfpathlineto{\pgfqpoint{1.497195in}{1.547607in}}%
\pgfpathlineto{\pgfqpoint{1.522480in}{1.457591in}}%
\pgfpathlineto{\pgfqpoint{1.547766in}{1.370805in}}%
\pgfpathlineto{\pgfqpoint{1.573051in}{1.288766in}}%
\pgfpathlineto{\pgfqpoint{1.598336in}{1.212415in}}%
\pgfpathlineto{\pgfqpoint{1.623622in}{1.142238in}}%
\pgfpathlineto{\pgfqpoint{1.648907in}{1.078387in}}%
\pgfpathlineto{\pgfqpoint{1.674192in}{1.020769in}}%
\pgfpathlineto{\pgfqpoint{1.699478in}{0.969126in}}%
\pgfpathlineto{\pgfqpoint{1.724763in}{0.923095in}}%
\pgfpathlineto{\pgfqpoint{1.750048in}{0.882252in}}%
\pgfpathlineto{\pgfqpoint{1.775334in}{0.846149in}}%
\pgfpathlineto{\pgfqpoint{1.800619in}{0.814331in}}%
\pgfpathlineto{\pgfqpoint{1.825904in}{0.786361in}}%
\pgfpathlineto{\pgfqpoint{1.851190in}{0.761822in}}%
\pgfpathlineto{\pgfqpoint{1.876475in}{0.740329in}}%
\pgfpathlineto{\pgfqpoint{1.901761in}{0.721528in}}%
\pgfpathlineto{\pgfqpoint{1.927046in}{0.705098in}}%
\pgfpathlineto{\pgfqpoint{1.952331in}{0.690753in}}%
\pgfpathlineto{\pgfqpoint{1.977617in}{0.678236in}}%
\pgfpathlineto{\pgfqpoint{2.002902in}{0.667318in}}%
\pgfpathlineto{\pgfqpoint{2.028187in}{0.657799in}}%
\pgfpathlineto{\pgfqpoint{2.053473in}{0.649502in}}%
\pgfpathlineto{\pgfqpoint{2.078758in}{0.642272in}}%
\pgfpathlineto{\pgfqpoint{2.104043in}{0.635972in}}%
\pgfpathlineto{\pgfqpoint{2.129329in}{0.630483in}}%
\pgfpathlineto{\pgfqpoint{2.154614in}{0.625702in}}%
\pgfpathlineto{\pgfqpoint{2.179899in}{0.621537in}}%
\pgfpathlineto{\pgfqpoint{2.205185in}{0.617910in}}%
\pgfpathlineto{\pgfqpoint{2.230470in}{0.614751in}}%
\pgfpathlineto{\pgfqpoint{2.255755in}{0.612001in}}%
\pgfpathlineto{\pgfqpoint{2.281041in}{0.609608in}}%
\pgfpathlineto{\pgfqpoint{2.306326in}{0.607525in}}%
\pgfpathlineto{\pgfqpoint{2.331612in}{0.605714in}}%
\pgfpathlineto{\pgfqpoint{2.356897in}{0.604139in}}%
\pgfpathlineto{\pgfqpoint{2.382182in}{0.602772in}}%
\pgfpathlineto{\pgfqpoint{2.407468in}{0.601584in}}%
\pgfpathlineto{\pgfqpoint{2.432753in}{0.600554in}}%
\pgfpathlineto{\pgfqpoint{2.458038in}{0.599662in}}%
\pgfpathlineto{\pgfqpoint{2.483324in}{0.598890in}}%
\pgfpathlineto{\pgfqpoint{2.508609in}{0.598224in}}%
\pgfpathlineto{\pgfqpoint{2.533894in}{0.597649in}}%
\pgfpathlineto{\pgfqpoint{2.559180in}{0.597154in}}%
\pgfpathlineto{\pgfqpoint{2.584465in}{0.596730in}}%
\pgfpathlineto{\pgfqpoint{2.609750in}{0.596367in}}%
\pgfpathlineto{\pgfqpoint{2.635036in}{0.596057in}}%
\pgfpathlineto{\pgfqpoint{2.660321in}{0.595795in}}%
\pgfpathlineto{\pgfqpoint{2.685607in}{0.595573in}}%
\pgfpathlineto{\pgfqpoint{2.710892in}{0.595387in}}%
\pgfpathlineto{\pgfqpoint{2.736177in}{0.595231in}}%
\pgfpathlineto{\pgfqpoint{2.761463in}{0.595104in}}%
\pgfpathlineto{\pgfqpoint{2.786748in}{0.594999in}}%
\pgfpathlineto{\pgfqpoint{2.812033in}{0.594916in}}%
\pgfpathlineto{\pgfqpoint{2.837319in}{0.594850in}}%
\pgfpathlineto{\pgfqpoint{2.862604in}{0.594799in}}%
\pgfpathlineto{\pgfqpoint{2.887889in}{0.594763in}}%
\pgfpathlineto{\pgfqpoint{2.913175in}{0.594737in}}%
\pgfpathlineto{\pgfqpoint{2.938460in}{0.594722in}}%
\pgfpathlineto{\pgfqpoint{2.963745in}{0.594716in}}%
\pgfpathlineto{\pgfqpoint{2.989031in}{0.594717in}}%
\pgfpathlineto{\pgfqpoint{3.014316in}{0.594724in}}%
\pgfpathlineto{\pgfqpoint{3.039601in}{0.594737in}}%
\pgfpathlineto{\pgfqpoint{3.064887in}{0.594754in}}%
\pgfpathlineto{\pgfqpoint{3.090172in}{0.594775in}}%
\pgfpathlineto{\pgfqpoint{3.115458in}{0.594800in}}%
\pgfpathlineto{\pgfqpoint{3.140743in}{0.594827in}}%
\pgfpathlineto{\pgfqpoint{3.166028in}{0.594856in}}%
\pgfpathlineto{\pgfqpoint{3.191314in}{0.594887in}}%
\pgfpathlineto{\pgfqpoint{3.216599in}{0.594919in}}%
\pgfpathlineto{\pgfqpoint{3.241884in}{0.594953in}}%
\pgfpathlineto{\pgfqpoint{3.267170in}{0.594987in}}%
\pgfpathlineto{\pgfqpoint{3.292455in}{0.595022in}}%
\pgfpathlineto{\pgfqpoint{3.317740in}{0.595057in}}%
\pgfpathlineto{\pgfqpoint{3.343026in}{0.595092in}}%
\pgfusepath{stroke}%
\end{pgfscope}%
\begin{pgfscope}%
\pgfpathrectangle{\pgfqpoint{0.708026in}{0.526079in}}{\pgfqpoint{2.635000in}{1.510000in}} %
\pgfusepath{clip}%
\pgfsetbuttcap%
\pgfsetroundjoin%
\pgfsetlinewidth{0.501875pt}%
\definecolor{currentstroke}{rgb}{0.000000,0.000000,0.000000}%
\pgfsetstrokecolor{currentstroke}%
\pgfsetdash{{1.850000pt}{0.800000pt}}{0.000000pt}%
\pgfpathmoveto{\pgfqpoint{0.694137in}{0.596518in}}%
\pgfpathlineto{\pgfqpoint{0.727988in}{0.596518in}}%
\pgfpathlineto{\pgfqpoint{0.767912in}{0.596518in}}%
\pgfpathlineto{\pgfqpoint{0.807836in}{0.596518in}}%
\pgfpathlineto{\pgfqpoint{0.847761in}{0.596518in}}%
\pgfpathlineto{\pgfqpoint{0.887685in}{0.596518in}}%
\pgfpathlineto{\pgfqpoint{0.927609in}{0.596518in}}%
\pgfpathlineto{\pgfqpoint{0.967533in}{0.596518in}}%
\pgfpathlineto{\pgfqpoint{1.007458in}{0.596518in}}%
\pgfpathlineto{\pgfqpoint{1.047382in}{0.596518in}}%
\pgfpathlineto{\pgfqpoint{1.087306in}{0.596518in}}%
\pgfpathlineto{\pgfqpoint{1.127230in}{0.596518in}}%
\pgfpathlineto{\pgfqpoint{1.167154in}{0.596518in}}%
\pgfpathlineto{\pgfqpoint{1.207079in}{0.596518in}}%
\pgfpathlineto{\pgfqpoint{1.247003in}{0.596518in}}%
\pgfpathlineto{\pgfqpoint{1.286927in}{0.596518in}}%
\pgfpathlineto{\pgfqpoint{1.326851in}{0.596518in}}%
\pgfpathlineto{\pgfqpoint{1.366776in}{0.596518in}}%
\pgfpathlineto{\pgfqpoint{1.406700in}{0.596518in}}%
\pgfpathlineto{\pgfqpoint{1.446624in}{0.596518in}}%
\pgfpathlineto{\pgfqpoint{1.486548in}{0.596518in}}%
\pgfpathlineto{\pgfqpoint{1.526473in}{0.596518in}}%
\pgfpathlineto{\pgfqpoint{1.566397in}{0.596518in}}%
\pgfpathlineto{\pgfqpoint{1.606321in}{0.596518in}}%
\pgfpathlineto{\pgfqpoint{1.646245in}{0.596518in}}%
\pgfpathlineto{\pgfqpoint{1.686170in}{0.596518in}}%
\pgfpathlineto{\pgfqpoint{1.726094in}{0.596518in}}%
\pgfpathlineto{\pgfqpoint{1.766018in}{0.596518in}}%
\pgfpathlineto{\pgfqpoint{1.805942in}{0.596518in}}%
\pgfpathlineto{\pgfqpoint{1.845867in}{0.596518in}}%
\pgfpathlineto{\pgfqpoint{1.885791in}{0.596518in}}%
\pgfpathlineto{\pgfqpoint{1.925715in}{0.596518in}}%
\pgfpathlineto{\pgfqpoint{1.965639in}{0.596518in}}%
\pgfpathlineto{\pgfqpoint{2.005564in}{0.596518in}}%
\pgfpathlineto{\pgfqpoint{2.045488in}{0.596518in}}%
\pgfpathlineto{\pgfqpoint{2.085412in}{0.596518in}}%
\pgfpathlineto{\pgfqpoint{2.125336in}{0.596518in}}%
\pgfpathlineto{\pgfqpoint{2.165261in}{0.596518in}}%
\pgfpathlineto{\pgfqpoint{2.205185in}{0.596518in}}%
\pgfpathlineto{\pgfqpoint{2.245109in}{0.596518in}}%
\pgfpathlineto{\pgfqpoint{2.285033in}{0.596518in}}%
\pgfpathlineto{\pgfqpoint{2.324958in}{0.596518in}}%
\pgfpathlineto{\pgfqpoint{2.364882in}{0.596518in}}%
\pgfpathlineto{\pgfqpoint{2.404806in}{0.596518in}}%
\pgfpathlineto{\pgfqpoint{2.444730in}{0.596518in}}%
\pgfpathlineto{\pgfqpoint{2.484654in}{0.596518in}}%
\pgfpathlineto{\pgfqpoint{2.524579in}{0.596518in}}%
\pgfpathlineto{\pgfqpoint{2.564503in}{0.596518in}}%
\pgfpathlineto{\pgfqpoint{2.604427in}{0.596518in}}%
\pgfpathlineto{\pgfqpoint{2.644351in}{0.596518in}}%
\pgfpathlineto{\pgfqpoint{2.684276in}{0.596518in}}%
\pgfpathlineto{\pgfqpoint{2.724200in}{0.596518in}}%
\pgfpathlineto{\pgfqpoint{2.764124in}{0.596518in}}%
\pgfpathlineto{\pgfqpoint{2.804048in}{0.596518in}}%
\pgfpathlineto{\pgfqpoint{2.843973in}{0.596518in}}%
\pgfpathlineto{\pgfqpoint{2.883897in}{0.596518in}}%
\pgfpathlineto{\pgfqpoint{2.923821in}{0.596518in}}%
\pgfpathlineto{\pgfqpoint{2.963745in}{0.596518in}}%
\pgfpathlineto{\pgfqpoint{3.003670in}{0.596518in}}%
\pgfpathlineto{\pgfqpoint{3.043594in}{0.596518in}}%
\pgfpathlineto{\pgfqpoint{3.083518in}{0.596518in}}%
\pgfpathlineto{\pgfqpoint{3.123442in}{0.596518in}}%
\pgfpathlineto{\pgfqpoint{3.163367in}{0.596518in}}%
\pgfpathlineto{\pgfqpoint{3.203291in}{0.596518in}}%
\pgfpathlineto{\pgfqpoint{3.243215in}{0.596518in}}%
\pgfpathlineto{\pgfqpoint{3.283139in}{0.596518in}}%
\pgfpathlineto{\pgfqpoint{3.323064in}{0.596518in}}%
\pgfpathlineto{\pgfqpoint{3.356915in}{0.596518in}}%
\pgfusepath{stroke}%
\end{pgfscope}%
\begin{pgfscope}%
\pgfsetrectcap%
\pgfsetmiterjoin%
\pgfsetlinewidth{0.803000pt}%
\definecolor{currentstroke}{rgb}{0.000000,0.000000,0.000000}%
\pgfsetstrokecolor{currentstroke}%
\pgfsetdash{}{0pt}%
\pgfpathmoveto{\pgfqpoint{0.708026in}{0.526079in}}%
\pgfpathlineto{\pgfqpoint{0.708026in}{2.036079in}}%
\pgfusepath{stroke}%
\end{pgfscope}%
\begin{pgfscope}%
\pgfsetrectcap%
\pgfsetmiterjoin%
\pgfsetlinewidth{0.803000pt}%
\definecolor{currentstroke}{rgb}{0.000000,0.000000,0.000000}%
\pgfsetstrokecolor{currentstroke}%
\pgfsetdash{}{0pt}%
\pgfpathmoveto{\pgfqpoint{3.343026in}{0.526079in}}%
\pgfpathlineto{\pgfqpoint{3.343026in}{2.036079in}}%
\pgfusepath{stroke}%
\end{pgfscope}%
\begin{pgfscope}%
\pgfsetrectcap%
\pgfsetmiterjoin%
\pgfsetlinewidth{0.803000pt}%
\definecolor{currentstroke}{rgb}{0.000000,0.000000,0.000000}%
\pgfsetstrokecolor{currentstroke}%
\pgfsetdash{}{0pt}%
\pgfpathmoveto{\pgfqpoint{0.708026in}{0.526079in}}%
\pgfpathlineto{\pgfqpoint{3.343026in}{0.526079in}}%
\pgfusepath{stroke}%
\end{pgfscope}%
\begin{pgfscope}%
\pgfsetrectcap%
\pgfsetmiterjoin%
\pgfsetlinewidth{0.803000pt}%
\definecolor{currentstroke}{rgb}{0.000000,0.000000,0.000000}%
\pgfsetstrokecolor{currentstroke}%
\pgfsetdash{}{0pt}%
\pgfpathmoveto{\pgfqpoint{0.708026in}{2.036079in}}%
\pgfpathlineto{\pgfqpoint{3.343026in}{2.036079in}}%
\pgfusepath{stroke}%
\end{pgfscope}%
\begin{pgfscope}%
\pgfsetbuttcap%
\pgfsetmiterjoin%
\definecolor{currentfill}{rgb}{1.000000,1.000000,1.000000}%
\pgfsetfillcolor{currentfill}%
\pgfsetfillopacity{0.800000}%
\pgfsetlinewidth{1.003750pt}%
\definecolor{currentstroke}{rgb}{0.800000,0.800000,0.800000}%
\pgfsetstrokecolor{currentstroke}%
\pgfsetstrokeopacity{0.800000}%
\pgfsetdash{}{0pt}%
\pgfpathmoveto{\pgfqpoint{2.199213in}{1.721111in}}%
\pgfpathlineto{\pgfqpoint{3.245803in}{1.721111in}}%
\pgfpathquadraticcurveto{\pgfqpoint{3.273581in}{1.721111in}}{\pgfqpoint{3.273581in}{1.748889in}}%
\pgfpathlineto{\pgfqpoint{3.273581in}{1.938857in}}%
\pgfpathquadraticcurveto{\pgfqpoint{3.273581in}{1.966635in}}{\pgfqpoint{3.245803in}{1.966635in}}%
\pgfpathlineto{\pgfqpoint{2.199213in}{1.966635in}}%
\pgfpathquadraticcurveto{\pgfqpoint{2.171436in}{1.966635in}}{\pgfqpoint{2.171436in}{1.938857in}}%
\pgfpathlineto{\pgfqpoint{2.171436in}{1.748889in}}%
\pgfpathquadraticcurveto{\pgfqpoint{2.171436in}{1.721111in}}{\pgfqpoint{2.199213in}{1.721111in}}%
\pgfpathclose%
\pgfusepath{stroke,fill}%
\end{pgfscope}%
\begin{pgfscope}%
\pgfsetrectcap%
\pgfsetroundjoin%
\pgfsetlinewidth{1.003750pt}%
\definecolor{currentstroke}{rgb}{0.627451,0.321569,0.176471}%
\pgfsetstrokecolor{currentstroke}%
\pgfsetdash{}{0pt}%
\pgfpathmoveto{\pgfqpoint{2.226991in}{1.854167in}}%
\pgfpathlineto{\pgfqpoint{2.504769in}{1.854167in}}%
\pgfusepath{stroke}%
\end{pgfscope}%
\begin{pgfscope}%
\pgftext[x=2.615880in,y=1.805556in,left,base]{\rmfamily\fontsize{10.000000}{12.000000}\selectfont \(\displaystyle \mathrm{R}\) - wave}%
\end{pgfscope}%
\end{pgfpicture}%
\makeatother%
\endgroup%

\caption{(a) Real part $\omega_\mr{r}=\mr{Re}(\omega)$ of numerical solutions of the dispersion relation (\ref{eq_dispersion_relation}) for parameters $\Omega_\mr{pe}=2|\Omega_\mr{ce}|$, $\nu_\mr{h}=0.005$, $\vpar=0.2c$ and $\vperp=0.6c$. (b) Corresponding imaginary parts $\gamma=\mr{Im}(\omega)$. Here, only the solution corresponding to the R-wave below the electron cyclotron frequency $|\Omega_\mr{ce}|$ is shown since the imaginary parts of the other two branches are close to zero.\label{fig_solutions_dispersion}}
\end{figure}
The other two solutions correspond to right-handed (R) and left-handed (L) circularly polarized waves (transversal waves with perturbations perpendicular to the background magnetic field only), respectively. The dispersion relation for these types of waves for an arbitrary hot electron equilibrium distribution function $f_\mr{h}^0$ reads \citep{Brambilla1998, Xiaoetal1998}
\begin{align}
0=D_{\mr{R/L}}(k,\omega)=1-\frac{c^2k^2}{\omega^2}-\frac{\Omega_\mr{pe}^2}{\omega(\omega\pm\Omega_\mr{ce})}+\nu_\mr{h}\frac{\Omega_\mr{pe}^2}{\omega}\int\frac{v_\perp}{2}\frac{\hat{G}F_\mr{h}^0}{\omega\pm\Omega_\mr{ce}-kv_\parallel}\mr{d}^3\mb{v},\label{eq_dispersion_relation_general}
\end{align}
where $\nu_\mr{h}=n_{\mr{h}0}/n_{\mr{c}0}$ is the ratio between hot and cold electron number densities, $\mr{d}^3\mb{v}=2\pi v_\perp\mr{d}v_\parallel v_\perp$, $F_\mr{h}^0$ the velocity part of the equilibrium distribution function, i.e. $f_\mr{h}^0(v_\perp,v_\parallel)=n_{\mr{h}0}F_\mr{h}^0(v_\perp,v_\parallel)$ and $\hat{G}$ is a differential operator measuring the anisotropy of the distribution function in velocity space:
\begin{align}
\hat{G}=\frac{\pa}{\pa v_\perp}+\frac{k}{\omega}\left(v_\perp\frac{\pa}{\pa v_\parallel}-v_\parallel\frac{\pa}{\pa v_\perp}\right).
\end{align}
In order to satisfy the steady-state Vlasov equation with the background magnetic field $\mb{B}_0$, it is straightforward to show that the equilibrium distribution function must be rotationally symmetric around the magnetic field and therefore only depends on $v_\perp^2=v_x^2+v_y^2$ and $v_\parallel=v_z$. For the special case of an anisotropic Maxwellian with generally different thermal velocities in parallel and perpendicular direction,
\begin{align}
F_\mr{h}^0(v_\perp,v_\parallel)=\frac{1}{(2\pi)^{3/2}v_{\mr{th}\parallel}v_{\mr{th}\perp}^2}\exp\left(-\frac{v_\perp^2}{2v_{\mr{th}\perp}^2}-\frac{v_\parallel^2}{2v_{\mr{th}\parallel}^2}\right),\label{eq_anisotropic_Maxwellian}
\end{align} 
the dispersion relation (\ref{eq_dispersion_relation_general}) transfers to
\begin{align}
0=D_{\mr{R/L}}(k,\omega)=1-\frac{c^2k^2}{\omega^2}-\frac{\Omega_\mr{pe}^2}{\omega(\omega\pm\Omega_\mr{ce})}+\nu_\mr{h}\frac{\Omega_\mr{pe}^2}{\omega^2}\left[\frac{\omega}{k\sqrt{2}v_{\mr{th}\parallel}}Z(\xi^\pm)-\left(1-\frac{v_{\mr{th}\perp}^2}{v_{\mr{th}\parallel}^2}\right)(1+\xi^\pm Z(\xi^\pm))\right],\label{eq_dispersion_relation}
\end{align}
where $\xi^\pm=(\omega\pm\Omega_\mr{ce})/k\sqrt{2}v_{\mr{th}\parallel}$ and $Z$ is the plasma dispersion function \citep{Friedetal1961} given by
\begin{align}
Z(\xi)=\sqrt{\pi}\mr{e}^{-\xi^2}\left(i-\frac{2}{\sqrt{\pi}}\int_0^\xi\mr{e}^{t^2}\mr{d}t\right)=\sqrt{\pi}\mr{e}^{-\xi^2}(i-\mr{erfi}(\xi))\label{eq_plasma_dispersion_function}.
\end{align}
In the absence of energetic electrons ($\nu_\mr{h}\rightarrow0$), the dispersion relation (\ref{eq_dispersion_relation}) transfers to the well-known cold plasma dispersion relation for electron waves, which only provides solutions with real oscillation frequencies $\omega_r:=\mr{Re}(\omega)$ for all wavenumbers $k$. This means that there is no wave growth or damping due to an imaginary part $\gamma:=\mr{Im}(\omega)$. However, depending on the temperature anisotropy of $F_\mr{h}^0$, the dispersion relation (\ref{eq_dispersion_relation}) provides solutions with $\gamma\neq0$ which is shown in Fig. \ref{fig_solutions_dispersion}, where we plot the real frequency $\omega_\mr{r}$ on the left-hand side and the growth rate $\gamma$ on the right-hand side. One can see that there are two solutions for R-waves and one solution for L-waves, which is known from the cold plasma theory \citep{Brambilla1998}. However, due to interaction of waves with fast electrons that meet the resonance condition $\omega=kv_\parallel\mp\Omega_\mr{ce}$, the lower branch below the electron cyclotron frequency becomes unstable for a certain range of wave numbers if the temperature anisotropy is sufficiently large.

We shall use these results for the verification of the developed numerical algorithms.

\section{Numerical methods}
\label{sec_numerical_methods}
In this section, we apply two kinds of numerical methods on the electron hybrid model which we have just discussed on the continuous level and for which the linear dispersion relation (\ref{eq_dispersion_relation}) is available. Since the latter corresponds to transverse electromagnetic waves, which, in the linear phase, are completely decoupled from longitudinal electrostatic waves, we neglect the $z$-components of the fields $\tilde{\mb{E}}$, $\tilde{\mb{B}}$ and $\tilde{\mb{j}}_\mr{c}$ in the model (\ref{eq_model_linearized}) and only solve for $x$- and $y$-components while retaining all velocity components in the kinetic equation. We start with an intuitive application of a combination of classical finite elements for solving field equations and the classical PIC method for solving the Vlasov equation followed by applying structure-preserving finite element PIC methods. 

\subsection{Standard finite element particle-in-cell}
\label{sec_standard}
As a first step, we write the momentum balance equation (\ref{eq_model_linearized_1}), Faraday's law (\ref{eq_model_linearized_3}) and Amp\'{e}re's law (\ref{eq_model_linearized_4}) in the compact form
\begin{subequations}
\label{eq_compact}
\begin{align}
\begin{cases}
\displaystyle\frac{\pa\mb{U}}{\pa t}+A_1\frac{\pa\mb{U}}{\pa z}+A_2\mb{U}=\mb{S},\vspace{0.2cm}\\
\displaystyle\mb{U}(0,t)=\mb{U}(L,t),\quad\mb{U}(z,t=0)=\mb{U}_0(z)
\end{cases}
\end{align}
\end{subequations}
for the vector of unknowns $\mb{U}=(\tilde{E}_x,\tilde{E}_y,\tilde{B}_x,\tilde{B}_y,\tilde{j}_{\mr{c}x},\tilde{j}_{\mr{c}y})$ with initial condition $\mb{U}_0$ and impose periodic boundary conditions on the domain $\Omega=(0,L)$, where $L$ is the length of the computational domain. The constant matrices $A_1,A_2\in\mathbb{R}^{6\times6}$ and the source term $\mb{S}$ are
\begin{subequations}
\begin{align}
A_1&=
\begin{pmatrix}
0 &0  &0 &c^2  &0 &0 \\
0 &0  &-c^2 &0 &0 &0 \\
0 &-1  &0 &0 &0 &0  \\
1 &0  &0 &0 &0 &0  \\
0 &0  &0 &0 &0 &0   \\
0 &0  &0 &0 &0 &0 
\end{pmatrix},
\end{align}
\begin{align}
A_2&=
\begin{pmatrix}
0 &0 &0 &0 &\mu_0c^2 &0 \\
0 &0 &0 &0 &0 &\mu_0c^2 \\
0 &0 &0 &0 &0 &0 \\
0 &0 &0 &0 &0 &0 \\
-\epsilon_0\Omega_{\mr{pe}}^2 &0 &0 &0 &0 &-\Omega_{\mr{ce}} \\
0 &-\epsilon_0\Omega_{\mr{pe}}^2 &0 &0 &\Omega_{\mr{ce}} &0 \\
\end{pmatrix},
\end{align}
\begin{align}
\textbf{S}&=
\begin{pmatrix}
-\mu_0c^2 j_{\mr{h}x} \\
-\mu_0c^2 j_{\mr{h}y} \\
0 \\
0 \\
0 \\
0
\end{pmatrix}\label{eq_source_term}.
\end{align}
\end{subequations}

\textbf{Semi-discretization in space}. Following classical finite element methods (see \citep{Doneaetal2003}, for instance), one assumes $\textbf{U}\in(H^1(\Omega))^6$, which means that all the unknown functions contained in $\textbf{U}$ are elements of the same space $H^1(\Omega)=\{u\in L^2(\Omega),\pa u/\pa z\in L^2(\Omega)\}$ with $L^2(\Omega)$ being the space of square integrable functions in the domain $\Omega$. Furthermore, the problem given in strong form is transformed into an equivalent weak formulation by multiplying the equations with a test function $V\in H^1$ (we shall use the allocations $H^1(\Omega)\rightarrow H^1$ and $L^2(\Omega)\rightarrow L^2$ for a shorter notation) and integrating over the domain $\Omega$. In our case (\ref{eq_compact}), the weak formulation reads: Find $\textbf{U}\in(H^1(\Omega))^6$ such that
\begin{align}
\int_0^L\frac{\pa \mb{U}}{\pa t}V\mr{d}z+A_1\int_0^L\frac{\pa\mb{U}}{\pa z}V\mr{d}z+A_2\int_0^L\mb{U}V\mr{d}z=\int_0^L\mb{S}V\mr{d}z\quad\quad\forall\,V\in H^1.
\end{align}
As a next step, we replace the function space $H^1$ by a finite-dimensional subspace $\mathcal{S}_h\subset H^1$ in which we look for the approximate solution $\mb{U}_h$ of the problem (\ref{eq_compact}). In addition to that, we use the same subspace for the trial function $\mb{U}_h$ and the test function $V_h$ (Bubnov-Galerkin-method). This leads to the following discrete version of the above problem: Find $\textbf{U}_h\in(\mathcal{S}_h)^6$ such that
\begin{align}
\int_0^L\frac{\pa \mb{U}_h}{\pa t}V_h\mr{d}z+A_1\int_0^L\frac{\pa\mb{U}_h}{\pa z}V_h\mr{d}z+A_2\int_0^L\mb{U}_hV_h\mr{d}z=\int_0^L\mb{S}V_h\mr{d}z\quad\quad\forall\,V_h\in\mathcal{S}_h.\label{eq_weak_discrete}
\end{align}
Expanding trial and test function in a basis of $\mathcal{S}_h$ denoted by $(\varphi_j)_{j=0,\ldots,N-1}$, where $N$ is the dimension of $\mathcal{S}_h$, 
\begin{align}
\mb{U}_h(z,t)=\sum_{j=0}^{N-1}\mb{u}_j(t)\varphi_j(z),\quad\quad V_h(z)=\sum_{j=0}^{N-1}v_j\varphi_j(z),\label{eq_expansion}
\end{align}
and substituting these expressions in the discrete weak formulation (\ref{eq_weak_discrete}) yields
\begin{align}
\sum_{i,j=0}^{N-1}v_i\frac{\mr{d}\mb{u}_j}{\mr{d}t}\underbrace{\int_0^L\varphi_i\varphi_j\mr{d}z}_{=:m_{ij}}+A_1\sum_{i,j=0}^{N-1}v_i\mb{u}_j\underbrace{\int_0^L\varphi_i\varphi_j^\prime\mr{d}z}_{=:c_{ij}}+A_2\sum_{i,j=0}^{N-1}v_i\mb{u}_j\underbrace{\int_0^L\varphi_i\varphi_j\mr{d}z}_{=:m_{ij}}=\sum_{i=0}^{N-1}v_i\int_0^L\mb{S}\varphi_i\mr{d}z,\label{eq_matrix_formulation_1}
\end{align}
where we have defined the entries of the mass matrix $\mathbb{M}:=(m_{ij})_{i,j=0,\ldots,N-1}\in\mathbb{R}^{N\times N}$ and the advection matrix $\mathbb{C}:=(c_{ij})_{i,j=0,\ldots,N-1}\in\mathbb{R}^{N\times N}$. With this, (\ref{eq_matrix_formulation_1}) can be expressed equivalently in the following semi-discrete block matrix form:
\begin{align}
\mathbb{V}\mathbb{M}_\mr{b}\frac{\mr{d}\mb{u}}{\mr{d}t}+\mathbb{V}\tilde{\mathbb{C}}\mb{u}+\mathbb{V}\tilde{\mathbb{M}}\mb{u}=\mathbb{V}\mathbb{S}.\label{eq_matrix_formulation_2}
\end{align}
In this matrix formulation, the vector $\mb{u}$ contains all the unknown finite element coefficients of the expansion (\ref{eq_expansion}), $\mb{u}=(\mb{u}_0,\mb{u}_1,\ldots,\mb{u}_{N-1})^\top$, and every $\mb{u}_j=(e_{xj},e_{yj},b_{xj},b_{yj},j_{cxj},j_{cyj})$ contains the respective coefficients of all six physical quantities which makes $\mb{u}\in\mathrm{R}^{6N}$. The block matrix $\mathbb{V}$ for the coefficients of the test function $V_h$ is
\begin{align}
\mathbb{V}:=\begin{pmatrix}
v_0I_6 &0 &\cdots &0 \\
0 &v_1I_6 &\cdots &0 \\
\vdots &\vdots &\ddots &\vdots \\
0 &0 &\cdots &v_{N-1}I_6\\
\end{pmatrix}\quad\in\mathbb{R}^{6N\times6N},\label{eq_matrix_test}
\end{align} 
where $I_6$ denotes the $6\times6$ identity matrix. Furthermore, we introduced the block matrices $\mathbb{M}_\mr{b}:=\mathbb{M}\otimes I_6\in\mathbb{R}^{6N\times 6N}$, $\tilde{\mathbb{C}}:=\mathbb{C}\otimes A_1\in\mathbb{R}^{6N\times 6N}$ and $\tilde{\mathbb{M}}:=\mathbb{M}\otimes A_2\in\mathbb{R}^{6N\times 6N}$. The vector $\mathbb{S}$ is given by
\begin{align}
&\mathbb{S}:=\begin{pmatrix}
\int_0^L\mathbf{S}\varphi_0(z)\text{d}z \\
\vdots \\
\int_0^L\mathbf{S}\varphi_{N-1}(z)\text{d}z
\end{pmatrix}\quad\in\mathbb{R}^{6N}.\label{eq_def_righthandside}
\end{align}
Since we want (\ref{eq_matrix_formulation_2}) to be true for all $\mathbb{V}$ of the form (\ref{eq_matrix_test}), we finally end up with the semi-discrete system
\begin{align}
\mathbb{M}_\mr{b}\frac{\mr{d}\mb{u}}{\mr{d}t}=-\tilde{\mathbb{C}}\mb{u}-\tilde{\mathbb{M}}\mb{u}+\mathbb{S}\label{eq_semi_discrete_system}
\end{align}
for the time evolution of all finite element coefficients $\mb{u}\in\mathbb{R}^{6N}$.

\textbf{Discretization in time}. Having done the spatial discretization, the next step is to apply a time stepping scheme on system (\ref{eq_semi_discrete_system}). Here, we use a second-order Crank-Nicolson scheme \citep{Cranketal1947} which consists of applying a mid-point rule on the quantities on the right-hand side. Denoting the time step by $n$, i.e. $t_n=n\Delta t$, the fully discrete matrix formulation for advancing $\mb{u}^n\rightarrow\mb{u}^{n+1}$ then reads
\begin{align}
\left(\mathbb{M}_\mr{b}+\frac{1}{2}\Delta t\tilde{\mathbb{C}}+\frac{1}{2}\Delta t\tilde{\mathbb{M}}\right)\mb{u}^{n+1}=\left(\mathbb{M}_\mr{b}-\frac{1}{2}\Delta t\tilde{\mathbb{C}}-\frac{1}{2}\Delta t\tilde{\mathbb{M}}\right)\mb{u}^n+\frac{1}{2}\Delta t\left(\mathbb{S}^{n+1}+\mathbb{S}^n\right).\label{eq_Crank_Nicolson}
\end{align}
We immediately see that this time stepping scheme involves the inversion of a large matrix on the left-hand side. However, this must be done only once in the very beginning of a simulation.

\textbf{Basis functions}. Let us now construct a basis of the finite-dimensional subspace $\mathcal{S}_h$ with $\dim\mathcal{S}_h=N$. We do this with a family of B-splines \citep{Ratnanietal2012}, which are piecewise polynomials of degree $p$. The set of basis functions is fully determined by a sequence of $m+1$ points (or knots) $0=z_0\leq z_1\leq\ldots\leq z_m=L$ which defines a knot vector $T=(z_0,z_1,\ldots,z_m)$. For degree $p=0$ the basis functions $(\varphi_j^{p=0})_{j=0,\ldots,m-1}$ are defined by
\begin{align}
\varphi_{j}^0(z)=\begin{cases}
1\quad z\in [z_j,\,z_{j+1})\\0 \quad\text{else}.
\end{cases}\label{eq_def_Bsplines_0}
\end{align}
Higher degrees are defined by the following recursion formula:
\begin{align}
\varphi_j^p(z)=w_j^p(z)\varphi_j^{p-1}(z)+(1-w_{j+1}^p)\varphi^{p-1}_{j+1}(z), \quad\quad w_j^p(z)=\frac{z-z_j}{z_{j+p}-z_j}.\label{eq_def_Bsplines_higher}
\end{align}
If the knot vector $T$ contains $r$ repeated knots one says that this knot has multiplicity $r$. Using multiple knots at the boundaries enables the application of Dirichlet boundary conditions by enforcing all the interior splines to vanish at the boundaries and setting the first and last spline there to one. This can be achieved by using $r=p+1$ equal knots for the left and right boundary, respectively. In this case $\dim\mathcal{S}_h=m-p$. However, since we are using periodic boundary conditions, we need a periodic basis. This can be achieved by extending the knot vector over the boundaries by $p$ additional points. The result is shown in Fig. \ref{fig_Bsplines_periodic} for generic degrees $p=1$ and $p=2$. In this case $\dim\mathcal{S}_h=m-2p$. Note in Fig. \ref{fig_Bsplines_periodic}, that B-splines which leave the domain at one boundary come back at the other boundary which can be seen by the respective color codings.
\begin{figure}[!t]
\centering
\includegraphics[scale=1]{01_Figures/Bsplines_p=1.pdf}
\includegraphics[scale=1]{01_Figures/Bsplines_p=2.pdf}
%\input{01_Figures/Bsplines_p=1.pgf}
%%% Creator: Matplotlib, PGF backend
%%
%% To include the figure in your LaTeX document, write
%%   \input{<filename>.pgf}
%%
%% Make sure the required packages are loaded in your preamble
%%   \usepackage{pgf}
%%
%% Figures using additional raster images can only be included by \input if
%% they are in the same directory as the main LaTeX file. For loading figures
%% from other directories you can use the `import` package
%%   \usepackage{import}
%% and then include the figures with
%%   \import{<path to file>}{<filename>.pgf}
%%
%% Matplotlib used the following preamble
%%   \usepackage{fontspec}
%%   \setmainfont{DejaVu Serif}
%%   \setsansfont{DejaVu Sans}
%%   \setmonofont{DejaVu Sans Mono}
%%
\begingroup%
\makeatletter%
\begin{pgfpicture}%
\pgfpathrectangle{\pgfpointorigin}{\pgfqpoint{3.198427in}{2.331214in}}%
\pgfusepath{use as bounding box, clip}%
\begin{pgfscope}%
\pgfsetbuttcap%
\pgfsetmiterjoin%
\definecolor{currentfill}{rgb}{1.000000,1.000000,1.000000}%
\pgfsetfillcolor{currentfill}%
\pgfsetlinewidth{0.000000pt}%
\definecolor{currentstroke}{rgb}{1.000000,1.000000,1.000000}%
\pgfsetstrokecolor{currentstroke}%
\pgfsetdash{}{0pt}%
\pgfpathmoveto{\pgfqpoint{0.000000in}{0.000000in}}%
\pgfpathlineto{\pgfqpoint{3.198427in}{0.000000in}}%
\pgfpathlineto{\pgfqpoint{3.198427in}{2.331214in}}%
\pgfpathlineto{\pgfqpoint{0.000000in}{2.331214in}}%
\pgfpathclose%
\pgfusepath{fill}%
\end{pgfscope}%
\begin{pgfscope}%
\pgfsetbuttcap%
\pgfsetmiterjoin%
\definecolor{currentfill}{rgb}{1.000000,1.000000,1.000000}%
\pgfsetfillcolor{currentfill}%
\pgfsetlinewidth{0.000000pt}%
\definecolor{currentstroke}{rgb}{0.000000,0.000000,0.000000}%
\pgfsetstrokecolor{currentstroke}%
\pgfsetstrokeopacity{0.000000}%
\pgfsetdash{}{0pt}%
\pgfpathmoveto{\pgfqpoint{0.374692in}{0.521603in}}%
\pgfpathlineto{\pgfqpoint{3.009692in}{0.521603in}}%
\pgfpathlineto{\pgfqpoint{3.009692in}{2.182603in}}%
\pgfpathlineto{\pgfqpoint{0.374692in}{2.182603in}}%
\pgfpathclose%
\pgfusepath{fill}%
\end{pgfscope}%
\begin{pgfscope}%
\pgfsetbuttcap%
\pgfsetroundjoin%
\definecolor{currentfill}{rgb}{0.000000,0.000000,0.000000}%
\pgfsetfillcolor{currentfill}%
\pgfsetlinewidth{0.803000pt}%
\definecolor{currentstroke}{rgb}{0.000000,0.000000,0.000000}%
\pgfsetstrokecolor{currentstroke}%
\pgfsetdash{}{0pt}%
\pgfsys@defobject{currentmarker}{\pgfqpoint{0.000000in}{-0.048611in}}{\pgfqpoint{0.000000in}{0.000000in}}{%
\pgfpathmoveto{\pgfqpoint{0.000000in}{0.000000in}}%
\pgfpathlineto{\pgfqpoint{0.000000in}{-0.048611in}}%
\pgfusepath{stroke,fill}%
}%
\begin{pgfscope}%
\pgfsys@transformshift{0.374692in}{0.521603in}%
\pgfsys@useobject{currentmarker}{}%
\end{pgfscope}%
\end{pgfscope}%
\begin{pgfscope}%
\pgftext[x=0.374692in,y=0.424381in,,top]{\rmfamily\fontsize{10.000000}{12.000000}\selectfont \(\displaystyle 0.0\)}%
\end{pgfscope}%
\begin{pgfscope}%
\pgfsetbuttcap%
\pgfsetroundjoin%
\definecolor{currentfill}{rgb}{0.000000,0.000000,0.000000}%
\pgfsetfillcolor{currentfill}%
\pgfsetlinewidth{0.803000pt}%
\definecolor{currentstroke}{rgb}{0.000000,0.000000,0.000000}%
\pgfsetstrokecolor{currentstroke}%
\pgfsetdash{}{0pt}%
\pgfsys@defobject{currentmarker}{\pgfqpoint{0.000000in}{-0.048611in}}{\pgfqpoint{0.000000in}{0.000000in}}{%
\pgfpathmoveto{\pgfqpoint{0.000000in}{0.000000in}}%
\pgfpathlineto{\pgfqpoint{0.000000in}{-0.048611in}}%
\pgfusepath{stroke,fill}%
}%
\begin{pgfscope}%
\pgfsys@transformshift{0.901692in}{0.521603in}%
\pgfsys@useobject{currentmarker}{}%
\end{pgfscope}%
\end{pgfscope}%
\begin{pgfscope}%
\pgftext[x=0.901692in,y=0.424381in,,top]{\rmfamily\fontsize{10.000000}{12.000000}\selectfont \(\displaystyle 0.2\)}%
\end{pgfscope}%
\begin{pgfscope}%
\pgfsetbuttcap%
\pgfsetroundjoin%
\definecolor{currentfill}{rgb}{0.000000,0.000000,0.000000}%
\pgfsetfillcolor{currentfill}%
\pgfsetlinewidth{0.803000pt}%
\definecolor{currentstroke}{rgb}{0.000000,0.000000,0.000000}%
\pgfsetstrokecolor{currentstroke}%
\pgfsetdash{}{0pt}%
\pgfsys@defobject{currentmarker}{\pgfqpoint{0.000000in}{-0.048611in}}{\pgfqpoint{0.000000in}{0.000000in}}{%
\pgfpathmoveto{\pgfqpoint{0.000000in}{0.000000in}}%
\pgfpathlineto{\pgfqpoint{0.000000in}{-0.048611in}}%
\pgfusepath{stroke,fill}%
}%
\begin{pgfscope}%
\pgfsys@transformshift{1.428692in}{0.521603in}%
\pgfsys@useobject{currentmarker}{}%
\end{pgfscope}%
\end{pgfscope}%
\begin{pgfscope}%
\pgftext[x=1.428692in,y=0.424381in,,top]{\rmfamily\fontsize{10.000000}{12.000000}\selectfont \(\displaystyle 0.4\)}%
\end{pgfscope}%
\begin{pgfscope}%
\pgfsetbuttcap%
\pgfsetroundjoin%
\definecolor{currentfill}{rgb}{0.000000,0.000000,0.000000}%
\pgfsetfillcolor{currentfill}%
\pgfsetlinewidth{0.803000pt}%
\definecolor{currentstroke}{rgb}{0.000000,0.000000,0.000000}%
\pgfsetstrokecolor{currentstroke}%
\pgfsetdash{}{0pt}%
\pgfsys@defobject{currentmarker}{\pgfqpoint{0.000000in}{-0.048611in}}{\pgfqpoint{0.000000in}{0.000000in}}{%
\pgfpathmoveto{\pgfqpoint{0.000000in}{0.000000in}}%
\pgfpathlineto{\pgfqpoint{0.000000in}{-0.048611in}}%
\pgfusepath{stroke,fill}%
}%
\begin{pgfscope}%
\pgfsys@transformshift{1.955692in}{0.521603in}%
\pgfsys@useobject{currentmarker}{}%
\end{pgfscope}%
\end{pgfscope}%
\begin{pgfscope}%
\pgftext[x=1.955692in,y=0.424381in,,top]{\rmfamily\fontsize{10.000000}{12.000000}\selectfont \(\displaystyle 0.6\)}%
\end{pgfscope}%
\begin{pgfscope}%
\pgfsetbuttcap%
\pgfsetroundjoin%
\definecolor{currentfill}{rgb}{0.000000,0.000000,0.000000}%
\pgfsetfillcolor{currentfill}%
\pgfsetlinewidth{0.803000pt}%
\definecolor{currentstroke}{rgb}{0.000000,0.000000,0.000000}%
\pgfsetstrokecolor{currentstroke}%
\pgfsetdash{}{0pt}%
\pgfsys@defobject{currentmarker}{\pgfqpoint{0.000000in}{-0.048611in}}{\pgfqpoint{0.000000in}{0.000000in}}{%
\pgfpathmoveto{\pgfqpoint{0.000000in}{0.000000in}}%
\pgfpathlineto{\pgfqpoint{0.000000in}{-0.048611in}}%
\pgfusepath{stroke,fill}%
}%
\begin{pgfscope}%
\pgfsys@transformshift{2.482692in}{0.521603in}%
\pgfsys@useobject{currentmarker}{}%
\end{pgfscope}%
\end{pgfscope}%
\begin{pgfscope}%
\pgftext[x=2.482692in,y=0.424381in,,top]{\rmfamily\fontsize{10.000000}{12.000000}\selectfont \(\displaystyle 0.8\)}%
\end{pgfscope}%
\begin{pgfscope}%
\pgfsetbuttcap%
\pgfsetroundjoin%
\definecolor{currentfill}{rgb}{0.000000,0.000000,0.000000}%
\pgfsetfillcolor{currentfill}%
\pgfsetlinewidth{0.803000pt}%
\definecolor{currentstroke}{rgb}{0.000000,0.000000,0.000000}%
\pgfsetstrokecolor{currentstroke}%
\pgfsetdash{}{0pt}%
\pgfsys@defobject{currentmarker}{\pgfqpoint{0.000000in}{-0.048611in}}{\pgfqpoint{0.000000in}{0.000000in}}{%
\pgfpathmoveto{\pgfqpoint{0.000000in}{0.000000in}}%
\pgfpathlineto{\pgfqpoint{0.000000in}{-0.048611in}}%
\pgfusepath{stroke,fill}%
}%
\begin{pgfscope}%
\pgfsys@transformshift{3.009692in}{0.521603in}%
\pgfsys@useobject{currentmarker}{}%
\end{pgfscope}%
\end{pgfscope}%
\begin{pgfscope}%
\pgftext[x=3.009692in,y=0.424381in,,top]{\rmfamily\fontsize{10.000000}{12.000000}\selectfont \(\displaystyle 1.0\)}%
\end{pgfscope}%
\begin{pgfscope}%
\pgftext[x=1.692192in,y=0.234413in,,top]{\rmfamily\fontsize{10.000000}{12.000000}\selectfont \(\displaystyle z\)}%
\end{pgfscope}%
\begin{pgfscope}%
\pgfsetbuttcap%
\pgfsetroundjoin%
\definecolor{currentfill}{rgb}{0.000000,0.000000,0.000000}%
\pgfsetfillcolor{currentfill}%
\pgfsetlinewidth{0.803000pt}%
\definecolor{currentstroke}{rgb}{0.000000,0.000000,0.000000}%
\pgfsetstrokecolor{currentstroke}%
\pgfsetdash{}{0pt}%
\pgfsys@defobject{currentmarker}{\pgfqpoint{-0.048611in}{0.000000in}}{\pgfqpoint{0.000000in}{0.000000in}}{%
\pgfpathmoveto{\pgfqpoint{0.000000in}{0.000000in}}%
\pgfpathlineto{\pgfqpoint{-0.048611in}{0.000000in}}%
\pgfusepath{stroke,fill}%
}%
\begin{pgfscope}%
\pgfsys@transformshift{0.374692in}{0.729228in}%
\pgfsys@useobject{currentmarker}{}%
\end{pgfscope}%
\end{pgfscope}%
\begin{pgfscope}%
\pgftext[x=0.100000in,y=0.676467in,left,base]{\rmfamily\fontsize{10.000000}{12.000000}\selectfont \(\displaystyle 0.0\)}%
\end{pgfscope}%
\begin{pgfscope}%
\pgfsetbuttcap%
\pgfsetroundjoin%
\definecolor{currentfill}{rgb}{0.000000,0.000000,0.000000}%
\pgfsetfillcolor{currentfill}%
\pgfsetlinewidth{0.803000pt}%
\definecolor{currentstroke}{rgb}{0.000000,0.000000,0.000000}%
\pgfsetstrokecolor{currentstroke}%
\pgfsetdash{}{0pt}%
\pgfsys@defobject{currentmarker}{\pgfqpoint{-0.048611in}{0.000000in}}{\pgfqpoint{0.000000in}{0.000000in}}{%
\pgfpathmoveto{\pgfqpoint{0.000000in}{0.000000in}}%
\pgfpathlineto{\pgfqpoint{-0.048611in}{0.000000in}}%
\pgfusepath{stroke,fill}%
}%
\begin{pgfscope}%
\pgfsys@transformshift{0.374692in}{1.144478in}%
\pgfsys@useobject{currentmarker}{}%
\end{pgfscope}%
\end{pgfscope}%
\begin{pgfscope}%
\pgftext[x=0.100000in,y=1.091717in,left,base]{\rmfamily\fontsize{10.000000}{12.000000}\selectfont \(\displaystyle 0.5\)}%
\end{pgfscope}%
\begin{pgfscope}%
\pgfsetbuttcap%
\pgfsetroundjoin%
\definecolor{currentfill}{rgb}{0.000000,0.000000,0.000000}%
\pgfsetfillcolor{currentfill}%
\pgfsetlinewidth{0.803000pt}%
\definecolor{currentstroke}{rgb}{0.000000,0.000000,0.000000}%
\pgfsetstrokecolor{currentstroke}%
\pgfsetdash{}{0pt}%
\pgfsys@defobject{currentmarker}{\pgfqpoint{-0.048611in}{0.000000in}}{\pgfqpoint{0.000000in}{0.000000in}}{%
\pgfpathmoveto{\pgfqpoint{0.000000in}{0.000000in}}%
\pgfpathlineto{\pgfqpoint{-0.048611in}{0.000000in}}%
\pgfusepath{stroke,fill}%
}%
\begin{pgfscope}%
\pgfsys@transformshift{0.374692in}{1.559728in}%
\pgfsys@useobject{currentmarker}{}%
\end{pgfscope}%
\end{pgfscope}%
\begin{pgfscope}%
\pgftext[x=0.100000in,y=1.506967in,left,base]{\rmfamily\fontsize{10.000000}{12.000000}\selectfont \(\displaystyle 1.0\)}%
\end{pgfscope}%
\begin{pgfscope}%
\pgfsetbuttcap%
\pgfsetroundjoin%
\definecolor{currentfill}{rgb}{0.000000,0.000000,0.000000}%
\pgfsetfillcolor{currentfill}%
\pgfsetlinewidth{0.803000pt}%
\definecolor{currentstroke}{rgb}{0.000000,0.000000,0.000000}%
\pgfsetstrokecolor{currentstroke}%
\pgfsetdash{}{0pt}%
\pgfsys@defobject{currentmarker}{\pgfqpoint{-0.048611in}{0.000000in}}{\pgfqpoint{0.000000in}{0.000000in}}{%
\pgfpathmoveto{\pgfqpoint{0.000000in}{0.000000in}}%
\pgfpathlineto{\pgfqpoint{-0.048611in}{0.000000in}}%
\pgfusepath{stroke,fill}%
}%
\begin{pgfscope}%
\pgfsys@transformshift{0.374692in}{1.974978in}%
\pgfsys@useobject{currentmarker}{}%
\end{pgfscope}%
\end{pgfscope}%
\begin{pgfscope}%
\pgftext[x=0.100000in,y=1.922217in,left,base]{\rmfamily\fontsize{10.000000}{12.000000}\selectfont \(\displaystyle 1.5\)}%
\end{pgfscope}%
\begin{pgfscope}%
\pgfpathrectangle{\pgfqpoint{0.374692in}{0.521603in}}{\pgfqpoint{2.635000in}{1.661000in}} %
\pgfusepath{clip}%
\pgfsetrectcap%
\pgfsetroundjoin%
\pgfsetlinewidth{1.003750pt}%
\definecolor{currentstroke}{rgb}{0.121569,0.466667,0.705882}%
\pgfsetstrokecolor{currentstroke}%
\pgfsetdash{}{0pt}%
\pgfpathmoveto{\pgfqpoint{0.374692in}{1.144478in}}%
\pgfpathlineto{\pgfqpoint{0.380015in}{1.136132in}}%
\pgfpathlineto{\pgfqpoint{0.385338in}{1.127870in}}%
\pgfpathlineto{\pgfqpoint{0.390662in}{1.119693in}}%
\pgfpathlineto{\pgfqpoint{0.395985in}{1.111601in}}%
\pgfpathlineto{\pgfqpoint{0.401308in}{1.103593in}}%
\pgfpathlineto{\pgfqpoint{0.406631in}{1.095670in}}%
\pgfpathlineto{\pgfqpoint{0.411955in}{1.087832in}}%
\pgfpathlineto{\pgfqpoint{0.417278in}{1.080079in}}%
\pgfpathlineto{\pgfqpoint{0.422601in}{1.072410in}}%
\pgfpathlineto{\pgfqpoint{0.427924in}{1.064826in}}%
\pgfpathlineto{\pgfqpoint{0.433248in}{1.057327in}}%
\pgfpathlineto{\pgfqpoint{0.438571in}{1.049913in}}%
\pgfpathlineto{\pgfqpoint{0.443894in}{1.042583in}}%
\pgfpathlineto{\pgfqpoint{0.449217in}{1.035338in}}%
\pgfpathlineto{\pgfqpoint{0.454540in}{1.028178in}}%
\pgfpathlineto{\pgfqpoint{0.459864in}{1.021102in}}%
\pgfpathlineto{\pgfqpoint{0.465187in}{1.014112in}}%
\pgfpathlineto{\pgfqpoint{0.470510in}{1.007206in}}%
\pgfpathlineto{\pgfqpoint{0.475833in}{1.000384in}}%
\pgfpathlineto{\pgfqpoint{0.481157in}{0.993648in}}%
\pgfpathlineto{\pgfqpoint{0.486480in}{0.986996in}}%
\pgfpathlineto{\pgfqpoint{0.491803in}{0.980429in}}%
\pgfpathlineto{\pgfqpoint{0.497126in}{0.973947in}}%
\pgfpathlineto{\pgfqpoint{0.502450in}{0.967549in}}%
\pgfpathlineto{\pgfqpoint{0.507773in}{0.961236in}}%
\pgfpathlineto{\pgfqpoint{0.513096in}{0.955008in}}%
\pgfpathlineto{\pgfqpoint{0.518419in}{0.948865in}}%
\pgfpathlineto{\pgfqpoint{0.523742in}{0.942806in}}%
\pgfpathlineto{\pgfqpoint{0.529066in}{0.936832in}}%
\pgfpathlineto{\pgfqpoint{0.534389in}{0.930943in}}%
\pgfpathlineto{\pgfqpoint{0.539712in}{0.925139in}}%
\pgfpathlineto{\pgfqpoint{0.545035in}{0.919419in}}%
\pgfpathlineto{\pgfqpoint{0.550359in}{0.913784in}}%
\pgfpathlineto{\pgfqpoint{0.555682in}{0.908234in}}%
\pgfpathlineto{\pgfqpoint{0.561005in}{0.902768in}}%
\pgfpathlineto{\pgfqpoint{0.566328in}{0.897387in}}%
\pgfpathlineto{\pgfqpoint{0.571652in}{0.892091in}}%
\pgfpathlineto{\pgfqpoint{0.576975in}{0.886880in}}%
\pgfpathlineto{\pgfqpoint{0.582298in}{0.881754in}}%
\pgfpathlineto{\pgfqpoint{0.587621in}{0.876712in}}%
\pgfpathlineto{\pgfqpoint{0.592944in}{0.871755in}}%
\pgfpathlineto{\pgfqpoint{0.598268in}{0.866882in}}%
\pgfpathlineto{\pgfqpoint{0.603591in}{0.862095in}}%
\pgfpathlineto{\pgfqpoint{0.608914in}{0.857392in}}%
\pgfpathlineto{\pgfqpoint{0.614237in}{0.852774in}}%
\pgfpathlineto{\pgfqpoint{0.619561in}{0.848240in}}%
\pgfpathlineto{\pgfqpoint{0.624884in}{0.843792in}}%
\pgfpathlineto{\pgfqpoint{0.630207in}{0.839428in}}%
\pgfpathlineto{\pgfqpoint{0.635530in}{0.835149in}}%
\pgfpathlineto{\pgfqpoint{0.640854in}{0.830954in}}%
\pgfpathlineto{\pgfqpoint{0.646177in}{0.826844in}}%
\pgfpathlineto{\pgfqpoint{0.651500in}{0.822820in}}%
\pgfpathlineto{\pgfqpoint{0.656823in}{0.818879in}}%
\pgfpathlineto{\pgfqpoint{0.662146in}{0.815024in}}%
\pgfpathlineto{\pgfqpoint{0.667470in}{0.811253in}}%
\pgfpathlineto{\pgfqpoint{0.672793in}{0.807567in}}%
\pgfpathlineto{\pgfqpoint{0.678116in}{0.803966in}}%
\pgfpathlineto{\pgfqpoint{0.683439in}{0.800449in}}%
\pgfpathlineto{\pgfqpoint{0.688763in}{0.797017in}}%
\pgfpathlineto{\pgfqpoint{0.694086in}{0.793670in}}%
\pgfpathlineto{\pgfqpoint{0.699409in}{0.790408in}}%
\pgfpathlineto{\pgfqpoint{0.704732in}{0.787230in}}%
\pgfpathlineto{\pgfqpoint{0.710056in}{0.784137in}}%
\pgfpathlineto{\pgfqpoint{0.715379in}{0.781129in}}%
\pgfpathlineto{\pgfqpoint{0.720702in}{0.778206in}}%
\pgfpathlineto{\pgfqpoint{0.726025in}{0.775367in}}%
\pgfpathlineto{\pgfqpoint{0.731349in}{0.772613in}}%
\pgfpathlineto{\pgfqpoint{0.736672in}{0.769944in}}%
\pgfpathlineto{\pgfqpoint{0.741995in}{0.767360in}}%
\pgfpathlineto{\pgfqpoint{0.747318in}{0.764860in}}%
\pgfpathlineto{\pgfqpoint{0.752641in}{0.762445in}}%
\pgfpathlineto{\pgfqpoint{0.757965in}{0.760115in}}%
\pgfpathlineto{\pgfqpoint{0.763288in}{0.757869in}}%
\pgfpathlineto{\pgfqpoint{0.768611in}{0.755708in}}%
\pgfpathlineto{\pgfqpoint{0.773934in}{0.753632in}}%
\pgfpathlineto{\pgfqpoint{0.779258in}{0.751641in}}%
\pgfpathlineto{\pgfqpoint{0.784581in}{0.749735in}}%
\pgfpathlineto{\pgfqpoint{0.789904in}{0.747913in}}%
\pgfpathlineto{\pgfqpoint{0.795227in}{0.746176in}}%
\pgfpathlineto{\pgfqpoint{0.800551in}{0.744523in}}%
\pgfpathlineto{\pgfqpoint{0.805874in}{0.742956in}}%
\pgfpathlineto{\pgfqpoint{0.811197in}{0.741473in}}%
\pgfpathlineto{\pgfqpoint{0.816520in}{0.740075in}}%
\pgfpathlineto{\pgfqpoint{0.821843in}{0.738761in}}%
\pgfpathlineto{\pgfqpoint{0.827167in}{0.737532in}}%
\pgfpathlineto{\pgfqpoint{0.832490in}{0.736389in}}%
\pgfpathlineto{\pgfqpoint{0.837813in}{0.735329in}}%
\pgfpathlineto{\pgfqpoint{0.843136in}{0.734355in}}%
\pgfpathlineto{\pgfqpoint{0.848460in}{0.733465in}}%
\pgfpathlineto{\pgfqpoint{0.853783in}{0.732660in}}%
\pgfpathlineto{\pgfqpoint{0.859106in}{0.731940in}}%
\pgfpathlineto{\pgfqpoint{0.864429in}{0.731304in}}%
\pgfpathlineto{\pgfqpoint{0.869753in}{0.730754in}}%
\pgfpathlineto{\pgfqpoint{0.875076in}{0.730288in}}%
\pgfpathlineto{\pgfqpoint{0.880399in}{0.729906in}}%
\pgfpathlineto{\pgfqpoint{0.885722in}{0.729610in}}%
\pgfpathlineto{\pgfqpoint{0.891045in}{0.729398in}}%
\pgfpathlineto{\pgfqpoint{0.896369in}{0.729271in}}%
\pgfpathlineto{\pgfqpoint{0.901692in}{0.729228in}}%
\pgfusepath{stroke}%
\end{pgfscope}%
\begin{pgfscope}%
\pgfpathrectangle{\pgfqpoint{0.374692in}{0.521603in}}{\pgfqpoint{2.635000in}{1.661000in}} %
\pgfusepath{clip}%
\pgfsetrectcap%
\pgfsetroundjoin%
\pgfsetlinewidth{1.003750pt}%
\definecolor{currentstroke}{rgb}{1.000000,0.498039,0.054902}%
\pgfsetstrokecolor{currentstroke}%
\pgfsetdash{}{0pt}%
\pgfpathmoveto{\pgfqpoint{0.374692in}{1.144478in}}%
\pgfpathlineto{\pgfqpoint{0.380015in}{1.152782in}}%
\pgfpathlineto{\pgfqpoint{0.385338in}{1.160917in}}%
\pgfpathlineto{\pgfqpoint{0.390662in}{1.168882in}}%
\pgfpathlineto{\pgfqpoint{0.395985in}{1.176678in}}%
\pgfpathlineto{\pgfqpoint{0.401308in}{1.184304in}}%
\pgfpathlineto{\pgfqpoint{0.406631in}{1.191761in}}%
\pgfpathlineto{\pgfqpoint{0.411955in}{1.199048in}}%
\pgfpathlineto{\pgfqpoint{0.417278in}{1.206166in}}%
\pgfpathlineto{\pgfqpoint{0.422601in}{1.213115in}}%
\pgfpathlineto{\pgfqpoint{0.427924in}{1.219894in}}%
\pgfpathlineto{\pgfqpoint{0.433248in}{1.226503in}}%
\pgfpathlineto{\pgfqpoint{0.438571in}{1.232943in}}%
\pgfpathlineto{\pgfqpoint{0.443894in}{1.239213in}}%
\pgfpathlineto{\pgfqpoint{0.449217in}{1.245314in}}%
\pgfpathlineto{\pgfqpoint{0.454540in}{1.251246in}}%
\pgfpathlineto{\pgfqpoint{0.459864in}{1.257008in}}%
\pgfpathlineto{\pgfqpoint{0.465187in}{1.262601in}}%
\pgfpathlineto{\pgfqpoint{0.470510in}{1.268024in}}%
\pgfpathlineto{\pgfqpoint{0.475833in}{1.273277in}}%
\pgfpathlineto{\pgfqpoint{0.481157in}{1.278362in}}%
\pgfpathlineto{\pgfqpoint{0.486480in}{1.283276in}}%
\pgfpathlineto{\pgfqpoint{0.491803in}{1.288022in}}%
\pgfpathlineto{\pgfqpoint{0.497126in}{1.292597in}}%
\pgfpathlineto{\pgfqpoint{0.502450in}{1.297004in}}%
\pgfpathlineto{\pgfqpoint{0.507773in}{1.301240in}}%
\pgfpathlineto{\pgfqpoint{0.513096in}{1.305308in}}%
\pgfpathlineto{\pgfqpoint{0.518419in}{1.309206in}}%
\pgfpathlineto{\pgfqpoint{0.523742in}{1.312934in}}%
\pgfpathlineto{\pgfqpoint{0.529066in}{1.316493in}}%
\pgfpathlineto{\pgfqpoint{0.534389in}{1.319882in}}%
\pgfpathlineto{\pgfqpoint{0.539712in}{1.323102in}}%
\pgfpathlineto{\pgfqpoint{0.545035in}{1.326153in}}%
\pgfpathlineto{\pgfqpoint{0.550359in}{1.329034in}}%
\pgfpathlineto{\pgfqpoint{0.555682in}{1.331745in}}%
\pgfpathlineto{\pgfqpoint{0.561005in}{1.334288in}}%
\pgfpathlineto{\pgfqpoint{0.566328in}{1.336660in}}%
\pgfpathlineto{\pgfqpoint{0.571652in}{1.338863in}}%
\pgfpathlineto{\pgfqpoint{0.576975in}{1.340897in}}%
\pgfpathlineto{\pgfqpoint{0.582298in}{1.342761in}}%
\pgfpathlineto{\pgfqpoint{0.587621in}{1.344456in}}%
\pgfpathlineto{\pgfqpoint{0.592944in}{1.345981in}}%
\pgfpathlineto{\pgfqpoint{0.598268in}{1.347337in}}%
\pgfpathlineto{\pgfqpoint{0.603591in}{1.348523in}}%
\pgfpathlineto{\pgfqpoint{0.608914in}{1.349540in}}%
\pgfpathlineto{\pgfqpoint{0.614237in}{1.350387in}}%
\pgfpathlineto{\pgfqpoint{0.619561in}{1.351065in}}%
\pgfpathlineto{\pgfqpoint{0.624884in}{1.351574in}}%
\pgfpathlineto{\pgfqpoint{0.630207in}{1.351913in}}%
\pgfpathlineto{\pgfqpoint{0.635530in}{1.352082in}}%
\pgfpathlineto{\pgfqpoint{0.640854in}{1.352082in}}%
\pgfpathlineto{\pgfqpoint{0.646177in}{1.351913in}}%
\pgfpathlineto{\pgfqpoint{0.651500in}{1.351574in}}%
\pgfpathlineto{\pgfqpoint{0.656823in}{1.351065in}}%
\pgfpathlineto{\pgfqpoint{0.662146in}{1.350387in}}%
\pgfpathlineto{\pgfqpoint{0.667470in}{1.349540in}}%
\pgfpathlineto{\pgfqpoint{0.672793in}{1.348523in}}%
\pgfpathlineto{\pgfqpoint{0.678116in}{1.347337in}}%
\pgfpathlineto{\pgfqpoint{0.683439in}{1.345981in}}%
\pgfpathlineto{\pgfqpoint{0.688763in}{1.344456in}}%
\pgfpathlineto{\pgfqpoint{0.694086in}{1.342761in}}%
\pgfpathlineto{\pgfqpoint{0.699409in}{1.340897in}}%
\pgfpathlineto{\pgfqpoint{0.704732in}{1.338863in}}%
\pgfpathlineto{\pgfqpoint{0.710056in}{1.336660in}}%
\pgfpathlineto{\pgfqpoint{0.715379in}{1.334288in}}%
\pgfpathlineto{\pgfqpoint{0.720702in}{1.331745in}}%
\pgfpathlineto{\pgfqpoint{0.726025in}{1.329034in}}%
\pgfpathlineto{\pgfqpoint{0.731349in}{1.326153in}}%
\pgfpathlineto{\pgfqpoint{0.736672in}{1.323102in}}%
\pgfpathlineto{\pgfqpoint{0.741995in}{1.319882in}}%
\pgfpathlineto{\pgfqpoint{0.747318in}{1.316493in}}%
\pgfpathlineto{\pgfqpoint{0.752641in}{1.312934in}}%
\pgfpathlineto{\pgfqpoint{0.757965in}{1.309206in}}%
\pgfpathlineto{\pgfqpoint{0.763288in}{1.305308in}}%
\pgfpathlineto{\pgfqpoint{0.768611in}{1.301240in}}%
\pgfpathlineto{\pgfqpoint{0.773934in}{1.297004in}}%
\pgfpathlineto{\pgfqpoint{0.779258in}{1.292597in}}%
\pgfpathlineto{\pgfqpoint{0.784581in}{1.288022in}}%
\pgfpathlineto{\pgfqpoint{0.789904in}{1.283276in}}%
\pgfpathlineto{\pgfqpoint{0.795227in}{1.278362in}}%
\pgfpathlineto{\pgfqpoint{0.800551in}{1.273277in}}%
\pgfpathlineto{\pgfqpoint{0.805874in}{1.268024in}}%
\pgfpathlineto{\pgfqpoint{0.811197in}{1.262601in}}%
\pgfpathlineto{\pgfqpoint{0.816520in}{1.257008in}}%
\pgfpathlineto{\pgfqpoint{0.821843in}{1.251246in}}%
\pgfpathlineto{\pgfqpoint{0.827167in}{1.245314in}}%
\pgfpathlineto{\pgfqpoint{0.832490in}{1.239213in}}%
\pgfpathlineto{\pgfqpoint{0.837813in}{1.232943in}}%
\pgfpathlineto{\pgfqpoint{0.843136in}{1.226503in}}%
\pgfpathlineto{\pgfqpoint{0.848460in}{1.219894in}}%
\pgfpathlineto{\pgfqpoint{0.853783in}{1.213115in}}%
\pgfpathlineto{\pgfqpoint{0.859106in}{1.206166in}}%
\pgfpathlineto{\pgfqpoint{0.864429in}{1.199048in}}%
\pgfpathlineto{\pgfqpoint{0.869753in}{1.191761in}}%
\pgfpathlineto{\pgfqpoint{0.875076in}{1.184304in}}%
\pgfpathlineto{\pgfqpoint{0.880399in}{1.176678in}}%
\pgfpathlineto{\pgfqpoint{0.885722in}{1.168882in}}%
\pgfpathlineto{\pgfqpoint{0.891045in}{1.160917in}}%
\pgfpathlineto{\pgfqpoint{0.896369in}{1.152782in}}%
\pgfpathlineto{\pgfqpoint{0.901692in}{1.144478in}}%
\pgfusepath{stroke}%
\end{pgfscope}%
\begin{pgfscope}%
\pgfpathrectangle{\pgfqpoint{0.374692in}{0.521603in}}{\pgfqpoint{2.635000in}{1.661000in}} %
\pgfusepath{clip}%
\pgfsetrectcap%
\pgfsetroundjoin%
\pgfsetlinewidth{1.003750pt}%
\definecolor{currentstroke}{rgb}{0.172549,0.627451,0.172549}%
\pgfsetstrokecolor{currentstroke}%
\pgfsetdash{}{0pt}%
\pgfpathmoveto{\pgfqpoint{0.374692in}{0.729228in}}%
\pgfpathlineto{\pgfqpoint{0.380015in}{0.729271in}}%
\pgfpathlineto{\pgfqpoint{0.385338in}{0.729398in}}%
\pgfpathlineto{\pgfqpoint{0.390662in}{0.729610in}}%
\pgfpathlineto{\pgfqpoint{0.395985in}{0.729906in}}%
\pgfpathlineto{\pgfqpoint{0.401308in}{0.730288in}}%
\pgfpathlineto{\pgfqpoint{0.406631in}{0.730754in}}%
\pgfpathlineto{\pgfqpoint{0.411955in}{0.731304in}}%
\pgfpathlineto{\pgfqpoint{0.417278in}{0.731940in}}%
\pgfpathlineto{\pgfqpoint{0.422601in}{0.732660in}}%
\pgfpathlineto{\pgfqpoint{0.427924in}{0.733465in}}%
\pgfpathlineto{\pgfqpoint{0.433248in}{0.734355in}}%
\pgfpathlineto{\pgfqpoint{0.438571in}{0.735329in}}%
\pgfpathlineto{\pgfqpoint{0.443894in}{0.736389in}}%
\pgfpathlineto{\pgfqpoint{0.449217in}{0.737532in}}%
\pgfpathlineto{\pgfqpoint{0.454540in}{0.738761in}}%
\pgfpathlineto{\pgfqpoint{0.459864in}{0.740075in}}%
\pgfpathlineto{\pgfqpoint{0.465187in}{0.741473in}}%
\pgfpathlineto{\pgfqpoint{0.470510in}{0.742956in}}%
\pgfpathlineto{\pgfqpoint{0.475833in}{0.744523in}}%
\pgfpathlineto{\pgfqpoint{0.481157in}{0.746176in}}%
\pgfpathlineto{\pgfqpoint{0.486480in}{0.747913in}}%
\pgfpathlineto{\pgfqpoint{0.491803in}{0.749735in}}%
\pgfpathlineto{\pgfqpoint{0.497126in}{0.751641in}}%
\pgfpathlineto{\pgfqpoint{0.502450in}{0.753632in}}%
\pgfpathlineto{\pgfqpoint{0.507773in}{0.755708in}}%
\pgfpathlineto{\pgfqpoint{0.513096in}{0.757869in}}%
\pgfpathlineto{\pgfqpoint{0.518419in}{0.760115in}}%
\pgfpathlineto{\pgfqpoint{0.523742in}{0.762445in}}%
\pgfpathlineto{\pgfqpoint{0.529066in}{0.764860in}}%
\pgfpathlineto{\pgfqpoint{0.534389in}{0.767360in}}%
\pgfpathlineto{\pgfqpoint{0.539712in}{0.769944in}}%
\pgfpathlineto{\pgfqpoint{0.545035in}{0.772613in}}%
\pgfpathlineto{\pgfqpoint{0.550359in}{0.775367in}}%
\pgfpathlineto{\pgfqpoint{0.555682in}{0.778206in}}%
\pgfpathlineto{\pgfqpoint{0.561005in}{0.781129in}}%
\pgfpathlineto{\pgfqpoint{0.566328in}{0.784137in}}%
\pgfpathlineto{\pgfqpoint{0.571652in}{0.787230in}}%
\pgfpathlineto{\pgfqpoint{0.576975in}{0.790408in}}%
\pgfpathlineto{\pgfqpoint{0.582298in}{0.793670in}}%
\pgfpathlineto{\pgfqpoint{0.587621in}{0.797017in}}%
\pgfpathlineto{\pgfqpoint{0.592944in}{0.800449in}}%
\pgfpathlineto{\pgfqpoint{0.598268in}{0.803966in}}%
\pgfpathlineto{\pgfqpoint{0.603591in}{0.807567in}}%
\pgfpathlineto{\pgfqpoint{0.608914in}{0.811253in}}%
\pgfpathlineto{\pgfqpoint{0.614237in}{0.815024in}}%
\pgfpathlineto{\pgfqpoint{0.619561in}{0.818879in}}%
\pgfpathlineto{\pgfqpoint{0.624884in}{0.822820in}}%
\pgfpathlineto{\pgfqpoint{0.630207in}{0.826844in}}%
\pgfpathlineto{\pgfqpoint{0.635530in}{0.830954in}}%
\pgfpathlineto{\pgfqpoint{0.640854in}{0.835149in}}%
\pgfpathlineto{\pgfqpoint{0.646177in}{0.839428in}}%
\pgfpathlineto{\pgfqpoint{0.651500in}{0.843792in}}%
\pgfpathlineto{\pgfqpoint{0.656823in}{0.848240in}}%
\pgfpathlineto{\pgfqpoint{0.662146in}{0.852774in}}%
\pgfpathlineto{\pgfqpoint{0.667470in}{0.857392in}}%
\pgfpathlineto{\pgfqpoint{0.672793in}{0.862095in}}%
\pgfpathlineto{\pgfqpoint{0.678116in}{0.866882in}}%
\pgfpathlineto{\pgfqpoint{0.683439in}{0.871755in}}%
\pgfpathlineto{\pgfqpoint{0.688763in}{0.876712in}}%
\pgfpathlineto{\pgfqpoint{0.694086in}{0.881754in}}%
\pgfpathlineto{\pgfqpoint{0.699409in}{0.886880in}}%
\pgfpathlineto{\pgfqpoint{0.704732in}{0.892091in}}%
\pgfpathlineto{\pgfqpoint{0.710056in}{0.897387in}}%
\pgfpathlineto{\pgfqpoint{0.715379in}{0.902768in}}%
\pgfpathlineto{\pgfqpoint{0.720702in}{0.908234in}}%
\pgfpathlineto{\pgfqpoint{0.726025in}{0.913784in}}%
\pgfpathlineto{\pgfqpoint{0.731349in}{0.919419in}}%
\pgfpathlineto{\pgfqpoint{0.736672in}{0.925139in}}%
\pgfpathlineto{\pgfqpoint{0.741995in}{0.930943in}}%
\pgfpathlineto{\pgfqpoint{0.747318in}{0.936832in}}%
\pgfpathlineto{\pgfqpoint{0.752641in}{0.942806in}}%
\pgfpathlineto{\pgfqpoint{0.757965in}{0.948865in}}%
\pgfpathlineto{\pgfqpoint{0.763288in}{0.955008in}}%
\pgfpathlineto{\pgfqpoint{0.768611in}{0.961236in}}%
\pgfpathlineto{\pgfqpoint{0.773934in}{0.967549in}}%
\pgfpathlineto{\pgfqpoint{0.779258in}{0.973947in}}%
\pgfpathlineto{\pgfqpoint{0.784581in}{0.980429in}}%
\pgfpathlineto{\pgfqpoint{0.789904in}{0.986996in}}%
\pgfpathlineto{\pgfqpoint{0.795227in}{0.993648in}}%
\pgfpathlineto{\pgfqpoint{0.800551in}{1.000384in}}%
\pgfpathlineto{\pgfqpoint{0.805874in}{1.007206in}}%
\pgfpathlineto{\pgfqpoint{0.811197in}{1.014112in}}%
\pgfpathlineto{\pgfqpoint{0.816520in}{1.021102in}}%
\pgfpathlineto{\pgfqpoint{0.821843in}{1.028178in}}%
\pgfpathlineto{\pgfqpoint{0.827167in}{1.035338in}}%
\pgfpathlineto{\pgfqpoint{0.832490in}{1.042583in}}%
\pgfpathlineto{\pgfqpoint{0.837813in}{1.049913in}}%
\pgfpathlineto{\pgfqpoint{0.843136in}{1.057327in}}%
\pgfpathlineto{\pgfqpoint{0.848460in}{1.064826in}}%
\pgfpathlineto{\pgfqpoint{0.853783in}{1.072410in}}%
\pgfpathlineto{\pgfqpoint{0.859106in}{1.080079in}}%
\pgfpathlineto{\pgfqpoint{0.864429in}{1.087832in}}%
\pgfpathlineto{\pgfqpoint{0.869753in}{1.095670in}}%
\pgfpathlineto{\pgfqpoint{0.875076in}{1.103593in}}%
\pgfpathlineto{\pgfqpoint{0.880399in}{1.111601in}}%
\pgfpathlineto{\pgfqpoint{0.885722in}{1.119693in}}%
\pgfpathlineto{\pgfqpoint{0.891045in}{1.127870in}}%
\pgfpathlineto{\pgfqpoint{0.896369in}{1.136132in}}%
\pgfpathlineto{\pgfqpoint{0.901692in}{1.144478in}}%
\pgfusepath{stroke}%
\end{pgfscope}%
\begin{pgfscope}%
\pgfpathrectangle{\pgfqpoint{0.374692in}{0.521603in}}{\pgfqpoint{2.635000in}{1.661000in}} %
\pgfusepath{clip}%
\pgfsetrectcap%
\pgfsetroundjoin%
\pgfsetlinewidth{1.003750pt}%
\definecolor{currentstroke}{rgb}{1.000000,0.498039,0.054902}%
\pgfsetstrokecolor{currentstroke}%
\pgfsetdash{}{0pt}%
\pgfpathmoveto{\pgfqpoint{0.901692in}{1.144478in}}%
\pgfpathlineto{\pgfqpoint{0.907015in}{1.136132in}}%
\pgfpathlineto{\pgfqpoint{0.912338in}{1.127870in}}%
\pgfpathlineto{\pgfqpoint{0.917662in}{1.119693in}}%
\pgfpathlineto{\pgfqpoint{0.922985in}{1.111601in}}%
\pgfpathlineto{\pgfqpoint{0.928308in}{1.103593in}}%
\pgfpathlineto{\pgfqpoint{0.933631in}{1.095670in}}%
\pgfpathlineto{\pgfqpoint{0.938955in}{1.087832in}}%
\pgfpathlineto{\pgfqpoint{0.944278in}{1.080079in}}%
\pgfpathlineto{\pgfqpoint{0.949601in}{1.072410in}}%
\pgfpathlineto{\pgfqpoint{0.954924in}{1.064826in}}%
\pgfpathlineto{\pgfqpoint{0.960248in}{1.057327in}}%
\pgfpathlineto{\pgfqpoint{0.965571in}{1.049913in}}%
\pgfpathlineto{\pgfqpoint{0.970894in}{1.042583in}}%
\pgfpathlineto{\pgfqpoint{0.976217in}{1.035338in}}%
\pgfpathlineto{\pgfqpoint{0.981540in}{1.028178in}}%
\pgfpathlineto{\pgfqpoint{0.986864in}{1.021102in}}%
\pgfpathlineto{\pgfqpoint{0.992187in}{1.014112in}}%
\pgfpathlineto{\pgfqpoint{0.997510in}{1.007206in}}%
\pgfpathlineto{\pgfqpoint{1.002833in}{1.000384in}}%
\pgfpathlineto{\pgfqpoint{1.008157in}{0.993648in}}%
\pgfpathlineto{\pgfqpoint{1.013480in}{0.986996in}}%
\pgfpathlineto{\pgfqpoint{1.018803in}{0.980429in}}%
\pgfpathlineto{\pgfqpoint{1.024126in}{0.973947in}}%
\pgfpathlineto{\pgfqpoint{1.029450in}{0.967549in}}%
\pgfpathlineto{\pgfqpoint{1.034773in}{0.961236in}}%
\pgfpathlineto{\pgfqpoint{1.040096in}{0.955008in}}%
\pgfpathlineto{\pgfqpoint{1.045419in}{0.948865in}}%
\pgfpathlineto{\pgfqpoint{1.050742in}{0.942806in}}%
\pgfpathlineto{\pgfqpoint{1.056066in}{0.936832in}}%
\pgfpathlineto{\pgfqpoint{1.061389in}{0.930943in}}%
\pgfpathlineto{\pgfqpoint{1.066712in}{0.925139in}}%
\pgfpathlineto{\pgfqpoint{1.072035in}{0.919419in}}%
\pgfpathlineto{\pgfqpoint{1.077359in}{0.913784in}}%
\pgfpathlineto{\pgfqpoint{1.082682in}{0.908234in}}%
\pgfpathlineto{\pgfqpoint{1.088005in}{0.902768in}}%
\pgfpathlineto{\pgfqpoint{1.093328in}{0.897387in}}%
\pgfpathlineto{\pgfqpoint{1.098652in}{0.892091in}}%
\pgfpathlineto{\pgfqpoint{1.103975in}{0.886880in}}%
\pgfpathlineto{\pgfqpoint{1.109298in}{0.881754in}}%
\pgfpathlineto{\pgfqpoint{1.114621in}{0.876712in}}%
\pgfpathlineto{\pgfqpoint{1.119944in}{0.871755in}}%
\pgfpathlineto{\pgfqpoint{1.125268in}{0.866882in}}%
\pgfpathlineto{\pgfqpoint{1.130591in}{0.862095in}}%
\pgfpathlineto{\pgfqpoint{1.135914in}{0.857392in}}%
\pgfpathlineto{\pgfqpoint{1.141237in}{0.852774in}}%
\pgfpathlineto{\pgfqpoint{1.146561in}{0.848240in}}%
\pgfpathlineto{\pgfqpoint{1.151884in}{0.843792in}}%
\pgfpathlineto{\pgfqpoint{1.157207in}{0.839428in}}%
\pgfpathlineto{\pgfqpoint{1.162530in}{0.835149in}}%
\pgfpathlineto{\pgfqpoint{1.167854in}{0.830954in}}%
\pgfpathlineto{\pgfqpoint{1.173177in}{0.826844in}}%
\pgfpathlineto{\pgfqpoint{1.178500in}{0.822820in}}%
\pgfpathlineto{\pgfqpoint{1.183823in}{0.818879in}}%
\pgfpathlineto{\pgfqpoint{1.189146in}{0.815024in}}%
\pgfpathlineto{\pgfqpoint{1.194470in}{0.811253in}}%
\pgfpathlineto{\pgfqpoint{1.199793in}{0.807567in}}%
\pgfpathlineto{\pgfqpoint{1.205116in}{0.803966in}}%
\pgfpathlineto{\pgfqpoint{1.210439in}{0.800449in}}%
\pgfpathlineto{\pgfqpoint{1.215763in}{0.797017in}}%
\pgfpathlineto{\pgfqpoint{1.221086in}{0.793670in}}%
\pgfpathlineto{\pgfqpoint{1.226409in}{0.790408in}}%
\pgfpathlineto{\pgfqpoint{1.231732in}{0.787230in}}%
\pgfpathlineto{\pgfqpoint{1.237056in}{0.784137in}}%
\pgfpathlineto{\pgfqpoint{1.242379in}{0.781129in}}%
\pgfpathlineto{\pgfqpoint{1.247702in}{0.778206in}}%
\pgfpathlineto{\pgfqpoint{1.253025in}{0.775367in}}%
\pgfpathlineto{\pgfqpoint{1.258349in}{0.772613in}}%
\pgfpathlineto{\pgfqpoint{1.263672in}{0.769944in}}%
\pgfpathlineto{\pgfqpoint{1.268995in}{0.767360in}}%
\pgfpathlineto{\pgfqpoint{1.274318in}{0.764860in}}%
\pgfpathlineto{\pgfqpoint{1.279641in}{0.762445in}}%
\pgfpathlineto{\pgfqpoint{1.284965in}{0.760115in}}%
\pgfpathlineto{\pgfqpoint{1.290288in}{0.757869in}}%
\pgfpathlineto{\pgfqpoint{1.295611in}{0.755708in}}%
\pgfpathlineto{\pgfqpoint{1.300934in}{0.753632in}}%
\pgfpathlineto{\pgfqpoint{1.306258in}{0.751641in}}%
\pgfpathlineto{\pgfqpoint{1.311581in}{0.749735in}}%
\pgfpathlineto{\pgfqpoint{1.316904in}{0.747913in}}%
\pgfpathlineto{\pgfqpoint{1.322227in}{0.746176in}}%
\pgfpathlineto{\pgfqpoint{1.327551in}{0.744523in}}%
\pgfpathlineto{\pgfqpoint{1.332874in}{0.742956in}}%
\pgfpathlineto{\pgfqpoint{1.338197in}{0.741473in}}%
\pgfpathlineto{\pgfqpoint{1.343520in}{0.740075in}}%
\pgfpathlineto{\pgfqpoint{1.348843in}{0.738761in}}%
\pgfpathlineto{\pgfqpoint{1.354167in}{0.737532in}}%
\pgfpathlineto{\pgfqpoint{1.359490in}{0.736389in}}%
\pgfpathlineto{\pgfqpoint{1.364813in}{0.735329in}}%
\pgfpathlineto{\pgfqpoint{1.370136in}{0.734355in}}%
\pgfpathlineto{\pgfqpoint{1.375460in}{0.733465in}}%
\pgfpathlineto{\pgfqpoint{1.380783in}{0.732660in}}%
\pgfpathlineto{\pgfqpoint{1.386106in}{0.731940in}}%
\pgfpathlineto{\pgfqpoint{1.391429in}{0.731304in}}%
\pgfpathlineto{\pgfqpoint{1.396753in}{0.730754in}}%
\pgfpathlineto{\pgfqpoint{1.402076in}{0.730288in}}%
\pgfpathlineto{\pgfqpoint{1.407399in}{0.729906in}}%
\pgfpathlineto{\pgfqpoint{1.412722in}{0.729610in}}%
\pgfpathlineto{\pgfqpoint{1.418045in}{0.729398in}}%
\pgfpathlineto{\pgfqpoint{1.423369in}{0.729271in}}%
\pgfpathlineto{\pgfqpoint{1.428692in}{0.729228in}}%
\pgfusepath{stroke}%
\end{pgfscope}%
\begin{pgfscope}%
\pgfpathrectangle{\pgfqpoint{0.374692in}{0.521603in}}{\pgfqpoint{2.635000in}{1.661000in}} %
\pgfusepath{clip}%
\pgfsetrectcap%
\pgfsetroundjoin%
\pgfsetlinewidth{1.003750pt}%
\definecolor{currentstroke}{rgb}{0.172549,0.627451,0.172549}%
\pgfsetstrokecolor{currentstroke}%
\pgfsetdash{}{0pt}%
\pgfpathmoveto{\pgfqpoint{0.901692in}{1.144478in}}%
\pgfpathlineto{\pgfqpoint{0.907015in}{1.152782in}}%
\pgfpathlineto{\pgfqpoint{0.912338in}{1.160917in}}%
\pgfpathlineto{\pgfqpoint{0.917662in}{1.168882in}}%
\pgfpathlineto{\pgfqpoint{0.922985in}{1.176678in}}%
\pgfpathlineto{\pgfqpoint{0.928308in}{1.184304in}}%
\pgfpathlineto{\pgfqpoint{0.933631in}{1.191761in}}%
\pgfpathlineto{\pgfqpoint{0.938955in}{1.199048in}}%
\pgfpathlineto{\pgfqpoint{0.944278in}{1.206166in}}%
\pgfpathlineto{\pgfqpoint{0.949601in}{1.213115in}}%
\pgfpathlineto{\pgfqpoint{0.954924in}{1.219894in}}%
\pgfpathlineto{\pgfqpoint{0.960248in}{1.226503in}}%
\pgfpathlineto{\pgfqpoint{0.965571in}{1.232943in}}%
\pgfpathlineto{\pgfqpoint{0.970894in}{1.239213in}}%
\pgfpathlineto{\pgfqpoint{0.976217in}{1.245314in}}%
\pgfpathlineto{\pgfqpoint{0.981540in}{1.251246in}}%
\pgfpathlineto{\pgfqpoint{0.986864in}{1.257008in}}%
\pgfpathlineto{\pgfqpoint{0.992187in}{1.262601in}}%
\pgfpathlineto{\pgfqpoint{0.997510in}{1.268024in}}%
\pgfpathlineto{\pgfqpoint{1.002833in}{1.273277in}}%
\pgfpathlineto{\pgfqpoint{1.008157in}{1.278362in}}%
\pgfpathlineto{\pgfqpoint{1.013480in}{1.283276in}}%
\pgfpathlineto{\pgfqpoint{1.018803in}{1.288022in}}%
\pgfpathlineto{\pgfqpoint{1.024126in}{1.292597in}}%
\pgfpathlineto{\pgfqpoint{1.029450in}{1.297004in}}%
\pgfpathlineto{\pgfqpoint{1.034773in}{1.301240in}}%
\pgfpathlineto{\pgfqpoint{1.040096in}{1.305308in}}%
\pgfpathlineto{\pgfqpoint{1.045419in}{1.309206in}}%
\pgfpathlineto{\pgfqpoint{1.050742in}{1.312934in}}%
\pgfpathlineto{\pgfqpoint{1.056066in}{1.316493in}}%
\pgfpathlineto{\pgfqpoint{1.061389in}{1.319882in}}%
\pgfpathlineto{\pgfqpoint{1.066712in}{1.323102in}}%
\pgfpathlineto{\pgfqpoint{1.072035in}{1.326153in}}%
\pgfpathlineto{\pgfqpoint{1.077359in}{1.329034in}}%
\pgfpathlineto{\pgfqpoint{1.082682in}{1.331745in}}%
\pgfpathlineto{\pgfqpoint{1.088005in}{1.334288in}}%
\pgfpathlineto{\pgfqpoint{1.093328in}{1.336660in}}%
\pgfpathlineto{\pgfqpoint{1.098652in}{1.338863in}}%
\pgfpathlineto{\pgfqpoint{1.103975in}{1.340897in}}%
\pgfpathlineto{\pgfqpoint{1.109298in}{1.342761in}}%
\pgfpathlineto{\pgfqpoint{1.114621in}{1.344456in}}%
\pgfpathlineto{\pgfqpoint{1.119944in}{1.345981in}}%
\pgfpathlineto{\pgfqpoint{1.125268in}{1.347337in}}%
\pgfpathlineto{\pgfqpoint{1.130591in}{1.348523in}}%
\pgfpathlineto{\pgfqpoint{1.135914in}{1.349540in}}%
\pgfpathlineto{\pgfqpoint{1.141237in}{1.350387in}}%
\pgfpathlineto{\pgfqpoint{1.146561in}{1.351065in}}%
\pgfpathlineto{\pgfqpoint{1.151884in}{1.351574in}}%
\pgfpathlineto{\pgfqpoint{1.157207in}{1.351913in}}%
\pgfpathlineto{\pgfqpoint{1.162530in}{1.352082in}}%
\pgfpathlineto{\pgfqpoint{1.167854in}{1.352082in}}%
\pgfpathlineto{\pgfqpoint{1.173177in}{1.351913in}}%
\pgfpathlineto{\pgfqpoint{1.178500in}{1.351574in}}%
\pgfpathlineto{\pgfqpoint{1.183823in}{1.351065in}}%
\pgfpathlineto{\pgfqpoint{1.189146in}{1.350387in}}%
\pgfpathlineto{\pgfqpoint{1.194470in}{1.349540in}}%
\pgfpathlineto{\pgfqpoint{1.199793in}{1.348523in}}%
\pgfpathlineto{\pgfqpoint{1.205116in}{1.347337in}}%
\pgfpathlineto{\pgfqpoint{1.210439in}{1.345981in}}%
\pgfpathlineto{\pgfqpoint{1.215763in}{1.344456in}}%
\pgfpathlineto{\pgfqpoint{1.221086in}{1.342761in}}%
\pgfpathlineto{\pgfqpoint{1.226409in}{1.340897in}}%
\pgfpathlineto{\pgfqpoint{1.231732in}{1.338863in}}%
\pgfpathlineto{\pgfqpoint{1.237056in}{1.336660in}}%
\pgfpathlineto{\pgfqpoint{1.242379in}{1.334288in}}%
\pgfpathlineto{\pgfqpoint{1.247702in}{1.331745in}}%
\pgfpathlineto{\pgfqpoint{1.253025in}{1.329034in}}%
\pgfpathlineto{\pgfqpoint{1.258349in}{1.326153in}}%
\pgfpathlineto{\pgfqpoint{1.263672in}{1.323102in}}%
\pgfpathlineto{\pgfqpoint{1.268995in}{1.319882in}}%
\pgfpathlineto{\pgfqpoint{1.274318in}{1.316493in}}%
\pgfpathlineto{\pgfqpoint{1.279641in}{1.312934in}}%
\pgfpathlineto{\pgfqpoint{1.284965in}{1.309206in}}%
\pgfpathlineto{\pgfqpoint{1.290288in}{1.305308in}}%
\pgfpathlineto{\pgfqpoint{1.295611in}{1.301240in}}%
\pgfpathlineto{\pgfqpoint{1.300934in}{1.297004in}}%
\pgfpathlineto{\pgfqpoint{1.306258in}{1.292597in}}%
\pgfpathlineto{\pgfqpoint{1.311581in}{1.288022in}}%
\pgfpathlineto{\pgfqpoint{1.316904in}{1.283276in}}%
\pgfpathlineto{\pgfqpoint{1.322227in}{1.278362in}}%
\pgfpathlineto{\pgfqpoint{1.327551in}{1.273277in}}%
\pgfpathlineto{\pgfqpoint{1.332874in}{1.268024in}}%
\pgfpathlineto{\pgfqpoint{1.338197in}{1.262601in}}%
\pgfpathlineto{\pgfqpoint{1.343520in}{1.257008in}}%
\pgfpathlineto{\pgfqpoint{1.348843in}{1.251246in}}%
\pgfpathlineto{\pgfqpoint{1.354167in}{1.245314in}}%
\pgfpathlineto{\pgfqpoint{1.359490in}{1.239213in}}%
\pgfpathlineto{\pgfqpoint{1.364813in}{1.232943in}}%
\pgfpathlineto{\pgfqpoint{1.370136in}{1.226503in}}%
\pgfpathlineto{\pgfqpoint{1.375460in}{1.219894in}}%
\pgfpathlineto{\pgfqpoint{1.380783in}{1.213115in}}%
\pgfpathlineto{\pgfqpoint{1.386106in}{1.206166in}}%
\pgfpathlineto{\pgfqpoint{1.391429in}{1.199048in}}%
\pgfpathlineto{\pgfqpoint{1.396753in}{1.191761in}}%
\pgfpathlineto{\pgfqpoint{1.402076in}{1.184304in}}%
\pgfpathlineto{\pgfqpoint{1.407399in}{1.176678in}}%
\pgfpathlineto{\pgfqpoint{1.412722in}{1.168882in}}%
\pgfpathlineto{\pgfqpoint{1.418045in}{1.160917in}}%
\pgfpathlineto{\pgfqpoint{1.423369in}{1.152782in}}%
\pgfpathlineto{\pgfqpoint{1.428692in}{1.144478in}}%
\pgfusepath{stroke}%
\end{pgfscope}%
\begin{pgfscope}%
\pgfpathrectangle{\pgfqpoint{0.374692in}{0.521603in}}{\pgfqpoint{2.635000in}{1.661000in}} %
\pgfusepath{clip}%
\pgfsetrectcap%
\pgfsetroundjoin%
\pgfsetlinewidth{1.003750pt}%
\definecolor{currentstroke}{rgb}{0.839216,0.152941,0.156863}%
\pgfsetstrokecolor{currentstroke}%
\pgfsetdash{}{0pt}%
\pgfpathmoveto{\pgfqpoint{0.901692in}{0.729228in}}%
\pgfpathlineto{\pgfqpoint{0.907015in}{0.729271in}}%
\pgfpathlineto{\pgfqpoint{0.912338in}{0.729398in}}%
\pgfpathlineto{\pgfqpoint{0.917662in}{0.729610in}}%
\pgfpathlineto{\pgfqpoint{0.922985in}{0.729906in}}%
\pgfpathlineto{\pgfqpoint{0.928308in}{0.730288in}}%
\pgfpathlineto{\pgfqpoint{0.933631in}{0.730754in}}%
\pgfpathlineto{\pgfqpoint{0.938955in}{0.731304in}}%
\pgfpathlineto{\pgfqpoint{0.944278in}{0.731940in}}%
\pgfpathlineto{\pgfqpoint{0.949601in}{0.732660in}}%
\pgfpathlineto{\pgfqpoint{0.954924in}{0.733465in}}%
\pgfpathlineto{\pgfqpoint{0.960248in}{0.734355in}}%
\pgfpathlineto{\pgfqpoint{0.965571in}{0.735329in}}%
\pgfpathlineto{\pgfqpoint{0.970894in}{0.736389in}}%
\pgfpathlineto{\pgfqpoint{0.976217in}{0.737532in}}%
\pgfpathlineto{\pgfqpoint{0.981540in}{0.738761in}}%
\pgfpathlineto{\pgfqpoint{0.986864in}{0.740075in}}%
\pgfpathlineto{\pgfqpoint{0.992187in}{0.741473in}}%
\pgfpathlineto{\pgfqpoint{0.997510in}{0.742956in}}%
\pgfpathlineto{\pgfqpoint{1.002833in}{0.744523in}}%
\pgfpathlineto{\pgfqpoint{1.008157in}{0.746176in}}%
\pgfpathlineto{\pgfqpoint{1.013480in}{0.747913in}}%
\pgfpathlineto{\pgfqpoint{1.018803in}{0.749735in}}%
\pgfpathlineto{\pgfqpoint{1.024126in}{0.751641in}}%
\pgfpathlineto{\pgfqpoint{1.029450in}{0.753632in}}%
\pgfpathlineto{\pgfqpoint{1.034773in}{0.755708in}}%
\pgfpathlineto{\pgfqpoint{1.040096in}{0.757869in}}%
\pgfpathlineto{\pgfqpoint{1.045419in}{0.760115in}}%
\pgfpathlineto{\pgfqpoint{1.050742in}{0.762445in}}%
\pgfpathlineto{\pgfqpoint{1.056066in}{0.764860in}}%
\pgfpathlineto{\pgfqpoint{1.061389in}{0.767360in}}%
\pgfpathlineto{\pgfqpoint{1.066712in}{0.769944in}}%
\pgfpathlineto{\pgfqpoint{1.072035in}{0.772613in}}%
\pgfpathlineto{\pgfqpoint{1.077359in}{0.775367in}}%
\pgfpathlineto{\pgfqpoint{1.082682in}{0.778206in}}%
\pgfpathlineto{\pgfqpoint{1.088005in}{0.781129in}}%
\pgfpathlineto{\pgfqpoint{1.093328in}{0.784137in}}%
\pgfpathlineto{\pgfqpoint{1.098652in}{0.787230in}}%
\pgfpathlineto{\pgfqpoint{1.103975in}{0.790408in}}%
\pgfpathlineto{\pgfqpoint{1.109298in}{0.793670in}}%
\pgfpathlineto{\pgfqpoint{1.114621in}{0.797017in}}%
\pgfpathlineto{\pgfqpoint{1.119944in}{0.800449in}}%
\pgfpathlineto{\pgfqpoint{1.125268in}{0.803966in}}%
\pgfpathlineto{\pgfqpoint{1.130591in}{0.807567in}}%
\pgfpathlineto{\pgfqpoint{1.135914in}{0.811253in}}%
\pgfpathlineto{\pgfqpoint{1.141237in}{0.815024in}}%
\pgfpathlineto{\pgfqpoint{1.146561in}{0.818879in}}%
\pgfpathlineto{\pgfqpoint{1.151884in}{0.822820in}}%
\pgfpathlineto{\pgfqpoint{1.157207in}{0.826844in}}%
\pgfpathlineto{\pgfqpoint{1.162530in}{0.830954in}}%
\pgfpathlineto{\pgfqpoint{1.167854in}{0.835149in}}%
\pgfpathlineto{\pgfqpoint{1.173177in}{0.839428in}}%
\pgfpathlineto{\pgfqpoint{1.178500in}{0.843792in}}%
\pgfpathlineto{\pgfqpoint{1.183823in}{0.848240in}}%
\pgfpathlineto{\pgfqpoint{1.189146in}{0.852774in}}%
\pgfpathlineto{\pgfqpoint{1.194470in}{0.857392in}}%
\pgfpathlineto{\pgfqpoint{1.199793in}{0.862095in}}%
\pgfpathlineto{\pgfqpoint{1.205116in}{0.866882in}}%
\pgfpathlineto{\pgfqpoint{1.210439in}{0.871755in}}%
\pgfpathlineto{\pgfqpoint{1.215763in}{0.876712in}}%
\pgfpathlineto{\pgfqpoint{1.221086in}{0.881754in}}%
\pgfpathlineto{\pgfqpoint{1.226409in}{0.886880in}}%
\pgfpathlineto{\pgfqpoint{1.231732in}{0.892091in}}%
\pgfpathlineto{\pgfqpoint{1.237056in}{0.897387in}}%
\pgfpathlineto{\pgfqpoint{1.242379in}{0.902768in}}%
\pgfpathlineto{\pgfqpoint{1.247702in}{0.908234in}}%
\pgfpathlineto{\pgfqpoint{1.253025in}{0.913784in}}%
\pgfpathlineto{\pgfqpoint{1.258349in}{0.919419in}}%
\pgfpathlineto{\pgfqpoint{1.263672in}{0.925139in}}%
\pgfpathlineto{\pgfqpoint{1.268995in}{0.930943in}}%
\pgfpathlineto{\pgfqpoint{1.274318in}{0.936832in}}%
\pgfpathlineto{\pgfqpoint{1.279641in}{0.942806in}}%
\pgfpathlineto{\pgfqpoint{1.284965in}{0.948865in}}%
\pgfpathlineto{\pgfqpoint{1.290288in}{0.955008in}}%
\pgfpathlineto{\pgfqpoint{1.295611in}{0.961236in}}%
\pgfpathlineto{\pgfqpoint{1.300934in}{0.967549in}}%
\pgfpathlineto{\pgfqpoint{1.306258in}{0.973947in}}%
\pgfpathlineto{\pgfqpoint{1.311581in}{0.980429in}}%
\pgfpathlineto{\pgfqpoint{1.316904in}{0.986996in}}%
\pgfpathlineto{\pgfqpoint{1.322227in}{0.993648in}}%
\pgfpathlineto{\pgfqpoint{1.327551in}{1.000384in}}%
\pgfpathlineto{\pgfqpoint{1.332874in}{1.007206in}}%
\pgfpathlineto{\pgfqpoint{1.338197in}{1.014112in}}%
\pgfpathlineto{\pgfqpoint{1.343520in}{1.021102in}}%
\pgfpathlineto{\pgfqpoint{1.348843in}{1.028178in}}%
\pgfpathlineto{\pgfqpoint{1.354167in}{1.035338in}}%
\pgfpathlineto{\pgfqpoint{1.359490in}{1.042583in}}%
\pgfpathlineto{\pgfqpoint{1.364813in}{1.049913in}}%
\pgfpathlineto{\pgfqpoint{1.370136in}{1.057327in}}%
\pgfpathlineto{\pgfqpoint{1.375460in}{1.064826in}}%
\pgfpathlineto{\pgfqpoint{1.380783in}{1.072410in}}%
\pgfpathlineto{\pgfqpoint{1.386106in}{1.080079in}}%
\pgfpathlineto{\pgfqpoint{1.391429in}{1.087832in}}%
\pgfpathlineto{\pgfqpoint{1.396753in}{1.095670in}}%
\pgfpathlineto{\pgfqpoint{1.402076in}{1.103593in}}%
\pgfpathlineto{\pgfqpoint{1.407399in}{1.111601in}}%
\pgfpathlineto{\pgfqpoint{1.412722in}{1.119693in}}%
\pgfpathlineto{\pgfqpoint{1.418045in}{1.127870in}}%
\pgfpathlineto{\pgfqpoint{1.423369in}{1.136132in}}%
\pgfpathlineto{\pgfqpoint{1.428692in}{1.144478in}}%
\pgfusepath{stroke}%
\end{pgfscope}%
\begin{pgfscope}%
\pgfpathrectangle{\pgfqpoint{0.374692in}{0.521603in}}{\pgfqpoint{2.635000in}{1.661000in}} %
\pgfusepath{clip}%
\pgfsetbuttcap%
\pgfsetroundjoin%
\pgfsetlinewidth{1.505625pt}%
\definecolor{currentstroke}{rgb}{0.000000,0.000000,0.000000}%
\pgfsetstrokecolor{currentstroke}%
\pgfsetdash{{5.550000pt}{2.400000pt}}{0.000000pt}%
\pgfpathmoveto{\pgfqpoint{0.901692in}{0.729228in}}%
\pgfpathlineto{\pgfqpoint{0.901692in}{0.798437in}}%
\pgfpathlineto{\pgfqpoint{0.901692in}{0.867645in}}%
\pgfpathlineto{\pgfqpoint{0.901692in}{0.936853in}}%
\pgfpathlineto{\pgfqpoint{0.901692in}{1.006062in}}%
\pgfpathlineto{\pgfqpoint{0.901692in}{1.075270in}}%
\pgfpathlineto{\pgfqpoint{0.901692in}{1.144478in}}%
\pgfpathlineto{\pgfqpoint{0.901692in}{1.213687in}}%
\pgfpathlineto{\pgfqpoint{0.901692in}{1.282895in}}%
\pgfpathlineto{\pgfqpoint{0.901692in}{1.352103in}}%
\pgfusepath{stroke}%
\end{pgfscope}%
\begin{pgfscope}%
\pgfpathrectangle{\pgfqpoint{0.374692in}{0.521603in}}{\pgfqpoint{2.635000in}{1.661000in}} %
\pgfusepath{clip}%
\pgfsetrectcap%
\pgfsetroundjoin%
\pgfsetlinewidth{1.003750pt}%
\definecolor{currentstroke}{rgb}{0.172549,0.627451,0.172549}%
\pgfsetstrokecolor{currentstroke}%
\pgfsetdash{}{0pt}%
\pgfpathmoveto{\pgfqpoint{1.428692in}{1.144478in}}%
\pgfpathlineto{\pgfqpoint{1.434015in}{1.136132in}}%
\pgfpathlineto{\pgfqpoint{1.439338in}{1.127870in}}%
\pgfpathlineto{\pgfqpoint{1.444662in}{1.119693in}}%
\pgfpathlineto{\pgfqpoint{1.449985in}{1.111601in}}%
\pgfpathlineto{\pgfqpoint{1.455308in}{1.103593in}}%
\pgfpathlineto{\pgfqpoint{1.460631in}{1.095670in}}%
\pgfpathlineto{\pgfqpoint{1.465955in}{1.087832in}}%
\pgfpathlineto{\pgfqpoint{1.471278in}{1.080079in}}%
\pgfpathlineto{\pgfqpoint{1.476601in}{1.072410in}}%
\pgfpathlineto{\pgfqpoint{1.481924in}{1.064826in}}%
\pgfpathlineto{\pgfqpoint{1.487248in}{1.057327in}}%
\pgfpathlineto{\pgfqpoint{1.492571in}{1.049913in}}%
\pgfpathlineto{\pgfqpoint{1.497894in}{1.042583in}}%
\pgfpathlineto{\pgfqpoint{1.503217in}{1.035338in}}%
\pgfpathlineto{\pgfqpoint{1.508540in}{1.028178in}}%
\pgfpathlineto{\pgfqpoint{1.513864in}{1.021102in}}%
\pgfpathlineto{\pgfqpoint{1.519187in}{1.014112in}}%
\pgfpathlineto{\pgfqpoint{1.524510in}{1.007206in}}%
\pgfpathlineto{\pgfqpoint{1.529833in}{1.000384in}}%
\pgfpathlineto{\pgfqpoint{1.535157in}{0.993648in}}%
\pgfpathlineto{\pgfqpoint{1.540480in}{0.986996in}}%
\pgfpathlineto{\pgfqpoint{1.545803in}{0.980429in}}%
\pgfpathlineto{\pgfqpoint{1.551126in}{0.973947in}}%
\pgfpathlineto{\pgfqpoint{1.556450in}{0.967549in}}%
\pgfpathlineto{\pgfqpoint{1.561773in}{0.961236in}}%
\pgfpathlineto{\pgfqpoint{1.567096in}{0.955008in}}%
\pgfpathlineto{\pgfqpoint{1.572419in}{0.948865in}}%
\pgfpathlineto{\pgfqpoint{1.577742in}{0.942806in}}%
\pgfpathlineto{\pgfqpoint{1.583066in}{0.936832in}}%
\pgfpathlineto{\pgfqpoint{1.588389in}{0.930943in}}%
\pgfpathlineto{\pgfqpoint{1.593712in}{0.925139in}}%
\pgfpathlineto{\pgfqpoint{1.599035in}{0.919419in}}%
\pgfpathlineto{\pgfqpoint{1.604359in}{0.913784in}}%
\pgfpathlineto{\pgfqpoint{1.609682in}{0.908234in}}%
\pgfpathlineto{\pgfqpoint{1.615005in}{0.902768in}}%
\pgfpathlineto{\pgfqpoint{1.620328in}{0.897387in}}%
\pgfpathlineto{\pgfqpoint{1.625652in}{0.892091in}}%
\pgfpathlineto{\pgfqpoint{1.630975in}{0.886880in}}%
\pgfpathlineto{\pgfqpoint{1.636298in}{0.881754in}}%
\pgfpathlineto{\pgfqpoint{1.641621in}{0.876712in}}%
\pgfpathlineto{\pgfqpoint{1.646944in}{0.871755in}}%
\pgfpathlineto{\pgfqpoint{1.652268in}{0.866882in}}%
\pgfpathlineto{\pgfqpoint{1.657591in}{0.862095in}}%
\pgfpathlineto{\pgfqpoint{1.662914in}{0.857392in}}%
\pgfpathlineto{\pgfqpoint{1.668237in}{0.852774in}}%
\pgfpathlineto{\pgfqpoint{1.673561in}{0.848240in}}%
\pgfpathlineto{\pgfqpoint{1.678884in}{0.843792in}}%
\pgfpathlineto{\pgfqpoint{1.684207in}{0.839428in}}%
\pgfpathlineto{\pgfqpoint{1.689530in}{0.835149in}}%
\pgfpathlineto{\pgfqpoint{1.694854in}{0.830954in}}%
\pgfpathlineto{\pgfqpoint{1.700177in}{0.826844in}}%
\pgfpathlineto{\pgfqpoint{1.705500in}{0.822820in}}%
\pgfpathlineto{\pgfqpoint{1.710823in}{0.818879in}}%
\pgfpathlineto{\pgfqpoint{1.716146in}{0.815024in}}%
\pgfpathlineto{\pgfqpoint{1.721470in}{0.811253in}}%
\pgfpathlineto{\pgfqpoint{1.726793in}{0.807567in}}%
\pgfpathlineto{\pgfqpoint{1.732116in}{0.803966in}}%
\pgfpathlineto{\pgfqpoint{1.737439in}{0.800449in}}%
\pgfpathlineto{\pgfqpoint{1.742763in}{0.797017in}}%
\pgfpathlineto{\pgfqpoint{1.748086in}{0.793670in}}%
\pgfpathlineto{\pgfqpoint{1.753409in}{0.790408in}}%
\pgfpathlineto{\pgfqpoint{1.758732in}{0.787230in}}%
\pgfpathlineto{\pgfqpoint{1.764056in}{0.784137in}}%
\pgfpathlineto{\pgfqpoint{1.769379in}{0.781129in}}%
\pgfpathlineto{\pgfqpoint{1.774702in}{0.778206in}}%
\pgfpathlineto{\pgfqpoint{1.780025in}{0.775367in}}%
\pgfpathlineto{\pgfqpoint{1.785349in}{0.772613in}}%
\pgfpathlineto{\pgfqpoint{1.790672in}{0.769944in}}%
\pgfpathlineto{\pgfqpoint{1.795995in}{0.767360in}}%
\pgfpathlineto{\pgfqpoint{1.801318in}{0.764860in}}%
\pgfpathlineto{\pgfqpoint{1.806641in}{0.762445in}}%
\pgfpathlineto{\pgfqpoint{1.811965in}{0.760115in}}%
\pgfpathlineto{\pgfqpoint{1.817288in}{0.757869in}}%
\pgfpathlineto{\pgfqpoint{1.822611in}{0.755708in}}%
\pgfpathlineto{\pgfqpoint{1.827934in}{0.753632in}}%
\pgfpathlineto{\pgfqpoint{1.833258in}{0.751641in}}%
\pgfpathlineto{\pgfqpoint{1.838581in}{0.749735in}}%
\pgfpathlineto{\pgfqpoint{1.843904in}{0.747913in}}%
\pgfpathlineto{\pgfqpoint{1.849227in}{0.746176in}}%
\pgfpathlineto{\pgfqpoint{1.854551in}{0.744523in}}%
\pgfpathlineto{\pgfqpoint{1.859874in}{0.742956in}}%
\pgfpathlineto{\pgfqpoint{1.865197in}{0.741473in}}%
\pgfpathlineto{\pgfqpoint{1.870520in}{0.740075in}}%
\pgfpathlineto{\pgfqpoint{1.875843in}{0.738761in}}%
\pgfpathlineto{\pgfqpoint{1.881167in}{0.737532in}}%
\pgfpathlineto{\pgfqpoint{1.886490in}{0.736389in}}%
\pgfpathlineto{\pgfqpoint{1.891813in}{0.735329in}}%
\pgfpathlineto{\pgfqpoint{1.897136in}{0.734355in}}%
\pgfpathlineto{\pgfqpoint{1.902460in}{0.733465in}}%
\pgfpathlineto{\pgfqpoint{1.907783in}{0.732660in}}%
\pgfpathlineto{\pgfqpoint{1.913106in}{0.731940in}}%
\pgfpathlineto{\pgfqpoint{1.918429in}{0.731304in}}%
\pgfpathlineto{\pgfqpoint{1.923753in}{0.730754in}}%
\pgfpathlineto{\pgfqpoint{1.929076in}{0.730288in}}%
\pgfpathlineto{\pgfqpoint{1.934399in}{0.729906in}}%
\pgfpathlineto{\pgfqpoint{1.939722in}{0.729610in}}%
\pgfpathlineto{\pgfqpoint{1.945045in}{0.729398in}}%
\pgfpathlineto{\pgfqpoint{1.950369in}{0.729271in}}%
\pgfpathlineto{\pgfqpoint{1.955692in}{0.729228in}}%
\pgfusepath{stroke}%
\end{pgfscope}%
\begin{pgfscope}%
\pgfpathrectangle{\pgfqpoint{0.374692in}{0.521603in}}{\pgfqpoint{2.635000in}{1.661000in}} %
\pgfusepath{clip}%
\pgfsetrectcap%
\pgfsetroundjoin%
\pgfsetlinewidth{1.003750pt}%
\definecolor{currentstroke}{rgb}{0.839216,0.152941,0.156863}%
\pgfsetstrokecolor{currentstroke}%
\pgfsetdash{}{0pt}%
\pgfpathmoveto{\pgfqpoint{1.428692in}{1.144478in}}%
\pgfpathlineto{\pgfqpoint{1.434015in}{1.152782in}}%
\pgfpathlineto{\pgfqpoint{1.439338in}{1.160917in}}%
\pgfpathlineto{\pgfqpoint{1.444662in}{1.168882in}}%
\pgfpathlineto{\pgfqpoint{1.449985in}{1.176678in}}%
\pgfpathlineto{\pgfqpoint{1.455308in}{1.184304in}}%
\pgfpathlineto{\pgfqpoint{1.460631in}{1.191761in}}%
\pgfpathlineto{\pgfqpoint{1.465955in}{1.199048in}}%
\pgfpathlineto{\pgfqpoint{1.471278in}{1.206166in}}%
\pgfpathlineto{\pgfqpoint{1.476601in}{1.213115in}}%
\pgfpathlineto{\pgfqpoint{1.481924in}{1.219894in}}%
\pgfpathlineto{\pgfqpoint{1.487248in}{1.226503in}}%
\pgfpathlineto{\pgfqpoint{1.492571in}{1.232943in}}%
\pgfpathlineto{\pgfqpoint{1.497894in}{1.239213in}}%
\pgfpathlineto{\pgfqpoint{1.503217in}{1.245314in}}%
\pgfpathlineto{\pgfqpoint{1.508540in}{1.251246in}}%
\pgfpathlineto{\pgfqpoint{1.513864in}{1.257008in}}%
\pgfpathlineto{\pgfqpoint{1.519187in}{1.262601in}}%
\pgfpathlineto{\pgfqpoint{1.524510in}{1.268024in}}%
\pgfpathlineto{\pgfqpoint{1.529833in}{1.273277in}}%
\pgfpathlineto{\pgfqpoint{1.535157in}{1.278362in}}%
\pgfpathlineto{\pgfqpoint{1.540480in}{1.283276in}}%
\pgfpathlineto{\pgfqpoint{1.545803in}{1.288022in}}%
\pgfpathlineto{\pgfqpoint{1.551126in}{1.292597in}}%
\pgfpathlineto{\pgfqpoint{1.556450in}{1.297004in}}%
\pgfpathlineto{\pgfqpoint{1.561773in}{1.301240in}}%
\pgfpathlineto{\pgfqpoint{1.567096in}{1.305308in}}%
\pgfpathlineto{\pgfqpoint{1.572419in}{1.309206in}}%
\pgfpathlineto{\pgfqpoint{1.577742in}{1.312934in}}%
\pgfpathlineto{\pgfqpoint{1.583066in}{1.316493in}}%
\pgfpathlineto{\pgfqpoint{1.588389in}{1.319882in}}%
\pgfpathlineto{\pgfqpoint{1.593712in}{1.323102in}}%
\pgfpathlineto{\pgfqpoint{1.599035in}{1.326153in}}%
\pgfpathlineto{\pgfqpoint{1.604359in}{1.329034in}}%
\pgfpathlineto{\pgfqpoint{1.609682in}{1.331745in}}%
\pgfpathlineto{\pgfqpoint{1.615005in}{1.334288in}}%
\pgfpathlineto{\pgfqpoint{1.620328in}{1.336660in}}%
\pgfpathlineto{\pgfqpoint{1.625652in}{1.338863in}}%
\pgfpathlineto{\pgfqpoint{1.630975in}{1.340897in}}%
\pgfpathlineto{\pgfqpoint{1.636298in}{1.342761in}}%
\pgfpathlineto{\pgfqpoint{1.641621in}{1.344456in}}%
\pgfpathlineto{\pgfqpoint{1.646944in}{1.345981in}}%
\pgfpathlineto{\pgfqpoint{1.652268in}{1.347337in}}%
\pgfpathlineto{\pgfqpoint{1.657591in}{1.348523in}}%
\pgfpathlineto{\pgfqpoint{1.662914in}{1.349540in}}%
\pgfpathlineto{\pgfqpoint{1.668237in}{1.350387in}}%
\pgfpathlineto{\pgfqpoint{1.673561in}{1.351065in}}%
\pgfpathlineto{\pgfqpoint{1.678884in}{1.351574in}}%
\pgfpathlineto{\pgfqpoint{1.684207in}{1.351913in}}%
\pgfpathlineto{\pgfqpoint{1.689530in}{1.352082in}}%
\pgfpathlineto{\pgfqpoint{1.694854in}{1.352082in}}%
\pgfpathlineto{\pgfqpoint{1.700177in}{1.351913in}}%
\pgfpathlineto{\pgfqpoint{1.705500in}{1.351574in}}%
\pgfpathlineto{\pgfqpoint{1.710823in}{1.351065in}}%
\pgfpathlineto{\pgfqpoint{1.716146in}{1.350387in}}%
\pgfpathlineto{\pgfqpoint{1.721470in}{1.349540in}}%
\pgfpathlineto{\pgfqpoint{1.726793in}{1.348523in}}%
\pgfpathlineto{\pgfqpoint{1.732116in}{1.347337in}}%
\pgfpathlineto{\pgfqpoint{1.737439in}{1.345981in}}%
\pgfpathlineto{\pgfqpoint{1.742763in}{1.344456in}}%
\pgfpathlineto{\pgfqpoint{1.748086in}{1.342761in}}%
\pgfpathlineto{\pgfqpoint{1.753409in}{1.340897in}}%
\pgfpathlineto{\pgfqpoint{1.758732in}{1.338863in}}%
\pgfpathlineto{\pgfqpoint{1.764056in}{1.336660in}}%
\pgfpathlineto{\pgfqpoint{1.769379in}{1.334288in}}%
\pgfpathlineto{\pgfqpoint{1.774702in}{1.331745in}}%
\pgfpathlineto{\pgfqpoint{1.780025in}{1.329034in}}%
\pgfpathlineto{\pgfqpoint{1.785349in}{1.326153in}}%
\pgfpathlineto{\pgfqpoint{1.790672in}{1.323102in}}%
\pgfpathlineto{\pgfqpoint{1.795995in}{1.319882in}}%
\pgfpathlineto{\pgfqpoint{1.801318in}{1.316493in}}%
\pgfpathlineto{\pgfqpoint{1.806641in}{1.312934in}}%
\pgfpathlineto{\pgfqpoint{1.811965in}{1.309206in}}%
\pgfpathlineto{\pgfqpoint{1.817288in}{1.305308in}}%
\pgfpathlineto{\pgfqpoint{1.822611in}{1.301240in}}%
\pgfpathlineto{\pgfqpoint{1.827934in}{1.297004in}}%
\pgfpathlineto{\pgfqpoint{1.833258in}{1.292597in}}%
\pgfpathlineto{\pgfqpoint{1.838581in}{1.288022in}}%
\pgfpathlineto{\pgfqpoint{1.843904in}{1.283276in}}%
\pgfpathlineto{\pgfqpoint{1.849227in}{1.278362in}}%
\pgfpathlineto{\pgfqpoint{1.854551in}{1.273277in}}%
\pgfpathlineto{\pgfqpoint{1.859874in}{1.268024in}}%
\pgfpathlineto{\pgfqpoint{1.865197in}{1.262601in}}%
\pgfpathlineto{\pgfqpoint{1.870520in}{1.257008in}}%
\pgfpathlineto{\pgfqpoint{1.875843in}{1.251246in}}%
\pgfpathlineto{\pgfqpoint{1.881167in}{1.245314in}}%
\pgfpathlineto{\pgfqpoint{1.886490in}{1.239213in}}%
\pgfpathlineto{\pgfqpoint{1.891813in}{1.232943in}}%
\pgfpathlineto{\pgfqpoint{1.897136in}{1.226503in}}%
\pgfpathlineto{\pgfqpoint{1.902460in}{1.219894in}}%
\pgfpathlineto{\pgfqpoint{1.907783in}{1.213115in}}%
\pgfpathlineto{\pgfqpoint{1.913106in}{1.206166in}}%
\pgfpathlineto{\pgfqpoint{1.918429in}{1.199048in}}%
\pgfpathlineto{\pgfqpoint{1.923753in}{1.191761in}}%
\pgfpathlineto{\pgfqpoint{1.929076in}{1.184304in}}%
\pgfpathlineto{\pgfqpoint{1.934399in}{1.176678in}}%
\pgfpathlineto{\pgfqpoint{1.939722in}{1.168882in}}%
\pgfpathlineto{\pgfqpoint{1.945045in}{1.160917in}}%
\pgfpathlineto{\pgfqpoint{1.950369in}{1.152782in}}%
\pgfpathlineto{\pgfqpoint{1.955692in}{1.144478in}}%
\pgfusepath{stroke}%
\end{pgfscope}%
\begin{pgfscope}%
\pgfpathrectangle{\pgfqpoint{0.374692in}{0.521603in}}{\pgfqpoint{2.635000in}{1.661000in}} %
\pgfusepath{clip}%
\pgfsetrectcap%
\pgfsetroundjoin%
\pgfsetlinewidth{1.003750pt}%
\definecolor{currentstroke}{rgb}{0.580392,0.403922,0.741176}%
\pgfsetstrokecolor{currentstroke}%
\pgfsetdash{}{0pt}%
\pgfpathmoveto{\pgfqpoint{1.428692in}{0.729228in}}%
\pgfpathlineto{\pgfqpoint{1.434015in}{0.729271in}}%
\pgfpathlineto{\pgfqpoint{1.439338in}{0.729398in}}%
\pgfpathlineto{\pgfqpoint{1.444662in}{0.729610in}}%
\pgfpathlineto{\pgfqpoint{1.449985in}{0.729906in}}%
\pgfpathlineto{\pgfqpoint{1.455308in}{0.730288in}}%
\pgfpathlineto{\pgfqpoint{1.460631in}{0.730754in}}%
\pgfpathlineto{\pgfqpoint{1.465955in}{0.731304in}}%
\pgfpathlineto{\pgfqpoint{1.471278in}{0.731940in}}%
\pgfpathlineto{\pgfqpoint{1.476601in}{0.732660in}}%
\pgfpathlineto{\pgfqpoint{1.481924in}{0.733465in}}%
\pgfpathlineto{\pgfqpoint{1.487248in}{0.734355in}}%
\pgfpathlineto{\pgfqpoint{1.492571in}{0.735329in}}%
\pgfpathlineto{\pgfqpoint{1.497894in}{0.736389in}}%
\pgfpathlineto{\pgfqpoint{1.503217in}{0.737532in}}%
\pgfpathlineto{\pgfqpoint{1.508540in}{0.738761in}}%
\pgfpathlineto{\pgfqpoint{1.513864in}{0.740075in}}%
\pgfpathlineto{\pgfqpoint{1.519187in}{0.741473in}}%
\pgfpathlineto{\pgfqpoint{1.524510in}{0.742956in}}%
\pgfpathlineto{\pgfqpoint{1.529833in}{0.744523in}}%
\pgfpathlineto{\pgfqpoint{1.535157in}{0.746176in}}%
\pgfpathlineto{\pgfqpoint{1.540480in}{0.747913in}}%
\pgfpathlineto{\pgfqpoint{1.545803in}{0.749735in}}%
\pgfpathlineto{\pgfqpoint{1.551126in}{0.751641in}}%
\pgfpathlineto{\pgfqpoint{1.556450in}{0.753632in}}%
\pgfpathlineto{\pgfqpoint{1.561773in}{0.755708in}}%
\pgfpathlineto{\pgfqpoint{1.567096in}{0.757869in}}%
\pgfpathlineto{\pgfqpoint{1.572419in}{0.760115in}}%
\pgfpathlineto{\pgfqpoint{1.577742in}{0.762445in}}%
\pgfpathlineto{\pgfqpoint{1.583066in}{0.764860in}}%
\pgfpathlineto{\pgfqpoint{1.588389in}{0.767360in}}%
\pgfpathlineto{\pgfqpoint{1.593712in}{0.769944in}}%
\pgfpathlineto{\pgfqpoint{1.599035in}{0.772613in}}%
\pgfpathlineto{\pgfqpoint{1.604359in}{0.775367in}}%
\pgfpathlineto{\pgfqpoint{1.609682in}{0.778206in}}%
\pgfpathlineto{\pgfqpoint{1.615005in}{0.781129in}}%
\pgfpathlineto{\pgfqpoint{1.620328in}{0.784137in}}%
\pgfpathlineto{\pgfqpoint{1.625652in}{0.787230in}}%
\pgfpathlineto{\pgfqpoint{1.630975in}{0.790408in}}%
\pgfpathlineto{\pgfqpoint{1.636298in}{0.793670in}}%
\pgfpathlineto{\pgfqpoint{1.641621in}{0.797017in}}%
\pgfpathlineto{\pgfqpoint{1.646944in}{0.800449in}}%
\pgfpathlineto{\pgfqpoint{1.652268in}{0.803966in}}%
\pgfpathlineto{\pgfqpoint{1.657591in}{0.807567in}}%
\pgfpathlineto{\pgfqpoint{1.662914in}{0.811253in}}%
\pgfpathlineto{\pgfqpoint{1.668237in}{0.815024in}}%
\pgfpathlineto{\pgfqpoint{1.673561in}{0.818879in}}%
\pgfpathlineto{\pgfqpoint{1.678884in}{0.822820in}}%
\pgfpathlineto{\pgfqpoint{1.684207in}{0.826844in}}%
\pgfpathlineto{\pgfqpoint{1.689530in}{0.830954in}}%
\pgfpathlineto{\pgfqpoint{1.694854in}{0.835149in}}%
\pgfpathlineto{\pgfqpoint{1.700177in}{0.839428in}}%
\pgfpathlineto{\pgfqpoint{1.705500in}{0.843792in}}%
\pgfpathlineto{\pgfqpoint{1.710823in}{0.848240in}}%
\pgfpathlineto{\pgfqpoint{1.716146in}{0.852774in}}%
\pgfpathlineto{\pgfqpoint{1.721470in}{0.857392in}}%
\pgfpathlineto{\pgfqpoint{1.726793in}{0.862095in}}%
\pgfpathlineto{\pgfqpoint{1.732116in}{0.866882in}}%
\pgfpathlineto{\pgfqpoint{1.737439in}{0.871755in}}%
\pgfpathlineto{\pgfqpoint{1.742763in}{0.876712in}}%
\pgfpathlineto{\pgfqpoint{1.748086in}{0.881754in}}%
\pgfpathlineto{\pgfqpoint{1.753409in}{0.886880in}}%
\pgfpathlineto{\pgfqpoint{1.758732in}{0.892091in}}%
\pgfpathlineto{\pgfqpoint{1.764056in}{0.897387in}}%
\pgfpathlineto{\pgfqpoint{1.769379in}{0.902768in}}%
\pgfpathlineto{\pgfqpoint{1.774702in}{0.908234in}}%
\pgfpathlineto{\pgfqpoint{1.780025in}{0.913784in}}%
\pgfpathlineto{\pgfqpoint{1.785349in}{0.919419in}}%
\pgfpathlineto{\pgfqpoint{1.790672in}{0.925139in}}%
\pgfpathlineto{\pgfqpoint{1.795995in}{0.930943in}}%
\pgfpathlineto{\pgfqpoint{1.801318in}{0.936832in}}%
\pgfpathlineto{\pgfqpoint{1.806641in}{0.942806in}}%
\pgfpathlineto{\pgfqpoint{1.811965in}{0.948865in}}%
\pgfpathlineto{\pgfqpoint{1.817288in}{0.955008in}}%
\pgfpathlineto{\pgfqpoint{1.822611in}{0.961236in}}%
\pgfpathlineto{\pgfqpoint{1.827934in}{0.967549in}}%
\pgfpathlineto{\pgfqpoint{1.833258in}{0.973947in}}%
\pgfpathlineto{\pgfqpoint{1.838581in}{0.980429in}}%
\pgfpathlineto{\pgfqpoint{1.843904in}{0.986996in}}%
\pgfpathlineto{\pgfqpoint{1.849227in}{0.993648in}}%
\pgfpathlineto{\pgfqpoint{1.854551in}{1.000384in}}%
\pgfpathlineto{\pgfqpoint{1.859874in}{1.007206in}}%
\pgfpathlineto{\pgfqpoint{1.865197in}{1.014112in}}%
\pgfpathlineto{\pgfqpoint{1.870520in}{1.021102in}}%
\pgfpathlineto{\pgfqpoint{1.875843in}{1.028178in}}%
\pgfpathlineto{\pgfqpoint{1.881167in}{1.035338in}}%
\pgfpathlineto{\pgfqpoint{1.886490in}{1.042583in}}%
\pgfpathlineto{\pgfqpoint{1.891813in}{1.049913in}}%
\pgfpathlineto{\pgfqpoint{1.897136in}{1.057327in}}%
\pgfpathlineto{\pgfqpoint{1.902460in}{1.064826in}}%
\pgfpathlineto{\pgfqpoint{1.907783in}{1.072410in}}%
\pgfpathlineto{\pgfqpoint{1.913106in}{1.080079in}}%
\pgfpathlineto{\pgfqpoint{1.918429in}{1.087832in}}%
\pgfpathlineto{\pgfqpoint{1.923753in}{1.095670in}}%
\pgfpathlineto{\pgfqpoint{1.929076in}{1.103593in}}%
\pgfpathlineto{\pgfqpoint{1.934399in}{1.111601in}}%
\pgfpathlineto{\pgfqpoint{1.939722in}{1.119693in}}%
\pgfpathlineto{\pgfqpoint{1.945045in}{1.127870in}}%
\pgfpathlineto{\pgfqpoint{1.950369in}{1.136132in}}%
\pgfpathlineto{\pgfqpoint{1.955692in}{1.144478in}}%
\pgfusepath{stroke}%
\end{pgfscope}%
\begin{pgfscope}%
\pgfpathrectangle{\pgfqpoint{0.374692in}{0.521603in}}{\pgfqpoint{2.635000in}{1.661000in}} %
\pgfusepath{clip}%
\pgfsetbuttcap%
\pgfsetroundjoin%
\pgfsetlinewidth{1.505625pt}%
\definecolor{currentstroke}{rgb}{0.000000,0.000000,0.000000}%
\pgfsetstrokecolor{currentstroke}%
\pgfsetdash{{5.550000pt}{2.400000pt}}{0.000000pt}%
\pgfpathmoveto{\pgfqpoint{1.428692in}{0.729228in}}%
\pgfpathlineto{\pgfqpoint{1.428692in}{0.798437in}}%
\pgfpathlineto{\pgfqpoint{1.428692in}{0.867645in}}%
\pgfpathlineto{\pgfqpoint{1.428692in}{0.936853in}}%
\pgfpathlineto{\pgfqpoint{1.428692in}{1.006062in}}%
\pgfpathlineto{\pgfqpoint{1.428692in}{1.075270in}}%
\pgfpathlineto{\pgfqpoint{1.428692in}{1.144478in}}%
\pgfpathlineto{\pgfqpoint{1.428692in}{1.213687in}}%
\pgfpathlineto{\pgfqpoint{1.428692in}{1.282895in}}%
\pgfpathlineto{\pgfqpoint{1.428692in}{1.352103in}}%
\pgfusepath{stroke}%
\end{pgfscope}%
\begin{pgfscope}%
\pgfpathrectangle{\pgfqpoint{0.374692in}{0.521603in}}{\pgfqpoint{2.635000in}{1.661000in}} %
\pgfusepath{clip}%
\pgfsetrectcap%
\pgfsetroundjoin%
\pgfsetlinewidth{1.003750pt}%
\definecolor{currentstroke}{rgb}{0.839216,0.152941,0.156863}%
\pgfsetstrokecolor{currentstroke}%
\pgfsetdash{}{0pt}%
\pgfpathmoveto{\pgfqpoint{1.955692in}{1.144478in}}%
\pgfpathlineto{\pgfqpoint{1.961015in}{1.136132in}}%
\pgfpathlineto{\pgfqpoint{1.966338in}{1.127870in}}%
\pgfpathlineto{\pgfqpoint{1.971662in}{1.119693in}}%
\pgfpathlineto{\pgfqpoint{1.976985in}{1.111601in}}%
\pgfpathlineto{\pgfqpoint{1.982308in}{1.103593in}}%
\pgfpathlineto{\pgfqpoint{1.987631in}{1.095670in}}%
\pgfpathlineto{\pgfqpoint{1.992955in}{1.087832in}}%
\pgfpathlineto{\pgfqpoint{1.998278in}{1.080079in}}%
\pgfpathlineto{\pgfqpoint{2.003601in}{1.072410in}}%
\pgfpathlineto{\pgfqpoint{2.008924in}{1.064826in}}%
\pgfpathlineto{\pgfqpoint{2.014248in}{1.057327in}}%
\pgfpathlineto{\pgfqpoint{2.019571in}{1.049913in}}%
\pgfpathlineto{\pgfqpoint{2.024894in}{1.042583in}}%
\pgfpathlineto{\pgfqpoint{2.030217in}{1.035338in}}%
\pgfpathlineto{\pgfqpoint{2.035540in}{1.028178in}}%
\pgfpathlineto{\pgfqpoint{2.040864in}{1.021102in}}%
\pgfpathlineto{\pgfqpoint{2.046187in}{1.014112in}}%
\pgfpathlineto{\pgfqpoint{2.051510in}{1.007206in}}%
\pgfpathlineto{\pgfqpoint{2.056833in}{1.000384in}}%
\pgfpathlineto{\pgfqpoint{2.062157in}{0.993648in}}%
\pgfpathlineto{\pgfqpoint{2.067480in}{0.986996in}}%
\pgfpathlineto{\pgfqpoint{2.072803in}{0.980429in}}%
\pgfpathlineto{\pgfqpoint{2.078126in}{0.973947in}}%
\pgfpathlineto{\pgfqpoint{2.083450in}{0.967549in}}%
\pgfpathlineto{\pgfqpoint{2.088773in}{0.961236in}}%
\pgfpathlineto{\pgfqpoint{2.094096in}{0.955008in}}%
\pgfpathlineto{\pgfqpoint{2.099419in}{0.948865in}}%
\pgfpathlineto{\pgfqpoint{2.104742in}{0.942806in}}%
\pgfpathlineto{\pgfqpoint{2.110066in}{0.936832in}}%
\pgfpathlineto{\pgfqpoint{2.115389in}{0.930943in}}%
\pgfpathlineto{\pgfqpoint{2.120712in}{0.925139in}}%
\pgfpathlineto{\pgfqpoint{2.126035in}{0.919419in}}%
\pgfpathlineto{\pgfqpoint{2.131359in}{0.913784in}}%
\pgfpathlineto{\pgfqpoint{2.136682in}{0.908234in}}%
\pgfpathlineto{\pgfqpoint{2.142005in}{0.902768in}}%
\pgfpathlineto{\pgfqpoint{2.147328in}{0.897387in}}%
\pgfpathlineto{\pgfqpoint{2.152652in}{0.892091in}}%
\pgfpathlineto{\pgfqpoint{2.157975in}{0.886880in}}%
\pgfpathlineto{\pgfqpoint{2.163298in}{0.881754in}}%
\pgfpathlineto{\pgfqpoint{2.168621in}{0.876712in}}%
\pgfpathlineto{\pgfqpoint{2.173944in}{0.871755in}}%
\pgfpathlineto{\pgfqpoint{2.179268in}{0.866882in}}%
\pgfpathlineto{\pgfqpoint{2.184591in}{0.862095in}}%
\pgfpathlineto{\pgfqpoint{2.189914in}{0.857392in}}%
\pgfpathlineto{\pgfqpoint{2.195237in}{0.852774in}}%
\pgfpathlineto{\pgfqpoint{2.200561in}{0.848240in}}%
\pgfpathlineto{\pgfqpoint{2.205884in}{0.843792in}}%
\pgfpathlineto{\pgfqpoint{2.211207in}{0.839428in}}%
\pgfpathlineto{\pgfqpoint{2.216530in}{0.835149in}}%
\pgfpathlineto{\pgfqpoint{2.221854in}{0.830954in}}%
\pgfpathlineto{\pgfqpoint{2.227177in}{0.826844in}}%
\pgfpathlineto{\pgfqpoint{2.232500in}{0.822820in}}%
\pgfpathlineto{\pgfqpoint{2.237823in}{0.818879in}}%
\pgfpathlineto{\pgfqpoint{2.243146in}{0.815024in}}%
\pgfpathlineto{\pgfqpoint{2.248470in}{0.811253in}}%
\pgfpathlineto{\pgfqpoint{2.253793in}{0.807567in}}%
\pgfpathlineto{\pgfqpoint{2.259116in}{0.803966in}}%
\pgfpathlineto{\pgfqpoint{2.264439in}{0.800449in}}%
\pgfpathlineto{\pgfqpoint{2.269763in}{0.797017in}}%
\pgfpathlineto{\pgfqpoint{2.275086in}{0.793670in}}%
\pgfpathlineto{\pgfqpoint{2.280409in}{0.790408in}}%
\pgfpathlineto{\pgfqpoint{2.285732in}{0.787230in}}%
\pgfpathlineto{\pgfqpoint{2.291056in}{0.784137in}}%
\pgfpathlineto{\pgfqpoint{2.296379in}{0.781129in}}%
\pgfpathlineto{\pgfqpoint{2.301702in}{0.778206in}}%
\pgfpathlineto{\pgfqpoint{2.307025in}{0.775367in}}%
\pgfpathlineto{\pgfqpoint{2.312349in}{0.772613in}}%
\pgfpathlineto{\pgfqpoint{2.317672in}{0.769944in}}%
\pgfpathlineto{\pgfqpoint{2.322995in}{0.767360in}}%
\pgfpathlineto{\pgfqpoint{2.328318in}{0.764860in}}%
\pgfpathlineto{\pgfqpoint{2.333641in}{0.762445in}}%
\pgfpathlineto{\pgfqpoint{2.338965in}{0.760115in}}%
\pgfpathlineto{\pgfqpoint{2.344288in}{0.757869in}}%
\pgfpathlineto{\pgfqpoint{2.349611in}{0.755708in}}%
\pgfpathlineto{\pgfqpoint{2.354934in}{0.753632in}}%
\pgfpathlineto{\pgfqpoint{2.360258in}{0.751641in}}%
\pgfpathlineto{\pgfqpoint{2.365581in}{0.749735in}}%
\pgfpathlineto{\pgfqpoint{2.370904in}{0.747913in}}%
\pgfpathlineto{\pgfqpoint{2.376227in}{0.746176in}}%
\pgfpathlineto{\pgfqpoint{2.381551in}{0.744523in}}%
\pgfpathlineto{\pgfqpoint{2.386874in}{0.742956in}}%
\pgfpathlineto{\pgfqpoint{2.392197in}{0.741473in}}%
\pgfpathlineto{\pgfqpoint{2.397520in}{0.740075in}}%
\pgfpathlineto{\pgfqpoint{2.402843in}{0.738761in}}%
\pgfpathlineto{\pgfqpoint{2.408167in}{0.737532in}}%
\pgfpathlineto{\pgfqpoint{2.413490in}{0.736389in}}%
\pgfpathlineto{\pgfqpoint{2.418813in}{0.735329in}}%
\pgfpathlineto{\pgfqpoint{2.424136in}{0.734355in}}%
\pgfpathlineto{\pgfqpoint{2.429460in}{0.733465in}}%
\pgfpathlineto{\pgfqpoint{2.434783in}{0.732660in}}%
\pgfpathlineto{\pgfqpoint{2.440106in}{0.731940in}}%
\pgfpathlineto{\pgfqpoint{2.445429in}{0.731304in}}%
\pgfpathlineto{\pgfqpoint{2.450753in}{0.730754in}}%
\pgfpathlineto{\pgfqpoint{2.456076in}{0.730288in}}%
\pgfpathlineto{\pgfqpoint{2.461399in}{0.729906in}}%
\pgfpathlineto{\pgfqpoint{2.466722in}{0.729610in}}%
\pgfpathlineto{\pgfqpoint{2.472045in}{0.729398in}}%
\pgfpathlineto{\pgfqpoint{2.477369in}{0.729271in}}%
\pgfpathlineto{\pgfqpoint{2.482692in}{0.729228in}}%
\pgfusepath{stroke}%
\end{pgfscope}%
\begin{pgfscope}%
\pgfpathrectangle{\pgfqpoint{0.374692in}{0.521603in}}{\pgfqpoint{2.635000in}{1.661000in}} %
\pgfusepath{clip}%
\pgfsetrectcap%
\pgfsetroundjoin%
\pgfsetlinewidth{1.003750pt}%
\definecolor{currentstroke}{rgb}{0.580392,0.403922,0.741176}%
\pgfsetstrokecolor{currentstroke}%
\pgfsetdash{}{0pt}%
\pgfpathmoveto{\pgfqpoint{1.955692in}{1.144478in}}%
\pgfpathlineto{\pgfqpoint{1.961015in}{1.152782in}}%
\pgfpathlineto{\pgfqpoint{1.966338in}{1.160917in}}%
\pgfpathlineto{\pgfqpoint{1.971662in}{1.168882in}}%
\pgfpathlineto{\pgfqpoint{1.976985in}{1.176678in}}%
\pgfpathlineto{\pgfqpoint{1.982308in}{1.184304in}}%
\pgfpathlineto{\pgfqpoint{1.987631in}{1.191761in}}%
\pgfpathlineto{\pgfqpoint{1.992955in}{1.199048in}}%
\pgfpathlineto{\pgfqpoint{1.998278in}{1.206166in}}%
\pgfpathlineto{\pgfqpoint{2.003601in}{1.213115in}}%
\pgfpathlineto{\pgfqpoint{2.008924in}{1.219894in}}%
\pgfpathlineto{\pgfqpoint{2.014248in}{1.226503in}}%
\pgfpathlineto{\pgfqpoint{2.019571in}{1.232943in}}%
\pgfpathlineto{\pgfqpoint{2.024894in}{1.239213in}}%
\pgfpathlineto{\pgfqpoint{2.030217in}{1.245314in}}%
\pgfpathlineto{\pgfqpoint{2.035540in}{1.251246in}}%
\pgfpathlineto{\pgfqpoint{2.040864in}{1.257008in}}%
\pgfpathlineto{\pgfqpoint{2.046187in}{1.262601in}}%
\pgfpathlineto{\pgfqpoint{2.051510in}{1.268024in}}%
\pgfpathlineto{\pgfqpoint{2.056833in}{1.273277in}}%
\pgfpathlineto{\pgfqpoint{2.062157in}{1.278362in}}%
\pgfpathlineto{\pgfqpoint{2.067480in}{1.283276in}}%
\pgfpathlineto{\pgfqpoint{2.072803in}{1.288022in}}%
\pgfpathlineto{\pgfqpoint{2.078126in}{1.292597in}}%
\pgfpathlineto{\pgfqpoint{2.083450in}{1.297004in}}%
\pgfpathlineto{\pgfqpoint{2.088773in}{1.301240in}}%
\pgfpathlineto{\pgfqpoint{2.094096in}{1.305308in}}%
\pgfpathlineto{\pgfqpoint{2.099419in}{1.309206in}}%
\pgfpathlineto{\pgfqpoint{2.104742in}{1.312934in}}%
\pgfpathlineto{\pgfqpoint{2.110066in}{1.316493in}}%
\pgfpathlineto{\pgfqpoint{2.115389in}{1.319882in}}%
\pgfpathlineto{\pgfqpoint{2.120712in}{1.323102in}}%
\pgfpathlineto{\pgfqpoint{2.126035in}{1.326153in}}%
\pgfpathlineto{\pgfqpoint{2.131359in}{1.329034in}}%
\pgfpathlineto{\pgfqpoint{2.136682in}{1.331745in}}%
\pgfpathlineto{\pgfqpoint{2.142005in}{1.334288in}}%
\pgfpathlineto{\pgfqpoint{2.147328in}{1.336660in}}%
\pgfpathlineto{\pgfqpoint{2.152652in}{1.338863in}}%
\pgfpathlineto{\pgfqpoint{2.157975in}{1.340897in}}%
\pgfpathlineto{\pgfqpoint{2.163298in}{1.342761in}}%
\pgfpathlineto{\pgfqpoint{2.168621in}{1.344456in}}%
\pgfpathlineto{\pgfqpoint{2.173944in}{1.345981in}}%
\pgfpathlineto{\pgfqpoint{2.179268in}{1.347337in}}%
\pgfpathlineto{\pgfqpoint{2.184591in}{1.348523in}}%
\pgfpathlineto{\pgfqpoint{2.189914in}{1.349540in}}%
\pgfpathlineto{\pgfqpoint{2.195237in}{1.350387in}}%
\pgfpathlineto{\pgfqpoint{2.200561in}{1.351065in}}%
\pgfpathlineto{\pgfqpoint{2.205884in}{1.351574in}}%
\pgfpathlineto{\pgfqpoint{2.211207in}{1.351913in}}%
\pgfpathlineto{\pgfqpoint{2.216530in}{1.352082in}}%
\pgfpathlineto{\pgfqpoint{2.221854in}{1.352082in}}%
\pgfpathlineto{\pgfqpoint{2.227177in}{1.351913in}}%
\pgfpathlineto{\pgfqpoint{2.232500in}{1.351574in}}%
\pgfpathlineto{\pgfqpoint{2.237823in}{1.351065in}}%
\pgfpathlineto{\pgfqpoint{2.243146in}{1.350387in}}%
\pgfpathlineto{\pgfqpoint{2.248470in}{1.349540in}}%
\pgfpathlineto{\pgfqpoint{2.253793in}{1.348523in}}%
\pgfpathlineto{\pgfqpoint{2.259116in}{1.347337in}}%
\pgfpathlineto{\pgfqpoint{2.264439in}{1.345981in}}%
\pgfpathlineto{\pgfqpoint{2.269763in}{1.344456in}}%
\pgfpathlineto{\pgfqpoint{2.275086in}{1.342761in}}%
\pgfpathlineto{\pgfqpoint{2.280409in}{1.340897in}}%
\pgfpathlineto{\pgfqpoint{2.285732in}{1.338863in}}%
\pgfpathlineto{\pgfqpoint{2.291056in}{1.336660in}}%
\pgfpathlineto{\pgfqpoint{2.296379in}{1.334288in}}%
\pgfpathlineto{\pgfqpoint{2.301702in}{1.331745in}}%
\pgfpathlineto{\pgfqpoint{2.307025in}{1.329034in}}%
\pgfpathlineto{\pgfqpoint{2.312349in}{1.326153in}}%
\pgfpathlineto{\pgfqpoint{2.317672in}{1.323102in}}%
\pgfpathlineto{\pgfqpoint{2.322995in}{1.319882in}}%
\pgfpathlineto{\pgfqpoint{2.328318in}{1.316493in}}%
\pgfpathlineto{\pgfqpoint{2.333641in}{1.312934in}}%
\pgfpathlineto{\pgfqpoint{2.338965in}{1.309206in}}%
\pgfpathlineto{\pgfqpoint{2.344288in}{1.305308in}}%
\pgfpathlineto{\pgfqpoint{2.349611in}{1.301240in}}%
\pgfpathlineto{\pgfqpoint{2.354934in}{1.297004in}}%
\pgfpathlineto{\pgfqpoint{2.360258in}{1.292597in}}%
\pgfpathlineto{\pgfqpoint{2.365581in}{1.288022in}}%
\pgfpathlineto{\pgfqpoint{2.370904in}{1.283276in}}%
\pgfpathlineto{\pgfqpoint{2.376227in}{1.278362in}}%
\pgfpathlineto{\pgfqpoint{2.381551in}{1.273277in}}%
\pgfpathlineto{\pgfqpoint{2.386874in}{1.268024in}}%
\pgfpathlineto{\pgfqpoint{2.392197in}{1.262601in}}%
\pgfpathlineto{\pgfqpoint{2.397520in}{1.257008in}}%
\pgfpathlineto{\pgfqpoint{2.402843in}{1.251246in}}%
\pgfpathlineto{\pgfqpoint{2.408167in}{1.245314in}}%
\pgfpathlineto{\pgfqpoint{2.413490in}{1.239213in}}%
\pgfpathlineto{\pgfqpoint{2.418813in}{1.232943in}}%
\pgfpathlineto{\pgfqpoint{2.424136in}{1.226503in}}%
\pgfpathlineto{\pgfqpoint{2.429460in}{1.219894in}}%
\pgfpathlineto{\pgfqpoint{2.434783in}{1.213115in}}%
\pgfpathlineto{\pgfqpoint{2.440106in}{1.206166in}}%
\pgfpathlineto{\pgfqpoint{2.445429in}{1.199048in}}%
\pgfpathlineto{\pgfqpoint{2.450753in}{1.191761in}}%
\pgfpathlineto{\pgfqpoint{2.456076in}{1.184304in}}%
\pgfpathlineto{\pgfqpoint{2.461399in}{1.176678in}}%
\pgfpathlineto{\pgfqpoint{2.466722in}{1.168882in}}%
\pgfpathlineto{\pgfqpoint{2.472045in}{1.160917in}}%
\pgfpathlineto{\pgfqpoint{2.477369in}{1.152782in}}%
\pgfpathlineto{\pgfqpoint{2.482692in}{1.144478in}}%
\pgfusepath{stroke}%
\end{pgfscope}%
\begin{pgfscope}%
\pgfpathrectangle{\pgfqpoint{0.374692in}{0.521603in}}{\pgfqpoint{2.635000in}{1.661000in}} %
\pgfusepath{clip}%
\pgfsetrectcap%
\pgfsetroundjoin%
\pgfsetlinewidth{1.003750pt}%
\definecolor{currentstroke}{rgb}{0.121569,0.466667,0.705882}%
\pgfsetstrokecolor{currentstroke}%
\pgfsetdash{}{0pt}%
\pgfpathmoveto{\pgfqpoint{1.955692in}{0.729228in}}%
\pgfpathlineto{\pgfqpoint{1.961015in}{0.729271in}}%
\pgfpathlineto{\pgfqpoint{1.966338in}{0.729398in}}%
\pgfpathlineto{\pgfqpoint{1.971662in}{0.729610in}}%
\pgfpathlineto{\pgfqpoint{1.976985in}{0.729906in}}%
\pgfpathlineto{\pgfqpoint{1.982308in}{0.730288in}}%
\pgfpathlineto{\pgfqpoint{1.987631in}{0.730754in}}%
\pgfpathlineto{\pgfqpoint{1.992955in}{0.731304in}}%
\pgfpathlineto{\pgfqpoint{1.998278in}{0.731940in}}%
\pgfpathlineto{\pgfqpoint{2.003601in}{0.732660in}}%
\pgfpathlineto{\pgfqpoint{2.008924in}{0.733465in}}%
\pgfpathlineto{\pgfqpoint{2.014248in}{0.734355in}}%
\pgfpathlineto{\pgfqpoint{2.019571in}{0.735329in}}%
\pgfpathlineto{\pgfqpoint{2.024894in}{0.736389in}}%
\pgfpathlineto{\pgfqpoint{2.030217in}{0.737532in}}%
\pgfpathlineto{\pgfqpoint{2.035540in}{0.738761in}}%
\pgfpathlineto{\pgfqpoint{2.040864in}{0.740075in}}%
\pgfpathlineto{\pgfqpoint{2.046187in}{0.741473in}}%
\pgfpathlineto{\pgfqpoint{2.051510in}{0.742956in}}%
\pgfpathlineto{\pgfqpoint{2.056833in}{0.744523in}}%
\pgfpathlineto{\pgfqpoint{2.062157in}{0.746176in}}%
\pgfpathlineto{\pgfqpoint{2.067480in}{0.747913in}}%
\pgfpathlineto{\pgfqpoint{2.072803in}{0.749735in}}%
\pgfpathlineto{\pgfqpoint{2.078126in}{0.751641in}}%
\pgfpathlineto{\pgfqpoint{2.083450in}{0.753632in}}%
\pgfpathlineto{\pgfqpoint{2.088773in}{0.755708in}}%
\pgfpathlineto{\pgfqpoint{2.094096in}{0.757869in}}%
\pgfpathlineto{\pgfqpoint{2.099419in}{0.760115in}}%
\pgfpathlineto{\pgfqpoint{2.104742in}{0.762445in}}%
\pgfpathlineto{\pgfqpoint{2.110066in}{0.764860in}}%
\pgfpathlineto{\pgfqpoint{2.115389in}{0.767360in}}%
\pgfpathlineto{\pgfqpoint{2.120712in}{0.769944in}}%
\pgfpathlineto{\pgfqpoint{2.126035in}{0.772613in}}%
\pgfpathlineto{\pgfqpoint{2.131359in}{0.775367in}}%
\pgfpathlineto{\pgfqpoint{2.136682in}{0.778206in}}%
\pgfpathlineto{\pgfqpoint{2.142005in}{0.781129in}}%
\pgfpathlineto{\pgfqpoint{2.147328in}{0.784137in}}%
\pgfpathlineto{\pgfqpoint{2.152652in}{0.787230in}}%
\pgfpathlineto{\pgfqpoint{2.157975in}{0.790408in}}%
\pgfpathlineto{\pgfqpoint{2.163298in}{0.793670in}}%
\pgfpathlineto{\pgfqpoint{2.168621in}{0.797017in}}%
\pgfpathlineto{\pgfqpoint{2.173944in}{0.800449in}}%
\pgfpathlineto{\pgfqpoint{2.179268in}{0.803966in}}%
\pgfpathlineto{\pgfqpoint{2.184591in}{0.807567in}}%
\pgfpathlineto{\pgfqpoint{2.189914in}{0.811253in}}%
\pgfpathlineto{\pgfqpoint{2.195237in}{0.815024in}}%
\pgfpathlineto{\pgfqpoint{2.200561in}{0.818879in}}%
\pgfpathlineto{\pgfqpoint{2.205884in}{0.822820in}}%
\pgfpathlineto{\pgfqpoint{2.211207in}{0.826844in}}%
\pgfpathlineto{\pgfqpoint{2.216530in}{0.830954in}}%
\pgfpathlineto{\pgfqpoint{2.221854in}{0.835149in}}%
\pgfpathlineto{\pgfqpoint{2.227177in}{0.839428in}}%
\pgfpathlineto{\pgfqpoint{2.232500in}{0.843792in}}%
\pgfpathlineto{\pgfqpoint{2.237823in}{0.848240in}}%
\pgfpathlineto{\pgfqpoint{2.243146in}{0.852774in}}%
\pgfpathlineto{\pgfqpoint{2.248470in}{0.857392in}}%
\pgfpathlineto{\pgfqpoint{2.253793in}{0.862095in}}%
\pgfpathlineto{\pgfqpoint{2.259116in}{0.866882in}}%
\pgfpathlineto{\pgfqpoint{2.264439in}{0.871755in}}%
\pgfpathlineto{\pgfqpoint{2.269763in}{0.876712in}}%
\pgfpathlineto{\pgfqpoint{2.275086in}{0.881754in}}%
\pgfpathlineto{\pgfqpoint{2.280409in}{0.886880in}}%
\pgfpathlineto{\pgfqpoint{2.285732in}{0.892091in}}%
\pgfpathlineto{\pgfqpoint{2.291056in}{0.897387in}}%
\pgfpathlineto{\pgfqpoint{2.296379in}{0.902768in}}%
\pgfpathlineto{\pgfqpoint{2.301702in}{0.908234in}}%
\pgfpathlineto{\pgfqpoint{2.307025in}{0.913784in}}%
\pgfpathlineto{\pgfqpoint{2.312349in}{0.919419in}}%
\pgfpathlineto{\pgfqpoint{2.317672in}{0.925139in}}%
\pgfpathlineto{\pgfqpoint{2.322995in}{0.930943in}}%
\pgfpathlineto{\pgfqpoint{2.328318in}{0.936832in}}%
\pgfpathlineto{\pgfqpoint{2.333641in}{0.942806in}}%
\pgfpathlineto{\pgfqpoint{2.338965in}{0.948865in}}%
\pgfpathlineto{\pgfqpoint{2.344288in}{0.955008in}}%
\pgfpathlineto{\pgfqpoint{2.349611in}{0.961236in}}%
\pgfpathlineto{\pgfqpoint{2.354934in}{0.967549in}}%
\pgfpathlineto{\pgfqpoint{2.360258in}{0.973947in}}%
\pgfpathlineto{\pgfqpoint{2.365581in}{0.980429in}}%
\pgfpathlineto{\pgfqpoint{2.370904in}{0.986996in}}%
\pgfpathlineto{\pgfqpoint{2.376227in}{0.993648in}}%
\pgfpathlineto{\pgfqpoint{2.381551in}{1.000384in}}%
\pgfpathlineto{\pgfqpoint{2.386874in}{1.007206in}}%
\pgfpathlineto{\pgfqpoint{2.392197in}{1.014112in}}%
\pgfpathlineto{\pgfqpoint{2.397520in}{1.021102in}}%
\pgfpathlineto{\pgfqpoint{2.402843in}{1.028178in}}%
\pgfpathlineto{\pgfqpoint{2.408167in}{1.035338in}}%
\pgfpathlineto{\pgfqpoint{2.413490in}{1.042583in}}%
\pgfpathlineto{\pgfqpoint{2.418813in}{1.049913in}}%
\pgfpathlineto{\pgfqpoint{2.424136in}{1.057327in}}%
\pgfpathlineto{\pgfqpoint{2.429460in}{1.064826in}}%
\pgfpathlineto{\pgfqpoint{2.434783in}{1.072410in}}%
\pgfpathlineto{\pgfqpoint{2.440106in}{1.080079in}}%
\pgfpathlineto{\pgfqpoint{2.445429in}{1.087832in}}%
\pgfpathlineto{\pgfqpoint{2.450753in}{1.095670in}}%
\pgfpathlineto{\pgfqpoint{2.456076in}{1.103593in}}%
\pgfpathlineto{\pgfqpoint{2.461399in}{1.111601in}}%
\pgfpathlineto{\pgfqpoint{2.466722in}{1.119693in}}%
\pgfpathlineto{\pgfqpoint{2.472045in}{1.127870in}}%
\pgfpathlineto{\pgfqpoint{2.477369in}{1.136132in}}%
\pgfpathlineto{\pgfqpoint{2.482692in}{1.144478in}}%
\pgfusepath{stroke}%
\end{pgfscope}%
\begin{pgfscope}%
\pgfpathrectangle{\pgfqpoint{0.374692in}{0.521603in}}{\pgfqpoint{2.635000in}{1.661000in}} %
\pgfusepath{clip}%
\pgfsetbuttcap%
\pgfsetroundjoin%
\pgfsetlinewidth{1.505625pt}%
\definecolor{currentstroke}{rgb}{0.000000,0.000000,0.000000}%
\pgfsetstrokecolor{currentstroke}%
\pgfsetdash{{5.550000pt}{2.400000pt}}{0.000000pt}%
\pgfpathmoveto{\pgfqpoint{1.955692in}{0.729228in}}%
\pgfpathlineto{\pgfqpoint{1.955692in}{0.798437in}}%
\pgfpathlineto{\pgfqpoint{1.955692in}{0.867645in}}%
\pgfpathlineto{\pgfqpoint{1.955692in}{0.936853in}}%
\pgfpathlineto{\pgfqpoint{1.955692in}{1.006062in}}%
\pgfpathlineto{\pgfqpoint{1.955692in}{1.075270in}}%
\pgfpathlineto{\pgfqpoint{1.955692in}{1.144478in}}%
\pgfpathlineto{\pgfqpoint{1.955692in}{1.213687in}}%
\pgfpathlineto{\pgfqpoint{1.955692in}{1.282895in}}%
\pgfpathlineto{\pgfqpoint{1.955692in}{1.352103in}}%
\pgfusepath{stroke}%
\end{pgfscope}%
\begin{pgfscope}%
\pgfpathrectangle{\pgfqpoint{0.374692in}{0.521603in}}{\pgfqpoint{2.635000in}{1.661000in}} %
\pgfusepath{clip}%
\pgfsetrectcap%
\pgfsetroundjoin%
\pgfsetlinewidth{1.003750pt}%
\definecolor{currentstroke}{rgb}{0.580392,0.403922,0.741176}%
\pgfsetstrokecolor{currentstroke}%
\pgfsetdash{}{0pt}%
\pgfpathmoveto{\pgfqpoint{2.482692in}{1.144478in}}%
\pgfpathlineto{\pgfqpoint{2.488015in}{1.136132in}}%
\pgfpathlineto{\pgfqpoint{2.493338in}{1.127870in}}%
\pgfpathlineto{\pgfqpoint{2.498662in}{1.119693in}}%
\pgfpathlineto{\pgfqpoint{2.503985in}{1.111601in}}%
\pgfpathlineto{\pgfqpoint{2.509308in}{1.103593in}}%
\pgfpathlineto{\pgfqpoint{2.514631in}{1.095670in}}%
\pgfpathlineto{\pgfqpoint{2.519955in}{1.087832in}}%
\pgfpathlineto{\pgfqpoint{2.525278in}{1.080079in}}%
\pgfpathlineto{\pgfqpoint{2.530601in}{1.072410in}}%
\pgfpathlineto{\pgfqpoint{2.535924in}{1.064826in}}%
\pgfpathlineto{\pgfqpoint{2.541248in}{1.057327in}}%
\pgfpathlineto{\pgfqpoint{2.546571in}{1.049913in}}%
\pgfpathlineto{\pgfqpoint{2.551894in}{1.042583in}}%
\pgfpathlineto{\pgfqpoint{2.557217in}{1.035338in}}%
\pgfpathlineto{\pgfqpoint{2.562540in}{1.028178in}}%
\pgfpathlineto{\pgfqpoint{2.567864in}{1.021102in}}%
\pgfpathlineto{\pgfqpoint{2.573187in}{1.014112in}}%
\pgfpathlineto{\pgfqpoint{2.578510in}{1.007206in}}%
\pgfpathlineto{\pgfqpoint{2.583833in}{1.000384in}}%
\pgfpathlineto{\pgfqpoint{2.589157in}{0.993648in}}%
\pgfpathlineto{\pgfqpoint{2.594480in}{0.986996in}}%
\pgfpathlineto{\pgfqpoint{2.599803in}{0.980429in}}%
\pgfpathlineto{\pgfqpoint{2.605126in}{0.973947in}}%
\pgfpathlineto{\pgfqpoint{2.610450in}{0.967549in}}%
\pgfpathlineto{\pgfqpoint{2.615773in}{0.961236in}}%
\pgfpathlineto{\pgfqpoint{2.621096in}{0.955008in}}%
\pgfpathlineto{\pgfqpoint{2.626419in}{0.948865in}}%
\pgfpathlineto{\pgfqpoint{2.631742in}{0.942806in}}%
\pgfpathlineto{\pgfqpoint{2.637066in}{0.936832in}}%
\pgfpathlineto{\pgfqpoint{2.642389in}{0.930943in}}%
\pgfpathlineto{\pgfqpoint{2.647712in}{0.925139in}}%
\pgfpathlineto{\pgfqpoint{2.653035in}{0.919419in}}%
\pgfpathlineto{\pgfqpoint{2.658359in}{0.913784in}}%
\pgfpathlineto{\pgfqpoint{2.663682in}{0.908234in}}%
\pgfpathlineto{\pgfqpoint{2.669005in}{0.902768in}}%
\pgfpathlineto{\pgfqpoint{2.674328in}{0.897387in}}%
\pgfpathlineto{\pgfqpoint{2.679652in}{0.892091in}}%
\pgfpathlineto{\pgfqpoint{2.684975in}{0.886880in}}%
\pgfpathlineto{\pgfqpoint{2.690298in}{0.881754in}}%
\pgfpathlineto{\pgfqpoint{2.695621in}{0.876712in}}%
\pgfpathlineto{\pgfqpoint{2.700944in}{0.871755in}}%
\pgfpathlineto{\pgfqpoint{2.706268in}{0.866882in}}%
\pgfpathlineto{\pgfqpoint{2.711591in}{0.862095in}}%
\pgfpathlineto{\pgfqpoint{2.716914in}{0.857392in}}%
\pgfpathlineto{\pgfqpoint{2.722237in}{0.852774in}}%
\pgfpathlineto{\pgfqpoint{2.727561in}{0.848240in}}%
\pgfpathlineto{\pgfqpoint{2.732884in}{0.843792in}}%
\pgfpathlineto{\pgfqpoint{2.738207in}{0.839428in}}%
\pgfpathlineto{\pgfqpoint{2.743530in}{0.835149in}}%
\pgfpathlineto{\pgfqpoint{2.748854in}{0.830954in}}%
\pgfpathlineto{\pgfqpoint{2.754177in}{0.826844in}}%
\pgfpathlineto{\pgfqpoint{2.759500in}{0.822820in}}%
\pgfpathlineto{\pgfqpoint{2.764823in}{0.818879in}}%
\pgfpathlineto{\pgfqpoint{2.770146in}{0.815024in}}%
\pgfpathlineto{\pgfqpoint{2.775470in}{0.811253in}}%
\pgfpathlineto{\pgfqpoint{2.780793in}{0.807567in}}%
\pgfpathlineto{\pgfqpoint{2.786116in}{0.803966in}}%
\pgfpathlineto{\pgfqpoint{2.791439in}{0.800449in}}%
\pgfpathlineto{\pgfqpoint{2.796763in}{0.797017in}}%
\pgfpathlineto{\pgfqpoint{2.802086in}{0.793670in}}%
\pgfpathlineto{\pgfqpoint{2.807409in}{0.790408in}}%
\pgfpathlineto{\pgfqpoint{2.812732in}{0.787230in}}%
\pgfpathlineto{\pgfqpoint{2.818056in}{0.784137in}}%
\pgfpathlineto{\pgfqpoint{2.823379in}{0.781129in}}%
\pgfpathlineto{\pgfqpoint{2.828702in}{0.778206in}}%
\pgfpathlineto{\pgfqpoint{2.834025in}{0.775367in}}%
\pgfpathlineto{\pgfqpoint{2.839349in}{0.772613in}}%
\pgfpathlineto{\pgfqpoint{2.844672in}{0.769944in}}%
\pgfpathlineto{\pgfqpoint{2.849995in}{0.767360in}}%
\pgfpathlineto{\pgfqpoint{2.855318in}{0.764860in}}%
\pgfpathlineto{\pgfqpoint{2.860641in}{0.762445in}}%
\pgfpathlineto{\pgfqpoint{2.865965in}{0.760115in}}%
\pgfpathlineto{\pgfqpoint{2.871288in}{0.757869in}}%
\pgfpathlineto{\pgfqpoint{2.876611in}{0.755708in}}%
\pgfpathlineto{\pgfqpoint{2.881934in}{0.753632in}}%
\pgfpathlineto{\pgfqpoint{2.887258in}{0.751641in}}%
\pgfpathlineto{\pgfqpoint{2.892581in}{0.749735in}}%
\pgfpathlineto{\pgfqpoint{2.897904in}{0.747913in}}%
\pgfpathlineto{\pgfqpoint{2.903227in}{0.746176in}}%
\pgfpathlineto{\pgfqpoint{2.908551in}{0.744523in}}%
\pgfpathlineto{\pgfqpoint{2.913874in}{0.742956in}}%
\pgfpathlineto{\pgfqpoint{2.919197in}{0.741473in}}%
\pgfpathlineto{\pgfqpoint{2.924520in}{0.740075in}}%
\pgfpathlineto{\pgfqpoint{2.929843in}{0.738761in}}%
\pgfpathlineto{\pgfqpoint{2.935167in}{0.737532in}}%
\pgfpathlineto{\pgfqpoint{2.940490in}{0.736389in}}%
\pgfpathlineto{\pgfqpoint{2.945813in}{0.735329in}}%
\pgfpathlineto{\pgfqpoint{2.951136in}{0.734355in}}%
\pgfpathlineto{\pgfqpoint{2.956460in}{0.733465in}}%
\pgfpathlineto{\pgfqpoint{2.961783in}{0.732660in}}%
\pgfpathlineto{\pgfqpoint{2.967106in}{0.731940in}}%
\pgfpathlineto{\pgfqpoint{2.972429in}{0.731304in}}%
\pgfpathlineto{\pgfqpoint{2.977753in}{0.730754in}}%
\pgfpathlineto{\pgfqpoint{2.983076in}{0.730288in}}%
\pgfpathlineto{\pgfqpoint{2.988399in}{0.729906in}}%
\pgfpathlineto{\pgfqpoint{2.993722in}{0.729610in}}%
\pgfpathlineto{\pgfqpoint{2.999045in}{0.729398in}}%
\pgfpathlineto{\pgfqpoint{3.004369in}{0.729271in}}%
\pgfpathlineto{\pgfqpoint{3.009692in}{0.729228in}}%
\pgfusepath{stroke}%
\end{pgfscope}%
\begin{pgfscope}%
\pgfpathrectangle{\pgfqpoint{0.374692in}{0.521603in}}{\pgfqpoint{2.635000in}{1.661000in}} %
\pgfusepath{clip}%
\pgfsetrectcap%
\pgfsetroundjoin%
\pgfsetlinewidth{1.003750pt}%
\definecolor{currentstroke}{rgb}{0.121569,0.466667,0.705882}%
\pgfsetstrokecolor{currentstroke}%
\pgfsetdash{}{0pt}%
\pgfpathmoveto{\pgfqpoint{2.482692in}{1.144478in}}%
\pgfpathlineto{\pgfqpoint{2.488015in}{1.152782in}}%
\pgfpathlineto{\pgfqpoint{2.493338in}{1.160917in}}%
\pgfpathlineto{\pgfqpoint{2.498662in}{1.168882in}}%
\pgfpathlineto{\pgfqpoint{2.503985in}{1.176678in}}%
\pgfpathlineto{\pgfqpoint{2.509308in}{1.184304in}}%
\pgfpathlineto{\pgfqpoint{2.514631in}{1.191761in}}%
\pgfpathlineto{\pgfqpoint{2.519955in}{1.199048in}}%
\pgfpathlineto{\pgfqpoint{2.525278in}{1.206166in}}%
\pgfpathlineto{\pgfqpoint{2.530601in}{1.213115in}}%
\pgfpathlineto{\pgfqpoint{2.535924in}{1.219894in}}%
\pgfpathlineto{\pgfqpoint{2.541248in}{1.226503in}}%
\pgfpathlineto{\pgfqpoint{2.546571in}{1.232943in}}%
\pgfpathlineto{\pgfqpoint{2.551894in}{1.239213in}}%
\pgfpathlineto{\pgfqpoint{2.557217in}{1.245314in}}%
\pgfpathlineto{\pgfqpoint{2.562540in}{1.251246in}}%
\pgfpathlineto{\pgfqpoint{2.567864in}{1.257008in}}%
\pgfpathlineto{\pgfqpoint{2.573187in}{1.262601in}}%
\pgfpathlineto{\pgfqpoint{2.578510in}{1.268024in}}%
\pgfpathlineto{\pgfqpoint{2.583833in}{1.273277in}}%
\pgfpathlineto{\pgfqpoint{2.589157in}{1.278362in}}%
\pgfpathlineto{\pgfqpoint{2.594480in}{1.283276in}}%
\pgfpathlineto{\pgfqpoint{2.599803in}{1.288022in}}%
\pgfpathlineto{\pgfqpoint{2.605126in}{1.292597in}}%
\pgfpathlineto{\pgfqpoint{2.610450in}{1.297004in}}%
\pgfpathlineto{\pgfqpoint{2.615773in}{1.301240in}}%
\pgfpathlineto{\pgfqpoint{2.621096in}{1.305308in}}%
\pgfpathlineto{\pgfqpoint{2.626419in}{1.309206in}}%
\pgfpathlineto{\pgfqpoint{2.631742in}{1.312934in}}%
\pgfpathlineto{\pgfqpoint{2.637066in}{1.316493in}}%
\pgfpathlineto{\pgfqpoint{2.642389in}{1.319882in}}%
\pgfpathlineto{\pgfqpoint{2.647712in}{1.323102in}}%
\pgfpathlineto{\pgfqpoint{2.653035in}{1.326153in}}%
\pgfpathlineto{\pgfqpoint{2.658359in}{1.329034in}}%
\pgfpathlineto{\pgfqpoint{2.663682in}{1.331745in}}%
\pgfpathlineto{\pgfqpoint{2.669005in}{1.334288in}}%
\pgfpathlineto{\pgfqpoint{2.674328in}{1.336660in}}%
\pgfpathlineto{\pgfqpoint{2.679652in}{1.338863in}}%
\pgfpathlineto{\pgfqpoint{2.684975in}{1.340897in}}%
\pgfpathlineto{\pgfqpoint{2.690298in}{1.342761in}}%
\pgfpathlineto{\pgfqpoint{2.695621in}{1.344456in}}%
\pgfpathlineto{\pgfqpoint{2.700944in}{1.345981in}}%
\pgfpathlineto{\pgfqpoint{2.706268in}{1.347337in}}%
\pgfpathlineto{\pgfqpoint{2.711591in}{1.348523in}}%
\pgfpathlineto{\pgfqpoint{2.716914in}{1.349540in}}%
\pgfpathlineto{\pgfqpoint{2.722237in}{1.350387in}}%
\pgfpathlineto{\pgfqpoint{2.727561in}{1.351065in}}%
\pgfpathlineto{\pgfqpoint{2.732884in}{1.351574in}}%
\pgfpathlineto{\pgfqpoint{2.738207in}{1.351913in}}%
\pgfpathlineto{\pgfqpoint{2.743530in}{1.352082in}}%
\pgfpathlineto{\pgfqpoint{2.748854in}{1.352082in}}%
\pgfpathlineto{\pgfqpoint{2.754177in}{1.351913in}}%
\pgfpathlineto{\pgfqpoint{2.759500in}{1.351574in}}%
\pgfpathlineto{\pgfqpoint{2.764823in}{1.351065in}}%
\pgfpathlineto{\pgfqpoint{2.770146in}{1.350387in}}%
\pgfpathlineto{\pgfqpoint{2.775470in}{1.349540in}}%
\pgfpathlineto{\pgfqpoint{2.780793in}{1.348523in}}%
\pgfpathlineto{\pgfqpoint{2.786116in}{1.347337in}}%
\pgfpathlineto{\pgfqpoint{2.791439in}{1.345981in}}%
\pgfpathlineto{\pgfqpoint{2.796763in}{1.344456in}}%
\pgfpathlineto{\pgfqpoint{2.802086in}{1.342761in}}%
\pgfpathlineto{\pgfqpoint{2.807409in}{1.340897in}}%
\pgfpathlineto{\pgfqpoint{2.812732in}{1.338863in}}%
\pgfpathlineto{\pgfqpoint{2.818056in}{1.336660in}}%
\pgfpathlineto{\pgfqpoint{2.823379in}{1.334288in}}%
\pgfpathlineto{\pgfqpoint{2.828702in}{1.331745in}}%
\pgfpathlineto{\pgfqpoint{2.834025in}{1.329034in}}%
\pgfpathlineto{\pgfqpoint{2.839349in}{1.326153in}}%
\pgfpathlineto{\pgfqpoint{2.844672in}{1.323102in}}%
\pgfpathlineto{\pgfqpoint{2.849995in}{1.319882in}}%
\pgfpathlineto{\pgfqpoint{2.855318in}{1.316493in}}%
\pgfpathlineto{\pgfqpoint{2.860641in}{1.312934in}}%
\pgfpathlineto{\pgfqpoint{2.865965in}{1.309206in}}%
\pgfpathlineto{\pgfqpoint{2.871288in}{1.305308in}}%
\pgfpathlineto{\pgfqpoint{2.876611in}{1.301240in}}%
\pgfpathlineto{\pgfqpoint{2.881934in}{1.297004in}}%
\pgfpathlineto{\pgfqpoint{2.887258in}{1.292597in}}%
\pgfpathlineto{\pgfqpoint{2.892581in}{1.288022in}}%
\pgfpathlineto{\pgfqpoint{2.897904in}{1.283276in}}%
\pgfpathlineto{\pgfqpoint{2.903227in}{1.278362in}}%
\pgfpathlineto{\pgfqpoint{2.908551in}{1.273277in}}%
\pgfpathlineto{\pgfqpoint{2.913874in}{1.268024in}}%
\pgfpathlineto{\pgfqpoint{2.919197in}{1.262601in}}%
\pgfpathlineto{\pgfqpoint{2.924520in}{1.257008in}}%
\pgfpathlineto{\pgfqpoint{2.929843in}{1.251246in}}%
\pgfpathlineto{\pgfqpoint{2.935167in}{1.245314in}}%
\pgfpathlineto{\pgfqpoint{2.940490in}{1.239213in}}%
\pgfpathlineto{\pgfqpoint{2.945813in}{1.232943in}}%
\pgfpathlineto{\pgfqpoint{2.951136in}{1.226503in}}%
\pgfpathlineto{\pgfqpoint{2.956460in}{1.219894in}}%
\pgfpathlineto{\pgfqpoint{2.961783in}{1.213115in}}%
\pgfpathlineto{\pgfqpoint{2.967106in}{1.206166in}}%
\pgfpathlineto{\pgfqpoint{2.972429in}{1.199048in}}%
\pgfpathlineto{\pgfqpoint{2.977753in}{1.191761in}}%
\pgfpathlineto{\pgfqpoint{2.983076in}{1.184304in}}%
\pgfpathlineto{\pgfqpoint{2.988399in}{1.176678in}}%
\pgfpathlineto{\pgfqpoint{2.993722in}{1.168882in}}%
\pgfpathlineto{\pgfqpoint{2.999045in}{1.160917in}}%
\pgfpathlineto{\pgfqpoint{3.004369in}{1.152782in}}%
\pgfpathlineto{\pgfqpoint{3.009692in}{1.144478in}}%
\pgfusepath{stroke}%
\end{pgfscope}%
\begin{pgfscope}%
\pgfpathrectangle{\pgfqpoint{0.374692in}{0.521603in}}{\pgfqpoint{2.635000in}{1.661000in}} %
\pgfusepath{clip}%
\pgfsetrectcap%
\pgfsetroundjoin%
\pgfsetlinewidth{1.003750pt}%
\definecolor{currentstroke}{rgb}{1.000000,0.498039,0.054902}%
\pgfsetstrokecolor{currentstroke}%
\pgfsetdash{}{0pt}%
\pgfpathmoveto{\pgfqpoint{2.482692in}{0.729228in}}%
\pgfpathlineto{\pgfqpoint{2.488015in}{0.729271in}}%
\pgfpathlineto{\pgfqpoint{2.493338in}{0.729398in}}%
\pgfpathlineto{\pgfqpoint{2.498662in}{0.729610in}}%
\pgfpathlineto{\pgfqpoint{2.503985in}{0.729906in}}%
\pgfpathlineto{\pgfqpoint{2.509308in}{0.730288in}}%
\pgfpathlineto{\pgfqpoint{2.514631in}{0.730754in}}%
\pgfpathlineto{\pgfqpoint{2.519955in}{0.731304in}}%
\pgfpathlineto{\pgfqpoint{2.525278in}{0.731940in}}%
\pgfpathlineto{\pgfqpoint{2.530601in}{0.732660in}}%
\pgfpathlineto{\pgfqpoint{2.535924in}{0.733465in}}%
\pgfpathlineto{\pgfqpoint{2.541248in}{0.734355in}}%
\pgfpathlineto{\pgfqpoint{2.546571in}{0.735329in}}%
\pgfpathlineto{\pgfqpoint{2.551894in}{0.736389in}}%
\pgfpathlineto{\pgfqpoint{2.557217in}{0.737532in}}%
\pgfpathlineto{\pgfqpoint{2.562540in}{0.738761in}}%
\pgfpathlineto{\pgfqpoint{2.567864in}{0.740075in}}%
\pgfpathlineto{\pgfqpoint{2.573187in}{0.741473in}}%
\pgfpathlineto{\pgfqpoint{2.578510in}{0.742956in}}%
\pgfpathlineto{\pgfqpoint{2.583833in}{0.744523in}}%
\pgfpathlineto{\pgfqpoint{2.589157in}{0.746176in}}%
\pgfpathlineto{\pgfqpoint{2.594480in}{0.747913in}}%
\pgfpathlineto{\pgfqpoint{2.599803in}{0.749735in}}%
\pgfpathlineto{\pgfqpoint{2.605126in}{0.751641in}}%
\pgfpathlineto{\pgfqpoint{2.610450in}{0.753632in}}%
\pgfpathlineto{\pgfqpoint{2.615773in}{0.755708in}}%
\pgfpathlineto{\pgfqpoint{2.621096in}{0.757869in}}%
\pgfpathlineto{\pgfqpoint{2.626419in}{0.760115in}}%
\pgfpathlineto{\pgfqpoint{2.631742in}{0.762445in}}%
\pgfpathlineto{\pgfqpoint{2.637066in}{0.764860in}}%
\pgfpathlineto{\pgfqpoint{2.642389in}{0.767360in}}%
\pgfpathlineto{\pgfqpoint{2.647712in}{0.769944in}}%
\pgfpathlineto{\pgfqpoint{2.653035in}{0.772613in}}%
\pgfpathlineto{\pgfqpoint{2.658359in}{0.775367in}}%
\pgfpathlineto{\pgfqpoint{2.663682in}{0.778206in}}%
\pgfpathlineto{\pgfqpoint{2.669005in}{0.781129in}}%
\pgfpathlineto{\pgfqpoint{2.674328in}{0.784137in}}%
\pgfpathlineto{\pgfqpoint{2.679652in}{0.787230in}}%
\pgfpathlineto{\pgfqpoint{2.684975in}{0.790408in}}%
\pgfpathlineto{\pgfqpoint{2.690298in}{0.793670in}}%
\pgfpathlineto{\pgfqpoint{2.695621in}{0.797017in}}%
\pgfpathlineto{\pgfqpoint{2.700944in}{0.800449in}}%
\pgfpathlineto{\pgfqpoint{2.706268in}{0.803966in}}%
\pgfpathlineto{\pgfqpoint{2.711591in}{0.807567in}}%
\pgfpathlineto{\pgfqpoint{2.716914in}{0.811253in}}%
\pgfpathlineto{\pgfqpoint{2.722237in}{0.815024in}}%
\pgfpathlineto{\pgfqpoint{2.727561in}{0.818879in}}%
\pgfpathlineto{\pgfqpoint{2.732884in}{0.822820in}}%
\pgfpathlineto{\pgfqpoint{2.738207in}{0.826844in}}%
\pgfpathlineto{\pgfqpoint{2.743530in}{0.830954in}}%
\pgfpathlineto{\pgfqpoint{2.748854in}{0.835149in}}%
\pgfpathlineto{\pgfqpoint{2.754177in}{0.839428in}}%
\pgfpathlineto{\pgfqpoint{2.759500in}{0.843792in}}%
\pgfpathlineto{\pgfqpoint{2.764823in}{0.848240in}}%
\pgfpathlineto{\pgfqpoint{2.770146in}{0.852774in}}%
\pgfpathlineto{\pgfqpoint{2.775470in}{0.857392in}}%
\pgfpathlineto{\pgfqpoint{2.780793in}{0.862095in}}%
\pgfpathlineto{\pgfqpoint{2.786116in}{0.866882in}}%
\pgfpathlineto{\pgfqpoint{2.791439in}{0.871755in}}%
\pgfpathlineto{\pgfqpoint{2.796763in}{0.876712in}}%
\pgfpathlineto{\pgfqpoint{2.802086in}{0.881754in}}%
\pgfpathlineto{\pgfqpoint{2.807409in}{0.886880in}}%
\pgfpathlineto{\pgfqpoint{2.812732in}{0.892091in}}%
\pgfpathlineto{\pgfqpoint{2.818056in}{0.897387in}}%
\pgfpathlineto{\pgfqpoint{2.823379in}{0.902768in}}%
\pgfpathlineto{\pgfqpoint{2.828702in}{0.908234in}}%
\pgfpathlineto{\pgfqpoint{2.834025in}{0.913784in}}%
\pgfpathlineto{\pgfqpoint{2.839349in}{0.919419in}}%
\pgfpathlineto{\pgfqpoint{2.844672in}{0.925139in}}%
\pgfpathlineto{\pgfqpoint{2.849995in}{0.930943in}}%
\pgfpathlineto{\pgfqpoint{2.855318in}{0.936832in}}%
\pgfpathlineto{\pgfqpoint{2.860641in}{0.942806in}}%
\pgfpathlineto{\pgfqpoint{2.865965in}{0.948865in}}%
\pgfpathlineto{\pgfqpoint{2.871288in}{0.955008in}}%
\pgfpathlineto{\pgfqpoint{2.876611in}{0.961236in}}%
\pgfpathlineto{\pgfqpoint{2.881934in}{0.967549in}}%
\pgfpathlineto{\pgfqpoint{2.887258in}{0.973947in}}%
\pgfpathlineto{\pgfqpoint{2.892581in}{0.980429in}}%
\pgfpathlineto{\pgfqpoint{2.897904in}{0.986996in}}%
\pgfpathlineto{\pgfqpoint{2.903227in}{0.993648in}}%
\pgfpathlineto{\pgfqpoint{2.908551in}{1.000384in}}%
\pgfpathlineto{\pgfqpoint{2.913874in}{1.007206in}}%
\pgfpathlineto{\pgfqpoint{2.919197in}{1.014112in}}%
\pgfpathlineto{\pgfqpoint{2.924520in}{1.021102in}}%
\pgfpathlineto{\pgfqpoint{2.929843in}{1.028178in}}%
\pgfpathlineto{\pgfqpoint{2.935167in}{1.035338in}}%
\pgfpathlineto{\pgfqpoint{2.940490in}{1.042583in}}%
\pgfpathlineto{\pgfqpoint{2.945813in}{1.049913in}}%
\pgfpathlineto{\pgfqpoint{2.951136in}{1.057327in}}%
\pgfpathlineto{\pgfqpoint{2.956460in}{1.064826in}}%
\pgfpathlineto{\pgfqpoint{2.961783in}{1.072410in}}%
\pgfpathlineto{\pgfqpoint{2.967106in}{1.080079in}}%
\pgfpathlineto{\pgfqpoint{2.972429in}{1.087832in}}%
\pgfpathlineto{\pgfqpoint{2.977753in}{1.095670in}}%
\pgfpathlineto{\pgfqpoint{2.983076in}{1.103593in}}%
\pgfpathlineto{\pgfqpoint{2.988399in}{1.111601in}}%
\pgfpathlineto{\pgfqpoint{2.993722in}{1.119693in}}%
\pgfpathlineto{\pgfqpoint{2.999045in}{1.127870in}}%
\pgfpathlineto{\pgfqpoint{3.004369in}{1.136132in}}%
\pgfpathlineto{\pgfqpoint{3.009692in}{1.144478in}}%
\pgfusepath{stroke}%
\end{pgfscope}%
\begin{pgfscope}%
\pgfpathrectangle{\pgfqpoint{0.374692in}{0.521603in}}{\pgfqpoint{2.635000in}{1.661000in}} %
\pgfusepath{clip}%
\pgfsetbuttcap%
\pgfsetroundjoin%
\pgfsetlinewidth{1.505625pt}%
\definecolor{currentstroke}{rgb}{0.000000,0.000000,0.000000}%
\pgfsetstrokecolor{currentstroke}%
\pgfsetdash{{5.550000pt}{2.400000pt}}{0.000000pt}%
\pgfpathmoveto{\pgfqpoint{2.482692in}{0.729228in}}%
\pgfpathlineto{\pgfqpoint{2.482692in}{0.798437in}}%
\pgfpathlineto{\pgfqpoint{2.482692in}{0.867645in}}%
\pgfpathlineto{\pgfqpoint{2.482692in}{0.936853in}}%
\pgfpathlineto{\pgfqpoint{2.482692in}{1.006062in}}%
\pgfpathlineto{\pgfqpoint{2.482692in}{1.075270in}}%
\pgfpathlineto{\pgfqpoint{2.482692in}{1.144478in}}%
\pgfpathlineto{\pgfqpoint{2.482692in}{1.213687in}}%
\pgfpathlineto{\pgfqpoint{2.482692in}{1.282895in}}%
\pgfpathlineto{\pgfqpoint{2.482692in}{1.352103in}}%
\pgfusepath{stroke}%
\end{pgfscope}%
\begin{pgfscope}%
\pgfpathrectangle{\pgfqpoint{0.374692in}{0.521603in}}{\pgfqpoint{2.635000in}{1.661000in}} %
\pgfusepath{clip}%
\pgfsetbuttcap%
\pgfsetroundjoin%
\definecolor{currentfill}{rgb}{1.000000,0.000000,0.000000}%
\pgfsetfillcolor{currentfill}%
\pgfsetlinewidth{1.003750pt}%
\definecolor{currentstroke}{rgb}{1.000000,0.000000,0.000000}%
\pgfsetstrokecolor{currentstroke}%
\pgfsetdash{}{0pt}%
\pgfsys@defobject{currentmarker}{\pgfqpoint{-0.020833in}{-0.020833in}}{\pgfqpoint{0.020833in}{0.020833in}}{%
\pgfpathmoveto{\pgfqpoint{0.000000in}{-0.020833in}}%
\pgfpathcurveto{\pgfqpoint{0.005525in}{-0.020833in}}{\pgfqpoint{0.010825in}{-0.018638in}}{\pgfqpoint{0.014731in}{-0.014731in}}%
\pgfpathcurveto{\pgfqpoint{0.018638in}{-0.010825in}}{\pgfqpoint{0.020833in}{-0.005525in}}{\pgfqpoint{0.020833in}{0.000000in}}%
\pgfpathcurveto{\pgfqpoint{0.020833in}{0.005525in}}{\pgfqpoint{0.018638in}{0.010825in}}{\pgfqpoint{0.014731in}{0.014731in}}%
\pgfpathcurveto{\pgfqpoint{0.010825in}{0.018638in}}{\pgfqpoint{0.005525in}{0.020833in}}{\pgfqpoint{0.000000in}{0.020833in}}%
\pgfpathcurveto{\pgfqpoint{-0.005525in}{0.020833in}}{\pgfqpoint{-0.010825in}{0.018638in}}{\pgfqpoint{-0.014731in}{0.014731in}}%
\pgfpathcurveto{\pgfqpoint{-0.018638in}{0.010825in}}{\pgfqpoint{-0.020833in}{0.005525in}}{\pgfqpoint{-0.020833in}{0.000000in}}%
\pgfpathcurveto{\pgfqpoint{-0.020833in}{-0.005525in}}{\pgfqpoint{-0.018638in}{-0.010825in}}{\pgfqpoint{-0.014731in}{-0.014731in}}%
\pgfpathcurveto{\pgfqpoint{-0.010825in}{-0.018638in}}{\pgfqpoint{-0.005525in}{-0.020833in}}{\pgfqpoint{0.000000in}{-0.020833in}}%
\pgfpathclose%
\pgfusepath{stroke,fill}%
}%
\begin{pgfscope}%
\pgfsys@transformshift{0.434086in}{0.729228in}%
\pgfsys@useobject{currentmarker}{}%
\end{pgfscope}%
\begin{pgfscope}%
\pgfsys@transformshift{0.638192in}{0.729228in}%
\pgfsys@useobject{currentmarker}{}%
\end{pgfscope}%
\begin{pgfscope}%
\pgfsys@transformshift{0.842298in}{0.729228in}%
\pgfsys@useobject{currentmarker}{}%
\end{pgfscope}%
\begin{pgfscope}%
\pgfsys@transformshift{0.961086in}{0.729228in}%
\pgfsys@useobject{currentmarker}{}%
\end{pgfscope}%
\begin{pgfscope}%
\pgfsys@transformshift{1.165192in}{0.729228in}%
\pgfsys@useobject{currentmarker}{}%
\end{pgfscope}%
\begin{pgfscope}%
\pgfsys@transformshift{1.369298in}{0.729228in}%
\pgfsys@useobject{currentmarker}{}%
\end{pgfscope}%
\begin{pgfscope}%
\pgfsys@transformshift{1.488086in}{0.729228in}%
\pgfsys@useobject{currentmarker}{}%
\end{pgfscope}%
\begin{pgfscope}%
\pgfsys@transformshift{1.692192in}{0.729228in}%
\pgfsys@useobject{currentmarker}{}%
\end{pgfscope}%
\begin{pgfscope}%
\pgfsys@transformshift{1.896298in}{0.729228in}%
\pgfsys@useobject{currentmarker}{}%
\end{pgfscope}%
\begin{pgfscope}%
\pgfsys@transformshift{2.015086in}{0.729228in}%
\pgfsys@useobject{currentmarker}{}%
\end{pgfscope}%
\begin{pgfscope}%
\pgfsys@transformshift{2.219192in}{0.729228in}%
\pgfsys@useobject{currentmarker}{}%
\end{pgfscope}%
\begin{pgfscope}%
\pgfsys@transformshift{2.423298in}{0.729228in}%
\pgfsys@useobject{currentmarker}{}%
\end{pgfscope}%
\begin{pgfscope}%
\pgfsys@transformshift{2.542086in}{0.729228in}%
\pgfsys@useobject{currentmarker}{}%
\end{pgfscope}%
\begin{pgfscope}%
\pgfsys@transformshift{2.746192in}{0.729228in}%
\pgfsys@useobject{currentmarker}{}%
\end{pgfscope}%
\begin{pgfscope}%
\pgfsys@transformshift{2.950298in}{0.729228in}%
\pgfsys@useobject{currentmarker}{}%
\end{pgfscope}%
\end{pgfscope}%
\begin{pgfscope}%
\pgfsetrectcap%
\pgfsetmiterjoin%
\pgfsetlinewidth{0.803000pt}%
\definecolor{currentstroke}{rgb}{0.000000,0.000000,0.000000}%
\pgfsetstrokecolor{currentstroke}%
\pgfsetdash{}{0pt}%
\pgfpathmoveto{\pgfqpoint{0.374692in}{0.521603in}}%
\pgfpathlineto{\pgfqpoint{0.374692in}{2.182603in}}%
\pgfusepath{stroke}%
\end{pgfscope}%
\begin{pgfscope}%
\pgfsetrectcap%
\pgfsetmiterjoin%
\pgfsetlinewidth{0.803000pt}%
\definecolor{currentstroke}{rgb}{0.000000,0.000000,0.000000}%
\pgfsetstrokecolor{currentstroke}%
\pgfsetdash{}{0pt}%
\pgfpathmoveto{\pgfqpoint{3.009692in}{0.521603in}}%
\pgfpathlineto{\pgfqpoint{3.009692in}{2.182603in}}%
\pgfusepath{stroke}%
\end{pgfscope}%
\begin{pgfscope}%
\pgfsetrectcap%
\pgfsetmiterjoin%
\pgfsetlinewidth{0.803000pt}%
\definecolor{currentstroke}{rgb}{0.000000,0.000000,0.000000}%
\pgfsetstrokecolor{currentstroke}%
\pgfsetdash{}{0pt}%
\pgfpathmoveto{\pgfqpoint{0.374692in}{0.521603in}}%
\pgfpathlineto{\pgfqpoint{3.009692in}{0.521603in}}%
\pgfusepath{stroke}%
\end{pgfscope}%
\begin{pgfscope}%
\pgfsetrectcap%
\pgfsetmiterjoin%
\pgfsetlinewidth{0.803000pt}%
\definecolor{currentstroke}{rgb}{0.000000,0.000000,0.000000}%
\pgfsetstrokecolor{currentstroke}%
\pgfsetdash{}{0pt}%
\pgfpathmoveto{\pgfqpoint{0.374692in}{2.182603in}}%
\pgfpathlineto{\pgfqpoint{3.009692in}{2.182603in}}%
\pgfusepath{stroke}%
\end{pgfscope}%
\begin{pgfscope}%
\pgfsetbuttcap%
\pgfsetmiterjoin%
\definecolor{currentfill}{rgb}{1.000000,1.000000,1.000000}%
\pgfsetfillcolor{currentfill}%
\pgfsetfillopacity{0.800000}%
\pgfsetlinewidth{1.003750pt}%
\definecolor{currentstroke}{rgb}{0.800000,0.800000,0.800000}%
\pgfsetstrokecolor{currentstroke}%
\pgfsetstrokeopacity{0.800000}%
\pgfsetdash{}{0pt}%
\pgfpathmoveto{\pgfqpoint{0.845236in}{1.661811in}}%
\pgfpathlineto{\pgfqpoint{2.912470in}{1.661811in}}%
\pgfpathquadraticcurveto{\pgfqpoint{2.940247in}{1.661811in}}{\pgfqpoint{2.940247in}{1.689589in}}%
\pgfpathlineto{\pgfqpoint{2.940247in}{2.085381in}}%
\pgfpathquadraticcurveto{\pgfqpoint{2.940247in}{2.113159in}}{\pgfqpoint{2.912470in}{2.113159in}}%
\pgfpathlineto{\pgfqpoint{0.845236in}{2.113159in}}%
\pgfpathquadraticcurveto{\pgfqpoint{0.817458in}{2.113159in}}{\pgfqpoint{0.817458in}{2.085381in}}%
\pgfpathlineto{\pgfqpoint{0.817458in}{1.689589in}}%
\pgfpathquadraticcurveto{\pgfqpoint{0.817458in}{1.661811in}}{\pgfqpoint{0.845236in}{1.661811in}}%
\pgfpathclose%
\pgfusepath{stroke,fill}%
\end{pgfscope}%
\begin{pgfscope}%
\pgfsetbuttcap%
\pgfsetroundjoin%
\pgfsetlinewidth{1.505625pt}%
\definecolor{currentstroke}{rgb}{0.000000,0.000000,0.000000}%
\pgfsetstrokecolor{currentstroke}%
\pgfsetdash{{5.550000pt}{2.400000pt}}{0.000000pt}%
\pgfpathmoveto{\pgfqpoint{0.873014in}{2.000691in}}%
\pgfpathlineto{\pgfqpoint{1.150792in}{2.000691in}}%
\pgfusepath{stroke}%
\end{pgfscope}%
\begin{pgfscope}%
\pgftext[x=1.261903in,y=1.952080in,left,base]{\rmfamily\fontsize{10.000000}{12.000000}\selectfont element boundaries}%
\end{pgfscope}%
\begin{pgfscope}%
\pgfsetbuttcap%
\pgfsetroundjoin%
\definecolor{currentfill}{rgb}{1.000000,0.000000,0.000000}%
\pgfsetfillcolor{currentfill}%
\pgfsetlinewidth{1.003750pt}%
\definecolor{currentstroke}{rgb}{1.000000,0.000000,0.000000}%
\pgfsetstrokecolor{currentstroke}%
\pgfsetdash{}{0pt}%
\pgfsys@defobject{currentmarker}{\pgfqpoint{-0.020833in}{-0.020833in}}{\pgfqpoint{0.020833in}{0.020833in}}{%
\pgfpathmoveto{\pgfqpoint{0.000000in}{-0.020833in}}%
\pgfpathcurveto{\pgfqpoint{0.005525in}{-0.020833in}}{\pgfqpoint{0.010825in}{-0.018638in}}{\pgfqpoint{0.014731in}{-0.014731in}}%
\pgfpathcurveto{\pgfqpoint{0.018638in}{-0.010825in}}{\pgfqpoint{0.020833in}{-0.005525in}}{\pgfqpoint{0.020833in}{0.000000in}}%
\pgfpathcurveto{\pgfqpoint{0.020833in}{0.005525in}}{\pgfqpoint{0.018638in}{0.010825in}}{\pgfqpoint{0.014731in}{0.014731in}}%
\pgfpathcurveto{\pgfqpoint{0.010825in}{0.018638in}}{\pgfqpoint{0.005525in}{0.020833in}}{\pgfqpoint{0.000000in}{0.020833in}}%
\pgfpathcurveto{\pgfqpoint{-0.005525in}{0.020833in}}{\pgfqpoint{-0.010825in}{0.018638in}}{\pgfqpoint{-0.014731in}{0.014731in}}%
\pgfpathcurveto{\pgfqpoint{-0.018638in}{0.010825in}}{\pgfqpoint{-0.020833in}{0.005525in}}{\pgfqpoint{-0.020833in}{0.000000in}}%
\pgfpathcurveto{\pgfqpoint{-0.020833in}{-0.005525in}}{\pgfqpoint{-0.018638in}{-0.010825in}}{\pgfqpoint{-0.014731in}{-0.014731in}}%
\pgfpathcurveto{\pgfqpoint{-0.010825in}{-0.018638in}}{\pgfqpoint{-0.005525in}{-0.020833in}}{\pgfqpoint{0.000000in}{-0.020833in}}%
\pgfpathclose%
\pgfusepath{stroke,fill}%
}%
\begin{pgfscope}%
\pgfsys@transformshift{1.011903in}{1.796834in}%
\pgfsys@useobject{currentmarker}{}%
\end{pgfscope}%
\end{pgfscope}%
\begin{pgfscope}%
\pgftext[x=1.261903in,y=1.748223in,left,base]{\rmfamily\fontsize{10.000000}{12.000000}\selectfont Gauss-Legendre points}%
\end{pgfscope}%
\end{pgfpicture}%
\makeatother%
\endgroup%

\caption{(a) Example for a periodic B-spline basis of degree $p=1$ on a domain of length $L=1$ discretized by $N_\mr{el}=5$ elements and the corresponding Gauss-Legendre quadrature points. In this special case, a B-spline basis is equivalent to the basis of linear Lagrange finite elements. (b) Same as (a) for degree $p=2$.\label{fig_Bsplines_periodic}}
\end{figure}
The elements of the discretized domain are naturally related to the knot sequence by simply using all interior knots together with the boundaries of the domain as the element boundaries which we denote by $(c_k)_{k=0,\ldots,N_\text{el}}$, where $N_\text{el}$ is the total number of elements and $c_0=0$ and $c_{N_\text{el}}=L$. Let us summarize some important properties of a B-spline basis \citep{Ratnanietal2012}:
\begin{itemize}
\item B-splines are piecewise polynomials of degree $p$,
\item B-splines are non-negative,
\item Compact support: there are exactly $p+1$ non-vanishing B-splines in each element and the support of the B-spline $\varphi_j^p$ is contained in $[z_j,\ldots,z_{j+p+1}]$,
\item B-splines form a partition of unity: $\sum_{j=0}^{N-1}\varphi_j^p(z)=1,\quad\forall z\in\mathbb{R}$,
\item If a knot $z_m$ has multiplicity $r$ then the B-spline is $\mathcal{C}^{(p-r)}$ at $z_m$.
\end{itemize}
Since B-splines are piecewise polynomials, all matrices (mass and advection matrix) can be computed exactly by using a quadrature rule of sufficient order. Here, we use the Gauss-Legendre quadrature rule with $p+1$ quadrature points per element which allows us to integrate exactly polynomials of an order up to $2p+1$.

\textbf{PIC}. Finally, we use a classical PIC solver \citep{Birdsalletal2004} to treat the source term and thus approximate the distribution function $f_\mr{h}$ by a sum of Dirac masses in the four-dimensional phase space
\begin{align}
f_\mr{h}(z,\mb{v},t)\approx\sum_{k=1}^{N_\mr{p}}w_k\delta(z-z_k(t))\delta(\mb{v}-\mb{v}_k(t)),\label{eq_particle_distribution_function}
\end{align}
where $N_\mr{p}$ is the number of particles, $w_k$ is the weight of the $k$-th particle and $\mb{v}_k=\mb{v}_k(t)$ and $z_k=z_k(t)$ are the particles' velocities and positions, respectively, satisfying the equations of motion
\begin{subequations}
\label{eq_motion_particles}
\begin{alignat}{2}
&\frac{\mr{d}\mb{v}_k}{\mr{d}t}=\frac{q_\mr{e}}{m_\mr{e}}\left[\mb{E}(z_k(t),t)+\mb{v}_k(t)\times\mb{B}(z_k(t),t)\right],\quad\quad &\mb{v}_k(0)=\mb{v}_k^0,\\
&\frac{\mr{d}z_k}{\mr{d}t}=v_{kz}, &z_k(0)=z_k^0.
\end{alignat}
\end{subequations}
We solve this set of ordinary differential equations in time with the classical Boris method \citep{Birdsalletal2004, Boris1970, Qinetal2013} which uses a staggered grid for positions and velocities, i.e. positions are computed at integer time steps ($z_k^n \rightarrow z_k^{n+1}$), whereas velocities are computed at interleaved time steps ($\mb{v}_k^{n-1/2}\rightarrow\mb{v}_k^{n+1/2}$). The meaning of the particles' weights $w_k$ in (\ref{eq_particle_distribution_function}) becomes clear if one uses a Monte Carlo interpretation for the evaluation of the integrals over the current contribution from the energetic electrons appearing in (\ref{eq_def_righthandside}):
\begin{align}
\int_0^Lj_{\mr{h}x/y}\varphi_j\mr{d}z\underset{\underset{\text{\text{see def.} (\ref{eq_model_full_hot_3})}}{\uparrow}}{=}q_\mr{e}\int_0^L\int\underbrace{v_{x/y}\frac{f_\mr{h}}{g_\mr{h}}\varphi_j}_{=:\mathcal{R}}g_\mr{h}\mr{d}^3\mb{v}\mr{d}z\approx q_\mr{e}\frac{1}{N_\mr{p}}\sum_{k=1}^{N_\mr{p}}v_{kx/y}(t)\frac{f_\mr{h}^0(z_k^0,\mb{v}_k^0)}{g_\mr{h}^0(z_k^0,\mb{v}_k^0)}\varphi_j(z_k(t))\label{eq_hotcurrent_weak}
\end{align} 
The last expression is an estimator of the expectation value of the random variable $\mathcal{R}:=v_{x/y}\varphi_jf_\mr{h}/g_\mr{h}$ distributed under the probability density function (PDF) $g_\mr{h}$ in phase space. Since $g_\mr{h}$ is a PDF it must be normalized to one. Note that we used that the distribution function $f_\mr{h}$ and the PDF $g_\mr{h}$ are constant along a particle trajectory according to the Vlasov equation, i.e. $f_\mr{h}(z_k(t),\mb{v}_k(t),t)=f_\mr{h}^0(z_k^0,\mb{v}_k^0)$. This means that the weights are fully determined from the initial distribution function $f_\mr{h}^0$ and the sampling distribution $g_\mr{h}^0$ from which the initial particles are drawn. Throughout this work we shall entirely use the sampling distribution 
\begin{align}
g_\mr{h}^0(z,v_x,v_y,v_z)=\frac{1}{L}\frac{1}{(2\pi)^{3/2}v_{\mr{th}\parallel}v_{\mr{th}\perp}^2}\exp\left(-\frac{v_x^2+v_y^2}{2v_{\mr{th}\perp}^2}-\frac{v_z^2}{2v_{\mr{th}\parallel}^2}\right).\label{eq_sampling_distribution}
\end{align}
Consequently, we sample uniformly in real space and normally in every velocity direction using standard random number generators. With this particular choice $w_k=1/N_\mr{p}\cdot f_\mr{h}^0(z_k^0,\mb{v}_k^0)/g_\mr{h}^0(z_k^0,\mb{v}_k^0)=n_{\mr{h}0}L/N_\mr{p}$ for the anisotropic Maxwellian $f_\mr{h}^0=n_{\mr{h}0}F_\mr{h}^0$ with $F_\mr{h}^0$ given in (\ref{eq_anisotropic_Maxwellian}). Finally, since the Boris method computes positions at integer time steps and velocities at interleaved time steps, we approximate the entries of the average vector $\Delta t/2\left(\mathbb{S}^{n+1}+\mathbb{S}^n\right)$ appearing on the right-hand side of (\ref{eq_Crank_Nicolson}) due to the Crank-Nicolson discretization in the following manner:
\begin{align}
-\frac{\mu_0c^2q_\mr{e}\Delta t}{2}\sum_{k=1}^{N_\mr{p}}w_k\left[v_{kx/y}^{n+1}\varphi_j(z_k^{n+1})+v_{kx/y}^{n}\varphi_j(z_k^{n})\right]\approx-\mu_0c^2q_\mr{e}\Delta t\sum_{k=1}^{N_\mr{p}}w_kv_{kx/y}^{n+1/2}\varphi_j\left(\frac{1}{2}(z_k^{n+1}+z_k^n)\right).\label{eq_average_source_term}
\end{align}

\textbf{Algorithm}. Let us summarize the algorithm for numerically solving the model (\ref{eq_model_linearized}) for transverse electromagnetic waves only:
\begin{enumerate}
\item Create a periodic B-spline basis of degree $p$ on a domain of length $L$ discretized by $N_\mr{el}$ elements (see (\ref{eq_def_Bsplines_0}) and (\ref{eq_def_Bsplines_higher})). This results in $N=N_\mr{el}$.
\item Assemble the mass matrix $\mathbb{M}$ and advection matrix $\mathbb{C}$ and from this, assemble the block matrices $\mathbb{M}_\mr{b}=\mathbb{M}\otimes I_6\in\mathbb{R}^{6N\times 6N}$, $\tilde{\mathbb{C}}=\mathbb{C}\otimes A_1\in\mathbb{R}^{6N\times 6N}$ and $\tilde{\mathbb{M}}=\mathbb{M}\otimes A_2\in\mathbb{R}^{6N\times 6N}$.
\item Load the initial fields $\mb{U}(z,t=0)$ and perform a $L^2$-projection to get the initial coefficients $\mb{u}^0\in\mathbb{R}^{6N}$.
\item Sample the initial positions $(z_k^0)_{k=1,\ldots,N_\mr{p}}$ and velocities $(v_{kx}^0,v_{ky}^0,v_{kz}^0)_{k=1,\ldots,N_\mr{p}}$ according to the sampling distribution (\ref{eq_sampling_distribution}) by using a random number generator and compute the weights $w_k=n_{\mr{h}0}L/N_\mr{p}$.
\item Compute the electric and magnetic field at the particle positions by noting that
\begin{subequations}
\label{eq_fields_particles}
\begin{align}
&B_{x/y}(z_k^n,t^n)=\tilde{B}_{hx/y}(z_k^n,t^n)=\sum_{j=0}^{N-1}b_{x/y}^n\varphi_j(z_k^n),\\
&B_z(z_k^n,t^n)=B_0,\\
&E_{x/y}(z_k^n,t^n)=\tilde{E}_{hx/y}(z_k^n,t^n)=\sum_{j=0}^{N-1}e_{x/y}^n\varphi_j(z_k^n),\\
&E_z(z_k^n,t^n)=0.
\end{align}
\end{subequations}
\item In order to initialize the Boris algorithm with interleaved particle position and velocities, compute the velocities $(v_{kx}^{-1/2},v_{ky}^{-1/2},v_{kz}^{-1/2})_{k=1,\ldots,N_\mr{p}}$ by applying the Boris algorithm with the time step $-\Delta t/2$.
\item Start the time loop:
	\begin{enumerate}[label*=\arabic*]
	\item Update the particle positions ($z_k^n \rightarrow z_k^{n+1}$) and 	velocities ($\mb{v}_k^{n-1/2}\rightarrow\mb{v}_k^{n+1/2}$) by applying the Boris algorithm 	with the time step $\Delta t$.\label{eq_time_loop_1}
	\item Assemble the source term $\Delta t/2\left(\mathbb{S}^{n+1}+\mathbb{S}^n\right)$ in the scheme (\ref{eq_Crank_Nicolson}) according to formula (\ref{eq_average_source_term}).
	\item Update the finite element coefficients ($\mb{u}^n\rightarrow\mb{u}^{n+1}$) according to the scheme (\ref{eq_Crank_Nicolson}) with the time step $\Delta t$.
	\item Compute the new fields at the particle positions according to formulas (\ref{eq_fields_particles}).
	\item Go to 7.1.
	\end{enumerate} 
\end{enumerate}


\begin{wrapfigure}{r}{8cm}
\vspace{-0.9cm}
\centering
\includegraphics[scale=0.2]{01_Figures/deRham1D.pdf}
\caption{Commuting diagram for involved function spaces in one spatial dimension with continuous spaces in the upper line and discrete subspaces in the lower line. The connection between the two sequences in made by the projectors $\Pi_0$ and $\Pi_1$.\vspace{-0.3cm} \label{fig_commuting_diagram}}
\end{wrapfigure}

\subsection{Geometric finite element particle-in-cell}
\label{sec_geometric}
In this section, we apply a structure-preserving finite element PIC method on the same model (\ref{eq_model_linearized}), once more with transverse electromagnetic field components ($x$- and $y$-components) only. The main difference compared to standard finite element approach is that we now look for the fields ($\tilde{E}_x$, $\tilde{E}_y$, $\tilde{B}_x$, $\tilde{B}_y$, $\tilde{j}_{\mr{c}x}$, $\tilde{j}_{\mr{c}y}$) in different function spaces $H^1$, respectively $L^2$. These spaces and the respective finite-dimensional subspaces $V_0\subset H^1$ and $V_1\subset L^2$ are related according to the commuting diagram depicted in Fig. \ref{fig_commuting_diagram}, where the upper line represents the sequence of spaces involved in Maxwell's equations and the lower line the finite-dimensional counterparts. The projectors $\Pi_0:H^1\rightarrow V_0$ and $\Pi_1:L^2\rightarrow V_1$ must be constructed carefully in order to assure the diagram to be commuting, i.e. $\Pi_1\pa\psi/\pa z=\pa/\pa z\Pi_0\psi$ \citep{Krausetal2017}. 

\textbf{Weak formulation}. In analogy to the previous section, we assume the domain to be $\Omega=(0,L)$ and impose periodic boundary conditions on all quantities. Obviously, we should look for $\tilde{\mb{E}}=(\tilde{E}_x,\tilde{E}_y)$ and $\tilde{\mb{j}}_\mr{c}=(\tilde{j}_{\mr{c}x},\tilde{j}_{\mr{c}y})$ in the same space since they are never connected via spatial derivatives in the same equation. The opposite is true for the magnetic field because in Maxwell's equations $\tilde{\mb{B}}=(\tilde{B}_x,\tilde{B}_y)$ is connected with the other two quantities via a spatial derivative and therefore $\tilde{\mb{B}}$ must be an element of a different space if we want to satisfy the diagram in Fig. \ref{fig_commuting_diagram}. Consequently, there are two options: Either we choose $\tilde{\mb{B}}\in (L^2)^2$ and $\tilde{\mb{E}}$, $\tilde{\mb{j}}_\mr{c}\in (H^1)^2$ or vice versa. We follow Kraus et al. \citep{Krausetal2017} and choose the former option. In order to obtain a weak formulation, we multiply by test functions $D_x$, $D_y\in H^1$, $C_x$, $C_y\in L^2$ and $O_x$, $O_y\in H^1$ and integrate over the domain $\Omega$. This results in the following formulation: 
find $(\tilde{E}_x,\tilde{E}_y,\tilde{B}_x,\tilde{B}_y,\tilde{j}_{\text{c}x},\tilde{j}_{\text{c}y})\in H^1\times H^1\times L^2\times L^2\times H^1\times H^1$ such that
\begin{subequations}
\label{eq_weak_gem}
\begin{alignat}{2}
	&\int_0^L\frac{\partial\tilde{E}_x}{\partial t}D_x\mathrm{d}z-c^2\int_0^L\tilde{B}_y\frac{\partial D_x}{\partial z}\mathrm{d}z+\mu_0c^2\int_0^L\tilde{j}_{\text{c}x}D_x\mathrm{d}z=-\mu_0c^2\int_0^Lj_{\text{h}x}D_x\mathrm{d}z \quad\quad&&\forall\,D_x\in H^1,\\
	&\int_0^L\frac{\partial\tilde{E}_y}{\partial t}D_y\mathrm{d}z+c^2\int_0^L\tilde{B}_x\frac{\partial D_y}{\partial z}\mathrm{d}z+\mu_0c^2\int_0^L\tilde{j}_{\text{c}y}D_y\mathrm{d}z=-\mu_0c^2\int_0^Lj_{\text{h}y}D_y\mathrm{d}z &&\forall\,D_y\in H^1,\\
	&\int_0^L\frac{\partial\tilde{B}_x}{\partial t}C_x\mathrm{d}z-\int_0^L\frac{\partial\tilde{E}_y}{\partial z}C_x\mathrm{d}z=0 &&\forall\,C_x\in L^2,\\
	&\int_0^L\frac{\partial\tilde{B}_y}{\partial t}C_y\mathrm{d}z+\int_0^L\frac{\partial\tilde{E}_x}{\partial z}C_y\mathrm{d}z=0 &&\forall\,C_y\in L^2,\\  
	&\int_0^L\frac{\partial\tilde{j}_{\text{c}x}}{\partial t}O_x\mathrm{d}z-\epsilon_0\Omega_\mathrm{pe}^2\int_0^L\tilde{E}_xO_x\mathrm{d}z-\Omega_\mathrm{ce}\int_0^L\tilde{j}_{\text{c}y}O_x\mathrm{d}z=0 &&\forall\,O_x\in H^1,\\
	&\int_0^L\frac{\partial\tilde{j}_{\text{c}y}}{\partial t}O_y\mathrm{d}z-\epsilon_0\Omega_\mathrm{pe}^2\int_0^L\tilde{E}_yO_y\mathrm{d}z+\Omega_\mathrm{ce}\int_0^L\tilde{j}_{\text{c}x}O_y\mathrm{d}z=0 &&\forall\,O_y\in H^1.
\end{alignat}
\end{subequations} 
Due to this particular choice for the function spaces, we have integrated by parts the terms involving the magnetic field in Amp\'{e}re's law in order for the weak formulation to be well-defined (this changes the sign). This has the consequence that these equations will be solved in a weak sense, whereas the other equations will be solved in a strong sense. Note that this procedure is actually not necessary for the last two equations since they do not involve spatial derivatives and are thus ordinary differential equations in time. However, for reasons of clarity, we continue with the above formulation. We will see later that all matrices due to the spatial discretization cancel out.

As a next step, we replace the spaces  $H^1$ and $L^2$ by their finite-dimensional counterparts $V_0\subset H^1$ and $V_1\subset L^2$ and denote the dimensions by $\dim V_0=N_0$ and $\dim V_1=N_1$ and the set of basis functions that span the spaces by $(\varphi^0_j)_{j=0,\ldots,N_0-1}$ and $(\varphi^1_{j+1/2})_{j=0,\ldots,N_1-1}$, respectively. The discrete version of (\ref{eq_weak_gem}) then simply reads: find $(\tilde{E}_{hx},\tilde{E}_{hy},\tilde{B}_{hx},\tilde{B}_{hy},\tilde{j}_{\text{c}x}^h,\tilde{j}_{\text{c}y}^h)\in V_0\times V_0\times V_1\times V_1\times V_0\times V_0$ such that
\begin{subequations}
\label{eq_weak_gem_discrete}
\begin{alignat}{2}
	&\int_0^L\frac{\partial\tilde{E}_{hx}}{\partial t}D_{hx}\mathrm{d}z-c^2\int_0^L\tilde{B}_{hy}\frac{\partial D_{hx}}{\partial z}\mathrm{d}z+\mu_0c^2\int_0^L\tilde{j}_{\text{c}x}^hD_{hx}\mathrm{d}z=-\mu_0c^2\int_0^Lj_{\text{h}x}D_{hx}\mathrm{d}z \quad\quad&&\forall\,D_{hx}\in V_0,\label{eq_weak_gem_discrete_1}\\
	&\int_0^L\frac{\partial\tilde{E}_{hy}}{\partial t}D_{hy}\mathrm{d}z+c^2\int_0^L\tilde{B}_{hx}\frac{\partial D_{hy}}{\partial z}\mathrm{d}z+\mu_0c^2\int_0^L\tilde{j}_{\text{c}y}^hD_{hy}\mathrm{d}z=-\mu_0c^2\int_0^Lj_{\text{h}y}D_{hy}\mathrm{d}z &&\forall\,D_{hy}\in V_0,\\
	&\int_0^L\frac{\partial\tilde{B}_{hx}}{\partial t}C_{hx}\mathrm{d}z-\int_0^L\frac{\partial\tilde{E}_{hy}}{\partial z}C_{hx}\mathrm{d}z=0 &&\forall\,C_{hx}\in V_1,\\
	&\int_0^L\frac{\partial\tilde{B}_{hy}}{\partial t}C_{hy}\mathrm{d}z+\int_0^L\frac{\partial\tilde{E}_{hx}}{\partial z}C_{hy}\mathrm{d}z=0 &&\forall\,C_{hy}\in V_1,\\  
	&\int_0^L\frac{\partial\tilde{j}_{\text{c}x}^h}{\partial t}O_{hx}\mathrm{d}z-\epsilon_0\Omega_\mathrm{pe}^2\int_0^L\tilde{E}_{hx}O_{hx}\mathrm{d}z-\Omega_\mathrm{ce}\int_0^L\tilde{j}_{\text{c}y}^hO_{hx}\mathrm{d}z=0 &&\forall\,O_{hx}\in V_0,\\
	&\int_0^L\frac{\partial\tilde{j}_{\text{c}y}^h}{\partial t}O_{hy}\mathrm{d}z-\epsilon_0\Omega_\mathrm{pe}^2\int_0^L\tilde{E}_{hy}O_{hy}\mathrm{d}z+\Omega_\mathrm{ce}\int_0^L\tilde{j}_{\text{c}x}^hO_{hy}\mathrm{d}z=0 &&\forall\,O_{hy}\in V_0.
\end{alignat}
\end{subequations}

\textbf{Commuting diagram}. There are multiple possibilities to construct the commuting diagram shown in Fig. \ref{fig_commuting_diagram}. The general procedure is to define a basis for the first subspace $V_0$, then to look for an appropriate basis for the next space $V_1$ in order to satisfy the sequence for differential operators in the lower line,  and finally to find the projectors such that the diagram is commuting. For the space $V_0$, we choose standard Lagrange finite elements\footnote{In doing FEEC, one is not restricted to Lagrange FEM. One can take any kind of basis for $V_0$, in particular splines.} of degree $p$ which are most easily defined on a reference element $I=[-1,1]$ together with a mapping $F_k:I\rightarrow\Omega_k$, $s\mapsto z$ on elements $\Omega_k=[c_k,c_{k+1}]$ on the physical domain $\Omega$, where $(c_k)_{k=0,\ldots,N_\mr{el}}$ denote the boundaries of $N_\mr{el}$ elements (and the elements are simply labeled by $0,\ldots,N_\mr{el}-1$). The mapping $F_k$ and its inverse $F_k^{-1}$ are given by
\begin{subequations}
\label{eq_mapping}
\begin{align}
&z=F_k(s):=c_k+\frac{s+1}{2}(c_{k+1}-c_k),\\
&s=F_k^{-1}(z):=\frac{2(z-c_k)}{c_{k+1}-c_k}-1.
\end{align}
\end{subequations}
The Lagrange \textit{shape} functions $(\eta_n(s))_{n=0,\ldots,p}$ of degree $p$ in the reference element $I$ are created from a sequence of knots $s_0=-1<\ldots<s_m<\ldots<1=s_p$ and are defined by $\eta_n(s_m)=\delta_{nm}$, which leads to the well-known formula
\begin{align}
\eta_n(s)=\prod_{m\neq n}\frac{s-s_m}{s_n-s_m}.\label{eq_def_Lagrange_shape}
\end{align}
\begin{figure}[!t]
\centering
\includegraphics[scale=1]{01_Figures/Lagrange_poly_p=2.pdf}
\includegraphics[scale=1]{01_Figures/Lagrange_histo_p=2.pdf}
%%% Creator: Matplotlib, PGF backend
%%
%% To include the figure in your LaTeX document, write
%%   \input{<filename>.pgf}
%%
%% Make sure the required packages are loaded in your preamble
%%   \usepackage{pgf}
%%
%% Figures using additional raster images can only be included by \input if
%% they are in the same directory as the main LaTeX file. For loading figures
%% from other directories you can use the `import` package
%%   \usepackage{import}
%% and then include the figures with
%%   \import{<path to file>}{<filename>.pgf}
%%
%% Matplotlib used the following preamble
%%   \usepackage{fontspec}
%%   \setmainfont{DejaVu Serif}
%%   \setsansfont{DejaVu Sans}
%%   \setmonofont{DejaVu Sans Mono}
%%
\begingroup%
\makeatletter%
\begin{pgfpicture}%
\pgfpathrectangle{\pgfpointorigin}{\pgfqpoint{3.198427in}{2.633214in}}%
\pgfusepath{use as bounding box, clip}%
\begin{pgfscope}%
\pgfsetbuttcap%
\pgfsetmiterjoin%
\definecolor{currentfill}{rgb}{1.000000,1.000000,1.000000}%
\pgfsetfillcolor{currentfill}%
\pgfsetlinewidth{0.000000pt}%
\definecolor{currentstroke}{rgb}{1.000000,1.000000,1.000000}%
\pgfsetstrokecolor{currentstroke}%
\pgfsetdash{}{0pt}%
\pgfpathmoveto{\pgfqpoint{0.000000in}{0.000000in}}%
\pgfpathlineto{\pgfqpoint{3.198427in}{0.000000in}}%
\pgfpathlineto{\pgfqpoint{3.198427in}{2.633214in}}%
\pgfpathlineto{\pgfqpoint{0.000000in}{2.633214in}}%
\pgfpathclose%
\pgfusepath{fill}%
\end{pgfscope}%
\begin{pgfscope}%
\pgfsetbuttcap%
\pgfsetmiterjoin%
\definecolor{currentfill}{rgb}{1.000000,1.000000,1.000000}%
\pgfsetfillcolor{currentfill}%
\pgfsetlinewidth{0.000000pt}%
\definecolor{currentstroke}{rgb}{0.000000,0.000000,0.000000}%
\pgfsetstrokecolor{currentstroke}%
\pgfsetstrokeopacity{0.000000}%
\pgfsetdash{}{0pt}%
\pgfpathmoveto{\pgfqpoint{0.374692in}{0.521603in}}%
\pgfpathlineto{\pgfqpoint{3.009692in}{0.521603in}}%
\pgfpathlineto{\pgfqpoint{3.009692in}{2.484603in}}%
\pgfpathlineto{\pgfqpoint{0.374692in}{2.484603in}}%
\pgfpathclose%
\pgfusepath{fill}%
\end{pgfscope}%
\begin{pgfscope}%
\pgfsetbuttcap%
\pgfsetroundjoin%
\definecolor{currentfill}{rgb}{0.000000,0.000000,0.000000}%
\pgfsetfillcolor{currentfill}%
\pgfsetlinewidth{0.803000pt}%
\definecolor{currentstroke}{rgb}{0.000000,0.000000,0.000000}%
\pgfsetstrokecolor{currentstroke}%
\pgfsetdash{}{0pt}%
\pgfsys@defobject{currentmarker}{\pgfqpoint{0.000000in}{-0.048611in}}{\pgfqpoint{0.000000in}{0.000000in}}{%
\pgfpathmoveto{\pgfqpoint{0.000000in}{0.000000in}}%
\pgfpathlineto{\pgfqpoint{0.000000in}{-0.048611in}}%
\pgfusepath{stroke,fill}%
}%
\begin{pgfscope}%
\pgfsys@transformshift{0.374692in}{0.521603in}%
\pgfsys@useobject{currentmarker}{}%
\end{pgfscope}%
\end{pgfscope}%
\begin{pgfscope}%
\pgftext[x=0.374692in,y=0.424381in,,top]{\rmfamily\fontsize{10.000000}{12.000000}\selectfont \(\displaystyle 0.0\)}%
\end{pgfscope}%
\begin{pgfscope}%
\pgfsetbuttcap%
\pgfsetroundjoin%
\definecolor{currentfill}{rgb}{0.000000,0.000000,0.000000}%
\pgfsetfillcolor{currentfill}%
\pgfsetlinewidth{0.803000pt}%
\definecolor{currentstroke}{rgb}{0.000000,0.000000,0.000000}%
\pgfsetstrokecolor{currentstroke}%
\pgfsetdash{}{0pt}%
\pgfsys@defobject{currentmarker}{\pgfqpoint{0.000000in}{-0.048611in}}{\pgfqpoint{0.000000in}{0.000000in}}{%
\pgfpathmoveto{\pgfqpoint{0.000000in}{0.000000in}}%
\pgfpathlineto{\pgfqpoint{0.000000in}{-0.048611in}}%
\pgfusepath{stroke,fill}%
}%
\begin{pgfscope}%
\pgfsys@transformshift{0.901692in}{0.521603in}%
\pgfsys@useobject{currentmarker}{}%
\end{pgfscope}%
\end{pgfscope}%
\begin{pgfscope}%
\pgftext[x=0.901692in,y=0.424381in,,top]{\rmfamily\fontsize{10.000000}{12.000000}\selectfont \(\displaystyle 0.2\)}%
\end{pgfscope}%
\begin{pgfscope}%
\pgfsetbuttcap%
\pgfsetroundjoin%
\definecolor{currentfill}{rgb}{0.000000,0.000000,0.000000}%
\pgfsetfillcolor{currentfill}%
\pgfsetlinewidth{0.803000pt}%
\definecolor{currentstroke}{rgb}{0.000000,0.000000,0.000000}%
\pgfsetstrokecolor{currentstroke}%
\pgfsetdash{}{0pt}%
\pgfsys@defobject{currentmarker}{\pgfqpoint{0.000000in}{-0.048611in}}{\pgfqpoint{0.000000in}{0.000000in}}{%
\pgfpathmoveto{\pgfqpoint{0.000000in}{0.000000in}}%
\pgfpathlineto{\pgfqpoint{0.000000in}{-0.048611in}}%
\pgfusepath{stroke,fill}%
}%
\begin{pgfscope}%
\pgfsys@transformshift{1.428692in}{0.521603in}%
\pgfsys@useobject{currentmarker}{}%
\end{pgfscope}%
\end{pgfscope}%
\begin{pgfscope}%
\pgftext[x=1.428692in,y=0.424381in,,top]{\rmfamily\fontsize{10.000000}{12.000000}\selectfont \(\displaystyle 0.4\)}%
\end{pgfscope}%
\begin{pgfscope}%
\pgfsetbuttcap%
\pgfsetroundjoin%
\definecolor{currentfill}{rgb}{0.000000,0.000000,0.000000}%
\pgfsetfillcolor{currentfill}%
\pgfsetlinewidth{0.803000pt}%
\definecolor{currentstroke}{rgb}{0.000000,0.000000,0.000000}%
\pgfsetstrokecolor{currentstroke}%
\pgfsetdash{}{0pt}%
\pgfsys@defobject{currentmarker}{\pgfqpoint{0.000000in}{-0.048611in}}{\pgfqpoint{0.000000in}{0.000000in}}{%
\pgfpathmoveto{\pgfqpoint{0.000000in}{0.000000in}}%
\pgfpathlineto{\pgfqpoint{0.000000in}{-0.048611in}}%
\pgfusepath{stroke,fill}%
}%
\begin{pgfscope}%
\pgfsys@transformshift{1.955692in}{0.521603in}%
\pgfsys@useobject{currentmarker}{}%
\end{pgfscope}%
\end{pgfscope}%
\begin{pgfscope}%
\pgftext[x=1.955692in,y=0.424381in,,top]{\rmfamily\fontsize{10.000000}{12.000000}\selectfont \(\displaystyle 0.6\)}%
\end{pgfscope}%
\begin{pgfscope}%
\pgfsetbuttcap%
\pgfsetroundjoin%
\definecolor{currentfill}{rgb}{0.000000,0.000000,0.000000}%
\pgfsetfillcolor{currentfill}%
\pgfsetlinewidth{0.803000pt}%
\definecolor{currentstroke}{rgb}{0.000000,0.000000,0.000000}%
\pgfsetstrokecolor{currentstroke}%
\pgfsetdash{}{0pt}%
\pgfsys@defobject{currentmarker}{\pgfqpoint{0.000000in}{-0.048611in}}{\pgfqpoint{0.000000in}{0.000000in}}{%
\pgfpathmoveto{\pgfqpoint{0.000000in}{0.000000in}}%
\pgfpathlineto{\pgfqpoint{0.000000in}{-0.048611in}}%
\pgfusepath{stroke,fill}%
}%
\begin{pgfscope}%
\pgfsys@transformshift{2.482692in}{0.521603in}%
\pgfsys@useobject{currentmarker}{}%
\end{pgfscope}%
\end{pgfscope}%
\begin{pgfscope}%
\pgftext[x=2.482692in,y=0.424381in,,top]{\rmfamily\fontsize{10.000000}{12.000000}\selectfont \(\displaystyle 0.8\)}%
\end{pgfscope}%
\begin{pgfscope}%
\pgfsetbuttcap%
\pgfsetroundjoin%
\definecolor{currentfill}{rgb}{0.000000,0.000000,0.000000}%
\pgfsetfillcolor{currentfill}%
\pgfsetlinewidth{0.803000pt}%
\definecolor{currentstroke}{rgb}{0.000000,0.000000,0.000000}%
\pgfsetstrokecolor{currentstroke}%
\pgfsetdash{}{0pt}%
\pgfsys@defobject{currentmarker}{\pgfqpoint{0.000000in}{-0.048611in}}{\pgfqpoint{0.000000in}{0.000000in}}{%
\pgfpathmoveto{\pgfqpoint{0.000000in}{0.000000in}}%
\pgfpathlineto{\pgfqpoint{0.000000in}{-0.048611in}}%
\pgfusepath{stroke,fill}%
}%
\begin{pgfscope}%
\pgfsys@transformshift{3.009692in}{0.521603in}%
\pgfsys@useobject{currentmarker}{}%
\end{pgfscope}%
\end{pgfscope}%
\begin{pgfscope}%
\pgftext[x=3.009692in,y=0.424381in,,top]{\rmfamily\fontsize{10.000000}{12.000000}\selectfont \(\displaystyle 1.0\)}%
\end{pgfscope}%
\begin{pgfscope}%
\pgftext[x=1.692192in,y=0.234413in,,top]{\rmfamily\fontsize{10.000000}{12.000000}\selectfont \(\displaystyle z\)}%
\end{pgfscope}%
\begin{pgfscope}%
\pgfsetbuttcap%
\pgfsetroundjoin%
\definecolor{currentfill}{rgb}{0.000000,0.000000,0.000000}%
\pgfsetfillcolor{currentfill}%
\pgfsetlinewidth{0.803000pt}%
\definecolor{currentstroke}{rgb}{0.000000,0.000000,0.000000}%
\pgfsetstrokecolor{currentstroke}%
\pgfsetdash{}{0pt}%
\pgfsys@defobject{currentmarker}{\pgfqpoint{-0.048611in}{0.000000in}}{\pgfqpoint{0.000000in}{0.000000in}}{%
\pgfpathmoveto{\pgfqpoint{0.000000in}{0.000000in}}%
\pgfpathlineto{\pgfqpoint{-0.048611in}{0.000000in}}%
\pgfusepath{stroke,fill}%
}%
\begin{pgfscope}%
\pgfsys@transformshift{0.374692in}{0.685187in}%
\pgfsys@useobject{currentmarker}{}%
\end{pgfscope}%
\end{pgfscope}%
\begin{pgfscope}%
\pgftext[x=0.100000in,y=0.632425in,left,base]{\rmfamily\fontsize{10.000000}{12.000000}\selectfont \(\displaystyle 0.0\)}%
\end{pgfscope}%
\begin{pgfscope}%
\pgfsetbuttcap%
\pgfsetroundjoin%
\definecolor{currentfill}{rgb}{0.000000,0.000000,0.000000}%
\pgfsetfillcolor{currentfill}%
\pgfsetlinewidth{0.803000pt}%
\definecolor{currentstroke}{rgb}{0.000000,0.000000,0.000000}%
\pgfsetstrokecolor{currentstroke}%
\pgfsetdash{}{0pt}%
\pgfsys@defobject{currentmarker}{\pgfqpoint{-0.048611in}{0.000000in}}{\pgfqpoint{0.000000in}{0.000000in}}{%
\pgfpathmoveto{\pgfqpoint{0.000000in}{0.000000in}}%
\pgfpathlineto{\pgfqpoint{-0.048611in}{0.000000in}}%
\pgfusepath{stroke,fill}%
}%
\begin{pgfscope}%
\pgfsys@transformshift{0.374692in}{1.012353in}%
\pgfsys@useobject{currentmarker}{}%
\end{pgfscope}%
\end{pgfscope}%
\begin{pgfscope}%
\pgftext[x=0.100000in,y=0.959592in,left,base]{\rmfamily\fontsize{10.000000}{12.000000}\selectfont \(\displaystyle 0.5\)}%
\end{pgfscope}%
\begin{pgfscope}%
\pgfsetbuttcap%
\pgfsetroundjoin%
\definecolor{currentfill}{rgb}{0.000000,0.000000,0.000000}%
\pgfsetfillcolor{currentfill}%
\pgfsetlinewidth{0.803000pt}%
\definecolor{currentstroke}{rgb}{0.000000,0.000000,0.000000}%
\pgfsetstrokecolor{currentstroke}%
\pgfsetdash{}{0pt}%
\pgfsys@defobject{currentmarker}{\pgfqpoint{-0.048611in}{0.000000in}}{\pgfqpoint{0.000000in}{0.000000in}}{%
\pgfpathmoveto{\pgfqpoint{0.000000in}{0.000000in}}%
\pgfpathlineto{\pgfqpoint{-0.048611in}{0.000000in}}%
\pgfusepath{stroke,fill}%
}%
\begin{pgfscope}%
\pgfsys@transformshift{0.374692in}{1.339520in}%
\pgfsys@useobject{currentmarker}{}%
\end{pgfscope}%
\end{pgfscope}%
\begin{pgfscope}%
\pgftext[x=0.100000in,y=1.286758in,left,base]{\rmfamily\fontsize{10.000000}{12.000000}\selectfont \(\displaystyle 1.0\)}%
\end{pgfscope}%
\begin{pgfscope}%
\pgfsetbuttcap%
\pgfsetroundjoin%
\definecolor{currentfill}{rgb}{0.000000,0.000000,0.000000}%
\pgfsetfillcolor{currentfill}%
\pgfsetlinewidth{0.803000pt}%
\definecolor{currentstroke}{rgb}{0.000000,0.000000,0.000000}%
\pgfsetstrokecolor{currentstroke}%
\pgfsetdash{}{0pt}%
\pgfsys@defobject{currentmarker}{\pgfqpoint{-0.048611in}{0.000000in}}{\pgfqpoint{0.000000in}{0.000000in}}{%
\pgfpathmoveto{\pgfqpoint{0.000000in}{0.000000in}}%
\pgfpathlineto{\pgfqpoint{-0.048611in}{0.000000in}}%
\pgfusepath{stroke,fill}%
}%
\begin{pgfscope}%
\pgfsys@transformshift{0.374692in}{1.666687in}%
\pgfsys@useobject{currentmarker}{}%
\end{pgfscope}%
\end{pgfscope}%
\begin{pgfscope}%
\pgftext[x=0.100000in,y=1.613925in,left,base]{\rmfamily\fontsize{10.000000}{12.000000}\selectfont \(\displaystyle 1.5\)}%
\end{pgfscope}%
\begin{pgfscope}%
\pgfsetbuttcap%
\pgfsetroundjoin%
\definecolor{currentfill}{rgb}{0.000000,0.000000,0.000000}%
\pgfsetfillcolor{currentfill}%
\pgfsetlinewidth{0.803000pt}%
\definecolor{currentstroke}{rgb}{0.000000,0.000000,0.000000}%
\pgfsetstrokecolor{currentstroke}%
\pgfsetdash{}{0pt}%
\pgfsys@defobject{currentmarker}{\pgfqpoint{-0.048611in}{0.000000in}}{\pgfqpoint{0.000000in}{0.000000in}}{%
\pgfpathmoveto{\pgfqpoint{0.000000in}{0.000000in}}%
\pgfpathlineto{\pgfqpoint{-0.048611in}{0.000000in}}%
\pgfusepath{stroke,fill}%
}%
\begin{pgfscope}%
\pgfsys@transformshift{0.374692in}{1.993853in}%
\pgfsys@useobject{currentmarker}{}%
\end{pgfscope}%
\end{pgfscope}%
\begin{pgfscope}%
\pgftext[x=0.100000in,y=1.941092in,left,base]{\rmfamily\fontsize{10.000000}{12.000000}\selectfont \(\displaystyle 2.0\)}%
\end{pgfscope}%
\begin{pgfscope}%
\pgfsetbuttcap%
\pgfsetroundjoin%
\definecolor{currentfill}{rgb}{0.000000,0.000000,0.000000}%
\pgfsetfillcolor{currentfill}%
\pgfsetlinewidth{0.803000pt}%
\definecolor{currentstroke}{rgb}{0.000000,0.000000,0.000000}%
\pgfsetstrokecolor{currentstroke}%
\pgfsetdash{}{0pt}%
\pgfsys@defobject{currentmarker}{\pgfqpoint{-0.048611in}{0.000000in}}{\pgfqpoint{0.000000in}{0.000000in}}{%
\pgfpathmoveto{\pgfqpoint{0.000000in}{0.000000in}}%
\pgfpathlineto{\pgfqpoint{-0.048611in}{0.000000in}}%
\pgfusepath{stroke,fill}%
}%
\begin{pgfscope}%
\pgfsys@transformshift{0.374692in}{2.321020in}%
\pgfsys@useobject{currentmarker}{}%
\end{pgfscope}%
\end{pgfscope}%
\begin{pgfscope}%
\pgftext[x=0.100000in,y=2.268258in,left,base]{\rmfamily\fontsize{10.000000}{12.000000}\selectfont \(\displaystyle 2.5\)}%
\end{pgfscope}%
\begin{pgfscope}%
\pgfpathrectangle{\pgfqpoint{0.374692in}{0.521603in}}{\pgfqpoint{2.635000in}{1.963000in}} %
\pgfusepath{clip}%
\pgfsetrectcap%
\pgfsetroundjoin%
\pgfsetlinewidth{1.003750pt}%
\definecolor{currentstroke}{rgb}{0.121569,0.466667,0.705882}%
\pgfsetstrokecolor{currentstroke}%
\pgfsetdash{}{0pt}%
\pgfpathmoveto{\pgfqpoint{0.374692in}{1.339520in}}%
\pgfpathlineto{\pgfqpoint{0.383564in}{1.319825in}}%
\pgfpathlineto{\pgfqpoint{0.392436in}{1.300398in}}%
\pgfpathlineto{\pgfqpoint{0.401308in}{1.281237in}}%
\pgfpathlineto{\pgfqpoint{0.410180in}{1.262343in}}%
\pgfpathlineto{\pgfqpoint{0.419052in}{1.243717in}}%
\pgfpathlineto{\pgfqpoint{0.427924in}{1.225357in}}%
\pgfpathlineto{\pgfqpoint{0.436796in}{1.207265in}}%
\pgfpathlineto{\pgfqpoint{0.445668in}{1.189439in}}%
\pgfpathlineto{\pgfqpoint{0.454540in}{1.171881in}}%
\pgfpathlineto{\pgfqpoint{0.463412in}{1.154590in}}%
\pgfpathlineto{\pgfqpoint{0.472285in}{1.137565in}}%
\pgfpathlineto{\pgfqpoint{0.481157in}{1.120808in}}%
\pgfpathlineto{\pgfqpoint{0.490029in}{1.104318in}}%
\pgfpathlineto{\pgfqpoint{0.498901in}{1.088095in}}%
\pgfpathlineto{\pgfqpoint{0.507773in}{1.072139in}}%
\pgfpathlineto{\pgfqpoint{0.516645in}{1.056450in}}%
\pgfpathlineto{\pgfqpoint{0.525517in}{1.041028in}}%
\pgfpathlineto{\pgfqpoint{0.534389in}{1.025873in}}%
\pgfpathlineto{\pgfqpoint{0.543261in}{1.010985in}}%
\pgfpathlineto{\pgfqpoint{0.552133in}{0.996364in}}%
\pgfpathlineto{\pgfqpoint{0.561005in}{0.982010in}}%
\pgfpathlineto{\pgfqpoint{0.569877in}{0.967923in}}%
\pgfpathlineto{\pgfqpoint{0.578749in}{0.954104in}}%
\pgfpathlineto{\pgfqpoint{0.587621in}{0.940551in}}%
\pgfpathlineto{\pgfqpoint{0.596493in}{0.927265in}}%
\pgfpathlineto{\pgfqpoint{0.605365in}{0.914247in}}%
\pgfpathlineto{\pgfqpoint{0.614237in}{0.901495in}}%
\pgfpathlineto{\pgfqpoint{0.623109in}{0.889011in}}%
\pgfpathlineto{\pgfqpoint{0.631982in}{0.876793in}}%
\pgfpathlineto{\pgfqpoint{0.640854in}{0.864843in}}%
\pgfpathlineto{\pgfqpoint{0.649726in}{0.853160in}}%
\pgfpathlineto{\pgfqpoint{0.658598in}{0.841743in}}%
\pgfpathlineto{\pgfqpoint{0.667470in}{0.830594in}}%
\pgfpathlineto{\pgfqpoint{0.676342in}{0.819712in}}%
\pgfpathlineto{\pgfqpoint{0.685214in}{0.809097in}}%
\pgfpathlineto{\pgfqpoint{0.694086in}{0.798749in}}%
\pgfpathlineto{\pgfqpoint{0.702958in}{0.788668in}}%
\pgfpathlineto{\pgfqpoint{0.711830in}{0.778854in}}%
\pgfpathlineto{\pgfqpoint{0.720702in}{0.769307in}}%
\pgfpathlineto{\pgfqpoint{0.729574in}{0.760027in}}%
\pgfpathlineto{\pgfqpoint{0.738446in}{0.751014in}}%
\pgfpathlineto{\pgfqpoint{0.747318in}{0.742268in}}%
\pgfpathlineto{\pgfqpoint{0.756190in}{0.733789in}}%
\pgfpathlineto{\pgfqpoint{0.765062in}{0.725578in}}%
\pgfpathlineto{\pgfqpoint{0.773934in}{0.717633in}}%
\pgfpathlineto{\pgfqpoint{0.782806in}{0.709955in}}%
\pgfpathlineto{\pgfqpoint{0.791678in}{0.702545in}}%
\pgfpathlineto{\pgfqpoint{0.800551in}{0.695401in}}%
\pgfpathlineto{\pgfqpoint{0.809423in}{0.688525in}}%
\pgfpathlineto{\pgfqpoint{0.818295in}{0.681915in}}%
\pgfpathlineto{\pgfqpoint{0.827167in}{0.675573in}}%
\pgfpathlineto{\pgfqpoint{0.836039in}{0.669498in}}%
\pgfpathlineto{\pgfqpoint{0.844911in}{0.663689in}}%
\pgfpathlineto{\pgfqpoint{0.853783in}{0.658148in}}%
\pgfpathlineto{\pgfqpoint{0.862655in}{0.652874in}}%
\pgfpathlineto{\pgfqpoint{0.871527in}{0.647867in}}%
\pgfpathlineto{\pgfqpoint{0.880399in}{0.643127in}}%
\pgfpathlineto{\pgfqpoint{0.889271in}{0.638654in}}%
\pgfpathlineto{\pgfqpoint{0.898143in}{0.634448in}}%
\pgfpathlineto{\pgfqpoint{0.907015in}{0.630509in}}%
\pgfpathlineto{\pgfqpoint{0.915887in}{0.626837in}}%
\pgfpathlineto{\pgfqpoint{0.924759in}{0.623432in}}%
\pgfpathlineto{\pgfqpoint{0.933631in}{0.620294in}}%
\pgfpathlineto{\pgfqpoint{0.942503in}{0.617423in}}%
\pgfpathlineto{\pgfqpoint{0.951375in}{0.614820in}}%
\pgfpathlineto{\pgfqpoint{0.960248in}{0.612483in}}%
\pgfpathlineto{\pgfqpoint{0.969120in}{0.610413in}}%
\pgfpathlineto{\pgfqpoint{0.977992in}{0.608611in}}%
\pgfpathlineto{\pgfqpoint{0.986864in}{0.607075in}}%
\pgfpathlineto{\pgfqpoint{0.995736in}{0.605807in}}%
\pgfpathlineto{\pgfqpoint{1.004608in}{0.604805in}}%
\pgfpathlineto{\pgfqpoint{1.013480in}{0.604071in}}%
\pgfpathlineto{\pgfqpoint{1.022352in}{0.603604in}}%
\pgfpathlineto{\pgfqpoint{1.031224in}{0.603403in}}%
\pgfpathlineto{\pgfqpoint{1.040096in}{0.603470in}}%
\pgfpathlineto{\pgfqpoint{1.048968in}{0.603804in}}%
\pgfpathlineto{\pgfqpoint{1.057840in}{0.604405in}}%
\pgfpathlineto{\pgfqpoint{1.066712in}{0.605273in}}%
\pgfpathlineto{\pgfqpoint{1.075584in}{0.606408in}}%
\pgfpathlineto{\pgfqpoint{1.084456in}{0.607810in}}%
\pgfpathlineto{\pgfqpoint{1.093328in}{0.609479in}}%
\pgfpathlineto{\pgfqpoint{1.102200in}{0.611415in}}%
\pgfpathlineto{\pgfqpoint{1.111072in}{0.613618in}}%
\pgfpathlineto{\pgfqpoint{1.119944in}{0.616088in}}%
\pgfpathlineto{\pgfqpoint{1.128817in}{0.618825in}}%
\pgfpathlineto{\pgfqpoint{1.137689in}{0.621830in}}%
\pgfpathlineto{\pgfqpoint{1.146561in}{0.625101in}}%
\pgfpathlineto{\pgfqpoint{1.155433in}{0.628639in}}%
\pgfpathlineto{\pgfqpoint{1.164305in}{0.632445in}}%
\pgfpathlineto{\pgfqpoint{1.173177in}{0.636517in}}%
\pgfpathlineto{\pgfqpoint{1.182049in}{0.640857in}}%
\pgfpathlineto{\pgfqpoint{1.190921in}{0.645463in}}%
\pgfpathlineto{\pgfqpoint{1.199793in}{0.650337in}}%
\pgfpathlineto{\pgfqpoint{1.208665in}{0.655478in}}%
\pgfpathlineto{\pgfqpoint{1.217537in}{0.660885in}}%
\pgfpathlineto{\pgfqpoint{1.226409in}{0.666560in}}%
\pgfpathlineto{\pgfqpoint{1.235281in}{0.672502in}}%
\pgfpathlineto{\pgfqpoint{1.244153in}{0.678711in}}%
\pgfpathlineto{\pgfqpoint{1.253025in}{0.685187in}}%
\pgfusepath{stroke}%
\end{pgfscope}%
\begin{pgfscope}%
\pgfpathrectangle{\pgfqpoint{0.374692in}{0.521603in}}{\pgfqpoint{2.635000in}{1.963000in}} %
\pgfusepath{clip}%
\pgfsetrectcap%
\pgfsetroundjoin%
\pgfsetlinewidth{1.003750pt}%
\definecolor{currentstroke}{rgb}{1.000000,0.498039,0.054902}%
\pgfsetstrokecolor{currentstroke}%
\pgfsetdash{}{0pt}%
\pgfpathmoveto{\pgfqpoint{0.374692in}{0.685187in}}%
\pgfpathlineto{\pgfqpoint{0.383564in}{0.711357in}}%
\pgfpathlineto{\pgfqpoint{0.392436in}{0.736994in}}%
\pgfpathlineto{\pgfqpoint{0.401308in}{0.762096in}}%
\pgfpathlineto{\pgfqpoint{0.410180in}{0.786665in}}%
\pgfpathlineto{\pgfqpoint{0.419052in}{0.810699in}}%
\pgfpathlineto{\pgfqpoint{0.427924in}{0.834199in}}%
\pgfpathlineto{\pgfqpoint{0.436796in}{0.857165in}}%
\pgfpathlineto{\pgfqpoint{0.445668in}{0.879597in}}%
\pgfpathlineto{\pgfqpoint{0.454540in}{0.901495in}}%
\pgfpathlineto{\pgfqpoint{0.463412in}{0.922859in}}%
\pgfpathlineto{\pgfqpoint{0.472285in}{0.943689in}}%
\pgfpathlineto{\pgfqpoint{0.481157in}{0.963984in}}%
\pgfpathlineto{\pgfqpoint{0.490029in}{0.983746in}}%
\pgfpathlineto{\pgfqpoint{0.498901in}{1.002973in}}%
\pgfpathlineto{\pgfqpoint{0.507773in}{1.021667in}}%
\pgfpathlineto{\pgfqpoint{0.516645in}{1.039826in}}%
\pgfpathlineto{\pgfqpoint{0.525517in}{1.057451in}}%
\pgfpathlineto{\pgfqpoint{0.534389in}{1.074542in}}%
\pgfpathlineto{\pgfqpoint{0.543261in}{1.091099in}}%
\pgfpathlineto{\pgfqpoint{0.552133in}{1.107122in}}%
\pgfpathlineto{\pgfqpoint{0.561005in}{1.122611in}}%
\pgfpathlineto{\pgfqpoint{0.569877in}{1.137565in}}%
\pgfpathlineto{\pgfqpoint{0.578749in}{1.151986in}}%
\pgfpathlineto{\pgfqpoint{0.587621in}{1.165872in}}%
\pgfpathlineto{\pgfqpoint{0.596493in}{1.179225in}}%
\pgfpathlineto{\pgfqpoint{0.605365in}{1.192043in}}%
\pgfpathlineto{\pgfqpoint{0.614237in}{1.204327in}}%
\pgfpathlineto{\pgfqpoint{0.623109in}{1.216077in}}%
\pgfpathlineto{\pgfqpoint{0.631982in}{1.227293in}}%
\pgfpathlineto{\pgfqpoint{0.640854in}{1.237975in}}%
\pgfpathlineto{\pgfqpoint{0.649726in}{1.248123in}}%
\pgfpathlineto{\pgfqpoint{0.658598in}{1.257737in}}%
\pgfpathlineto{\pgfqpoint{0.667470in}{1.266816in}}%
\pgfpathlineto{\pgfqpoint{0.676342in}{1.275362in}}%
\pgfpathlineto{\pgfqpoint{0.685214in}{1.283373in}}%
\pgfpathlineto{\pgfqpoint{0.694086in}{1.290851in}}%
\pgfpathlineto{\pgfqpoint{0.702958in}{1.297794in}}%
\pgfpathlineto{\pgfqpoint{0.711830in}{1.304203in}}%
\pgfpathlineto{\pgfqpoint{0.720702in}{1.310078in}}%
\pgfpathlineto{\pgfqpoint{0.729574in}{1.315419in}}%
\pgfpathlineto{\pgfqpoint{0.738446in}{1.320226in}}%
\pgfpathlineto{\pgfqpoint{0.747318in}{1.324499in}}%
\pgfpathlineto{\pgfqpoint{0.756190in}{1.328237in}}%
\pgfpathlineto{\pgfqpoint{0.765062in}{1.331442in}}%
\pgfpathlineto{\pgfqpoint{0.773934in}{1.334112in}}%
\pgfpathlineto{\pgfqpoint{0.782806in}{1.336249in}}%
\pgfpathlineto{\pgfqpoint{0.791678in}{1.337851in}}%
\pgfpathlineto{\pgfqpoint{0.800551in}{1.338919in}}%
\pgfpathlineto{\pgfqpoint{0.809423in}{1.339453in}}%
\pgfpathlineto{\pgfqpoint{0.818295in}{1.339453in}}%
\pgfpathlineto{\pgfqpoint{0.827167in}{1.338919in}}%
\pgfpathlineto{\pgfqpoint{0.836039in}{1.337851in}}%
\pgfpathlineto{\pgfqpoint{0.844911in}{1.336249in}}%
\pgfpathlineto{\pgfqpoint{0.853783in}{1.334112in}}%
\pgfpathlineto{\pgfqpoint{0.862655in}{1.331442in}}%
\pgfpathlineto{\pgfqpoint{0.871527in}{1.328237in}}%
\pgfpathlineto{\pgfqpoint{0.880399in}{1.324499in}}%
\pgfpathlineto{\pgfqpoint{0.889271in}{1.320226in}}%
\pgfpathlineto{\pgfqpoint{0.898143in}{1.315419in}}%
\pgfpathlineto{\pgfqpoint{0.907015in}{1.310078in}}%
\pgfpathlineto{\pgfqpoint{0.915887in}{1.304203in}}%
\pgfpathlineto{\pgfqpoint{0.924759in}{1.297794in}}%
\pgfpathlineto{\pgfqpoint{0.933631in}{1.290851in}}%
\pgfpathlineto{\pgfqpoint{0.942503in}{1.283373in}}%
\pgfpathlineto{\pgfqpoint{0.951375in}{1.275362in}}%
\pgfpathlineto{\pgfqpoint{0.960248in}{1.266816in}}%
\pgfpathlineto{\pgfqpoint{0.969120in}{1.257737in}}%
\pgfpathlineto{\pgfqpoint{0.977992in}{1.248123in}}%
\pgfpathlineto{\pgfqpoint{0.986864in}{1.237975in}}%
\pgfpathlineto{\pgfqpoint{0.995736in}{1.227293in}}%
\pgfpathlineto{\pgfqpoint{1.004608in}{1.216077in}}%
\pgfpathlineto{\pgfqpoint{1.013480in}{1.204327in}}%
\pgfpathlineto{\pgfqpoint{1.022352in}{1.192043in}}%
\pgfpathlineto{\pgfqpoint{1.031224in}{1.179225in}}%
\pgfpathlineto{\pgfqpoint{1.040096in}{1.165872in}}%
\pgfpathlineto{\pgfqpoint{1.048968in}{1.151986in}}%
\pgfpathlineto{\pgfqpoint{1.057840in}{1.137565in}}%
\pgfpathlineto{\pgfqpoint{1.066712in}{1.122611in}}%
\pgfpathlineto{\pgfqpoint{1.075584in}{1.107122in}}%
\pgfpathlineto{\pgfqpoint{1.084456in}{1.091099in}}%
\pgfpathlineto{\pgfqpoint{1.093328in}{1.074542in}}%
\pgfpathlineto{\pgfqpoint{1.102200in}{1.057451in}}%
\pgfpathlineto{\pgfqpoint{1.111072in}{1.039826in}}%
\pgfpathlineto{\pgfqpoint{1.119944in}{1.021667in}}%
\pgfpathlineto{\pgfqpoint{1.128817in}{1.002973in}}%
\pgfpathlineto{\pgfqpoint{1.137689in}{0.983746in}}%
\pgfpathlineto{\pgfqpoint{1.146561in}{0.963984in}}%
\pgfpathlineto{\pgfqpoint{1.155433in}{0.943689in}}%
\pgfpathlineto{\pgfqpoint{1.164305in}{0.922859in}}%
\pgfpathlineto{\pgfqpoint{1.173177in}{0.901495in}}%
\pgfpathlineto{\pgfqpoint{1.182049in}{0.879597in}}%
\pgfpathlineto{\pgfqpoint{1.190921in}{0.857165in}}%
\pgfpathlineto{\pgfqpoint{1.199793in}{0.834199in}}%
\pgfpathlineto{\pgfqpoint{1.208665in}{0.810699in}}%
\pgfpathlineto{\pgfqpoint{1.217537in}{0.786665in}}%
\pgfpathlineto{\pgfqpoint{1.226409in}{0.762096in}}%
\pgfpathlineto{\pgfqpoint{1.235281in}{0.736994in}}%
\pgfpathlineto{\pgfqpoint{1.244153in}{0.711357in}}%
\pgfpathlineto{\pgfqpoint{1.253025in}{0.685187in}}%
\pgfusepath{stroke}%
\end{pgfscope}%
\begin{pgfscope}%
\pgfpathrectangle{\pgfqpoint{0.374692in}{0.521603in}}{\pgfqpoint{2.635000in}{1.963000in}} %
\pgfusepath{clip}%
\pgfsetrectcap%
\pgfsetroundjoin%
\pgfsetlinewidth{1.003750pt}%
\definecolor{currentstroke}{rgb}{0.172549,0.627451,0.172549}%
\pgfsetstrokecolor{currentstroke}%
\pgfsetdash{}{0pt}%
\pgfpathmoveto{\pgfqpoint{0.374692in}{0.685187in}}%
\pgfpathlineto{\pgfqpoint{0.383564in}{0.678711in}}%
\pgfpathlineto{\pgfqpoint{0.392436in}{0.672502in}}%
\pgfpathlineto{\pgfqpoint{0.401308in}{0.666560in}}%
\pgfpathlineto{\pgfqpoint{0.410180in}{0.660885in}}%
\pgfpathlineto{\pgfqpoint{0.419052in}{0.655478in}}%
\pgfpathlineto{\pgfqpoint{0.427924in}{0.650337in}}%
\pgfpathlineto{\pgfqpoint{0.436796in}{0.645463in}}%
\pgfpathlineto{\pgfqpoint{0.445668in}{0.640857in}}%
\pgfpathlineto{\pgfqpoint{0.454540in}{0.636517in}}%
\pgfpathlineto{\pgfqpoint{0.463412in}{0.632445in}}%
\pgfpathlineto{\pgfqpoint{0.472285in}{0.628639in}}%
\pgfpathlineto{\pgfqpoint{0.481157in}{0.625101in}}%
\pgfpathlineto{\pgfqpoint{0.490029in}{0.621830in}}%
\pgfpathlineto{\pgfqpoint{0.498901in}{0.618825in}}%
\pgfpathlineto{\pgfqpoint{0.507773in}{0.616088in}}%
\pgfpathlineto{\pgfqpoint{0.516645in}{0.613618in}}%
\pgfpathlineto{\pgfqpoint{0.525517in}{0.611415in}}%
\pgfpathlineto{\pgfqpoint{0.534389in}{0.609479in}}%
\pgfpathlineto{\pgfqpoint{0.543261in}{0.607810in}}%
\pgfpathlineto{\pgfqpoint{0.552133in}{0.606408in}}%
\pgfpathlineto{\pgfqpoint{0.561005in}{0.605273in}}%
\pgfpathlineto{\pgfqpoint{0.569877in}{0.604405in}}%
\pgfpathlineto{\pgfqpoint{0.578749in}{0.603804in}}%
\pgfpathlineto{\pgfqpoint{0.587621in}{0.603470in}}%
\pgfpathlineto{\pgfqpoint{0.596493in}{0.603403in}}%
\pgfpathlineto{\pgfqpoint{0.605365in}{0.603604in}}%
\pgfpathlineto{\pgfqpoint{0.614237in}{0.604071in}}%
\pgfpathlineto{\pgfqpoint{0.623109in}{0.604805in}}%
\pgfpathlineto{\pgfqpoint{0.631982in}{0.605807in}}%
\pgfpathlineto{\pgfqpoint{0.640854in}{0.607075in}}%
\pgfpathlineto{\pgfqpoint{0.649726in}{0.608611in}}%
\pgfpathlineto{\pgfqpoint{0.658598in}{0.610413in}}%
\pgfpathlineto{\pgfqpoint{0.667470in}{0.612483in}}%
\pgfpathlineto{\pgfqpoint{0.676342in}{0.614820in}}%
\pgfpathlineto{\pgfqpoint{0.685214in}{0.617423in}}%
\pgfpathlineto{\pgfqpoint{0.694086in}{0.620294in}}%
\pgfpathlineto{\pgfqpoint{0.702958in}{0.623432in}}%
\pgfpathlineto{\pgfqpoint{0.711830in}{0.626837in}}%
\pgfpathlineto{\pgfqpoint{0.720702in}{0.630509in}}%
\pgfpathlineto{\pgfqpoint{0.729574in}{0.634448in}}%
\pgfpathlineto{\pgfqpoint{0.738446in}{0.638654in}}%
\pgfpathlineto{\pgfqpoint{0.747318in}{0.643127in}}%
\pgfpathlineto{\pgfqpoint{0.756190in}{0.647867in}}%
\pgfpathlineto{\pgfqpoint{0.765062in}{0.652874in}}%
\pgfpathlineto{\pgfqpoint{0.773934in}{0.658148in}}%
\pgfpathlineto{\pgfqpoint{0.782806in}{0.663689in}}%
\pgfpathlineto{\pgfqpoint{0.791678in}{0.669498in}}%
\pgfpathlineto{\pgfqpoint{0.800551in}{0.675573in}}%
\pgfpathlineto{\pgfqpoint{0.809423in}{0.681915in}}%
\pgfpathlineto{\pgfqpoint{0.818295in}{0.688525in}}%
\pgfpathlineto{\pgfqpoint{0.827167in}{0.695401in}}%
\pgfpathlineto{\pgfqpoint{0.836039in}{0.702545in}}%
\pgfpathlineto{\pgfqpoint{0.844911in}{0.709955in}}%
\pgfpathlineto{\pgfqpoint{0.853783in}{0.717633in}}%
\pgfpathlineto{\pgfqpoint{0.862655in}{0.725578in}}%
\pgfpathlineto{\pgfqpoint{0.871527in}{0.733789in}}%
\pgfpathlineto{\pgfqpoint{0.880399in}{0.742268in}}%
\pgfpathlineto{\pgfqpoint{0.889271in}{0.751014in}}%
\pgfpathlineto{\pgfqpoint{0.898143in}{0.760027in}}%
\pgfpathlineto{\pgfqpoint{0.907015in}{0.769307in}}%
\pgfpathlineto{\pgfqpoint{0.915887in}{0.778854in}}%
\pgfpathlineto{\pgfqpoint{0.924759in}{0.788668in}}%
\pgfpathlineto{\pgfqpoint{0.933631in}{0.798749in}}%
\pgfpathlineto{\pgfqpoint{0.942503in}{0.809097in}}%
\pgfpathlineto{\pgfqpoint{0.951375in}{0.819712in}}%
\pgfpathlineto{\pgfqpoint{0.960248in}{0.830594in}}%
\pgfpathlineto{\pgfqpoint{0.969120in}{0.841743in}}%
\pgfpathlineto{\pgfqpoint{0.977992in}{0.853160in}}%
\pgfpathlineto{\pgfqpoint{0.986864in}{0.864843in}}%
\pgfpathlineto{\pgfqpoint{0.995736in}{0.876793in}}%
\pgfpathlineto{\pgfqpoint{1.004608in}{0.889011in}}%
\pgfpathlineto{\pgfqpoint{1.013480in}{0.901495in}}%
\pgfpathlineto{\pgfqpoint{1.022352in}{0.914247in}}%
\pgfpathlineto{\pgfqpoint{1.031224in}{0.927265in}}%
\pgfpathlineto{\pgfqpoint{1.040096in}{0.940551in}}%
\pgfpathlineto{\pgfqpoint{1.048968in}{0.954104in}}%
\pgfpathlineto{\pgfqpoint{1.057840in}{0.967923in}}%
\pgfpathlineto{\pgfqpoint{1.066712in}{0.982010in}}%
\pgfpathlineto{\pgfqpoint{1.075584in}{0.996364in}}%
\pgfpathlineto{\pgfqpoint{1.084456in}{1.010985in}}%
\pgfpathlineto{\pgfqpoint{1.093328in}{1.025873in}}%
\pgfpathlineto{\pgfqpoint{1.102200in}{1.041028in}}%
\pgfpathlineto{\pgfqpoint{1.111072in}{1.056450in}}%
\pgfpathlineto{\pgfqpoint{1.119944in}{1.072139in}}%
\pgfpathlineto{\pgfqpoint{1.128817in}{1.088095in}}%
\pgfpathlineto{\pgfqpoint{1.137689in}{1.104318in}}%
\pgfpathlineto{\pgfqpoint{1.146561in}{1.120808in}}%
\pgfpathlineto{\pgfqpoint{1.155433in}{1.137565in}}%
\pgfpathlineto{\pgfqpoint{1.164305in}{1.154590in}}%
\pgfpathlineto{\pgfqpoint{1.173177in}{1.171881in}}%
\pgfpathlineto{\pgfqpoint{1.182049in}{1.189439in}}%
\pgfpathlineto{\pgfqpoint{1.190921in}{1.207265in}}%
\pgfpathlineto{\pgfqpoint{1.199793in}{1.225357in}}%
\pgfpathlineto{\pgfqpoint{1.208665in}{1.243717in}}%
\pgfpathlineto{\pgfqpoint{1.217537in}{1.262343in}}%
\pgfpathlineto{\pgfqpoint{1.226409in}{1.281237in}}%
\pgfpathlineto{\pgfqpoint{1.235281in}{1.300398in}}%
\pgfpathlineto{\pgfqpoint{1.244153in}{1.319825in}}%
\pgfpathlineto{\pgfqpoint{1.253025in}{1.339520in}}%
\pgfusepath{stroke}%
\end{pgfscope}%
\begin{pgfscope}%
\pgfpathrectangle{\pgfqpoint{0.374692in}{0.521603in}}{\pgfqpoint{2.635000in}{1.963000in}} %
\pgfusepath{clip}%
\pgfsetbuttcap%
\pgfsetroundjoin%
\pgfsetlinewidth{1.505625pt}%
\definecolor{currentstroke}{rgb}{0.000000,0.000000,0.000000}%
\pgfsetstrokecolor{currentstroke}%
\pgfsetdash{{5.550000pt}{2.400000pt}}{0.000000pt}%
\pgfpathmoveto{\pgfqpoint{1.253025in}{0.521603in}}%
\pgfpathlineto{\pgfqpoint{1.253025in}{0.630659in}}%
\pgfpathlineto{\pgfqpoint{1.253025in}{0.739714in}}%
\pgfpathlineto{\pgfqpoint{1.253025in}{0.848770in}}%
\pgfpathlineto{\pgfqpoint{1.253025in}{0.957826in}}%
\pgfpathlineto{\pgfqpoint{1.253025in}{1.066881in}}%
\pgfpathlineto{\pgfqpoint{1.253025in}{1.175937in}}%
\pgfpathlineto{\pgfqpoint{1.253025in}{1.284992in}}%
\pgfpathlineto{\pgfqpoint{1.253025in}{1.394048in}}%
\pgfpathlineto{\pgfqpoint{1.253025in}{1.503103in}}%
\pgfusepath{stroke}%
\end{pgfscope}%
\begin{pgfscope}%
\pgfpathrectangle{\pgfqpoint{0.374692in}{0.521603in}}{\pgfqpoint{2.635000in}{1.963000in}} %
\pgfusepath{clip}%
\pgfsetrectcap%
\pgfsetroundjoin%
\pgfsetlinewidth{1.003750pt}%
\definecolor{currentstroke}{rgb}{0.172549,0.627451,0.172549}%
\pgfsetstrokecolor{currentstroke}%
\pgfsetdash{}{0pt}%
\pgfpathmoveto{\pgfqpoint{1.253025in}{1.339520in}}%
\pgfpathlineto{\pgfqpoint{1.261897in}{1.319825in}}%
\pgfpathlineto{\pgfqpoint{1.270769in}{1.300398in}}%
\pgfpathlineto{\pgfqpoint{1.279641in}{1.281237in}}%
\pgfpathlineto{\pgfqpoint{1.288513in}{1.262343in}}%
\pgfpathlineto{\pgfqpoint{1.297386in}{1.243717in}}%
\pgfpathlineto{\pgfqpoint{1.306258in}{1.225357in}}%
\pgfpathlineto{\pgfqpoint{1.315130in}{1.207265in}}%
\pgfpathlineto{\pgfqpoint{1.324002in}{1.189439in}}%
\pgfpathlineto{\pgfqpoint{1.332874in}{1.171881in}}%
\pgfpathlineto{\pgfqpoint{1.341746in}{1.154590in}}%
\pgfpathlineto{\pgfqpoint{1.350618in}{1.137565in}}%
\pgfpathlineto{\pgfqpoint{1.359490in}{1.120808in}}%
\pgfpathlineto{\pgfqpoint{1.368362in}{1.104318in}}%
\pgfpathlineto{\pgfqpoint{1.377234in}{1.088095in}}%
\pgfpathlineto{\pgfqpoint{1.386106in}{1.072139in}}%
\pgfpathlineto{\pgfqpoint{1.394978in}{1.056450in}}%
\pgfpathlineto{\pgfqpoint{1.403850in}{1.041028in}}%
\pgfpathlineto{\pgfqpoint{1.412722in}{1.025873in}}%
\pgfpathlineto{\pgfqpoint{1.421594in}{1.010985in}}%
\pgfpathlineto{\pgfqpoint{1.430466in}{0.996364in}}%
\pgfpathlineto{\pgfqpoint{1.439338in}{0.982010in}}%
\pgfpathlineto{\pgfqpoint{1.448210in}{0.967923in}}%
\pgfpathlineto{\pgfqpoint{1.457083in}{0.954104in}}%
\pgfpathlineto{\pgfqpoint{1.465955in}{0.940551in}}%
\pgfpathlineto{\pgfqpoint{1.474827in}{0.927265in}}%
\pgfpathlineto{\pgfqpoint{1.483699in}{0.914247in}}%
\pgfpathlineto{\pgfqpoint{1.492571in}{0.901495in}}%
\pgfpathlineto{\pgfqpoint{1.501443in}{0.889011in}}%
\pgfpathlineto{\pgfqpoint{1.510315in}{0.876793in}}%
\pgfpathlineto{\pgfqpoint{1.519187in}{0.864843in}}%
\pgfpathlineto{\pgfqpoint{1.528059in}{0.853160in}}%
\pgfpathlineto{\pgfqpoint{1.536931in}{0.841743in}}%
\pgfpathlineto{\pgfqpoint{1.545803in}{0.830594in}}%
\pgfpathlineto{\pgfqpoint{1.554675in}{0.819712in}}%
\pgfpathlineto{\pgfqpoint{1.563547in}{0.809097in}}%
\pgfpathlineto{\pgfqpoint{1.572419in}{0.798749in}}%
\pgfpathlineto{\pgfqpoint{1.581291in}{0.788668in}}%
\pgfpathlineto{\pgfqpoint{1.590163in}{0.778854in}}%
\pgfpathlineto{\pgfqpoint{1.599035in}{0.769307in}}%
\pgfpathlineto{\pgfqpoint{1.607907in}{0.760027in}}%
\pgfpathlineto{\pgfqpoint{1.616779in}{0.751014in}}%
\pgfpathlineto{\pgfqpoint{1.625652in}{0.742268in}}%
\pgfpathlineto{\pgfqpoint{1.634524in}{0.733789in}}%
\pgfpathlineto{\pgfqpoint{1.643396in}{0.725578in}}%
\pgfpathlineto{\pgfqpoint{1.652268in}{0.717633in}}%
\pgfpathlineto{\pgfqpoint{1.661140in}{0.709955in}}%
\pgfpathlineto{\pgfqpoint{1.670012in}{0.702545in}}%
\pgfpathlineto{\pgfqpoint{1.678884in}{0.695401in}}%
\pgfpathlineto{\pgfqpoint{1.687756in}{0.688525in}}%
\pgfpathlineto{\pgfqpoint{1.696628in}{0.681915in}}%
\pgfpathlineto{\pgfqpoint{1.705500in}{0.675573in}}%
\pgfpathlineto{\pgfqpoint{1.714372in}{0.669498in}}%
\pgfpathlineto{\pgfqpoint{1.723244in}{0.663689in}}%
\pgfpathlineto{\pgfqpoint{1.732116in}{0.658148in}}%
\pgfpathlineto{\pgfqpoint{1.740988in}{0.652874in}}%
\pgfpathlineto{\pgfqpoint{1.749860in}{0.647867in}}%
\pgfpathlineto{\pgfqpoint{1.758732in}{0.643127in}}%
\pgfpathlineto{\pgfqpoint{1.767604in}{0.638654in}}%
\pgfpathlineto{\pgfqpoint{1.776476in}{0.634448in}}%
\pgfpathlineto{\pgfqpoint{1.785349in}{0.630509in}}%
\pgfpathlineto{\pgfqpoint{1.794221in}{0.626837in}}%
\pgfpathlineto{\pgfqpoint{1.803093in}{0.623432in}}%
\pgfpathlineto{\pgfqpoint{1.811965in}{0.620294in}}%
\pgfpathlineto{\pgfqpoint{1.820837in}{0.617423in}}%
\pgfpathlineto{\pgfqpoint{1.829709in}{0.614820in}}%
\pgfpathlineto{\pgfqpoint{1.838581in}{0.612483in}}%
\pgfpathlineto{\pgfqpoint{1.847453in}{0.610413in}}%
\pgfpathlineto{\pgfqpoint{1.856325in}{0.608611in}}%
\pgfpathlineto{\pgfqpoint{1.865197in}{0.607075in}}%
\pgfpathlineto{\pgfqpoint{1.874069in}{0.605807in}}%
\pgfpathlineto{\pgfqpoint{1.882941in}{0.604805in}}%
\pgfpathlineto{\pgfqpoint{1.891813in}{0.604071in}}%
\pgfpathlineto{\pgfqpoint{1.900685in}{0.603604in}}%
\pgfpathlineto{\pgfqpoint{1.909557in}{0.603403in}}%
\pgfpathlineto{\pgfqpoint{1.918429in}{0.603470in}}%
\pgfpathlineto{\pgfqpoint{1.927301in}{0.603804in}}%
\pgfpathlineto{\pgfqpoint{1.936173in}{0.604405in}}%
\pgfpathlineto{\pgfqpoint{1.945045in}{0.605273in}}%
\pgfpathlineto{\pgfqpoint{1.953918in}{0.606408in}}%
\pgfpathlineto{\pgfqpoint{1.962790in}{0.607810in}}%
\pgfpathlineto{\pgfqpoint{1.971662in}{0.609479in}}%
\pgfpathlineto{\pgfqpoint{1.980534in}{0.611415in}}%
\pgfpathlineto{\pgfqpoint{1.989406in}{0.613618in}}%
\pgfpathlineto{\pgfqpoint{1.998278in}{0.616088in}}%
\pgfpathlineto{\pgfqpoint{2.007150in}{0.618825in}}%
\pgfpathlineto{\pgfqpoint{2.016022in}{0.621830in}}%
\pgfpathlineto{\pgfqpoint{2.024894in}{0.625101in}}%
\pgfpathlineto{\pgfqpoint{2.033766in}{0.628639in}}%
\pgfpathlineto{\pgfqpoint{2.042638in}{0.632445in}}%
\pgfpathlineto{\pgfqpoint{2.051510in}{0.636517in}}%
\pgfpathlineto{\pgfqpoint{2.060382in}{0.640857in}}%
\pgfpathlineto{\pgfqpoint{2.069254in}{0.645463in}}%
\pgfpathlineto{\pgfqpoint{2.078126in}{0.650337in}}%
\pgfpathlineto{\pgfqpoint{2.086998in}{0.655478in}}%
\pgfpathlineto{\pgfqpoint{2.095870in}{0.660885in}}%
\pgfpathlineto{\pgfqpoint{2.104742in}{0.666560in}}%
\pgfpathlineto{\pgfqpoint{2.113615in}{0.672502in}}%
\pgfpathlineto{\pgfqpoint{2.122487in}{0.678711in}}%
\pgfpathlineto{\pgfqpoint{2.131359in}{0.685187in}}%
\pgfusepath{stroke}%
\end{pgfscope}%
\begin{pgfscope}%
\pgfpathrectangle{\pgfqpoint{0.374692in}{0.521603in}}{\pgfqpoint{2.635000in}{1.963000in}} %
\pgfusepath{clip}%
\pgfsetrectcap%
\pgfsetroundjoin%
\pgfsetlinewidth{1.003750pt}%
\definecolor{currentstroke}{rgb}{0.839216,0.152941,0.156863}%
\pgfsetstrokecolor{currentstroke}%
\pgfsetdash{}{0pt}%
\pgfpathmoveto{\pgfqpoint{1.253025in}{0.685187in}}%
\pgfpathlineto{\pgfqpoint{1.261897in}{0.711357in}}%
\pgfpathlineto{\pgfqpoint{1.270769in}{0.736994in}}%
\pgfpathlineto{\pgfqpoint{1.279641in}{0.762096in}}%
\pgfpathlineto{\pgfqpoint{1.288513in}{0.786665in}}%
\pgfpathlineto{\pgfqpoint{1.297386in}{0.810699in}}%
\pgfpathlineto{\pgfqpoint{1.306258in}{0.834199in}}%
\pgfpathlineto{\pgfqpoint{1.315130in}{0.857165in}}%
\pgfpathlineto{\pgfqpoint{1.324002in}{0.879597in}}%
\pgfpathlineto{\pgfqpoint{1.332874in}{0.901495in}}%
\pgfpathlineto{\pgfqpoint{1.341746in}{0.922859in}}%
\pgfpathlineto{\pgfqpoint{1.350618in}{0.943689in}}%
\pgfpathlineto{\pgfqpoint{1.359490in}{0.963984in}}%
\pgfpathlineto{\pgfqpoint{1.368362in}{0.983746in}}%
\pgfpathlineto{\pgfqpoint{1.377234in}{1.002973in}}%
\pgfpathlineto{\pgfqpoint{1.386106in}{1.021667in}}%
\pgfpathlineto{\pgfqpoint{1.394978in}{1.039826in}}%
\pgfpathlineto{\pgfqpoint{1.403850in}{1.057451in}}%
\pgfpathlineto{\pgfqpoint{1.412722in}{1.074542in}}%
\pgfpathlineto{\pgfqpoint{1.421594in}{1.091099in}}%
\pgfpathlineto{\pgfqpoint{1.430466in}{1.107122in}}%
\pgfpathlineto{\pgfqpoint{1.439338in}{1.122611in}}%
\pgfpathlineto{\pgfqpoint{1.448210in}{1.137565in}}%
\pgfpathlineto{\pgfqpoint{1.457083in}{1.151986in}}%
\pgfpathlineto{\pgfqpoint{1.465955in}{1.165872in}}%
\pgfpathlineto{\pgfqpoint{1.474827in}{1.179225in}}%
\pgfpathlineto{\pgfqpoint{1.483699in}{1.192043in}}%
\pgfpathlineto{\pgfqpoint{1.492571in}{1.204327in}}%
\pgfpathlineto{\pgfqpoint{1.501443in}{1.216077in}}%
\pgfpathlineto{\pgfqpoint{1.510315in}{1.227293in}}%
\pgfpathlineto{\pgfqpoint{1.519187in}{1.237975in}}%
\pgfpathlineto{\pgfqpoint{1.528059in}{1.248123in}}%
\pgfpathlineto{\pgfqpoint{1.536931in}{1.257737in}}%
\pgfpathlineto{\pgfqpoint{1.545803in}{1.266816in}}%
\pgfpathlineto{\pgfqpoint{1.554675in}{1.275362in}}%
\pgfpathlineto{\pgfqpoint{1.563547in}{1.283373in}}%
\pgfpathlineto{\pgfqpoint{1.572419in}{1.290851in}}%
\pgfpathlineto{\pgfqpoint{1.581291in}{1.297794in}}%
\pgfpathlineto{\pgfqpoint{1.590163in}{1.304203in}}%
\pgfpathlineto{\pgfqpoint{1.599035in}{1.310078in}}%
\pgfpathlineto{\pgfqpoint{1.607907in}{1.315419in}}%
\pgfpathlineto{\pgfqpoint{1.616779in}{1.320226in}}%
\pgfpathlineto{\pgfqpoint{1.625652in}{1.324499in}}%
\pgfpathlineto{\pgfqpoint{1.634524in}{1.328237in}}%
\pgfpathlineto{\pgfqpoint{1.643396in}{1.331442in}}%
\pgfpathlineto{\pgfqpoint{1.652268in}{1.334112in}}%
\pgfpathlineto{\pgfqpoint{1.661140in}{1.336249in}}%
\pgfpathlineto{\pgfqpoint{1.670012in}{1.337851in}}%
\pgfpathlineto{\pgfqpoint{1.678884in}{1.338919in}}%
\pgfpathlineto{\pgfqpoint{1.687756in}{1.339453in}}%
\pgfpathlineto{\pgfqpoint{1.696628in}{1.339453in}}%
\pgfpathlineto{\pgfqpoint{1.705500in}{1.338919in}}%
\pgfpathlineto{\pgfqpoint{1.714372in}{1.337851in}}%
\pgfpathlineto{\pgfqpoint{1.723244in}{1.336249in}}%
\pgfpathlineto{\pgfqpoint{1.732116in}{1.334112in}}%
\pgfpathlineto{\pgfqpoint{1.740988in}{1.331442in}}%
\pgfpathlineto{\pgfqpoint{1.749860in}{1.328237in}}%
\pgfpathlineto{\pgfqpoint{1.758732in}{1.324499in}}%
\pgfpathlineto{\pgfqpoint{1.767604in}{1.320226in}}%
\pgfpathlineto{\pgfqpoint{1.776476in}{1.315419in}}%
\pgfpathlineto{\pgfqpoint{1.785349in}{1.310078in}}%
\pgfpathlineto{\pgfqpoint{1.794221in}{1.304203in}}%
\pgfpathlineto{\pgfqpoint{1.803093in}{1.297794in}}%
\pgfpathlineto{\pgfqpoint{1.811965in}{1.290851in}}%
\pgfpathlineto{\pgfqpoint{1.820837in}{1.283373in}}%
\pgfpathlineto{\pgfqpoint{1.829709in}{1.275362in}}%
\pgfpathlineto{\pgfqpoint{1.838581in}{1.266816in}}%
\pgfpathlineto{\pgfqpoint{1.847453in}{1.257737in}}%
\pgfpathlineto{\pgfqpoint{1.856325in}{1.248123in}}%
\pgfpathlineto{\pgfqpoint{1.865197in}{1.237975in}}%
\pgfpathlineto{\pgfqpoint{1.874069in}{1.227293in}}%
\pgfpathlineto{\pgfqpoint{1.882941in}{1.216077in}}%
\pgfpathlineto{\pgfqpoint{1.891813in}{1.204327in}}%
\pgfpathlineto{\pgfqpoint{1.900685in}{1.192043in}}%
\pgfpathlineto{\pgfqpoint{1.909557in}{1.179225in}}%
\pgfpathlineto{\pgfqpoint{1.918429in}{1.165872in}}%
\pgfpathlineto{\pgfqpoint{1.927301in}{1.151986in}}%
\pgfpathlineto{\pgfqpoint{1.936173in}{1.137565in}}%
\pgfpathlineto{\pgfqpoint{1.945045in}{1.122611in}}%
\pgfpathlineto{\pgfqpoint{1.953918in}{1.107122in}}%
\pgfpathlineto{\pgfqpoint{1.962790in}{1.091099in}}%
\pgfpathlineto{\pgfqpoint{1.971662in}{1.074542in}}%
\pgfpathlineto{\pgfqpoint{1.980534in}{1.057451in}}%
\pgfpathlineto{\pgfqpoint{1.989406in}{1.039826in}}%
\pgfpathlineto{\pgfqpoint{1.998278in}{1.021667in}}%
\pgfpathlineto{\pgfqpoint{2.007150in}{1.002973in}}%
\pgfpathlineto{\pgfqpoint{2.016022in}{0.983746in}}%
\pgfpathlineto{\pgfqpoint{2.024894in}{0.963984in}}%
\pgfpathlineto{\pgfqpoint{2.033766in}{0.943689in}}%
\pgfpathlineto{\pgfqpoint{2.042638in}{0.922859in}}%
\pgfpathlineto{\pgfqpoint{2.051510in}{0.901495in}}%
\pgfpathlineto{\pgfqpoint{2.060382in}{0.879597in}}%
\pgfpathlineto{\pgfqpoint{2.069254in}{0.857165in}}%
\pgfpathlineto{\pgfqpoint{2.078126in}{0.834199in}}%
\pgfpathlineto{\pgfqpoint{2.086998in}{0.810699in}}%
\pgfpathlineto{\pgfqpoint{2.095870in}{0.786665in}}%
\pgfpathlineto{\pgfqpoint{2.104742in}{0.762096in}}%
\pgfpathlineto{\pgfqpoint{2.113615in}{0.736994in}}%
\pgfpathlineto{\pgfqpoint{2.122487in}{0.711357in}}%
\pgfpathlineto{\pgfqpoint{2.131359in}{0.685187in}}%
\pgfusepath{stroke}%
\end{pgfscope}%
\begin{pgfscope}%
\pgfpathrectangle{\pgfqpoint{0.374692in}{0.521603in}}{\pgfqpoint{2.635000in}{1.963000in}} %
\pgfusepath{clip}%
\pgfsetrectcap%
\pgfsetroundjoin%
\pgfsetlinewidth{1.003750pt}%
\definecolor{currentstroke}{rgb}{0.580392,0.403922,0.741176}%
\pgfsetstrokecolor{currentstroke}%
\pgfsetdash{}{0pt}%
\pgfpathmoveto{\pgfqpoint{1.253025in}{0.685187in}}%
\pgfpathlineto{\pgfqpoint{1.261897in}{0.678711in}}%
\pgfpathlineto{\pgfqpoint{1.270769in}{0.672502in}}%
\pgfpathlineto{\pgfqpoint{1.279641in}{0.666560in}}%
\pgfpathlineto{\pgfqpoint{1.288513in}{0.660885in}}%
\pgfpathlineto{\pgfqpoint{1.297386in}{0.655478in}}%
\pgfpathlineto{\pgfqpoint{1.306258in}{0.650337in}}%
\pgfpathlineto{\pgfqpoint{1.315130in}{0.645463in}}%
\pgfpathlineto{\pgfqpoint{1.324002in}{0.640857in}}%
\pgfpathlineto{\pgfqpoint{1.332874in}{0.636517in}}%
\pgfpathlineto{\pgfqpoint{1.341746in}{0.632445in}}%
\pgfpathlineto{\pgfqpoint{1.350618in}{0.628639in}}%
\pgfpathlineto{\pgfqpoint{1.359490in}{0.625101in}}%
\pgfpathlineto{\pgfqpoint{1.368362in}{0.621830in}}%
\pgfpathlineto{\pgfqpoint{1.377234in}{0.618825in}}%
\pgfpathlineto{\pgfqpoint{1.386106in}{0.616088in}}%
\pgfpathlineto{\pgfqpoint{1.394978in}{0.613618in}}%
\pgfpathlineto{\pgfqpoint{1.403850in}{0.611415in}}%
\pgfpathlineto{\pgfqpoint{1.412722in}{0.609479in}}%
\pgfpathlineto{\pgfqpoint{1.421594in}{0.607810in}}%
\pgfpathlineto{\pgfqpoint{1.430466in}{0.606408in}}%
\pgfpathlineto{\pgfqpoint{1.439338in}{0.605273in}}%
\pgfpathlineto{\pgfqpoint{1.448210in}{0.604405in}}%
\pgfpathlineto{\pgfqpoint{1.457083in}{0.603804in}}%
\pgfpathlineto{\pgfqpoint{1.465955in}{0.603470in}}%
\pgfpathlineto{\pgfqpoint{1.474827in}{0.603403in}}%
\pgfpathlineto{\pgfqpoint{1.483699in}{0.603604in}}%
\pgfpathlineto{\pgfqpoint{1.492571in}{0.604071in}}%
\pgfpathlineto{\pgfqpoint{1.501443in}{0.604805in}}%
\pgfpathlineto{\pgfqpoint{1.510315in}{0.605807in}}%
\pgfpathlineto{\pgfqpoint{1.519187in}{0.607075in}}%
\pgfpathlineto{\pgfqpoint{1.528059in}{0.608611in}}%
\pgfpathlineto{\pgfqpoint{1.536931in}{0.610413in}}%
\pgfpathlineto{\pgfqpoint{1.545803in}{0.612483in}}%
\pgfpathlineto{\pgfqpoint{1.554675in}{0.614820in}}%
\pgfpathlineto{\pgfqpoint{1.563547in}{0.617423in}}%
\pgfpathlineto{\pgfqpoint{1.572419in}{0.620294in}}%
\pgfpathlineto{\pgfqpoint{1.581291in}{0.623432in}}%
\pgfpathlineto{\pgfqpoint{1.590163in}{0.626837in}}%
\pgfpathlineto{\pgfqpoint{1.599035in}{0.630509in}}%
\pgfpathlineto{\pgfqpoint{1.607907in}{0.634448in}}%
\pgfpathlineto{\pgfqpoint{1.616779in}{0.638654in}}%
\pgfpathlineto{\pgfqpoint{1.625652in}{0.643127in}}%
\pgfpathlineto{\pgfqpoint{1.634524in}{0.647867in}}%
\pgfpathlineto{\pgfqpoint{1.643396in}{0.652874in}}%
\pgfpathlineto{\pgfqpoint{1.652268in}{0.658148in}}%
\pgfpathlineto{\pgfqpoint{1.661140in}{0.663689in}}%
\pgfpathlineto{\pgfqpoint{1.670012in}{0.669498in}}%
\pgfpathlineto{\pgfqpoint{1.678884in}{0.675573in}}%
\pgfpathlineto{\pgfqpoint{1.687756in}{0.681915in}}%
\pgfpathlineto{\pgfqpoint{1.696628in}{0.688525in}}%
\pgfpathlineto{\pgfqpoint{1.705500in}{0.695401in}}%
\pgfpathlineto{\pgfqpoint{1.714372in}{0.702545in}}%
\pgfpathlineto{\pgfqpoint{1.723244in}{0.709955in}}%
\pgfpathlineto{\pgfqpoint{1.732116in}{0.717633in}}%
\pgfpathlineto{\pgfqpoint{1.740988in}{0.725578in}}%
\pgfpathlineto{\pgfqpoint{1.749860in}{0.733789in}}%
\pgfpathlineto{\pgfqpoint{1.758732in}{0.742268in}}%
\pgfpathlineto{\pgfqpoint{1.767604in}{0.751014in}}%
\pgfpathlineto{\pgfqpoint{1.776476in}{0.760027in}}%
\pgfpathlineto{\pgfqpoint{1.785349in}{0.769307in}}%
\pgfpathlineto{\pgfqpoint{1.794221in}{0.778854in}}%
\pgfpathlineto{\pgfqpoint{1.803093in}{0.788668in}}%
\pgfpathlineto{\pgfqpoint{1.811965in}{0.798749in}}%
\pgfpathlineto{\pgfqpoint{1.820837in}{0.809097in}}%
\pgfpathlineto{\pgfqpoint{1.829709in}{0.819712in}}%
\pgfpathlineto{\pgfqpoint{1.838581in}{0.830594in}}%
\pgfpathlineto{\pgfqpoint{1.847453in}{0.841743in}}%
\pgfpathlineto{\pgfqpoint{1.856325in}{0.853160in}}%
\pgfpathlineto{\pgfqpoint{1.865197in}{0.864843in}}%
\pgfpathlineto{\pgfqpoint{1.874069in}{0.876793in}}%
\pgfpathlineto{\pgfqpoint{1.882941in}{0.889011in}}%
\pgfpathlineto{\pgfqpoint{1.891813in}{0.901495in}}%
\pgfpathlineto{\pgfqpoint{1.900685in}{0.914247in}}%
\pgfpathlineto{\pgfqpoint{1.909557in}{0.927265in}}%
\pgfpathlineto{\pgfqpoint{1.918429in}{0.940551in}}%
\pgfpathlineto{\pgfqpoint{1.927301in}{0.954104in}}%
\pgfpathlineto{\pgfqpoint{1.936173in}{0.967923in}}%
\pgfpathlineto{\pgfqpoint{1.945045in}{0.982010in}}%
\pgfpathlineto{\pgfqpoint{1.953918in}{0.996364in}}%
\pgfpathlineto{\pgfqpoint{1.962790in}{1.010985in}}%
\pgfpathlineto{\pgfqpoint{1.971662in}{1.025873in}}%
\pgfpathlineto{\pgfqpoint{1.980534in}{1.041028in}}%
\pgfpathlineto{\pgfqpoint{1.989406in}{1.056450in}}%
\pgfpathlineto{\pgfqpoint{1.998278in}{1.072139in}}%
\pgfpathlineto{\pgfqpoint{2.007150in}{1.088095in}}%
\pgfpathlineto{\pgfqpoint{2.016022in}{1.104318in}}%
\pgfpathlineto{\pgfqpoint{2.024894in}{1.120808in}}%
\pgfpathlineto{\pgfqpoint{2.033766in}{1.137565in}}%
\pgfpathlineto{\pgfqpoint{2.042638in}{1.154590in}}%
\pgfpathlineto{\pgfqpoint{2.051510in}{1.171881in}}%
\pgfpathlineto{\pgfqpoint{2.060382in}{1.189439in}}%
\pgfpathlineto{\pgfqpoint{2.069254in}{1.207265in}}%
\pgfpathlineto{\pgfqpoint{2.078126in}{1.225357in}}%
\pgfpathlineto{\pgfqpoint{2.086998in}{1.243717in}}%
\pgfpathlineto{\pgfqpoint{2.095870in}{1.262343in}}%
\pgfpathlineto{\pgfqpoint{2.104742in}{1.281237in}}%
\pgfpathlineto{\pgfqpoint{2.113615in}{1.300398in}}%
\pgfpathlineto{\pgfqpoint{2.122487in}{1.319825in}}%
\pgfpathlineto{\pgfqpoint{2.131359in}{1.339520in}}%
\pgfusepath{stroke}%
\end{pgfscope}%
\begin{pgfscope}%
\pgfpathrectangle{\pgfqpoint{0.374692in}{0.521603in}}{\pgfqpoint{2.635000in}{1.963000in}} %
\pgfusepath{clip}%
\pgfsetrectcap%
\pgfsetroundjoin%
\pgfsetlinewidth{1.003750pt}%
\definecolor{currentstroke}{rgb}{0.580392,0.403922,0.741176}%
\pgfsetstrokecolor{currentstroke}%
\pgfsetdash{}{0pt}%
\pgfpathmoveto{\pgfqpoint{2.131359in}{1.339520in}}%
\pgfpathlineto{\pgfqpoint{2.140231in}{1.319825in}}%
\pgfpathlineto{\pgfqpoint{2.149103in}{1.300398in}}%
\pgfpathlineto{\pgfqpoint{2.157975in}{1.281237in}}%
\pgfpathlineto{\pgfqpoint{2.166847in}{1.262343in}}%
\pgfpathlineto{\pgfqpoint{2.175719in}{1.243717in}}%
\pgfpathlineto{\pgfqpoint{2.184591in}{1.225357in}}%
\pgfpathlineto{\pgfqpoint{2.193463in}{1.207265in}}%
\pgfpathlineto{\pgfqpoint{2.202335in}{1.189439in}}%
\pgfpathlineto{\pgfqpoint{2.211207in}{1.171881in}}%
\pgfpathlineto{\pgfqpoint{2.220079in}{1.154590in}}%
\pgfpathlineto{\pgfqpoint{2.228951in}{1.137565in}}%
\pgfpathlineto{\pgfqpoint{2.237823in}{1.120808in}}%
\pgfpathlineto{\pgfqpoint{2.246695in}{1.104318in}}%
\pgfpathlineto{\pgfqpoint{2.255567in}{1.088095in}}%
\pgfpathlineto{\pgfqpoint{2.264439in}{1.072139in}}%
\pgfpathlineto{\pgfqpoint{2.273311in}{1.056450in}}%
\pgfpathlineto{\pgfqpoint{2.282184in}{1.041028in}}%
\pgfpathlineto{\pgfqpoint{2.291056in}{1.025873in}}%
\pgfpathlineto{\pgfqpoint{2.299928in}{1.010985in}}%
\pgfpathlineto{\pgfqpoint{2.308800in}{0.996364in}}%
\pgfpathlineto{\pgfqpoint{2.317672in}{0.982010in}}%
\pgfpathlineto{\pgfqpoint{2.326544in}{0.967923in}}%
\pgfpathlineto{\pgfqpoint{2.335416in}{0.954104in}}%
\pgfpathlineto{\pgfqpoint{2.344288in}{0.940551in}}%
\pgfpathlineto{\pgfqpoint{2.353160in}{0.927265in}}%
\pgfpathlineto{\pgfqpoint{2.362032in}{0.914247in}}%
\pgfpathlineto{\pgfqpoint{2.370904in}{0.901495in}}%
\pgfpathlineto{\pgfqpoint{2.379776in}{0.889011in}}%
\pgfpathlineto{\pgfqpoint{2.388648in}{0.876793in}}%
\pgfpathlineto{\pgfqpoint{2.397520in}{0.864843in}}%
\pgfpathlineto{\pgfqpoint{2.406392in}{0.853160in}}%
\pgfpathlineto{\pgfqpoint{2.415264in}{0.841743in}}%
\pgfpathlineto{\pgfqpoint{2.424136in}{0.830594in}}%
\pgfpathlineto{\pgfqpoint{2.433008in}{0.819712in}}%
\pgfpathlineto{\pgfqpoint{2.441880in}{0.809097in}}%
\pgfpathlineto{\pgfqpoint{2.450753in}{0.798749in}}%
\pgfpathlineto{\pgfqpoint{2.459625in}{0.788668in}}%
\pgfpathlineto{\pgfqpoint{2.468497in}{0.778854in}}%
\pgfpathlineto{\pgfqpoint{2.477369in}{0.769307in}}%
\pgfpathlineto{\pgfqpoint{2.486241in}{0.760027in}}%
\pgfpathlineto{\pgfqpoint{2.495113in}{0.751014in}}%
\pgfpathlineto{\pgfqpoint{2.503985in}{0.742268in}}%
\pgfpathlineto{\pgfqpoint{2.512857in}{0.733789in}}%
\pgfpathlineto{\pgfqpoint{2.521729in}{0.725578in}}%
\pgfpathlineto{\pgfqpoint{2.530601in}{0.717633in}}%
\pgfpathlineto{\pgfqpoint{2.539473in}{0.709955in}}%
\pgfpathlineto{\pgfqpoint{2.548345in}{0.702545in}}%
\pgfpathlineto{\pgfqpoint{2.557217in}{0.695401in}}%
\pgfpathlineto{\pgfqpoint{2.566089in}{0.688525in}}%
\pgfpathlineto{\pgfqpoint{2.574961in}{0.681915in}}%
\pgfpathlineto{\pgfqpoint{2.583833in}{0.675573in}}%
\pgfpathlineto{\pgfqpoint{2.592705in}{0.669498in}}%
\pgfpathlineto{\pgfqpoint{2.601577in}{0.663689in}}%
\pgfpathlineto{\pgfqpoint{2.610450in}{0.658148in}}%
\pgfpathlineto{\pgfqpoint{2.619322in}{0.652874in}}%
\pgfpathlineto{\pgfqpoint{2.628194in}{0.647867in}}%
\pgfpathlineto{\pgfqpoint{2.637066in}{0.643127in}}%
\pgfpathlineto{\pgfqpoint{2.645938in}{0.638654in}}%
\pgfpathlineto{\pgfqpoint{2.654810in}{0.634448in}}%
\pgfpathlineto{\pgfqpoint{2.663682in}{0.630509in}}%
\pgfpathlineto{\pgfqpoint{2.672554in}{0.626837in}}%
\pgfpathlineto{\pgfqpoint{2.681426in}{0.623432in}}%
\pgfpathlineto{\pgfqpoint{2.690298in}{0.620294in}}%
\pgfpathlineto{\pgfqpoint{2.699170in}{0.617423in}}%
\pgfpathlineto{\pgfqpoint{2.708042in}{0.614820in}}%
\pgfpathlineto{\pgfqpoint{2.716914in}{0.612483in}}%
\pgfpathlineto{\pgfqpoint{2.725786in}{0.610413in}}%
\pgfpathlineto{\pgfqpoint{2.734658in}{0.608611in}}%
\pgfpathlineto{\pgfqpoint{2.743530in}{0.607075in}}%
\pgfpathlineto{\pgfqpoint{2.752402in}{0.605807in}}%
\pgfpathlineto{\pgfqpoint{2.761274in}{0.604805in}}%
\pgfpathlineto{\pgfqpoint{2.770146in}{0.604071in}}%
\pgfpathlineto{\pgfqpoint{2.779019in}{0.603604in}}%
\pgfpathlineto{\pgfqpoint{2.787891in}{0.603403in}}%
\pgfpathlineto{\pgfqpoint{2.796763in}{0.603470in}}%
\pgfpathlineto{\pgfqpoint{2.805635in}{0.603804in}}%
\pgfpathlineto{\pgfqpoint{2.814507in}{0.604405in}}%
\pgfpathlineto{\pgfqpoint{2.823379in}{0.605273in}}%
\pgfpathlineto{\pgfqpoint{2.832251in}{0.606408in}}%
\pgfpathlineto{\pgfqpoint{2.841123in}{0.607810in}}%
\pgfpathlineto{\pgfqpoint{2.849995in}{0.609479in}}%
\pgfpathlineto{\pgfqpoint{2.858867in}{0.611415in}}%
\pgfpathlineto{\pgfqpoint{2.867739in}{0.613618in}}%
\pgfpathlineto{\pgfqpoint{2.876611in}{0.616088in}}%
\pgfpathlineto{\pgfqpoint{2.885483in}{0.618825in}}%
\pgfpathlineto{\pgfqpoint{2.894355in}{0.621830in}}%
\pgfpathlineto{\pgfqpoint{2.903227in}{0.625101in}}%
\pgfpathlineto{\pgfqpoint{2.912099in}{0.628639in}}%
\pgfpathlineto{\pgfqpoint{2.920971in}{0.632445in}}%
\pgfpathlineto{\pgfqpoint{2.929843in}{0.636517in}}%
\pgfpathlineto{\pgfqpoint{2.938716in}{0.640857in}}%
\pgfpathlineto{\pgfqpoint{2.947588in}{0.645463in}}%
\pgfpathlineto{\pgfqpoint{2.956460in}{0.650337in}}%
\pgfpathlineto{\pgfqpoint{2.965332in}{0.655478in}}%
\pgfpathlineto{\pgfqpoint{2.974204in}{0.660885in}}%
\pgfpathlineto{\pgfqpoint{2.983076in}{0.666560in}}%
\pgfpathlineto{\pgfqpoint{2.991948in}{0.672502in}}%
\pgfpathlineto{\pgfqpoint{3.000820in}{0.678711in}}%
\pgfpathlineto{\pgfqpoint{3.009692in}{0.685187in}}%
\pgfusepath{stroke}%
\end{pgfscope}%
\begin{pgfscope}%
\pgfpathrectangle{\pgfqpoint{0.374692in}{0.521603in}}{\pgfqpoint{2.635000in}{1.963000in}} %
\pgfusepath{clip}%
\pgfsetrectcap%
\pgfsetroundjoin%
\pgfsetlinewidth{1.003750pt}%
\definecolor{currentstroke}{rgb}{0.549020,0.337255,0.294118}%
\pgfsetstrokecolor{currentstroke}%
\pgfsetdash{}{0pt}%
\pgfpathmoveto{\pgfqpoint{2.131359in}{0.685187in}}%
\pgfpathlineto{\pgfqpoint{2.140231in}{0.711357in}}%
\pgfpathlineto{\pgfqpoint{2.149103in}{0.736994in}}%
\pgfpathlineto{\pgfqpoint{2.157975in}{0.762096in}}%
\pgfpathlineto{\pgfqpoint{2.166847in}{0.786665in}}%
\pgfpathlineto{\pgfqpoint{2.175719in}{0.810699in}}%
\pgfpathlineto{\pgfqpoint{2.184591in}{0.834199in}}%
\pgfpathlineto{\pgfqpoint{2.193463in}{0.857165in}}%
\pgfpathlineto{\pgfqpoint{2.202335in}{0.879597in}}%
\pgfpathlineto{\pgfqpoint{2.211207in}{0.901495in}}%
\pgfpathlineto{\pgfqpoint{2.220079in}{0.922859in}}%
\pgfpathlineto{\pgfqpoint{2.228951in}{0.943689in}}%
\pgfpathlineto{\pgfqpoint{2.237823in}{0.963984in}}%
\pgfpathlineto{\pgfqpoint{2.246695in}{0.983746in}}%
\pgfpathlineto{\pgfqpoint{2.255567in}{1.002973in}}%
\pgfpathlineto{\pgfqpoint{2.264439in}{1.021667in}}%
\pgfpathlineto{\pgfqpoint{2.273311in}{1.039826in}}%
\pgfpathlineto{\pgfqpoint{2.282184in}{1.057451in}}%
\pgfpathlineto{\pgfqpoint{2.291056in}{1.074542in}}%
\pgfpathlineto{\pgfqpoint{2.299928in}{1.091099in}}%
\pgfpathlineto{\pgfqpoint{2.308800in}{1.107122in}}%
\pgfpathlineto{\pgfqpoint{2.317672in}{1.122611in}}%
\pgfpathlineto{\pgfqpoint{2.326544in}{1.137565in}}%
\pgfpathlineto{\pgfqpoint{2.335416in}{1.151986in}}%
\pgfpathlineto{\pgfqpoint{2.344288in}{1.165872in}}%
\pgfpathlineto{\pgfqpoint{2.353160in}{1.179225in}}%
\pgfpathlineto{\pgfqpoint{2.362032in}{1.192043in}}%
\pgfpathlineto{\pgfqpoint{2.370904in}{1.204327in}}%
\pgfpathlineto{\pgfqpoint{2.379776in}{1.216077in}}%
\pgfpathlineto{\pgfqpoint{2.388648in}{1.227293in}}%
\pgfpathlineto{\pgfqpoint{2.397520in}{1.237975in}}%
\pgfpathlineto{\pgfqpoint{2.406392in}{1.248123in}}%
\pgfpathlineto{\pgfqpoint{2.415264in}{1.257737in}}%
\pgfpathlineto{\pgfqpoint{2.424136in}{1.266816in}}%
\pgfpathlineto{\pgfqpoint{2.433008in}{1.275362in}}%
\pgfpathlineto{\pgfqpoint{2.441880in}{1.283373in}}%
\pgfpathlineto{\pgfqpoint{2.450753in}{1.290851in}}%
\pgfpathlineto{\pgfqpoint{2.459625in}{1.297794in}}%
\pgfpathlineto{\pgfqpoint{2.468497in}{1.304203in}}%
\pgfpathlineto{\pgfqpoint{2.477369in}{1.310078in}}%
\pgfpathlineto{\pgfqpoint{2.486241in}{1.315419in}}%
\pgfpathlineto{\pgfqpoint{2.495113in}{1.320226in}}%
\pgfpathlineto{\pgfqpoint{2.503985in}{1.324499in}}%
\pgfpathlineto{\pgfqpoint{2.512857in}{1.328237in}}%
\pgfpathlineto{\pgfqpoint{2.521729in}{1.331442in}}%
\pgfpathlineto{\pgfqpoint{2.530601in}{1.334112in}}%
\pgfpathlineto{\pgfqpoint{2.539473in}{1.336249in}}%
\pgfpathlineto{\pgfqpoint{2.548345in}{1.337851in}}%
\pgfpathlineto{\pgfqpoint{2.557217in}{1.338919in}}%
\pgfpathlineto{\pgfqpoint{2.566089in}{1.339453in}}%
\pgfpathlineto{\pgfqpoint{2.574961in}{1.339453in}}%
\pgfpathlineto{\pgfqpoint{2.583833in}{1.338919in}}%
\pgfpathlineto{\pgfqpoint{2.592705in}{1.337851in}}%
\pgfpathlineto{\pgfqpoint{2.601577in}{1.336249in}}%
\pgfpathlineto{\pgfqpoint{2.610450in}{1.334112in}}%
\pgfpathlineto{\pgfqpoint{2.619322in}{1.331442in}}%
\pgfpathlineto{\pgfqpoint{2.628194in}{1.328237in}}%
\pgfpathlineto{\pgfqpoint{2.637066in}{1.324499in}}%
\pgfpathlineto{\pgfqpoint{2.645938in}{1.320226in}}%
\pgfpathlineto{\pgfqpoint{2.654810in}{1.315419in}}%
\pgfpathlineto{\pgfqpoint{2.663682in}{1.310078in}}%
\pgfpathlineto{\pgfqpoint{2.672554in}{1.304203in}}%
\pgfpathlineto{\pgfqpoint{2.681426in}{1.297794in}}%
\pgfpathlineto{\pgfqpoint{2.690298in}{1.290851in}}%
\pgfpathlineto{\pgfqpoint{2.699170in}{1.283373in}}%
\pgfpathlineto{\pgfqpoint{2.708042in}{1.275362in}}%
\pgfpathlineto{\pgfqpoint{2.716914in}{1.266816in}}%
\pgfpathlineto{\pgfqpoint{2.725786in}{1.257737in}}%
\pgfpathlineto{\pgfqpoint{2.734658in}{1.248123in}}%
\pgfpathlineto{\pgfqpoint{2.743530in}{1.237975in}}%
\pgfpathlineto{\pgfqpoint{2.752402in}{1.227293in}}%
\pgfpathlineto{\pgfqpoint{2.761274in}{1.216077in}}%
\pgfpathlineto{\pgfqpoint{2.770146in}{1.204327in}}%
\pgfpathlineto{\pgfqpoint{2.779019in}{1.192043in}}%
\pgfpathlineto{\pgfqpoint{2.787891in}{1.179225in}}%
\pgfpathlineto{\pgfqpoint{2.796763in}{1.165872in}}%
\pgfpathlineto{\pgfqpoint{2.805635in}{1.151986in}}%
\pgfpathlineto{\pgfqpoint{2.814507in}{1.137565in}}%
\pgfpathlineto{\pgfqpoint{2.823379in}{1.122611in}}%
\pgfpathlineto{\pgfqpoint{2.832251in}{1.107122in}}%
\pgfpathlineto{\pgfqpoint{2.841123in}{1.091099in}}%
\pgfpathlineto{\pgfqpoint{2.849995in}{1.074542in}}%
\pgfpathlineto{\pgfqpoint{2.858867in}{1.057451in}}%
\pgfpathlineto{\pgfqpoint{2.867739in}{1.039826in}}%
\pgfpathlineto{\pgfqpoint{2.876611in}{1.021667in}}%
\pgfpathlineto{\pgfqpoint{2.885483in}{1.002973in}}%
\pgfpathlineto{\pgfqpoint{2.894355in}{0.983746in}}%
\pgfpathlineto{\pgfqpoint{2.903227in}{0.963984in}}%
\pgfpathlineto{\pgfqpoint{2.912099in}{0.943689in}}%
\pgfpathlineto{\pgfqpoint{2.920971in}{0.922859in}}%
\pgfpathlineto{\pgfqpoint{2.929843in}{0.901495in}}%
\pgfpathlineto{\pgfqpoint{2.938716in}{0.879597in}}%
\pgfpathlineto{\pgfqpoint{2.947588in}{0.857165in}}%
\pgfpathlineto{\pgfqpoint{2.956460in}{0.834199in}}%
\pgfpathlineto{\pgfqpoint{2.965332in}{0.810699in}}%
\pgfpathlineto{\pgfqpoint{2.974204in}{0.786665in}}%
\pgfpathlineto{\pgfqpoint{2.983076in}{0.762096in}}%
\pgfpathlineto{\pgfqpoint{2.991948in}{0.736994in}}%
\pgfpathlineto{\pgfqpoint{3.000820in}{0.711357in}}%
\pgfpathlineto{\pgfqpoint{3.009692in}{0.685187in}}%
\pgfusepath{stroke}%
\end{pgfscope}%
\begin{pgfscope}%
\pgfpathrectangle{\pgfqpoint{0.374692in}{0.521603in}}{\pgfqpoint{2.635000in}{1.963000in}} %
\pgfusepath{clip}%
\pgfsetrectcap%
\pgfsetroundjoin%
\pgfsetlinewidth{1.003750pt}%
\definecolor{currentstroke}{rgb}{0.121569,0.466667,0.705882}%
\pgfsetstrokecolor{currentstroke}%
\pgfsetdash{}{0pt}%
\pgfpathmoveto{\pgfqpoint{2.131359in}{0.685187in}}%
\pgfpathlineto{\pgfqpoint{2.140231in}{0.678711in}}%
\pgfpathlineto{\pgfqpoint{2.149103in}{0.672502in}}%
\pgfpathlineto{\pgfqpoint{2.157975in}{0.666560in}}%
\pgfpathlineto{\pgfqpoint{2.166847in}{0.660885in}}%
\pgfpathlineto{\pgfqpoint{2.175719in}{0.655478in}}%
\pgfpathlineto{\pgfqpoint{2.184591in}{0.650337in}}%
\pgfpathlineto{\pgfqpoint{2.193463in}{0.645463in}}%
\pgfpathlineto{\pgfqpoint{2.202335in}{0.640857in}}%
\pgfpathlineto{\pgfqpoint{2.211207in}{0.636517in}}%
\pgfpathlineto{\pgfqpoint{2.220079in}{0.632445in}}%
\pgfpathlineto{\pgfqpoint{2.228951in}{0.628639in}}%
\pgfpathlineto{\pgfqpoint{2.237823in}{0.625101in}}%
\pgfpathlineto{\pgfqpoint{2.246695in}{0.621830in}}%
\pgfpathlineto{\pgfqpoint{2.255567in}{0.618825in}}%
\pgfpathlineto{\pgfqpoint{2.264439in}{0.616088in}}%
\pgfpathlineto{\pgfqpoint{2.273311in}{0.613618in}}%
\pgfpathlineto{\pgfqpoint{2.282184in}{0.611415in}}%
\pgfpathlineto{\pgfqpoint{2.291056in}{0.609479in}}%
\pgfpathlineto{\pgfqpoint{2.299928in}{0.607810in}}%
\pgfpathlineto{\pgfqpoint{2.308800in}{0.606408in}}%
\pgfpathlineto{\pgfqpoint{2.317672in}{0.605273in}}%
\pgfpathlineto{\pgfqpoint{2.326544in}{0.604405in}}%
\pgfpathlineto{\pgfqpoint{2.335416in}{0.603804in}}%
\pgfpathlineto{\pgfqpoint{2.344288in}{0.603470in}}%
\pgfpathlineto{\pgfqpoint{2.353160in}{0.603403in}}%
\pgfpathlineto{\pgfqpoint{2.362032in}{0.603604in}}%
\pgfpathlineto{\pgfqpoint{2.370904in}{0.604071in}}%
\pgfpathlineto{\pgfqpoint{2.379776in}{0.604805in}}%
\pgfpathlineto{\pgfqpoint{2.388648in}{0.605807in}}%
\pgfpathlineto{\pgfqpoint{2.397520in}{0.607075in}}%
\pgfpathlineto{\pgfqpoint{2.406392in}{0.608611in}}%
\pgfpathlineto{\pgfqpoint{2.415264in}{0.610413in}}%
\pgfpathlineto{\pgfqpoint{2.424136in}{0.612483in}}%
\pgfpathlineto{\pgfqpoint{2.433008in}{0.614820in}}%
\pgfpathlineto{\pgfqpoint{2.441880in}{0.617423in}}%
\pgfpathlineto{\pgfqpoint{2.450753in}{0.620294in}}%
\pgfpathlineto{\pgfqpoint{2.459625in}{0.623432in}}%
\pgfpathlineto{\pgfqpoint{2.468497in}{0.626837in}}%
\pgfpathlineto{\pgfqpoint{2.477369in}{0.630509in}}%
\pgfpathlineto{\pgfqpoint{2.486241in}{0.634448in}}%
\pgfpathlineto{\pgfqpoint{2.495113in}{0.638654in}}%
\pgfpathlineto{\pgfqpoint{2.503985in}{0.643127in}}%
\pgfpathlineto{\pgfqpoint{2.512857in}{0.647867in}}%
\pgfpathlineto{\pgfqpoint{2.521729in}{0.652874in}}%
\pgfpathlineto{\pgfqpoint{2.530601in}{0.658148in}}%
\pgfpathlineto{\pgfqpoint{2.539473in}{0.663689in}}%
\pgfpathlineto{\pgfqpoint{2.548345in}{0.669498in}}%
\pgfpathlineto{\pgfqpoint{2.557217in}{0.675573in}}%
\pgfpathlineto{\pgfqpoint{2.566089in}{0.681915in}}%
\pgfpathlineto{\pgfqpoint{2.574961in}{0.688525in}}%
\pgfpathlineto{\pgfqpoint{2.583833in}{0.695401in}}%
\pgfpathlineto{\pgfqpoint{2.592705in}{0.702545in}}%
\pgfpathlineto{\pgfqpoint{2.601577in}{0.709955in}}%
\pgfpathlineto{\pgfqpoint{2.610450in}{0.717633in}}%
\pgfpathlineto{\pgfqpoint{2.619322in}{0.725578in}}%
\pgfpathlineto{\pgfqpoint{2.628194in}{0.733789in}}%
\pgfpathlineto{\pgfqpoint{2.637066in}{0.742268in}}%
\pgfpathlineto{\pgfqpoint{2.645938in}{0.751014in}}%
\pgfpathlineto{\pgfqpoint{2.654810in}{0.760027in}}%
\pgfpathlineto{\pgfqpoint{2.663682in}{0.769307in}}%
\pgfpathlineto{\pgfqpoint{2.672554in}{0.778854in}}%
\pgfpathlineto{\pgfqpoint{2.681426in}{0.788668in}}%
\pgfpathlineto{\pgfqpoint{2.690298in}{0.798749in}}%
\pgfpathlineto{\pgfqpoint{2.699170in}{0.809097in}}%
\pgfpathlineto{\pgfqpoint{2.708042in}{0.819712in}}%
\pgfpathlineto{\pgfqpoint{2.716914in}{0.830594in}}%
\pgfpathlineto{\pgfqpoint{2.725786in}{0.841743in}}%
\pgfpathlineto{\pgfqpoint{2.734658in}{0.853160in}}%
\pgfpathlineto{\pgfqpoint{2.743530in}{0.864843in}}%
\pgfpathlineto{\pgfqpoint{2.752402in}{0.876793in}}%
\pgfpathlineto{\pgfqpoint{2.761274in}{0.889011in}}%
\pgfpathlineto{\pgfqpoint{2.770146in}{0.901495in}}%
\pgfpathlineto{\pgfqpoint{2.779019in}{0.914247in}}%
\pgfpathlineto{\pgfqpoint{2.787891in}{0.927265in}}%
\pgfpathlineto{\pgfqpoint{2.796763in}{0.940551in}}%
\pgfpathlineto{\pgfqpoint{2.805635in}{0.954104in}}%
\pgfpathlineto{\pgfqpoint{2.814507in}{0.967923in}}%
\pgfpathlineto{\pgfqpoint{2.823379in}{0.982010in}}%
\pgfpathlineto{\pgfqpoint{2.832251in}{0.996364in}}%
\pgfpathlineto{\pgfqpoint{2.841123in}{1.010985in}}%
\pgfpathlineto{\pgfqpoint{2.849995in}{1.025873in}}%
\pgfpathlineto{\pgfqpoint{2.858867in}{1.041028in}}%
\pgfpathlineto{\pgfqpoint{2.867739in}{1.056450in}}%
\pgfpathlineto{\pgfqpoint{2.876611in}{1.072139in}}%
\pgfpathlineto{\pgfqpoint{2.885483in}{1.088095in}}%
\pgfpathlineto{\pgfqpoint{2.894355in}{1.104318in}}%
\pgfpathlineto{\pgfqpoint{2.903227in}{1.120808in}}%
\pgfpathlineto{\pgfqpoint{2.912099in}{1.137565in}}%
\pgfpathlineto{\pgfqpoint{2.920971in}{1.154590in}}%
\pgfpathlineto{\pgfqpoint{2.929843in}{1.171881in}}%
\pgfpathlineto{\pgfqpoint{2.938716in}{1.189439in}}%
\pgfpathlineto{\pgfqpoint{2.947588in}{1.207265in}}%
\pgfpathlineto{\pgfqpoint{2.956460in}{1.225357in}}%
\pgfpathlineto{\pgfqpoint{2.965332in}{1.243717in}}%
\pgfpathlineto{\pgfqpoint{2.974204in}{1.262343in}}%
\pgfpathlineto{\pgfqpoint{2.983076in}{1.281237in}}%
\pgfpathlineto{\pgfqpoint{2.991948in}{1.300398in}}%
\pgfpathlineto{\pgfqpoint{3.000820in}{1.319825in}}%
\pgfpathlineto{\pgfqpoint{3.009692in}{1.339520in}}%
\pgfusepath{stroke}%
\end{pgfscope}%
\begin{pgfscope}%
\pgfpathrectangle{\pgfqpoint{0.374692in}{0.521603in}}{\pgfqpoint{2.635000in}{1.963000in}} %
\pgfusepath{clip}%
\pgfsetbuttcap%
\pgfsetroundjoin%
\pgfsetlinewidth{1.505625pt}%
\definecolor{currentstroke}{rgb}{0.000000,0.000000,0.000000}%
\pgfsetstrokecolor{currentstroke}%
\pgfsetdash{{5.550000pt}{2.400000pt}}{0.000000pt}%
\pgfpathmoveto{\pgfqpoint{2.131359in}{0.521603in}}%
\pgfpathlineto{\pgfqpoint{2.131359in}{0.630659in}}%
\pgfpathlineto{\pgfqpoint{2.131359in}{0.739714in}}%
\pgfpathlineto{\pgfqpoint{2.131359in}{0.848770in}}%
\pgfpathlineto{\pgfqpoint{2.131359in}{0.957826in}}%
\pgfpathlineto{\pgfqpoint{2.131359in}{1.066881in}}%
\pgfpathlineto{\pgfqpoint{2.131359in}{1.175937in}}%
\pgfpathlineto{\pgfqpoint{2.131359in}{1.284992in}}%
\pgfpathlineto{\pgfqpoint{2.131359in}{1.394048in}}%
\pgfpathlineto{\pgfqpoint{2.131359in}{1.503103in}}%
\pgfusepath{stroke}%
\end{pgfscope}%
\begin{pgfscope}%
\pgfpathrectangle{\pgfqpoint{0.374692in}{0.521603in}}{\pgfqpoint{2.635000in}{1.963000in}} %
\pgfusepath{clip}%
\pgfsetbuttcap%
\pgfsetroundjoin%
\definecolor{currentfill}{rgb}{1.000000,0.000000,0.000000}%
\pgfsetfillcolor{currentfill}%
\pgfsetlinewidth{1.003750pt}%
\definecolor{currentstroke}{rgb}{1.000000,0.000000,0.000000}%
\pgfsetstrokecolor{currentstroke}%
\pgfsetdash{}{0pt}%
\pgfsys@defobject{currentmarker}{\pgfqpoint{-0.020833in}{-0.020833in}}{\pgfqpoint{0.020833in}{0.020833in}}{%
\pgfpathmoveto{\pgfqpoint{0.000000in}{-0.020833in}}%
\pgfpathcurveto{\pgfqpoint{0.005525in}{-0.020833in}}{\pgfqpoint{0.010825in}{-0.018638in}}{\pgfqpoint{0.014731in}{-0.014731in}}%
\pgfpathcurveto{\pgfqpoint{0.018638in}{-0.010825in}}{\pgfqpoint{0.020833in}{-0.005525in}}{\pgfqpoint{0.020833in}{0.000000in}}%
\pgfpathcurveto{\pgfqpoint{0.020833in}{0.005525in}}{\pgfqpoint{0.018638in}{0.010825in}}{\pgfqpoint{0.014731in}{0.014731in}}%
\pgfpathcurveto{\pgfqpoint{0.010825in}{0.018638in}}{\pgfqpoint{0.005525in}{0.020833in}}{\pgfqpoint{0.000000in}{0.020833in}}%
\pgfpathcurveto{\pgfqpoint{-0.005525in}{0.020833in}}{\pgfqpoint{-0.010825in}{0.018638in}}{\pgfqpoint{-0.014731in}{0.014731in}}%
\pgfpathcurveto{\pgfqpoint{-0.018638in}{0.010825in}}{\pgfqpoint{-0.020833in}{0.005525in}}{\pgfqpoint{-0.020833in}{0.000000in}}%
\pgfpathcurveto{\pgfqpoint{-0.020833in}{-0.005525in}}{\pgfqpoint{-0.018638in}{-0.010825in}}{\pgfqpoint{-0.014731in}{-0.014731in}}%
\pgfpathcurveto{\pgfqpoint{-0.010825in}{-0.018638in}}{\pgfqpoint{-0.005525in}{-0.020833in}}{\pgfqpoint{0.000000in}{-0.020833in}}%
\pgfpathclose%
\pgfusepath{stroke,fill}%
}%
\begin{pgfscope}%
\pgfsys@transformshift{0.374692in}{0.685187in}%
\pgfsys@useobject{currentmarker}{}%
\end{pgfscope}%
\begin{pgfscope}%
\pgfsys@transformshift{0.813859in}{0.685187in}%
\pgfsys@useobject{currentmarker}{}%
\end{pgfscope}%
\begin{pgfscope}%
\pgfsys@transformshift{1.253025in}{0.685187in}%
\pgfsys@useobject{currentmarker}{}%
\end{pgfscope}%
\end{pgfscope}%
\begin{pgfscope}%
\pgfpathrectangle{\pgfqpoint{0.374692in}{0.521603in}}{\pgfqpoint{2.635000in}{1.963000in}} %
\pgfusepath{clip}%
\pgfsetbuttcap%
\pgfsetroundjoin%
\definecolor{currentfill}{rgb}{1.000000,0.000000,0.000000}%
\pgfsetfillcolor{currentfill}%
\pgfsetlinewidth{1.003750pt}%
\definecolor{currentstroke}{rgb}{1.000000,0.000000,0.000000}%
\pgfsetstrokecolor{currentstroke}%
\pgfsetdash{}{0pt}%
\pgfsys@defobject{currentmarker}{\pgfqpoint{-0.020833in}{-0.020833in}}{\pgfqpoint{0.020833in}{0.020833in}}{%
\pgfpathmoveto{\pgfqpoint{0.000000in}{-0.020833in}}%
\pgfpathcurveto{\pgfqpoint{0.005525in}{-0.020833in}}{\pgfqpoint{0.010825in}{-0.018638in}}{\pgfqpoint{0.014731in}{-0.014731in}}%
\pgfpathcurveto{\pgfqpoint{0.018638in}{-0.010825in}}{\pgfqpoint{0.020833in}{-0.005525in}}{\pgfqpoint{0.020833in}{0.000000in}}%
\pgfpathcurveto{\pgfqpoint{0.020833in}{0.005525in}}{\pgfqpoint{0.018638in}{0.010825in}}{\pgfqpoint{0.014731in}{0.014731in}}%
\pgfpathcurveto{\pgfqpoint{0.010825in}{0.018638in}}{\pgfqpoint{0.005525in}{0.020833in}}{\pgfqpoint{0.000000in}{0.020833in}}%
\pgfpathcurveto{\pgfqpoint{-0.005525in}{0.020833in}}{\pgfqpoint{-0.010825in}{0.018638in}}{\pgfqpoint{-0.014731in}{0.014731in}}%
\pgfpathcurveto{\pgfqpoint{-0.018638in}{0.010825in}}{\pgfqpoint{-0.020833in}{0.005525in}}{\pgfqpoint{-0.020833in}{0.000000in}}%
\pgfpathcurveto{\pgfqpoint{-0.020833in}{-0.005525in}}{\pgfqpoint{-0.018638in}{-0.010825in}}{\pgfqpoint{-0.014731in}{-0.014731in}}%
\pgfpathcurveto{\pgfqpoint{-0.010825in}{-0.018638in}}{\pgfqpoint{-0.005525in}{-0.020833in}}{\pgfqpoint{0.000000in}{-0.020833in}}%
\pgfpathclose%
\pgfusepath{stroke,fill}%
}%
\begin{pgfscope}%
\pgfsys@transformshift{0.374692in}{0.685187in}%
\pgfsys@useobject{currentmarker}{}%
\end{pgfscope}%
\begin{pgfscope}%
\pgfsys@transformshift{0.813859in}{0.685187in}%
\pgfsys@useobject{currentmarker}{}%
\end{pgfscope}%
\begin{pgfscope}%
\pgfsys@transformshift{1.253025in}{0.685187in}%
\pgfsys@useobject{currentmarker}{}%
\end{pgfscope}%
\end{pgfscope}%
\begin{pgfscope}%
\pgfpathrectangle{\pgfqpoint{0.374692in}{0.521603in}}{\pgfqpoint{2.635000in}{1.963000in}} %
\pgfusepath{clip}%
\pgfsetbuttcap%
\pgfsetroundjoin%
\definecolor{currentfill}{rgb}{1.000000,0.000000,0.000000}%
\pgfsetfillcolor{currentfill}%
\pgfsetlinewidth{1.003750pt}%
\definecolor{currentstroke}{rgb}{1.000000,0.000000,0.000000}%
\pgfsetstrokecolor{currentstroke}%
\pgfsetdash{}{0pt}%
\pgfsys@defobject{currentmarker}{\pgfqpoint{-0.020833in}{-0.020833in}}{\pgfqpoint{0.020833in}{0.020833in}}{%
\pgfpathmoveto{\pgfqpoint{0.000000in}{-0.020833in}}%
\pgfpathcurveto{\pgfqpoint{0.005525in}{-0.020833in}}{\pgfqpoint{0.010825in}{-0.018638in}}{\pgfqpoint{0.014731in}{-0.014731in}}%
\pgfpathcurveto{\pgfqpoint{0.018638in}{-0.010825in}}{\pgfqpoint{0.020833in}{-0.005525in}}{\pgfqpoint{0.020833in}{0.000000in}}%
\pgfpathcurveto{\pgfqpoint{0.020833in}{0.005525in}}{\pgfqpoint{0.018638in}{0.010825in}}{\pgfqpoint{0.014731in}{0.014731in}}%
\pgfpathcurveto{\pgfqpoint{0.010825in}{0.018638in}}{\pgfqpoint{0.005525in}{0.020833in}}{\pgfqpoint{0.000000in}{0.020833in}}%
\pgfpathcurveto{\pgfqpoint{-0.005525in}{0.020833in}}{\pgfqpoint{-0.010825in}{0.018638in}}{\pgfqpoint{-0.014731in}{0.014731in}}%
\pgfpathcurveto{\pgfqpoint{-0.018638in}{0.010825in}}{\pgfqpoint{-0.020833in}{0.005525in}}{\pgfqpoint{-0.020833in}{0.000000in}}%
\pgfpathcurveto{\pgfqpoint{-0.020833in}{-0.005525in}}{\pgfqpoint{-0.018638in}{-0.010825in}}{\pgfqpoint{-0.014731in}{-0.014731in}}%
\pgfpathcurveto{\pgfqpoint{-0.010825in}{-0.018638in}}{\pgfqpoint{-0.005525in}{-0.020833in}}{\pgfqpoint{0.000000in}{-0.020833in}}%
\pgfpathclose%
\pgfusepath{stroke,fill}%
}%
\begin{pgfscope}%
\pgfsys@transformshift{1.253025in}{0.685187in}%
\pgfsys@useobject{currentmarker}{}%
\end{pgfscope}%
\begin{pgfscope}%
\pgfsys@transformshift{1.692192in}{0.685187in}%
\pgfsys@useobject{currentmarker}{}%
\end{pgfscope}%
\begin{pgfscope}%
\pgfsys@transformshift{2.131359in}{0.685187in}%
\pgfsys@useobject{currentmarker}{}%
\end{pgfscope}%
\end{pgfscope}%
\begin{pgfscope}%
\pgfpathrectangle{\pgfqpoint{0.374692in}{0.521603in}}{\pgfqpoint{2.635000in}{1.963000in}} %
\pgfusepath{clip}%
\pgfsetbuttcap%
\pgfsetroundjoin%
\definecolor{currentfill}{rgb}{1.000000,0.000000,0.000000}%
\pgfsetfillcolor{currentfill}%
\pgfsetlinewidth{1.003750pt}%
\definecolor{currentstroke}{rgb}{1.000000,0.000000,0.000000}%
\pgfsetstrokecolor{currentstroke}%
\pgfsetdash{}{0pt}%
\pgfsys@defobject{currentmarker}{\pgfqpoint{-0.020833in}{-0.020833in}}{\pgfqpoint{0.020833in}{0.020833in}}{%
\pgfpathmoveto{\pgfqpoint{0.000000in}{-0.020833in}}%
\pgfpathcurveto{\pgfqpoint{0.005525in}{-0.020833in}}{\pgfqpoint{0.010825in}{-0.018638in}}{\pgfqpoint{0.014731in}{-0.014731in}}%
\pgfpathcurveto{\pgfqpoint{0.018638in}{-0.010825in}}{\pgfqpoint{0.020833in}{-0.005525in}}{\pgfqpoint{0.020833in}{0.000000in}}%
\pgfpathcurveto{\pgfqpoint{0.020833in}{0.005525in}}{\pgfqpoint{0.018638in}{0.010825in}}{\pgfqpoint{0.014731in}{0.014731in}}%
\pgfpathcurveto{\pgfqpoint{0.010825in}{0.018638in}}{\pgfqpoint{0.005525in}{0.020833in}}{\pgfqpoint{0.000000in}{0.020833in}}%
\pgfpathcurveto{\pgfqpoint{-0.005525in}{0.020833in}}{\pgfqpoint{-0.010825in}{0.018638in}}{\pgfqpoint{-0.014731in}{0.014731in}}%
\pgfpathcurveto{\pgfqpoint{-0.018638in}{0.010825in}}{\pgfqpoint{-0.020833in}{0.005525in}}{\pgfqpoint{-0.020833in}{0.000000in}}%
\pgfpathcurveto{\pgfqpoint{-0.020833in}{-0.005525in}}{\pgfqpoint{-0.018638in}{-0.010825in}}{\pgfqpoint{-0.014731in}{-0.014731in}}%
\pgfpathcurveto{\pgfqpoint{-0.010825in}{-0.018638in}}{\pgfqpoint{-0.005525in}{-0.020833in}}{\pgfqpoint{0.000000in}{-0.020833in}}%
\pgfpathclose%
\pgfusepath{stroke,fill}%
}%
\begin{pgfscope}%
\pgfsys@transformshift{1.253025in}{0.685187in}%
\pgfsys@useobject{currentmarker}{}%
\end{pgfscope}%
\begin{pgfscope}%
\pgfsys@transformshift{1.692192in}{0.685187in}%
\pgfsys@useobject{currentmarker}{}%
\end{pgfscope}%
\begin{pgfscope}%
\pgfsys@transformshift{2.131359in}{0.685187in}%
\pgfsys@useobject{currentmarker}{}%
\end{pgfscope}%
\end{pgfscope}%
\begin{pgfscope}%
\pgfpathrectangle{\pgfqpoint{0.374692in}{0.521603in}}{\pgfqpoint{2.635000in}{1.963000in}} %
\pgfusepath{clip}%
\pgfsetbuttcap%
\pgfsetroundjoin%
\definecolor{currentfill}{rgb}{0.000000,0.000000,0.000000}%
\pgfsetfillcolor{currentfill}%
\pgfsetlinewidth{1.003750pt}%
\definecolor{currentstroke}{rgb}{0.000000,0.000000,0.000000}%
\pgfsetstrokecolor{currentstroke}%
\pgfsetdash{}{0pt}%
\pgfsys@defobject{currentmarker}{\pgfqpoint{-0.020833in}{-0.020833in}}{\pgfqpoint{0.020833in}{0.020833in}}{%
\pgfpathmoveto{\pgfqpoint{0.000000in}{-0.020833in}}%
\pgfpathcurveto{\pgfqpoint{0.005525in}{-0.020833in}}{\pgfqpoint{0.010825in}{-0.018638in}}{\pgfqpoint{0.014731in}{-0.014731in}}%
\pgfpathcurveto{\pgfqpoint{0.018638in}{-0.010825in}}{\pgfqpoint{0.020833in}{-0.005525in}}{\pgfqpoint{0.020833in}{0.000000in}}%
\pgfpathcurveto{\pgfqpoint{0.020833in}{0.005525in}}{\pgfqpoint{0.018638in}{0.010825in}}{\pgfqpoint{0.014731in}{0.014731in}}%
\pgfpathcurveto{\pgfqpoint{0.010825in}{0.018638in}}{\pgfqpoint{0.005525in}{0.020833in}}{\pgfqpoint{0.000000in}{0.020833in}}%
\pgfpathcurveto{\pgfqpoint{-0.005525in}{0.020833in}}{\pgfqpoint{-0.010825in}{0.018638in}}{\pgfqpoint{-0.014731in}{0.014731in}}%
\pgfpathcurveto{\pgfqpoint{-0.018638in}{0.010825in}}{\pgfqpoint{-0.020833in}{0.005525in}}{\pgfqpoint{-0.020833in}{0.000000in}}%
\pgfpathcurveto{\pgfqpoint{-0.020833in}{-0.005525in}}{\pgfqpoint{-0.018638in}{-0.010825in}}{\pgfqpoint{-0.014731in}{-0.014731in}}%
\pgfpathcurveto{\pgfqpoint{-0.010825in}{-0.018638in}}{\pgfqpoint{-0.005525in}{-0.020833in}}{\pgfqpoint{0.000000in}{-0.020833in}}%
\pgfpathclose%
\pgfusepath{stroke,fill}%
}%
\begin{pgfscope}%
\pgfsys@transformshift{2.131359in}{0.685187in}%
\pgfsys@useobject{currentmarker}{}%
\end{pgfscope}%
\begin{pgfscope}%
\pgfsys@transformshift{2.570525in}{0.685187in}%
\pgfsys@useobject{currentmarker}{}%
\end{pgfscope}%
\end{pgfscope}%
\begin{pgfscope}%
\pgfpathrectangle{\pgfqpoint{0.374692in}{0.521603in}}{\pgfqpoint{2.635000in}{1.963000in}} %
\pgfusepath{clip}%
\pgfsetbuttcap%
\pgfsetroundjoin%
\definecolor{currentfill}{rgb}{1.000000,0.000000,0.000000}%
\pgfsetfillcolor{currentfill}%
\pgfsetlinewidth{1.003750pt}%
\definecolor{currentstroke}{rgb}{1.000000,0.000000,0.000000}%
\pgfsetstrokecolor{currentstroke}%
\pgfsetdash{}{0pt}%
\pgfsys@defobject{currentmarker}{\pgfqpoint{-0.020833in}{-0.020833in}}{\pgfqpoint{0.020833in}{0.020833in}}{%
\pgfpathmoveto{\pgfqpoint{0.000000in}{-0.020833in}}%
\pgfpathcurveto{\pgfqpoint{0.005525in}{-0.020833in}}{\pgfqpoint{0.010825in}{-0.018638in}}{\pgfqpoint{0.014731in}{-0.014731in}}%
\pgfpathcurveto{\pgfqpoint{0.018638in}{-0.010825in}}{\pgfqpoint{0.020833in}{-0.005525in}}{\pgfqpoint{0.020833in}{0.000000in}}%
\pgfpathcurveto{\pgfqpoint{0.020833in}{0.005525in}}{\pgfqpoint{0.018638in}{0.010825in}}{\pgfqpoint{0.014731in}{0.014731in}}%
\pgfpathcurveto{\pgfqpoint{0.010825in}{0.018638in}}{\pgfqpoint{0.005525in}{0.020833in}}{\pgfqpoint{0.000000in}{0.020833in}}%
\pgfpathcurveto{\pgfqpoint{-0.005525in}{0.020833in}}{\pgfqpoint{-0.010825in}{0.018638in}}{\pgfqpoint{-0.014731in}{0.014731in}}%
\pgfpathcurveto{\pgfqpoint{-0.018638in}{0.010825in}}{\pgfqpoint{-0.020833in}{0.005525in}}{\pgfqpoint{-0.020833in}{0.000000in}}%
\pgfpathcurveto{\pgfqpoint{-0.020833in}{-0.005525in}}{\pgfqpoint{-0.018638in}{-0.010825in}}{\pgfqpoint{-0.014731in}{-0.014731in}}%
\pgfpathcurveto{\pgfqpoint{-0.010825in}{-0.018638in}}{\pgfqpoint{-0.005525in}{-0.020833in}}{\pgfqpoint{0.000000in}{-0.020833in}}%
\pgfpathclose%
\pgfusepath{stroke,fill}%
}%
\begin{pgfscope}%
\pgfsys@transformshift{2.131359in}{0.685187in}%
\pgfsys@useobject{currentmarker}{}%
\end{pgfscope}%
\begin{pgfscope}%
\pgfsys@transformshift{2.570525in}{0.685187in}%
\pgfsys@useobject{currentmarker}{}%
\end{pgfscope}%
\end{pgfscope}%
\begin{pgfscope}%
\pgfsetrectcap%
\pgfsetmiterjoin%
\pgfsetlinewidth{0.803000pt}%
\definecolor{currentstroke}{rgb}{0.000000,0.000000,0.000000}%
\pgfsetstrokecolor{currentstroke}%
\pgfsetdash{}{0pt}%
\pgfpathmoveto{\pgfqpoint{0.374692in}{0.521603in}}%
\pgfpathlineto{\pgfqpoint{0.374692in}{2.484603in}}%
\pgfusepath{stroke}%
\end{pgfscope}%
\begin{pgfscope}%
\pgfsetrectcap%
\pgfsetmiterjoin%
\pgfsetlinewidth{0.803000pt}%
\definecolor{currentstroke}{rgb}{0.000000,0.000000,0.000000}%
\pgfsetstrokecolor{currentstroke}%
\pgfsetdash{}{0pt}%
\pgfpathmoveto{\pgfqpoint{3.009692in}{0.521603in}}%
\pgfpathlineto{\pgfqpoint{3.009692in}{2.484603in}}%
\pgfusepath{stroke}%
\end{pgfscope}%
\begin{pgfscope}%
\pgfsetrectcap%
\pgfsetmiterjoin%
\pgfsetlinewidth{0.803000pt}%
\definecolor{currentstroke}{rgb}{0.000000,0.000000,0.000000}%
\pgfsetstrokecolor{currentstroke}%
\pgfsetdash{}{0pt}%
\pgfpathmoveto{\pgfqpoint{0.374692in}{0.521603in}}%
\pgfpathlineto{\pgfqpoint{3.009692in}{0.521603in}}%
\pgfusepath{stroke}%
\end{pgfscope}%
\begin{pgfscope}%
\pgfsetrectcap%
\pgfsetmiterjoin%
\pgfsetlinewidth{0.803000pt}%
\definecolor{currentstroke}{rgb}{0.000000,0.000000,0.000000}%
\pgfsetstrokecolor{currentstroke}%
\pgfsetdash{}{0pt}%
\pgfpathmoveto{\pgfqpoint{0.374692in}{2.484603in}}%
\pgfpathlineto{\pgfqpoint{3.009692in}{2.484603in}}%
\pgfusepath{stroke}%
\end{pgfscope}%
\begin{pgfscope}%
\pgfsetbuttcap%
\pgfsetmiterjoin%
\definecolor{currentfill}{rgb}{1.000000,1.000000,1.000000}%
\pgfsetfillcolor{currentfill}%
\pgfsetfillopacity{0.800000}%
\pgfsetlinewidth{1.003750pt}%
\definecolor{currentstroke}{rgb}{0.800000,0.800000,0.800000}%
\pgfsetstrokecolor{currentstroke}%
\pgfsetstrokeopacity{0.800000}%
\pgfsetdash{}{0pt}%
\pgfpathmoveto{\pgfqpoint{1.450976in}{1.759954in}}%
\pgfpathlineto{\pgfqpoint{2.912470in}{1.759954in}}%
\pgfpathquadraticcurveto{\pgfqpoint{2.940247in}{1.759954in}}{\pgfqpoint{2.940247in}{1.787732in}}%
\pgfpathlineto{\pgfqpoint{2.940247in}{2.387381in}}%
\pgfpathquadraticcurveto{\pgfqpoint{2.940247in}{2.415159in}}{\pgfqpoint{2.912470in}{2.415159in}}%
\pgfpathlineto{\pgfqpoint{1.450976in}{2.415159in}}%
\pgfpathquadraticcurveto{\pgfqpoint{1.423198in}{2.415159in}}{\pgfqpoint{1.423198in}{2.387381in}}%
\pgfpathlineto{\pgfqpoint{1.423198in}{1.787732in}}%
\pgfpathquadraticcurveto{\pgfqpoint{1.423198in}{1.759954in}}{\pgfqpoint{1.450976in}{1.759954in}}%
\pgfpathclose%
\pgfusepath{stroke,fill}%
\end{pgfscope}%
\begin{pgfscope}%
\pgfsetbuttcap%
\pgfsetroundjoin%
\pgfsetlinewidth{1.505625pt}%
\definecolor{currentstroke}{rgb}{0.000000,0.000000,0.000000}%
\pgfsetstrokecolor{currentstroke}%
\pgfsetdash{{5.550000pt}{2.400000pt}}{0.000000pt}%
\pgfpathmoveto{\pgfqpoint{1.478754in}{2.302691in}}%
\pgfpathlineto{\pgfqpoint{1.756532in}{2.302691in}}%
\pgfusepath{stroke}%
\end{pgfscope}%
\begin{pgfscope}%
\pgftext[x=1.867643in,y=2.254080in,left,base]{\rmfamily\fontsize{10.000000}{12.000000}\selectfont el. boundaries}%
\end{pgfscope}%
\begin{pgfscope}%
\pgfsetbuttcap%
\pgfsetroundjoin%
\definecolor{currentfill}{rgb}{0.000000,0.000000,0.000000}%
\pgfsetfillcolor{currentfill}%
\pgfsetlinewidth{1.003750pt}%
\definecolor{currentstroke}{rgb}{0.000000,0.000000,0.000000}%
\pgfsetstrokecolor{currentstroke}%
\pgfsetdash{}{0pt}%
\pgfsys@defobject{currentmarker}{\pgfqpoint{-0.020833in}{-0.020833in}}{\pgfqpoint{0.020833in}{0.020833in}}{%
\pgfpathmoveto{\pgfqpoint{0.000000in}{-0.020833in}}%
\pgfpathcurveto{\pgfqpoint{0.005525in}{-0.020833in}}{\pgfqpoint{0.010825in}{-0.018638in}}{\pgfqpoint{0.014731in}{-0.014731in}}%
\pgfpathcurveto{\pgfqpoint{0.018638in}{-0.010825in}}{\pgfqpoint{0.020833in}{-0.005525in}}{\pgfqpoint{0.020833in}{0.000000in}}%
\pgfpathcurveto{\pgfqpoint{0.020833in}{0.005525in}}{\pgfqpoint{0.018638in}{0.010825in}}{\pgfqpoint{0.014731in}{0.014731in}}%
\pgfpathcurveto{\pgfqpoint{0.010825in}{0.018638in}}{\pgfqpoint{0.005525in}{0.020833in}}{\pgfqpoint{0.000000in}{0.020833in}}%
\pgfpathcurveto{\pgfqpoint{-0.005525in}{0.020833in}}{\pgfqpoint{-0.010825in}{0.018638in}}{\pgfqpoint{-0.014731in}{0.014731in}}%
\pgfpathcurveto{\pgfqpoint{-0.018638in}{0.010825in}}{\pgfqpoint{-0.020833in}{0.005525in}}{\pgfqpoint{-0.020833in}{0.000000in}}%
\pgfpathcurveto{\pgfqpoint{-0.020833in}{-0.005525in}}{\pgfqpoint{-0.018638in}{-0.010825in}}{\pgfqpoint{-0.014731in}{-0.014731in}}%
\pgfpathcurveto{\pgfqpoint{-0.010825in}{-0.018638in}}{\pgfqpoint{-0.005525in}{-0.020833in}}{\pgfqpoint{0.000000in}{-0.020833in}}%
\pgfpathclose%
\pgfusepath{stroke,fill}%
}%
\begin{pgfscope}%
\pgfsys@transformshift{1.617643in}{2.098834in}%
\pgfsys@useobject{currentmarker}{}%
\end{pgfscope}%
\end{pgfscope}%
\begin{pgfscope}%
\pgftext[x=1.867643in,y=2.050223in,left,base]{\rmfamily\fontsize{10.000000}{12.000000}\selectfont local knots \(\displaystyle s_m\)}%
\end{pgfscope}%
\begin{pgfscope}%
\pgfsetbuttcap%
\pgfsetroundjoin%
\definecolor{currentfill}{rgb}{1.000000,0.000000,0.000000}%
\pgfsetfillcolor{currentfill}%
\pgfsetlinewidth{1.003750pt}%
\definecolor{currentstroke}{rgb}{1.000000,0.000000,0.000000}%
\pgfsetstrokecolor{currentstroke}%
\pgfsetdash{}{0pt}%
\pgfsys@defobject{currentmarker}{\pgfqpoint{-0.020833in}{-0.020833in}}{\pgfqpoint{0.020833in}{0.020833in}}{%
\pgfpathmoveto{\pgfqpoint{0.000000in}{-0.020833in}}%
\pgfpathcurveto{\pgfqpoint{0.005525in}{-0.020833in}}{\pgfqpoint{0.010825in}{-0.018638in}}{\pgfqpoint{0.014731in}{-0.014731in}}%
\pgfpathcurveto{\pgfqpoint{0.018638in}{-0.010825in}}{\pgfqpoint{0.020833in}{-0.005525in}}{\pgfqpoint{0.020833in}{0.000000in}}%
\pgfpathcurveto{\pgfqpoint{0.020833in}{0.005525in}}{\pgfqpoint{0.018638in}{0.010825in}}{\pgfqpoint{0.014731in}{0.014731in}}%
\pgfpathcurveto{\pgfqpoint{0.010825in}{0.018638in}}{\pgfqpoint{0.005525in}{0.020833in}}{\pgfqpoint{0.000000in}{0.020833in}}%
\pgfpathcurveto{\pgfqpoint{-0.005525in}{0.020833in}}{\pgfqpoint{-0.010825in}{0.018638in}}{\pgfqpoint{-0.014731in}{0.014731in}}%
\pgfpathcurveto{\pgfqpoint{-0.018638in}{0.010825in}}{\pgfqpoint{-0.020833in}{0.005525in}}{\pgfqpoint{-0.020833in}{0.000000in}}%
\pgfpathcurveto{\pgfqpoint{-0.020833in}{-0.005525in}}{\pgfqpoint{-0.018638in}{-0.010825in}}{\pgfqpoint{-0.014731in}{-0.014731in}}%
\pgfpathcurveto{\pgfqpoint{-0.010825in}{-0.018638in}}{\pgfqpoint{-0.005525in}{-0.020833in}}{\pgfqpoint{0.000000in}{-0.020833in}}%
\pgfpathclose%
\pgfusepath{stroke,fill}%
}%
\begin{pgfscope}%
\pgfsys@transformshift{1.617643in}{1.894977in}%
\pgfsys@useobject{currentmarker}{}%
\end{pgfscope}%
\end{pgfscope}%
\begin{pgfscope}%
\pgftext[x=1.867643in,y=1.846366in,left,base]{\rmfamily\fontsize{10.000000}{12.000000}\selectfont global knots \(\displaystyle z_i\)}%
\end{pgfscope}%
\begin{pgfscope}%
\pgfsetbuttcap%
\pgfsetmiterjoin%
\definecolor{currentfill}{rgb}{1.000000,1.000000,1.000000}%
\pgfsetfillcolor{currentfill}%
\pgfsetlinewidth{0.000000pt}%
\definecolor{currentstroke}{rgb}{0.000000,0.000000,0.000000}%
\pgfsetstrokecolor{currentstroke}%
\pgfsetstrokeopacity{0.000000}%
\pgfsetdash{}{0pt}%
\pgfpathmoveto{\pgfqpoint{0.629692in}{1.756603in}}%
\pgfpathlineto{\pgfqpoint{1.309692in}{1.756603in}}%
\pgfpathlineto{\pgfqpoint{1.309692in}{2.224603in}}%
\pgfpathlineto{\pgfqpoint{0.629692in}{2.224603in}}%
\pgfpathclose%
\pgfusepath{fill}%
\end{pgfscope}%
\begin{pgfscope}%
\pgfsetbuttcap%
\pgfsetroundjoin%
\definecolor{currentfill}{rgb}{0.000000,0.000000,0.000000}%
\pgfsetfillcolor{currentfill}%
\pgfsetlinewidth{0.803000pt}%
\definecolor{currentstroke}{rgb}{0.000000,0.000000,0.000000}%
\pgfsetstrokecolor{currentstroke}%
\pgfsetdash{}{0pt}%
\pgfsys@defobject{currentmarker}{\pgfqpoint{0.000000in}{-0.048611in}}{\pgfqpoint{0.000000in}{0.000000in}}{%
\pgfpathmoveto{\pgfqpoint{0.000000in}{0.000000in}}%
\pgfpathlineto{\pgfqpoint{0.000000in}{-0.048611in}}%
\pgfusepath{stroke,fill}%
}%
\begin{pgfscope}%
\pgfsys@transformshift{0.660601in}{1.756603in}%
\pgfsys@useobject{currentmarker}{}%
\end{pgfscope}%
\end{pgfscope}%
\begin{pgfscope}%
\pgftext[x=0.660601in,y=1.659381in,,top]{\rmfamily\fontsize{10.000000}{12.000000}\selectfont \(\displaystyle -1\)}%
\end{pgfscope}%
\begin{pgfscope}%
\pgfsetbuttcap%
\pgfsetroundjoin%
\definecolor{currentfill}{rgb}{0.000000,0.000000,0.000000}%
\pgfsetfillcolor{currentfill}%
\pgfsetlinewidth{0.803000pt}%
\definecolor{currentstroke}{rgb}{0.000000,0.000000,0.000000}%
\pgfsetstrokecolor{currentstroke}%
\pgfsetdash{}{0pt}%
\pgfsys@defobject{currentmarker}{\pgfqpoint{0.000000in}{-0.048611in}}{\pgfqpoint{0.000000in}{0.000000in}}{%
\pgfpathmoveto{\pgfqpoint{0.000000in}{0.000000in}}%
\pgfpathlineto{\pgfqpoint{0.000000in}{-0.048611in}}%
\pgfusepath{stroke,fill}%
}%
\begin{pgfscope}%
\pgfsys@transformshift{0.969692in}{1.756603in}%
\pgfsys@useobject{currentmarker}{}%
\end{pgfscope}%
\end{pgfscope}%
\begin{pgfscope}%
\pgftext[x=0.969692in,y=1.659381in,,top]{\rmfamily\fontsize{10.000000}{12.000000}\selectfont \(\displaystyle 0\)}%
\end{pgfscope}%
\begin{pgfscope}%
\pgfsetbuttcap%
\pgfsetroundjoin%
\definecolor{currentfill}{rgb}{0.000000,0.000000,0.000000}%
\pgfsetfillcolor{currentfill}%
\pgfsetlinewidth{0.803000pt}%
\definecolor{currentstroke}{rgb}{0.000000,0.000000,0.000000}%
\pgfsetstrokecolor{currentstroke}%
\pgfsetdash{}{0pt}%
\pgfsys@defobject{currentmarker}{\pgfqpoint{0.000000in}{-0.048611in}}{\pgfqpoint{0.000000in}{0.000000in}}{%
\pgfpathmoveto{\pgfqpoint{0.000000in}{0.000000in}}%
\pgfpathlineto{\pgfqpoint{0.000000in}{-0.048611in}}%
\pgfusepath{stroke,fill}%
}%
\begin{pgfscope}%
\pgfsys@transformshift{1.278783in}{1.756603in}%
\pgfsys@useobject{currentmarker}{}%
\end{pgfscope}%
\end{pgfscope}%
\begin{pgfscope}%
\pgftext[x=1.278783in,y=1.659381in,,top]{\rmfamily\fontsize{10.000000}{12.000000}\selectfont \(\displaystyle 1\)}%
\end{pgfscope}%
\begin{pgfscope}%
\pgftext[x=0.969692in,y=1.469413in,,top]{\rmfamily\fontsize{10.000000}{12.000000}\selectfont s}%
\end{pgfscope}%
\begin{pgfscope}%
\pgfsetbuttcap%
\pgfsetroundjoin%
\definecolor{currentfill}{rgb}{0.000000,0.000000,0.000000}%
\pgfsetfillcolor{currentfill}%
\pgfsetlinewidth{0.803000pt}%
\definecolor{currentstroke}{rgb}{0.000000,0.000000,0.000000}%
\pgfsetstrokecolor{currentstroke}%
\pgfsetdash{}{0pt}%
\pgfsys@defobject{currentmarker}{\pgfqpoint{-0.048611in}{0.000000in}}{\pgfqpoint{0.000000in}{0.000000in}}{%
\pgfpathmoveto{\pgfqpoint{0.000000in}{0.000000in}}%
\pgfpathlineto{\pgfqpoint{-0.048611in}{0.000000in}}%
\pgfusepath{stroke,fill}%
}%
\begin{pgfscope}%
\pgfsys@transformshift{0.629692in}{1.825145in}%
\pgfsys@useobject{currentmarker}{}%
\end{pgfscope}%
\end{pgfscope}%
\begin{pgfscope}%
\pgftext[x=0.463025in,y=1.772383in,left,base]{\rmfamily\fontsize{10.000000}{12.000000}\selectfont \(\displaystyle 0\)}%
\end{pgfscope}%
\begin{pgfscope}%
\pgfsetbuttcap%
\pgfsetroundjoin%
\definecolor{currentfill}{rgb}{0.000000,0.000000,0.000000}%
\pgfsetfillcolor{currentfill}%
\pgfsetlinewidth{0.803000pt}%
\definecolor{currentstroke}{rgb}{0.000000,0.000000,0.000000}%
\pgfsetstrokecolor{currentstroke}%
\pgfsetdash{}{0pt}%
\pgfsys@defobject{currentmarker}{\pgfqpoint{-0.048611in}{0.000000in}}{\pgfqpoint{0.000000in}{0.000000in}}{%
\pgfpathmoveto{\pgfqpoint{0.000000in}{0.000000in}}%
\pgfpathlineto{\pgfqpoint{-0.048611in}{0.000000in}}%
\pgfusepath{stroke,fill}%
}%
\begin{pgfscope}%
\pgfsys@transformshift{0.629692in}{2.203331in}%
\pgfsys@useobject{currentmarker}{}%
\end{pgfscope}%
\end{pgfscope}%
\begin{pgfscope}%
\pgftext[x=0.463025in,y=2.150569in,left,base]{\rmfamily\fontsize{10.000000}{12.000000}\selectfont \(\displaystyle 1\)}%
\end{pgfscope}%
\begin{pgfscope}%
\pgfpathrectangle{\pgfqpoint{0.629692in}{1.756603in}}{\pgfqpoint{0.680000in}{0.468000in}} %
\pgfusepath{clip}%
\pgfsetrectcap%
\pgfsetroundjoin%
\pgfsetlinewidth{1.003750pt}%
\definecolor{currentstroke}{rgb}{0.121569,0.466667,0.705882}%
\pgfsetstrokecolor{currentstroke}%
\pgfsetdash{}{0pt}%
\pgfpathmoveto{\pgfqpoint{0.660601in}{2.203331in}}%
\pgfpathlineto{\pgfqpoint{0.666845in}{2.191948in}}%
\pgfpathlineto{\pgfqpoint{0.673090in}{2.180719in}}%
\pgfpathlineto{\pgfqpoint{0.679334in}{2.169645in}}%
\pgfpathlineto{\pgfqpoint{0.685578in}{2.158725in}}%
\pgfpathlineto{\pgfqpoint{0.691822in}{2.147959in}}%
\pgfpathlineto{\pgfqpoint{0.698067in}{2.137348in}}%
\pgfpathlineto{\pgfqpoint{0.704311in}{2.126891in}}%
\pgfpathlineto{\pgfqpoint{0.710555in}{2.116588in}}%
\pgfpathlineto{\pgfqpoint{0.716799in}{2.106440in}}%
\pgfpathlineto{\pgfqpoint{0.723044in}{2.096446in}}%
\pgfpathlineto{\pgfqpoint{0.729288in}{2.086606in}}%
\pgfpathlineto{\pgfqpoint{0.735532in}{2.076921in}}%
\pgfpathlineto{\pgfqpoint{0.741776in}{2.067390in}}%
\pgfpathlineto{\pgfqpoint{0.748021in}{2.058014in}}%
\pgfpathlineto{\pgfqpoint{0.754265in}{2.048792in}}%
\pgfpathlineto{\pgfqpoint{0.760509in}{2.039724in}}%
\pgfpathlineto{\pgfqpoint{0.766753in}{2.030810in}}%
\pgfpathlineto{\pgfqpoint{0.772998in}{2.022051in}}%
\pgfpathlineto{\pgfqpoint{0.779242in}{2.013447in}}%
\pgfpathlineto{\pgfqpoint{0.785486in}{2.004996in}}%
\pgfpathlineto{\pgfqpoint{0.791731in}{1.996700in}}%
\pgfpathlineto{\pgfqpoint{0.797975in}{1.988558in}}%
\pgfpathlineto{\pgfqpoint{0.804219in}{1.980571in}}%
\pgfpathlineto{\pgfqpoint{0.810463in}{1.972738in}}%
\pgfpathlineto{\pgfqpoint{0.816708in}{1.965059in}}%
\pgfpathlineto{\pgfqpoint{0.822952in}{1.957535in}}%
\pgfpathlineto{\pgfqpoint{0.829196in}{1.950165in}}%
\pgfpathlineto{\pgfqpoint{0.835440in}{1.942949in}}%
\pgfpathlineto{\pgfqpoint{0.841685in}{1.935888in}}%
\pgfpathlineto{\pgfqpoint{0.847929in}{1.928981in}}%
\pgfpathlineto{\pgfqpoint{0.854173in}{1.922228in}}%
\pgfpathlineto{\pgfqpoint{0.860417in}{1.915630in}}%
\pgfpathlineto{\pgfqpoint{0.866662in}{1.909186in}}%
\pgfpathlineto{\pgfqpoint{0.872906in}{1.902896in}}%
\pgfpathlineto{\pgfqpoint{0.879150in}{1.896761in}}%
\pgfpathlineto{\pgfqpoint{0.885394in}{1.890780in}}%
\pgfpathlineto{\pgfqpoint{0.891639in}{1.884954in}}%
\pgfpathlineto{\pgfqpoint{0.897883in}{1.879281in}}%
\pgfpathlineto{\pgfqpoint{0.904127in}{1.873763in}}%
\pgfpathlineto{\pgfqpoint{0.910371in}{1.868400in}}%
\pgfpathlineto{\pgfqpoint{0.916616in}{1.863191in}}%
\pgfpathlineto{\pgfqpoint{0.922860in}{1.858136in}}%
\pgfpathlineto{\pgfqpoint{0.929104in}{1.853235in}}%
\pgfpathlineto{\pgfqpoint{0.935349in}{1.848489in}}%
\pgfpathlineto{\pgfqpoint{0.941593in}{1.843898in}}%
\pgfpathlineto{\pgfqpoint{0.947837in}{1.839460in}}%
\pgfpathlineto{\pgfqpoint{0.954081in}{1.835177in}}%
\pgfpathlineto{\pgfqpoint{0.960326in}{1.831048in}}%
\pgfpathlineto{\pgfqpoint{0.966570in}{1.827074in}}%
\pgfpathlineto{\pgfqpoint{0.972814in}{1.823254in}}%
\pgfpathlineto{\pgfqpoint{0.979058in}{1.819588in}}%
\pgfpathlineto{\pgfqpoint{0.985303in}{1.816077in}}%
\pgfpathlineto{\pgfqpoint{0.991547in}{1.812720in}}%
\pgfpathlineto{\pgfqpoint{0.997791in}{1.809517in}}%
\pgfpathlineto{\pgfqpoint{1.004035in}{1.806469in}}%
\pgfpathlineto{\pgfqpoint{1.010280in}{1.803575in}}%
\pgfpathlineto{\pgfqpoint{1.016524in}{1.800835in}}%
\pgfpathlineto{\pgfqpoint{1.022768in}{1.798250in}}%
\pgfpathlineto{\pgfqpoint{1.029012in}{1.795819in}}%
\pgfpathlineto{\pgfqpoint{1.035257in}{1.793542in}}%
\pgfpathlineto{\pgfqpoint{1.041501in}{1.791420in}}%
\pgfpathlineto{\pgfqpoint{1.047745in}{1.789452in}}%
\pgfpathlineto{\pgfqpoint{1.053989in}{1.787638in}}%
\pgfpathlineto{\pgfqpoint{1.060234in}{1.785979in}}%
\pgfpathlineto{\pgfqpoint{1.066478in}{1.784474in}}%
\pgfpathlineto{\pgfqpoint{1.072722in}{1.783124in}}%
\pgfpathlineto{\pgfqpoint{1.078967in}{1.781928in}}%
\pgfpathlineto{\pgfqpoint{1.085211in}{1.780886in}}%
\pgfpathlineto{\pgfqpoint{1.091455in}{1.779998in}}%
\pgfpathlineto{\pgfqpoint{1.097699in}{1.779265in}}%
\pgfpathlineto{\pgfqpoint{1.103944in}{1.778686in}}%
\pgfpathlineto{\pgfqpoint{1.110188in}{1.778262in}}%
\pgfpathlineto{\pgfqpoint{1.116432in}{1.777992in}}%
\pgfpathlineto{\pgfqpoint{1.122676in}{1.777876in}}%
\pgfpathlineto{\pgfqpoint{1.128921in}{1.777915in}}%
\pgfpathlineto{\pgfqpoint{1.135165in}{1.778108in}}%
\pgfpathlineto{\pgfqpoint{1.141409in}{1.778455in}}%
\pgfpathlineto{\pgfqpoint{1.147653in}{1.778956in}}%
\pgfpathlineto{\pgfqpoint{1.153898in}{1.779612in}}%
\pgfpathlineto{\pgfqpoint{1.160142in}{1.780423in}}%
\pgfpathlineto{\pgfqpoint{1.166386in}{1.781387in}}%
\pgfpathlineto{\pgfqpoint{1.172630in}{1.782506in}}%
\pgfpathlineto{\pgfqpoint{1.178875in}{1.783780in}}%
\pgfpathlineto{\pgfqpoint{1.185119in}{1.785207in}}%
\pgfpathlineto{\pgfqpoint{1.191363in}{1.786790in}}%
\pgfpathlineto{\pgfqpoint{1.197607in}{1.788526in}}%
\pgfpathlineto{\pgfqpoint{1.203852in}{1.790417in}}%
\pgfpathlineto{\pgfqpoint{1.210096in}{1.792462in}}%
\pgfpathlineto{\pgfqpoint{1.216340in}{1.794661in}}%
\pgfpathlineto{\pgfqpoint{1.222585in}{1.797015in}}%
\pgfpathlineto{\pgfqpoint{1.228829in}{1.799523in}}%
\pgfpathlineto{\pgfqpoint{1.235073in}{1.802186in}}%
\pgfpathlineto{\pgfqpoint{1.241317in}{1.805002in}}%
\pgfpathlineto{\pgfqpoint{1.247562in}{1.807974in}}%
\pgfpathlineto{\pgfqpoint{1.253806in}{1.811099in}}%
\pgfpathlineto{\pgfqpoint{1.260050in}{1.814379in}}%
\pgfpathlineto{\pgfqpoint{1.266294in}{1.817813in}}%
\pgfpathlineto{\pgfqpoint{1.272539in}{1.821402in}}%
\pgfpathlineto{\pgfqpoint{1.278783in}{1.825145in}}%
\pgfusepath{stroke}%
\end{pgfscope}%
\begin{pgfscope}%
\pgfpathrectangle{\pgfqpoint{0.629692in}{1.756603in}}{\pgfqpoint{0.680000in}{0.468000in}} %
\pgfusepath{clip}%
\pgfsetrectcap%
\pgfsetroundjoin%
\pgfsetlinewidth{1.003750pt}%
\definecolor{currentstroke}{rgb}{1.000000,0.498039,0.054902}%
\pgfsetstrokecolor{currentstroke}%
\pgfsetdash{}{0pt}%
\pgfpathmoveto{\pgfqpoint{0.660601in}{1.825145in}}%
\pgfpathlineto{\pgfqpoint{0.666845in}{1.840270in}}%
\pgfpathlineto{\pgfqpoint{0.673090in}{1.855088in}}%
\pgfpathlineto{\pgfqpoint{0.679334in}{1.869596in}}%
\pgfpathlineto{\pgfqpoint{0.685578in}{1.883796in}}%
\pgfpathlineto{\pgfqpoint{0.691822in}{1.897687in}}%
\pgfpathlineto{\pgfqpoint{0.698067in}{1.911270in}}%
\pgfpathlineto{\pgfqpoint{0.704311in}{1.924543in}}%
\pgfpathlineto{\pgfqpoint{0.710555in}{1.937508in}}%
\pgfpathlineto{\pgfqpoint{0.716799in}{1.950165in}}%
\pgfpathlineto{\pgfqpoint{0.723044in}{1.962512in}}%
\pgfpathlineto{\pgfqpoint{0.729288in}{1.974551in}}%
\pgfpathlineto{\pgfqpoint{0.735532in}{1.986282in}}%
\pgfpathlineto{\pgfqpoint{0.741776in}{1.997703in}}%
\pgfpathlineto{\pgfqpoint{0.748021in}{2.008816in}}%
\pgfpathlineto{\pgfqpoint{0.754265in}{2.019620in}}%
\pgfpathlineto{\pgfqpoint{0.760509in}{2.030116in}}%
\pgfpathlineto{\pgfqpoint{0.766753in}{2.040303in}}%
\pgfpathlineto{\pgfqpoint{0.772998in}{2.050181in}}%
\pgfpathlineto{\pgfqpoint{0.779242in}{2.059750in}}%
\pgfpathlineto{\pgfqpoint{0.785486in}{2.069011in}}%
\pgfpathlineto{\pgfqpoint{0.791731in}{2.077963in}}%
\pgfpathlineto{\pgfqpoint{0.797975in}{2.086606in}}%
\pgfpathlineto{\pgfqpoint{0.804219in}{2.094941in}}%
\pgfpathlineto{\pgfqpoint{0.810463in}{2.102967in}}%
\pgfpathlineto{\pgfqpoint{0.816708in}{2.110684in}}%
\pgfpathlineto{\pgfqpoint{0.822952in}{2.118093in}}%
\pgfpathlineto{\pgfqpoint{0.829196in}{2.125193in}}%
\pgfpathlineto{\pgfqpoint{0.835440in}{2.131984in}}%
\pgfpathlineto{\pgfqpoint{0.841685in}{2.138467in}}%
\pgfpathlineto{\pgfqpoint{0.847929in}{2.144641in}}%
\pgfpathlineto{\pgfqpoint{0.854173in}{2.150506in}}%
\pgfpathlineto{\pgfqpoint{0.860417in}{2.156062in}}%
\pgfpathlineto{\pgfqpoint{0.866662in}{2.161310in}}%
\pgfpathlineto{\pgfqpoint{0.872906in}{2.166249in}}%
\pgfpathlineto{\pgfqpoint{0.879150in}{2.170879in}}%
\pgfpathlineto{\pgfqpoint{0.885394in}{2.175201in}}%
\pgfpathlineto{\pgfqpoint{0.891639in}{2.179214in}}%
\pgfpathlineto{\pgfqpoint{0.897883in}{2.182918in}}%
\pgfpathlineto{\pgfqpoint{0.904127in}{2.186314in}}%
\pgfpathlineto{\pgfqpoint{0.910371in}{2.189401in}}%
\pgfpathlineto{\pgfqpoint{0.916616in}{2.192179in}}%
\pgfpathlineto{\pgfqpoint{0.922860in}{2.194649in}}%
\pgfpathlineto{\pgfqpoint{0.929104in}{2.196809in}}%
\pgfpathlineto{\pgfqpoint{0.935349in}{2.198662in}}%
\pgfpathlineto{\pgfqpoint{0.941593in}{2.200205in}}%
\pgfpathlineto{\pgfqpoint{0.947837in}{2.201440in}}%
\pgfpathlineto{\pgfqpoint{0.954081in}{2.202366in}}%
\pgfpathlineto{\pgfqpoint{0.960326in}{2.202983in}}%
\pgfpathlineto{\pgfqpoint{0.966570in}{2.203292in}}%
\pgfpathlineto{\pgfqpoint{0.972814in}{2.203292in}}%
\pgfpathlineto{\pgfqpoint{0.979058in}{2.202983in}}%
\pgfpathlineto{\pgfqpoint{0.985303in}{2.202366in}}%
\pgfpathlineto{\pgfqpoint{0.991547in}{2.201440in}}%
\pgfpathlineto{\pgfqpoint{0.997791in}{2.200205in}}%
\pgfpathlineto{\pgfqpoint{1.004035in}{2.198662in}}%
\pgfpathlineto{\pgfqpoint{1.010280in}{2.196809in}}%
\pgfpathlineto{\pgfqpoint{1.016524in}{2.194649in}}%
\pgfpathlineto{\pgfqpoint{1.022768in}{2.192179in}}%
\pgfpathlineto{\pgfqpoint{1.029012in}{2.189401in}}%
\pgfpathlineto{\pgfqpoint{1.035257in}{2.186314in}}%
\pgfpathlineto{\pgfqpoint{1.041501in}{2.182918in}}%
\pgfpathlineto{\pgfqpoint{1.047745in}{2.179214in}}%
\pgfpathlineto{\pgfqpoint{1.053989in}{2.175201in}}%
\pgfpathlineto{\pgfqpoint{1.060234in}{2.170879in}}%
\pgfpathlineto{\pgfqpoint{1.066478in}{2.166249in}}%
\pgfpathlineto{\pgfqpoint{1.072722in}{2.161310in}}%
\pgfpathlineto{\pgfqpoint{1.078967in}{2.156062in}}%
\pgfpathlineto{\pgfqpoint{1.085211in}{2.150506in}}%
\pgfpathlineto{\pgfqpoint{1.091455in}{2.144641in}}%
\pgfpathlineto{\pgfqpoint{1.097699in}{2.138467in}}%
\pgfpathlineto{\pgfqpoint{1.103944in}{2.131984in}}%
\pgfpathlineto{\pgfqpoint{1.110188in}{2.125193in}}%
\pgfpathlineto{\pgfqpoint{1.116432in}{2.118093in}}%
\pgfpathlineto{\pgfqpoint{1.122676in}{2.110684in}}%
\pgfpathlineto{\pgfqpoint{1.128921in}{2.102967in}}%
\pgfpathlineto{\pgfqpoint{1.135165in}{2.094941in}}%
\pgfpathlineto{\pgfqpoint{1.141409in}{2.086606in}}%
\pgfpathlineto{\pgfqpoint{1.147653in}{2.077963in}}%
\pgfpathlineto{\pgfqpoint{1.153898in}{2.069011in}}%
\pgfpathlineto{\pgfqpoint{1.160142in}{2.059750in}}%
\pgfpathlineto{\pgfqpoint{1.166386in}{2.050181in}}%
\pgfpathlineto{\pgfqpoint{1.172630in}{2.040303in}}%
\pgfpathlineto{\pgfqpoint{1.178875in}{2.030116in}}%
\pgfpathlineto{\pgfqpoint{1.185119in}{2.019620in}}%
\pgfpathlineto{\pgfqpoint{1.191363in}{2.008816in}}%
\pgfpathlineto{\pgfqpoint{1.197607in}{1.997703in}}%
\pgfpathlineto{\pgfqpoint{1.203852in}{1.986282in}}%
\pgfpathlineto{\pgfqpoint{1.210096in}{1.974551in}}%
\pgfpathlineto{\pgfqpoint{1.216340in}{1.962512in}}%
\pgfpathlineto{\pgfqpoint{1.222585in}{1.950165in}}%
\pgfpathlineto{\pgfqpoint{1.228829in}{1.937508in}}%
\pgfpathlineto{\pgfqpoint{1.235073in}{1.924543in}}%
\pgfpathlineto{\pgfqpoint{1.241317in}{1.911270in}}%
\pgfpathlineto{\pgfqpoint{1.247562in}{1.897687in}}%
\pgfpathlineto{\pgfqpoint{1.253806in}{1.883796in}}%
\pgfpathlineto{\pgfqpoint{1.260050in}{1.869596in}}%
\pgfpathlineto{\pgfqpoint{1.266294in}{1.855088in}}%
\pgfpathlineto{\pgfqpoint{1.272539in}{1.840270in}}%
\pgfpathlineto{\pgfqpoint{1.278783in}{1.825145in}}%
\pgfusepath{stroke}%
\end{pgfscope}%
\begin{pgfscope}%
\pgfpathrectangle{\pgfqpoint{0.629692in}{1.756603in}}{\pgfqpoint{0.680000in}{0.468000in}} %
\pgfusepath{clip}%
\pgfsetrectcap%
\pgfsetroundjoin%
\pgfsetlinewidth{1.003750pt}%
\definecolor{currentstroke}{rgb}{0.172549,0.627451,0.172549}%
\pgfsetstrokecolor{currentstroke}%
\pgfsetdash{}{0pt}%
\pgfpathmoveto{\pgfqpoint{0.660601in}{1.825145in}}%
\pgfpathlineto{\pgfqpoint{0.666845in}{1.821402in}}%
\pgfpathlineto{\pgfqpoint{0.673090in}{1.817813in}}%
\pgfpathlineto{\pgfqpoint{0.679334in}{1.814379in}}%
\pgfpathlineto{\pgfqpoint{0.685578in}{1.811099in}}%
\pgfpathlineto{\pgfqpoint{0.691822in}{1.807974in}}%
\pgfpathlineto{\pgfqpoint{0.698067in}{1.805002in}}%
\pgfpathlineto{\pgfqpoint{0.704311in}{1.802186in}}%
\pgfpathlineto{\pgfqpoint{0.710555in}{1.799523in}}%
\pgfpathlineto{\pgfqpoint{0.716799in}{1.797015in}}%
\pgfpathlineto{\pgfqpoint{0.723044in}{1.794661in}}%
\pgfpathlineto{\pgfqpoint{0.729288in}{1.792462in}}%
\pgfpathlineto{\pgfqpoint{0.735532in}{1.790417in}}%
\pgfpathlineto{\pgfqpoint{0.741776in}{1.788526in}}%
\pgfpathlineto{\pgfqpoint{0.748021in}{1.786790in}}%
\pgfpathlineto{\pgfqpoint{0.754265in}{1.785207in}}%
\pgfpathlineto{\pgfqpoint{0.760509in}{1.783780in}}%
\pgfpathlineto{\pgfqpoint{0.766753in}{1.782506in}}%
\pgfpathlineto{\pgfqpoint{0.772998in}{1.781387in}}%
\pgfpathlineto{\pgfqpoint{0.779242in}{1.780423in}}%
\pgfpathlineto{\pgfqpoint{0.785486in}{1.779612in}}%
\pgfpathlineto{\pgfqpoint{0.791731in}{1.778956in}}%
\pgfpathlineto{\pgfqpoint{0.797975in}{1.778455in}}%
\pgfpathlineto{\pgfqpoint{0.804219in}{1.778108in}}%
\pgfpathlineto{\pgfqpoint{0.810463in}{1.777915in}}%
\pgfpathlineto{\pgfqpoint{0.816708in}{1.777876in}}%
\pgfpathlineto{\pgfqpoint{0.822952in}{1.777992in}}%
\pgfpathlineto{\pgfqpoint{0.829196in}{1.778262in}}%
\pgfpathlineto{\pgfqpoint{0.835440in}{1.778686in}}%
\pgfpathlineto{\pgfqpoint{0.841685in}{1.779265in}}%
\pgfpathlineto{\pgfqpoint{0.847929in}{1.779998in}}%
\pgfpathlineto{\pgfqpoint{0.854173in}{1.780886in}}%
\pgfpathlineto{\pgfqpoint{0.860417in}{1.781928in}}%
\pgfpathlineto{\pgfqpoint{0.866662in}{1.783124in}}%
\pgfpathlineto{\pgfqpoint{0.872906in}{1.784474in}}%
\pgfpathlineto{\pgfqpoint{0.879150in}{1.785979in}}%
\pgfpathlineto{\pgfqpoint{0.885394in}{1.787638in}}%
\pgfpathlineto{\pgfqpoint{0.891639in}{1.789452in}}%
\pgfpathlineto{\pgfqpoint{0.897883in}{1.791420in}}%
\pgfpathlineto{\pgfqpoint{0.904127in}{1.793542in}}%
\pgfpathlineto{\pgfqpoint{0.910371in}{1.795819in}}%
\pgfpathlineto{\pgfqpoint{0.916616in}{1.798250in}}%
\pgfpathlineto{\pgfqpoint{0.922860in}{1.800835in}}%
\pgfpathlineto{\pgfqpoint{0.929104in}{1.803575in}}%
\pgfpathlineto{\pgfqpoint{0.935349in}{1.806469in}}%
\pgfpathlineto{\pgfqpoint{0.941593in}{1.809517in}}%
\pgfpathlineto{\pgfqpoint{0.947837in}{1.812720in}}%
\pgfpathlineto{\pgfqpoint{0.954081in}{1.816077in}}%
\pgfpathlineto{\pgfqpoint{0.960326in}{1.819588in}}%
\pgfpathlineto{\pgfqpoint{0.966570in}{1.823254in}}%
\pgfpathlineto{\pgfqpoint{0.972814in}{1.827074in}}%
\pgfpathlineto{\pgfqpoint{0.979058in}{1.831048in}}%
\pgfpathlineto{\pgfqpoint{0.985303in}{1.835177in}}%
\pgfpathlineto{\pgfqpoint{0.991547in}{1.839460in}}%
\pgfpathlineto{\pgfqpoint{0.997791in}{1.843898in}}%
\pgfpathlineto{\pgfqpoint{1.004035in}{1.848489in}}%
\pgfpathlineto{\pgfqpoint{1.010280in}{1.853235in}}%
\pgfpathlineto{\pgfqpoint{1.016524in}{1.858136in}}%
\pgfpathlineto{\pgfqpoint{1.022768in}{1.863191in}}%
\pgfpathlineto{\pgfqpoint{1.029012in}{1.868400in}}%
\pgfpathlineto{\pgfqpoint{1.035257in}{1.873763in}}%
\pgfpathlineto{\pgfqpoint{1.041501in}{1.879281in}}%
\pgfpathlineto{\pgfqpoint{1.047745in}{1.884954in}}%
\pgfpathlineto{\pgfqpoint{1.053989in}{1.890780in}}%
\pgfpathlineto{\pgfqpoint{1.060234in}{1.896761in}}%
\pgfpathlineto{\pgfqpoint{1.066478in}{1.902896in}}%
\pgfpathlineto{\pgfqpoint{1.072722in}{1.909186in}}%
\pgfpathlineto{\pgfqpoint{1.078967in}{1.915630in}}%
\pgfpathlineto{\pgfqpoint{1.085211in}{1.922228in}}%
\pgfpathlineto{\pgfqpoint{1.091455in}{1.928981in}}%
\pgfpathlineto{\pgfqpoint{1.097699in}{1.935888in}}%
\pgfpathlineto{\pgfqpoint{1.103944in}{1.942949in}}%
\pgfpathlineto{\pgfqpoint{1.110188in}{1.950165in}}%
\pgfpathlineto{\pgfqpoint{1.116432in}{1.957535in}}%
\pgfpathlineto{\pgfqpoint{1.122676in}{1.965059in}}%
\pgfpathlineto{\pgfqpoint{1.128921in}{1.972738in}}%
\pgfpathlineto{\pgfqpoint{1.135165in}{1.980571in}}%
\pgfpathlineto{\pgfqpoint{1.141409in}{1.988558in}}%
\pgfpathlineto{\pgfqpoint{1.147653in}{1.996700in}}%
\pgfpathlineto{\pgfqpoint{1.153898in}{2.004996in}}%
\pgfpathlineto{\pgfqpoint{1.160142in}{2.013447in}}%
\pgfpathlineto{\pgfqpoint{1.166386in}{2.022051in}}%
\pgfpathlineto{\pgfqpoint{1.172630in}{2.030810in}}%
\pgfpathlineto{\pgfqpoint{1.178875in}{2.039724in}}%
\pgfpathlineto{\pgfqpoint{1.185119in}{2.048792in}}%
\pgfpathlineto{\pgfqpoint{1.191363in}{2.058014in}}%
\pgfpathlineto{\pgfqpoint{1.197607in}{2.067390in}}%
\pgfpathlineto{\pgfqpoint{1.203852in}{2.076921in}}%
\pgfpathlineto{\pgfqpoint{1.210096in}{2.086606in}}%
\pgfpathlineto{\pgfqpoint{1.216340in}{2.096446in}}%
\pgfpathlineto{\pgfqpoint{1.222585in}{2.106440in}}%
\pgfpathlineto{\pgfqpoint{1.228829in}{2.116588in}}%
\pgfpathlineto{\pgfqpoint{1.235073in}{2.126891in}}%
\pgfpathlineto{\pgfqpoint{1.241317in}{2.137348in}}%
\pgfpathlineto{\pgfqpoint{1.247562in}{2.147959in}}%
\pgfpathlineto{\pgfqpoint{1.253806in}{2.158725in}}%
\pgfpathlineto{\pgfqpoint{1.260050in}{2.169645in}}%
\pgfpathlineto{\pgfqpoint{1.266294in}{2.180719in}}%
\pgfpathlineto{\pgfqpoint{1.272539in}{2.191948in}}%
\pgfpathlineto{\pgfqpoint{1.278783in}{2.203331in}}%
\pgfusepath{stroke}%
\end{pgfscope}%
\begin{pgfscope}%
\pgfpathrectangle{\pgfqpoint{0.629692in}{1.756603in}}{\pgfqpoint{0.680000in}{0.468000in}} %
\pgfusepath{clip}%
\pgfsetbuttcap%
\pgfsetroundjoin%
\definecolor{currentfill}{rgb}{0.000000,0.000000,0.000000}%
\pgfsetfillcolor{currentfill}%
\pgfsetlinewidth{1.003750pt}%
\definecolor{currentstroke}{rgb}{0.000000,0.000000,0.000000}%
\pgfsetstrokecolor{currentstroke}%
\pgfsetdash{}{0pt}%
\pgfsys@defobject{currentmarker}{\pgfqpoint{-0.020833in}{-0.020833in}}{\pgfqpoint{0.020833in}{0.020833in}}{%
\pgfpathmoveto{\pgfqpoint{0.000000in}{-0.020833in}}%
\pgfpathcurveto{\pgfqpoint{0.005525in}{-0.020833in}}{\pgfqpoint{0.010825in}{-0.018638in}}{\pgfqpoint{0.014731in}{-0.014731in}}%
\pgfpathcurveto{\pgfqpoint{0.018638in}{-0.010825in}}{\pgfqpoint{0.020833in}{-0.005525in}}{\pgfqpoint{0.020833in}{0.000000in}}%
\pgfpathcurveto{\pgfqpoint{0.020833in}{0.005525in}}{\pgfqpoint{0.018638in}{0.010825in}}{\pgfqpoint{0.014731in}{0.014731in}}%
\pgfpathcurveto{\pgfqpoint{0.010825in}{0.018638in}}{\pgfqpoint{0.005525in}{0.020833in}}{\pgfqpoint{0.000000in}{0.020833in}}%
\pgfpathcurveto{\pgfqpoint{-0.005525in}{0.020833in}}{\pgfqpoint{-0.010825in}{0.018638in}}{\pgfqpoint{-0.014731in}{0.014731in}}%
\pgfpathcurveto{\pgfqpoint{-0.018638in}{0.010825in}}{\pgfqpoint{-0.020833in}{0.005525in}}{\pgfqpoint{-0.020833in}{0.000000in}}%
\pgfpathcurveto{\pgfqpoint{-0.020833in}{-0.005525in}}{\pgfqpoint{-0.018638in}{-0.010825in}}{\pgfqpoint{-0.014731in}{-0.014731in}}%
\pgfpathcurveto{\pgfqpoint{-0.010825in}{-0.018638in}}{\pgfqpoint{-0.005525in}{-0.020833in}}{\pgfqpoint{0.000000in}{-0.020833in}}%
\pgfpathclose%
\pgfusepath{stroke,fill}%
}%
\begin{pgfscope}%
\pgfsys@transformshift{0.660601in}{1.825145in}%
\pgfsys@useobject{currentmarker}{}%
\end{pgfscope}%
\begin{pgfscope}%
\pgfsys@transformshift{0.969692in}{1.825145in}%
\pgfsys@useobject{currentmarker}{}%
\end{pgfscope}%
\begin{pgfscope}%
\pgfsys@transformshift{1.278783in}{1.825145in}%
\pgfsys@useobject{currentmarker}{}%
\end{pgfscope}%
\end{pgfscope}%
\begin{pgfscope}%
\pgfsetrectcap%
\pgfsetmiterjoin%
\pgfsetlinewidth{0.803000pt}%
\definecolor{currentstroke}{rgb}{0.000000,0.000000,0.000000}%
\pgfsetstrokecolor{currentstroke}%
\pgfsetdash{}{0pt}%
\pgfpathmoveto{\pgfqpoint{0.629692in}{1.756603in}}%
\pgfpathlineto{\pgfqpoint{0.629692in}{2.224603in}}%
\pgfusepath{stroke}%
\end{pgfscope}%
\begin{pgfscope}%
\pgfsetrectcap%
\pgfsetmiterjoin%
\pgfsetlinewidth{0.803000pt}%
\definecolor{currentstroke}{rgb}{0.000000,0.000000,0.000000}%
\pgfsetstrokecolor{currentstroke}%
\pgfsetdash{}{0pt}%
\pgfpathmoveto{\pgfqpoint{1.309692in}{1.756603in}}%
\pgfpathlineto{\pgfqpoint{1.309692in}{2.224603in}}%
\pgfusepath{stroke}%
\end{pgfscope}%
\begin{pgfscope}%
\pgfsetrectcap%
\pgfsetmiterjoin%
\pgfsetlinewidth{0.803000pt}%
\definecolor{currentstroke}{rgb}{0.000000,0.000000,0.000000}%
\pgfsetstrokecolor{currentstroke}%
\pgfsetdash{}{0pt}%
\pgfpathmoveto{\pgfqpoint{0.629692in}{1.756603in}}%
\pgfpathlineto{\pgfqpoint{1.309692in}{1.756603in}}%
\pgfusepath{stroke}%
\end{pgfscope}%
\begin{pgfscope}%
\pgfsetrectcap%
\pgfsetmiterjoin%
\pgfsetlinewidth{0.803000pt}%
\definecolor{currentstroke}{rgb}{0.000000,0.000000,0.000000}%
\pgfsetstrokecolor{currentstroke}%
\pgfsetdash{}{0pt}%
\pgfpathmoveto{\pgfqpoint{0.629692in}{2.224603in}}%
\pgfpathlineto{\pgfqpoint{1.309692in}{2.224603in}}%
\pgfusepath{stroke}%
\end{pgfscope}%
\begin{pgfscope}%
\pgftext[x=0.969692in,y=2.307937in,,base]{\rmfamily\fontsize{12.000000}{14.400000}\selectfont Local \(\displaystyle \eta_n\)}%
\end{pgfscope}%
\end{pgfpicture}%
\makeatother%
\endgroup%

%%% Creator: Matplotlib, PGF backend
%%
%% To include the figure in your LaTeX document, write
%%   \input{<filename>.pgf}
%%
%% Make sure the required packages are loaded in your preamble
%%   \usepackage{pgf}
%%
%% Figures using additional raster images can only be included by \input if
%% they are in the same directory as the main LaTeX file. For loading figures
%% from other directories you can use the `import` package
%%   \usepackage{import}
%% and then include the figures with
%%   \import{<path to file>}{<filename>.pgf}
%%
%% Matplotlib used the following preamble
%%   \usepackage{fontspec}
%%   \setmainfont{DejaVu Serif}
%%   \setsansfont{DejaVu Sans}
%%   \setmonofont{DejaVu Sans Mono}
%%
\begingroup%
\makeatletter%
\begin{pgfpicture}%
\pgfpathrectangle{\pgfpointorigin}{\pgfqpoint{3.198427in}{2.637365in}}%
\pgfusepath{use as bounding box, clip}%
\begin{pgfscope}%
\pgfsetbuttcap%
\pgfsetmiterjoin%
\definecolor{currentfill}{rgb}{1.000000,1.000000,1.000000}%
\pgfsetfillcolor{currentfill}%
\pgfsetlinewidth{0.000000pt}%
\definecolor{currentstroke}{rgb}{1.000000,1.000000,1.000000}%
\pgfsetstrokecolor{currentstroke}%
\pgfsetdash{}{0pt}%
\pgfpathmoveto{\pgfqpoint{0.000000in}{0.000000in}}%
\pgfpathlineto{\pgfqpoint{3.198427in}{0.000000in}}%
\pgfpathlineto{\pgfqpoint{3.198427in}{2.637365in}}%
\pgfpathlineto{\pgfqpoint{0.000000in}{2.637365in}}%
\pgfpathclose%
\pgfusepath{fill}%
\end{pgfscope}%
\begin{pgfscope}%
\pgfsetbuttcap%
\pgfsetmiterjoin%
\definecolor{currentfill}{rgb}{1.000000,1.000000,1.000000}%
\pgfsetfillcolor{currentfill}%
\pgfsetlinewidth{0.000000pt}%
\definecolor{currentstroke}{rgb}{0.000000,0.000000,0.000000}%
\pgfsetstrokecolor{currentstroke}%
\pgfsetstrokeopacity{0.000000}%
\pgfsetdash{}{0pt}%
\pgfpathmoveto{\pgfqpoint{0.374692in}{0.521603in}}%
\pgfpathlineto{\pgfqpoint{3.009692in}{0.521603in}}%
\pgfpathlineto{\pgfqpoint{3.009692in}{2.484603in}}%
\pgfpathlineto{\pgfqpoint{0.374692in}{2.484603in}}%
\pgfpathclose%
\pgfusepath{fill}%
\end{pgfscope}%
\begin{pgfscope}%
\pgfsetbuttcap%
\pgfsetroundjoin%
\definecolor{currentfill}{rgb}{0.000000,0.000000,0.000000}%
\pgfsetfillcolor{currentfill}%
\pgfsetlinewidth{0.803000pt}%
\definecolor{currentstroke}{rgb}{0.000000,0.000000,0.000000}%
\pgfsetstrokecolor{currentstroke}%
\pgfsetdash{}{0pt}%
\pgfsys@defobject{currentmarker}{\pgfqpoint{0.000000in}{-0.048611in}}{\pgfqpoint{0.000000in}{0.000000in}}{%
\pgfpathmoveto{\pgfqpoint{0.000000in}{0.000000in}}%
\pgfpathlineto{\pgfqpoint{0.000000in}{-0.048611in}}%
\pgfusepath{stroke,fill}%
}%
\begin{pgfscope}%
\pgfsys@transformshift{0.374692in}{0.521603in}%
\pgfsys@useobject{currentmarker}{}%
\end{pgfscope}%
\end{pgfscope}%
\begin{pgfscope}%
\pgftext[x=0.374692in,y=0.424381in,,top]{\rmfamily\fontsize{10.000000}{12.000000}\selectfont \(\displaystyle 0.0\)}%
\end{pgfscope}%
\begin{pgfscope}%
\pgfsetbuttcap%
\pgfsetroundjoin%
\definecolor{currentfill}{rgb}{0.000000,0.000000,0.000000}%
\pgfsetfillcolor{currentfill}%
\pgfsetlinewidth{0.803000pt}%
\definecolor{currentstroke}{rgb}{0.000000,0.000000,0.000000}%
\pgfsetstrokecolor{currentstroke}%
\pgfsetdash{}{0pt}%
\pgfsys@defobject{currentmarker}{\pgfqpoint{0.000000in}{-0.048611in}}{\pgfqpoint{0.000000in}{0.000000in}}{%
\pgfpathmoveto{\pgfqpoint{0.000000in}{0.000000in}}%
\pgfpathlineto{\pgfqpoint{0.000000in}{-0.048611in}}%
\pgfusepath{stroke,fill}%
}%
\begin{pgfscope}%
\pgfsys@transformshift{0.901692in}{0.521603in}%
\pgfsys@useobject{currentmarker}{}%
\end{pgfscope}%
\end{pgfscope}%
\begin{pgfscope}%
\pgftext[x=0.901692in,y=0.424381in,,top]{\rmfamily\fontsize{10.000000}{12.000000}\selectfont \(\displaystyle 0.2\)}%
\end{pgfscope}%
\begin{pgfscope}%
\pgfsetbuttcap%
\pgfsetroundjoin%
\definecolor{currentfill}{rgb}{0.000000,0.000000,0.000000}%
\pgfsetfillcolor{currentfill}%
\pgfsetlinewidth{0.803000pt}%
\definecolor{currentstroke}{rgb}{0.000000,0.000000,0.000000}%
\pgfsetstrokecolor{currentstroke}%
\pgfsetdash{}{0pt}%
\pgfsys@defobject{currentmarker}{\pgfqpoint{0.000000in}{-0.048611in}}{\pgfqpoint{0.000000in}{0.000000in}}{%
\pgfpathmoveto{\pgfqpoint{0.000000in}{0.000000in}}%
\pgfpathlineto{\pgfqpoint{0.000000in}{-0.048611in}}%
\pgfusepath{stroke,fill}%
}%
\begin{pgfscope}%
\pgfsys@transformshift{1.428692in}{0.521603in}%
\pgfsys@useobject{currentmarker}{}%
\end{pgfscope}%
\end{pgfscope}%
\begin{pgfscope}%
\pgftext[x=1.428692in,y=0.424381in,,top]{\rmfamily\fontsize{10.000000}{12.000000}\selectfont \(\displaystyle 0.4\)}%
\end{pgfscope}%
\begin{pgfscope}%
\pgfsetbuttcap%
\pgfsetroundjoin%
\definecolor{currentfill}{rgb}{0.000000,0.000000,0.000000}%
\pgfsetfillcolor{currentfill}%
\pgfsetlinewidth{0.803000pt}%
\definecolor{currentstroke}{rgb}{0.000000,0.000000,0.000000}%
\pgfsetstrokecolor{currentstroke}%
\pgfsetdash{}{0pt}%
\pgfsys@defobject{currentmarker}{\pgfqpoint{0.000000in}{-0.048611in}}{\pgfqpoint{0.000000in}{0.000000in}}{%
\pgfpathmoveto{\pgfqpoint{0.000000in}{0.000000in}}%
\pgfpathlineto{\pgfqpoint{0.000000in}{-0.048611in}}%
\pgfusepath{stroke,fill}%
}%
\begin{pgfscope}%
\pgfsys@transformshift{1.955692in}{0.521603in}%
\pgfsys@useobject{currentmarker}{}%
\end{pgfscope}%
\end{pgfscope}%
\begin{pgfscope}%
\pgftext[x=1.955692in,y=0.424381in,,top]{\rmfamily\fontsize{10.000000}{12.000000}\selectfont \(\displaystyle 0.6\)}%
\end{pgfscope}%
\begin{pgfscope}%
\pgfsetbuttcap%
\pgfsetroundjoin%
\definecolor{currentfill}{rgb}{0.000000,0.000000,0.000000}%
\pgfsetfillcolor{currentfill}%
\pgfsetlinewidth{0.803000pt}%
\definecolor{currentstroke}{rgb}{0.000000,0.000000,0.000000}%
\pgfsetstrokecolor{currentstroke}%
\pgfsetdash{}{0pt}%
\pgfsys@defobject{currentmarker}{\pgfqpoint{0.000000in}{-0.048611in}}{\pgfqpoint{0.000000in}{0.000000in}}{%
\pgfpathmoveto{\pgfqpoint{0.000000in}{0.000000in}}%
\pgfpathlineto{\pgfqpoint{0.000000in}{-0.048611in}}%
\pgfusepath{stroke,fill}%
}%
\begin{pgfscope}%
\pgfsys@transformshift{2.482692in}{0.521603in}%
\pgfsys@useobject{currentmarker}{}%
\end{pgfscope}%
\end{pgfscope}%
\begin{pgfscope}%
\pgftext[x=2.482692in,y=0.424381in,,top]{\rmfamily\fontsize{10.000000}{12.000000}\selectfont \(\displaystyle 0.8\)}%
\end{pgfscope}%
\begin{pgfscope}%
\pgfsetbuttcap%
\pgfsetroundjoin%
\definecolor{currentfill}{rgb}{0.000000,0.000000,0.000000}%
\pgfsetfillcolor{currentfill}%
\pgfsetlinewidth{0.803000pt}%
\definecolor{currentstroke}{rgb}{0.000000,0.000000,0.000000}%
\pgfsetstrokecolor{currentstroke}%
\pgfsetdash{}{0pt}%
\pgfsys@defobject{currentmarker}{\pgfqpoint{0.000000in}{-0.048611in}}{\pgfqpoint{0.000000in}{0.000000in}}{%
\pgfpathmoveto{\pgfqpoint{0.000000in}{0.000000in}}%
\pgfpathlineto{\pgfqpoint{0.000000in}{-0.048611in}}%
\pgfusepath{stroke,fill}%
}%
\begin{pgfscope}%
\pgfsys@transformshift{3.009692in}{0.521603in}%
\pgfsys@useobject{currentmarker}{}%
\end{pgfscope}%
\end{pgfscope}%
\begin{pgfscope}%
\pgftext[x=3.009692in,y=0.424381in,,top]{\rmfamily\fontsize{10.000000}{12.000000}\selectfont \(\displaystyle 1.0\)}%
\end{pgfscope}%
\begin{pgfscope}%
\pgftext[x=1.692192in,y=0.234413in,,top]{\rmfamily\fontsize{10.000000}{12.000000}\selectfont \(\displaystyle z\)}%
\end{pgfscope}%
\begin{pgfscope}%
\pgfsetbuttcap%
\pgfsetroundjoin%
\definecolor{currentfill}{rgb}{0.000000,0.000000,0.000000}%
\pgfsetfillcolor{currentfill}%
\pgfsetlinewidth{0.803000pt}%
\definecolor{currentstroke}{rgb}{0.000000,0.000000,0.000000}%
\pgfsetstrokecolor{currentstroke}%
\pgfsetdash{}{0pt}%
\pgfsys@defobject{currentmarker}{\pgfqpoint{-0.048611in}{0.000000in}}{\pgfqpoint{0.000000in}{0.000000in}}{%
\pgfpathmoveto{\pgfqpoint{0.000000in}{0.000000in}}%
\pgfpathlineto{\pgfqpoint{-0.048611in}{0.000000in}}%
\pgfusepath{stroke,fill}%
}%
\begin{pgfscope}%
\pgfsys@transformshift{0.374692in}{0.521603in}%
\pgfsys@useobject{currentmarker}{}%
\end{pgfscope}%
\end{pgfscope}%
\begin{pgfscope}%
\pgftext[x=0.100000in,y=0.468842in,left,base]{\rmfamily\fontsize{10.000000}{12.000000}\selectfont \(\displaystyle -1\)}%
\end{pgfscope}%
\begin{pgfscope}%
\pgfsetbuttcap%
\pgfsetroundjoin%
\definecolor{currentfill}{rgb}{0.000000,0.000000,0.000000}%
\pgfsetfillcolor{currentfill}%
\pgfsetlinewidth{0.803000pt}%
\definecolor{currentstroke}{rgb}{0.000000,0.000000,0.000000}%
\pgfsetstrokecolor{currentstroke}%
\pgfsetdash{}{0pt}%
\pgfsys@defobject{currentmarker}{\pgfqpoint{-0.048611in}{0.000000in}}{\pgfqpoint{0.000000in}{0.000000in}}{%
\pgfpathmoveto{\pgfqpoint{0.000000in}{0.000000in}}%
\pgfpathlineto{\pgfqpoint{-0.048611in}{0.000000in}}%
\pgfusepath{stroke,fill}%
}%
\begin{pgfscope}%
\pgfsys@transformshift{0.374692in}{0.848770in}%
\pgfsys@useobject{currentmarker}{}%
\end{pgfscope}%
\end{pgfscope}%
\begin{pgfscope}%
\pgftext[x=0.208025in,y=0.796008in,left,base]{\rmfamily\fontsize{10.000000}{12.000000}\selectfont \(\displaystyle 0\)}%
\end{pgfscope}%
\begin{pgfscope}%
\pgfsetbuttcap%
\pgfsetroundjoin%
\definecolor{currentfill}{rgb}{0.000000,0.000000,0.000000}%
\pgfsetfillcolor{currentfill}%
\pgfsetlinewidth{0.803000pt}%
\definecolor{currentstroke}{rgb}{0.000000,0.000000,0.000000}%
\pgfsetstrokecolor{currentstroke}%
\pgfsetdash{}{0pt}%
\pgfsys@defobject{currentmarker}{\pgfqpoint{-0.048611in}{0.000000in}}{\pgfqpoint{0.000000in}{0.000000in}}{%
\pgfpathmoveto{\pgfqpoint{0.000000in}{0.000000in}}%
\pgfpathlineto{\pgfqpoint{-0.048611in}{0.000000in}}%
\pgfusepath{stroke,fill}%
}%
\begin{pgfscope}%
\pgfsys@transformshift{0.374692in}{1.175937in}%
\pgfsys@useobject{currentmarker}{}%
\end{pgfscope}%
\end{pgfscope}%
\begin{pgfscope}%
\pgftext[x=0.208025in,y=1.123175in,left,base]{\rmfamily\fontsize{10.000000}{12.000000}\selectfont \(\displaystyle 1\)}%
\end{pgfscope}%
\begin{pgfscope}%
\pgfsetbuttcap%
\pgfsetroundjoin%
\definecolor{currentfill}{rgb}{0.000000,0.000000,0.000000}%
\pgfsetfillcolor{currentfill}%
\pgfsetlinewidth{0.803000pt}%
\definecolor{currentstroke}{rgb}{0.000000,0.000000,0.000000}%
\pgfsetstrokecolor{currentstroke}%
\pgfsetdash{}{0pt}%
\pgfsys@defobject{currentmarker}{\pgfqpoint{-0.048611in}{0.000000in}}{\pgfqpoint{0.000000in}{0.000000in}}{%
\pgfpathmoveto{\pgfqpoint{0.000000in}{0.000000in}}%
\pgfpathlineto{\pgfqpoint{-0.048611in}{0.000000in}}%
\pgfusepath{stroke,fill}%
}%
\begin{pgfscope}%
\pgfsys@transformshift{0.374692in}{1.503103in}%
\pgfsys@useobject{currentmarker}{}%
\end{pgfscope}%
\end{pgfscope}%
\begin{pgfscope}%
\pgftext[x=0.208025in,y=1.450342in,left,base]{\rmfamily\fontsize{10.000000}{12.000000}\selectfont \(\displaystyle 2\)}%
\end{pgfscope}%
\begin{pgfscope}%
\pgfsetbuttcap%
\pgfsetroundjoin%
\definecolor{currentfill}{rgb}{0.000000,0.000000,0.000000}%
\pgfsetfillcolor{currentfill}%
\pgfsetlinewidth{0.803000pt}%
\definecolor{currentstroke}{rgb}{0.000000,0.000000,0.000000}%
\pgfsetstrokecolor{currentstroke}%
\pgfsetdash{}{0pt}%
\pgfsys@defobject{currentmarker}{\pgfqpoint{-0.048611in}{0.000000in}}{\pgfqpoint{0.000000in}{0.000000in}}{%
\pgfpathmoveto{\pgfqpoint{0.000000in}{0.000000in}}%
\pgfpathlineto{\pgfqpoint{-0.048611in}{0.000000in}}%
\pgfusepath{stroke,fill}%
}%
\begin{pgfscope}%
\pgfsys@transformshift{0.374692in}{1.830270in}%
\pgfsys@useobject{currentmarker}{}%
\end{pgfscope}%
\end{pgfscope}%
\begin{pgfscope}%
\pgftext[x=0.208025in,y=1.777508in,left,base]{\rmfamily\fontsize{10.000000}{12.000000}\selectfont \(\displaystyle 3\)}%
\end{pgfscope}%
\begin{pgfscope}%
\pgfsetbuttcap%
\pgfsetroundjoin%
\definecolor{currentfill}{rgb}{0.000000,0.000000,0.000000}%
\pgfsetfillcolor{currentfill}%
\pgfsetlinewidth{0.803000pt}%
\definecolor{currentstroke}{rgb}{0.000000,0.000000,0.000000}%
\pgfsetstrokecolor{currentstroke}%
\pgfsetdash{}{0pt}%
\pgfsys@defobject{currentmarker}{\pgfqpoint{-0.048611in}{0.000000in}}{\pgfqpoint{0.000000in}{0.000000in}}{%
\pgfpathmoveto{\pgfqpoint{0.000000in}{0.000000in}}%
\pgfpathlineto{\pgfqpoint{-0.048611in}{0.000000in}}%
\pgfusepath{stroke,fill}%
}%
\begin{pgfscope}%
\pgfsys@transformshift{0.374692in}{2.157437in}%
\pgfsys@useobject{currentmarker}{}%
\end{pgfscope}%
\end{pgfscope}%
\begin{pgfscope}%
\pgftext[x=0.208025in,y=2.104675in,left,base]{\rmfamily\fontsize{10.000000}{12.000000}\selectfont \(\displaystyle 4\)}%
\end{pgfscope}%
\begin{pgfscope}%
\pgfsetbuttcap%
\pgfsetroundjoin%
\definecolor{currentfill}{rgb}{0.000000,0.000000,0.000000}%
\pgfsetfillcolor{currentfill}%
\pgfsetlinewidth{0.803000pt}%
\definecolor{currentstroke}{rgb}{0.000000,0.000000,0.000000}%
\pgfsetstrokecolor{currentstroke}%
\pgfsetdash{}{0pt}%
\pgfsys@defobject{currentmarker}{\pgfqpoint{-0.048611in}{0.000000in}}{\pgfqpoint{0.000000in}{0.000000in}}{%
\pgfpathmoveto{\pgfqpoint{0.000000in}{0.000000in}}%
\pgfpathlineto{\pgfqpoint{-0.048611in}{0.000000in}}%
\pgfusepath{stroke,fill}%
}%
\begin{pgfscope}%
\pgfsys@transformshift{0.374692in}{2.484603in}%
\pgfsys@useobject{currentmarker}{}%
\end{pgfscope}%
\end{pgfscope}%
\begin{pgfscope}%
\pgftext[x=0.208025in,y=2.431842in,left,base]{\rmfamily\fontsize{10.000000}{12.000000}\selectfont \(\displaystyle 5\)}%
\end{pgfscope}%
\begin{pgfscope}%
\pgfpathrectangle{\pgfqpoint{0.374692in}{0.521603in}}{\pgfqpoint{2.635000in}{1.963000in}} %
\pgfusepath{clip}%
\pgfsetbuttcap%
\pgfsetroundjoin%
\pgfsetlinewidth{1.505625pt}%
\definecolor{currentstroke}{rgb}{0.000000,0.000000,0.000000}%
\pgfsetstrokecolor{currentstroke}%
\pgfsetdash{{5.550000pt}{2.400000pt}}{0.000000pt}%
\pgfpathmoveto{\pgfqpoint{2.131359in}{0.521603in}}%
\pgfpathlineto{\pgfqpoint{2.131359in}{0.630659in}}%
\pgfpathlineto{\pgfqpoint{2.131359in}{0.739714in}}%
\pgfpathlineto{\pgfqpoint{2.131359in}{0.848770in}}%
\pgfpathlineto{\pgfqpoint{2.131359in}{0.957826in}}%
\pgfpathlineto{\pgfqpoint{2.131359in}{1.066881in}}%
\pgfpathlineto{\pgfqpoint{2.131359in}{1.175937in}}%
\pgfpathlineto{\pgfqpoint{2.131359in}{1.284992in}}%
\pgfpathlineto{\pgfqpoint{2.131359in}{1.394048in}}%
\pgfpathlineto{\pgfqpoint{2.131359in}{1.503103in}}%
\pgfusepath{stroke}%
\end{pgfscope}%
\begin{pgfscope}%
\pgfpathrectangle{\pgfqpoint{0.374692in}{0.521603in}}{\pgfqpoint{2.635000in}{1.963000in}} %
\pgfusepath{clip}%
\pgfsetrectcap%
\pgfsetroundjoin%
\pgfsetlinewidth{1.003750pt}%
\definecolor{currentstroke}{rgb}{0.121569,0.466667,0.705882}%
\pgfsetstrokecolor{currentstroke}%
\pgfsetdash{}{0pt}%
\pgfpathmoveto{\pgfqpoint{0.374692in}{1.339520in}}%
\pgfpathlineto{\pgfqpoint{0.383564in}{1.332911in}}%
\pgfpathlineto{\pgfqpoint{0.392436in}{1.326301in}}%
\pgfpathlineto{\pgfqpoint{0.401308in}{1.319692in}}%
\pgfpathlineto{\pgfqpoint{0.410180in}{1.313082in}}%
\pgfpathlineto{\pgfqpoint{0.419052in}{1.306473in}}%
\pgfpathlineto{\pgfqpoint{0.427924in}{1.299863in}}%
\pgfpathlineto{\pgfqpoint{0.436796in}{1.293254in}}%
\pgfpathlineto{\pgfqpoint{0.445668in}{1.286645in}}%
\pgfpathlineto{\pgfqpoint{0.454540in}{1.280035in}}%
\pgfpathlineto{\pgfqpoint{0.463412in}{1.273426in}}%
\pgfpathlineto{\pgfqpoint{0.472285in}{1.266816in}}%
\pgfpathlineto{\pgfqpoint{0.481157in}{1.260207in}}%
\pgfpathlineto{\pgfqpoint{0.490029in}{1.253597in}}%
\pgfpathlineto{\pgfqpoint{0.498901in}{1.246988in}}%
\pgfpathlineto{\pgfqpoint{0.507773in}{1.240379in}}%
\pgfpathlineto{\pgfqpoint{0.516645in}{1.233769in}}%
\pgfpathlineto{\pgfqpoint{0.525517in}{1.227160in}}%
\pgfpathlineto{\pgfqpoint{0.534389in}{1.220550in}}%
\pgfpathlineto{\pgfqpoint{0.543261in}{1.213941in}}%
\pgfpathlineto{\pgfqpoint{0.552133in}{1.207331in}}%
\pgfpathlineto{\pgfqpoint{0.561005in}{1.200722in}}%
\pgfpathlineto{\pgfqpoint{0.569877in}{1.194113in}}%
\pgfpathlineto{\pgfqpoint{0.578749in}{1.187503in}}%
\pgfpathlineto{\pgfqpoint{0.587621in}{1.180894in}}%
\pgfpathlineto{\pgfqpoint{0.596493in}{1.174284in}}%
\pgfpathlineto{\pgfqpoint{0.605365in}{1.167675in}}%
\pgfpathlineto{\pgfqpoint{0.614237in}{1.161065in}}%
\pgfpathlineto{\pgfqpoint{0.623109in}{1.154456in}}%
\pgfpathlineto{\pgfqpoint{0.631982in}{1.147847in}}%
\pgfpathlineto{\pgfqpoint{0.640854in}{1.141237in}}%
\pgfpathlineto{\pgfqpoint{0.649726in}{1.134628in}}%
\pgfpathlineto{\pgfqpoint{0.658598in}{1.128018in}}%
\pgfpathlineto{\pgfqpoint{0.667470in}{1.121409in}}%
\pgfpathlineto{\pgfqpoint{0.676342in}{1.114799in}}%
\pgfpathlineto{\pgfqpoint{0.685214in}{1.108190in}}%
\pgfpathlineto{\pgfqpoint{0.694086in}{1.101581in}}%
\pgfpathlineto{\pgfqpoint{0.702958in}{1.094971in}}%
\pgfpathlineto{\pgfqpoint{0.711830in}{1.088362in}}%
\pgfpathlineto{\pgfqpoint{0.720702in}{1.081752in}}%
\pgfpathlineto{\pgfqpoint{0.729574in}{1.075143in}}%
\pgfpathlineto{\pgfqpoint{0.738446in}{1.068533in}}%
\pgfpathlineto{\pgfqpoint{0.747318in}{1.061924in}}%
\pgfpathlineto{\pgfqpoint{0.756190in}{1.055315in}}%
\pgfpathlineto{\pgfqpoint{0.765062in}{1.048705in}}%
\pgfpathlineto{\pgfqpoint{0.773934in}{1.042096in}}%
\pgfpathlineto{\pgfqpoint{0.782806in}{1.035486in}}%
\pgfpathlineto{\pgfqpoint{0.791678in}{1.028877in}}%
\pgfpathlineto{\pgfqpoint{0.800551in}{1.022267in}}%
\pgfpathlineto{\pgfqpoint{0.809423in}{1.015658in}}%
\pgfpathlineto{\pgfqpoint{0.818295in}{1.009049in}}%
\pgfpathlineto{\pgfqpoint{0.827167in}{1.002439in}}%
\pgfpathlineto{\pgfqpoint{0.836039in}{0.995830in}}%
\pgfpathlineto{\pgfqpoint{0.844911in}{0.989220in}}%
\pgfpathlineto{\pgfqpoint{0.853783in}{0.982611in}}%
\pgfpathlineto{\pgfqpoint{0.862655in}{0.976001in}}%
\pgfpathlineto{\pgfqpoint{0.871527in}{0.969392in}}%
\pgfpathlineto{\pgfqpoint{0.880399in}{0.962783in}}%
\pgfpathlineto{\pgfqpoint{0.889271in}{0.956173in}}%
\pgfpathlineto{\pgfqpoint{0.898143in}{0.949564in}}%
\pgfpathlineto{\pgfqpoint{0.907015in}{0.942954in}}%
\pgfpathlineto{\pgfqpoint{0.915887in}{0.936345in}}%
\pgfpathlineto{\pgfqpoint{0.924759in}{0.929735in}}%
\pgfpathlineto{\pgfqpoint{0.933631in}{0.923126in}}%
\pgfpathlineto{\pgfqpoint{0.942503in}{0.916517in}}%
\pgfpathlineto{\pgfqpoint{0.951375in}{0.909907in}}%
\pgfpathlineto{\pgfqpoint{0.960248in}{0.903298in}}%
\pgfpathlineto{\pgfqpoint{0.969120in}{0.896688in}}%
\pgfpathlineto{\pgfqpoint{0.977992in}{0.890079in}}%
\pgfpathlineto{\pgfqpoint{0.986864in}{0.883469in}}%
\pgfpathlineto{\pgfqpoint{0.995736in}{0.876860in}}%
\pgfpathlineto{\pgfqpoint{1.004608in}{0.870251in}}%
\pgfpathlineto{\pgfqpoint{1.013480in}{0.863641in}}%
\pgfpathlineto{\pgfqpoint{1.022352in}{0.857032in}}%
\pgfpathlineto{\pgfqpoint{1.031224in}{0.850422in}}%
\pgfpathlineto{\pgfqpoint{1.040096in}{0.843813in}}%
\pgfpathlineto{\pgfqpoint{1.048968in}{0.837204in}}%
\pgfpathlineto{\pgfqpoint{1.057840in}{0.830594in}}%
\pgfpathlineto{\pgfqpoint{1.066712in}{0.823985in}}%
\pgfpathlineto{\pgfqpoint{1.075584in}{0.817375in}}%
\pgfpathlineto{\pgfqpoint{1.084456in}{0.810766in}}%
\pgfpathlineto{\pgfqpoint{1.093328in}{0.804156in}}%
\pgfpathlineto{\pgfqpoint{1.102200in}{0.797547in}}%
\pgfpathlineto{\pgfqpoint{1.111072in}{0.790938in}}%
\pgfpathlineto{\pgfqpoint{1.119944in}{0.784328in}}%
\pgfpathlineto{\pgfqpoint{1.128817in}{0.777719in}}%
\pgfpathlineto{\pgfqpoint{1.137689in}{0.771109in}}%
\pgfpathlineto{\pgfqpoint{1.146561in}{0.764500in}}%
\pgfpathlineto{\pgfqpoint{1.155433in}{0.757890in}}%
\pgfpathlineto{\pgfqpoint{1.164305in}{0.751281in}}%
\pgfpathlineto{\pgfqpoint{1.173177in}{0.744672in}}%
\pgfpathlineto{\pgfqpoint{1.182049in}{0.738062in}}%
\pgfpathlineto{\pgfqpoint{1.190921in}{0.731453in}}%
\pgfpathlineto{\pgfqpoint{1.199793in}{0.724843in}}%
\pgfpathlineto{\pgfqpoint{1.208665in}{0.718234in}}%
\pgfpathlineto{\pgfqpoint{1.217537in}{0.711624in}}%
\pgfpathlineto{\pgfqpoint{1.226409in}{0.705015in}}%
\pgfpathlineto{\pgfqpoint{1.235281in}{0.698406in}}%
\pgfpathlineto{\pgfqpoint{1.244153in}{0.691796in}}%
\pgfpathlineto{\pgfqpoint{1.253025in}{0.685187in}}%
\pgfusepath{stroke}%
\end{pgfscope}%
\begin{pgfscope}%
\pgfpathrectangle{\pgfqpoint{0.374692in}{0.521603in}}{\pgfqpoint{2.635000in}{1.963000in}} %
\pgfusepath{clip}%
\pgfsetrectcap%
\pgfsetroundjoin%
\pgfsetlinewidth{1.003750pt}%
\definecolor{currentstroke}{rgb}{1.000000,0.498039,0.054902}%
\pgfsetstrokecolor{currentstroke}%
\pgfsetdash{}{0pt}%
\pgfpathmoveto{\pgfqpoint{0.374692in}{0.685187in}}%
\pgfpathlineto{\pgfqpoint{0.383564in}{0.691796in}}%
\pgfpathlineto{\pgfqpoint{0.392436in}{0.698406in}}%
\pgfpathlineto{\pgfqpoint{0.401308in}{0.705015in}}%
\pgfpathlineto{\pgfqpoint{0.410180in}{0.711624in}}%
\pgfpathlineto{\pgfqpoint{0.419052in}{0.718234in}}%
\pgfpathlineto{\pgfqpoint{0.427924in}{0.724843in}}%
\pgfpathlineto{\pgfqpoint{0.436796in}{0.731453in}}%
\pgfpathlineto{\pgfqpoint{0.445668in}{0.738062in}}%
\pgfpathlineto{\pgfqpoint{0.454540in}{0.744672in}}%
\pgfpathlineto{\pgfqpoint{0.463412in}{0.751281in}}%
\pgfpathlineto{\pgfqpoint{0.472285in}{0.757890in}}%
\pgfpathlineto{\pgfqpoint{0.481157in}{0.764500in}}%
\pgfpathlineto{\pgfqpoint{0.490029in}{0.771109in}}%
\pgfpathlineto{\pgfqpoint{0.498901in}{0.777719in}}%
\pgfpathlineto{\pgfqpoint{0.507773in}{0.784328in}}%
\pgfpathlineto{\pgfqpoint{0.516645in}{0.790938in}}%
\pgfpathlineto{\pgfqpoint{0.525517in}{0.797547in}}%
\pgfpathlineto{\pgfqpoint{0.534389in}{0.804156in}}%
\pgfpathlineto{\pgfqpoint{0.543261in}{0.810766in}}%
\pgfpathlineto{\pgfqpoint{0.552133in}{0.817375in}}%
\pgfpathlineto{\pgfqpoint{0.561005in}{0.823985in}}%
\pgfpathlineto{\pgfqpoint{0.569877in}{0.830594in}}%
\pgfpathlineto{\pgfqpoint{0.578749in}{0.837204in}}%
\pgfpathlineto{\pgfqpoint{0.587621in}{0.843813in}}%
\pgfpathlineto{\pgfqpoint{0.596493in}{0.850422in}}%
\pgfpathlineto{\pgfqpoint{0.605365in}{0.857032in}}%
\pgfpathlineto{\pgfqpoint{0.614237in}{0.863641in}}%
\pgfpathlineto{\pgfqpoint{0.623109in}{0.870251in}}%
\pgfpathlineto{\pgfqpoint{0.631982in}{0.876860in}}%
\pgfpathlineto{\pgfqpoint{0.640854in}{0.883469in}}%
\pgfpathlineto{\pgfqpoint{0.649726in}{0.890079in}}%
\pgfpathlineto{\pgfqpoint{0.658598in}{0.896688in}}%
\pgfpathlineto{\pgfqpoint{0.667470in}{0.903298in}}%
\pgfpathlineto{\pgfqpoint{0.676342in}{0.909907in}}%
\pgfpathlineto{\pgfqpoint{0.685214in}{0.916517in}}%
\pgfpathlineto{\pgfqpoint{0.694086in}{0.923126in}}%
\pgfpathlineto{\pgfqpoint{0.702958in}{0.929735in}}%
\pgfpathlineto{\pgfqpoint{0.711830in}{0.936345in}}%
\pgfpathlineto{\pgfqpoint{0.720702in}{0.942954in}}%
\pgfpathlineto{\pgfqpoint{0.729574in}{0.949564in}}%
\pgfpathlineto{\pgfqpoint{0.738446in}{0.956173in}}%
\pgfpathlineto{\pgfqpoint{0.747318in}{0.962783in}}%
\pgfpathlineto{\pgfqpoint{0.756190in}{0.969392in}}%
\pgfpathlineto{\pgfqpoint{0.765062in}{0.976001in}}%
\pgfpathlineto{\pgfqpoint{0.773934in}{0.982611in}}%
\pgfpathlineto{\pgfqpoint{0.782806in}{0.989220in}}%
\pgfpathlineto{\pgfqpoint{0.791678in}{0.995830in}}%
\pgfpathlineto{\pgfqpoint{0.800551in}{1.002439in}}%
\pgfpathlineto{\pgfqpoint{0.809423in}{1.009049in}}%
\pgfpathlineto{\pgfqpoint{0.818295in}{1.015658in}}%
\pgfpathlineto{\pgfqpoint{0.827167in}{1.022267in}}%
\pgfpathlineto{\pgfqpoint{0.836039in}{1.028877in}}%
\pgfpathlineto{\pgfqpoint{0.844911in}{1.035486in}}%
\pgfpathlineto{\pgfqpoint{0.853783in}{1.042096in}}%
\pgfpathlineto{\pgfqpoint{0.862655in}{1.048705in}}%
\pgfpathlineto{\pgfqpoint{0.871527in}{1.055315in}}%
\pgfpathlineto{\pgfqpoint{0.880399in}{1.061924in}}%
\pgfpathlineto{\pgfqpoint{0.889271in}{1.068533in}}%
\pgfpathlineto{\pgfqpoint{0.898143in}{1.075143in}}%
\pgfpathlineto{\pgfqpoint{0.907015in}{1.081752in}}%
\pgfpathlineto{\pgfqpoint{0.915887in}{1.088362in}}%
\pgfpathlineto{\pgfqpoint{0.924759in}{1.094971in}}%
\pgfpathlineto{\pgfqpoint{0.933631in}{1.101581in}}%
\pgfpathlineto{\pgfqpoint{0.942503in}{1.108190in}}%
\pgfpathlineto{\pgfqpoint{0.951375in}{1.114799in}}%
\pgfpathlineto{\pgfqpoint{0.960248in}{1.121409in}}%
\pgfpathlineto{\pgfqpoint{0.969120in}{1.128018in}}%
\pgfpathlineto{\pgfqpoint{0.977992in}{1.134628in}}%
\pgfpathlineto{\pgfqpoint{0.986864in}{1.141237in}}%
\pgfpathlineto{\pgfqpoint{0.995736in}{1.147847in}}%
\pgfpathlineto{\pgfqpoint{1.004608in}{1.154456in}}%
\pgfpathlineto{\pgfqpoint{1.013480in}{1.161065in}}%
\pgfpathlineto{\pgfqpoint{1.022352in}{1.167675in}}%
\pgfpathlineto{\pgfqpoint{1.031224in}{1.174284in}}%
\pgfpathlineto{\pgfqpoint{1.040096in}{1.180894in}}%
\pgfpathlineto{\pgfqpoint{1.048968in}{1.187503in}}%
\pgfpathlineto{\pgfqpoint{1.057840in}{1.194113in}}%
\pgfpathlineto{\pgfqpoint{1.066712in}{1.200722in}}%
\pgfpathlineto{\pgfqpoint{1.075584in}{1.207331in}}%
\pgfpathlineto{\pgfqpoint{1.084456in}{1.213941in}}%
\pgfpathlineto{\pgfqpoint{1.093328in}{1.220550in}}%
\pgfpathlineto{\pgfqpoint{1.102200in}{1.227160in}}%
\pgfpathlineto{\pgfqpoint{1.111072in}{1.233769in}}%
\pgfpathlineto{\pgfqpoint{1.119944in}{1.240379in}}%
\pgfpathlineto{\pgfqpoint{1.128817in}{1.246988in}}%
\pgfpathlineto{\pgfqpoint{1.137689in}{1.253597in}}%
\pgfpathlineto{\pgfqpoint{1.146561in}{1.260207in}}%
\pgfpathlineto{\pgfqpoint{1.155433in}{1.266816in}}%
\pgfpathlineto{\pgfqpoint{1.164305in}{1.273426in}}%
\pgfpathlineto{\pgfqpoint{1.173177in}{1.280035in}}%
\pgfpathlineto{\pgfqpoint{1.182049in}{1.286645in}}%
\pgfpathlineto{\pgfqpoint{1.190921in}{1.293254in}}%
\pgfpathlineto{\pgfqpoint{1.199793in}{1.299863in}}%
\pgfpathlineto{\pgfqpoint{1.208665in}{1.306473in}}%
\pgfpathlineto{\pgfqpoint{1.217537in}{1.313082in}}%
\pgfpathlineto{\pgfqpoint{1.226409in}{1.319692in}}%
\pgfpathlineto{\pgfqpoint{1.235281in}{1.326301in}}%
\pgfpathlineto{\pgfqpoint{1.244153in}{1.332911in}}%
\pgfpathlineto{\pgfqpoint{1.253025in}{1.339520in}}%
\pgfusepath{stroke}%
\end{pgfscope}%
\begin{pgfscope}%
\pgfpathrectangle{\pgfqpoint{0.374692in}{0.521603in}}{\pgfqpoint{2.635000in}{1.963000in}} %
\pgfusepath{clip}%
\pgfsetbuttcap%
\pgfsetroundjoin%
\definecolor{currentfill}{rgb}{1.000000,0.000000,0.000000}%
\pgfsetfillcolor{currentfill}%
\pgfsetlinewidth{1.003750pt}%
\definecolor{currentstroke}{rgb}{1.000000,0.000000,0.000000}%
\pgfsetstrokecolor{currentstroke}%
\pgfsetdash{}{0pt}%
\pgfsys@defobject{currentmarker}{\pgfqpoint{-0.020833in}{-0.020833in}}{\pgfqpoint{0.020833in}{0.020833in}}{%
\pgfpathmoveto{\pgfqpoint{0.000000in}{-0.020833in}}%
\pgfpathcurveto{\pgfqpoint{0.005525in}{-0.020833in}}{\pgfqpoint{0.010825in}{-0.018638in}}{\pgfqpoint{0.014731in}{-0.014731in}}%
\pgfpathcurveto{\pgfqpoint{0.018638in}{-0.010825in}}{\pgfqpoint{0.020833in}{-0.005525in}}{\pgfqpoint{0.020833in}{0.000000in}}%
\pgfpathcurveto{\pgfqpoint{0.020833in}{0.005525in}}{\pgfqpoint{0.018638in}{0.010825in}}{\pgfqpoint{0.014731in}{0.014731in}}%
\pgfpathcurveto{\pgfqpoint{0.010825in}{0.018638in}}{\pgfqpoint{0.005525in}{0.020833in}}{\pgfqpoint{0.000000in}{0.020833in}}%
\pgfpathcurveto{\pgfqpoint{-0.005525in}{0.020833in}}{\pgfqpoint{-0.010825in}{0.018638in}}{\pgfqpoint{-0.014731in}{0.014731in}}%
\pgfpathcurveto{\pgfqpoint{-0.018638in}{0.010825in}}{\pgfqpoint{-0.020833in}{0.005525in}}{\pgfqpoint{-0.020833in}{0.000000in}}%
\pgfpathcurveto{\pgfqpoint{-0.020833in}{-0.005525in}}{\pgfqpoint{-0.018638in}{-0.010825in}}{\pgfqpoint{-0.014731in}{-0.014731in}}%
\pgfpathcurveto{\pgfqpoint{-0.010825in}{-0.018638in}}{\pgfqpoint{-0.005525in}{-0.020833in}}{\pgfqpoint{0.000000in}{-0.020833in}}%
\pgfpathclose%
\pgfusepath{stroke,fill}%
}%
\begin{pgfscope}%
\pgfsys@transformshift{0.374692in}{0.848770in}%
\pgfsys@useobject{currentmarker}{}%
\end{pgfscope}%
\begin{pgfscope}%
\pgfsys@transformshift{0.813859in}{0.848770in}%
\pgfsys@useobject{currentmarker}{}%
\end{pgfscope}%
\begin{pgfscope}%
\pgfsys@transformshift{1.253025in}{0.848770in}%
\pgfsys@useobject{currentmarker}{}%
\end{pgfscope}%
\end{pgfscope}%
\begin{pgfscope}%
\pgfpathrectangle{\pgfqpoint{0.374692in}{0.521603in}}{\pgfqpoint{2.635000in}{1.963000in}} %
\pgfusepath{clip}%
\pgfsetbuttcap%
\pgfsetroundjoin%
\definecolor{currentfill}{rgb}{1.000000,0.000000,0.000000}%
\pgfsetfillcolor{currentfill}%
\pgfsetlinewidth{1.003750pt}%
\definecolor{currentstroke}{rgb}{1.000000,0.000000,0.000000}%
\pgfsetstrokecolor{currentstroke}%
\pgfsetdash{}{0pt}%
\pgfsys@defobject{currentmarker}{\pgfqpoint{-0.020833in}{-0.020833in}}{\pgfqpoint{0.020833in}{0.020833in}}{%
\pgfpathmoveto{\pgfqpoint{0.000000in}{-0.020833in}}%
\pgfpathcurveto{\pgfqpoint{0.005525in}{-0.020833in}}{\pgfqpoint{0.010825in}{-0.018638in}}{\pgfqpoint{0.014731in}{-0.014731in}}%
\pgfpathcurveto{\pgfqpoint{0.018638in}{-0.010825in}}{\pgfqpoint{0.020833in}{-0.005525in}}{\pgfqpoint{0.020833in}{0.000000in}}%
\pgfpathcurveto{\pgfqpoint{0.020833in}{0.005525in}}{\pgfqpoint{0.018638in}{0.010825in}}{\pgfqpoint{0.014731in}{0.014731in}}%
\pgfpathcurveto{\pgfqpoint{0.010825in}{0.018638in}}{\pgfqpoint{0.005525in}{0.020833in}}{\pgfqpoint{0.000000in}{0.020833in}}%
\pgfpathcurveto{\pgfqpoint{-0.005525in}{0.020833in}}{\pgfqpoint{-0.010825in}{0.018638in}}{\pgfqpoint{-0.014731in}{0.014731in}}%
\pgfpathcurveto{\pgfqpoint{-0.018638in}{0.010825in}}{\pgfqpoint{-0.020833in}{0.005525in}}{\pgfqpoint{-0.020833in}{0.000000in}}%
\pgfpathcurveto{\pgfqpoint{-0.020833in}{-0.005525in}}{\pgfqpoint{-0.018638in}{-0.010825in}}{\pgfqpoint{-0.014731in}{-0.014731in}}%
\pgfpathcurveto{\pgfqpoint{-0.010825in}{-0.018638in}}{\pgfqpoint{-0.005525in}{-0.020833in}}{\pgfqpoint{0.000000in}{-0.020833in}}%
\pgfpathclose%
\pgfusepath{stroke,fill}%
}%
\begin{pgfscope}%
\pgfsys@transformshift{0.374692in}{0.848770in}%
\pgfsys@useobject{currentmarker}{}%
\end{pgfscope}%
\begin{pgfscope}%
\pgfsys@transformshift{0.813859in}{0.848770in}%
\pgfsys@useobject{currentmarker}{}%
\end{pgfscope}%
\begin{pgfscope}%
\pgfsys@transformshift{1.253025in}{0.848770in}%
\pgfsys@useobject{currentmarker}{}%
\end{pgfscope}%
\end{pgfscope}%
\begin{pgfscope}%
\pgfpathrectangle{\pgfqpoint{0.374692in}{0.521603in}}{\pgfqpoint{2.635000in}{1.963000in}} %
\pgfusepath{clip}%
\pgfsetbuttcap%
\pgfsetroundjoin%
\pgfsetlinewidth{1.505625pt}%
\definecolor{currentstroke}{rgb}{0.000000,0.000000,0.000000}%
\pgfsetstrokecolor{currentstroke}%
\pgfsetdash{{5.550000pt}{2.400000pt}}{0.000000pt}%
\pgfpathmoveto{\pgfqpoint{1.253025in}{0.521603in}}%
\pgfpathlineto{\pgfqpoint{1.253025in}{0.630659in}}%
\pgfpathlineto{\pgfqpoint{1.253025in}{0.739714in}}%
\pgfpathlineto{\pgfqpoint{1.253025in}{0.848770in}}%
\pgfpathlineto{\pgfqpoint{1.253025in}{0.957826in}}%
\pgfpathlineto{\pgfqpoint{1.253025in}{1.066881in}}%
\pgfpathlineto{\pgfqpoint{1.253025in}{1.175937in}}%
\pgfpathlineto{\pgfqpoint{1.253025in}{1.284992in}}%
\pgfpathlineto{\pgfqpoint{1.253025in}{1.394048in}}%
\pgfpathlineto{\pgfqpoint{1.253025in}{1.503103in}}%
\pgfusepath{stroke}%
\end{pgfscope}%
\begin{pgfscope}%
\pgfpathrectangle{\pgfqpoint{0.374692in}{0.521603in}}{\pgfqpoint{2.635000in}{1.963000in}} %
\pgfusepath{clip}%
\pgfsetrectcap%
\pgfsetroundjoin%
\pgfsetlinewidth{1.003750pt}%
\definecolor{currentstroke}{rgb}{0.172549,0.627451,0.172549}%
\pgfsetstrokecolor{currentstroke}%
\pgfsetdash{}{0pt}%
\pgfpathmoveto{\pgfqpoint{1.253025in}{1.339520in}}%
\pgfpathlineto{\pgfqpoint{1.261897in}{1.332911in}}%
\pgfpathlineto{\pgfqpoint{1.270769in}{1.326301in}}%
\pgfpathlineto{\pgfqpoint{1.279641in}{1.319692in}}%
\pgfpathlineto{\pgfqpoint{1.288513in}{1.313082in}}%
\pgfpathlineto{\pgfqpoint{1.297386in}{1.306473in}}%
\pgfpathlineto{\pgfqpoint{1.306258in}{1.299863in}}%
\pgfpathlineto{\pgfqpoint{1.315130in}{1.293254in}}%
\pgfpathlineto{\pgfqpoint{1.324002in}{1.286645in}}%
\pgfpathlineto{\pgfqpoint{1.332874in}{1.280035in}}%
\pgfpathlineto{\pgfqpoint{1.341746in}{1.273426in}}%
\pgfpathlineto{\pgfqpoint{1.350618in}{1.266816in}}%
\pgfpathlineto{\pgfqpoint{1.359490in}{1.260207in}}%
\pgfpathlineto{\pgfqpoint{1.368362in}{1.253597in}}%
\pgfpathlineto{\pgfqpoint{1.377234in}{1.246988in}}%
\pgfpathlineto{\pgfqpoint{1.386106in}{1.240379in}}%
\pgfpathlineto{\pgfqpoint{1.394978in}{1.233769in}}%
\pgfpathlineto{\pgfqpoint{1.403850in}{1.227160in}}%
\pgfpathlineto{\pgfqpoint{1.412722in}{1.220550in}}%
\pgfpathlineto{\pgfqpoint{1.421594in}{1.213941in}}%
\pgfpathlineto{\pgfqpoint{1.430466in}{1.207331in}}%
\pgfpathlineto{\pgfqpoint{1.439338in}{1.200722in}}%
\pgfpathlineto{\pgfqpoint{1.448210in}{1.194113in}}%
\pgfpathlineto{\pgfqpoint{1.457083in}{1.187503in}}%
\pgfpathlineto{\pgfqpoint{1.465955in}{1.180894in}}%
\pgfpathlineto{\pgfqpoint{1.474827in}{1.174284in}}%
\pgfpathlineto{\pgfqpoint{1.483699in}{1.167675in}}%
\pgfpathlineto{\pgfqpoint{1.492571in}{1.161065in}}%
\pgfpathlineto{\pgfqpoint{1.501443in}{1.154456in}}%
\pgfpathlineto{\pgfqpoint{1.510315in}{1.147847in}}%
\pgfpathlineto{\pgfqpoint{1.519187in}{1.141237in}}%
\pgfpathlineto{\pgfqpoint{1.528059in}{1.134628in}}%
\pgfpathlineto{\pgfqpoint{1.536931in}{1.128018in}}%
\pgfpathlineto{\pgfqpoint{1.545803in}{1.121409in}}%
\pgfpathlineto{\pgfqpoint{1.554675in}{1.114799in}}%
\pgfpathlineto{\pgfqpoint{1.563547in}{1.108190in}}%
\pgfpathlineto{\pgfqpoint{1.572419in}{1.101581in}}%
\pgfpathlineto{\pgfqpoint{1.581291in}{1.094971in}}%
\pgfpathlineto{\pgfqpoint{1.590163in}{1.088362in}}%
\pgfpathlineto{\pgfqpoint{1.599035in}{1.081752in}}%
\pgfpathlineto{\pgfqpoint{1.607907in}{1.075143in}}%
\pgfpathlineto{\pgfqpoint{1.616779in}{1.068533in}}%
\pgfpathlineto{\pgfqpoint{1.625652in}{1.061924in}}%
\pgfpathlineto{\pgfqpoint{1.634524in}{1.055315in}}%
\pgfpathlineto{\pgfqpoint{1.643396in}{1.048705in}}%
\pgfpathlineto{\pgfqpoint{1.652268in}{1.042096in}}%
\pgfpathlineto{\pgfqpoint{1.661140in}{1.035486in}}%
\pgfpathlineto{\pgfqpoint{1.670012in}{1.028877in}}%
\pgfpathlineto{\pgfqpoint{1.678884in}{1.022267in}}%
\pgfpathlineto{\pgfqpoint{1.687756in}{1.015658in}}%
\pgfpathlineto{\pgfqpoint{1.696628in}{1.009049in}}%
\pgfpathlineto{\pgfqpoint{1.705500in}{1.002439in}}%
\pgfpathlineto{\pgfqpoint{1.714372in}{0.995830in}}%
\pgfpathlineto{\pgfqpoint{1.723244in}{0.989220in}}%
\pgfpathlineto{\pgfqpoint{1.732116in}{0.982611in}}%
\pgfpathlineto{\pgfqpoint{1.740988in}{0.976001in}}%
\pgfpathlineto{\pgfqpoint{1.749860in}{0.969392in}}%
\pgfpathlineto{\pgfqpoint{1.758732in}{0.962783in}}%
\pgfpathlineto{\pgfqpoint{1.767604in}{0.956173in}}%
\pgfpathlineto{\pgfqpoint{1.776476in}{0.949564in}}%
\pgfpathlineto{\pgfqpoint{1.785349in}{0.942954in}}%
\pgfpathlineto{\pgfqpoint{1.794221in}{0.936345in}}%
\pgfpathlineto{\pgfqpoint{1.803093in}{0.929735in}}%
\pgfpathlineto{\pgfqpoint{1.811965in}{0.923126in}}%
\pgfpathlineto{\pgfqpoint{1.820837in}{0.916517in}}%
\pgfpathlineto{\pgfqpoint{1.829709in}{0.909907in}}%
\pgfpathlineto{\pgfqpoint{1.838581in}{0.903298in}}%
\pgfpathlineto{\pgfqpoint{1.847453in}{0.896688in}}%
\pgfpathlineto{\pgfqpoint{1.856325in}{0.890079in}}%
\pgfpathlineto{\pgfqpoint{1.865197in}{0.883469in}}%
\pgfpathlineto{\pgfqpoint{1.874069in}{0.876860in}}%
\pgfpathlineto{\pgfqpoint{1.882941in}{0.870251in}}%
\pgfpathlineto{\pgfqpoint{1.891813in}{0.863641in}}%
\pgfpathlineto{\pgfqpoint{1.900685in}{0.857032in}}%
\pgfpathlineto{\pgfqpoint{1.909557in}{0.850422in}}%
\pgfpathlineto{\pgfqpoint{1.918429in}{0.843813in}}%
\pgfpathlineto{\pgfqpoint{1.927301in}{0.837204in}}%
\pgfpathlineto{\pgfqpoint{1.936173in}{0.830594in}}%
\pgfpathlineto{\pgfqpoint{1.945045in}{0.823985in}}%
\pgfpathlineto{\pgfqpoint{1.953918in}{0.817375in}}%
\pgfpathlineto{\pgfqpoint{1.962790in}{0.810766in}}%
\pgfpathlineto{\pgfqpoint{1.971662in}{0.804156in}}%
\pgfpathlineto{\pgfqpoint{1.980534in}{0.797547in}}%
\pgfpathlineto{\pgfqpoint{1.989406in}{0.790938in}}%
\pgfpathlineto{\pgfqpoint{1.998278in}{0.784328in}}%
\pgfpathlineto{\pgfqpoint{2.007150in}{0.777719in}}%
\pgfpathlineto{\pgfqpoint{2.016022in}{0.771109in}}%
\pgfpathlineto{\pgfqpoint{2.024894in}{0.764500in}}%
\pgfpathlineto{\pgfqpoint{2.033766in}{0.757890in}}%
\pgfpathlineto{\pgfqpoint{2.042638in}{0.751281in}}%
\pgfpathlineto{\pgfqpoint{2.051510in}{0.744672in}}%
\pgfpathlineto{\pgfqpoint{2.060382in}{0.738062in}}%
\pgfpathlineto{\pgfqpoint{2.069254in}{0.731453in}}%
\pgfpathlineto{\pgfqpoint{2.078126in}{0.724843in}}%
\pgfpathlineto{\pgfqpoint{2.086998in}{0.718234in}}%
\pgfpathlineto{\pgfqpoint{2.095870in}{0.711624in}}%
\pgfpathlineto{\pgfqpoint{2.104742in}{0.705015in}}%
\pgfpathlineto{\pgfqpoint{2.113615in}{0.698406in}}%
\pgfpathlineto{\pgfqpoint{2.122487in}{0.691796in}}%
\pgfpathlineto{\pgfqpoint{2.131359in}{0.685187in}}%
\pgfusepath{stroke}%
\end{pgfscope}%
\begin{pgfscope}%
\pgfpathrectangle{\pgfqpoint{0.374692in}{0.521603in}}{\pgfqpoint{2.635000in}{1.963000in}} %
\pgfusepath{clip}%
\pgfsetrectcap%
\pgfsetroundjoin%
\pgfsetlinewidth{1.003750pt}%
\definecolor{currentstroke}{rgb}{0.839216,0.152941,0.156863}%
\pgfsetstrokecolor{currentstroke}%
\pgfsetdash{}{0pt}%
\pgfpathmoveto{\pgfqpoint{1.253025in}{0.685187in}}%
\pgfpathlineto{\pgfqpoint{1.261897in}{0.691796in}}%
\pgfpathlineto{\pgfqpoint{1.270769in}{0.698406in}}%
\pgfpathlineto{\pgfqpoint{1.279641in}{0.705015in}}%
\pgfpathlineto{\pgfqpoint{1.288513in}{0.711624in}}%
\pgfpathlineto{\pgfqpoint{1.297386in}{0.718234in}}%
\pgfpathlineto{\pgfqpoint{1.306258in}{0.724843in}}%
\pgfpathlineto{\pgfqpoint{1.315130in}{0.731453in}}%
\pgfpathlineto{\pgfqpoint{1.324002in}{0.738062in}}%
\pgfpathlineto{\pgfqpoint{1.332874in}{0.744672in}}%
\pgfpathlineto{\pgfqpoint{1.341746in}{0.751281in}}%
\pgfpathlineto{\pgfqpoint{1.350618in}{0.757890in}}%
\pgfpathlineto{\pgfqpoint{1.359490in}{0.764500in}}%
\pgfpathlineto{\pgfqpoint{1.368362in}{0.771109in}}%
\pgfpathlineto{\pgfqpoint{1.377234in}{0.777719in}}%
\pgfpathlineto{\pgfqpoint{1.386106in}{0.784328in}}%
\pgfpathlineto{\pgfqpoint{1.394978in}{0.790938in}}%
\pgfpathlineto{\pgfqpoint{1.403850in}{0.797547in}}%
\pgfpathlineto{\pgfqpoint{1.412722in}{0.804156in}}%
\pgfpathlineto{\pgfqpoint{1.421594in}{0.810766in}}%
\pgfpathlineto{\pgfqpoint{1.430466in}{0.817375in}}%
\pgfpathlineto{\pgfqpoint{1.439338in}{0.823985in}}%
\pgfpathlineto{\pgfqpoint{1.448210in}{0.830594in}}%
\pgfpathlineto{\pgfqpoint{1.457083in}{0.837204in}}%
\pgfpathlineto{\pgfqpoint{1.465955in}{0.843813in}}%
\pgfpathlineto{\pgfqpoint{1.474827in}{0.850422in}}%
\pgfpathlineto{\pgfqpoint{1.483699in}{0.857032in}}%
\pgfpathlineto{\pgfqpoint{1.492571in}{0.863641in}}%
\pgfpathlineto{\pgfqpoint{1.501443in}{0.870251in}}%
\pgfpathlineto{\pgfqpoint{1.510315in}{0.876860in}}%
\pgfpathlineto{\pgfqpoint{1.519187in}{0.883469in}}%
\pgfpathlineto{\pgfqpoint{1.528059in}{0.890079in}}%
\pgfpathlineto{\pgfqpoint{1.536931in}{0.896688in}}%
\pgfpathlineto{\pgfqpoint{1.545803in}{0.903298in}}%
\pgfpathlineto{\pgfqpoint{1.554675in}{0.909907in}}%
\pgfpathlineto{\pgfqpoint{1.563547in}{0.916517in}}%
\pgfpathlineto{\pgfqpoint{1.572419in}{0.923126in}}%
\pgfpathlineto{\pgfqpoint{1.581291in}{0.929735in}}%
\pgfpathlineto{\pgfqpoint{1.590163in}{0.936345in}}%
\pgfpathlineto{\pgfqpoint{1.599035in}{0.942954in}}%
\pgfpathlineto{\pgfqpoint{1.607907in}{0.949564in}}%
\pgfpathlineto{\pgfqpoint{1.616779in}{0.956173in}}%
\pgfpathlineto{\pgfqpoint{1.625652in}{0.962783in}}%
\pgfpathlineto{\pgfqpoint{1.634524in}{0.969392in}}%
\pgfpathlineto{\pgfqpoint{1.643396in}{0.976001in}}%
\pgfpathlineto{\pgfqpoint{1.652268in}{0.982611in}}%
\pgfpathlineto{\pgfqpoint{1.661140in}{0.989220in}}%
\pgfpathlineto{\pgfqpoint{1.670012in}{0.995830in}}%
\pgfpathlineto{\pgfqpoint{1.678884in}{1.002439in}}%
\pgfpathlineto{\pgfqpoint{1.687756in}{1.009049in}}%
\pgfpathlineto{\pgfqpoint{1.696628in}{1.015658in}}%
\pgfpathlineto{\pgfqpoint{1.705500in}{1.022267in}}%
\pgfpathlineto{\pgfqpoint{1.714372in}{1.028877in}}%
\pgfpathlineto{\pgfqpoint{1.723244in}{1.035486in}}%
\pgfpathlineto{\pgfqpoint{1.732116in}{1.042096in}}%
\pgfpathlineto{\pgfqpoint{1.740988in}{1.048705in}}%
\pgfpathlineto{\pgfqpoint{1.749860in}{1.055315in}}%
\pgfpathlineto{\pgfqpoint{1.758732in}{1.061924in}}%
\pgfpathlineto{\pgfqpoint{1.767604in}{1.068533in}}%
\pgfpathlineto{\pgfqpoint{1.776476in}{1.075143in}}%
\pgfpathlineto{\pgfqpoint{1.785349in}{1.081752in}}%
\pgfpathlineto{\pgfqpoint{1.794221in}{1.088362in}}%
\pgfpathlineto{\pgfqpoint{1.803093in}{1.094971in}}%
\pgfpathlineto{\pgfqpoint{1.811965in}{1.101581in}}%
\pgfpathlineto{\pgfqpoint{1.820837in}{1.108190in}}%
\pgfpathlineto{\pgfqpoint{1.829709in}{1.114799in}}%
\pgfpathlineto{\pgfqpoint{1.838581in}{1.121409in}}%
\pgfpathlineto{\pgfqpoint{1.847453in}{1.128018in}}%
\pgfpathlineto{\pgfqpoint{1.856325in}{1.134628in}}%
\pgfpathlineto{\pgfqpoint{1.865197in}{1.141237in}}%
\pgfpathlineto{\pgfqpoint{1.874069in}{1.147847in}}%
\pgfpathlineto{\pgfqpoint{1.882941in}{1.154456in}}%
\pgfpathlineto{\pgfqpoint{1.891813in}{1.161065in}}%
\pgfpathlineto{\pgfqpoint{1.900685in}{1.167675in}}%
\pgfpathlineto{\pgfqpoint{1.909557in}{1.174284in}}%
\pgfpathlineto{\pgfqpoint{1.918429in}{1.180894in}}%
\pgfpathlineto{\pgfqpoint{1.927301in}{1.187503in}}%
\pgfpathlineto{\pgfqpoint{1.936173in}{1.194113in}}%
\pgfpathlineto{\pgfqpoint{1.945045in}{1.200722in}}%
\pgfpathlineto{\pgfqpoint{1.953918in}{1.207331in}}%
\pgfpathlineto{\pgfqpoint{1.962790in}{1.213941in}}%
\pgfpathlineto{\pgfqpoint{1.971662in}{1.220550in}}%
\pgfpathlineto{\pgfqpoint{1.980534in}{1.227160in}}%
\pgfpathlineto{\pgfqpoint{1.989406in}{1.233769in}}%
\pgfpathlineto{\pgfqpoint{1.998278in}{1.240379in}}%
\pgfpathlineto{\pgfqpoint{2.007150in}{1.246988in}}%
\pgfpathlineto{\pgfqpoint{2.016022in}{1.253597in}}%
\pgfpathlineto{\pgfqpoint{2.024894in}{1.260207in}}%
\pgfpathlineto{\pgfqpoint{2.033766in}{1.266816in}}%
\pgfpathlineto{\pgfqpoint{2.042638in}{1.273426in}}%
\pgfpathlineto{\pgfqpoint{2.051510in}{1.280035in}}%
\pgfpathlineto{\pgfqpoint{2.060382in}{1.286645in}}%
\pgfpathlineto{\pgfqpoint{2.069254in}{1.293254in}}%
\pgfpathlineto{\pgfqpoint{2.078126in}{1.299863in}}%
\pgfpathlineto{\pgfqpoint{2.086998in}{1.306473in}}%
\pgfpathlineto{\pgfqpoint{2.095870in}{1.313082in}}%
\pgfpathlineto{\pgfqpoint{2.104742in}{1.319692in}}%
\pgfpathlineto{\pgfqpoint{2.113615in}{1.326301in}}%
\pgfpathlineto{\pgfqpoint{2.122487in}{1.332911in}}%
\pgfpathlineto{\pgfqpoint{2.131359in}{1.339520in}}%
\pgfusepath{stroke}%
\end{pgfscope}%
\begin{pgfscope}%
\pgfpathrectangle{\pgfqpoint{0.374692in}{0.521603in}}{\pgfqpoint{2.635000in}{1.963000in}} %
\pgfusepath{clip}%
\pgfsetbuttcap%
\pgfsetroundjoin%
\definecolor{currentfill}{rgb}{1.000000,0.000000,0.000000}%
\pgfsetfillcolor{currentfill}%
\pgfsetlinewidth{1.003750pt}%
\definecolor{currentstroke}{rgb}{1.000000,0.000000,0.000000}%
\pgfsetstrokecolor{currentstroke}%
\pgfsetdash{}{0pt}%
\pgfsys@defobject{currentmarker}{\pgfqpoint{-0.020833in}{-0.020833in}}{\pgfqpoint{0.020833in}{0.020833in}}{%
\pgfpathmoveto{\pgfqpoint{0.000000in}{-0.020833in}}%
\pgfpathcurveto{\pgfqpoint{0.005525in}{-0.020833in}}{\pgfqpoint{0.010825in}{-0.018638in}}{\pgfqpoint{0.014731in}{-0.014731in}}%
\pgfpathcurveto{\pgfqpoint{0.018638in}{-0.010825in}}{\pgfqpoint{0.020833in}{-0.005525in}}{\pgfqpoint{0.020833in}{0.000000in}}%
\pgfpathcurveto{\pgfqpoint{0.020833in}{0.005525in}}{\pgfqpoint{0.018638in}{0.010825in}}{\pgfqpoint{0.014731in}{0.014731in}}%
\pgfpathcurveto{\pgfqpoint{0.010825in}{0.018638in}}{\pgfqpoint{0.005525in}{0.020833in}}{\pgfqpoint{0.000000in}{0.020833in}}%
\pgfpathcurveto{\pgfqpoint{-0.005525in}{0.020833in}}{\pgfqpoint{-0.010825in}{0.018638in}}{\pgfqpoint{-0.014731in}{0.014731in}}%
\pgfpathcurveto{\pgfqpoint{-0.018638in}{0.010825in}}{\pgfqpoint{-0.020833in}{0.005525in}}{\pgfqpoint{-0.020833in}{0.000000in}}%
\pgfpathcurveto{\pgfqpoint{-0.020833in}{-0.005525in}}{\pgfqpoint{-0.018638in}{-0.010825in}}{\pgfqpoint{-0.014731in}{-0.014731in}}%
\pgfpathcurveto{\pgfqpoint{-0.010825in}{-0.018638in}}{\pgfqpoint{-0.005525in}{-0.020833in}}{\pgfqpoint{0.000000in}{-0.020833in}}%
\pgfpathclose%
\pgfusepath{stroke,fill}%
}%
\begin{pgfscope}%
\pgfsys@transformshift{1.253025in}{0.848770in}%
\pgfsys@useobject{currentmarker}{}%
\end{pgfscope}%
\begin{pgfscope}%
\pgfsys@transformshift{1.692192in}{0.848770in}%
\pgfsys@useobject{currentmarker}{}%
\end{pgfscope}%
\begin{pgfscope}%
\pgfsys@transformshift{2.131359in}{0.848770in}%
\pgfsys@useobject{currentmarker}{}%
\end{pgfscope}%
\end{pgfscope}%
\begin{pgfscope}%
\pgfpathrectangle{\pgfqpoint{0.374692in}{0.521603in}}{\pgfqpoint{2.635000in}{1.963000in}} %
\pgfusepath{clip}%
\pgfsetbuttcap%
\pgfsetroundjoin%
\definecolor{currentfill}{rgb}{1.000000,0.000000,0.000000}%
\pgfsetfillcolor{currentfill}%
\pgfsetlinewidth{1.003750pt}%
\definecolor{currentstroke}{rgb}{1.000000,0.000000,0.000000}%
\pgfsetstrokecolor{currentstroke}%
\pgfsetdash{}{0pt}%
\pgfsys@defobject{currentmarker}{\pgfqpoint{-0.020833in}{-0.020833in}}{\pgfqpoint{0.020833in}{0.020833in}}{%
\pgfpathmoveto{\pgfqpoint{0.000000in}{-0.020833in}}%
\pgfpathcurveto{\pgfqpoint{0.005525in}{-0.020833in}}{\pgfqpoint{0.010825in}{-0.018638in}}{\pgfqpoint{0.014731in}{-0.014731in}}%
\pgfpathcurveto{\pgfqpoint{0.018638in}{-0.010825in}}{\pgfqpoint{0.020833in}{-0.005525in}}{\pgfqpoint{0.020833in}{0.000000in}}%
\pgfpathcurveto{\pgfqpoint{0.020833in}{0.005525in}}{\pgfqpoint{0.018638in}{0.010825in}}{\pgfqpoint{0.014731in}{0.014731in}}%
\pgfpathcurveto{\pgfqpoint{0.010825in}{0.018638in}}{\pgfqpoint{0.005525in}{0.020833in}}{\pgfqpoint{0.000000in}{0.020833in}}%
\pgfpathcurveto{\pgfqpoint{-0.005525in}{0.020833in}}{\pgfqpoint{-0.010825in}{0.018638in}}{\pgfqpoint{-0.014731in}{0.014731in}}%
\pgfpathcurveto{\pgfqpoint{-0.018638in}{0.010825in}}{\pgfqpoint{-0.020833in}{0.005525in}}{\pgfqpoint{-0.020833in}{0.000000in}}%
\pgfpathcurveto{\pgfqpoint{-0.020833in}{-0.005525in}}{\pgfqpoint{-0.018638in}{-0.010825in}}{\pgfqpoint{-0.014731in}{-0.014731in}}%
\pgfpathcurveto{\pgfqpoint{-0.010825in}{-0.018638in}}{\pgfqpoint{-0.005525in}{-0.020833in}}{\pgfqpoint{0.000000in}{-0.020833in}}%
\pgfpathclose%
\pgfusepath{stroke,fill}%
}%
\begin{pgfscope}%
\pgfsys@transformshift{1.253025in}{0.848770in}%
\pgfsys@useobject{currentmarker}{}%
\end{pgfscope}%
\begin{pgfscope}%
\pgfsys@transformshift{1.692192in}{0.848770in}%
\pgfsys@useobject{currentmarker}{}%
\end{pgfscope}%
\begin{pgfscope}%
\pgfsys@transformshift{2.131359in}{0.848770in}%
\pgfsys@useobject{currentmarker}{}%
\end{pgfscope}%
\end{pgfscope}%
\begin{pgfscope}%
\pgfpathrectangle{\pgfqpoint{0.374692in}{0.521603in}}{\pgfqpoint{2.635000in}{1.963000in}} %
\pgfusepath{clip}%
\pgfsetrectcap%
\pgfsetroundjoin%
\pgfsetlinewidth{1.003750pt}%
\definecolor{currentstroke}{rgb}{0.580392,0.403922,0.741176}%
\pgfsetstrokecolor{currentstroke}%
\pgfsetdash{}{0pt}%
\pgfpathmoveto{\pgfqpoint{2.131359in}{1.339520in}}%
\pgfpathlineto{\pgfqpoint{2.140231in}{1.332911in}}%
\pgfpathlineto{\pgfqpoint{2.149103in}{1.326301in}}%
\pgfpathlineto{\pgfqpoint{2.157975in}{1.319692in}}%
\pgfpathlineto{\pgfqpoint{2.166847in}{1.313082in}}%
\pgfpathlineto{\pgfqpoint{2.175719in}{1.306473in}}%
\pgfpathlineto{\pgfqpoint{2.184591in}{1.299863in}}%
\pgfpathlineto{\pgfqpoint{2.193463in}{1.293254in}}%
\pgfpathlineto{\pgfqpoint{2.202335in}{1.286645in}}%
\pgfpathlineto{\pgfqpoint{2.211207in}{1.280035in}}%
\pgfpathlineto{\pgfqpoint{2.220079in}{1.273426in}}%
\pgfpathlineto{\pgfqpoint{2.228951in}{1.266816in}}%
\pgfpathlineto{\pgfqpoint{2.237823in}{1.260207in}}%
\pgfpathlineto{\pgfqpoint{2.246695in}{1.253597in}}%
\pgfpathlineto{\pgfqpoint{2.255567in}{1.246988in}}%
\pgfpathlineto{\pgfqpoint{2.264439in}{1.240379in}}%
\pgfpathlineto{\pgfqpoint{2.273311in}{1.233769in}}%
\pgfpathlineto{\pgfqpoint{2.282184in}{1.227160in}}%
\pgfpathlineto{\pgfqpoint{2.291056in}{1.220550in}}%
\pgfpathlineto{\pgfqpoint{2.299928in}{1.213941in}}%
\pgfpathlineto{\pgfqpoint{2.308800in}{1.207331in}}%
\pgfpathlineto{\pgfqpoint{2.317672in}{1.200722in}}%
\pgfpathlineto{\pgfqpoint{2.326544in}{1.194113in}}%
\pgfpathlineto{\pgfqpoint{2.335416in}{1.187503in}}%
\pgfpathlineto{\pgfqpoint{2.344288in}{1.180894in}}%
\pgfpathlineto{\pgfqpoint{2.353160in}{1.174284in}}%
\pgfpathlineto{\pgfqpoint{2.362032in}{1.167675in}}%
\pgfpathlineto{\pgfqpoint{2.370904in}{1.161065in}}%
\pgfpathlineto{\pgfqpoint{2.379776in}{1.154456in}}%
\pgfpathlineto{\pgfqpoint{2.388648in}{1.147847in}}%
\pgfpathlineto{\pgfqpoint{2.397520in}{1.141237in}}%
\pgfpathlineto{\pgfqpoint{2.406392in}{1.134628in}}%
\pgfpathlineto{\pgfqpoint{2.415264in}{1.128018in}}%
\pgfpathlineto{\pgfqpoint{2.424136in}{1.121409in}}%
\pgfpathlineto{\pgfqpoint{2.433008in}{1.114799in}}%
\pgfpathlineto{\pgfqpoint{2.441880in}{1.108190in}}%
\pgfpathlineto{\pgfqpoint{2.450753in}{1.101581in}}%
\pgfpathlineto{\pgfqpoint{2.459625in}{1.094971in}}%
\pgfpathlineto{\pgfqpoint{2.468497in}{1.088362in}}%
\pgfpathlineto{\pgfqpoint{2.477369in}{1.081752in}}%
\pgfpathlineto{\pgfqpoint{2.486241in}{1.075143in}}%
\pgfpathlineto{\pgfqpoint{2.495113in}{1.068533in}}%
\pgfpathlineto{\pgfqpoint{2.503985in}{1.061924in}}%
\pgfpathlineto{\pgfqpoint{2.512857in}{1.055315in}}%
\pgfpathlineto{\pgfqpoint{2.521729in}{1.048705in}}%
\pgfpathlineto{\pgfqpoint{2.530601in}{1.042096in}}%
\pgfpathlineto{\pgfqpoint{2.539473in}{1.035486in}}%
\pgfpathlineto{\pgfqpoint{2.548345in}{1.028877in}}%
\pgfpathlineto{\pgfqpoint{2.557217in}{1.022267in}}%
\pgfpathlineto{\pgfqpoint{2.566089in}{1.015658in}}%
\pgfpathlineto{\pgfqpoint{2.574961in}{1.009049in}}%
\pgfpathlineto{\pgfqpoint{2.583833in}{1.002439in}}%
\pgfpathlineto{\pgfqpoint{2.592705in}{0.995830in}}%
\pgfpathlineto{\pgfqpoint{2.601577in}{0.989220in}}%
\pgfpathlineto{\pgfqpoint{2.610450in}{0.982611in}}%
\pgfpathlineto{\pgfqpoint{2.619322in}{0.976001in}}%
\pgfpathlineto{\pgfqpoint{2.628194in}{0.969392in}}%
\pgfpathlineto{\pgfqpoint{2.637066in}{0.962783in}}%
\pgfpathlineto{\pgfqpoint{2.645938in}{0.956173in}}%
\pgfpathlineto{\pgfqpoint{2.654810in}{0.949564in}}%
\pgfpathlineto{\pgfqpoint{2.663682in}{0.942954in}}%
\pgfpathlineto{\pgfqpoint{2.672554in}{0.936345in}}%
\pgfpathlineto{\pgfqpoint{2.681426in}{0.929735in}}%
\pgfpathlineto{\pgfqpoint{2.690298in}{0.923126in}}%
\pgfpathlineto{\pgfqpoint{2.699170in}{0.916517in}}%
\pgfpathlineto{\pgfqpoint{2.708042in}{0.909907in}}%
\pgfpathlineto{\pgfqpoint{2.716914in}{0.903298in}}%
\pgfpathlineto{\pgfqpoint{2.725786in}{0.896688in}}%
\pgfpathlineto{\pgfqpoint{2.734658in}{0.890079in}}%
\pgfpathlineto{\pgfqpoint{2.743530in}{0.883469in}}%
\pgfpathlineto{\pgfqpoint{2.752402in}{0.876860in}}%
\pgfpathlineto{\pgfqpoint{2.761274in}{0.870251in}}%
\pgfpathlineto{\pgfqpoint{2.770146in}{0.863641in}}%
\pgfpathlineto{\pgfqpoint{2.779019in}{0.857032in}}%
\pgfpathlineto{\pgfqpoint{2.787891in}{0.850422in}}%
\pgfpathlineto{\pgfqpoint{2.796763in}{0.843813in}}%
\pgfpathlineto{\pgfqpoint{2.805635in}{0.837204in}}%
\pgfpathlineto{\pgfqpoint{2.814507in}{0.830594in}}%
\pgfpathlineto{\pgfqpoint{2.823379in}{0.823985in}}%
\pgfpathlineto{\pgfqpoint{2.832251in}{0.817375in}}%
\pgfpathlineto{\pgfqpoint{2.841123in}{0.810766in}}%
\pgfpathlineto{\pgfqpoint{2.849995in}{0.804156in}}%
\pgfpathlineto{\pgfqpoint{2.858867in}{0.797547in}}%
\pgfpathlineto{\pgfqpoint{2.867739in}{0.790938in}}%
\pgfpathlineto{\pgfqpoint{2.876611in}{0.784328in}}%
\pgfpathlineto{\pgfqpoint{2.885483in}{0.777719in}}%
\pgfpathlineto{\pgfqpoint{2.894355in}{0.771109in}}%
\pgfpathlineto{\pgfqpoint{2.903227in}{0.764500in}}%
\pgfpathlineto{\pgfqpoint{2.912099in}{0.757890in}}%
\pgfpathlineto{\pgfqpoint{2.920971in}{0.751281in}}%
\pgfpathlineto{\pgfqpoint{2.929843in}{0.744672in}}%
\pgfpathlineto{\pgfqpoint{2.938716in}{0.738062in}}%
\pgfpathlineto{\pgfqpoint{2.947588in}{0.731453in}}%
\pgfpathlineto{\pgfqpoint{2.956460in}{0.724843in}}%
\pgfpathlineto{\pgfqpoint{2.965332in}{0.718234in}}%
\pgfpathlineto{\pgfqpoint{2.974204in}{0.711624in}}%
\pgfpathlineto{\pgfqpoint{2.983076in}{0.705015in}}%
\pgfpathlineto{\pgfqpoint{2.991948in}{0.698406in}}%
\pgfpathlineto{\pgfqpoint{3.000820in}{0.691796in}}%
\pgfpathlineto{\pgfqpoint{3.009692in}{0.685187in}}%
\pgfusepath{stroke}%
\end{pgfscope}%
\begin{pgfscope}%
\pgfpathrectangle{\pgfqpoint{0.374692in}{0.521603in}}{\pgfqpoint{2.635000in}{1.963000in}} %
\pgfusepath{clip}%
\pgfsetrectcap%
\pgfsetroundjoin%
\pgfsetlinewidth{1.003750pt}%
\definecolor{currentstroke}{rgb}{0.549020,0.337255,0.294118}%
\pgfsetstrokecolor{currentstroke}%
\pgfsetdash{}{0pt}%
\pgfpathmoveto{\pgfqpoint{2.131359in}{0.685187in}}%
\pgfpathlineto{\pgfqpoint{2.140231in}{0.691796in}}%
\pgfpathlineto{\pgfqpoint{2.149103in}{0.698406in}}%
\pgfpathlineto{\pgfqpoint{2.157975in}{0.705015in}}%
\pgfpathlineto{\pgfqpoint{2.166847in}{0.711624in}}%
\pgfpathlineto{\pgfqpoint{2.175719in}{0.718234in}}%
\pgfpathlineto{\pgfqpoint{2.184591in}{0.724843in}}%
\pgfpathlineto{\pgfqpoint{2.193463in}{0.731453in}}%
\pgfpathlineto{\pgfqpoint{2.202335in}{0.738062in}}%
\pgfpathlineto{\pgfqpoint{2.211207in}{0.744672in}}%
\pgfpathlineto{\pgfqpoint{2.220079in}{0.751281in}}%
\pgfpathlineto{\pgfqpoint{2.228951in}{0.757890in}}%
\pgfpathlineto{\pgfqpoint{2.237823in}{0.764500in}}%
\pgfpathlineto{\pgfqpoint{2.246695in}{0.771109in}}%
\pgfpathlineto{\pgfqpoint{2.255567in}{0.777719in}}%
\pgfpathlineto{\pgfqpoint{2.264439in}{0.784328in}}%
\pgfpathlineto{\pgfqpoint{2.273311in}{0.790938in}}%
\pgfpathlineto{\pgfqpoint{2.282184in}{0.797547in}}%
\pgfpathlineto{\pgfqpoint{2.291056in}{0.804156in}}%
\pgfpathlineto{\pgfqpoint{2.299928in}{0.810766in}}%
\pgfpathlineto{\pgfqpoint{2.308800in}{0.817375in}}%
\pgfpathlineto{\pgfqpoint{2.317672in}{0.823985in}}%
\pgfpathlineto{\pgfqpoint{2.326544in}{0.830594in}}%
\pgfpathlineto{\pgfqpoint{2.335416in}{0.837204in}}%
\pgfpathlineto{\pgfqpoint{2.344288in}{0.843813in}}%
\pgfpathlineto{\pgfqpoint{2.353160in}{0.850422in}}%
\pgfpathlineto{\pgfqpoint{2.362032in}{0.857032in}}%
\pgfpathlineto{\pgfqpoint{2.370904in}{0.863641in}}%
\pgfpathlineto{\pgfqpoint{2.379776in}{0.870251in}}%
\pgfpathlineto{\pgfqpoint{2.388648in}{0.876860in}}%
\pgfpathlineto{\pgfqpoint{2.397520in}{0.883469in}}%
\pgfpathlineto{\pgfqpoint{2.406392in}{0.890079in}}%
\pgfpathlineto{\pgfqpoint{2.415264in}{0.896688in}}%
\pgfpathlineto{\pgfqpoint{2.424136in}{0.903298in}}%
\pgfpathlineto{\pgfqpoint{2.433008in}{0.909907in}}%
\pgfpathlineto{\pgfqpoint{2.441880in}{0.916517in}}%
\pgfpathlineto{\pgfqpoint{2.450753in}{0.923126in}}%
\pgfpathlineto{\pgfqpoint{2.459625in}{0.929735in}}%
\pgfpathlineto{\pgfqpoint{2.468497in}{0.936345in}}%
\pgfpathlineto{\pgfqpoint{2.477369in}{0.942954in}}%
\pgfpathlineto{\pgfqpoint{2.486241in}{0.949564in}}%
\pgfpathlineto{\pgfqpoint{2.495113in}{0.956173in}}%
\pgfpathlineto{\pgfqpoint{2.503985in}{0.962783in}}%
\pgfpathlineto{\pgfqpoint{2.512857in}{0.969392in}}%
\pgfpathlineto{\pgfqpoint{2.521729in}{0.976001in}}%
\pgfpathlineto{\pgfqpoint{2.530601in}{0.982611in}}%
\pgfpathlineto{\pgfqpoint{2.539473in}{0.989220in}}%
\pgfpathlineto{\pgfqpoint{2.548345in}{0.995830in}}%
\pgfpathlineto{\pgfqpoint{2.557217in}{1.002439in}}%
\pgfpathlineto{\pgfqpoint{2.566089in}{1.009049in}}%
\pgfpathlineto{\pgfqpoint{2.574961in}{1.015658in}}%
\pgfpathlineto{\pgfqpoint{2.583833in}{1.022267in}}%
\pgfpathlineto{\pgfqpoint{2.592705in}{1.028877in}}%
\pgfpathlineto{\pgfqpoint{2.601577in}{1.035486in}}%
\pgfpathlineto{\pgfqpoint{2.610450in}{1.042096in}}%
\pgfpathlineto{\pgfqpoint{2.619322in}{1.048705in}}%
\pgfpathlineto{\pgfqpoint{2.628194in}{1.055315in}}%
\pgfpathlineto{\pgfqpoint{2.637066in}{1.061924in}}%
\pgfpathlineto{\pgfqpoint{2.645938in}{1.068533in}}%
\pgfpathlineto{\pgfqpoint{2.654810in}{1.075143in}}%
\pgfpathlineto{\pgfqpoint{2.663682in}{1.081752in}}%
\pgfpathlineto{\pgfqpoint{2.672554in}{1.088362in}}%
\pgfpathlineto{\pgfqpoint{2.681426in}{1.094971in}}%
\pgfpathlineto{\pgfqpoint{2.690298in}{1.101581in}}%
\pgfpathlineto{\pgfqpoint{2.699170in}{1.108190in}}%
\pgfpathlineto{\pgfqpoint{2.708042in}{1.114799in}}%
\pgfpathlineto{\pgfqpoint{2.716914in}{1.121409in}}%
\pgfpathlineto{\pgfqpoint{2.725786in}{1.128018in}}%
\pgfpathlineto{\pgfqpoint{2.734658in}{1.134628in}}%
\pgfpathlineto{\pgfqpoint{2.743530in}{1.141237in}}%
\pgfpathlineto{\pgfqpoint{2.752402in}{1.147847in}}%
\pgfpathlineto{\pgfqpoint{2.761274in}{1.154456in}}%
\pgfpathlineto{\pgfqpoint{2.770146in}{1.161065in}}%
\pgfpathlineto{\pgfqpoint{2.779019in}{1.167675in}}%
\pgfpathlineto{\pgfqpoint{2.787891in}{1.174284in}}%
\pgfpathlineto{\pgfqpoint{2.796763in}{1.180894in}}%
\pgfpathlineto{\pgfqpoint{2.805635in}{1.187503in}}%
\pgfpathlineto{\pgfqpoint{2.814507in}{1.194113in}}%
\pgfpathlineto{\pgfqpoint{2.823379in}{1.200722in}}%
\pgfpathlineto{\pgfqpoint{2.832251in}{1.207331in}}%
\pgfpathlineto{\pgfqpoint{2.841123in}{1.213941in}}%
\pgfpathlineto{\pgfqpoint{2.849995in}{1.220550in}}%
\pgfpathlineto{\pgfqpoint{2.858867in}{1.227160in}}%
\pgfpathlineto{\pgfqpoint{2.867739in}{1.233769in}}%
\pgfpathlineto{\pgfqpoint{2.876611in}{1.240379in}}%
\pgfpathlineto{\pgfqpoint{2.885483in}{1.246988in}}%
\pgfpathlineto{\pgfqpoint{2.894355in}{1.253597in}}%
\pgfpathlineto{\pgfqpoint{2.903227in}{1.260207in}}%
\pgfpathlineto{\pgfqpoint{2.912099in}{1.266816in}}%
\pgfpathlineto{\pgfqpoint{2.920971in}{1.273426in}}%
\pgfpathlineto{\pgfqpoint{2.929843in}{1.280035in}}%
\pgfpathlineto{\pgfqpoint{2.938716in}{1.286645in}}%
\pgfpathlineto{\pgfqpoint{2.947588in}{1.293254in}}%
\pgfpathlineto{\pgfqpoint{2.956460in}{1.299863in}}%
\pgfpathlineto{\pgfqpoint{2.965332in}{1.306473in}}%
\pgfpathlineto{\pgfqpoint{2.974204in}{1.313082in}}%
\pgfpathlineto{\pgfqpoint{2.983076in}{1.319692in}}%
\pgfpathlineto{\pgfqpoint{2.991948in}{1.326301in}}%
\pgfpathlineto{\pgfqpoint{3.000820in}{1.332911in}}%
\pgfpathlineto{\pgfqpoint{3.009692in}{1.339520in}}%
\pgfusepath{stroke}%
\end{pgfscope}%
\begin{pgfscope}%
\pgfpathrectangle{\pgfqpoint{0.374692in}{0.521603in}}{\pgfqpoint{2.635000in}{1.963000in}} %
\pgfusepath{clip}%
\pgfsetbuttcap%
\pgfsetroundjoin%
\definecolor{currentfill}{rgb}{0.000000,0.000000,0.000000}%
\pgfsetfillcolor{currentfill}%
\pgfsetlinewidth{1.003750pt}%
\definecolor{currentstroke}{rgb}{0.000000,0.000000,0.000000}%
\pgfsetstrokecolor{currentstroke}%
\pgfsetdash{}{0pt}%
\pgfsys@defobject{currentmarker}{\pgfqpoint{-0.020833in}{-0.020833in}}{\pgfqpoint{0.020833in}{0.020833in}}{%
\pgfpathmoveto{\pgfqpoint{0.000000in}{-0.020833in}}%
\pgfpathcurveto{\pgfqpoint{0.005525in}{-0.020833in}}{\pgfqpoint{0.010825in}{-0.018638in}}{\pgfqpoint{0.014731in}{-0.014731in}}%
\pgfpathcurveto{\pgfqpoint{0.018638in}{-0.010825in}}{\pgfqpoint{0.020833in}{-0.005525in}}{\pgfqpoint{0.020833in}{0.000000in}}%
\pgfpathcurveto{\pgfqpoint{0.020833in}{0.005525in}}{\pgfqpoint{0.018638in}{0.010825in}}{\pgfqpoint{0.014731in}{0.014731in}}%
\pgfpathcurveto{\pgfqpoint{0.010825in}{0.018638in}}{\pgfqpoint{0.005525in}{0.020833in}}{\pgfqpoint{0.000000in}{0.020833in}}%
\pgfpathcurveto{\pgfqpoint{-0.005525in}{0.020833in}}{\pgfqpoint{-0.010825in}{0.018638in}}{\pgfqpoint{-0.014731in}{0.014731in}}%
\pgfpathcurveto{\pgfqpoint{-0.018638in}{0.010825in}}{\pgfqpoint{-0.020833in}{0.005525in}}{\pgfqpoint{-0.020833in}{0.000000in}}%
\pgfpathcurveto{\pgfqpoint{-0.020833in}{-0.005525in}}{\pgfqpoint{-0.018638in}{-0.010825in}}{\pgfqpoint{-0.014731in}{-0.014731in}}%
\pgfpathcurveto{\pgfqpoint{-0.010825in}{-0.018638in}}{\pgfqpoint{-0.005525in}{-0.020833in}}{\pgfqpoint{0.000000in}{-0.020833in}}%
\pgfpathclose%
\pgfusepath{stroke,fill}%
}%
\begin{pgfscope}%
\pgfsys@transformshift{2.131359in}{0.848770in}%
\pgfsys@useobject{currentmarker}{}%
\end{pgfscope}%
\begin{pgfscope}%
\pgfsys@transformshift{2.570525in}{0.848770in}%
\pgfsys@useobject{currentmarker}{}%
\end{pgfscope}%
\end{pgfscope}%
\begin{pgfscope}%
\pgfpathrectangle{\pgfqpoint{0.374692in}{0.521603in}}{\pgfqpoint{2.635000in}{1.963000in}} %
\pgfusepath{clip}%
\pgfsetbuttcap%
\pgfsetroundjoin%
\definecolor{currentfill}{rgb}{1.000000,0.000000,0.000000}%
\pgfsetfillcolor{currentfill}%
\pgfsetlinewidth{1.003750pt}%
\definecolor{currentstroke}{rgb}{1.000000,0.000000,0.000000}%
\pgfsetstrokecolor{currentstroke}%
\pgfsetdash{}{0pt}%
\pgfsys@defobject{currentmarker}{\pgfqpoint{-0.020833in}{-0.020833in}}{\pgfqpoint{0.020833in}{0.020833in}}{%
\pgfpathmoveto{\pgfqpoint{0.000000in}{-0.020833in}}%
\pgfpathcurveto{\pgfqpoint{0.005525in}{-0.020833in}}{\pgfqpoint{0.010825in}{-0.018638in}}{\pgfqpoint{0.014731in}{-0.014731in}}%
\pgfpathcurveto{\pgfqpoint{0.018638in}{-0.010825in}}{\pgfqpoint{0.020833in}{-0.005525in}}{\pgfqpoint{0.020833in}{0.000000in}}%
\pgfpathcurveto{\pgfqpoint{0.020833in}{0.005525in}}{\pgfqpoint{0.018638in}{0.010825in}}{\pgfqpoint{0.014731in}{0.014731in}}%
\pgfpathcurveto{\pgfqpoint{0.010825in}{0.018638in}}{\pgfqpoint{0.005525in}{0.020833in}}{\pgfqpoint{0.000000in}{0.020833in}}%
\pgfpathcurveto{\pgfqpoint{-0.005525in}{0.020833in}}{\pgfqpoint{-0.010825in}{0.018638in}}{\pgfqpoint{-0.014731in}{0.014731in}}%
\pgfpathcurveto{\pgfqpoint{-0.018638in}{0.010825in}}{\pgfqpoint{-0.020833in}{0.005525in}}{\pgfqpoint{-0.020833in}{0.000000in}}%
\pgfpathcurveto{\pgfqpoint{-0.020833in}{-0.005525in}}{\pgfqpoint{-0.018638in}{-0.010825in}}{\pgfqpoint{-0.014731in}{-0.014731in}}%
\pgfpathcurveto{\pgfqpoint{-0.010825in}{-0.018638in}}{\pgfqpoint{-0.005525in}{-0.020833in}}{\pgfqpoint{0.000000in}{-0.020833in}}%
\pgfpathclose%
\pgfusepath{stroke,fill}%
}%
\begin{pgfscope}%
\pgfsys@transformshift{2.131359in}{0.848770in}%
\pgfsys@useobject{currentmarker}{}%
\end{pgfscope}%
\begin{pgfscope}%
\pgfsys@transformshift{2.570525in}{0.848770in}%
\pgfsys@useobject{currentmarker}{}%
\end{pgfscope}%
\end{pgfscope}%
\begin{pgfscope}%
\pgfsetrectcap%
\pgfsetmiterjoin%
\pgfsetlinewidth{0.803000pt}%
\definecolor{currentstroke}{rgb}{0.000000,0.000000,0.000000}%
\pgfsetstrokecolor{currentstroke}%
\pgfsetdash{}{0pt}%
\pgfpathmoveto{\pgfqpoint{0.374692in}{0.521603in}}%
\pgfpathlineto{\pgfqpoint{0.374692in}{2.484603in}}%
\pgfusepath{stroke}%
\end{pgfscope}%
\begin{pgfscope}%
\pgfsetrectcap%
\pgfsetmiterjoin%
\pgfsetlinewidth{0.803000pt}%
\definecolor{currentstroke}{rgb}{0.000000,0.000000,0.000000}%
\pgfsetstrokecolor{currentstroke}%
\pgfsetdash{}{0pt}%
\pgfpathmoveto{\pgfqpoint{3.009692in}{0.521603in}}%
\pgfpathlineto{\pgfqpoint{3.009692in}{2.484603in}}%
\pgfusepath{stroke}%
\end{pgfscope}%
\begin{pgfscope}%
\pgfsetrectcap%
\pgfsetmiterjoin%
\pgfsetlinewidth{0.803000pt}%
\definecolor{currentstroke}{rgb}{0.000000,0.000000,0.000000}%
\pgfsetstrokecolor{currentstroke}%
\pgfsetdash{}{0pt}%
\pgfpathmoveto{\pgfqpoint{0.374692in}{0.521603in}}%
\pgfpathlineto{\pgfqpoint{3.009692in}{0.521603in}}%
\pgfusepath{stroke}%
\end{pgfscope}%
\begin{pgfscope}%
\pgfsetrectcap%
\pgfsetmiterjoin%
\pgfsetlinewidth{0.803000pt}%
\definecolor{currentstroke}{rgb}{0.000000,0.000000,0.000000}%
\pgfsetstrokecolor{currentstroke}%
\pgfsetdash{}{0pt}%
\pgfpathmoveto{\pgfqpoint{0.374692in}{2.484603in}}%
\pgfpathlineto{\pgfqpoint{3.009692in}{2.484603in}}%
\pgfusepath{stroke}%
\end{pgfscope}%
\begin{pgfscope}%
\pgfsetbuttcap%
\pgfsetmiterjoin%
\definecolor{currentfill}{rgb}{1.000000,1.000000,1.000000}%
\pgfsetfillcolor{currentfill}%
\pgfsetfillopacity{0.800000}%
\pgfsetlinewidth{1.003750pt}%
\definecolor{currentstroke}{rgb}{0.800000,0.800000,0.800000}%
\pgfsetstrokecolor{currentstroke}%
\pgfsetstrokeopacity{0.800000}%
\pgfsetdash{}{0pt}%
\pgfpathmoveto{\pgfqpoint{1.450976in}{1.759954in}}%
\pgfpathlineto{\pgfqpoint{2.912470in}{1.759954in}}%
\pgfpathquadraticcurveto{\pgfqpoint{2.940247in}{1.759954in}}{\pgfqpoint{2.940247in}{1.787732in}}%
\pgfpathlineto{\pgfqpoint{2.940247in}{2.387381in}}%
\pgfpathquadraticcurveto{\pgfqpoint{2.940247in}{2.415159in}}{\pgfqpoint{2.912470in}{2.415159in}}%
\pgfpathlineto{\pgfqpoint{1.450976in}{2.415159in}}%
\pgfpathquadraticcurveto{\pgfqpoint{1.423198in}{2.415159in}}{\pgfqpoint{1.423198in}{2.387381in}}%
\pgfpathlineto{\pgfqpoint{1.423198in}{1.787732in}}%
\pgfpathquadraticcurveto{\pgfqpoint{1.423198in}{1.759954in}}{\pgfqpoint{1.450976in}{1.759954in}}%
\pgfpathclose%
\pgfusepath{stroke,fill}%
\end{pgfscope}%
\begin{pgfscope}%
\pgfsetbuttcap%
\pgfsetroundjoin%
\pgfsetlinewidth{1.505625pt}%
\definecolor{currentstroke}{rgb}{0.000000,0.000000,0.000000}%
\pgfsetstrokecolor{currentstroke}%
\pgfsetdash{{5.550000pt}{2.400000pt}}{0.000000pt}%
\pgfpathmoveto{\pgfqpoint{1.478754in}{2.302691in}}%
\pgfpathlineto{\pgfqpoint{1.756532in}{2.302691in}}%
\pgfusepath{stroke}%
\end{pgfscope}%
\begin{pgfscope}%
\pgftext[x=1.867643in,y=2.254080in,left,base]{\rmfamily\fontsize{10.000000}{12.000000}\selectfont el. boundaries}%
\end{pgfscope}%
\begin{pgfscope}%
\pgfsetbuttcap%
\pgfsetroundjoin%
\definecolor{currentfill}{rgb}{0.000000,0.000000,0.000000}%
\pgfsetfillcolor{currentfill}%
\pgfsetlinewidth{1.003750pt}%
\definecolor{currentstroke}{rgb}{0.000000,0.000000,0.000000}%
\pgfsetstrokecolor{currentstroke}%
\pgfsetdash{}{0pt}%
\pgfsys@defobject{currentmarker}{\pgfqpoint{-0.020833in}{-0.020833in}}{\pgfqpoint{0.020833in}{0.020833in}}{%
\pgfpathmoveto{\pgfqpoint{0.000000in}{-0.020833in}}%
\pgfpathcurveto{\pgfqpoint{0.005525in}{-0.020833in}}{\pgfqpoint{0.010825in}{-0.018638in}}{\pgfqpoint{0.014731in}{-0.014731in}}%
\pgfpathcurveto{\pgfqpoint{0.018638in}{-0.010825in}}{\pgfqpoint{0.020833in}{-0.005525in}}{\pgfqpoint{0.020833in}{0.000000in}}%
\pgfpathcurveto{\pgfqpoint{0.020833in}{0.005525in}}{\pgfqpoint{0.018638in}{0.010825in}}{\pgfqpoint{0.014731in}{0.014731in}}%
\pgfpathcurveto{\pgfqpoint{0.010825in}{0.018638in}}{\pgfqpoint{0.005525in}{0.020833in}}{\pgfqpoint{0.000000in}{0.020833in}}%
\pgfpathcurveto{\pgfqpoint{-0.005525in}{0.020833in}}{\pgfqpoint{-0.010825in}{0.018638in}}{\pgfqpoint{-0.014731in}{0.014731in}}%
\pgfpathcurveto{\pgfqpoint{-0.018638in}{0.010825in}}{\pgfqpoint{-0.020833in}{0.005525in}}{\pgfqpoint{-0.020833in}{0.000000in}}%
\pgfpathcurveto{\pgfqpoint{-0.020833in}{-0.005525in}}{\pgfqpoint{-0.018638in}{-0.010825in}}{\pgfqpoint{-0.014731in}{-0.014731in}}%
\pgfpathcurveto{\pgfqpoint{-0.010825in}{-0.018638in}}{\pgfqpoint{-0.005525in}{-0.020833in}}{\pgfqpoint{0.000000in}{-0.020833in}}%
\pgfpathclose%
\pgfusepath{stroke,fill}%
}%
\begin{pgfscope}%
\pgfsys@transformshift{1.617643in}{2.098834in}%
\pgfsys@useobject{currentmarker}{}%
\end{pgfscope}%
\end{pgfscope}%
\begin{pgfscope}%
\pgftext[x=1.867643in,y=2.050223in,left,base]{\rmfamily\fontsize{10.000000}{12.000000}\selectfont local knots \(\displaystyle s_m\)}%
\end{pgfscope}%
\begin{pgfscope}%
\pgfsetbuttcap%
\pgfsetroundjoin%
\definecolor{currentfill}{rgb}{1.000000,0.000000,0.000000}%
\pgfsetfillcolor{currentfill}%
\pgfsetlinewidth{1.003750pt}%
\definecolor{currentstroke}{rgb}{1.000000,0.000000,0.000000}%
\pgfsetstrokecolor{currentstroke}%
\pgfsetdash{}{0pt}%
\pgfsys@defobject{currentmarker}{\pgfqpoint{-0.020833in}{-0.020833in}}{\pgfqpoint{0.020833in}{0.020833in}}{%
\pgfpathmoveto{\pgfqpoint{0.000000in}{-0.020833in}}%
\pgfpathcurveto{\pgfqpoint{0.005525in}{-0.020833in}}{\pgfqpoint{0.010825in}{-0.018638in}}{\pgfqpoint{0.014731in}{-0.014731in}}%
\pgfpathcurveto{\pgfqpoint{0.018638in}{-0.010825in}}{\pgfqpoint{0.020833in}{-0.005525in}}{\pgfqpoint{0.020833in}{0.000000in}}%
\pgfpathcurveto{\pgfqpoint{0.020833in}{0.005525in}}{\pgfqpoint{0.018638in}{0.010825in}}{\pgfqpoint{0.014731in}{0.014731in}}%
\pgfpathcurveto{\pgfqpoint{0.010825in}{0.018638in}}{\pgfqpoint{0.005525in}{0.020833in}}{\pgfqpoint{0.000000in}{0.020833in}}%
\pgfpathcurveto{\pgfqpoint{-0.005525in}{0.020833in}}{\pgfqpoint{-0.010825in}{0.018638in}}{\pgfqpoint{-0.014731in}{0.014731in}}%
\pgfpathcurveto{\pgfqpoint{-0.018638in}{0.010825in}}{\pgfqpoint{-0.020833in}{0.005525in}}{\pgfqpoint{-0.020833in}{0.000000in}}%
\pgfpathcurveto{\pgfqpoint{-0.020833in}{-0.005525in}}{\pgfqpoint{-0.018638in}{-0.010825in}}{\pgfqpoint{-0.014731in}{-0.014731in}}%
\pgfpathcurveto{\pgfqpoint{-0.010825in}{-0.018638in}}{\pgfqpoint{-0.005525in}{-0.020833in}}{\pgfqpoint{0.000000in}{-0.020833in}}%
\pgfpathclose%
\pgfusepath{stroke,fill}%
}%
\begin{pgfscope}%
\pgfsys@transformshift{1.617643in}{1.894977in}%
\pgfsys@useobject{currentmarker}{}%
\end{pgfscope}%
\end{pgfscope}%
\begin{pgfscope}%
\pgftext[x=1.867643in,y=1.846366in,left,base]{\rmfamily\fontsize{10.000000}{12.000000}\selectfont global knots \(\displaystyle z_i\)}%
\end{pgfscope}%
\begin{pgfscope}%
\pgfsetbuttcap%
\pgfsetmiterjoin%
\definecolor{currentfill}{rgb}{1.000000,1.000000,1.000000}%
\pgfsetfillcolor{currentfill}%
\pgfsetlinewidth{0.000000pt}%
\definecolor{currentstroke}{rgb}{0.000000,0.000000,0.000000}%
\pgfsetstrokecolor{currentstroke}%
\pgfsetstrokeopacity{0.000000}%
\pgfsetdash{}{0pt}%
\pgfpathmoveto{\pgfqpoint{0.629692in}{1.756603in}}%
\pgfpathlineto{\pgfqpoint{1.309692in}{1.756603in}}%
\pgfpathlineto{\pgfqpoint{1.309692in}{2.224603in}}%
\pgfpathlineto{\pgfqpoint{0.629692in}{2.224603in}}%
\pgfpathclose%
\pgfusepath{fill}%
\end{pgfscope}%
\begin{pgfscope}%
\pgfsetbuttcap%
\pgfsetroundjoin%
\definecolor{currentfill}{rgb}{0.000000,0.000000,0.000000}%
\pgfsetfillcolor{currentfill}%
\pgfsetlinewidth{0.803000pt}%
\definecolor{currentstroke}{rgb}{0.000000,0.000000,0.000000}%
\pgfsetstrokecolor{currentstroke}%
\pgfsetdash{}{0pt}%
\pgfsys@defobject{currentmarker}{\pgfqpoint{0.000000in}{-0.048611in}}{\pgfqpoint{0.000000in}{0.000000in}}{%
\pgfpathmoveto{\pgfqpoint{0.000000in}{0.000000in}}%
\pgfpathlineto{\pgfqpoint{0.000000in}{-0.048611in}}%
\pgfusepath{stroke,fill}%
}%
\begin{pgfscope}%
\pgfsys@transformshift{0.660601in}{1.756603in}%
\pgfsys@useobject{currentmarker}{}%
\end{pgfscope}%
\end{pgfscope}%
\begin{pgfscope}%
\pgftext[x=0.660601in,y=1.659381in,,top]{\rmfamily\fontsize{10.000000}{12.000000}\selectfont \(\displaystyle -1\)}%
\end{pgfscope}%
\begin{pgfscope}%
\pgfsetbuttcap%
\pgfsetroundjoin%
\definecolor{currentfill}{rgb}{0.000000,0.000000,0.000000}%
\pgfsetfillcolor{currentfill}%
\pgfsetlinewidth{0.803000pt}%
\definecolor{currentstroke}{rgb}{0.000000,0.000000,0.000000}%
\pgfsetstrokecolor{currentstroke}%
\pgfsetdash{}{0pt}%
\pgfsys@defobject{currentmarker}{\pgfqpoint{0.000000in}{-0.048611in}}{\pgfqpoint{0.000000in}{0.000000in}}{%
\pgfpathmoveto{\pgfqpoint{0.000000in}{0.000000in}}%
\pgfpathlineto{\pgfqpoint{0.000000in}{-0.048611in}}%
\pgfusepath{stroke,fill}%
}%
\begin{pgfscope}%
\pgfsys@transformshift{0.969692in}{1.756603in}%
\pgfsys@useobject{currentmarker}{}%
\end{pgfscope}%
\end{pgfscope}%
\begin{pgfscope}%
\pgftext[x=0.969692in,y=1.659381in,,top]{\rmfamily\fontsize{10.000000}{12.000000}\selectfont \(\displaystyle 0\)}%
\end{pgfscope}%
\begin{pgfscope}%
\pgfsetbuttcap%
\pgfsetroundjoin%
\definecolor{currentfill}{rgb}{0.000000,0.000000,0.000000}%
\pgfsetfillcolor{currentfill}%
\pgfsetlinewidth{0.803000pt}%
\definecolor{currentstroke}{rgb}{0.000000,0.000000,0.000000}%
\pgfsetstrokecolor{currentstroke}%
\pgfsetdash{}{0pt}%
\pgfsys@defobject{currentmarker}{\pgfqpoint{0.000000in}{-0.048611in}}{\pgfqpoint{0.000000in}{0.000000in}}{%
\pgfpathmoveto{\pgfqpoint{0.000000in}{0.000000in}}%
\pgfpathlineto{\pgfqpoint{0.000000in}{-0.048611in}}%
\pgfusepath{stroke,fill}%
}%
\begin{pgfscope}%
\pgfsys@transformshift{1.278783in}{1.756603in}%
\pgfsys@useobject{currentmarker}{}%
\end{pgfscope}%
\end{pgfscope}%
\begin{pgfscope}%
\pgftext[x=1.278783in,y=1.659381in,,top]{\rmfamily\fontsize{10.000000}{12.000000}\selectfont \(\displaystyle 1\)}%
\end{pgfscope}%
\begin{pgfscope}%
\pgftext[x=0.969692in,y=1.469413in,,top]{\rmfamily\fontsize{10.000000}{12.000000}\selectfont s}%
\end{pgfscope}%
\begin{pgfscope}%
\pgfsetbuttcap%
\pgfsetroundjoin%
\definecolor{currentfill}{rgb}{0.000000,0.000000,0.000000}%
\pgfsetfillcolor{currentfill}%
\pgfsetlinewidth{0.803000pt}%
\definecolor{currentstroke}{rgb}{0.000000,0.000000,0.000000}%
\pgfsetstrokecolor{currentstroke}%
\pgfsetdash{}{0pt}%
\pgfsys@defobject{currentmarker}{\pgfqpoint{-0.048611in}{0.000000in}}{\pgfqpoint{0.000000in}{0.000000in}}{%
\pgfpathmoveto{\pgfqpoint{0.000000in}{0.000000in}}%
\pgfpathlineto{\pgfqpoint{-0.048611in}{0.000000in}}%
\pgfusepath{stroke,fill}%
}%
\begin{pgfscope}%
\pgfsys@transformshift{0.629692in}{1.884240in}%
\pgfsys@useobject{currentmarker}{}%
\end{pgfscope}%
\end{pgfscope}%
\begin{pgfscope}%
\pgftext[x=0.463025in,y=1.831478in,left,base]{\rmfamily\fontsize{10.000000}{12.000000}\selectfont \(\displaystyle 0\)}%
\end{pgfscope}%
\begin{pgfscope}%
\pgfsetbuttcap%
\pgfsetroundjoin%
\definecolor{currentfill}{rgb}{0.000000,0.000000,0.000000}%
\pgfsetfillcolor{currentfill}%
\pgfsetlinewidth{0.803000pt}%
\definecolor{currentstroke}{rgb}{0.000000,0.000000,0.000000}%
\pgfsetstrokecolor{currentstroke}%
\pgfsetdash{}{0pt}%
\pgfsys@defobject{currentmarker}{\pgfqpoint{-0.048611in}{0.000000in}}{\pgfqpoint{0.000000in}{0.000000in}}{%
\pgfpathmoveto{\pgfqpoint{0.000000in}{0.000000in}}%
\pgfpathlineto{\pgfqpoint{-0.048611in}{0.000000in}}%
\pgfusepath{stroke,fill}%
}%
\begin{pgfscope}%
\pgfsys@transformshift{0.629692in}{2.096967in}%
\pgfsys@useobject{currentmarker}{}%
\end{pgfscope}%
\end{pgfscope}%
\begin{pgfscope}%
\pgftext[x=0.463025in,y=2.044205in,left,base]{\rmfamily\fontsize{10.000000}{12.000000}\selectfont \(\displaystyle 1\)}%
\end{pgfscope}%
\begin{pgfscope}%
\pgfpathrectangle{\pgfqpoint{0.629692in}{1.756603in}}{\pgfqpoint{0.680000in}{0.468000in}} %
\pgfusepath{clip}%
\pgfsetrectcap%
\pgfsetroundjoin%
\pgfsetlinewidth{1.003750pt}%
\definecolor{currentstroke}{rgb}{0.121569,0.466667,0.705882}%
\pgfsetstrokecolor{currentstroke}%
\pgfsetdash{}{0pt}%
\pgfpathmoveto{\pgfqpoint{0.660601in}{2.203331in}}%
\pgfpathlineto{\pgfqpoint{0.666845in}{2.199033in}}%
\pgfpathlineto{\pgfqpoint{0.673090in}{2.194736in}}%
\pgfpathlineto{\pgfqpoint{0.679334in}{2.190438in}}%
\pgfpathlineto{\pgfqpoint{0.685578in}{2.186141in}}%
\pgfpathlineto{\pgfqpoint{0.691822in}{2.181843in}}%
\pgfpathlineto{\pgfqpoint{0.698067in}{2.177545in}}%
\pgfpathlineto{\pgfqpoint{0.704311in}{2.173248in}}%
\pgfpathlineto{\pgfqpoint{0.710555in}{2.168950in}}%
\pgfpathlineto{\pgfqpoint{0.716799in}{2.164653in}}%
\pgfpathlineto{\pgfqpoint{0.723044in}{2.160355in}}%
\pgfpathlineto{\pgfqpoint{0.729288in}{2.156058in}}%
\pgfpathlineto{\pgfqpoint{0.735532in}{2.151760in}}%
\pgfpathlineto{\pgfqpoint{0.741776in}{2.147463in}}%
\pgfpathlineto{\pgfqpoint{0.748021in}{2.143165in}}%
\pgfpathlineto{\pgfqpoint{0.754265in}{2.138868in}}%
\pgfpathlineto{\pgfqpoint{0.760509in}{2.134570in}}%
\pgfpathlineto{\pgfqpoint{0.766753in}{2.130273in}}%
\pgfpathlineto{\pgfqpoint{0.772998in}{2.125975in}}%
\pgfpathlineto{\pgfqpoint{0.779242in}{2.121678in}}%
\pgfpathlineto{\pgfqpoint{0.785486in}{2.117380in}}%
\pgfpathlineto{\pgfqpoint{0.791731in}{2.113083in}}%
\pgfpathlineto{\pgfqpoint{0.797975in}{2.108785in}}%
\pgfpathlineto{\pgfqpoint{0.804219in}{2.104488in}}%
\pgfpathlineto{\pgfqpoint{0.810463in}{2.100190in}}%
\pgfpathlineto{\pgfqpoint{0.816708in}{2.095893in}}%
\pgfpathlineto{\pgfqpoint{0.822952in}{2.091595in}}%
\pgfpathlineto{\pgfqpoint{0.829196in}{2.087298in}}%
\pgfpathlineto{\pgfqpoint{0.835440in}{2.083000in}}%
\pgfpathlineto{\pgfqpoint{0.841685in}{2.078703in}}%
\pgfpathlineto{\pgfqpoint{0.847929in}{2.074405in}}%
\pgfpathlineto{\pgfqpoint{0.854173in}{2.070107in}}%
\pgfpathlineto{\pgfqpoint{0.860417in}{2.065810in}}%
\pgfpathlineto{\pgfqpoint{0.866662in}{2.061512in}}%
\pgfpathlineto{\pgfqpoint{0.872906in}{2.057215in}}%
\pgfpathlineto{\pgfqpoint{0.879150in}{2.052917in}}%
\pgfpathlineto{\pgfqpoint{0.885394in}{2.048620in}}%
\pgfpathlineto{\pgfqpoint{0.891639in}{2.044322in}}%
\pgfpathlineto{\pgfqpoint{0.897883in}{2.040025in}}%
\pgfpathlineto{\pgfqpoint{0.904127in}{2.035727in}}%
\pgfpathlineto{\pgfqpoint{0.910371in}{2.031430in}}%
\pgfpathlineto{\pgfqpoint{0.916616in}{2.027132in}}%
\pgfpathlineto{\pgfqpoint{0.922860in}{2.022835in}}%
\pgfpathlineto{\pgfqpoint{0.929104in}{2.018537in}}%
\pgfpathlineto{\pgfqpoint{0.935349in}{2.014240in}}%
\pgfpathlineto{\pgfqpoint{0.941593in}{2.009942in}}%
\pgfpathlineto{\pgfqpoint{0.947837in}{2.005645in}}%
\pgfpathlineto{\pgfqpoint{0.954081in}{2.001347in}}%
\pgfpathlineto{\pgfqpoint{0.960326in}{1.997050in}}%
\pgfpathlineto{\pgfqpoint{0.966570in}{1.992752in}}%
\pgfpathlineto{\pgfqpoint{0.972814in}{1.988455in}}%
\pgfpathlineto{\pgfqpoint{0.979058in}{1.984157in}}%
\pgfpathlineto{\pgfqpoint{0.985303in}{1.979860in}}%
\pgfpathlineto{\pgfqpoint{0.991547in}{1.975562in}}%
\pgfpathlineto{\pgfqpoint{0.997791in}{1.971264in}}%
\pgfpathlineto{\pgfqpoint{1.004035in}{1.966967in}}%
\pgfpathlineto{\pgfqpoint{1.010280in}{1.962669in}}%
\pgfpathlineto{\pgfqpoint{1.016524in}{1.958372in}}%
\pgfpathlineto{\pgfqpoint{1.022768in}{1.954074in}}%
\pgfpathlineto{\pgfqpoint{1.029012in}{1.949777in}}%
\pgfpathlineto{\pgfqpoint{1.035257in}{1.945479in}}%
\pgfpathlineto{\pgfqpoint{1.041501in}{1.941182in}}%
\pgfpathlineto{\pgfqpoint{1.047745in}{1.936884in}}%
\pgfpathlineto{\pgfqpoint{1.053989in}{1.932587in}}%
\pgfpathlineto{\pgfqpoint{1.060234in}{1.928289in}}%
\pgfpathlineto{\pgfqpoint{1.066478in}{1.923992in}}%
\pgfpathlineto{\pgfqpoint{1.072722in}{1.919694in}}%
\pgfpathlineto{\pgfqpoint{1.078967in}{1.915397in}}%
\pgfpathlineto{\pgfqpoint{1.085211in}{1.911099in}}%
\pgfpathlineto{\pgfqpoint{1.091455in}{1.906802in}}%
\pgfpathlineto{\pgfqpoint{1.097699in}{1.902504in}}%
\pgfpathlineto{\pgfqpoint{1.103944in}{1.898207in}}%
\pgfpathlineto{\pgfqpoint{1.110188in}{1.893909in}}%
\pgfpathlineto{\pgfqpoint{1.116432in}{1.889612in}}%
\pgfpathlineto{\pgfqpoint{1.122676in}{1.885314in}}%
\pgfpathlineto{\pgfqpoint{1.128921in}{1.881017in}}%
\pgfpathlineto{\pgfqpoint{1.135165in}{1.876719in}}%
\pgfpathlineto{\pgfqpoint{1.141409in}{1.872422in}}%
\pgfpathlineto{\pgfqpoint{1.147653in}{1.868124in}}%
\pgfpathlineto{\pgfqpoint{1.153898in}{1.863826in}}%
\pgfpathlineto{\pgfqpoint{1.160142in}{1.859529in}}%
\pgfpathlineto{\pgfqpoint{1.166386in}{1.855231in}}%
\pgfpathlineto{\pgfqpoint{1.172630in}{1.850934in}}%
\pgfpathlineto{\pgfqpoint{1.178875in}{1.846636in}}%
\pgfpathlineto{\pgfqpoint{1.185119in}{1.842339in}}%
\pgfpathlineto{\pgfqpoint{1.191363in}{1.838041in}}%
\pgfpathlineto{\pgfqpoint{1.197607in}{1.833744in}}%
\pgfpathlineto{\pgfqpoint{1.203852in}{1.829446in}}%
\pgfpathlineto{\pgfqpoint{1.210096in}{1.825149in}}%
\pgfpathlineto{\pgfqpoint{1.216340in}{1.820851in}}%
\pgfpathlineto{\pgfqpoint{1.222585in}{1.816554in}}%
\pgfpathlineto{\pgfqpoint{1.228829in}{1.812256in}}%
\pgfpathlineto{\pgfqpoint{1.235073in}{1.807959in}}%
\pgfpathlineto{\pgfqpoint{1.241317in}{1.803661in}}%
\pgfpathlineto{\pgfqpoint{1.247562in}{1.799364in}}%
\pgfpathlineto{\pgfqpoint{1.253806in}{1.795066in}}%
\pgfpathlineto{\pgfqpoint{1.260050in}{1.790769in}}%
\pgfpathlineto{\pgfqpoint{1.266294in}{1.786471in}}%
\pgfpathlineto{\pgfqpoint{1.272539in}{1.782174in}}%
\pgfpathlineto{\pgfqpoint{1.278783in}{1.777876in}}%
\pgfusepath{stroke}%
\end{pgfscope}%
\begin{pgfscope}%
\pgfpathrectangle{\pgfqpoint{0.629692in}{1.756603in}}{\pgfqpoint{0.680000in}{0.468000in}} %
\pgfusepath{clip}%
\pgfsetrectcap%
\pgfsetroundjoin%
\pgfsetlinewidth{1.003750pt}%
\definecolor{currentstroke}{rgb}{1.000000,0.498039,0.054902}%
\pgfsetstrokecolor{currentstroke}%
\pgfsetdash{}{0pt}%
\pgfpathmoveto{\pgfqpoint{0.660601in}{1.777876in}}%
\pgfpathlineto{\pgfqpoint{0.666845in}{1.782174in}}%
\pgfpathlineto{\pgfqpoint{0.673090in}{1.786471in}}%
\pgfpathlineto{\pgfqpoint{0.679334in}{1.790769in}}%
\pgfpathlineto{\pgfqpoint{0.685578in}{1.795066in}}%
\pgfpathlineto{\pgfqpoint{0.691822in}{1.799364in}}%
\pgfpathlineto{\pgfqpoint{0.698067in}{1.803661in}}%
\pgfpathlineto{\pgfqpoint{0.704311in}{1.807959in}}%
\pgfpathlineto{\pgfqpoint{0.710555in}{1.812256in}}%
\pgfpathlineto{\pgfqpoint{0.716799in}{1.816554in}}%
\pgfpathlineto{\pgfqpoint{0.723044in}{1.820851in}}%
\pgfpathlineto{\pgfqpoint{0.729288in}{1.825149in}}%
\pgfpathlineto{\pgfqpoint{0.735532in}{1.829446in}}%
\pgfpathlineto{\pgfqpoint{0.741776in}{1.833744in}}%
\pgfpathlineto{\pgfqpoint{0.748021in}{1.838041in}}%
\pgfpathlineto{\pgfqpoint{0.754265in}{1.842339in}}%
\pgfpathlineto{\pgfqpoint{0.760509in}{1.846636in}}%
\pgfpathlineto{\pgfqpoint{0.766753in}{1.850934in}}%
\pgfpathlineto{\pgfqpoint{0.772998in}{1.855231in}}%
\pgfpathlineto{\pgfqpoint{0.779242in}{1.859529in}}%
\pgfpathlineto{\pgfqpoint{0.785486in}{1.863826in}}%
\pgfpathlineto{\pgfqpoint{0.791731in}{1.868124in}}%
\pgfpathlineto{\pgfqpoint{0.797975in}{1.872422in}}%
\pgfpathlineto{\pgfqpoint{0.804219in}{1.876719in}}%
\pgfpathlineto{\pgfqpoint{0.810463in}{1.881017in}}%
\pgfpathlineto{\pgfqpoint{0.816708in}{1.885314in}}%
\pgfpathlineto{\pgfqpoint{0.822952in}{1.889612in}}%
\pgfpathlineto{\pgfqpoint{0.829196in}{1.893909in}}%
\pgfpathlineto{\pgfqpoint{0.835440in}{1.898207in}}%
\pgfpathlineto{\pgfqpoint{0.841685in}{1.902504in}}%
\pgfpathlineto{\pgfqpoint{0.847929in}{1.906802in}}%
\pgfpathlineto{\pgfqpoint{0.854173in}{1.911099in}}%
\pgfpathlineto{\pgfqpoint{0.860417in}{1.915397in}}%
\pgfpathlineto{\pgfqpoint{0.866662in}{1.919694in}}%
\pgfpathlineto{\pgfqpoint{0.872906in}{1.923992in}}%
\pgfpathlineto{\pgfqpoint{0.879150in}{1.928289in}}%
\pgfpathlineto{\pgfqpoint{0.885394in}{1.932587in}}%
\pgfpathlineto{\pgfqpoint{0.891639in}{1.936884in}}%
\pgfpathlineto{\pgfqpoint{0.897883in}{1.941182in}}%
\pgfpathlineto{\pgfqpoint{0.904127in}{1.945479in}}%
\pgfpathlineto{\pgfqpoint{0.910371in}{1.949777in}}%
\pgfpathlineto{\pgfqpoint{0.916616in}{1.954074in}}%
\pgfpathlineto{\pgfqpoint{0.922860in}{1.958372in}}%
\pgfpathlineto{\pgfqpoint{0.929104in}{1.962669in}}%
\pgfpathlineto{\pgfqpoint{0.935349in}{1.966967in}}%
\pgfpathlineto{\pgfqpoint{0.941593in}{1.971264in}}%
\pgfpathlineto{\pgfqpoint{0.947837in}{1.975562in}}%
\pgfpathlineto{\pgfqpoint{0.954081in}{1.979860in}}%
\pgfpathlineto{\pgfqpoint{0.960326in}{1.984157in}}%
\pgfpathlineto{\pgfqpoint{0.966570in}{1.988455in}}%
\pgfpathlineto{\pgfqpoint{0.972814in}{1.992752in}}%
\pgfpathlineto{\pgfqpoint{0.979058in}{1.997050in}}%
\pgfpathlineto{\pgfqpoint{0.985303in}{2.001347in}}%
\pgfpathlineto{\pgfqpoint{0.991547in}{2.005645in}}%
\pgfpathlineto{\pgfqpoint{0.997791in}{2.009942in}}%
\pgfpathlineto{\pgfqpoint{1.004035in}{2.014240in}}%
\pgfpathlineto{\pgfqpoint{1.010280in}{2.018537in}}%
\pgfpathlineto{\pgfqpoint{1.016524in}{2.022835in}}%
\pgfpathlineto{\pgfqpoint{1.022768in}{2.027132in}}%
\pgfpathlineto{\pgfqpoint{1.029012in}{2.031430in}}%
\pgfpathlineto{\pgfqpoint{1.035257in}{2.035727in}}%
\pgfpathlineto{\pgfqpoint{1.041501in}{2.040025in}}%
\pgfpathlineto{\pgfqpoint{1.047745in}{2.044322in}}%
\pgfpathlineto{\pgfqpoint{1.053989in}{2.048620in}}%
\pgfpathlineto{\pgfqpoint{1.060234in}{2.052917in}}%
\pgfpathlineto{\pgfqpoint{1.066478in}{2.057215in}}%
\pgfpathlineto{\pgfqpoint{1.072722in}{2.061512in}}%
\pgfpathlineto{\pgfqpoint{1.078967in}{2.065810in}}%
\pgfpathlineto{\pgfqpoint{1.085211in}{2.070107in}}%
\pgfpathlineto{\pgfqpoint{1.091455in}{2.074405in}}%
\pgfpathlineto{\pgfqpoint{1.097699in}{2.078703in}}%
\pgfpathlineto{\pgfqpoint{1.103944in}{2.083000in}}%
\pgfpathlineto{\pgfqpoint{1.110188in}{2.087298in}}%
\pgfpathlineto{\pgfqpoint{1.116432in}{2.091595in}}%
\pgfpathlineto{\pgfqpoint{1.122676in}{2.095893in}}%
\pgfpathlineto{\pgfqpoint{1.128921in}{2.100190in}}%
\pgfpathlineto{\pgfqpoint{1.135165in}{2.104488in}}%
\pgfpathlineto{\pgfqpoint{1.141409in}{2.108785in}}%
\pgfpathlineto{\pgfqpoint{1.147653in}{2.113083in}}%
\pgfpathlineto{\pgfqpoint{1.153898in}{2.117380in}}%
\pgfpathlineto{\pgfqpoint{1.160142in}{2.121678in}}%
\pgfpathlineto{\pgfqpoint{1.166386in}{2.125975in}}%
\pgfpathlineto{\pgfqpoint{1.172630in}{2.130273in}}%
\pgfpathlineto{\pgfqpoint{1.178875in}{2.134570in}}%
\pgfpathlineto{\pgfqpoint{1.185119in}{2.138868in}}%
\pgfpathlineto{\pgfqpoint{1.191363in}{2.143165in}}%
\pgfpathlineto{\pgfqpoint{1.197607in}{2.147463in}}%
\pgfpathlineto{\pgfqpoint{1.203852in}{2.151760in}}%
\pgfpathlineto{\pgfqpoint{1.210096in}{2.156058in}}%
\pgfpathlineto{\pgfqpoint{1.216340in}{2.160355in}}%
\pgfpathlineto{\pgfqpoint{1.222585in}{2.164653in}}%
\pgfpathlineto{\pgfqpoint{1.228829in}{2.168950in}}%
\pgfpathlineto{\pgfqpoint{1.235073in}{2.173248in}}%
\pgfpathlineto{\pgfqpoint{1.241317in}{2.177545in}}%
\pgfpathlineto{\pgfqpoint{1.247562in}{2.181843in}}%
\pgfpathlineto{\pgfqpoint{1.253806in}{2.186141in}}%
\pgfpathlineto{\pgfqpoint{1.260050in}{2.190438in}}%
\pgfpathlineto{\pgfqpoint{1.266294in}{2.194736in}}%
\pgfpathlineto{\pgfqpoint{1.272539in}{2.199033in}}%
\pgfpathlineto{\pgfqpoint{1.278783in}{2.203331in}}%
\pgfusepath{stroke}%
\end{pgfscope}%
\begin{pgfscope}%
\pgfpathrectangle{\pgfqpoint{0.629692in}{1.756603in}}{\pgfqpoint{0.680000in}{0.468000in}} %
\pgfusepath{clip}%
\pgfsetbuttcap%
\pgfsetroundjoin%
\definecolor{currentfill}{rgb}{0.000000,0.000000,0.000000}%
\pgfsetfillcolor{currentfill}%
\pgfsetlinewidth{1.003750pt}%
\definecolor{currentstroke}{rgb}{0.000000,0.000000,0.000000}%
\pgfsetstrokecolor{currentstroke}%
\pgfsetdash{}{0pt}%
\pgfsys@defobject{currentmarker}{\pgfqpoint{-0.020833in}{-0.020833in}}{\pgfqpoint{0.020833in}{0.020833in}}{%
\pgfpathmoveto{\pgfqpoint{0.000000in}{-0.020833in}}%
\pgfpathcurveto{\pgfqpoint{0.005525in}{-0.020833in}}{\pgfqpoint{0.010825in}{-0.018638in}}{\pgfqpoint{0.014731in}{-0.014731in}}%
\pgfpathcurveto{\pgfqpoint{0.018638in}{-0.010825in}}{\pgfqpoint{0.020833in}{-0.005525in}}{\pgfqpoint{0.020833in}{0.000000in}}%
\pgfpathcurveto{\pgfqpoint{0.020833in}{0.005525in}}{\pgfqpoint{0.018638in}{0.010825in}}{\pgfqpoint{0.014731in}{0.014731in}}%
\pgfpathcurveto{\pgfqpoint{0.010825in}{0.018638in}}{\pgfqpoint{0.005525in}{0.020833in}}{\pgfqpoint{0.000000in}{0.020833in}}%
\pgfpathcurveto{\pgfqpoint{-0.005525in}{0.020833in}}{\pgfqpoint{-0.010825in}{0.018638in}}{\pgfqpoint{-0.014731in}{0.014731in}}%
\pgfpathcurveto{\pgfqpoint{-0.018638in}{0.010825in}}{\pgfqpoint{-0.020833in}{0.005525in}}{\pgfqpoint{-0.020833in}{0.000000in}}%
\pgfpathcurveto{\pgfqpoint{-0.020833in}{-0.005525in}}{\pgfqpoint{-0.018638in}{-0.010825in}}{\pgfqpoint{-0.014731in}{-0.014731in}}%
\pgfpathcurveto{\pgfqpoint{-0.010825in}{-0.018638in}}{\pgfqpoint{-0.005525in}{-0.020833in}}{\pgfqpoint{0.000000in}{-0.020833in}}%
\pgfpathclose%
\pgfusepath{stroke,fill}%
}%
\begin{pgfscope}%
\pgfsys@transformshift{0.660601in}{1.884240in}%
\pgfsys@useobject{currentmarker}{}%
\end{pgfscope}%
\begin{pgfscope}%
\pgfsys@transformshift{0.969692in}{1.884240in}%
\pgfsys@useobject{currentmarker}{}%
\end{pgfscope}%
\begin{pgfscope}%
\pgfsys@transformshift{1.278783in}{1.884240in}%
\pgfsys@useobject{currentmarker}{}%
\end{pgfscope}%
\end{pgfscope}%
\begin{pgfscope}%
\pgfsetrectcap%
\pgfsetmiterjoin%
\pgfsetlinewidth{0.803000pt}%
\definecolor{currentstroke}{rgb}{0.000000,0.000000,0.000000}%
\pgfsetstrokecolor{currentstroke}%
\pgfsetdash{}{0pt}%
\pgfpathmoveto{\pgfqpoint{0.629692in}{1.756603in}}%
\pgfpathlineto{\pgfqpoint{0.629692in}{2.224603in}}%
\pgfusepath{stroke}%
\end{pgfscope}%
\begin{pgfscope}%
\pgfsetrectcap%
\pgfsetmiterjoin%
\pgfsetlinewidth{0.803000pt}%
\definecolor{currentstroke}{rgb}{0.000000,0.000000,0.000000}%
\pgfsetstrokecolor{currentstroke}%
\pgfsetdash{}{0pt}%
\pgfpathmoveto{\pgfqpoint{1.309692in}{1.756603in}}%
\pgfpathlineto{\pgfqpoint{1.309692in}{2.224603in}}%
\pgfusepath{stroke}%
\end{pgfscope}%
\begin{pgfscope}%
\pgfsetrectcap%
\pgfsetmiterjoin%
\pgfsetlinewidth{0.803000pt}%
\definecolor{currentstroke}{rgb}{0.000000,0.000000,0.000000}%
\pgfsetstrokecolor{currentstroke}%
\pgfsetdash{}{0pt}%
\pgfpathmoveto{\pgfqpoint{0.629692in}{1.756603in}}%
\pgfpathlineto{\pgfqpoint{1.309692in}{1.756603in}}%
\pgfusepath{stroke}%
\end{pgfscope}%
\begin{pgfscope}%
\pgfsetrectcap%
\pgfsetmiterjoin%
\pgfsetlinewidth{0.803000pt}%
\definecolor{currentstroke}{rgb}{0.000000,0.000000,0.000000}%
\pgfsetstrokecolor{currentstroke}%
\pgfsetdash{}{0pt}%
\pgfpathmoveto{\pgfqpoint{0.629692in}{2.224603in}}%
\pgfpathlineto{\pgfqpoint{1.309692in}{2.224603in}}%
\pgfusepath{stroke}%
\end{pgfscope}%
\begin{pgfscope}%
\pgftext[x=0.969692in,y=2.307937in,,base]{\rmfamily\fontsize{12.000000}{14.400000}\selectfont Local \(\displaystyle \chi_{n+1/2}\)}%
\end{pgfscope}%
\end{pgfpicture}%
\makeatother%
\endgroup%

\caption{(a) Lagrange shape functions of degree $p=2$ in the reference element $I=[-1,1]$ and the corresponding periodic basis functions on a physical domain of length $L=1$ which has been discretized by $N_\mr{el}=3$ elements of equal length. (b) Corresponding local histopolation shape and basis functions.\label{fig_Lagrange}}
\end{figure}
\hspace{-2.3mm} The construction of the \textit{basis} functions on the physical domain is then done by noting that we need continuity at the shared degrees of freedom at the element boundaries in order for $V_0$ to be a subspace of $H^1$. This leads to a total number of $N_0=pN_\mr{el}$ basis functions in case of periodic boundary conditions and we get the formula $j=\mr{mod}(pk+n,N_0)_{n=0,\ldots,p;k=0\ldots N_\mr{el}-1}$ to go from shape to basis functions. The corresponding projector $\Pi_0$ on this basis acting on some continuous function $E\in H^1$ we define by
\begin{align}
\Pi_0:H^1\rightarrow V_0,\quad(\Pi_0 E)(z_i)=E(z_i),\label{eq_def_projector0}
\end{align}
where $(z_i)_{i=0,\ldots,N_0-1}$ is the global knot sequence on the physical domain which satisfies $\varphi^0_j(z_i)=\delta_{ij}$. Denoting the projected function by $E_h:=\Pi_0E$ we thus have
\begin{align}
E(z_i)=E_h(z_i)=\sum_{j=0}^{N_0-1}e_j\varphi_j^0(z_i)=e_j,
\end{align}
which means that the finite element coefficients are the values of the function at the knot sequence $(z_i)_{i=0,\ldots,N_0-1}$. As a next step, we consider the space $V_1$ and define the shape functions $(\chi_{n+1/2})_{n=0,\ldots,p-1}$ in the reference element $I$ by
\begin{align}
\int_{s_m}^{s_{m+1}}\chi_{n+1/2}(s)\mr{d}s=\delta_{nm},\label{eq_def_LHP}
\end{align}
where $s_0=-1<\ldots<s_m<\ldots<1=s_p$ is the same local knots sequence as for the usual Lagrange shape functions. The polynomials $(\chi_{n+1/2})_{n=0,\ldots,p-1}$ are called \textit{Lagrange histopolation polynomials} (LHPs). Some simple considerations yield that the solution of these equations is given by linear combinations of first order derivatives of the Lagrange shape functions $(\eta_n(s))_{n=0,\ldots,p}$,
\begin{align}
\chi_{n+1/2}(s)=\sum_{m=n+1}^p\frac{\mr{d}}{\mr{d}s}\eta_m(s),\label{eq_def_Lagrange_histo}
\end{align}
which can be verified by plugging this in the definition (\ref{eq_def_LHP}) and using the property $\eta_n(s_m)=\delta_{nm}$. In order to get a basis on the physical domain, these shape functions are just put next to each other since there are no shared degrees of freedom at the element boundaries at which continuity must be enforced. This also has the consequence that the total number of basis function is again $N_1=pN_\mr{el}$, however, in contrast to the previous case, there are now $p$ non-vanishing basis function per element (and not $p+1$) which means that we get the conversion formula $j=(pk+n)_{n=0,\ldots,p-1;k=0,\ldots,N_\mr{el}-1}$ to go from shape to basis functions. We define the corresponding projector $\Pi_1$ acting on some square integrable function $B\in L^2$ by
\begin{align}
\Pi_1:L^2\rightarrow V_1,\quad\int_{z_i}^{z_{i+1}}(\Pi_1 B)(z)\mr{d}z=\int_{z_i}^{z_{i+1}}B(z)\mr{d}z.\label{eq_def_projector1}
\end{align}
Note that $i=0,\ldots,N_0-1$ and thus $z_{N_0}=L$ is just the right end of the domain. Again, denoting the projected function by $B_h:=\Pi_1B$ we have
\begin{align}
\int_{z_i}^{z_{i+1}}B(z)\mr{d}z=\int_{z_i}^{z_{i+1}}B_h(z)\mr{d}z=\sum_{j=0}^{N_1-1}b_{j+1/2}\int_{z_i}^{z_{i+1}}\varphi_{j+1/2}^1(z)\mr{d}z=\frac{c_{k+1}-c_k}{2}b_{i+1/2}\quad\quad\forall\,z_i\in[c_k,c_{k+1}),\label{eq_coefficients_V1}
\end{align}
where $(c_{k+1}-c_k)/2$ is the Jacobian originating from evaluating the integral in (\ref{eq_coefficients_V1}) on the reference element $I$. This choice for the bases of the space $V_0$ and $V_1$ together with the projectors $\Pi_0$ in (\ref{eq_def_projector0}) and $\Pi_1$ in (\ref{eq_def_projector1}) leads to the following consideration: take $\psi\in H^1$ and note that
\begin{align}
&\int_{z_i}^{z_{i+1}}(\Pi_1\frac{\pa\psi}{\pa z})(z)\mr{d}z\underset{\underset{\text{(\ref{eq_def_projector1})}}{\uparrow}}{=}\int_{z_i}^{z_{i+1}}\frac{\pa\psi}{\pa z}(z)\mr{d}z=\psi(z_{i+1})-\psi(z_i)\underset{\underset{\text{(\ref{eq_def_projector0})}}{\uparrow}}{=}(\Pi_0\psi)(z_{i+1})-(\Pi_0\psi)(z_{i})=\int_{z_i}^{z_{i+1}}\frac{\pa}{\pa z}(\Pi_0\psi)(z)\mr{d}z.
\end{align}
Since the integrations from $z_i$ to $z_{i+1}$ for $i=0,\ldots,N_0-1$ uniquely define an element of $V_1$, we get $\Pi_1\pa\psi/\pa z=\pa/\pa z(\Pi_0\psi)$ and hence the diagram is commuting. 

\textbf{Semi-discratization in space}. In order to obtain a matrix formulation out of the (discrete) weak formulation (\ref{eq_weak_gem_discrete}), we express all quantities in their respective basis by
\begin{align}
\tilde{E}_{hx/y}(z,t)=\sum_{j=0}^{N_0-1}e_{x/yj}(t)\varphi_j^0(z),\quad\tilde{B}_{hx/y}(z,t)=\sum_{j=0}^{N_1-1}b_{x/yj+1/2}(t)\varphi_{j+1/2}^1(z),\quad\tilde{j}_{\mr{c}x/y}^h(z,t)=\sum_{j=0}^{N_0-1}y_{x/yj}(t)\varphi_j^0(z),
\end{align}
and substitute this in the weak formulation (\ref{eq_weak_gem_discrete}). The same is done for the test functions $D_{hx/y}\in V_0$, $C_{hx/y}\in V_1$ and $O_{hx/y}\in V_0$. Let us do this in an exemplary way for the $x$-component of Amper\'{e}re's law (\ref{eq_weak_gem_discrete_1}) by noting that the spatial derivative in the second term is acting on the test function $D_{hx}\in V_0$ with coefficients $(d_{xj})_{j=0,\ldots,N_0-1}$. According to the diagram in Fig. \ref{fig_commuting_diagram}, this has the consequence that the function $\pa D_{hx}/\pa z$ must now be an element of the space $V_1$ with new coefficients $(d_{xj+1/2})_{j=0,\ldots,N_1-1}$, which are given by formula (\ref{eq_coefficients_V1}):
\begin{align}
\frac{c_{k+1}-c_k}{2}d_{xj+1/2}=\int_{z_j}^{z_{j+1}}\frac{\pa D_{hx}}{\pa z}\mr{d}z=\sum_{i=0}^{N_0-1}d_{xi}\int_{z_j}^{z_{j+1}}\frac{\pa}{\pa z}\varphi_i^0(z)\mr{d}z=\sum_{i=0}^{N_0-1}d_{xi}\left[\varphi_i^0(z_{j+1})-\varphi_i^0(z_{j})\right]=d_{xj+1}-d_{xj}.
\end{align} 
For a uniform mesh $c_{k+1}-c_k=h$ we hence get from (\ref{eq_weak_gem_discrete_1}) 
\begin{align}
\begin{split}
&\sum_{i,j}^{N_0-1}\frac{\mr{d}e_{xj}}{\mr{d}t}d_{xi}\underbrace{\int_0^L\varphi_i^0\varphi_j^0\mr{d}z}_{=:m_{0ij}}-\frac{2c^2}{h}\sum_{i,j=0}^{N_1-1}b_{yj+1/2}(d_{xi+1}-d_{xi})\underbrace{\int_0^L\varphi_{i+1/2}^1\varphi_{j+1/2}^1\mr{d}z}_{=:m_{1ij}}+\mu_0c^2\sum_{i,j=0}^{N_0-1}y_{xj}d_{xi}\underbrace{\int_0^L\varphi_i^0\varphi_j^0\mr{d}z}_{=:m_{0ij}}\\
=&-\mu_0c^2\sum_{i=0}^{N_0-1}d_{xi}\underbrace{\int_0^Lj_{\mr{h}x}\varphi_i^0\mr{d}z}_{=:\bar{j}_{\mr{h}xi}}.
\end{split}
\end{align}
Here, we defined the entries of the two mass matrices $\mathbb{M}_0:=(m_{0ij})_{i,j=0,\ldots,N_0-1}\in\mathbb{R}^{N_0\times N_0}$ and $\mathbb{M}_1:=(m_{1ij})_{i,j=0,\ldots,N_1-1}\in\mathbb{R}^{N_1\times N_1}$, respectively, as well as the vector $\bar{\mb{j}}_{\mr{h}x}:=(\bar{j}_{\mr{h}xi})_{i=0,\ldots,N_0-1}\in\mathbb{R}^{N_0}$ for the right-hand side, which is coupled to the PIC part of the algorithm in the exact same way as it was done in (\ref{eq_hotcurrent_weak}). All together, this leads to the equivalent matrix formulation
\begin{subequations}
\begin{align}
&\textbf{d}_x^\top\mathbb{M}_0\frac{\text{d}\textbf{e}_x}{\text{d}t}-c^2(\mathbb{G}\textbf{d}_x)^\top\mathbb{M}_1\textbf{b}_y+\mu_0c^2\textbf{d}_x^\top\mathbb{M}_0\textbf{y}_x=-\mu_0c^2q_\mr{e}\textbf{d}_x^\top\mathbb{Q}^0\mathbb{W}\mb{V}_x,\qquad\forall \mb{d}_x\in\mathbb{R}^{N_0},\\[3mm]\Leftrightarrow\quad&\mathbb{M}_0\frac{\text{d}\textbf{e}_x}{\text{d}t}-c^2\mathbb{G}^\top\mathbb{M}_1\textbf{b}_y+\mu_0c^2\mathbb{M}_0\textbf{y}_x=-\mu_0c^2q_\mr{e}\mathbb{Q}^0\mathbb{W}\mb{V}_x,
\end{align}
\end{subequations}
were we introduced the vector $\mb{V}_x=(v_{1x}\ldots,v_{N_\mr{p}x})^\top\in\mathbb{R}^{N_\mr{p}}$ holding the particles' velocities in $x$-direction. The matrices $\mathbb{Q}^0\in\mathbb{R}^{N_0\times N_\mr{p}}$ and $\mathbb{W}\in\mathbb{R}^{N_\mr{p}\times N_\mr{p}}$ defined by
\begin{subequations}
\begin{align}
&\mathbb{Q}^0=\mathbb{Q}^0(\mb{Z}):=(\varphi_i^0(z_k))_{i=0,\ldots,N_0-1;k=1\ldots,N_\mr{p}},\label{eq_def_Q0}\\
&\mathbb{W}:=\mr{diag}(w_1,\ldots,w_{N_\mr{p}}),\label{eq_def_W}
\end{align}
\end{subequations}
with $\mb{Z}=(z_1\ldots,z_{N_\mr{p}})^\top\in\mathbb{R}^{N_\mr{p}}$ being the particle positions, simply result from writing (\ref{eq_hotcurrent_weak}) in terms of matrix-vector multiplications. Finally, we introduced the discrete gradient matrix
\begin{align}
\mathbb{G}:=\frac{2}{h}
\begin{pmatrix}
 -1 & 1  &  &  &  \\
  & -1 & 1 & &   \\
  &  & \ddots & \ddots &  \\
  &  &   & -1 & 1 \\
  1 &  &  &    & -1  
\end{pmatrix} \quad\in\mathbb{R}^{N_1\times N_0},\label{eq_discrete_gradient}
\end{align}
where the last row is due to periodic boundary conditions and thus $d_{N_0}=d_0$, for instance.

Doing the same for the other equations in (\ref{eq_weak_gem_discrete}) as well as for the equations of motion for the particles (\ref{eq_motion_particles}), leads to the following semi-discrete system for the ten variables $\mb{\mb{u}}=(\mb{e}_x,\mb{e}_y,\mb{b}_x,\mb{b}_y,\mb{y}_x,\mb{y}_y,\mb{Z},\mb{V}_x,\mb{V}_y,\mb{V}_z)\in\mathbb{R}^{4N_0+2N_1+4N_\mr{p}}$:
\begin{subequations}
\label{eq_semi}
\begin{align}
    &\mathbb{M}_0\frac{\mathrm{d}\textbf{e}_x}{\mathrm{d} t}=c^2\mathbb{G}^\top\mathbb{M}_1\textbf{b}_y-\mu_0c^2\mathbb{M}_0\textbf{y}_x-\mu_0c^2q_\text{e}\mathbb{Q}^0\mathbb{W}\textbf{V}_x,\label{eq_semi1}\\
    &\mathbb{M}_0\frac{\mathrm{d}\textbf{e}_y}{\mathrm{d} t}=-c^2\mathbb{G}^\top\mathbb{M}_1\textbf{b}_x-\mu_0c^2\mathbb{M}_0\textbf{y}_y-\mu_0c^2q_\text{e}\mathbb{Q}^0\mathbb{W}\textbf{V}_y,\label{eq_semi2}\\
    &\frac{\mathrm{d}\textbf{b}_x}{\mathrm{d} t}=\mathbb{G}\textbf{e}_y,\label{eq_semi3}\\
    &\frac{\mathrm{d}\textbf{b}_y}{\mathrm{d} t}=-\mathbb{G}\textbf{e}_x,\label{eq_semi4}\\
    &\frac{\mathrm{d}\textbf{y}_x}{\mathrm{d} t}=\epsilon_0\Omega_\text{pe}^2\textbf{e}_x+\Omega_\text{ce}\textbf{y}_y,\label{eq_semi5}\\
    &\frac{\mathrm{d}\textbf{y}_y}{\mathrm{d} t}=\epsilon_0\Omega_\text{pe}^2\textbf{e}_y-\Omega_\text{ce}\textbf{y}_x,\label{eq_semi6}\\
    &\frac{\mathrm{d}\textbf{Z}}{\mathrm{d} t}=\textbf{V}_z,\label{eq_semi7}\\
    &\frac{\mathrm{d}\textbf{V}_x}{\mathrm{d} t}=\frac{q_\text{e}}{m_\text{e}}[(\mathbb{Q}^0)^\top\textbf{e}_x-\mathbb{B}_y\textbf{V}_z+B_0\textbf{V}_y],\label{eq_semi8}\\
    &\frac{\mathrm{d}\textbf{V}_y}{\mathrm{d} t}=\frac{q_\text{e}}{m_\text{e}}[(\mathbb{Q}^0)^\top\textbf{e}_y+\mathbb{B}_x\textbf{V}_z-B_0\textbf{V}_x], \label{eq_semi9}\\
    &\frac{\mathrm{d}\textbf{V}_z}{\mathrm{d} t}=\frac{q_\text{e}}{m_\text{e}}[\mathbb{B}_y\textbf{V}_x-\mathbb{B}_x\textbf{V}_y],\label{eq_semi10},
\end{align}
\end{subequations}
where the matrices $\mathbb{Q}^1\in\mathbb{R}^{N_1\times N_\mr{p}}$ and $\mathbb{B}_{x/y}\in\mathbb{R}^{N_\mr{p}\times N_\mr{p}}$ defined by
\begin{align}
&\mathbb{Q}^1=\mathbb{Q}^1(\mb{Z}):=(\varphi_{i+1/2}^1(z_k))_{i=0,\ldots,N_1-1;k=1\ldots,N_\mr{p}},\label{eq_def_Q1}\\
&\mathbb{B}_{x/y}=\mathbb{B}_{x/y}(\mb{Z},\mb{b}_{x/y}):=\mr{diag}\left[(\mathbb{Q}^1)^\top(\mb{Z})\mb{b}_{x/y}\right],\label{eq_def_Bxy}
\end{align}
arise naturally after writing the particles' equations of motion (\ref{eq_motion_particles}) in matrix-vector form and noting that the discrete electric and magnetic fields can be expressed in their respective bases (see (\ref{eq_fields_particles})).

In order to analyze the semi-discrete system of equations (\ref{eq_semi}), we define the system's discrete Hamiltonian ${H_h:\mathbb{R}^n\rightarrow\mathbb{R}}$, $\mb{u}\mapsto H_h(\mb{u})$ ($n=4N_0+2N_1+4N_\mr{p}$) by replacing the continuous functions in the energy (\ref{eq_total_energy}) by their discrete counterparts. This results in
\begin{align}
\label{eq_discrete_Hamiltonian}
\begin{split}
H_h(\mb{u}):=&\underbrace{\frac{\epsilon_0}{2}(\mb{e}_x^\top\mathbb{M}_0\mb{e}_x+\mb{e}_y^\top\mathbb{M}_0\mb{e}_y)}_{H_E}+\underbrace{\frac{1}{2\mu_0}(\mb{b}_x^\top\mathbb{M}_1\mb{b}_x+\mb{b}_y^\top\mathbb{M}_1\mb{b}_y)}_{H_B}+\underbrace{\frac{1}{2\epsilon_0\Omega_\mr{pe}^2}(\mb{y}_x^\top\mathbb{M}_0\mb{y}_x+\mb{y}_y^\top\mathbb{M}_0\mb{y}_y)}_{H_Y}\\
+&\underbrace{\frac{m_\mr{e}}{2}\mb{V}_x^\top\mathbb{W}\mb{V}_x}_{H_x}+\underbrace{\frac{m_\mr{e}}{2}\mb{V}_y^\top\mathbb{W}\mb{V}_y}_{H_y}+\underbrace{\frac{m_\mr{e}}{2}\mb{V}_z^\top\mathbb{W}\mb{V}_z}_{H_z}.
\end{split}
\end{align} 
Using this discrete Hamiltonian, it is straightforward to show that the semi-discrete system (\ref{eq_semi}) can be equivalently written in a noncanonical Hamiltonian structure for the dynamics of the variable $\mb{u}$:
\begin{align}
\frac{\mr{d}\mb{u}}{\mr{d}t}=\mathbb{J}(\mb{u})\nabla_\mb{u}H_h(\mb{u}).\label{eq_Hamiltonian_structure}
\end{align}
\begin{lemma}
The matrix $\mathbb{J}$ in (\ref{eq_Hamiltonian_structure}) is skew-symmetric and satisfies the Jacobi identity,
\begin{align}
\sum_{l}\left(\frac{\pa \mathbb{J}_{ab}}{\pa u_l}\mathbb{J}_{lc}+\frac{\pa \mathbb{J}_{bc}}{\pa u_l}\mathbb{J}_{la}+\frac{\pa \mathbb{J}_{ca}}{\pa u_l}\mathbb{J}_{lb}\right)=0,\qquad\forall\ a,b,c.\label{eq_Jacobi}
\end{align}
\begin{table}
\centering
\caption{Block index triples for which the terms in (\ref{eq_Jacobi_2}) are not equal to zero.\label{tab_Jacobi1}}
\begin{tabular}{|c|c|}
\hline
\rule{0pt}{2ex}  Term & Block indices (i,j,k)\\
\hline \rule{0pt}{2ex}
\uproman{1} &(9,10,2)\,\, (10,9,2) \\
\uproman{2} &(2,9,10)\,\, (2,10,9) \\
\uproman{3} &(9,2,10)\,\, (10,2,9) \\
\uproman{4} &(8,10,1)\,\, (10,8,1) \\
\uproman{5} &(1,8,10)\,\, (1,10,8) \\
\uproman{6} &(8,1,10)\,\, (10,1,8) \\
\uproman{7} &(1,8,10)\,\, (8,1,10)\,\, (2,9,10)\,\, (9,2,10)\,\, (8,10,10)\,\, (10,8,10)\,\, (9,10,10)\,\, (10,9,10) \\
\uproman{8} &(10,1,8)\,\, (10,8,1)\,\, (10,2,9)\,\, (10,9,2)\,\, (10,8,10)\,\, (10,10,8)\,\, (10,9,10)\,\, (10,10,9) \\
\uproman{9} &(1,10,8)\,\, (8,10,1)\,\, (2,10,9)\,\, (9,10,2)\,\, (8,10,10)\,\, (10,10,8)\,\, (9,10,10)\,\, (10,10,9) \\
\hline
\end{tabular}
\end{table}
\begin{proof}
The matrix $\mathbb{J}$ is written explicitly in \ref{sec_appendix1} in a $10\times10$ block structure. From this, the skew-symmetry $\mathbb{J}^\top=-\mathbb{J}$ is obvious. To proof the Jacobi identity we again take advantage of the $10\times10$ block structure of $\mathbb{J}$ and denote the $(i,j)$-th block by $\hat{\mathbb{J}}_{i,j}$($1\leq i\leq 10$, $1\leq j\leq 10$). Due to the fact that only very few blocks depend on the unknown $\mb{u}$, namely $\hat{\mathbb{J}}_{1,8}$, $\hat{\mathbb{J}}_{8,1}$, $\hat{\mathbb{J}}_{2,9}$, $\hat{\mathbb{J}}_{9,2}$, $\hat{\mathbb{J}}_{8,10}$, $\hat{\mathbb{J}}_{10,8}$, $\hat{\mathbb{J}}_{9,10}$ and $\hat{\mathbb{J}}_{10,9}$ via $\mathbb{B}_x=\mathbb{B}_x(\mb{Z},\mb{b}_x)$, $\mathbb{B}_y=\mathbb{B}_y(\mb{Z},\mb{b}_y)$ and $\mathbb{Q}^0=\mathbb{Q}^0(\mb{Z})$, the Jacobi identity (\ref{eq_Jacobi}) reduces to
\begin{align}
\label{eq_Jacobi_2}
\begin{split}
0&=\underbrace{\frac{\pa\hat{\mathbb{J}}_{i,j}}{\pa\mb{b}_x}\hat{\mathbb{J}}_{3,k=2}}_{\text{\uproman{1}}}+\underbrace{\frac{\pa\hat{\mathbb{J}}_{j,k}}{\pa\mb{b}_x}\hat{\mathbb{J}}_{3,i=2}}_{\text{\uproman{2}}}+\underbrace{\frac{\pa\hat{\mathbb{J}}_{k,i}}{\pa\mb{b}_x}\hat{\mathbb{J}}_{3,j=2}}_{\text{\uproman{3}}}+\underbrace{\frac{\pa\hat{\mathbb{J}}_{i,j}}{\pa\mb{b}_y}\hat{\mathbb{J}}_{4,k=1}}_{\text{\uproman{4}}}+\underbrace{\frac{\pa\hat{\mathbb{J}}_{j,k}}{\pa\mb{b}_y}\hat{\mathbb{J}}_{4,i=1}}_{\text{\uproman{5}}}+\underbrace{\frac{\pa\hat{\mathbb{J}}_{k,i}}{\pa\mb{b}_y}\hat{\mathbb{J}}_{4,j=1}}_{\text{\uproman{6}}}\\
&+\underbrace{\frac{\pa\hat{\mathbb{J}}_{i,j}}{\pa\mb{Z}}\hat{\mathbb{J}}_{7,k=10}}_{\text{\uproman{7}}}+\underbrace{\frac{\pa\hat{\mathbb{J}}_{j,k}}{\pa\mb{Z}}\hat{\mathbb{J}}_{7,i=10}}_{\text{\uproman{8}}}+\underbrace{\frac{\pa\hat{\mathbb{J}}_{k,i}}{\pa\mb{Z}}\hat{\mathbb{J}}_{7,j=10}}_{\text{\uproman{9}}},\qquad\forall\ i,j,k.
\end{split}
\end{align}
Here, we could already identify one block index in each (e.g. $k=2$ for term \uproman{1} or $k=1$ for term \uproman{4}). The other indices can be determined from the aforementioned dependencies of the matrices $\mathbb{B}_x$, $\mathbb{B}_y$ and $\mathbb{Q}^0$ on $\mb{b}_{x/y}$ and $\mb{Z}$, respectively. In Tab., \ref{tab_Jacobi1} we list the resulting block index combinations giving a non-zero contribution for each term \uproman{1},\,\ldots\,,\uproman{9}. Summing terms corresponding to identical index triples leads to 18 different index triples listed in Tab. \ref{tab_Jacobi2} for which the Jacobi identity in the form (\ref{eq_Jacobi_2}) needs to proven. Since the Jacobi identity gives the same expression for cyclic permutations of $(i,j,k)$, there are always three index triples which are equivalent. Consequently, there are only six distinct expressions that need to be checked. It is immediately clear that the last two expressions in Tab. \ref{tab_Jacobi2} are equal to zero and that the first and second and the third and fourth expression, respectively, are the same up to the sign. The remaining two expressions only differ with respect to $\pa\mathbb{B}_x/\pa\mb{b}_x$ and $\pa\mathbb{B}_y/\pa\mb{b}_y$. Because of the definitions (\ref{eq_def_Bxy}) of $\mathbb{B}_x$ and $\mathbb{B}_y$, respectively, these terms are again equivalent which means that we only have to proof one combination explicitly, for example
\begin{align}
\sum_l\frac{\pa(\mathbb{B}_y(\mb{Z},\mb{b}_y)\mathbb{W}^{-1})_{ab}}{\pa b_{yl+1/2}}(\mathbb{G}\mathbb{M}_0^{-1})_{lc}=\sum_l\frac{\pa(\mathbb{M}_0^{-1}\mathbb{Q}^0(\mb{Z}))_{ca}}{\pa z_{l}}(\mathbb{W}^{-1})_{lb},\qquad\forall\ a,b,c.\label{eq_proof_Jacobi_1}
\end{align} 
Writing all matrix products explicitly yields
\begin{align}
\sum_{l,m,n,r}(\mathbb{Q}^{1})^\top_{am}\delta_{an}\underbrace{\frac{\pa b_{ym+1/2}}{\pa b_{y_{l+1/2}}}}_{=\delta_{lm}}\delta_{nb}\frac{1}{w_n}\mathbb{G}_{lr}(\mathbb{M}_0^{-1})_{rc}=\sum_{l,m}(\mathbb{M}_0^{-1})_{cm}\underbrace{\frac{\pa\varphi^0_{m}(z_a)}{\pa z_l}}_{=\delta_{al}(\mr{d}\varphi_m^0/\mr{d}z)(z_a)}\delta_{lb}\frac{1}{w_l}\,.
\end{align}
As a next step, we eliminate all sums involving a Kronecker delta. This results in
\begin{align}
\delta_{ab}\frac{1}{w_a}\sum_{m,r}\varphi_{m+1/2}^1(z_a)\mathbb{G}_{mr}(\mathbb{M}_0^{-1})_{rc}=\delta_{ab}\frac{1}{w_a}\sum_m(\mathbb{M}_0^{-1})_{cm}\frac{\mr{d}\varphi_m^0}{\mr{d}z}(z_a).
\end{align}
Using that the discrete gradient matrix (\ref{eq_discrete_gradient}) can be written as $\mathbb{G}_{mr}=2(\delta_{mr-1}-\delta_{mr})/h$ and performing the sum over $m$ yields
\begin{align}
\delta_{ab}\frac{1}{w_a}\frac{2}{h}\sum_r(\mathbb{M}_0^{-1})_{rc}(\varphi_{r-1/2}^1(z_a)-\varphi_{r+1/2}^1(z_a))=\delta_{ab}\frac{1}{w_a}\sum_r(\mathbb{M}_0^{-1})_{rc}\frac{\mr{d}\varphi_r^0}{\mr{d}z}(z_a),
\end{align}
where we have used the symmetry of the inverse of the mass matrix $(\mathbb{M}_0^{-1})_{rc}=(\mathbb{M}_0^{-1})_{cr}$. Furthermore, we renamed the summation index on the right-hand-side from $m$ to $r$.
Since the basis function on both sides are evaluated at the same particle position $z_a$ it remains to show that
\begin{align}
\varphi_{r-1/2}^1-\varphi_{r+1/2}^1=\frac{h}{2}\frac{\mr{d}\varphi_r^0}{\mr{d}z},
\end{align}
which is true due to our particular choice of basis functions satisfying the commuting diagram in Fig. \ref{fig_commuting_diagram}. By using the mappings $F_k$ and $F_k^{-1}$ in  (\ref{eq_mapping}) from real space to the reference element $I=[-1,1]$ and by using the definition (\ref{eq_def_LHP}) of the LHPs $(\chi_{n+1/2})_{n=0,\ldots,p-1}$, we get
\begin{align}
\begin{split}
&\varphi_{r-1/2}^1(F_k(s))-\varphi_{r+1/2}^1(F_k(s))=\chi_{n-1/2}(s)-\chi_{n+1/2}(s)\\=&\sum_{m=n}^p\frac{\mr{d}}{\mr{d}s}\eta_m(s)-\sum_{m=n+1}^p\frac{\mr{d}}{\mr{d}s}\eta_m(s)=\frac{\mr{d}}{\mr{d}s}\eta_n(s)\\
=&\frac{\mr{d}}{\mr{d}s}\eta_n(F_k^{-1}(F_k(s)))=\frac{\mr{d}}{\mr{d}s}\varphi^0_r(F_k(s))=\frac{\mr{d}F_k}{\mr{d}s}\frac{\mr{d}\varphi_r^0}{\mr{d}z}=\frac{h}{2}\frac{\mr{d}\varphi_r^0}{\mr{d}z},
\end{split}
\end{align}
which completes the proof of the Jacobi identity (\ref{eq_Jacobi}).
\end{proof}
\end{lemma}

With the stated properties of $\mathbb{J}$, we can define the following Poisson bracket, a bilinear, anti-symmetric bracket, that satisfies Leibniz' rule and the Jacobi identity: 
\begin{align}
\{R,S\}=\nabla_\mb{u}R^\top\mathbb{J}(\mb{u})\nabla_\mb{u}S
\end{align} 
for two functions $R,S:\mathbb{R}^n\rightarrow\mathbb{R},\mb{u}\mapsto R,S(\mb{u})$ of the dynamical variables $\mb{u}$. This means that the time evolution of an arbitrary function $R$ can be written as
\begin{align}
\frac{\mr{d}}{\mr{d}t}R(\mb{u}(t))=\nabla_\mb{u}R^\top\frac{\mr{d}\mb{u}}{\mr{d}t}\underset{\underset{\text{(\ref{eq_Hamiltonian_structure})}}{\uparrow}}{=}\nabla_\mb{u}R^\top\mathbb{J}(\mb{u})\nabla_\mb{u}H_h=\{R,H_h\},
\end{align}
and taking $R=H_h$ and using the anti-symmetry of the bracket yields
\begin{align}
\frac{\mr{d}}{\mr{d}t}H_h(\mb{u}(t))=\{H_h,H_h\}=-\{H_h,H_h\}=0,
\end{align}
which means that the semi-discrete system (\ref{eq_Hamiltonian_structure}) exactly conserves the discrete Hamiltonian (\ref{eq_discrete_Hamiltonian}).

\textbf{Discretization in time}. We once more follow \citep{Krausetal2017} and choose a splitting scheme for the integration of the Hamiltonian system (\ref{eq_Hamiltonian_structure}) in time. For Hamiltonian systems there are in principle two options: The first one is to split the Poisson matrix $\mathbb{J}$ and to keep the full Hamiltonian. If each of the subsystems can then be solved analytically, this yields exact energy conservation. Or one splits the Hamiltonian while keeping the full Poisson matrix. This yields so-called Poisson integrators which have the advantage that some invariants, the so-called Casimir invariants of Hamiltonian systems, are preserved exactly even on the fully discretized level. We choose the latter option and consequently split the Hamiltonian (\ref{eq_discrete_Hamiltonian}) into the six parts 
\begin{align}
H_h=H_E+H_B+H_Y+H_x+H_y+H_z,
\end{align} 
in order to obtain six subsystems which still have the form (\ref{eq_Hamiltonian_structure}), however, with a simpler Hamiltonian, respectively. We find that each of the subsystems can be solved analytically in the way listed in \ref{sec_appendix3}, which means that we get a set of six Poisson integrators denoted by $\Phi_{\Delta t}^E$, $\Phi_{\Delta t}^B$, $\Phi_{\Delta t}^Y$, $\Phi_{\Delta t}^x$, $\Phi_{\Delta t}^y$ and $\Phi_{\Delta t}^z$, which can be applied successively in some specific order to advance $\mb{u}$ by a time step $\Delta t$. The easiest composition is the first-order Lie-Trotter splitting \citep{Trotter1959}, which consists of simply applying each integrator one after the other:
\begin{align}
\Phi_{\Delta t}^L:=\Phi_{\Delta t}^z\circ\Phi_{\Delta t}^y\circ\Phi_{\Delta t}^x\circ\Phi_{\Delta t}^Y\circ\Phi_{\Delta t}^B\circ\Phi_{\Delta t}^E.\label{eq_LieTrotter}
\end{align}
It is important to note that the input to each integrator must be the output of the previous integrator which has the consequence that if the magnetic field coefficients $\mb{b}_x$ and $\mb{b}_y$ change, for instance, the matrices $\mathbb{B}_{x/y}=\mathbb{B}_{x/y}(\mb{Z},\mb{b}_{x/y})$ need to be updated. Furthermore, we use the second order, symmetric Strang splitting \citep{Strang1968}
\begin{align}
\Phi_{\Delta t}^S:=\Phi_{\Delta t/2}^z\circ\Phi_{\Delta t/2}^y\circ\Phi_{\Delta t/2}^x\circ\Phi_{\Delta t/2}^Y\circ\Phi_{\Delta t/2}^B\circ\Phi_{\Delta t/2}^E\circ\Phi_{\Delta t}^E\circ\Phi_{\Delta t}^B\circ\Phi_{\Delta t}^Y\circ\Phi_{\Delta t}^x\circ\Phi_{\Delta t}^y\circ\Phi_{\Delta t}^z.\label{eq_Strang}
\end{align} 
Higher order splitting schemes can e.g. be found in \citep{McLachlanetal2012}.

\textbf{Algorithm}. Finally, like it was done in the previous section, we want to summarize the algorithm for for numerically solving the model (\ref{eq_model_linearized}) for transverse electromagnetic waves only:
\begin{enumerate}
\item Create a periodic basis of Lagrange polynomials $(\varphi_j^0(z))_{j=0,\ldots,N_0-1}$ of degree $p$ on a domain $L$ discretized by $N_\text{el}$ elements using the definition of the shape functions (\ref{eq_def_Lagrange_shape}) on the reference element $I=[-1,1]$ and the formulas (\ref{eq_mapping}) for transformations on the physical domain. This results in $N_0=pN_\text{el}$.
\item Create the corresponding basis of Lagrange histopolation polynomials $(\varphi_{j+1/2}^1(z))$ $_{j=0,\ldots,N_1-1}$ using the definition of the shape functions (\ref{eq_def_Lagrange_histo}) on the reference element $I=[-1,1]$ and the formulas (\ref{eq_mapping}) for transformations on the physical domain. This results in $N_1=pN_\text{el}$.
\item Assemble the global mass matrices $\mathbb{M}_0$ and $\mathbb{M}_1$.
\item Load the initial fields $\tilde{E}_x(z,t=0)$, $\tilde{E}_y(z,t=0)$, $\tilde{B}_x(z,t=0)$, $\tilde{B}_y(z,t=0)$, $\tilde{j}_{\text{c}x}(z,t=0)$, $\tilde{j}_{\text{c}y}(z,t=0)$ and use the projectors $\Pi_0$ (\ref{eq_def_projector0}) and $\Pi_1$ (\ref{eq_def_projector1}) in order to get the initial finite element coefficients $\textbf{e}_x^0$, $\textbf{e}_y^0$, $\textbf{b}_x^0$, $\textbf{b}_y^0$, $\textbf{y}_x^0$, $\textbf{y}_y^0$.
\item Sample the initial positions $(z_k^0)_{k=1,\ldots,N_\mr{p}}$ and velocities $(v_{kx}^0,v_{ky}^0,v_{kz}^0)_{k=1,\ldots,N_\mr{p}}$ according to the sampling distribution (\ref{eq_sampling_distribution}) by using a random number generator and compute the weights $w_k=n_{\mr{h}0}L/N_\mr{p}$.
\item Assemble the matrices $\mathbb{G}$ (\ref{eq_discrete_gradient}), $\mathbb{Q}^0(\textbf{Z}^0)$ (\ref{eq_def_Q0}), $\mathbb{Q}^1(\textbf{Z}^0)$ (\ref{eq_def_Q1}), $\mathbb{B}_x(\textbf{Z}^0,\textbf{b}_x^0)$ (\ref{eq_def_Bxy}), $\mathbb{B}_y(\textbf{Z}^0,\textbf{b}_y^0)$ (\ref{eq_def_Bxy}) and $\mathbb{W}$ (\ref{eq_def_W}).
\item Start the time loop:
	\begin{enumerate}[label*=\arabic*]
	\item Apply one of the time integrators (\ref{eq_LieTrotter}) (Lie-Trotter) or (\ref{eq_Strang}) (Strang) for a time step $\Delta t$ in order to update $\textbf{e}_x^n$, $\textbf{e}_y^n$, $\textbf{b}_x^n$, $\textbf{b}_y^n$, $\textbf{y}_x^n$, $\textbf{y}_y^n$, $\textbf{Z}^n$, $\textbf{V}_x^n$, $\textbf{V}_y^n$, $\textbf{V}_z^n$ $\rightarrow$ $\textbf{e}_x^{n+1}$, $\textbf{e}_y^{n+1}$, $\textbf{b}_x^{n+1}$, $\textbf{b}_y^{n+1}$, $\textbf{y}_x^{n+1}$, $\textbf{y}_y^{n+1}$, $\textbf{Z}^{n+1}$, $\textbf{V}_x^{n+1}$, $\textbf{V}_y^{n+1}$, $\textbf{V}_z^{n+1}$. The single integrators are listed in \ref{sec_appendix3}.
	\item Go to 7.1
	\end{enumerate}
\end{enumerate}


\section{Numerical experiments}
\label{sec_numerical_results}
In this section, we present results of two runs performed with each algorithm developed in the previous two sections (Sec. \ref{sec_standard} and Sec. \ref{sec_geometric}). In the first run, we excite the instability stated in section \ref{sec_dispersion} for a single wavenumber $k$, while in the second run we excite multiple modes without expecting an instability.

\subsection{Run 1: Single $k$-mode}
For the first run, we initialize the codes as follows: We choose an anisotropic Maxwellian for the energetic electrons and perturb the $x$-component of the magnetic wave field by $\tilde{B}_x(z,t=0)=a\sin(kz)$ in order to seed the instability for one particular $k$-mode. The amplitude $a$ is chosen with respect to the background magnetic field such that it is small enough to start in the linear phase, but large enough to reach the nonlinear phase within a reasonable simulation time. All other field quantities are initially zero, which means that there is no electric field and cold plasma current at $t=0$. All physical and numerical parameters of the run are listed in Tab. \ref{tab_parameters}. Note that we have chosen a polynomial degree of $p=1$ in order to get basis functions which are as similar as possible for the two codes since B-splines and Lagrange polynomials are the same only for this degree (see Fig. \ref{fig_Bsplines_periodic}a). In this case, the main difference between the two codes is that the magnetic field is still expressed with piecewise linear functions for standard finite elements, but with piecewise constant functions for structure-preserving geometric finite elements. 
\begin{figure}[!t]
\centering
\includegraphics[scale=1]{01_Figures/comparison_1e5.pdf}
%%% Creator: Matplotlib, PGF backend
%%
%% To include the figure in your LaTeX document, write
%%   \input{<filename>.pgf}
%%
%% Make sure the required packages are loaded in your preamble
%%   \usepackage{pgf}
%%
%% Figures using additional raster images can only be included by \input if
%% they are in the same directory as the main LaTeX file. For loading figures
%% from other directories you can use the `import` package
%%   \usepackage{import}
%% and then include the figures with
%%   \import{<path to file>}{<filename>.pgf}
%%
%% Matplotlib used the following preamble
%%   \usepackage{fontspec}
%%   \setmainfont{DejaVu Serif}
%%   \setsansfont{DejaVu Sans}
%%   \setmonofont{DejaVu Sans Mono}
%%
\begingroup%
\makeatletter%
\begin{pgfpicture}%
\pgfpathrectangle{\pgfpointorigin}{\pgfqpoint{6.308836in}{2.497040in}}%
\pgfusepath{use as bounding box, clip}%
\begin{pgfscope}%
\pgfsetbuttcap%
\pgfsetmiterjoin%
\definecolor{currentfill}{rgb}{1.000000,1.000000,1.000000}%
\pgfsetfillcolor{currentfill}%
\pgfsetlinewidth{0.000000pt}%
\definecolor{currentstroke}{rgb}{1.000000,1.000000,1.000000}%
\pgfsetstrokecolor{currentstroke}%
\pgfsetdash{}{0pt}%
\pgfpathmoveto{\pgfqpoint{0.000000in}{0.000000in}}%
\pgfpathlineto{\pgfqpoint{6.308836in}{0.000000in}}%
\pgfpathlineto{\pgfqpoint{6.308836in}{2.497040in}}%
\pgfpathlineto{\pgfqpoint{0.000000in}{2.497040in}}%
\pgfpathclose%
\pgfusepath{fill}%
\end{pgfscope}%
\begin{pgfscope}%
\pgfsetbuttcap%
\pgfsetmiterjoin%
\definecolor{currentfill}{rgb}{1.000000,1.000000,1.000000}%
\pgfsetfillcolor{currentfill}%
\pgfsetlinewidth{0.000000pt}%
\definecolor{currentstroke}{rgb}{0.000000,0.000000,0.000000}%
\pgfsetstrokecolor{currentstroke}%
\pgfsetstrokeopacity{0.000000}%
\pgfsetdash{}{0pt}%
\pgfpathmoveto{\pgfqpoint{0.679669in}{0.526079in}}%
\pgfpathlineto{\pgfqpoint{3.038365in}{0.526079in}}%
\pgfpathlineto{\pgfqpoint{3.038365in}{2.187079in}}%
\pgfpathlineto{\pgfqpoint{0.679669in}{2.187079in}}%
\pgfpathclose%
\pgfusepath{fill}%
\end{pgfscope}%
\begin{pgfscope}%
\pgfsetbuttcap%
\pgfsetroundjoin%
\definecolor{currentfill}{rgb}{0.000000,0.000000,0.000000}%
\pgfsetfillcolor{currentfill}%
\pgfsetlinewidth{0.803000pt}%
\definecolor{currentstroke}{rgb}{0.000000,0.000000,0.000000}%
\pgfsetstrokecolor{currentstroke}%
\pgfsetdash{}{0pt}%
\pgfsys@defobject{currentmarker}{\pgfqpoint{0.000000in}{-0.048611in}}{\pgfqpoint{0.000000in}{0.000000in}}{%
\pgfpathmoveto{\pgfqpoint{0.000000in}{0.000000in}}%
\pgfpathlineto{\pgfqpoint{0.000000in}{-0.048611in}}%
\pgfusepath{stroke,fill}%
}%
\begin{pgfscope}%
\pgfsys@transformshift{0.679669in}{0.526079in}%
\pgfsys@useobject{currentmarker}{}%
\end{pgfscope}%
\end{pgfscope}%
\begin{pgfscope}%
\pgftext[x=0.679669in,y=0.428857in,,top]{\rmfamily\fontsize{10.000000}{12.000000}\selectfont \(\displaystyle 0\)}%
\end{pgfscope}%
\begin{pgfscope}%
\pgfsetbuttcap%
\pgfsetroundjoin%
\definecolor{currentfill}{rgb}{0.000000,0.000000,0.000000}%
\pgfsetfillcolor{currentfill}%
\pgfsetlinewidth{0.803000pt}%
\definecolor{currentstroke}{rgb}{0.000000,0.000000,0.000000}%
\pgfsetstrokecolor{currentstroke}%
\pgfsetdash{}{0pt}%
\pgfsys@defobject{currentmarker}{\pgfqpoint{0.000000in}{-0.048611in}}{\pgfqpoint{0.000000in}{0.000000in}}{%
\pgfpathmoveto{\pgfqpoint{0.000000in}{0.000000in}}%
\pgfpathlineto{\pgfqpoint{0.000000in}{-0.048611in}}%
\pgfusepath{stroke,fill}%
}%
\begin{pgfscope}%
\pgfsys@transformshift{1.269343in}{0.526079in}%
\pgfsys@useobject{currentmarker}{}%
\end{pgfscope}%
\end{pgfscope}%
\begin{pgfscope}%
\pgftext[x=1.269343in,y=0.428857in,,top]{\rmfamily\fontsize{10.000000}{12.000000}\selectfont \(\displaystyle 50\)}%
\end{pgfscope}%
\begin{pgfscope}%
\pgfsetbuttcap%
\pgfsetroundjoin%
\definecolor{currentfill}{rgb}{0.000000,0.000000,0.000000}%
\pgfsetfillcolor{currentfill}%
\pgfsetlinewidth{0.803000pt}%
\definecolor{currentstroke}{rgb}{0.000000,0.000000,0.000000}%
\pgfsetstrokecolor{currentstroke}%
\pgfsetdash{}{0pt}%
\pgfsys@defobject{currentmarker}{\pgfqpoint{0.000000in}{-0.048611in}}{\pgfqpoint{0.000000in}{0.000000in}}{%
\pgfpathmoveto{\pgfqpoint{0.000000in}{0.000000in}}%
\pgfpathlineto{\pgfqpoint{0.000000in}{-0.048611in}}%
\pgfusepath{stroke,fill}%
}%
\begin{pgfscope}%
\pgfsys@transformshift{1.859017in}{0.526079in}%
\pgfsys@useobject{currentmarker}{}%
\end{pgfscope}%
\end{pgfscope}%
\begin{pgfscope}%
\pgftext[x=1.859017in,y=0.428857in,,top]{\rmfamily\fontsize{10.000000}{12.000000}\selectfont \(\displaystyle 100\)}%
\end{pgfscope}%
\begin{pgfscope}%
\pgfsetbuttcap%
\pgfsetroundjoin%
\definecolor{currentfill}{rgb}{0.000000,0.000000,0.000000}%
\pgfsetfillcolor{currentfill}%
\pgfsetlinewidth{0.803000pt}%
\definecolor{currentstroke}{rgb}{0.000000,0.000000,0.000000}%
\pgfsetstrokecolor{currentstroke}%
\pgfsetdash{}{0pt}%
\pgfsys@defobject{currentmarker}{\pgfqpoint{0.000000in}{-0.048611in}}{\pgfqpoint{0.000000in}{0.000000in}}{%
\pgfpathmoveto{\pgfqpoint{0.000000in}{0.000000in}}%
\pgfpathlineto{\pgfqpoint{0.000000in}{-0.048611in}}%
\pgfusepath{stroke,fill}%
}%
\begin{pgfscope}%
\pgfsys@transformshift{2.448691in}{0.526079in}%
\pgfsys@useobject{currentmarker}{}%
\end{pgfscope}%
\end{pgfscope}%
\begin{pgfscope}%
\pgftext[x=2.448691in,y=0.428857in,,top]{\rmfamily\fontsize{10.000000}{12.000000}\selectfont \(\displaystyle 150\)}%
\end{pgfscope}%
\begin{pgfscope}%
\pgfsetbuttcap%
\pgfsetroundjoin%
\definecolor{currentfill}{rgb}{0.000000,0.000000,0.000000}%
\pgfsetfillcolor{currentfill}%
\pgfsetlinewidth{0.803000pt}%
\definecolor{currentstroke}{rgb}{0.000000,0.000000,0.000000}%
\pgfsetstrokecolor{currentstroke}%
\pgfsetdash{}{0pt}%
\pgfsys@defobject{currentmarker}{\pgfqpoint{0.000000in}{-0.048611in}}{\pgfqpoint{0.000000in}{0.000000in}}{%
\pgfpathmoveto{\pgfqpoint{0.000000in}{0.000000in}}%
\pgfpathlineto{\pgfqpoint{0.000000in}{-0.048611in}}%
\pgfusepath{stroke,fill}%
}%
\begin{pgfscope}%
\pgfsys@transformshift{3.038365in}{0.526079in}%
\pgfsys@useobject{currentmarker}{}%
\end{pgfscope}%
\end{pgfscope}%
\begin{pgfscope}%
\pgftext[x=3.038365in,y=0.428857in,,top]{\rmfamily\fontsize{10.000000}{12.000000}\selectfont \(\displaystyle 200\)}%
\end{pgfscope}%
\begin{pgfscope}%
\pgftext[x=1.859017in,y=0.238889in,,top]{\rmfamily\fontsize{10.000000}{12.000000}\selectfont \(\displaystyle t|\Omega_\mathrm{ce}|\)}%
\end{pgfscope}%
\begin{pgfscope}%
\pgfsetbuttcap%
\pgfsetroundjoin%
\definecolor{currentfill}{rgb}{0.000000,0.000000,0.000000}%
\pgfsetfillcolor{currentfill}%
\pgfsetlinewidth{0.803000pt}%
\definecolor{currentstroke}{rgb}{0.000000,0.000000,0.000000}%
\pgfsetstrokecolor{currentstroke}%
\pgfsetdash{}{0pt}%
\pgfsys@defobject{currentmarker}{\pgfqpoint{-0.048611in}{0.000000in}}{\pgfqpoint{0.000000in}{0.000000in}}{%
\pgfpathmoveto{\pgfqpoint{0.000000in}{0.000000in}}%
\pgfpathlineto{\pgfqpoint{-0.048611in}{0.000000in}}%
\pgfusepath{stroke,fill}%
}%
\begin{pgfscope}%
\pgfsys@transformshift{0.679669in}{0.526079in}%
\pgfsys@useobject{currentmarker}{}%
\end{pgfscope}%
\end{pgfscope}%
\begin{pgfscope}%
\pgftext[x=0.294444in,y=0.473318in,left,base]{\rmfamily\fontsize{10.000000}{12.000000}\selectfont \(\displaystyle 10^{-7}\)}%
\end{pgfscope}%
\begin{pgfscope}%
\pgfsetbuttcap%
\pgfsetroundjoin%
\definecolor{currentfill}{rgb}{0.000000,0.000000,0.000000}%
\pgfsetfillcolor{currentfill}%
\pgfsetlinewidth{0.803000pt}%
\definecolor{currentstroke}{rgb}{0.000000,0.000000,0.000000}%
\pgfsetstrokecolor{currentstroke}%
\pgfsetdash{}{0pt}%
\pgfsys@defobject{currentmarker}{\pgfqpoint{-0.048611in}{0.000000in}}{\pgfqpoint{0.000000in}{0.000000in}}{%
\pgfpathmoveto{\pgfqpoint{0.000000in}{0.000000in}}%
\pgfpathlineto{\pgfqpoint{-0.048611in}{0.000000in}}%
\pgfusepath{stroke,fill}%
}%
\begin{pgfscope}%
\pgfsys@transformshift{0.679669in}{1.190479in}%
\pgfsys@useobject{currentmarker}{}%
\end{pgfscope}%
\end{pgfscope}%
\begin{pgfscope}%
\pgftext[x=0.294444in,y=1.137718in,left,base]{\rmfamily\fontsize{10.000000}{12.000000}\selectfont \(\displaystyle 10^{-5}\)}%
\end{pgfscope}%
\begin{pgfscope}%
\pgfsetbuttcap%
\pgfsetroundjoin%
\definecolor{currentfill}{rgb}{0.000000,0.000000,0.000000}%
\pgfsetfillcolor{currentfill}%
\pgfsetlinewidth{0.803000pt}%
\definecolor{currentstroke}{rgb}{0.000000,0.000000,0.000000}%
\pgfsetstrokecolor{currentstroke}%
\pgfsetdash{}{0pt}%
\pgfsys@defobject{currentmarker}{\pgfqpoint{-0.048611in}{0.000000in}}{\pgfqpoint{0.000000in}{0.000000in}}{%
\pgfpathmoveto{\pgfqpoint{0.000000in}{0.000000in}}%
\pgfpathlineto{\pgfqpoint{-0.048611in}{0.000000in}}%
\pgfusepath{stroke,fill}%
}%
\begin{pgfscope}%
\pgfsys@transformshift{0.679669in}{1.854879in}%
\pgfsys@useobject{currentmarker}{}%
\end{pgfscope}%
\end{pgfscope}%
\begin{pgfscope}%
\pgftext[x=0.294444in,y=1.802118in,left,base]{\rmfamily\fontsize{10.000000}{12.000000}\selectfont \(\displaystyle 10^{-3}\)}%
\end{pgfscope}%
\begin{pgfscope}%
\pgftext[x=0.238889in,y=1.356579in,,bottom,rotate=90.000000]{\rmfamily\fontsize{10.000000}{12.000000}\selectfont \(\displaystyle \mathcal{E} / \mathcal{E}(0)\)}%
\end{pgfscope}%
\begin{pgfscope}%
\pgfpathrectangle{\pgfqpoint{0.679669in}{0.526079in}}{\pgfqpoint{2.358696in}{1.661000in}} %
\pgfusepath{clip}%
\pgfsetrectcap%
\pgfsetroundjoin%
\pgfsetlinewidth{1.003750pt}%
\definecolor{currentstroke}{rgb}{1.000000,0.549020,0.000000}%
\pgfsetstrokecolor{currentstroke}%
\pgfsetdash{}{0pt}%
\pgfpathmoveto{\pgfqpoint{0.680577in}{0.512191in}}%
\pgfpathlineto{\pgfqpoint{0.683207in}{0.929254in}}%
\pgfpathlineto{\pgfqpoint{0.684681in}{0.952506in}}%
\pgfpathlineto{\pgfqpoint{0.685271in}{0.949203in}}%
\pgfpathlineto{\pgfqpoint{0.687188in}{0.914037in}}%
\pgfpathlineto{\pgfqpoint{0.688809in}{0.923936in}}%
\pgfpathlineto{\pgfqpoint{0.691905in}{0.983835in}}%
\pgfpathlineto{\pgfqpoint{0.692347in}{0.980808in}}%
\pgfpathlineto{\pgfqpoint{0.693969in}{0.943674in}}%
\pgfpathlineto{\pgfqpoint{0.695443in}{0.946134in}}%
\pgfpathlineto{\pgfqpoint{0.696327in}{0.953098in}}%
\pgfpathlineto{\pgfqpoint{0.696917in}{0.948433in}}%
\pgfpathlineto{\pgfqpoint{0.698981in}{0.893082in}}%
\pgfpathlineto{\pgfqpoint{0.699718in}{0.861271in}}%
\pgfpathlineto{\pgfqpoint{0.700455in}{0.883450in}}%
\pgfpathlineto{\pgfqpoint{0.702814in}{0.986974in}}%
\pgfpathlineto{\pgfqpoint{0.703698in}{0.976875in}}%
\pgfpathlineto{\pgfqpoint{0.703846in}{0.976215in}}%
\pgfpathlineto{\pgfqpoint{0.704288in}{0.979432in}}%
\pgfpathlineto{\pgfqpoint{0.704878in}{0.985455in}}%
\pgfpathlineto{\pgfqpoint{0.705467in}{0.978073in}}%
\pgfpathlineto{\pgfqpoint{0.708711in}{0.881275in}}%
\pgfpathlineto{\pgfqpoint{0.710185in}{0.893208in}}%
\pgfpathlineto{\pgfqpoint{0.711069in}{0.939816in}}%
\pgfpathlineto{\pgfqpoint{0.712543in}{1.011727in}}%
\pgfpathlineto{\pgfqpoint{0.713281in}{1.004701in}}%
\pgfpathlineto{\pgfqpoint{0.713723in}{1.001671in}}%
\pgfpathlineto{\pgfqpoint{0.714313in}{1.009767in}}%
\pgfpathlineto{\pgfqpoint{0.718735in}{1.089790in}}%
\pgfpathlineto{\pgfqpoint{0.718882in}{1.089489in}}%
\pgfpathlineto{\pgfqpoint{0.719620in}{1.069721in}}%
\pgfpathlineto{\pgfqpoint{0.721241in}{0.975678in}}%
\pgfpathlineto{\pgfqpoint{0.722126in}{1.015569in}}%
\pgfpathlineto{\pgfqpoint{0.724484in}{1.062604in}}%
\pgfpathlineto{\pgfqpoint{0.724927in}{1.066927in}}%
\pgfpathlineto{\pgfqpoint{0.725369in}{1.058248in}}%
\pgfpathlineto{\pgfqpoint{0.726990in}{0.959862in}}%
\pgfpathlineto{\pgfqpoint{0.728022in}{1.003785in}}%
\pgfpathlineto{\pgfqpoint{0.729644in}{1.085549in}}%
\pgfpathlineto{\pgfqpoint{0.730381in}{1.072367in}}%
\pgfpathlineto{\pgfqpoint{0.732887in}{0.963013in}}%
\pgfpathlineto{\pgfqpoint{0.733624in}{0.904040in}}%
\pgfpathlineto{\pgfqpoint{0.734361in}{0.969196in}}%
\pgfpathlineto{\pgfqpoint{0.735983in}{1.080998in}}%
\pgfpathlineto{\pgfqpoint{0.736573in}{1.073887in}}%
\pgfpathlineto{\pgfqpoint{0.739963in}{1.004321in}}%
\pgfpathlineto{\pgfqpoint{0.740406in}{1.011488in}}%
\pgfpathlineto{\pgfqpoint{0.742617in}{1.121651in}}%
\pgfpathlineto{\pgfqpoint{0.743501in}{1.096019in}}%
\pgfpathlineto{\pgfqpoint{0.746450in}{1.013455in}}%
\pgfpathlineto{\pgfqpoint{0.746745in}{1.020615in}}%
\pgfpathlineto{\pgfqpoint{0.748808in}{1.141561in}}%
\pgfpathlineto{\pgfqpoint{0.749693in}{1.110895in}}%
\pgfpathlineto{\pgfqpoint{0.751020in}{1.025512in}}%
\pgfpathlineto{\pgfqpoint{0.751904in}{1.062379in}}%
\pgfpathlineto{\pgfqpoint{0.755295in}{1.145023in}}%
\pgfpathlineto{\pgfqpoint{0.755442in}{1.145908in}}%
\pgfpathlineto{\pgfqpoint{0.755737in}{1.142575in}}%
\pgfpathlineto{\pgfqpoint{0.756769in}{1.070917in}}%
\pgfpathlineto{\pgfqpoint{0.757654in}{0.991620in}}%
\pgfpathlineto{\pgfqpoint{0.758391in}{1.035589in}}%
\pgfpathlineto{\pgfqpoint{0.760749in}{1.111515in}}%
\pgfpathlineto{\pgfqpoint{0.762076in}{1.122629in}}%
\pgfpathlineto{\pgfqpoint{0.762666in}{1.113368in}}%
\pgfpathlineto{\pgfqpoint{0.764140in}{1.028215in}}%
\pgfpathlineto{\pgfqpoint{0.764877in}{1.069856in}}%
\pgfpathlineto{\pgfqpoint{0.766351in}{1.146132in}}%
\pgfpathlineto{\pgfqpoint{0.767088in}{1.134249in}}%
\pgfpathlineto{\pgfqpoint{0.768120in}{1.117283in}}%
\pgfpathlineto{\pgfqpoint{0.769005in}{1.122491in}}%
\pgfpathlineto{\pgfqpoint{0.772543in}{1.153466in}}%
\pgfpathlineto{\pgfqpoint{0.773132in}{1.146687in}}%
\pgfpathlineto{\pgfqpoint{0.774607in}{1.093902in}}%
\pgfpathlineto{\pgfqpoint{0.775491in}{1.117695in}}%
\pgfpathlineto{\pgfqpoint{0.778145in}{1.162978in}}%
\pgfpathlineto{\pgfqpoint{0.778734in}{1.153391in}}%
\pgfpathlineto{\pgfqpoint{0.780946in}{1.096168in}}%
\pgfpathlineto{\pgfqpoint{0.781535in}{1.103562in}}%
\pgfpathlineto{\pgfqpoint{0.783304in}{1.178478in}}%
\pgfpathlineto{\pgfqpoint{0.784336in}{1.155995in}}%
\pgfpathlineto{\pgfqpoint{0.786400in}{1.075291in}}%
\pgfpathlineto{\pgfqpoint{0.787285in}{1.097715in}}%
\pgfpathlineto{\pgfqpoint{0.789349in}{1.164010in}}%
\pgfpathlineto{\pgfqpoint{0.789938in}{1.153101in}}%
\pgfpathlineto{\pgfqpoint{0.791707in}{1.086834in}}%
\pgfpathlineto{\pgfqpoint{0.792592in}{1.105187in}}%
\pgfpathlineto{\pgfqpoint{0.795540in}{1.175687in}}%
\pgfpathlineto{\pgfqpoint{0.795688in}{1.174927in}}%
\pgfpathlineto{\pgfqpoint{0.796572in}{1.148346in}}%
\pgfpathlineto{\pgfqpoint{0.798046in}{1.102572in}}%
\pgfpathlineto{\pgfqpoint{0.798783in}{1.113692in}}%
\pgfpathlineto{\pgfqpoint{0.800847in}{1.175153in}}%
\pgfpathlineto{\pgfqpoint{0.801732in}{1.164758in}}%
\pgfpathlineto{\pgfqpoint{0.803648in}{1.116948in}}%
\pgfpathlineto{\pgfqpoint{0.804533in}{1.136531in}}%
\pgfpathlineto{\pgfqpoint{0.806596in}{1.189606in}}%
\pgfpathlineto{\pgfqpoint{0.807334in}{1.177204in}}%
\pgfpathlineto{\pgfqpoint{0.809987in}{1.134222in}}%
\pgfpathlineto{\pgfqpoint{0.810135in}{1.135007in}}%
\pgfpathlineto{\pgfqpoint{0.811461in}{1.173359in}}%
\pgfpathlineto{\pgfqpoint{0.813525in}{1.202552in}}%
\pgfpathlineto{\pgfqpoint{0.813967in}{1.201622in}}%
\pgfpathlineto{\pgfqpoint{0.815147in}{1.188886in}}%
\pgfpathlineto{\pgfqpoint{0.816326in}{1.167976in}}%
\pgfpathlineto{\pgfqpoint{0.817211in}{1.176262in}}%
\pgfpathlineto{\pgfqpoint{0.819569in}{1.189837in}}%
\pgfpathlineto{\pgfqpoint{0.819864in}{1.189351in}}%
\pgfpathlineto{\pgfqpoint{0.822223in}{1.168185in}}%
\pgfpathlineto{\pgfqpoint{0.822960in}{1.158681in}}%
\pgfpathlineto{\pgfqpoint{0.823550in}{1.168245in}}%
\pgfpathlineto{\pgfqpoint{0.824729in}{1.189218in}}%
\pgfpathlineto{\pgfqpoint{0.825466in}{1.180218in}}%
\pgfpathlineto{\pgfqpoint{0.826645in}{1.162867in}}%
\pgfpathlineto{\pgfqpoint{0.827382in}{1.170341in}}%
\pgfpathlineto{\pgfqpoint{0.828267in}{1.178495in}}%
\pgfpathlineto{\pgfqpoint{0.829004in}{1.172048in}}%
\pgfpathlineto{\pgfqpoint{0.830183in}{1.158683in}}%
\pgfpathlineto{\pgfqpoint{0.830773in}{1.166037in}}%
\pgfpathlineto{\pgfqpoint{0.831952in}{1.189554in}}%
\pgfpathlineto{\pgfqpoint{0.832542in}{1.178793in}}%
\pgfpathlineto{\pgfqpoint{0.833574in}{1.152691in}}%
\pgfpathlineto{\pgfqpoint{0.834311in}{1.163908in}}%
\pgfpathlineto{\pgfqpoint{0.835196in}{1.173629in}}%
\pgfpathlineto{\pgfqpoint{0.836228in}{1.172649in}}%
\pgfpathlineto{\pgfqpoint{0.838144in}{1.189579in}}%
\pgfpathlineto{\pgfqpoint{0.839029in}{1.180574in}}%
\pgfpathlineto{\pgfqpoint{0.840060in}{1.168956in}}%
\pgfpathlineto{\pgfqpoint{0.840798in}{1.175533in}}%
\pgfpathlineto{\pgfqpoint{0.842272in}{1.204328in}}%
\pgfpathlineto{\pgfqpoint{0.843156in}{1.193494in}}%
\pgfpathlineto{\pgfqpoint{0.845957in}{1.169857in}}%
\pgfpathlineto{\pgfqpoint{0.846105in}{1.170919in}}%
\pgfpathlineto{\pgfqpoint{0.848021in}{1.203755in}}%
\pgfpathlineto{\pgfqpoint{0.849053in}{1.195753in}}%
\pgfpathlineto{\pgfqpoint{0.852149in}{1.162080in}}%
\pgfpathlineto{\pgfqpoint{0.852886in}{1.165083in}}%
\pgfpathlineto{\pgfqpoint{0.854065in}{1.185657in}}%
\pgfpathlineto{\pgfqpoint{0.854950in}{1.199938in}}%
\pgfpathlineto{\pgfqpoint{0.855539in}{1.189972in}}%
\pgfpathlineto{\pgfqpoint{0.856129in}{1.180714in}}%
\pgfpathlineto{\pgfqpoint{0.856719in}{1.193710in}}%
\pgfpathlineto{\pgfqpoint{0.858930in}{1.223558in}}%
\pgfpathlineto{\pgfqpoint{0.860846in}{1.230230in}}%
\pgfpathlineto{\pgfqpoint{0.861289in}{1.227371in}}%
\pgfpathlineto{\pgfqpoint{0.862763in}{1.196540in}}%
\pgfpathlineto{\pgfqpoint{0.863500in}{1.212457in}}%
\pgfpathlineto{\pgfqpoint{0.865711in}{1.241244in}}%
\pgfpathlineto{\pgfqpoint{0.866006in}{1.240897in}}%
\pgfpathlineto{\pgfqpoint{0.866301in}{1.240800in}}%
\pgfpathlineto{\pgfqpoint{0.866596in}{1.242102in}}%
\pgfpathlineto{\pgfqpoint{0.867628in}{1.252478in}}%
\pgfpathlineto{\pgfqpoint{0.868070in}{1.246682in}}%
\pgfpathlineto{\pgfqpoint{0.869249in}{1.212701in}}%
\pgfpathlineto{\pgfqpoint{0.869986in}{1.234797in}}%
\pgfpathlineto{\pgfqpoint{0.871313in}{1.269145in}}%
\pgfpathlineto{\pgfqpoint{0.872050in}{1.258742in}}%
\pgfpathlineto{\pgfqpoint{0.873230in}{1.236354in}}%
\pgfpathlineto{\pgfqpoint{0.874114in}{1.244587in}}%
\pgfpathlineto{\pgfqpoint{0.874262in}{1.244771in}}%
\pgfpathlineto{\pgfqpoint{0.874409in}{1.244113in}}%
\pgfpathlineto{\pgfqpoint{0.875736in}{1.225592in}}%
\pgfpathlineto{\pgfqpoint{0.876473in}{1.236941in}}%
\pgfpathlineto{\pgfqpoint{0.878094in}{1.276363in}}%
\pgfpathlineto{\pgfqpoint{0.878832in}{1.265585in}}%
\pgfpathlineto{\pgfqpoint{0.881338in}{1.239175in}}%
\pgfpathlineto{\pgfqpoint{0.881927in}{1.230901in}}%
\pgfpathlineto{\pgfqpoint{0.882517in}{1.240567in}}%
\pgfpathlineto{\pgfqpoint{0.884581in}{1.294045in}}%
\pgfpathlineto{\pgfqpoint{0.885318in}{1.282603in}}%
\pgfpathlineto{\pgfqpoint{0.887087in}{1.243715in}}%
\pgfpathlineto{\pgfqpoint{0.887971in}{1.248065in}}%
\pgfpathlineto{\pgfqpoint{0.889151in}{1.269851in}}%
\pgfpathlineto{\pgfqpoint{0.890772in}{1.310152in}}%
\pgfpathlineto{\pgfqpoint{0.891362in}{1.298995in}}%
\pgfpathlineto{\pgfqpoint{0.893131in}{1.236591in}}%
\pgfpathlineto{\pgfqpoint{0.894163in}{1.249882in}}%
\pgfpathlineto{\pgfqpoint{0.896964in}{1.293511in}}%
\pgfpathlineto{\pgfqpoint{0.897406in}{1.290337in}}%
\pgfpathlineto{\pgfqpoint{0.899175in}{1.231461in}}%
\pgfpathlineto{\pgfqpoint{0.900207in}{1.258533in}}%
\pgfpathlineto{\pgfqpoint{0.902861in}{1.306758in}}%
\pgfpathlineto{\pgfqpoint{0.903008in}{1.306994in}}%
\pgfpathlineto{\pgfqpoint{0.903450in}{1.305111in}}%
\pgfpathlineto{\pgfqpoint{0.904630in}{1.274918in}}%
\pgfpathlineto{\pgfqpoint{0.905514in}{1.252612in}}%
\pgfpathlineto{\pgfqpoint{0.906251in}{1.267676in}}%
\pgfpathlineto{\pgfqpoint{0.908020in}{1.324464in}}%
\pgfpathlineto{\pgfqpoint{0.908905in}{1.315707in}}%
\pgfpathlineto{\pgfqpoint{0.911706in}{1.261730in}}%
\pgfpathlineto{\pgfqpoint{0.912590in}{1.279138in}}%
\pgfpathlineto{\pgfqpoint{0.914212in}{1.331338in}}%
\pgfpathlineto{\pgfqpoint{0.914949in}{1.320680in}}%
\pgfpathlineto{\pgfqpoint{0.917160in}{1.282892in}}%
\pgfpathlineto{\pgfqpoint{0.917750in}{1.283788in}}%
\pgfpathlineto{\pgfqpoint{0.918340in}{1.282585in}}%
\pgfpathlineto{\pgfqpoint{0.918635in}{1.281766in}}%
\pgfpathlineto{\pgfqpoint{0.919077in}{1.284272in}}%
\pgfpathlineto{\pgfqpoint{0.920698in}{1.319003in}}%
\pgfpathlineto{\pgfqpoint{0.921583in}{1.301449in}}%
\pgfpathlineto{\pgfqpoint{0.922762in}{1.280380in}}%
\pgfpathlineto{\pgfqpoint{0.923499in}{1.286590in}}%
\pgfpathlineto{\pgfqpoint{0.927185in}{1.346408in}}%
\pgfpathlineto{\pgfqpoint{0.927922in}{1.336865in}}%
\pgfpathlineto{\pgfqpoint{0.929396in}{1.295947in}}%
\pgfpathlineto{\pgfqpoint{0.930133in}{1.311962in}}%
\pgfpathlineto{\pgfqpoint{0.932639in}{1.349522in}}%
\pgfpathlineto{\pgfqpoint{0.933524in}{1.359435in}}%
\pgfpathlineto{\pgfqpoint{0.934113in}{1.353547in}}%
\pgfpathlineto{\pgfqpoint{0.936030in}{1.307942in}}%
\pgfpathlineto{\pgfqpoint{0.936914in}{1.323487in}}%
\pgfpathlineto{\pgfqpoint{0.938241in}{1.353408in}}%
\pgfpathlineto{\pgfqpoint{0.938978in}{1.346755in}}%
\pgfpathlineto{\pgfqpoint{0.942074in}{1.295483in}}%
\pgfpathlineto{\pgfqpoint{0.942811in}{1.310973in}}%
\pgfpathlineto{\pgfqpoint{0.944285in}{1.354887in}}%
\pgfpathlineto{\pgfqpoint{0.945022in}{1.346087in}}%
\pgfpathlineto{\pgfqpoint{0.947234in}{1.301727in}}%
\pgfpathlineto{\pgfqpoint{0.947971in}{1.311650in}}%
\pgfpathlineto{\pgfqpoint{0.950772in}{1.375798in}}%
\pgfpathlineto{\pgfqpoint{0.951509in}{1.359771in}}%
\pgfpathlineto{\pgfqpoint{0.952836in}{1.320783in}}%
\pgfpathlineto{\pgfqpoint{0.953573in}{1.330496in}}%
\pgfpathlineto{\pgfqpoint{0.956668in}{1.392654in}}%
\pgfpathlineto{\pgfqpoint{0.957553in}{1.380379in}}%
\pgfpathlineto{\pgfqpoint{0.959027in}{1.337980in}}%
\pgfpathlineto{\pgfqpoint{0.959764in}{1.351541in}}%
\pgfpathlineto{\pgfqpoint{0.961681in}{1.394339in}}%
\pgfpathlineto{\pgfqpoint{0.962270in}{1.391810in}}%
\pgfpathlineto{\pgfqpoint{0.963745in}{1.369058in}}%
\pgfpathlineto{\pgfqpoint{0.965219in}{1.316957in}}%
\pgfpathlineto{\pgfqpoint{0.965956in}{1.335687in}}%
\pgfpathlineto{\pgfqpoint{0.968020in}{1.396678in}}%
\pgfpathlineto{\pgfqpoint{0.968609in}{1.390736in}}%
\pgfpathlineto{\pgfqpoint{0.971705in}{1.335985in}}%
\pgfpathlineto{\pgfqpoint{0.972147in}{1.343204in}}%
\pgfpathlineto{\pgfqpoint{0.974064in}{1.402689in}}%
\pgfpathlineto{\pgfqpoint{0.974948in}{1.388778in}}%
\pgfpathlineto{\pgfqpoint{0.976865in}{1.348327in}}%
\pgfpathlineto{\pgfqpoint{0.977749in}{1.352332in}}%
\pgfpathlineto{\pgfqpoint{0.978929in}{1.378048in}}%
\pgfpathlineto{\pgfqpoint{0.980403in}{1.409504in}}%
\pgfpathlineto{\pgfqpoint{0.980993in}{1.401560in}}%
\pgfpathlineto{\pgfqpoint{0.982762in}{1.356273in}}%
\pgfpathlineto{\pgfqpoint{0.983646in}{1.372104in}}%
\pgfpathlineto{\pgfqpoint{0.986300in}{1.397469in}}%
\pgfpathlineto{\pgfqpoint{0.986447in}{1.397992in}}%
\pgfpathlineto{\pgfqpoint{0.986889in}{1.395055in}}%
\pgfpathlineto{\pgfqpoint{0.988953in}{1.337786in}}%
\pgfpathlineto{\pgfqpoint{0.989985in}{1.361702in}}%
\pgfpathlineto{\pgfqpoint{0.991312in}{1.387404in}}%
\pgfpathlineto{\pgfqpoint{0.992049in}{1.383282in}}%
\pgfpathlineto{\pgfqpoint{0.993818in}{1.361887in}}%
\pgfpathlineto{\pgfqpoint{0.995440in}{1.315888in}}%
\pgfpathlineto{\pgfqpoint{0.996029in}{1.331094in}}%
\pgfpathlineto{\pgfqpoint{0.997651in}{1.383725in}}%
\pgfpathlineto{\pgfqpoint{0.998388in}{1.375777in}}%
\pgfpathlineto{\pgfqpoint{1.001631in}{1.344508in}}%
\pgfpathlineto{\pgfqpoint{1.001779in}{1.344895in}}%
\pgfpathlineto{\pgfqpoint{1.002516in}{1.359873in}}%
\pgfpathlineto{\pgfqpoint{1.004137in}{1.403467in}}%
\pgfpathlineto{\pgfqpoint{1.004874in}{1.391817in}}%
\pgfpathlineto{\pgfqpoint{1.006054in}{1.373911in}}%
\pgfpathlineto{\pgfqpoint{1.006791in}{1.378341in}}%
\pgfpathlineto{\pgfqpoint{1.010476in}{1.408676in}}%
\pgfpathlineto{\pgfqpoint{1.011066in}{1.404750in}}%
\pgfpathlineto{\pgfqpoint{1.012982in}{1.369114in}}%
\pgfpathlineto{\pgfqpoint{1.014014in}{1.378182in}}%
\pgfpathlineto{\pgfqpoint{1.014309in}{1.378706in}}%
\pgfpathlineto{\pgfqpoint{1.014751in}{1.376528in}}%
\pgfpathlineto{\pgfqpoint{1.015341in}{1.373817in}}%
\pgfpathlineto{\pgfqpoint{1.015931in}{1.377923in}}%
\pgfpathlineto{\pgfqpoint{1.017110in}{1.388613in}}%
\pgfpathlineto{\pgfqpoint{1.017700in}{1.383121in}}%
\pgfpathlineto{\pgfqpoint{1.019321in}{1.355790in}}%
\pgfpathlineto{\pgfqpoint{1.020058in}{1.365906in}}%
\pgfpathlineto{\pgfqpoint{1.022712in}{1.390094in}}%
\pgfpathlineto{\pgfqpoint{1.023744in}{1.403137in}}%
\pgfpathlineto{\pgfqpoint{1.024481in}{1.395095in}}%
\pgfpathlineto{\pgfqpoint{1.025808in}{1.369004in}}%
\pgfpathlineto{\pgfqpoint{1.026545in}{1.379105in}}%
\pgfpathlineto{\pgfqpoint{1.028019in}{1.399940in}}%
\pgfpathlineto{\pgfqpoint{1.028756in}{1.396170in}}%
\pgfpathlineto{\pgfqpoint{1.029198in}{1.395042in}}%
\pgfpathlineto{\pgfqpoint{1.029641in}{1.397466in}}%
\pgfpathlineto{\pgfqpoint{1.030673in}{1.405594in}}%
\pgfpathlineto{\pgfqpoint{1.031262in}{1.401550in}}%
\pgfpathlineto{\pgfqpoint{1.032294in}{1.389512in}}%
\pgfpathlineto{\pgfqpoint{1.033031in}{1.396485in}}%
\pgfpathlineto{\pgfqpoint{1.034505in}{1.421591in}}%
\pgfpathlineto{\pgfqpoint{1.035243in}{1.413247in}}%
\pgfpathlineto{\pgfqpoint{1.036422in}{1.395881in}}%
\pgfpathlineto{\pgfqpoint{1.037306in}{1.399690in}}%
\pgfpathlineto{\pgfqpoint{1.037601in}{1.400219in}}%
\pgfpathlineto{\pgfqpoint{1.038191in}{1.398184in}}%
\pgfpathlineto{\pgfqpoint{1.038781in}{1.396347in}}%
\pgfpathlineto{\pgfqpoint{1.039223in}{1.398701in}}%
\pgfpathlineto{\pgfqpoint{1.041139in}{1.426765in}}%
\pgfpathlineto{\pgfqpoint{1.042024in}{1.415355in}}%
\pgfpathlineto{\pgfqpoint{1.043351in}{1.398665in}}%
\pgfpathlineto{\pgfqpoint{1.044088in}{1.401054in}}%
\pgfpathlineto{\pgfqpoint{1.047036in}{1.422872in}}%
\pgfpathlineto{\pgfqpoint{1.047478in}{1.426674in}}%
\pgfpathlineto{\pgfqpoint{1.048215in}{1.418130in}}%
\pgfpathlineto{\pgfqpoint{1.049542in}{1.395985in}}%
\pgfpathlineto{\pgfqpoint{1.050279in}{1.401338in}}%
\pgfpathlineto{\pgfqpoint{1.053965in}{1.432281in}}%
\pgfpathlineto{\pgfqpoint{1.054112in}{1.432390in}}%
\pgfpathlineto{\pgfqpoint{1.054407in}{1.430915in}}%
\pgfpathlineto{\pgfqpoint{1.055881in}{1.408949in}}%
\pgfpathlineto{\pgfqpoint{1.056766in}{1.419860in}}%
\pgfpathlineto{\pgfqpoint{1.058535in}{1.445220in}}%
\pgfpathlineto{\pgfqpoint{1.059272in}{1.441033in}}%
\pgfpathlineto{\pgfqpoint{1.062515in}{1.420945in}}%
\pgfpathlineto{\pgfqpoint{1.062957in}{1.423204in}}%
\pgfpathlineto{\pgfqpoint{1.064874in}{1.452891in}}%
\pgfpathlineto{\pgfqpoint{1.065906in}{1.440313in}}%
\pgfpathlineto{\pgfqpoint{1.066495in}{1.434708in}}%
\pgfpathlineto{\pgfqpoint{1.067380in}{1.440203in}}%
\pgfpathlineto{\pgfqpoint{1.067969in}{1.442126in}}%
\pgfpathlineto{\pgfqpoint{1.069001in}{1.440687in}}%
\pgfpathlineto{\pgfqpoint{1.070033in}{1.454214in}}%
\pgfpathlineto{\pgfqpoint{1.071213in}{1.469026in}}%
\pgfpathlineto{\pgfqpoint{1.071950in}{1.461458in}}%
\pgfpathlineto{\pgfqpoint{1.073424in}{1.433088in}}%
\pgfpathlineto{\pgfqpoint{1.074161in}{1.442852in}}%
\pgfpathlineto{\pgfqpoint{1.076667in}{1.463716in}}%
\pgfpathlineto{\pgfqpoint{1.077699in}{1.470434in}}%
\pgfpathlineto{\pgfqpoint{1.078141in}{1.466814in}}%
\pgfpathlineto{\pgfqpoint{1.079763in}{1.431349in}}%
\pgfpathlineto{\pgfqpoint{1.080647in}{1.446594in}}%
\pgfpathlineto{\pgfqpoint{1.082122in}{1.468120in}}%
\pgfpathlineto{\pgfqpoint{1.082859in}{1.465048in}}%
\pgfpathlineto{\pgfqpoint{1.086397in}{1.436486in}}%
\pgfpathlineto{\pgfqpoint{1.086986in}{1.446153in}}%
\pgfpathlineto{\pgfqpoint{1.088608in}{1.477984in}}%
\pgfpathlineto{\pgfqpoint{1.089345in}{1.474615in}}%
\pgfpathlineto{\pgfqpoint{1.092441in}{1.457930in}}%
\pgfpathlineto{\pgfqpoint{1.092883in}{1.460300in}}%
\pgfpathlineto{\pgfqpoint{1.094947in}{1.501194in}}%
\pgfpathlineto{\pgfqpoint{1.095979in}{1.487008in}}%
\pgfpathlineto{\pgfqpoint{1.097306in}{1.465965in}}%
\pgfpathlineto{\pgfqpoint{1.098043in}{1.471364in}}%
\pgfpathlineto{\pgfqpoint{1.100991in}{1.502635in}}%
\pgfpathlineto{\pgfqpoint{1.101876in}{1.491572in}}%
\pgfpathlineto{\pgfqpoint{1.103645in}{1.450073in}}%
\pgfpathlineto{\pgfqpoint{1.104529in}{1.463283in}}%
\pgfpathlineto{\pgfqpoint{1.107035in}{1.488832in}}%
\pgfpathlineto{\pgfqpoint{1.107330in}{1.489085in}}%
\pgfpathlineto{\pgfqpoint{1.107625in}{1.487774in}}%
\pgfpathlineto{\pgfqpoint{1.108804in}{1.464511in}}%
\pgfpathlineto{\pgfqpoint{1.109836in}{1.445615in}}%
\pgfpathlineto{\pgfqpoint{1.110573in}{1.455739in}}%
\pgfpathlineto{\pgfqpoint{1.112932in}{1.491513in}}%
\pgfpathlineto{\pgfqpoint{1.113374in}{1.490620in}}%
\pgfpathlineto{\pgfqpoint{1.114701in}{1.477978in}}%
\pgfpathlineto{\pgfqpoint{1.116028in}{1.465077in}}%
\pgfpathlineto{\pgfqpoint{1.116618in}{1.470555in}}%
\pgfpathlineto{\pgfqpoint{1.118681in}{1.509371in}}%
\pgfpathlineto{\pgfqpoint{1.119566in}{1.502860in}}%
\pgfpathlineto{\pgfqpoint{1.122367in}{1.490062in}}%
\pgfpathlineto{\pgfqpoint{1.122514in}{1.490385in}}%
\pgfpathlineto{\pgfqpoint{1.123546in}{1.502657in}}%
\pgfpathlineto{\pgfqpoint{1.124578in}{1.513766in}}%
\pgfpathlineto{\pgfqpoint{1.125315in}{1.508885in}}%
\pgfpathlineto{\pgfqpoint{1.127084in}{1.488661in}}%
\pgfpathlineto{\pgfqpoint{1.127969in}{1.490785in}}%
\pgfpathlineto{\pgfqpoint{1.130327in}{1.500100in}}%
\pgfpathlineto{\pgfqpoint{1.131212in}{1.504443in}}%
\pgfpathlineto{\pgfqpoint{1.131802in}{1.500228in}}%
\pgfpathlineto{\pgfqpoint{1.133276in}{1.481565in}}%
\pgfpathlineto{\pgfqpoint{1.134308in}{1.485595in}}%
\pgfpathlineto{\pgfqpoint{1.137551in}{1.503379in}}%
\pgfpathlineto{\pgfqpoint{1.137993in}{1.504575in}}%
\pgfpathlineto{\pgfqpoint{1.138435in}{1.502092in}}%
\pgfpathlineto{\pgfqpoint{1.139467in}{1.491764in}}%
\pgfpathlineto{\pgfqpoint{1.140204in}{1.497361in}}%
\pgfpathlineto{\pgfqpoint{1.143890in}{1.529446in}}%
\pgfpathlineto{\pgfqpoint{1.144332in}{1.530648in}}%
\pgfpathlineto{\pgfqpoint{1.144922in}{1.527559in}}%
\pgfpathlineto{\pgfqpoint{1.146249in}{1.515401in}}%
\pgfpathlineto{\pgfqpoint{1.146986in}{1.519371in}}%
\pgfpathlineto{\pgfqpoint{1.148902in}{1.537920in}}%
\pgfpathlineto{\pgfqpoint{1.149787in}{1.533774in}}%
\pgfpathlineto{\pgfqpoint{1.152882in}{1.516970in}}%
\pgfpathlineto{\pgfqpoint{1.153767in}{1.523576in}}%
\pgfpathlineto{\pgfqpoint{1.155094in}{1.533837in}}%
\pgfpathlineto{\pgfqpoint{1.155683in}{1.530288in}}%
\pgfpathlineto{\pgfqpoint{1.157895in}{1.510342in}}%
\pgfpathlineto{\pgfqpoint{1.158779in}{1.511572in}}%
\pgfpathlineto{\pgfqpoint{1.159959in}{1.518144in}}%
\pgfpathlineto{\pgfqpoint{1.161580in}{1.533052in}}%
\pgfpathlineto{\pgfqpoint{1.162317in}{1.528694in}}%
\pgfpathlineto{\pgfqpoint{1.163791in}{1.518841in}}%
\pgfpathlineto{\pgfqpoint{1.164529in}{1.521845in}}%
\pgfpathlineto{\pgfqpoint{1.168067in}{1.547161in}}%
\pgfpathlineto{\pgfqpoint{1.169246in}{1.542756in}}%
\pgfpathlineto{\pgfqpoint{1.170278in}{1.538387in}}%
\pgfpathlineto{\pgfqpoint{1.170868in}{1.541076in}}%
\pgfpathlineto{\pgfqpoint{1.173521in}{1.550757in}}%
\pgfpathlineto{\pgfqpoint{1.174406in}{1.551257in}}%
\pgfpathlineto{\pgfqpoint{1.174848in}{1.550405in}}%
\pgfpathlineto{\pgfqpoint{1.176764in}{1.542883in}}%
\pgfpathlineto{\pgfqpoint{1.177501in}{1.546392in}}%
\pgfpathlineto{\pgfqpoint{1.178533in}{1.551009in}}%
\pgfpathlineto{\pgfqpoint{1.179270in}{1.548796in}}%
\pgfpathlineto{\pgfqpoint{1.182808in}{1.539854in}}%
\pgfpathlineto{\pgfqpoint{1.183251in}{1.539109in}}%
\pgfpathlineto{\pgfqpoint{1.183693in}{1.541399in}}%
\pgfpathlineto{\pgfqpoint{1.185020in}{1.552405in}}%
\pgfpathlineto{\pgfqpoint{1.185904in}{1.549219in}}%
\pgfpathlineto{\pgfqpoint{1.186936in}{1.546111in}}%
\pgfpathlineto{\pgfqpoint{1.187673in}{1.547824in}}%
\pgfpathlineto{\pgfqpoint{1.191948in}{1.565983in}}%
\pgfpathlineto{\pgfqpoint{1.192538in}{1.564537in}}%
\pgfpathlineto{\pgfqpoint{1.193570in}{1.560944in}}%
\pgfpathlineto{\pgfqpoint{1.194160in}{1.563392in}}%
\pgfpathlineto{\pgfqpoint{1.195634in}{1.572328in}}%
\pgfpathlineto{\pgfqpoint{1.196666in}{1.571408in}}%
\pgfpathlineto{\pgfqpoint{1.198582in}{1.575168in}}%
\pgfpathlineto{\pgfqpoint{1.199467in}{1.573029in}}%
\pgfpathlineto{\pgfqpoint{1.200056in}{1.571808in}}%
\pgfpathlineto{\pgfqpoint{1.200646in}{1.573310in}}%
\pgfpathlineto{\pgfqpoint{1.202120in}{1.581663in}}%
\pgfpathlineto{\pgfqpoint{1.203005in}{1.578278in}}%
\pgfpathlineto{\pgfqpoint{1.203742in}{1.576539in}}%
\pgfpathlineto{\pgfqpoint{1.204479in}{1.578463in}}%
\pgfpathlineto{\pgfqpoint{1.205069in}{1.579451in}}%
\pgfpathlineto{\pgfqpoint{1.205658in}{1.578108in}}%
\pgfpathlineto{\pgfqpoint{1.206690in}{1.575007in}}%
\pgfpathlineto{\pgfqpoint{1.207427in}{1.577070in}}%
\pgfpathlineto{\pgfqpoint{1.208459in}{1.580228in}}%
\pgfpathlineto{\pgfqpoint{1.209196in}{1.578982in}}%
\pgfpathlineto{\pgfqpoint{1.210081in}{1.577711in}}%
\pgfpathlineto{\pgfqpoint{1.210671in}{1.578897in}}%
\pgfpathlineto{\pgfqpoint{1.215388in}{1.591517in}}%
\pgfpathlineto{\pgfqpoint{1.215535in}{1.591409in}}%
\pgfpathlineto{\pgfqpoint{1.216715in}{1.589068in}}%
\pgfpathlineto{\pgfqpoint{1.217304in}{1.590603in}}%
\pgfpathlineto{\pgfqpoint{1.219958in}{1.595560in}}%
\pgfpathlineto{\pgfqpoint{1.222169in}{1.600980in}}%
\pgfpathlineto{\pgfqpoint{1.223054in}{1.599079in}}%
\pgfpathlineto{\pgfqpoint{1.223938in}{1.597687in}}%
\pgfpathlineto{\pgfqpoint{1.224528in}{1.598848in}}%
\pgfpathlineto{\pgfqpoint{1.228213in}{1.606757in}}%
\pgfpathlineto{\pgfqpoint{1.228656in}{1.606158in}}%
\pgfpathlineto{\pgfqpoint{1.230425in}{1.601348in}}%
\pgfpathlineto{\pgfqpoint{1.231162in}{1.603054in}}%
\pgfpathlineto{\pgfqpoint{1.235289in}{1.613576in}}%
\pgfpathlineto{\pgfqpoint{1.236616in}{1.611098in}}%
\pgfpathlineto{\pgfqpoint{1.237353in}{1.613322in}}%
\pgfpathlineto{\pgfqpoint{1.238680in}{1.617213in}}%
\pgfpathlineto{\pgfqpoint{1.239270in}{1.615857in}}%
\pgfpathlineto{\pgfqpoint{1.240596in}{1.611867in}}%
\pgfpathlineto{\pgfqpoint{1.241334in}{1.613702in}}%
\pgfpathlineto{\pgfqpoint{1.244577in}{1.621133in}}%
\pgfpathlineto{\pgfqpoint{1.245166in}{1.622253in}}%
\pgfpathlineto{\pgfqpoint{1.245756in}{1.620765in}}%
\pgfpathlineto{\pgfqpoint{1.246788in}{1.617236in}}%
\pgfpathlineto{\pgfqpoint{1.247525in}{1.619432in}}%
\pgfpathlineto{\pgfqpoint{1.249736in}{1.625682in}}%
\pgfpathlineto{\pgfqpoint{1.250179in}{1.625491in}}%
\pgfpathlineto{\pgfqpoint{1.253422in}{1.624635in}}%
\pgfpathlineto{\pgfqpoint{1.253569in}{1.625007in}}%
\pgfpathlineto{\pgfqpoint{1.259024in}{1.638948in}}%
\pgfpathlineto{\pgfqpoint{1.260351in}{1.640389in}}%
\pgfpathlineto{\pgfqpoint{1.262414in}{1.645151in}}%
\pgfpathlineto{\pgfqpoint{1.263004in}{1.644415in}}%
\pgfpathlineto{\pgfqpoint{1.264331in}{1.642520in}}%
\pgfpathlineto{\pgfqpoint{1.265068in}{1.643489in}}%
\pgfpathlineto{\pgfqpoint{1.268606in}{1.647795in}}%
\pgfpathlineto{\pgfqpoint{1.269048in}{1.646845in}}%
\pgfpathlineto{\pgfqpoint{1.270375in}{1.643887in}}%
\pgfpathlineto{\pgfqpoint{1.271112in}{1.644911in}}%
\pgfpathlineto{\pgfqpoint{1.275535in}{1.652665in}}%
\pgfpathlineto{\pgfqpoint{1.275829in}{1.652388in}}%
\pgfpathlineto{\pgfqpoint{1.276861in}{1.650458in}}%
\pgfpathlineto{\pgfqpoint{1.277451in}{1.651816in}}%
\pgfpathlineto{\pgfqpoint{1.281579in}{1.663741in}}%
\pgfpathlineto{\pgfqpoint{1.282316in}{1.664008in}}%
\pgfpathlineto{\pgfqpoint{1.282906in}{1.663177in}}%
\pgfpathlineto{\pgfqpoint{1.283643in}{1.662427in}}%
\pgfpathlineto{\pgfqpoint{1.284232in}{1.663723in}}%
\pgfpathlineto{\pgfqpoint{1.286001in}{1.670861in}}%
\pgfpathlineto{\pgfqpoint{1.287181in}{1.670127in}}%
\pgfpathlineto{\pgfqpoint{1.288507in}{1.671699in}}%
\pgfpathlineto{\pgfqpoint{1.289097in}{1.670466in}}%
\pgfpathlineto{\pgfqpoint{1.290277in}{1.667850in}}%
\pgfpathlineto{\pgfqpoint{1.291014in}{1.669270in}}%
\pgfpathlineto{\pgfqpoint{1.292488in}{1.671181in}}%
\pgfpathlineto{\pgfqpoint{1.293077in}{1.670900in}}%
\pgfpathlineto{\pgfqpoint{1.295584in}{1.667912in}}%
\pgfpathlineto{\pgfqpoint{1.296910in}{1.664506in}}%
\pgfpathlineto{\pgfqpoint{1.297647in}{1.666632in}}%
\pgfpathlineto{\pgfqpoint{1.299416in}{1.672824in}}%
\pgfpathlineto{\pgfqpoint{1.300154in}{1.672354in}}%
\pgfpathlineto{\pgfqpoint{1.302954in}{1.673566in}}%
\pgfpathlineto{\pgfqpoint{1.303986in}{1.677342in}}%
\pgfpathlineto{\pgfqpoint{1.306345in}{1.688013in}}%
\pgfpathlineto{\pgfqpoint{1.307230in}{1.687702in}}%
\pgfpathlineto{\pgfqpoint{1.308556in}{1.690191in}}%
\pgfpathlineto{\pgfqpoint{1.311652in}{1.698586in}}%
\pgfpathlineto{\pgfqpoint{1.312537in}{1.699345in}}%
\pgfpathlineto{\pgfqpoint{1.313274in}{1.698585in}}%
\pgfpathlineto{\pgfqpoint{1.315338in}{1.698705in}}%
\pgfpathlineto{\pgfqpoint{1.318728in}{1.701715in}}%
\pgfpathlineto{\pgfqpoint{1.319171in}{1.700788in}}%
\pgfpathlineto{\pgfqpoint{1.320792in}{1.697789in}}%
\pgfpathlineto{\pgfqpoint{1.321529in}{1.698288in}}%
\pgfpathlineto{\pgfqpoint{1.323593in}{1.701125in}}%
\pgfpathlineto{\pgfqpoint{1.324772in}{1.702208in}}%
\pgfpathlineto{\pgfqpoint{1.325510in}{1.701750in}}%
\pgfpathlineto{\pgfqpoint{1.326689in}{1.702929in}}%
\pgfpathlineto{\pgfqpoint{1.335092in}{1.722381in}}%
\pgfpathlineto{\pgfqpoint{1.336124in}{1.721493in}}%
\pgfpathlineto{\pgfqpoint{1.337303in}{1.722610in}}%
\pgfpathlineto{\pgfqpoint{1.340399in}{1.725475in}}%
\pgfpathlineto{\pgfqpoint{1.342020in}{1.724412in}}%
\pgfpathlineto{\pgfqpoint{1.343200in}{1.721359in}}%
\pgfpathlineto{\pgfqpoint{1.344084in}{1.722855in}}%
\pgfpathlineto{\pgfqpoint{1.344674in}{1.723272in}}%
\pgfpathlineto{\pgfqpoint{1.345411in}{1.722420in}}%
\pgfpathlineto{\pgfqpoint{1.347180in}{1.722684in}}%
\pgfpathlineto{\pgfqpoint{1.348802in}{1.721548in}}%
\pgfpathlineto{\pgfqpoint{1.349539in}{1.721511in}}%
\pgfpathlineto{\pgfqpoint{1.349981in}{1.722424in}}%
\pgfpathlineto{\pgfqpoint{1.354846in}{1.732447in}}%
\pgfpathlineto{\pgfqpoint{1.356173in}{1.733703in}}%
\pgfpathlineto{\pgfqpoint{1.359858in}{1.742992in}}%
\pgfpathlineto{\pgfqpoint{1.361332in}{1.745216in}}%
\pgfpathlineto{\pgfqpoint{1.364723in}{1.752235in}}%
\pgfpathlineto{\pgfqpoint{1.365607in}{1.750001in}}%
\pgfpathlineto{\pgfqpoint{1.366639in}{1.748350in}}%
\pgfpathlineto{\pgfqpoint{1.367524in}{1.748981in}}%
\pgfpathlineto{\pgfqpoint{1.370620in}{1.752269in}}%
\pgfpathlineto{\pgfqpoint{1.371357in}{1.752955in}}%
\pgfpathlineto{\pgfqpoint{1.371946in}{1.752038in}}%
\pgfpathlineto{\pgfqpoint{1.373126in}{1.749853in}}%
\pgfpathlineto{\pgfqpoint{1.373863in}{1.751024in}}%
\pgfpathlineto{\pgfqpoint{1.377843in}{1.757778in}}%
\pgfpathlineto{\pgfqpoint{1.377990in}{1.757696in}}%
\pgfpathlineto{\pgfqpoint{1.379170in}{1.757335in}}%
\pgfpathlineto{\pgfqpoint{1.379612in}{1.758057in}}%
\pgfpathlineto{\pgfqpoint{1.384624in}{1.768669in}}%
\pgfpathlineto{\pgfqpoint{1.385804in}{1.768078in}}%
\pgfpathlineto{\pgfqpoint{1.386983in}{1.770480in}}%
\pgfpathlineto{\pgfqpoint{1.388605in}{1.773514in}}%
\pgfpathlineto{\pgfqpoint{1.389342in}{1.773216in}}%
\pgfpathlineto{\pgfqpoint{1.393617in}{1.774876in}}%
\pgfpathlineto{\pgfqpoint{1.394796in}{1.777118in}}%
\pgfpathlineto{\pgfqpoint{1.395533in}{1.776109in}}%
\pgfpathlineto{\pgfqpoint{1.396713in}{1.775444in}}%
\pgfpathlineto{\pgfqpoint{1.397302in}{1.775830in}}%
\pgfpathlineto{\pgfqpoint{1.400840in}{1.779286in}}%
\pgfpathlineto{\pgfqpoint{1.401725in}{1.779663in}}%
\pgfpathlineto{\pgfqpoint{1.402462in}{1.779069in}}%
\pgfpathlineto{\pgfqpoint{1.403789in}{1.780074in}}%
\pgfpathlineto{\pgfqpoint{1.407916in}{1.785733in}}%
\pgfpathlineto{\pgfqpoint{1.408064in}{1.785649in}}%
\pgfpathlineto{\pgfqpoint{1.409538in}{1.784719in}}%
\pgfpathlineto{\pgfqpoint{1.410128in}{1.785753in}}%
\pgfpathlineto{\pgfqpoint{1.413076in}{1.789123in}}%
\pgfpathlineto{\pgfqpoint{1.417351in}{1.793373in}}%
\pgfpathlineto{\pgfqpoint{1.419268in}{1.793969in}}%
\pgfpathlineto{\pgfqpoint{1.423248in}{1.797222in}}%
\pgfpathlineto{\pgfqpoint{1.428260in}{1.802907in}}%
\pgfpathlineto{\pgfqpoint{1.430471in}{1.803638in}}%
\pgfpathlineto{\pgfqpoint{1.438874in}{1.808444in}}%
\pgfpathlineto{\pgfqpoint{1.440201in}{1.810372in}}%
\pgfpathlineto{\pgfqpoint{1.442412in}{1.811929in}}%
\pgfpathlineto{\pgfqpoint{1.445803in}{1.815523in}}%
\pgfpathlineto{\pgfqpoint{1.451405in}{1.820971in}}%
\pgfpathlineto{\pgfqpoint{1.457302in}{1.822587in}}%
\pgfpathlineto{\pgfqpoint{1.458186in}{1.823022in}}%
\pgfpathlineto{\pgfqpoint{1.458776in}{1.822374in}}%
\pgfpathlineto{\pgfqpoint{1.460692in}{1.822554in}}%
\pgfpathlineto{\pgfqpoint{1.463935in}{1.824488in}}%
\pgfpathlineto{\pgfqpoint{1.466884in}{1.826728in}}%
\pgfpathlineto{\pgfqpoint{1.478235in}{1.839363in}}%
\pgfpathlineto{\pgfqpoint{1.478382in}{1.839248in}}%
\pgfpathlineto{\pgfqpoint{1.480594in}{1.839229in}}%
\pgfpathlineto{\pgfqpoint{1.482363in}{1.838594in}}%
\pgfpathlineto{\pgfqpoint{1.486785in}{1.838947in}}%
\pgfpathlineto{\pgfqpoint{1.490029in}{1.841603in}}%
\pgfpathlineto{\pgfqpoint{1.494009in}{1.847643in}}%
\pgfpathlineto{\pgfqpoint{1.502117in}{1.856850in}}%
\pgfpathlineto{\pgfqpoint{1.504476in}{1.857161in}}%
\pgfpathlineto{\pgfqpoint{1.509046in}{1.856216in}}%
\pgfpathlineto{\pgfqpoint{1.510815in}{1.856563in}}%
\pgfpathlineto{\pgfqpoint{1.514205in}{1.858035in}}%
\pgfpathlineto{\pgfqpoint{1.517154in}{1.860775in}}%
\pgfpathlineto{\pgfqpoint{1.521576in}{1.866889in}}%
\pgfpathlineto{\pgfqpoint{1.531306in}{1.873909in}}%
\pgfpathlineto{\pgfqpoint{1.536613in}{1.872602in}}%
\pgfpathlineto{\pgfqpoint{1.541625in}{1.875461in}}%
\pgfpathlineto{\pgfqpoint{1.550765in}{1.884979in}}%
\pgfpathlineto{\pgfqpoint{1.553861in}{1.887210in}}%
\pgfpathlineto{\pgfqpoint{1.555482in}{1.887424in}}%
\pgfpathlineto{\pgfqpoint{1.555777in}{1.887162in}}%
\pgfpathlineto{\pgfqpoint{1.558873in}{1.886639in}}%
\pgfpathlineto{\pgfqpoint{1.566981in}{1.891161in}}%
\pgfpathlineto{\pgfqpoint{1.580396in}{1.901618in}}%
\pgfpathlineto{\pgfqpoint{1.583639in}{1.903871in}}%
\pgfpathlineto{\pgfqpoint{1.585556in}{1.904233in}}%
\pgfpathlineto{\pgfqpoint{1.588209in}{1.904814in}}%
\pgfpathlineto{\pgfqpoint{1.592779in}{1.907090in}}%
\pgfpathlineto{\pgfqpoint{1.594253in}{1.908639in}}%
\pgfpathlineto{\pgfqpoint{1.597202in}{1.910800in}}%
\pgfpathlineto{\pgfqpoint{1.606047in}{1.915149in}}%
\pgfpathlineto{\pgfqpoint{1.609732in}{1.916602in}}%
\pgfpathlineto{\pgfqpoint{1.611207in}{1.917431in}}%
\pgfpathlineto{\pgfqpoint{1.615629in}{1.919854in}}%
\pgfpathlineto{\pgfqpoint{1.617398in}{1.920094in}}%
\pgfpathlineto{\pgfqpoint{1.621231in}{1.922442in}}%
\pgfpathlineto{\pgfqpoint{1.630371in}{1.924791in}}%
\pgfpathlineto{\pgfqpoint{1.633909in}{1.926800in}}%
\pgfpathlineto{\pgfqpoint{1.636268in}{1.929207in}}%
\pgfpathlineto{\pgfqpoint{1.639511in}{1.930983in}}%
\pgfpathlineto{\pgfqpoint{1.639806in}{1.930644in}}%
\pgfpathlineto{\pgfqpoint{1.640985in}{1.930252in}}%
\pgfpathlineto{\pgfqpoint{1.641427in}{1.930791in}}%
\pgfpathlineto{\pgfqpoint{1.644228in}{1.932103in}}%
\pgfpathlineto{\pgfqpoint{1.653958in}{1.931340in}}%
\pgfpathlineto{\pgfqpoint{1.657349in}{1.933636in}}%
\pgfpathlineto{\pgfqpoint{1.658823in}{1.934700in}}%
\pgfpathlineto{\pgfqpoint{1.663540in}{1.938705in}}%
\pgfpathlineto{\pgfqpoint{1.665457in}{1.939408in}}%
\pgfpathlineto{\pgfqpoint{1.668700in}{1.940702in}}%
\pgfpathlineto{\pgfqpoint{1.670469in}{1.939037in}}%
\pgfpathlineto{\pgfqpoint{1.671943in}{1.939754in}}%
\pgfpathlineto{\pgfqpoint{1.674302in}{1.939883in}}%
\pgfpathlineto{\pgfqpoint{1.678282in}{1.939717in}}%
\pgfpathlineto{\pgfqpoint{1.680935in}{1.941059in}}%
\pgfpathlineto{\pgfqpoint{1.683294in}{1.942700in}}%
\pgfpathlineto{\pgfqpoint{1.689781in}{1.948759in}}%
\pgfpathlineto{\pgfqpoint{1.692582in}{1.949861in}}%
\pgfpathlineto{\pgfqpoint{1.692729in}{1.949673in}}%
\pgfpathlineto{\pgfqpoint{1.695677in}{1.947950in}}%
\pgfpathlineto{\pgfqpoint{1.699510in}{1.946458in}}%
\pgfpathlineto{\pgfqpoint{1.701279in}{1.945197in}}%
\pgfpathlineto{\pgfqpoint{1.701869in}{1.945710in}}%
\pgfpathlineto{\pgfqpoint{1.705112in}{1.946331in}}%
\pgfpathlineto{\pgfqpoint{1.707323in}{1.947129in}}%
\pgfpathlineto{\pgfqpoint{1.712630in}{1.951242in}}%
\pgfpathlineto{\pgfqpoint{1.714547in}{1.953162in}}%
\pgfpathlineto{\pgfqpoint{1.716906in}{1.953669in}}%
\pgfpathlineto{\pgfqpoint{1.719854in}{1.953263in}}%
\pgfpathlineto{\pgfqpoint{1.722802in}{1.952697in}}%
\pgfpathlineto{\pgfqpoint{1.725456in}{1.950823in}}%
\pgfpathlineto{\pgfqpoint{1.730026in}{1.951492in}}%
\pgfpathlineto{\pgfqpoint{1.731500in}{1.952392in}}%
\pgfpathlineto{\pgfqpoint{1.735185in}{1.956210in}}%
\pgfpathlineto{\pgfqpoint{1.738429in}{1.958680in}}%
\pgfpathlineto{\pgfqpoint{1.740787in}{1.959745in}}%
\pgfpathlineto{\pgfqpoint{1.743883in}{1.959999in}}%
\pgfpathlineto{\pgfqpoint{1.747274in}{1.959617in}}%
\pgfpathlineto{\pgfqpoint{1.750075in}{1.959099in}}%
\pgfpathlineto{\pgfqpoint{1.753760in}{1.957990in}}%
\pgfpathlineto{\pgfqpoint{1.755382in}{1.958667in}}%
\pgfpathlineto{\pgfqpoint{1.759067in}{1.960589in}}%
\pgfpathlineto{\pgfqpoint{1.761573in}{1.961259in}}%
\pgfpathlineto{\pgfqpoint{1.764374in}{1.963054in}}%
\pgfpathlineto{\pgfqpoint{1.768355in}{1.963581in}}%
\pgfpathlineto{\pgfqpoint{1.770124in}{1.963863in}}%
\pgfpathlineto{\pgfqpoint{1.770418in}{1.963497in}}%
\pgfpathlineto{\pgfqpoint{1.772925in}{1.962502in}}%
\pgfpathlineto{\pgfqpoint{1.777052in}{1.962513in}}%
\pgfpathlineto{\pgfqpoint{1.779264in}{1.962204in}}%
\pgfpathlineto{\pgfqpoint{1.783244in}{1.963467in}}%
\pgfpathlineto{\pgfqpoint{1.785308in}{1.963896in}}%
\pgfpathlineto{\pgfqpoint{1.788256in}{1.965431in}}%
\pgfpathlineto{\pgfqpoint{1.791647in}{1.965802in}}%
\pgfpathlineto{\pgfqpoint{1.794595in}{1.966591in}}%
\pgfpathlineto{\pgfqpoint{1.796954in}{1.966839in}}%
\pgfpathlineto{\pgfqpoint{1.801376in}{1.967872in}}%
\pgfpathlineto{\pgfqpoint{1.802998in}{1.968742in}}%
\pgfpathlineto{\pgfqpoint{1.806389in}{1.969675in}}%
\pgfpathlineto{\pgfqpoint{1.809337in}{1.969245in}}%
\pgfpathlineto{\pgfqpoint{1.811990in}{1.969385in}}%
\pgfpathlineto{\pgfqpoint{1.816266in}{1.969752in}}%
\pgfpathlineto{\pgfqpoint{1.818624in}{1.969093in}}%
\pgfpathlineto{\pgfqpoint{1.820983in}{1.969523in}}%
\pgfpathlineto{\pgfqpoint{1.830123in}{1.972614in}}%
\pgfpathlineto{\pgfqpoint{1.833219in}{1.972250in}}%
\pgfpathlineto{\pgfqpoint{1.835430in}{1.972254in}}%
\pgfpathlineto{\pgfqpoint{1.840000in}{1.970967in}}%
\pgfpathlineto{\pgfqpoint{1.841916in}{1.970344in}}%
\pgfpathlineto{\pgfqpoint{1.844570in}{1.970019in}}%
\pgfpathlineto{\pgfqpoint{1.849287in}{1.972200in}}%
\pgfpathlineto{\pgfqpoint{1.851499in}{1.974032in}}%
\pgfpathlineto{\pgfqpoint{1.854300in}{1.975013in}}%
\pgfpathlineto{\pgfqpoint{1.857101in}{1.974592in}}%
\pgfpathlineto{\pgfqpoint{1.860196in}{1.973841in}}%
\pgfpathlineto{\pgfqpoint{1.864619in}{1.971744in}}%
\pgfpathlineto{\pgfqpoint{1.870221in}{1.971560in}}%
\pgfpathlineto{\pgfqpoint{1.872432in}{1.971989in}}%
\pgfpathlineto{\pgfqpoint{1.875823in}{1.974394in}}%
\pgfpathlineto{\pgfqpoint{1.878181in}{1.975474in}}%
\pgfpathlineto{\pgfqpoint{1.882162in}{1.975507in}}%
\pgfpathlineto{\pgfqpoint{1.884373in}{1.974549in}}%
\pgfpathlineto{\pgfqpoint{1.887027in}{1.972883in}}%
\pgfpathlineto{\pgfqpoint{1.891744in}{1.970507in}}%
\pgfpathlineto{\pgfqpoint{1.893218in}{1.971082in}}%
\pgfpathlineto{\pgfqpoint{1.896314in}{1.972830in}}%
\pgfpathlineto{\pgfqpoint{1.899704in}{1.974871in}}%
\pgfpathlineto{\pgfqpoint{1.903243in}{1.976190in}}%
\pgfpathlineto{\pgfqpoint{1.907960in}{1.976711in}}%
\pgfpathlineto{\pgfqpoint{1.912382in}{1.974700in}}%
\pgfpathlineto{\pgfqpoint{1.918279in}{1.972481in}}%
\pgfpathlineto{\pgfqpoint{1.921080in}{1.972148in}}%
\pgfpathlineto{\pgfqpoint{1.923734in}{1.974202in}}%
\pgfpathlineto{\pgfqpoint{1.925650in}{1.974584in}}%
\pgfpathlineto{\pgfqpoint{1.932874in}{1.974775in}}%
\pgfpathlineto{\pgfqpoint{1.935675in}{1.974094in}}%
\pgfpathlineto{\pgfqpoint{1.937296in}{1.974853in}}%
\pgfpathlineto{\pgfqpoint{1.937886in}{1.974007in}}%
\pgfpathlineto{\pgfqpoint{1.940834in}{1.971899in}}%
\pgfpathlineto{\pgfqpoint{1.940982in}{1.972009in}}%
\pgfpathlineto{\pgfqpoint{1.944372in}{1.973362in}}%
\pgfpathlineto{\pgfqpoint{1.944520in}{1.973199in}}%
\pgfpathlineto{\pgfqpoint{1.945994in}{1.971683in}}%
\pgfpathlineto{\pgfqpoint{1.946584in}{1.972656in}}%
\pgfpathlineto{\pgfqpoint{1.949532in}{1.975322in}}%
\pgfpathlineto{\pgfqpoint{1.953660in}{1.976356in}}%
\pgfpathlineto{\pgfqpoint{1.954839in}{1.977776in}}%
\pgfpathlineto{\pgfqpoint{1.955576in}{1.976866in}}%
\pgfpathlineto{\pgfqpoint{1.958672in}{1.974778in}}%
\pgfpathlineto{\pgfqpoint{1.961915in}{1.976086in}}%
\pgfpathlineto{\pgfqpoint{1.962063in}{1.975879in}}%
\pgfpathlineto{\pgfqpoint{1.963684in}{1.973728in}}%
\pgfpathlineto{\pgfqpoint{1.964421in}{1.974520in}}%
\pgfpathlineto{\pgfqpoint{1.967517in}{1.975895in}}%
\pgfpathlineto{\pgfqpoint{1.967664in}{1.975713in}}%
\pgfpathlineto{\pgfqpoint{1.970760in}{1.973520in}}%
\pgfpathlineto{\pgfqpoint{1.974740in}{1.974495in}}%
\pgfpathlineto{\pgfqpoint{1.976215in}{1.972565in}}%
\pgfpathlineto{\pgfqpoint{1.976952in}{1.973587in}}%
\pgfpathlineto{\pgfqpoint{1.979605in}{1.975167in}}%
\pgfpathlineto{\pgfqpoint{1.989630in}{1.973940in}}%
\pgfpathlineto{\pgfqpoint{1.991988in}{1.974683in}}%
\pgfpathlineto{\pgfqpoint{1.995674in}{1.974181in}}%
\pgfpathlineto{\pgfqpoint{2.002603in}{1.973468in}}%
\pgfpathlineto{\pgfqpoint{2.005698in}{1.971368in}}%
\pgfpathlineto{\pgfqpoint{2.006583in}{1.970915in}}%
\pgfpathlineto{\pgfqpoint{2.007173in}{1.971906in}}%
\pgfpathlineto{\pgfqpoint{2.009236in}{1.972967in}}%
\pgfpathlineto{\pgfqpoint{2.021914in}{1.973801in}}%
\pgfpathlineto{\pgfqpoint{2.026927in}{1.971500in}}%
\pgfpathlineto{\pgfqpoint{2.028401in}{1.970569in}}%
\pgfpathlineto{\pgfqpoint{2.029875in}{1.968612in}}%
\pgfpathlineto{\pgfqpoint{2.030612in}{1.969168in}}%
\pgfpathlineto{\pgfqpoint{2.033855in}{1.969604in}}%
\pgfpathlineto{\pgfqpoint{2.036656in}{1.969005in}}%
\pgfpathlineto{\pgfqpoint{2.040489in}{1.970953in}}%
\pgfpathlineto{\pgfqpoint{2.046828in}{1.969743in}}%
\pgfpathlineto{\pgfqpoint{2.049777in}{1.969337in}}%
\pgfpathlineto{\pgfqpoint{2.051693in}{1.968606in}}%
\pgfpathlineto{\pgfqpoint{2.054789in}{1.965867in}}%
\pgfpathlineto{\pgfqpoint{2.058327in}{1.966432in}}%
\pgfpathlineto{\pgfqpoint{2.060096in}{1.965093in}}%
\pgfpathlineto{\pgfqpoint{2.060685in}{1.966069in}}%
\pgfpathlineto{\pgfqpoint{2.064076in}{1.968946in}}%
\pgfpathlineto{\pgfqpoint{2.066582in}{1.969525in}}%
\pgfpathlineto{\pgfqpoint{2.069088in}{1.971736in}}%
\pgfpathlineto{\pgfqpoint{2.069973in}{1.970621in}}%
\pgfpathlineto{\pgfqpoint{2.073216in}{1.970214in}}%
\pgfpathlineto{\pgfqpoint{2.074838in}{1.970295in}}%
\pgfpathlineto{\pgfqpoint{2.074985in}{1.970158in}}%
\pgfpathlineto{\pgfqpoint{2.080145in}{1.967127in}}%
\pgfpathlineto{\pgfqpoint{2.081471in}{1.967380in}}%
\pgfpathlineto{\pgfqpoint{2.081766in}{1.966866in}}%
\pgfpathlineto{\pgfqpoint{2.083683in}{1.964904in}}%
\pgfpathlineto{\pgfqpoint{2.084125in}{1.965157in}}%
\pgfpathlineto{\pgfqpoint{2.088990in}{1.966898in}}%
\pgfpathlineto{\pgfqpoint{2.090611in}{1.967510in}}%
\pgfpathlineto{\pgfqpoint{2.093265in}{1.969447in}}%
\pgfpathlineto{\pgfqpoint{2.094592in}{1.968416in}}%
\pgfpathlineto{\pgfqpoint{2.096656in}{1.967815in}}%
\pgfpathlineto{\pgfqpoint{2.099309in}{1.967712in}}%
\pgfpathlineto{\pgfqpoint{2.102700in}{1.964277in}}%
\pgfpathlineto{\pgfqpoint{2.105648in}{1.963659in}}%
\pgfpathlineto{\pgfqpoint{2.108007in}{1.961811in}}%
\pgfpathlineto{\pgfqpoint{2.108449in}{1.962381in}}%
\pgfpathlineto{\pgfqpoint{2.111397in}{1.964390in}}%
\pgfpathlineto{\pgfqpoint{2.113461in}{1.962520in}}%
\pgfpathlineto{\pgfqpoint{2.114346in}{1.963763in}}%
\pgfpathlineto{\pgfqpoint{2.117442in}{1.966455in}}%
\pgfpathlineto{\pgfqpoint{2.117884in}{1.966010in}}%
\pgfpathlineto{\pgfqpoint{2.120243in}{1.964314in}}%
\pgfpathlineto{\pgfqpoint{2.120685in}{1.964962in}}%
\pgfpathlineto{\pgfqpoint{2.122159in}{1.966882in}}%
\pgfpathlineto{\pgfqpoint{2.122896in}{1.966382in}}%
\pgfpathlineto{\pgfqpoint{2.127613in}{1.964946in}}%
\pgfpathlineto{\pgfqpoint{2.129235in}{1.964998in}}%
\pgfpathlineto{\pgfqpoint{2.129530in}{1.964481in}}%
\pgfpathlineto{\pgfqpoint{2.132331in}{1.961893in}}%
\pgfpathlineto{\pgfqpoint{2.133363in}{1.962899in}}%
\pgfpathlineto{\pgfqpoint{2.134837in}{1.964967in}}%
\pgfpathlineto{\pgfqpoint{2.135574in}{1.964336in}}%
\pgfpathlineto{\pgfqpoint{2.138080in}{1.962180in}}%
\pgfpathlineto{\pgfqpoint{2.138670in}{1.962716in}}%
\pgfpathlineto{\pgfqpoint{2.141766in}{1.965138in}}%
\pgfpathlineto{\pgfqpoint{2.142060in}{1.964796in}}%
\pgfpathlineto{\pgfqpoint{2.143535in}{1.962545in}}%
\pgfpathlineto{\pgfqpoint{2.144272in}{1.963620in}}%
\pgfpathlineto{\pgfqpoint{2.147073in}{1.965894in}}%
\pgfpathlineto{\pgfqpoint{2.147220in}{1.965789in}}%
\pgfpathlineto{\pgfqpoint{2.150463in}{1.962788in}}%
\pgfpathlineto{\pgfqpoint{2.151200in}{1.963879in}}%
\pgfpathlineto{\pgfqpoint{2.152527in}{1.964787in}}%
\pgfpathlineto{\pgfqpoint{2.153117in}{1.964210in}}%
\pgfpathlineto{\pgfqpoint{2.156360in}{1.961207in}}%
\pgfpathlineto{\pgfqpoint{2.156802in}{1.961860in}}%
\pgfpathlineto{\pgfqpoint{2.159161in}{1.962999in}}%
\pgfpathlineto{\pgfqpoint{2.160930in}{1.961062in}}%
\pgfpathlineto{\pgfqpoint{2.162994in}{1.961479in}}%
\pgfpathlineto{\pgfqpoint{2.165205in}{1.963195in}}%
\pgfpathlineto{\pgfqpoint{2.165942in}{1.961491in}}%
\pgfpathlineto{\pgfqpoint{2.168154in}{1.960283in}}%
\pgfpathlineto{\pgfqpoint{2.169333in}{1.961645in}}%
\pgfpathlineto{\pgfqpoint{2.170365in}{1.962363in}}%
\pgfpathlineto{\pgfqpoint{2.171102in}{1.961791in}}%
\pgfpathlineto{\pgfqpoint{2.174198in}{1.959946in}}%
\pgfpathlineto{\pgfqpoint{2.174787in}{1.960957in}}%
\pgfpathlineto{\pgfqpoint{2.177293in}{1.962109in}}%
\pgfpathlineto{\pgfqpoint{2.181863in}{1.961414in}}%
\pgfpathlineto{\pgfqpoint{2.182601in}{1.962014in}}%
\pgfpathlineto{\pgfqpoint{2.183338in}{1.961022in}}%
\pgfpathlineto{\pgfqpoint{2.186286in}{1.958433in}}%
\pgfpathlineto{\pgfqpoint{2.186581in}{1.958782in}}%
\pgfpathlineto{\pgfqpoint{2.188645in}{1.960310in}}%
\pgfpathlineto{\pgfqpoint{2.189087in}{1.959972in}}%
\pgfpathlineto{\pgfqpoint{2.194099in}{1.957681in}}%
\pgfpathlineto{\pgfqpoint{2.195131in}{1.958308in}}%
\pgfpathlineto{\pgfqpoint{2.195721in}{1.957302in}}%
\pgfpathlineto{\pgfqpoint{2.197048in}{1.955650in}}%
\pgfpathlineto{\pgfqpoint{2.197785in}{1.956040in}}%
\pgfpathlineto{\pgfqpoint{2.205156in}{1.958741in}}%
\pgfpathlineto{\pgfqpoint{2.207219in}{1.959007in}}%
\pgfpathlineto{\pgfqpoint{2.208546in}{1.957440in}}%
\pgfpathlineto{\pgfqpoint{2.209578in}{1.956680in}}%
\pgfpathlineto{\pgfqpoint{2.210168in}{1.957301in}}%
\pgfpathlineto{\pgfqpoint{2.212674in}{1.957894in}}%
\pgfpathlineto{\pgfqpoint{2.214148in}{1.955270in}}%
\pgfpathlineto{\pgfqpoint{2.216654in}{1.954196in}}%
\pgfpathlineto{\pgfqpoint{2.218276in}{1.955424in}}%
\pgfpathlineto{\pgfqpoint{2.219013in}{1.954378in}}%
\pgfpathlineto{\pgfqpoint{2.221814in}{1.951925in}}%
\pgfpathlineto{\pgfqpoint{2.222404in}{1.952686in}}%
\pgfpathlineto{\pgfqpoint{2.225057in}{1.954196in}}%
\pgfpathlineto{\pgfqpoint{2.233902in}{1.955729in}}%
\pgfpathlineto{\pgfqpoint{2.238914in}{1.954039in}}%
\pgfpathlineto{\pgfqpoint{2.240831in}{1.954094in}}%
\pgfpathlineto{\pgfqpoint{2.243779in}{1.952650in}}%
\pgfpathlineto{\pgfqpoint{2.245990in}{1.951467in}}%
\pgfpathlineto{\pgfqpoint{2.252919in}{1.954423in}}%
\pgfpathlineto{\pgfqpoint{2.255425in}{1.956026in}}%
\pgfpathlineto{\pgfqpoint{2.261764in}{1.956718in}}%
\pgfpathlineto{\pgfqpoint{2.264713in}{1.954724in}}%
\pgfpathlineto{\pgfqpoint{2.267956in}{1.952978in}}%
\pgfpathlineto{\pgfqpoint{2.275622in}{1.952079in}}%
\pgfpathlineto{\pgfqpoint{2.279012in}{1.954394in}}%
\pgfpathlineto{\pgfqpoint{2.281076in}{1.954495in}}%
\pgfpathlineto{\pgfqpoint{2.286383in}{1.955130in}}%
\pgfpathlineto{\pgfqpoint{2.288447in}{1.955011in}}%
\pgfpathlineto{\pgfqpoint{2.290363in}{1.954270in}}%
\pgfpathlineto{\pgfqpoint{2.300830in}{1.953404in}}%
\pgfpathlineto{\pgfqpoint{2.303484in}{1.954587in}}%
\pgfpathlineto{\pgfqpoint{2.309233in}{1.957042in}}%
\pgfpathlineto{\pgfqpoint{2.313213in}{1.956826in}}%
\pgfpathlineto{\pgfqpoint{2.315425in}{1.956665in}}%
\pgfpathlineto{\pgfqpoint{2.323238in}{1.954765in}}%
\pgfpathlineto{\pgfqpoint{2.327660in}{1.955507in}}%
\pgfpathlineto{\pgfqpoint{2.329872in}{1.955602in}}%
\pgfpathlineto{\pgfqpoint{2.331641in}{1.956431in}}%
\pgfpathlineto{\pgfqpoint{2.334294in}{1.956250in}}%
\pgfpathlineto{\pgfqpoint{2.337243in}{1.956491in}}%
\pgfpathlineto{\pgfqpoint{2.343434in}{1.956535in}}%
\pgfpathlineto{\pgfqpoint{2.346677in}{1.956734in}}%
\pgfpathlineto{\pgfqpoint{2.354933in}{1.957325in}}%
\pgfpathlineto{\pgfqpoint{2.375719in}{1.960550in}}%
\pgfpathlineto{\pgfqpoint{2.378962in}{1.960166in}}%
\pgfpathlineto{\pgfqpoint{2.379994in}{1.960810in}}%
\pgfpathlineto{\pgfqpoint{2.380584in}{1.959913in}}%
\pgfpathlineto{\pgfqpoint{2.382647in}{1.959272in}}%
\pgfpathlineto{\pgfqpoint{2.386480in}{1.959171in}}%
\pgfpathlineto{\pgfqpoint{2.388397in}{1.958676in}}%
\pgfpathlineto{\pgfqpoint{2.389429in}{1.959928in}}%
\pgfpathlineto{\pgfqpoint{2.391345in}{1.960408in}}%
\pgfpathlineto{\pgfqpoint{2.397832in}{1.963198in}}%
\pgfpathlineto{\pgfqpoint{2.402107in}{1.961210in}}%
\pgfpathlineto{\pgfqpoint{2.403728in}{1.960535in}}%
\pgfpathlineto{\pgfqpoint{2.407856in}{1.958484in}}%
\pgfpathlineto{\pgfqpoint{2.409920in}{1.958736in}}%
\pgfpathlineto{\pgfqpoint{2.411247in}{1.957752in}}%
\pgfpathlineto{\pgfqpoint{2.412868in}{1.958229in}}%
\pgfpathlineto{\pgfqpoint{2.415374in}{1.959584in}}%
\pgfpathlineto{\pgfqpoint{2.423630in}{1.962570in}}%
\pgfpathlineto{\pgfqpoint{2.425546in}{1.962344in}}%
\pgfpathlineto{\pgfqpoint{2.431001in}{1.959881in}}%
\pgfpathlineto{\pgfqpoint{2.431885in}{1.959301in}}%
\pgfpathlineto{\pgfqpoint{2.432622in}{1.960106in}}%
\pgfpathlineto{\pgfqpoint{2.434096in}{1.959433in}}%
\pgfpathlineto{\pgfqpoint{2.437340in}{1.959235in}}%
\pgfpathlineto{\pgfqpoint{2.441025in}{1.961719in}}%
\pgfpathlineto{\pgfqpoint{2.441320in}{1.961421in}}%
\pgfpathlineto{\pgfqpoint{2.442057in}{1.961489in}}%
\pgfpathlineto{\pgfqpoint{2.442499in}{1.962237in}}%
\pgfpathlineto{\pgfqpoint{2.445448in}{1.964408in}}%
\pgfpathlineto{\pgfqpoint{2.447806in}{1.964654in}}%
\pgfpathlineto{\pgfqpoint{2.453851in}{1.963320in}}%
\pgfpathlineto{\pgfqpoint{2.456209in}{1.961917in}}%
\pgfpathlineto{\pgfqpoint{2.462401in}{1.959439in}}%
\pgfpathlineto{\pgfqpoint{2.462548in}{1.959688in}}%
\pgfpathlineto{\pgfqpoint{2.463728in}{1.961109in}}%
\pgfpathlineto{\pgfqpoint{2.464465in}{1.960223in}}%
\pgfpathlineto{\pgfqpoint{2.466086in}{1.960683in}}%
\pgfpathlineto{\pgfqpoint{2.469919in}{1.962126in}}%
\pgfpathlineto{\pgfqpoint{2.471246in}{1.962510in}}%
\pgfpathlineto{\pgfqpoint{2.471836in}{1.961931in}}%
\pgfpathlineto{\pgfqpoint{2.473310in}{1.962692in}}%
\pgfpathlineto{\pgfqpoint{2.476406in}{1.963104in}}%
\pgfpathlineto{\pgfqpoint{2.480533in}{1.961323in}}%
\pgfpathlineto{\pgfqpoint{2.482007in}{1.960349in}}%
\pgfpathlineto{\pgfqpoint{2.488346in}{1.960025in}}%
\pgfpathlineto{\pgfqpoint{2.501172in}{1.962950in}}%
\pgfpathlineto{\pgfqpoint{2.505742in}{1.962429in}}%
\pgfpathlineto{\pgfqpoint{2.507953in}{1.962334in}}%
\pgfpathlineto{\pgfqpoint{2.511049in}{1.962056in}}%
\pgfpathlineto{\pgfqpoint{2.514292in}{1.962320in}}%
\pgfpathlineto{\pgfqpoint{2.516651in}{1.961949in}}%
\pgfpathlineto{\pgfqpoint{2.522253in}{1.961295in}}%
\pgfpathlineto{\pgfqpoint{2.524022in}{1.962184in}}%
\pgfpathlineto{\pgfqpoint{2.526675in}{1.962042in}}%
\pgfpathlineto{\pgfqpoint{2.553948in}{1.959694in}}%
\pgfpathlineto{\pgfqpoint{2.556159in}{1.960081in}}%
\pgfpathlineto{\pgfqpoint{2.568395in}{1.960307in}}%
\pgfpathlineto{\pgfqpoint{2.572080in}{1.958850in}}%
\pgfpathlineto{\pgfqpoint{2.578124in}{1.958443in}}%
\pgfpathlineto{\pgfqpoint{2.588296in}{1.960608in}}%
\pgfpathlineto{\pgfqpoint{2.592719in}{1.959647in}}%
\pgfpathlineto{\pgfqpoint{2.595225in}{1.958322in}}%
\pgfpathlineto{\pgfqpoint{2.601859in}{1.957387in}}%
\pgfpathlineto{\pgfqpoint{2.605102in}{1.958563in}}%
\pgfpathlineto{\pgfqpoint{2.607903in}{1.959915in}}%
\pgfpathlineto{\pgfqpoint{2.610556in}{1.960959in}}%
\pgfpathlineto{\pgfqpoint{2.614537in}{1.961180in}}%
\pgfpathlineto{\pgfqpoint{2.615569in}{1.961556in}}%
\pgfpathlineto{\pgfqpoint{2.616158in}{1.960797in}}%
\pgfpathlineto{\pgfqpoint{2.619549in}{1.958343in}}%
\pgfpathlineto{\pgfqpoint{2.623382in}{1.956267in}}%
\pgfpathlineto{\pgfqpoint{2.624414in}{1.955578in}}%
\pgfpathlineto{\pgfqpoint{2.625151in}{1.956225in}}%
\pgfpathlineto{\pgfqpoint{2.629279in}{1.956983in}}%
\pgfpathlineto{\pgfqpoint{2.631637in}{1.957722in}}%
\pgfpathlineto{\pgfqpoint{2.635175in}{1.960178in}}%
\pgfpathlineto{\pgfqpoint{2.636649in}{1.959267in}}%
\pgfpathlineto{\pgfqpoint{2.638271in}{1.960867in}}%
\pgfpathlineto{\pgfqpoint{2.640630in}{1.960313in}}%
\pgfpathlineto{\pgfqpoint{2.643726in}{1.958854in}}%
\pgfpathlineto{\pgfqpoint{2.649622in}{1.956747in}}%
\pgfpathlineto{\pgfqpoint{2.652865in}{1.956716in}}%
\pgfpathlineto{\pgfqpoint{2.654487in}{1.955729in}}%
\pgfpathlineto{\pgfqpoint{2.655077in}{1.956609in}}%
\pgfpathlineto{\pgfqpoint{2.657583in}{1.957742in}}%
\pgfpathlineto{\pgfqpoint{2.663480in}{1.959398in}}%
\pgfpathlineto{\pgfqpoint{2.667755in}{1.957329in}}%
\pgfpathlineto{\pgfqpoint{2.669966in}{1.957369in}}%
\pgfpathlineto{\pgfqpoint{2.674831in}{1.955916in}}%
\pgfpathlineto{\pgfqpoint{2.676010in}{1.956701in}}%
\pgfpathlineto{\pgfqpoint{2.676600in}{1.955974in}}%
\pgfpathlineto{\pgfqpoint{2.678664in}{1.955512in}}%
\pgfpathlineto{\pgfqpoint{2.683971in}{1.956524in}}%
\pgfpathlineto{\pgfqpoint{2.685003in}{1.956404in}}%
\pgfpathlineto{\pgfqpoint{2.685445in}{1.957183in}}%
\pgfpathlineto{\pgfqpoint{2.686329in}{1.958120in}}%
\pgfpathlineto{\pgfqpoint{2.687214in}{1.957508in}}%
\pgfpathlineto{\pgfqpoint{2.701514in}{1.957678in}}%
\pgfpathlineto{\pgfqpoint{2.704904in}{1.957104in}}%
\pgfpathlineto{\pgfqpoint{2.706673in}{1.957508in}}%
\pgfpathlineto{\pgfqpoint{2.706968in}{1.957066in}}%
\pgfpathlineto{\pgfqpoint{2.709179in}{1.956462in}}%
\pgfpathlineto{\pgfqpoint{2.716108in}{1.956817in}}%
\pgfpathlineto{\pgfqpoint{2.718319in}{1.956922in}}%
\pgfpathlineto{\pgfqpoint{2.720383in}{1.956262in}}%
\pgfpathlineto{\pgfqpoint{2.722300in}{1.957207in}}%
\pgfpathlineto{\pgfqpoint{2.730408in}{1.957180in}}%
\pgfpathlineto{\pgfqpoint{2.733061in}{1.956448in}}%
\pgfpathlineto{\pgfqpoint{2.737631in}{1.955642in}}%
\pgfpathlineto{\pgfqpoint{2.738221in}{1.955503in}}%
\pgfpathlineto{\pgfqpoint{2.738810in}{1.956489in}}%
\pgfpathlineto{\pgfqpoint{2.741169in}{1.957495in}}%
\pgfpathlineto{\pgfqpoint{2.744265in}{1.957967in}}%
\pgfpathlineto{\pgfqpoint{2.745739in}{1.958998in}}%
\pgfpathlineto{\pgfqpoint{2.747066in}{1.960412in}}%
\pgfpathlineto{\pgfqpoint{2.747803in}{1.959843in}}%
\pgfpathlineto{\pgfqpoint{2.750751in}{1.959853in}}%
\pgfpathlineto{\pgfqpoint{2.754437in}{1.960642in}}%
\pgfpathlineto{\pgfqpoint{2.755911in}{1.958955in}}%
\pgfpathlineto{\pgfqpoint{2.756796in}{1.959523in}}%
\pgfpathlineto{\pgfqpoint{2.760628in}{1.957392in}}%
\pgfpathlineto{\pgfqpoint{2.762840in}{1.956665in}}%
\pgfpathlineto{\pgfqpoint{2.767704in}{1.957086in}}%
\pgfpathlineto{\pgfqpoint{2.768589in}{1.957319in}}%
\pgfpathlineto{\pgfqpoint{2.768884in}{1.957892in}}%
\pgfpathlineto{\pgfqpoint{2.771390in}{1.959850in}}%
\pgfpathlineto{\pgfqpoint{2.772569in}{1.960485in}}%
\pgfpathlineto{\pgfqpoint{2.773159in}{1.959810in}}%
\pgfpathlineto{\pgfqpoint{2.774486in}{1.960611in}}%
\pgfpathlineto{\pgfqpoint{2.777139in}{1.962668in}}%
\pgfpathlineto{\pgfqpoint{2.777729in}{1.961622in}}%
\pgfpathlineto{\pgfqpoint{2.779645in}{1.960993in}}%
\pgfpathlineto{\pgfqpoint{2.785247in}{1.958044in}}%
\pgfpathlineto{\pgfqpoint{2.786132in}{1.957509in}}%
\pgfpathlineto{\pgfqpoint{2.786721in}{1.958269in}}%
\pgfpathlineto{\pgfqpoint{2.788638in}{1.958378in}}%
\pgfpathlineto{\pgfqpoint{2.789817in}{1.959281in}}%
\pgfpathlineto{\pgfqpoint{2.790407in}{1.958067in}}%
\pgfpathlineto{\pgfqpoint{2.791291in}{1.956946in}}%
\pgfpathlineto{\pgfqpoint{2.792029in}{1.958051in}}%
\pgfpathlineto{\pgfqpoint{2.795419in}{1.960632in}}%
\pgfpathlineto{\pgfqpoint{2.798073in}{1.960922in}}%
\pgfpathlineto{\pgfqpoint{2.799105in}{1.962492in}}%
\pgfpathlineto{\pgfqpoint{2.800431in}{1.964252in}}%
\pgfpathlineto{\pgfqpoint{2.801168in}{1.963822in}}%
\pgfpathlineto{\pgfqpoint{2.819596in}{1.963319in}}%
\pgfpathlineto{\pgfqpoint{2.820628in}{1.963678in}}%
\pgfpathlineto{\pgfqpoint{2.821217in}{1.962993in}}%
\pgfpathlineto{\pgfqpoint{2.821807in}{1.962967in}}%
\pgfpathlineto{\pgfqpoint{2.822249in}{1.963760in}}%
\pgfpathlineto{\pgfqpoint{2.825198in}{1.966821in}}%
\pgfpathlineto{\pgfqpoint{2.829031in}{1.965502in}}%
\pgfpathlineto{\pgfqpoint{2.830505in}{1.967373in}}%
\pgfpathlineto{\pgfqpoint{2.831094in}{1.966400in}}%
\pgfpathlineto{\pgfqpoint{2.833895in}{1.964028in}}%
\pgfpathlineto{\pgfqpoint{2.840382in}{1.963185in}}%
\pgfpathlineto{\pgfqpoint{2.843183in}{1.964166in}}%
\pgfpathlineto{\pgfqpoint{2.843330in}{1.964002in}}%
\pgfpathlineto{\pgfqpoint{2.846131in}{1.962485in}}%
\pgfpathlineto{\pgfqpoint{2.847458in}{1.964742in}}%
\pgfpathlineto{\pgfqpoint{2.848195in}{1.965163in}}%
\pgfpathlineto{\pgfqpoint{2.849079in}{1.964555in}}%
\pgfpathlineto{\pgfqpoint{2.853355in}{1.966377in}}%
\pgfpathlineto{\pgfqpoint{2.855566in}{1.966426in}}%
\pgfpathlineto{\pgfqpoint{2.859546in}{1.966276in}}%
\pgfpathlineto{\pgfqpoint{2.860726in}{1.967962in}}%
\pgfpathlineto{\pgfqpoint{2.861463in}{1.966676in}}%
\pgfpathlineto{\pgfqpoint{2.863379in}{1.966096in}}%
\pgfpathlineto{\pgfqpoint{2.865296in}{1.967321in}}%
\pgfpathlineto{\pgfqpoint{2.867654in}{1.967666in}}%
\pgfpathlineto{\pgfqpoint{2.869718in}{1.966158in}}%
\pgfpathlineto{\pgfqpoint{2.870308in}{1.967043in}}%
\pgfpathlineto{\pgfqpoint{2.872519in}{1.968076in}}%
\pgfpathlineto{\pgfqpoint{2.874435in}{1.967421in}}%
\pgfpathlineto{\pgfqpoint{2.875910in}{1.968031in}}%
\pgfpathlineto{\pgfqpoint{2.878711in}{1.969853in}}%
\pgfpathlineto{\pgfqpoint{2.879300in}{1.968970in}}%
\pgfpathlineto{\pgfqpoint{2.881512in}{1.968854in}}%
\pgfpathlineto{\pgfqpoint{2.882838in}{1.969814in}}%
\pgfpathlineto{\pgfqpoint{2.884755in}{1.970382in}}%
\pgfpathlineto{\pgfqpoint{2.888735in}{1.969989in}}%
\pgfpathlineto{\pgfqpoint{2.891536in}{1.969419in}}%
\pgfpathlineto{\pgfqpoint{2.892715in}{1.967723in}}%
\pgfpathlineto{\pgfqpoint{2.893452in}{1.968633in}}%
\pgfpathlineto{\pgfqpoint{2.896106in}{1.969534in}}%
\pgfpathlineto{\pgfqpoint{2.899644in}{1.966392in}}%
\pgfpathlineto{\pgfqpoint{2.900381in}{1.967778in}}%
\pgfpathlineto{\pgfqpoint{2.901560in}{1.968684in}}%
\pgfpathlineto{\pgfqpoint{2.902150in}{1.968276in}}%
\pgfpathlineto{\pgfqpoint{2.905836in}{1.967383in}}%
\pgfpathlineto{\pgfqpoint{2.908784in}{1.970037in}}%
\pgfpathlineto{\pgfqpoint{2.909079in}{1.969721in}}%
\pgfpathlineto{\pgfqpoint{2.910258in}{1.968504in}}%
\pgfpathlineto{\pgfqpoint{2.911143in}{1.969334in}}%
\pgfpathlineto{\pgfqpoint{2.916892in}{1.969164in}}%
\pgfpathlineto{\pgfqpoint{2.922789in}{1.967966in}}%
\pgfpathlineto{\pgfqpoint{2.924410in}{1.968647in}}%
\pgfpathlineto{\pgfqpoint{2.927654in}{1.967933in}}%
\pgfpathlineto{\pgfqpoint{2.930454in}{1.968292in}}%
\pgfpathlineto{\pgfqpoint{2.931781in}{1.970249in}}%
\pgfpathlineto{\pgfqpoint{2.932666in}{1.969561in}}%
\pgfpathlineto{\pgfqpoint{2.935909in}{1.971210in}}%
\pgfpathlineto{\pgfqpoint{2.939300in}{1.972683in}}%
\pgfpathlineto{\pgfqpoint{2.941511in}{1.972676in}}%
\pgfpathlineto{\pgfqpoint{2.944312in}{1.973244in}}%
\pgfpathlineto{\pgfqpoint{2.953894in}{1.969687in}}%
\pgfpathlineto{\pgfqpoint{2.956400in}{1.970184in}}%
\pgfpathlineto{\pgfqpoint{2.959349in}{1.970894in}}%
\pgfpathlineto{\pgfqpoint{2.963624in}{1.971825in}}%
\pgfpathlineto{\pgfqpoint{2.968488in}{1.972944in}}%
\pgfpathlineto{\pgfqpoint{2.977334in}{1.966964in}}%
\pgfpathlineto{\pgfqpoint{2.980135in}{1.967310in}}%
\pgfpathlineto{\pgfqpoint{2.983083in}{1.965499in}}%
\pgfpathlineto{\pgfqpoint{2.983967in}{1.966834in}}%
\pgfpathlineto{\pgfqpoint{2.986031in}{1.968317in}}%
\pgfpathlineto{\pgfqpoint{2.990896in}{1.970336in}}%
\pgfpathlineto{\pgfqpoint{2.992518in}{1.970673in}}%
\pgfpathlineto{\pgfqpoint{2.992813in}{1.970177in}}%
\pgfpathlineto{\pgfqpoint{2.993992in}{1.968213in}}%
\pgfpathlineto{\pgfqpoint{2.994876in}{1.969069in}}%
\pgfpathlineto{\pgfqpoint{2.998267in}{1.968987in}}%
\pgfpathlineto{\pgfqpoint{3.000921in}{1.966549in}}%
\pgfpathlineto{\pgfqpoint{3.001363in}{1.967242in}}%
\pgfpathlineto{\pgfqpoint{3.003574in}{1.968293in}}%
\pgfpathlineto{\pgfqpoint{3.005490in}{1.967327in}}%
\pgfpathlineto{\pgfqpoint{3.006817in}{1.966532in}}%
\pgfpathlineto{\pgfqpoint{3.007407in}{1.967231in}}%
\pgfpathlineto{\pgfqpoint{3.010208in}{1.969215in}}%
\pgfpathlineto{\pgfqpoint{3.010355in}{1.969103in}}%
\pgfpathlineto{\pgfqpoint{3.013009in}{1.968758in}}%
\pgfpathlineto{\pgfqpoint{3.014188in}{1.970093in}}%
\pgfpathlineto{\pgfqpoint{3.015368in}{1.972125in}}%
\pgfpathlineto{\pgfqpoint{3.016105in}{1.971276in}}%
\pgfpathlineto{\pgfqpoint{3.018906in}{1.969768in}}%
\pgfpathlineto{\pgfqpoint{3.022149in}{1.971062in}}%
\pgfpathlineto{\pgfqpoint{3.022296in}{1.970863in}}%
\pgfpathlineto{\pgfqpoint{3.024950in}{1.968076in}}%
\pgfpathlineto{\pgfqpoint{3.025392in}{1.968315in}}%
\pgfpathlineto{\pgfqpoint{3.028340in}{1.968371in}}%
\pgfpathlineto{\pgfqpoint{3.031289in}{1.965973in}}%
\pgfpathlineto{\pgfqpoint{3.031436in}{1.966094in}}%
\pgfpathlineto{\pgfqpoint{3.032910in}{1.968680in}}%
\pgfpathlineto{\pgfqpoint{3.033942in}{1.967076in}}%
\pgfpathlineto{\pgfqpoint{3.036743in}{1.964977in}}%
\pgfpathlineto{\pgfqpoint{3.037480in}{1.966233in}}%
\pgfpathlineto{\pgfqpoint{3.038365in}{1.967243in}}%
\pgfpathlineto{\pgfqpoint{3.038365in}{1.967243in}}%
\pgfusepath{stroke}%
\end{pgfscope}%
\begin{pgfscope}%
\pgfpathrectangle{\pgfqpoint{0.679669in}{0.526079in}}{\pgfqpoint{2.358696in}{1.661000in}} %
\pgfusepath{clip}%
\pgfsetrectcap%
\pgfsetroundjoin%
\pgfsetlinewidth{1.003750pt}%
\definecolor{currentstroke}{rgb}{0.501961,0.000000,0.501961}%
\pgfsetstrokecolor{currentstroke}%
\pgfsetdash{}{0pt}%
\pgfpathmoveto{\pgfqpoint{0.680119in}{0.512191in}}%
\pgfpathlineto{\pgfqpoint{0.682175in}{0.841208in}}%
\pgfpathlineto{\pgfqpoint{0.685713in}{0.919078in}}%
\pgfpathlineto{\pgfqpoint{0.686745in}{0.939180in}}%
\pgfpathlineto{\pgfqpoint{0.687482in}{0.931192in}}%
\pgfpathlineto{\pgfqpoint{0.690431in}{0.873826in}}%
\pgfpathlineto{\pgfqpoint{0.691905in}{0.879942in}}%
\pgfpathlineto{\pgfqpoint{0.692642in}{0.906568in}}%
\pgfpathlineto{\pgfqpoint{0.695001in}{0.959075in}}%
\pgfpathlineto{\pgfqpoint{0.695296in}{0.960067in}}%
\pgfpathlineto{\pgfqpoint{0.695738in}{0.956815in}}%
\pgfpathlineto{\pgfqpoint{0.697949in}{0.930530in}}%
\pgfpathlineto{\pgfqpoint{0.698391in}{0.935361in}}%
\pgfpathlineto{\pgfqpoint{0.701192in}{1.014375in}}%
\pgfpathlineto{\pgfqpoint{0.701929in}{1.008819in}}%
\pgfpathlineto{\pgfqpoint{0.702666in}{1.002355in}}%
\pgfpathlineto{\pgfqpoint{0.703256in}{1.009633in}}%
\pgfpathlineto{\pgfqpoint{0.706794in}{1.046057in}}%
\pgfpathlineto{\pgfqpoint{0.708416in}{1.060954in}}%
\pgfpathlineto{\pgfqpoint{0.709595in}{1.053591in}}%
\pgfpathlineto{\pgfqpoint{0.710627in}{1.038275in}}%
\pgfpathlineto{\pgfqpoint{0.712396in}{0.948624in}}%
\pgfpathlineto{\pgfqpoint{0.713281in}{0.985097in}}%
\pgfpathlineto{\pgfqpoint{0.715492in}{1.038317in}}%
\pgfpathlineto{\pgfqpoint{0.715639in}{1.038230in}}%
\pgfpathlineto{\pgfqpoint{0.715787in}{1.038512in}}%
\pgfpathlineto{\pgfqpoint{0.716229in}{1.036669in}}%
\pgfpathlineto{\pgfqpoint{0.717113in}{1.028730in}}%
\pgfpathlineto{\pgfqpoint{0.718882in}{0.956310in}}%
\pgfpathlineto{\pgfqpoint{0.719620in}{1.000589in}}%
\pgfpathlineto{\pgfqpoint{0.721389in}{1.099155in}}%
\pgfpathlineto{\pgfqpoint{0.721978in}{1.091142in}}%
\pgfpathlineto{\pgfqpoint{0.724927in}{1.030991in}}%
\pgfpathlineto{\pgfqpoint{0.725369in}{1.041704in}}%
\pgfpathlineto{\pgfqpoint{0.726990in}{1.114129in}}%
\pgfpathlineto{\pgfqpoint{0.727875in}{1.095878in}}%
\pgfpathlineto{\pgfqpoint{0.729791in}{0.977325in}}%
\pgfpathlineto{\pgfqpoint{0.731266in}{1.008179in}}%
\pgfpathlineto{\pgfqpoint{0.731708in}{1.010500in}}%
\pgfpathlineto{\pgfqpoint{0.733035in}{1.076775in}}%
\pgfpathlineto{\pgfqpoint{0.733624in}{1.090716in}}%
\pgfpathlineto{\pgfqpoint{0.734361in}{1.068944in}}%
\pgfpathlineto{\pgfqpoint{0.735836in}{0.982345in}}%
\pgfpathlineto{\pgfqpoint{0.736573in}{1.013826in}}%
\pgfpathlineto{\pgfqpoint{0.739816in}{1.114984in}}%
\pgfpathlineto{\pgfqpoint{0.740111in}{1.116159in}}%
\pgfpathlineto{\pgfqpoint{0.740553in}{1.108643in}}%
\pgfpathlineto{\pgfqpoint{0.741880in}{1.019787in}}%
\pgfpathlineto{\pgfqpoint{0.742617in}{0.984215in}}%
\pgfpathlineto{\pgfqpoint{0.743354in}{1.021396in}}%
\pgfpathlineto{\pgfqpoint{0.746302in}{1.116565in}}%
\pgfpathlineto{\pgfqpoint{0.746597in}{1.119853in}}%
\pgfpathlineto{\pgfqpoint{0.747039in}{1.112435in}}%
\pgfpathlineto{\pgfqpoint{0.748808in}{0.999974in}}%
\pgfpathlineto{\pgfqpoint{0.749693in}{1.059902in}}%
\pgfpathlineto{\pgfqpoint{0.751020in}{1.132290in}}%
\pgfpathlineto{\pgfqpoint{0.751757in}{1.114135in}}%
\pgfpathlineto{\pgfqpoint{0.755442in}{0.986468in}}%
\pgfpathlineto{\pgfqpoint{0.755737in}{0.995750in}}%
\pgfpathlineto{\pgfqpoint{0.757801in}{1.146951in}}%
\pgfpathlineto{\pgfqpoint{0.758685in}{1.120511in}}%
\pgfpathlineto{\pgfqpoint{0.760012in}{1.084798in}}%
\pgfpathlineto{\pgfqpoint{0.760897in}{1.088205in}}%
\pgfpathlineto{\pgfqpoint{0.762224in}{1.103086in}}%
\pgfpathlineto{\pgfqpoint{0.763993in}{1.168523in}}%
\pgfpathlineto{\pgfqpoint{0.764730in}{1.142509in}}%
\pgfpathlineto{\pgfqpoint{0.766351in}{0.989960in}}%
\pgfpathlineto{\pgfqpoint{0.767383in}{1.044311in}}%
\pgfpathlineto{\pgfqpoint{0.769742in}{1.096684in}}%
\pgfpathlineto{\pgfqpoint{0.770184in}{1.097758in}}%
\pgfpathlineto{\pgfqpoint{0.770774in}{1.095186in}}%
\pgfpathlineto{\pgfqpoint{0.771658in}{1.075107in}}%
\pgfpathlineto{\pgfqpoint{0.772395in}{1.058218in}}%
\pgfpathlineto{\pgfqpoint{0.772985in}{1.076551in}}%
\pgfpathlineto{\pgfqpoint{0.774754in}{1.166816in}}%
\pgfpathlineto{\pgfqpoint{0.775639in}{1.150064in}}%
\pgfpathlineto{\pgfqpoint{0.777850in}{1.111035in}}%
\pgfpathlineto{\pgfqpoint{0.778292in}{1.115876in}}%
\pgfpathlineto{\pgfqpoint{0.780651in}{1.189661in}}%
\pgfpathlineto{\pgfqpoint{0.781535in}{1.177027in}}%
\pgfpathlineto{\pgfqpoint{0.783010in}{1.070859in}}%
\pgfpathlineto{\pgfqpoint{0.783599in}{1.048927in}}%
\pgfpathlineto{\pgfqpoint{0.784336in}{1.079564in}}%
\pgfpathlineto{\pgfqpoint{0.786253in}{1.149853in}}%
\pgfpathlineto{\pgfqpoint{0.786842in}{1.140339in}}%
\pgfpathlineto{\pgfqpoint{0.788611in}{1.035283in}}%
\pgfpathlineto{\pgfqpoint{0.789349in}{0.998861in}}%
\pgfpathlineto{\pgfqpoint{0.789938in}{1.031946in}}%
\pgfpathlineto{\pgfqpoint{0.792149in}{1.145397in}}%
\pgfpathlineto{\pgfqpoint{0.792592in}{1.143010in}}%
\pgfpathlineto{\pgfqpoint{0.795393in}{1.078146in}}%
\pgfpathlineto{\pgfqpoint{0.796130in}{1.099223in}}%
\pgfpathlineto{\pgfqpoint{0.798046in}{1.174269in}}%
\pgfpathlineto{\pgfqpoint{0.798636in}{1.170537in}}%
\pgfpathlineto{\pgfqpoint{0.799963in}{1.126811in}}%
\pgfpathlineto{\pgfqpoint{0.800552in}{1.113616in}}%
\pgfpathlineto{\pgfqpoint{0.801437in}{1.129196in}}%
\pgfpathlineto{\pgfqpoint{0.803648in}{1.196442in}}%
\pgfpathlineto{\pgfqpoint{0.804385in}{1.182646in}}%
\pgfpathlineto{\pgfqpoint{0.806744in}{1.062394in}}%
\pgfpathlineto{\pgfqpoint{0.807923in}{1.105685in}}%
\pgfpathlineto{\pgfqpoint{0.810282in}{1.172838in}}%
\pgfpathlineto{\pgfqpoint{0.810872in}{1.168217in}}%
\pgfpathlineto{\pgfqpoint{0.811609in}{1.162355in}}%
\pgfpathlineto{\pgfqpoint{0.812346in}{1.168584in}}%
\pgfpathlineto{\pgfqpoint{0.816326in}{1.236370in}}%
\pgfpathlineto{\pgfqpoint{0.817505in}{1.224427in}}%
\pgfpathlineto{\pgfqpoint{0.823992in}{1.126508in}}%
\pgfpathlineto{\pgfqpoint{0.824876in}{1.106958in}}%
\pgfpathlineto{\pgfqpoint{0.825613in}{1.118967in}}%
\pgfpathlineto{\pgfqpoint{0.826645in}{1.134557in}}%
\pgfpathlineto{\pgfqpoint{0.827530in}{1.128451in}}%
\pgfpathlineto{\pgfqpoint{0.827677in}{1.128131in}}%
\pgfpathlineto{\pgfqpoint{0.827972in}{1.130871in}}%
\pgfpathlineto{\pgfqpoint{0.829594in}{1.201312in}}%
\pgfpathlineto{\pgfqpoint{0.830773in}{1.225144in}}%
\pgfpathlineto{\pgfqpoint{0.831658in}{1.221223in}}%
\pgfpathlineto{\pgfqpoint{0.832395in}{1.235977in}}%
\pgfpathlineto{\pgfqpoint{0.833869in}{1.270222in}}%
\pgfpathlineto{\pgfqpoint{0.834753in}{1.266130in}}%
\pgfpathlineto{\pgfqpoint{0.836375in}{1.257232in}}%
\pgfpathlineto{\pgfqpoint{0.840503in}{1.165821in}}%
\pgfpathlineto{\pgfqpoint{0.842272in}{1.088238in}}%
\pgfpathlineto{\pgfqpoint{0.843156in}{1.118502in}}%
\pgfpathlineto{\pgfqpoint{0.846547in}{1.230304in}}%
\pgfpathlineto{\pgfqpoint{0.847284in}{1.227061in}}%
\pgfpathlineto{\pgfqpoint{0.847726in}{1.228531in}}%
\pgfpathlineto{\pgfqpoint{0.849348in}{1.259144in}}%
\pgfpathlineto{\pgfqpoint{0.851559in}{1.280404in}}%
\pgfpathlineto{\pgfqpoint{0.851854in}{1.280072in}}%
\pgfpathlineto{\pgfqpoint{0.853181in}{1.269303in}}%
\pgfpathlineto{\pgfqpoint{0.860846in}{1.142820in}}%
\pgfpathlineto{\pgfqpoint{0.861584in}{1.162707in}}%
\pgfpathlineto{\pgfqpoint{0.864385in}{1.237131in}}%
\pgfpathlineto{\pgfqpoint{0.869397in}{1.311446in}}%
\pgfpathlineto{\pgfqpoint{0.870134in}{1.297087in}}%
\pgfpathlineto{\pgfqpoint{0.871755in}{1.262755in}}%
\pgfpathlineto{\pgfqpoint{0.872493in}{1.265758in}}%
\pgfpathlineto{\pgfqpoint{0.872640in}{1.266087in}}%
\pgfpathlineto{\pgfqpoint{0.873082in}{1.263439in}}%
\pgfpathlineto{\pgfqpoint{0.874409in}{1.241250in}}%
\pgfpathlineto{\pgfqpoint{0.875293in}{1.252695in}}%
\pgfpathlineto{\pgfqpoint{0.875588in}{1.254056in}}%
\pgfpathlineto{\pgfqpoint{0.876031in}{1.248921in}}%
\pgfpathlineto{\pgfqpoint{0.877652in}{1.171110in}}%
\pgfpathlineto{\pgfqpoint{0.878094in}{1.160298in}}%
\pgfpathlineto{\pgfqpoint{0.878684in}{1.181281in}}%
\pgfpathlineto{\pgfqpoint{0.882075in}{1.295217in}}%
\pgfpathlineto{\pgfqpoint{0.882370in}{1.297030in}}%
\pgfpathlineto{\pgfqpoint{0.882959in}{1.291825in}}%
\pgfpathlineto{\pgfqpoint{0.884139in}{1.277013in}}%
\pgfpathlineto{\pgfqpoint{0.884728in}{1.283288in}}%
\pgfpathlineto{\pgfqpoint{0.887529in}{1.348015in}}%
\pgfpathlineto{\pgfqpoint{0.888414in}{1.341921in}}%
\pgfpathlineto{\pgfqpoint{0.889888in}{1.289606in}}%
\pgfpathlineto{\pgfqpoint{0.890920in}{1.252370in}}%
\pgfpathlineto{\pgfqpoint{0.891657in}{1.267772in}}%
\pgfpathlineto{\pgfqpoint{0.892689in}{1.290189in}}%
\pgfpathlineto{\pgfqpoint{0.893426in}{1.279902in}}%
\pgfpathlineto{\pgfqpoint{0.896669in}{1.195866in}}%
\pgfpathlineto{\pgfqpoint{0.897406in}{1.212358in}}%
\pgfpathlineto{\pgfqpoint{0.899765in}{1.315467in}}%
\pgfpathlineto{\pgfqpoint{0.900649in}{1.310182in}}%
\pgfpathlineto{\pgfqpoint{0.901387in}{1.307067in}}%
\pgfpathlineto{\pgfqpoint{0.902271in}{1.309299in}}%
\pgfpathlineto{\pgfqpoint{0.903450in}{1.320010in}}%
\pgfpathlineto{\pgfqpoint{0.905514in}{1.366997in}}%
\pgfpathlineto{\pgfqpoint{0.906399in}{1.350338in}}%
\pgfpathlineto{\pgfqpoint{0.908905in}{1.260663in}}%
\pgfpathlineto{\pgfqpoint{0.909937in}{1.264197in}}%
\pgfpathlineto{\pgfqpoint{0.911558in}{1.283809in}}%
\pgfpathlineto{\pgfqpoint{0.912148in}{1.276024in}}%
\pgfpathlineto{\pgfqpoint{0.913475in}{1.185362in}}%
\pgfpathlineto{\pgfqpoint{0.914065in}{1.148349in}}%
\pgfpathlineto{\pgfqpoint{0.914802in}{1.184894in}}%
\pgfpathlineto{\pgfqpoint{0.917750in}{1.315451in}}%
\pgfpathlineto{\pgfqpoint{0.918782in}{1.322646in}}%
\pgfpathlineto{\pgfqpoint{0.919372in}{1.316394in}}%
\pgfpathlineto{\pgfqpoint{0.920698in}{1.284005in}}%
\pgfpathlineto{\pgfqpoint{0.921435in}{1.300860in}}%
\pgfpathlineto{\pgfqpoint{0.923057in}{1.332057in}}%
\pgfpathlineto{\pgfqpoint{0.923647in}{1.327740in}}%
\pgfpathlineto{\pgfqpoint{0.926890in}{1.267579in}}%
\pgfpathlineto{\pgfqpoint{0.927332in}{1.263499in}}%
\pgfpathlineto{\pgfqpoint{0.927922in}{1.275802in}}%
\pgfpathlineto{\pgfqpoint{0.929396in}{1.315440in}}%
\pgfpathlineto{\pgfqpoint{0.929986in}{1.302598in}}%
\pgfpathlineto{\pgfqpoint{0.933376in}{1.206654in}}%
\pgfpathlineto{\pgfqpoint{0.933524in}{1.207561in}}%
\pgfpathlineto{\pgfqpoint{0.934556in}{1.263799in}}%
\pgfpathlineto{\pgfqpoint{0.936177in}{1.324272in}}%
\pgfpathlineto{\pgfqpoint{0.936767in}{1.318078in}}%
\pgfpathlineto{\pgfqpoint{0.938094in}{1.284334in}}%
\pgfpathlineto{\pgfqpoint{0.938978in}{1.298216in}}%
\pgfpathlineto{\pgfqpoint{0.941927in}{1.345313in}}%
\pgfpathlineto{\pgfqpoint{0.942369in}{1.340060in}}%
\pgfpathlineto{\pgfqpoint{0.943696in}{1.255629in}}%
\pgfpathlineto{\pgfqpoint{0.944433in}{1.219728in}}%
\pgfpathlineto{\pgfqpoint{0.945317in}{1.244954in}}%
\pgfpathlineto{\pgfqpoint{0.947381in}{1.299649in}}%
\pgfpathlineto{\pgfqpoint{0.947823in}{1.295314in}}%
\pgfpathlineto{\pgfqpoint{0.949592in}{1.233451in}}%
\pgfpathlineto{\pgfqpoint{0.950624in}{1.175614in}}%
\pgfpathlineto{\pgfqpoint{0.951214in}{1.216979in}}%
\pgfpathlineto{\pgfqpoint{0.953278in}{1.339988in}}%
\pgfpathlineto{\pgfqpoint{0.953720in}{1.337951in}}%
\pgfpathlineto{\pgfqpoint{0.954899in}{1.319554in}}%
\pgfpathlineto{\pgfqpoint{0.956374in}{1.275063in}}%
\pgfpathlineto{\pgfqpoint{0.957111in}{1.289513in}}%
\pgfpathlineto{\pgfqpoint{0.959175in}{1.373550in}}%
\pgfpathlineto{\pgfqpoint{0.959912in}{1.354453in}}%
\pgfpathlineto{\pgfqpoint{0.962123in}{1.267778in}}%
\pgfpathlineto{\pgfqpoint{0.963007in}{1.273532in}}%
\pgfpathlineto{\pgfqpoint{0.964187in}{1.314003in}}%
\pgfpathlineto{\pgfqpoint{0.965219in}{1.343904in}}%
\pgfpathlineto{\pgfqpoint{0.965808in}{1.331139in}}%
\pgfpathlineto{\pgfqpoint{0.967725in}{1.225349in}}%
\pgfpathlineto{\pgfqpoint{0.968609in}{1.259801in}}%
\pgfpathlineto{\pgfqpoint{0.971705in}{1.372613in}}%
\pgfpathlineto{\pgfqpoint{0.972147in}{1.367712in}}%
\pgfpathlineto{\pgfqpoint{0.973916in}{1.291515in}}%
\pgfpathlineto{\pgfqpoint{0.974948in}{1.324206in}}%
\pgfpathlineto{\pgfqpoint{0.976570in}{1.370000in}}%
\pgfpathlineto{\pgfqpoint{0.977307in}{1.364117in}}%
\pgfpathlineto{\pgfqpoint{0.978634in}{1.328852in}}%
\pgfpathlineto{\pgfqpoint{0.980403in}{1.216349in}}%
\pgfpathlineto{\pgfqpoint{0.981287in}{1.264320in}}%
\pgfpathlineto{\pgfqpoint{0.982614in}{1.329718in}}%
\pgfpathlineto{\pgfqpoint{0.983351in}{1.316257in}}%
\pgfpathlineto{\pgfqpoint{0.986447in}{1.258063in}}%
\pgfpathlineto{\pgfqpoint{0.986889in}{1.266632in}}%
\pgfpathlineto{\pgfqpoint{0.988953in}{1.359122in}}%
\pgfpathlineto{\pgfqpoint{0.989838in}{1.340655in}}%
\pgfpathlineto{\pgfqpoint{0.991312in}{1.293373in}}%
\pgfpathlineto{\pgfqpoint{0.992196in}{1.301376in}}%
\pgfpathlineto{\pgfqpoint{0.993818in}{1.325446in}}%
\pgfpathlineto{\pgfqpoint{0.995145in}{1.359396in}}%
\pgfpathlineto{\pgfqpoint{0.995882in}{1.343553in}}%
\pgfpathlineto{\pgfqpoint{0.997798in}{1.195170in}}%
\pgfpathlineto{\pgfqpoint{0.999125in}{1.238224in}}%
\pgfpathlineto{\pgfqpoint{1.001631in}{1.285688in}}%
\pgfpathlineto{\pgfqpoint{1.002073in}{1.281509in}}%
\pgfpathlineto{\pgfqpoint{1.003105in}{1.216001in}}%
\pgfpathlineto{\pgfqpoint{1.003842in}{1.165862in}}%
\pgfpathlineto{\pgfqpoint{1.004579in}{1.221636in}}%
\pgfpathlineto{\pgfqpoint{1.007086in}{1.329611in}}%
\pgfpathlineto{\pgfqpoint{1.008118in}{1.338565in}}%
\pgfpathlineto{\pgfqpoint{1.009002in}{1.345415in}}%
\pgfpathlineto{\pgfqpoint{1.009592in}{1.339797in}}%
\pgfpathlineto{\pgfqpoint{1.010476in}{1.328839in}}%
\pgfpathlineto{\pgfqpoint{1.011213in}{1.337749in}}%
\pgfpathlineto{\pgfqpoint{1.012540in}{1.361623in}}%
\pgfpathlineto{\pgfqpoint{1.013130in}{1.352104in}}%
\pgfpathlineto{\pgfqpoint{1.017257in}{1.208760in}}%
\pgfpathlineto{\pgfqpoint{1.018142in}{1.235382in}}%
\pgfpathlineto{\pgfqpoint{1.019469in}{1.275535in}}%
\pgfpathlineto{\pgfqpoint{1.020058in}{1.265141in}}%
\pgfpathlineto{\pgfqpoint{1.020943in}{1.247892in}}%
\pgfpathlineto{\pgfqpoint{1.021533in}{1.258487in}}%
\pgfpathlineto{\pgfqpoint{1.025955in}{1.371552in}}%
\pgfpathlineto{\pgfqpoint{1.026545in}{1.364763in}}%
\pgfpathlineto{\pgfqpoint{1.028019in}{1.338854in}}%
\pgfpathlineto{\pgfqpoint{1.029051in}{1.342183in}}%
\pgfpathlineto{\pgfqpoint{1.029788in}{1.332824in}}%
\pgfpathlineto{\pgfqpoint{1.030820in}{1.316174in}}%
\pgfpathlineto{\pgfqpoint{1.031704in}{1.324504in}}%
\pgfpathlineto{\pgfqpoint{1.032147in}{1.326972in}}%
\pgfpathlineto{\pgfqpoint{1.032589in}{1.321059in}}%
\pgfpathlineto{\pgfqpoint{1.034063in}{1.235893in}}%
\pgfpathlineto{\pgfqpoint{1.034653in}{1.214637in}}%
\pgfpathlineto{\pgfqpoint{1.035390in}{1.242022in}}%
\pgfpathlineto{\pgfqpoint{1.036422in}{1.273795in}}%
\pgfpathlineto{\pgfqpoint{1.037454in}{1.272235in}}%
\pgfpathlineto{\pgfqpoint{1.039370in}{1.312480in}}%
\pgfpathlineto{\pgfqpoint{1.040402in}{1.297275in}}%
\pgfpathlineto{\pgfqpoint{1.040697in}{1.294007in}}%
\pgfpathlineto{\pgfqpoint{1.041287in}{1.303454in}}%
\pgfpathlineto{\pgfqpoint{1.043793in}{1.376337in}}%
\pgfpathlineto{\pgfqpoint{1.044530in}{1.372575in}}%
\pgfpathlineto{\pgfqpoint{1.046152in}{1.351296in}}%
\pgfpathlineto{\pgfqpoint{1.047773in}{1.287594in}}%
\pgfpathlineto{\pgfqpoint{1.048658in}{1.316279in}}%
\pgfpathlineto{\pgfqpoint{1.049395in}{1.331613in}}%
\pgfpathlineto{\pgfqpoint{1.050132in}{1.322542in}}%
\pgfpathlineto{\pgfqpoint{1.053817in}{1.217856in}}%
\pgfpathlineto{\pgfqpoint{1.054554in}{1.241553in}}%
\pgfpathlineto{\pgfqpoint{1.056029in}{1.317022in}}%
\pgfpathlineto{\pgfqpoint{1.056766in}{1.303285in}}%
\pgfpathlineto{\pgfqpoint{1.058092in}{1.261703in}}%
\pgfpathlineto{\pgfqpoint{1.058682in}{1.276722in}}%
\pgfpathlineto{\pgfqpoint{1.062515in}{1.391650in}}%
\pgfpathlineto{\pgfqpoint{1.062662in}{1.391980in}}%
\pgfpathlineto{\pgfqpoint{1.062957in}{1.390562in}}%
\pgfpathlineto{\pgfqpoint{1.063989in}{1.361849in}}%
\pgfpathlineto{\pgfqpoint{1.065021in}{1.332731in}}%
\pgfpathlineto{\pgfqpoint{1.065758in}{1.349823in}}%
\pgfpathlineto{\pgfqpoint{1.066495in}{1.363828in}}%
\pgfpathlineto{\pgfqpoint{1.067380in}{1.352931in}}%
\pgfpathlineto{\pgfqpoint{1.070033in}{1.286426in}}%
\pgfpathlineto{\pgfqpoint{1.071213in}{1.209826in}}%
\pgfpathlineto{\pgfqpoint{1.071950in}{1.249254in}}%
\pgfpathlineto{\pgfqpoint{1.073571in}{1.345882in}}%
\pgfpathlineto{\pgfqpoint{1.074161in}{1.334064in}}%
\pgfpathlineto{\pgfqpoint{1.076667in}{1.300303in}}%
\pgfpathlineto{\pgfqpoint{1.077404in}{1.287082in}}%
\pgfpathlineto{\pgfqpoint{1.077994in}{1.298658in}}%
\pgfpathlineto{\pgfqpoint{1.079763in}{1.388949in}}%
\pgfpathlineto{\pgfqpoint{1.080647in}{1.368223in}}%
\pgfpathlineto{\pgfqpoint{1.081974in}{1.326247in}}%
\pgfpathlineto{\pgfqpoint{1.082859in}{1.335541in}}%
\pgfpathlineto{\pgfqpoint{1.086249in}{1.375829in}}%
\pgfpathlineto{\pgfqpoint{1.086692in}{1.370038in}}%
\pgfpathlineto{\pgfqpoint{1.088608in}{1.254029in}}%
\pgfpathlineto{\pgfqpoint{1.089935in}{1.299007in}}%
\pgfpathlineto{\pgfqpoint{1.092736in}{1.365292in}}%
\pgfpathlineto{\pgfqpoint{1.093031in}{1.361428in}}%
\pgfpathlineto{\pgfqpoint{1.094652in}{1.277552in}}%
\pgfpathlineto{\pgfqpoint{1.095537in}{1.326397in}}%
\pgfpathlineto{\pgfqpoint{1.097748in}{1.430068in}}%
\pgfpathlineto{\pgfqpoint{1.098190in}{1.429425in}}%
\pgfpathlineto{\pgfqpoint{1.099222in}{1.428034in}}%
\pgfpathlineto{\pgfqpoint{1.100254in}{1.397972in}}%
\pgfpathlineto{\pgfqpoint{1.101139in}{1.373829in}}%
\pgfpathlineto{\pgfqpoint{1.101876in}{1.387132in}}%
\pgfpathlineto{\pgfqpoint{1.103350in}{1.422494in}}%
\pgfpathlineto{\pgfqpoint{1.103940in}{1.412505in}}%
\pgfpathlineto{\pgfqpoint{1.106740in}{1.236759in}}%
\pgfpathlineto{\pgfqpoint{1.107330in}{1.213858in}}%
\pgfpathlineto{\pgfqpoint{1.107920in}{1.240501in}}%
\pgfpathlineto{\pgfqpoint{1.109984in}{1.383756in}}%
\pgfpathlineto{\pgfqpoint{1.110721in}{1.367191in}}%
\pgfpathlineto{\pgfqpoint{1.112048in}{1.322455in}}%
\pgfpathlineto{\pgfqpoint{1.112785in}{1.337703in}}%
\pgfpathlineto{\pgfqpoint{1.116175in}{1.451970in}}%
\pgfpathlineto{\pgfqpoint{1.117060in}{1.435415in}}%
\pgfpathlineto{\pgfqpoint{1.118681in}{1.376628in}}%
\pgfpathlineto{\pgfqpoint{1.119566in}{1.389321in}}%
\pgfpathlineto{\pgfqpoint{1.120303in}{1.395829in}}%
\pgfpathlineto{\pgfqpoint{1.121188in}{1.393132in}}%
\pgfpathlineto{\pgfqpoint{1.122219in}{1.385518in}}%
\pgfpathlineto{\pgfqpoint{1.123399in}{1.327377in}}%
\pgfpathlineto{\pgfqpoint{1.124873in}{1.205183in}}%
\pgfpathlineto{\pgfqpoint{1.125610in}{1.248377in}}%
\pgfpathlineto{\pgfqpoint{1.127821in}{1.313309in}}%
\pgfpathlineto{\pgfqpoint{1.129001in}{1.327935in}}%
\pgfpathlineto{\pgfqpoint{1.131949in}{1.380435in}}%
\pgfpathlineto{\pgfqpoint{1.134013in}{1.435781in}}%
\pgfpathlineto{\pgfqpoint{1.134455in}{1.435567in}}%
\pgfpathlineto{\pgfqpoint{1.135340in}{1.430848in}}%
\pgfpathlineto{\pgfqpoint{1.137109in}{1.391947in}}%
\pgfpathlineto{\pgfqpoint{1.138141in}{1.365650in}}%
\pgfpathlineto{\pgfqpoint{1.139025in}{1.378896in}}%
\pgfpathlineto{\pgfqpoint{1.139467in}{1.383554in}}%
\pgfpathlineto{\pgfqpoint{1.140057in}{1.373401in}}%
\pgfpathlineto{\pgfqpoint{1.144185in}{1.243699in}}%
\pgfpathlineto{\pgfqpoint{1.144627in}{1.259187in}}%
\pgfpathlineto{\pgfqpoint{1.146543in}{1.360951in}}%
\pgfpathlineto{\pgfqpoint{1.147428in}{1.348028in}}%
\pgfpathlineto{\pgfqpoint{1.148607in}{1.318016in}}%
\pgfpathlineto{\pgfqpoint{1.149197in}{1.331376in}}%
\pgfpathlineto{\pgfqpoint{1.152735in}{1.413414in}}%
\pgfpathlineto{\pgfqpoint{1.153177in}{1.409685in}}%
\pgfpathlineto{\pgfqpoint{1.155389in}{1.342450in}}%
\pgfpathlineto{\pgfqpoint{1.156715in}{1.369036in}}%
\pgfpathlineto{\pgfqpoint{1.157305in}{1.372794in}}%
\pgfpathlineto{\pgfqpoint{1.157895in}{1.367427in}}%
\pgfpathlineto{\pgfqpoint{1.159811in}{1.318348in}}%
\pgfpathlineto{\pgfqpoint{1.161433in}{1.210723in}}%
\pgfpathlineto{\pgfqpoint{1.162317in}{1.259477in}}%
\pgfpathlineto{\pgfqpoint{1.163939in}{1.326668in}}%
\pgfpathlineto{\pgfqpoint{1.164529in}{1.321919in}}%
\pgfpathlineto{\pgfqpoint{1.167477in}{1.284892in}}%
\pgfpathlineto{\pgfqpoint{1.168067in}{1.293708in}}%
\pgfpathlineto{\pgfqpoint{1.170425in}{1.378898in}}%
\pgfpathlineto{\pgfqpoint{1.171310in}{1.361533in}}%
\pgfpathlineto{\pgfqpoint{1.172342in}{1.342174in}}%
\pgfpathlineto{\pgfqpoint{1.173374in}{1.343206in}}%
\pgfpathlineto{\pgfqpoint{1.173963in}{1.341944in}}%
\pgfpathlineto{\pgfqpoint{1.174406in}{1.343453in}}%
\pgfpathlineto{\pgfqpoint{1.176469in}{1.372868in}}%
\pgfpathlineto{\pgfqpoint{1.177207in}{1.360578in}}%
\pgfpathlineto{\pgfqpoint{1.178976in}{1.300861in}}%
\pgfpathlineto{\pgfqpoint{1.180007in}{1.309422in}}%
\pgfpathlineto{\pgfqpoint{1.180892in}{1.315721in}}%
\pgfpathlineto{\pgfqpoint{1.181629in}{1.312055in}}%
\pgfpathlineto{\pgfqpoint{1.181924in}{1.310951in}}%
\pgfpathlineto{\pgfqpoint{1.182366in}{1.314761in}}%
\pgfpathlineto{\pgfqpoint{1.183251in}{1.327879in}}%
\pgfpathlineto{\pgfqpoint{1.183840in}{1.314847in}}%
\pgfpathlineto{\pgfqpoint{1.184872in}{1.275158in}}%
\pgfpathlineto{\pgfqpoint{1.185609in}{1.297583in}}%
\pgfpathlineto{\pgfqpoint{1.187378in}{1.342753in}}%
\pgfpathlineto{\pgfqpoint{1.187968in}{1.337203in}}%
\pgfpathlineto{\pgfqpoint{1.188558in}{1.330702in}}%
\pgfpathlineto{\pgfqpoint{1.189147in}{1.338345in}}%
\pgfpathlineto{\pgfqpoint{1.190032in}{1.356180in}}%
\pgfpathlineto{\pgfqpoint{1.190769in}{1.343308in}}%
\pgfpathlineto{\pgfqpoint{1.191801in}{1.317741in}}%
\pgfpathlineto{\pgfqpoint{1.192538in}{1.334066in}}%
\pgfpathlineto{\pgfqpoint{1.193717in}{1.361920in}}%
\pgfpathlineto{\pgfqpoint{1.194307in}{1.350939in}}%
\pgfpathlineto{\pgfqpoint{1.195634in}{1.311728in}}%
\pgfpathlineto{\pgfqpoint{1.196813in}{1.315022in}}%
\pgfpathlineto{\pgfqpoint{1.197993in}{1.300811in}}%
\pgfpathlineto{\pgfqpoint{1.198582in}{1.308214in}}%
\pgfpathlineto{\pgfqpoint{1.200351in}{1.348065in}}%
\pgfpathlineto{\pgfqpoint{1.201088in}{1.332554in}}%
\pgfpathlineto{\pgfqpoint{1.201825in}{1.312510in}}%
\pgfpathlineto{\pgfqpoint{1.202563in}{1.330236in}}%
\pgfpathlineto{\pgfqpoint{1.203889in}{1.369183in}}%
\pgfpathlineto{\pgfqpoint{1.204774in}{1.363261in}}%
\pgfpathlineto{\pgfqpoint{1.205069in}{1.362576in}}%
\pgfpathlineto{\pgfqpoint{1.205511in}{1.365688in}}%
\pgfpathlineto{\pgfqpoint{1.206543in}{1.377039in}}%
\pgfpathlineto{\pgfqpoint{1.207132in}{1.371302in}}%
\pgfpathlineto{\pgfqpoint{1.208459in}{1.345262in}}%
\pgfpathlineto{\pgfqpoint{1.209344in}{1.353166in}}%
\pgfpathlineto{\pgfqpoint{1.210081in}{1.358948in}}%
\pgfpathlineto{\pgfqpoint{1.210818in}{1.352576in}}%
\pgfpathlineto{\pgfqpoint{1.212587in}{1.324772in}}%
\pgfpathlineto{\pgfqpoint{1.213914in}{1.325637in}}%
\pgfpathlineto{\pgfqpoint{1.215240in}{1.306259in}}%
\pgfpathlineto{\pgfqpoint{1.215830in}{1.315615in}}%
\pgfpathlineto{\pgfqpoint{1.216862in}{1.332868in}}%
\pgfpathlineto{\pgfqpoint{1.217452in}{1.324412in}}%
\pgfpathlineto{\pgfqpoint{1.218336in}{1.312765in}}%
\pgfpathlineto{\pgfqpoint{1.219073in}{1.321947in}}%
\pgfpathlineto{\pgfqpoint{1.220695in}{1.332232in}}%
\pgfpathlineto{\pgfqpoint{1.221137in}{1.331910in}}%
\pgfpathlineto{\pgfqpoint{1.221432in}{1.331675in}}%
\pgfpathlineto{\pgfqpoint{1.221727in}{1.332362in}}%
\pgfpathlineto{\pgfqpoint{1.222611in}{1.348098in}}%
\pgfpathlineto{\pgfqpoint{1.224086in}{1.376220in}}%
\pgfpathlineto{\pgfqpoint{1.224970in}{1.373117in}}%
\pgfpathlineto{\pgfqpoint{1.227771in}{1.355318in}}%
\pgfpathlineto{\pgfqpoint{1.228213in}{1.352372in}}%
\pgfpathlineto{\pgfqpoint{1.228803in}{1.357246in}}%
\pgfpathlineto{\pgfqpoint{1.230130in}{1.376080in}}%
\pgfpathlineto{\pgfqpoint{1.230867in}{1.369754in}}%
\pgfpathlineto{\pgfqpoint{1.232636in}{1.329462in}}%
\pgfpathlineto{\pgfqpoint{1.233815in}{1.339718in}}%
\pgfpathlineto{\pgfqpoint{1.234257in}{1.337369in}}%
\pgfpathlineto{\pgfqpoint{1.234995in}{1.329995in}}%
\pgfpathlineto{\pgfqpoint{1.235584in}{1.337709in}}%
\pgfpathlineto{\pgfqpoint{1.236616in}{1.357802in}}%
\pgfpathlineto{\pgfqpoint{1.237353in}{1.346956in}}%
\pgfpathlineto{\pgfqpoint{1.238533in}{1.322522in}}%
\pgfpathlineto{\pgfqpoint{1.239122in}{1.333388in}}%
\pgfpathlineto{\pgfqpoint{1.240744in}{1.375746in}}%
\pgfpathlineto{\pgfqpoint{1.241481in}{1.366941in}}%
\pgfpathlineto{\pgfqpoint{1.242808in}{1.346124in}}%
\pgfpathlineto{\pgfqpoint{1.243692in}{1.349192in}}%
\pgfpathlineto{\pgfqpoint{1.244429in}{1.343130in}}%
\pgfpathlineto{\pgfqpoint{1.244872in}{1.339942in}}%
\pgfpathlineto{\pgfqpoint{1.245314in}{1.345116in}}%
\pgfpathlineto{\pgfqpoint{1.246788in}{1.384457in}}%
\pgfpathlineto{\pgfqpoint{1.247673in}{1.370177in}}%
\pgfpathlineto{\pgfqpoint{1.249147in}{1.341555in}}%
\pgfpathlineto{\pgfqpoint{1.249884in}{1.346104in}}%
\pgfpathlineto{\pgfqpoint{1.253422in}{1.389886in}}%
\pgfpathlineto{\pgfqpoint{1.254159in}{1.381590in}}%
\pgfpathlineto{\pgfqpoint{1.255338in}{1.358740in}}%
\pgfpathlineto{\pgfqpoint{1.256075in}{1.369926in}}%
\pgfpathlineto{\pgfqpoint{1.257402in}{1.395221in}}%
\pgfpathlineto{\pgfqpoint{1.258287in}{1.391194in}}%
\pgfpathlineto{\pgfqpoint{1.258582in}{1.390742in}}%
\pgfpathlineto{\pgfqpoint{1.259171in}{1.393226in}}%
\pgfpathlineto{\pgfqpoint{1.259761in}{1.395241in}}%
\pgfpathlineto{\pgfqpoint{1.260351in}{1.391351in}}%
\pgfpathlineto{\pgfqpoint{1.262267in}{1.370259in}}%
\pgfpathlineto{\pgfqpoint{1.263152in}{1.374317in}}%
\pgfpathlineto{\pgfqpoint{1.263741in}{1.377431in}}%
\pgfpathlineto{\pgfqpoint{1.264478in}{1.373005in}}%
\pgfpathlineto{\pgfqpoint{1.265658in}{1.359708in}}%
\pgfpathlineto{\pgfqpoint{1.266395in}{1.365017in}}%
\pgfpathlineto{\pgfqpoint{1.266690in}{1.366196in}}%
\pgfpathlineto{\pgfqpoint{1.267132in}{1.362701in}}%
\pgfpathlineto{\pgfqpoint{1.268459in}{1.338709in}}%
\pgfpathlineto{\pgfqpoint{1.269196in}{1.352426in}}%
\pgfpathlineto{\pgfqpoint{1.270670in}{1.380818in}}%
\pgfpathlineto{\pgfqpoint{1.271554in}{1.376974in}}%
\pgfpathlineto{\pgfqpoint{1.271997in}{1.375427in}}%
\pgfpathlineto{\pgfqpoint{1.272586in}{1.378343in}}%
\pgfpathlineto{\pgfqpoint{1.273618in}{1.387423in}}%
\pgfpathlineto{\pgfqpoint{1.274355in}{1.381653in}}%
\pgfpathlineto{\pgfqpoint{1.275240in}{1.371306in}}%
\pgfpathlineto{\pgfqpoint{1.275829in}{1.378944in}}%
\pgfpathlineto{\pgfqpoint{1.277156in}{1.404135in}}%
\pgfpathlineto{\pgfqpoint{1.277893in}{1.393163in}}%
\pgfpathlineto{\pgfqpoint{1.278630in}{1.384240in}}%
\pgfpathlineto{\pgfqpoint{1.279368in}{1.391087in}}%
\pgfpathlineto{\pgfqpoint{1.279957in}{1.394549in}}%
\pgfpathlineto{\pgfqpoint{1.280547in}{1.388523in}}%
\pgfpathlineto{\pgfqpoint{1.281579in}{1.376300in}}%
\pgfpathlineto{\pgfqpoint{1.282316in}{1.383428in}}%
\pgfpathlineto{\pgfqpoint{1.283790in}{1.407418in}}%
\pgfpathlineto{\pgfqpoint{1.284527in}{1.398561in}}%
\pgfpathlineto{\pgfqpoint{1.285854in}{1.369668in}}%
\pgfpathlineto{\pgfqpoint{1.286886in}{1.375087in}}%
\pgfpathlineto{\pgfqpoint{1.287623in}{1.365509in}}%
\pgfpathlineto{\pgfqpoint{1.288360in}{1.355182in}}%
\pgfpathlineto{\pgfqpoint{1.288950in}{1.363037in}}%
\pgfpathlineto{\pgfqpoint{1.290277in}{1.390566in}}%
\pgfpathlineto{\pgfqpoint{1.291014in}{1.380339in}}%
\pgfpathlineto{\pgfqpoint{1.293815in}{1.350511in}}%
\pgfpathlineto{\pgfqpoint{1.294404in}{1.351625in}}%
\pgfpathlineto{\pgfqpoint{1.295436in}{1.364190in}}%
\pgfpathlineto{\pgfqpoint{1.297058in}{1.400715in}}%
\pgfpathlineto{\pgfqpoint{1.297795in}{1.388429in}}%
\pgfpathlineto{\pgfqpoint{1.298974in}{1.363563in}}%
\pgfpathlineto{\pgfqpoint{1.299711in}{1.372361in}}%
\pgfpathlineto{\pgfqpoint{1.303249in}{1.410502in}}%
\pgfpathlineto{\pgfqpoint{1.303397in}{1.409778in}}%
\pgfpathlineto{\pgfqpoint{1.304429in}{1.388595in}}%
\pgfpathlineto{\pgfqpoint{1.305755in}{1.354612in}}%
\pgfpathlineto{\pgfqpoint{1.306493in}{1.364876in}}%
\pgfpathlineto{\pgfqpoint{1.309146in}{1.397638in}}%
\pgfpathlineto{\pgfqpoint{1.309736in}{1.402075in}}%
\pgfpathlineto{\pgfqpoint{1.310473in}{1.395454in}}%
\pgfpathlineto{\pgfqpoint{1.311947in}{1.369798in}}%
\pgfpathlineto{\pgfqpoint{1.312684in}{1.379758in}}%
\pgfpathlineto{\pgfqpoint{1.314895in}{1.406934in}}%
\pgfpathlineto{\pgfqpoint{1.315338in}{1.405931in}}%
\pgfpathlineto{\pgfqpoint{1.318433in}{1.381631in}}%
\pgfpathlineto{\pgfqpoint{1.319171in}{1.390907in}}%
\pgfpathlineto{\pgfqpoint{1.320350in}{1.411806in}}%
\pgfpathlineto{\pgfqpoint{1.321234in}{1.404327in}}%
\pgfpathlineto{\pgfqpoint{1.324330in}{1.368471in}}%
\pgfpathlineto{\pgfqpoint{1.324920in}{1.374201in}}%
\pgfpathlineto{\pgfqpoint{1.326247in}{1.390679in}}%
\pgfpathlineto{\pgfqpoint{1.327131in}{1.386727in}}%
\pgfpathlineto{\pgfqpoint{1.328016in}{1.381819in}}%
\pgfpathlineto{\pgfqpoint{1.328605in}{1.385021in}}%
\pgfpathlineto{\pgfqpoint{1.332733in}{1.428292in}}%
\pgfpathlineto{\pgfqpoint{1.333028in}{1.426964in}}%
\pgfpathlineto{\pgfqpoint{1.334355in}{1.409413in}}%
\pgfpathlineto{\pgfqpoint{1.335239in}{1.418304in}}%
\pgfpathlineto{\pgfqpoint{1.336566in}{1.434797in}}%
\pgfpathlineto{\pgfqpoint{1.337303in}{1.430827in}}%
\pgfpathlineto{\pgfqpoint{1.341431in}{1.388503in}}%
\pgfpathlineto{\pgfqpoint{1.341873in}{1.392966in}}%
\pgfpathlineto{\pgfqpoint{1.343200in}{1.419783in}}%
\pgfpathlineto{\pgfqpoint{1.343937in}{1.407543in}}%
\pgfpathlineto{\pgfqpoint{1.344527in}{1.399596in}}%
\pgfpathlineto{\pgfqpoint{1.345264in}{1.407297in}}%
\pgfpathlineto{\pgfqpoint{1.349391in}{1.451881in}}%
\pgfpathlineto{\pgfqpoint{1.349686in}{1.452489in}}%
\pgfpathlineto{\pgfqpoint{1.350128in}{1.449984in}}%
\pgfpathlineto{\pgfqpoint{1.352045in}{1.430257in}}%
\pgfpathlineto{\pgfqpoint{1.353077in}{1.434230in}}%
\pgfpathlineto{\pgfqpoint{1.353666in}{1.430330in}}%
\pgfpathlineto{\pgfqpoint{1.354404in}{1.423340in}}%
\pgfpathlineto{\pgfqpoint{1.354993in}{1.428825in}}%
\pgfpathlineto{\pgfqpoint{1.355878in}{1.439872in}}%
\pgfpathlineto{\pgfqpoint{1.356467in}{1.430645in}}%
\pgfpathlineto{\pgfqpoint{1.358089in}{1.389964in}}%
\pgfpathlineto{\pgfqpoint{1.358974in}{1.400916in}}%
\pgfpathlineto{\pgfqpoint{1.362217in}{1.441862in}}%
\pgfpathlineto{\pgfqpoint{1.362806in}{1.445424in}}%
\pgfpathlineto{\pgfqpoint{1.363396in}{1.441511in}}%
\pgfpathlineto{\pgfqpoint{1.364428in}{1.429718in}}%
\pgfpathlineto{\pgfqpoint{1.364870in}{1.436648in}}%
\pgfpathlineto{\pgfqpoint{1.366787in}{1.482161in}}%
\pgfpathlineto{\pgfqpoint{1.367819in}{1.480850in}}%
\pgfpathlineto{\pgfqpoint{1.368703in}{1.477780in}}%
\pgfpathlineto{\pgfqpoint{1.371357in}{1.451628in}}%
\pgfpathlineto{\pgfqpoint{1.372241in}{1.462980in}}%
\pgfpathlineto{\pgfqpoint{1.372831in}{1.469147in}}%
\pgfpathlineto{\pgfqpoint{1.373568in}{1.462193in}}%
\pgfpathlineto{\pgfqpoint{1.377401in}{1.400492in}}%
\pgfpathlineto{\pgfqpoint{1.377696in}{1.402968in}}%
\pgfpathlineto{\pgfqpoint{1.380349in}{1.449640in}}%
\pgfpathlineto{\pgfqpoint{1.381234in}{1.445813in}}%
\pgfpathlineto{\pgfqpoint{1.381529in}{1.444715in}}%
\pgfpathlineto{\pgfqpoint{1.382118in}{1.448775in}}%
\pgfpathlineto{\pgfqpoint{1.386099in}{1.502148in}}%
\pgfpathlineto{\pgfqpoint{1.386688in}{1.497398in}}%
\pgfpathlineto{\pgfqpoint{1.388457in}{1.470468in}}%
\pgfpathlineto{\pgfqpoint{1.389931in}{1.471381in}}%
\pgfpathlineto{\pgfqpoint{1.390668in}{1.468217in}}%
\pgfpathlineto{\pgfqpoint{1.391406in}{1.470712in}}%
\pgfpathlineto{\pgfqpoint{1.392290in}{1.474149in}}%
\pgfpathlineto{\pgfqpoint{1.393027in}{1.471207in}}%
\pgfpathlineto{\pgfqpoint{1.394501in}{1.447121in}}%
\pgfpathlineto{\pgfqpoint{1.395386in}{1.459471in}}%
\pgfpathlineto{\pgfqpoint{1.396713in}{1.474957in}}%
\pgfpathlineto{\pgfqpoint{1.397450in}{1.473730in}}%
\pgfpathlineto{\pgfqpoint{1.399514in}{1.471442in}}%
\pgfpathlineto{\pgfqpoint{1.401283in}{1.447308in}}%
\pgfpathlineto{\pgfqpoint{1.402315in}{1.459103in}}%
\pgfpathlineto{\pgfqpoint{1.403199in}{1.466043in}}%
\pgfpathlineto{\pgfqpoint{1.403936in}{1.461210in}}%
\pgfpathlineto{\pgfqpoint{1.407179in}{1.437569in}}%
\pgfpathlineto{\pgfqpoint{1.407474in}{1.435604in}}%
\pgfpathlineto{\pgfqpoint{1.408064in}{1.440651in}}%
\pgfpathlineto{\pgfqpoint{1.409538in}{1.464046in}}%
\pgfpathlineto{\pgfqpoint{1.410275in}{1.457469in}}%
\pgfpathlineto{\pgfqpoint{1.411454in}{1.442815in}}%
\pgfpathlineto{\pgfqpoint{1.412044in}{1.448890in}}%
\pgfpathlineto{\pgfqpoint{1.414550in}{1.466138in}}%
\pgfpathlineto{\pgfqpoint{1.416319in}{1.484239in}}%
\pgfpathlineto{\pgfqpoint{1.417204in}{1.473962in}}%
\pgfpathlineto{\pgfqpoint{1.417793in}{1.468001in}}%
\pgfpathlineto{\pgfqpoint{1.418531in}{1.474614in}}%
\pgfpathlineto{\pgfqpoint{1.419563in}{1.483743in}}%
\pgfpathlineto{\pgfqpoint{1.420300in}{1.478564in}}%
\pgfpathlineto{\pgfqpoint{1.421479in}{1.471847in}}%
\pgfpathlineto{\pgfqpoint{1.422216in}{1.473829in}}%
\pgfpathlineto{\pgfqpoint{1.422658in}{1.474453in}}%
\pgfpathlineto{\pgfqpoint{1.423101in}{1.472534in}}%
\pgfpathlineto{\pgfqpoint{1.424427in}{1.459621in}}%
\pgfpathlineto{\pgfqpoint{1.425312in}{1.466570in}}%
\pgfpathlineto{\pgfqpoint{1.426196in}{1.474004in}}%
\pgfpathlineto{\pgfqpoint{1.426786in}{1.467852in}}%
\pgfpathlineto{\pgfqpoint{1.428113in}{1.446812in}}%
\pgfpathlineto{\pgfqpoint{1.428850in}{1.454997in}}%
\pgfpathlineto{\pgfqpoint{1.429440in}{1.459824in}}%
\pgfpathlineto{\pgfqpoint{1.430324in}{1.455293in}}%
\pgfpathlineto{\pgfqpoint{1.430619in}{1.454421in}}%
\pgfpathlineto{\pgfqpoint{1.431209in}{1.457669in}}%
\pgfpathlineto{\pgfqpoint{1.432388in}{1.464706in}}%
\pgfpathlineto{\pgfqpoint{1.432978in}{1.461777in}}%
\pgfpathlineto{\pgfqpoint{1.434452in}{1.445889in}}%
\pgfpathlineto{\pgfqpoint{1.435336in}{1.453207in}}%
\pgfpathlineto{\pgfqpoint{1.439317in}{1.486469in}}%
\pgfpathlineto{\pgfqpoint{1.439611in}{1.486941in}}%
\pgfpathlineto{\pgfqpoint{1.440054in}{1.484243in}}%
\pgfpathlineto{\pgfqpoint{1.440938in}{1.476951in}}%
\pgfpathlineto{\pgfqpoint{1.441528in}{1.481405in}}%
\pgfpathlineto{\pgfqpoint{1.443149in}{1.496252in}}%
\pgfpathlineto{\pgfqpoint{1.443887in}{1.495215in}}%
\pgfpathlineto{\pgfqpoint{1.444476in}{1.494954in}}%
\pgfpathlineto{\pgfqpoint{1.445066in}{1.496090in}}%
\pgfpathlineto{\pgfqpoint{1.445508in}{1.496792in}}%
\pgfpathlineto{\pgfqpoint{1.445950in}{1.495180in}}%
\pgfpathlineto{\pgfqpoint{1.447719in}{1.471913in}}%
\pgfpathlineto{\pgfqpoint{1.448751in}{1.482120in}}%
\pgfpathlineto{\pgfqpoint{1.449488in}{1.486284in}}%
\pgfpathlineto{\pgfqpoint{1.450078in}{1.482323in}}%
\pgfpathlineto{\pgfqpoint{1.453764in}{1.453500in}}%
\pgfpathlineto{\pgfqpoint{1.454058in}{1.452531in}}%
\pgfpathlineto{\pgfqpoint{1.454501in}{1.455484in}}%
\pgfpathlineto{\pgfqpoint{1.456122in}{1.475599in}}%
\pgfpathlineto{\pgfqpoint{1.456859in}{1.467731in}}%
\pgfpathlineto{\pgfqpoint{1.458039in}{1.449165in}}%
\pgfpathlineto{\pgfqpoint{1.458776in}{1.456418in}}%
\pgfpathlineto{\pgfqpoint{1.459218in}{1.459445in}}%
\pgfpathlineto{\pgfqpoint{1.460103in}{1.454392in}}%
\pgfpathlineto{\pgfqpoint{1.460692in}{1.451782in}}%
\pgfpathlineto{\pgfqpoint{1.461282in}{1.455416in}}%
\pgfpathlineto{\pgfqpoint{1.462904in}{1.472782in}}%
\pgfpathlineto{\pgfqpoint{1.463788in}{1.467434in}}%
\pgfpathlineto{\pgfqpoint{1.464378in}{1.464505in}}%
\pgfpathlineto{\pgfqpoint{1.464967in}{1.467915in}}%
\pgfpathlineto{\pgfqpoint{1.468948in}{1.498612in}}%
\pgfpathlineto{\pgfqpoint{1.469095in}{1.498454in}}%
\pgfpathlineto{\pgfqpoint{1.469832in}{1.490195in}}%
\pgfpathlineto{\pgfqpoint{1.471012in}{1.472454in}}%
\pgfpathlineto{\pgfqpoint{1.471749in}{1.478536in}}%
\pgfpathlineto{\pgfqpoint{1.472781in}{1.487969in}}%
\pgfpathlineto{\pgfqpoint{1.473665in}{1.483519in}}%
\pgfpathlineto{\pgfqpoint{1.476908in}{1.468804in}}%
\pgfpathlineto{\pgfqpoint{1.477498in}{1.467251in}}%
\pgfpathlineto{\pgfqpoint{1.477940in}{1.469381in}}%
\pgfpathlineto{\pgfqpoint{1.479414in}{1.485964in}}%
\pgfpathlineto{\pgfqpoint{1.480152in}{1.479390in}}%
\pgfpathlineto{\pgfqpoint{1.481183in}{1.468977in}}%
\pgfpathlineto{\pgfqpoint{1.481921in}{1.473935in}}%
\pgfpathlineto{\pgfqpoint{1.482658in}{1.477565in}}%
\pgfpathlineto{\pgfqpoint{1.483395in}{1.474920in}}%
\pgfpathlineto{\pgfqpoint{1.484132in}{1.471085in}}%
\pgfpathlineto{\pgfqpoint{1.484721in}{1.474471in}}%
\pgfpathlineto{\pgfqpoint{1.486048in}{1.491953in}}%
\pgfpathlineto{\pgfqpoint{1.486785in}{1.484964in}}%
\pgfpathlineto{\pgfqpoint{1.488260in}{1.470259in}}%
\pgfpathlineto{\pgfqpoint{1.488849in}{1.473435in}}%
\pgfpathlineto{\pgfqpoint{1.492240in}{1.492955in}}%
\pgfpathlineto{\pgfqpoint{1.492535in}{1.493507in}}%
\pgfpathlineto{\pgfqpoint{1.492977in}{1.491147in}}%
\pgfpathlineto{\pgfqpoint{1.494009in}{1.483210in}}%
\pgfpathlineto{\pgfqpoint{1.494746in}{1.486741in}}%
\pgfpathlineto{\pgfqpoint{1.496515in}{1.504144in}}%
\pgfpathlineto{\pgfqpoint{1.497399in}{1.500061in}}%
\pgfpathlineto{\pgfqpoint{1.497989in}{1.498088in}}%
\pgfpathlineto{\pgfqpoint{1.498726in}{1.500733in}}%
\pgfpathlineto{\pgfqpoint{1.499021in}{1.501307in}}%
\pgfpathlineto{\pgfqpoint{1.499758in}{1.499231in}}%
\pgfpathlineto{\pgfqpoint{1.500790in}{1.497448in}}%
\pgfpathlineto{\pgfqpoint{1.501380in}{1.498421in}}%
\pgfpathlineto{\pgfqpoint{1.502707in}{1.511496in}}%
\pgfpathlineto{\pgfqpoint{1.503149in}{1.513905in}}%
\pgfpathlineto{\pgfqpoint{1.503886in}{1.509546in}}%
\pgfpathlineto{\pgfqpoint{1.506982in}{1.493851in}}%
\pgfpathlineto{\pgfqpoint{1.508014in}{1.493451in}}%
\pgfpathlineto{\pgfqpoint{1.508308in}{1.494092in}}%
\pgfpathlineto{\pgfqpoint{1.509635in}{1.501583in}}%
\pgfpathlineto{\pgfqpoint{1.510372in}{1.496600in}}%
\pgfpathlineto{\pgfqpoint{1.511846in}{1.481381in}}%
\pgfpathlineto{\pgfqpoint{1.512584in}{1.486213in}}%
\pgfpathlineto{\pgfqpoint{1.516269in}{1.515327in}}%
\pgfpathlineto{\pgfqpoint{1.516859in}{1.514530in}}%
\pgfpathlineto{\pgfqpoint{1.518480in}{1.505801in}}%
\pgfpathlineto{\pgfqpoint{1.519217in}{1.509936in}}%
\pgfpathlineto{\pgfqpoint{1.521576in}{1.517815in}}%
\pgfpathlineto{\pgfqpoint{1.522461in}{1.520179in}}%
\pgfpathlineto{\pgfqpoint{1.523050in}{1.518167in}}%
\pgfpathlineto{\pgfqpoint{1.524967in}{1.490784in}}%
\pgfpathlineto{\pgfqpoint{1.526146in}{1.501512in}}%
\pgfpathlineto{\pgfqpoint{1.527915in}{1.510694in}}%
\pgfpathlineto{\pgfqpoint{1.528357in}{1.509321in}}%
\pgfpathlineto{\pgfqpoint{1.530863in}{1.494264in}}%
\pgfpathlineto{\pgfqpoint{1.531453in}{1.497683in}}%
\pgfpathlineto{\pgfqpoint{1.534107in}{1.523641in}}%
\pgfpathlineto{\pgfqpoint{1.534549in}{1.522864in}}%
\pgfpathlineto{\pgfqpoint{1.537497in}{1.515829in}}%
\pgfpathlineto{\pgfqpoint{1.537645in}{1.516039in}}%
\pgfpathlineto{\pgfqpoint{1.538677in}{1.525856in}}%
\pgfpathlineto{\pgfqpoint{1.539119in}{1.528285in}}%
\pgfpathlineto{\pgfqpoint{1.539856in}{1.524526in}}%
\pgfpathlineto{\pgfqpoint{1.543984in}{1.494368in}}%
\pgfpathlineto{\pgfqpoint{1.544279in}{1.494931in}}%
\pgfpathlineto{\pgfqpoint{1.545900in}{1.507454in}}%
\pgfpathlineto{\pgfqpoint{1.546785in}{1.502405in}}%
\pgfpathlineto{\pgfqpoint{1.548111in}{1.490651in}}%
\pgfpathlineto{\pgfqpoint{1.548849in}{1.496543in}}%
\pgfpathlineto{\pgfqpoint{1.552681in}{1.532386in}}%
\pgfpathlineto{\pgfqpoint{1.552829in}{1.532195in}}%
\pgfpathlineto{\pgfqpoint{1.554008in}{1.526257in}}%
\pgfpathlineto{\pgfqpoint{1.554450in}{1.524687in}}%
\pgfpathlineto{\pgfqpoint{1.555040in}{1.527864in}}%
\pgfpathlineto{\pgfqpoint{1.556514in}{1.542090in}}%
\pgfpathlineto{\pgfqpoint{1.557546in}{1.540491in}}%
\pgfpathlineto{\pgfqpoint{1.558873in}{1.538182in}}%
\pgfpathlineto{\pgfqpoint{1.561379in}{1.517280in}}%
\pgfpathlineto{\pgfqpoint{1.562706in}{1.520759in}}%
\pgfpathlineto{\pgfqpoint{1.563148in}{1.520237in}}%
\pgfpathlineto{\pgfqpoint{1.563296in}{1.519472in}}%
\pgfpathlineto{\pgfqpoint{1.566686in}{1.498358in}}%
\pgfpathlineto{\pgfqpoint{1.567423in}{1.500234in}}%
\pgfpathlineto{\pgfqpoint{1.569045in}{1.518592in}}%
\pgfpathlineto{\pgfqpoint{1.570077in}{1.527702in}}%
\pgfpathlineto{\pgfqpoint{1.570814in}{1.522922in}}%
\pgfpathlineto{\pgfqpoint{1.571551in}{1.519733in}}%
\pgfpathlineto{\pgfqpoint{1.572435in}{1.521543in}}%
\pgfpathlineto{\pgfqpoint{1.573762in}{1.524276in}}%
\pgfpathlineto{\pgfqpoint{1.576268in}{1.540542in}}%
\pgfpathlineto{\pgfqpoint{1.577005in}{1.535849in}}%
\pgfpathlineto{\pgfqpoint{1.578185in}{1.526765in}}%
\pgfpathlineto{\pgfqpoint{1.578922in}{1.529461in}}%
\pgfpathlineto{\pgfqpoint{1.582313in}{1.545707in}}%
\pgfpathlineto{\pgfqpoint{1.582607in}{1.546052in}}%
\pgfpathlineto{\pgfqpoint{1.582902in}{1.544610in}}%
\pgfpathlineto{\pgfqpoint{1.584376in}{1.525637in}}%
\pgfpathlineto{\pgfqpoint{1.585408in}{1.533844in}}%
\pgfpathlineto{\pgfqpoint{1.587177in}{1.548600in}}%
\pgfpathlineto{\pgfqpoint{1.587914in}{1.545330in}}%
\pgfpathlineto{\pgfqpoint{1.590863in}{1.530642in}}%
\pgfpathlineto{\pgfqpoint{1.591158in}{1.531585in}}%
\pgfpathlineto{\pgfqpoint{1.593221in}{1.553565in}}%
\pgfpathlineto{\pgfqpoint{1.594106in}{1.543656in}}%
\pgfpathlineto{\pgfqpoint{1.596022in}{1.523333in}}%
\pgfpathlineto{\pgfqpoint{1.596612in}{1.524065in}}%
\pgfpathlineto{\pgfqpoint{1.598234in}{1.531063in}}%
\pgfpathlineto{\pgfqpoint{1.599560in}{1.538934in}}%
\pgfpathlineto{\pgfqpoint{1.600150in}{1.534562in}}%
\pgfpathlineto{\pgfqpoint{1.601624in}{1.514500in}}%
\pgfpathlineto{\pgfqpoint{1.602361in}{1.522643in}}%
\pgfpathlineto{\pgfqpoint{1.605457in}{1.551277in}}%
\pgfpathlineto{\pgfqpoint{1.606047in}{1.552658in}}%
\pgfpathlineto{\pgfqpoint{1.606784in}{1.551229in}}%
\pgfpathlineto{\pgfqpoint{1.608111in}{1.545299in}}%
\pgfpathlineto{\pgfqpoint{1.608700in}{1.549384in}}%
\pgfpathlineto{\pgfqpoint{1.610322in}{1.572696in}}%
\pgfpathlineto{\pgfqpoint{1.611207in}{1.568051in}}%
\pgfpathlineto{\pgfqpoint{1.614745in}{1.538877in}}%
\pgfpathlineto{\pgfqpoint{1.615629in}{1.547365in}}%
\pgfpathlineto{\pgfqpoint{1.616514in}{1.555331in}}%
\pgfpathlineto{\pgfqpoint{1.617251in}{1.550373in}}%
\pgfpathlineto{\pgfqpoint{1.619462in}{1.516541in}}%
\pgfpathlineto{\pgfqpoint{1.620936in}{1.519489in}}%
\pgfpathlineto{\pgfqpoint{1.621378in}{1.520408in}}%
\pgfpathlineto{\pgfqpoint{1.622705in}{1.544884in}}%
\pgfpathlineto{\pgfqpoint{1.623295in}{1.549224in}}%
\pgfpathlineto{\pgfqpoint{1.624032in}{1.542474in}}%
\pgfpathlineto{\pgfqpoint{1.625506in}{1.528479in}}%
\pgfpathlineto{\pgfqpoint{1.626096in}{1.532916in}}%
\pgfpathlineto{\pgfqpoint{1.629929in}{1.570845in}}%
\pgfpathlineto{\pgfqpoint{1.630371in}{1.569559in}}%
\pgfpathlineto{\pgfqpoint{1.631993in}{1.546885in}}%
\pgfpathlineto{\pgfqpoint{1.633024in}{1.558719in}}%
\pgfpathlineto{\pgfqpoint{1.634351in}{1.568741in}}%
\pgfpathlineto{\pgfqpoint{1.634941in}{1.567196in}}%
\pgfpathlineto{\pgfqpoint{1.638774in}{1.536334in}}%
\pgfpathlineto{\pgfqpoint{1.639658in}{1.547191in}}%
\pgfpathlineto{\pgfqpoint{1.640690in}{1.561427in}}%
\pgfpathlineto{\pgfqpoint{1.641427in}{1.553049in}}%
\pgfpathlineto{\pgfqpoint{1.643933in}{1.534536in}}%
\pgfpathlineto{\pgfqpoint{1.644376in}{1.535730in}}%
\pgfpathlineto{\pgfqpoint{1.647324in}{1.564669in}}%
\pgfpathlineto{\pgfqpoint{1.648061in}{1.556108in}}%
\pgfpathlineto{\pgfqpoint{1.649240in}{1.534322in}}%
\pgfpathlineto{\pgfqpoint{1.650125in}{1.543433in}}%
\pgfpathlineto{\pgfqpoint{1.652926in}{1.562549in}}%
\pgfpathlineto{\pgfqpoint{1.653368in}{1.563653in}}%
\pgfpathlineto{\pgfqpoint{1.653810in}{1.560845in}}%
\pgfpathlineto{\pgfqpoint{1.655727in}{1.536047in}}%
\pgfpathlineto{\pgfqpoint{1.656611in}{1.542974in}}%
\pgfpathlineto{\pgfqpoint{1.658233in}{1.564863in}}%
\pgfpathlineto{\pgfqpoint{1.659265in}{1.561144in}}%
\pgfpathlineto{\pgfqpoint{1.660149in}{1.564488in}}%
\pgfpathlineto{\pgfqpoint{1.660739in}{1.560736in}}%
\pgfpathlineto{\pgfqpoint{1.661771in}{1.548596in}}%
\pgfpathlineto{\pgfqpoint{1.662508in}{1.554254in}}%
\pgfpathlineto{\pgfqpoint{1.664425in}{1.577010in}}%
\pgfpathlineto{\pgfqpoint{1.665162in}{1.572865in}}%
\pgfpathlineto{\pgfqpoint{1.668552in}{1.549372in}}%
\pgfpathlineto{\pgfqpoint{1.668847in}{1.549025in}}%
\pgfpathlineto{\pgfqpoint{1.669289in}{1.550791in}}%
\pgfpathlineto{\pgfqpoint{1.670764in}{1.568055in}}%
\pgfpathlineto{\pgfqpoint{1.671501in}{1.558577in}}%
\pgfpathlineto{\pgfqpoint{1.674007in}{1.535283in}}%
\pgfpathlineto{\pgfqpoint{1.674596in}{1.538337in}}%
\pgfpathlineto{\pgfqpoint{1.677250in}{1.562342in}}%
\pgfpathlineto{\pgfqpoint{1.677987in}{1.558067in}}%
\pgfpathlineto{\pgfqpoint{1.679314in}{1.543518in}}%
\pgfpathlineto{\pgfqpoint{1.680051in}{1.550467in}}%
\pgfpathlineto{\pgfqpoint{1.683147in}{1.582374in}}%
\pgfpathlineto{\pgfqpoint{1.683294in}{1.582534in}}%
\pgfpathlineto{\pgfqpoint{1.683736in}{1.580748in}}%
\pgfpathlineto{\pgfqpoint{1.685653in}{1.565227in}}%
\pgfpathlineto{\pgfqpoint{1.686390in}{1.568396in}}%
\pgfpathlineto{\pgfqpoint{1.688159in}{1.585289in}}%
\pgfpathlineto{\pgfqpoint{1.688896in}{1.579974in}}%
\pgfpathlineto{\pgfqpoint{1.691992in}{1.546680in}}%
\pgfpathlineto{\pgfqpoint{1.692582in}{1.551302in}}%
\pgfpathlineto{\pgfqpoint{1.694056in}{1.574186in}}%
\pgfpathlineto{\pgfqpoint{1.695088in}{1.570150in}}%
\pgfpathlineto{\pgfqpoint{1.696120in}{1.564250in}}%
\pgfpathlineto{\pgfqpoint{1.697299in}{1.552829in}}%
\pgfpathlineto{\pgfqpoint{1.698036in}{1.557468in}}%
\pgfpathlineto{\pgfqpoint{1.700690in}{1.580956in}}%
\pgfpathlineto{\pgfqpoint{1.701574in}{1.571380in}}%
\pgfpathlineto{\pgfqpoint{1.703933in}{1.553739in}}%
\pgfpathlineto{\pgfqpoint{1.704080in}{1.553856in}}%
\pgfpathlineto{\pgfqpoint{1.704965in}{1.557778in}}%
\pgfpathlineto{\pgfqpoint{1.706439in}{1.570421in}}%
\pgfpathlineto{\pgfqpoint{1.707176in}{1.566560in}}%
\pgfpathlineto{\pgfqpoint{1.709535in}{1.540691in}}%
\pgfpathlineto{\pgfqpoint{1.710419in}{1.551595in}}%
\pgfpathlineto{\pgfqpoint{1.712925in}{1.571938in}}%
\pgfpathlineto{\pgfqpoint{1.713220in}{1.573027in}}%
\pgfpathlineto{\pgfqpoint{1.713810in}{1.569071in}}%
\pgfpathlineto{\pgfqpoint{1.715579in}{1.551934in}}%
\pgfpathlineto{\pgfqpoint{1.716316in}{1.555504in}}%
\pgfpathlineto{\pgfqpoint{1.718527in}{1.581381in}}%
\pgfpathlineto{\pgfqpoint{1.719412in}{1.576269in}}%
\pgfpathlineto{\pgfqpoint{1.722065in}{1.554883in}}%
\pgfpathlineto{\pgfqpoint{1.722655in}{1.559784in}}%
\pgfpathlineto{\pgfqpoint{1.724129in}{1.581608in}}%
\pgfpathlineto{\pgfqpoint{1.725014in}{1.577317in}}%
\pgfpathlineto{\pgfqpoint{1.727667in}{1.552399in}}%
\pgfpathlineto{\pgfqpoint{1.728699in}{1.555039in}}%
\pgfpathlineto{\pgfqpoint{1.730321in}{1.576033in}}%
\pgfpathlineto{\pgfqpoint{1.730910in}{1.579063in}}%
\pgfpathlineto{\pgfqpoint{1.731500in}{1.573517in}}%
\pgfpathlineto{\pgfqpoint{1.732974in}{1.551630in}}%
\pgfpathlineto{\pgfqpoint{1.733859in}{1.557919in}}%
\pgfpathlineto{\pgfqpoint{1.736807in}{1.579851in}}%
\pgfpathlineto{\pgfqpoint{1.737692in}{1.575488in}}%
\pgfpathlineto{\pgfqpoint{1.739461in}{1.555274in}}%
\pgfpathlineto{\pgfqpoint{1.740198in}{1.562224in}}%
\pgfpathlineto{\pgfqpoint{1.742114in}{1.582678in}}%
\pgfpathlineto{\pgfqpoint{1.742851in}{1.581838in}}%
\pgfpathlineto{\pgfqpoint{1.743883in}{1.583013in}}%
\pgfpathlineto{\pgfqpoint{1.744178in}{1.581515in}}%
\pgfpathlineto{\pgfqpoint{1.745505in}{1.565339in}}%
\pgfpathlineto{\pgfqpoint{1.746389in}{1.574384in}}%
\pgfpathlineto{\pgfqpoint{1.748601in}{1.595820in}}%
\pgfpathlineto{\pgfqpoint{1.749043in}{1.593824in}}%
\pgfpathlineto{\pgfqpoint{1.752286in}{1.570176in}}%
\pgfpathlineto{\pgfqpoint{1.752728in}{1.572263in}}%
\pgfpathlineto{\pgfqpoint{1.754202in}{1.589201in}}%
\pgfpathlineto{\pgfqpoint{1.755087in}{1.582898in}}%
\pgfpathlineto{\pgfqpoint{1.757888in}{1.554293in}}%
\pgfpathlineto{\pgfqpoint{1.758330in}{1.555834in}}%
\pgfpathlineto{\pgfqpoint{1.760984in}{1.577226in}}%
\pgfpathlineto{\pgfqpoint{1.761573in}{1.571960in}}%
\pgfpathlineto{\pgfqpoint{1.763048in}{1.547762in}}%
\pgfpathlineto{\pgfqpoint{1.763785in}{1.556189in}}%
\pgfpathlineto{\pgfqpoint{1.766880in}{1.590480in}}%
\pgfpathlineto{\pgfqpoint{1.767323in}{1.589141in}}%
\pgfpathlineto{\pgfqpoint{1.769387in}{1.570752in}}%
\pgfpathlineto{\pgfqpoint{1.770271in}{1.580166in}}%
\pgfpathlineto{\pgfqpoint{1.771893in}{1.603411in}}%
\pgfpathlineto{\pgfqpoint{1.772630in}{1.598547in}}%
\pgfpathlineto{\pgfqpoint{1.775873in}{1.569957in}}%
\pgfpathlineto{\pgfqpoint{1.776610in}{1.574926in}}%
\pgfpathlineto{\pgfqpoint{1.777937in}{1.588213in}}%
\pgfpathlineto{\pgfqpoint{1.778674in}{1.583342in}}%
\pgfpathlineto{\pgfqpoint{1.781770in}{1.552237in}}%
\pgfpathlineto{\pgfqpoint{1.782359in}{1.552975in}}%
\pgfpathlineto{\pgfqpoint{1.783244in}{1.560686in}}%
\pgfpathlineto{\pgfqpoint{1.784571in}{1.576541in}}%
\pgfpathlineto{\pgfqpoint{1.785308in}{1.569699in}}%
\pgfpathlineto{\pgfqpoint{1.786782in}{1.556952in}}%
\pgfpathlineto{\pgfqpoint{1.787519in}{1.559409in}}%
\pgfpathlineto{\pgfqpoint{1.790762in}{1.589895in}}%
\pgfpathlineto{\pgfqpoint{1.791794in}{1.583604in}}%
\pgfpathlineto{\pgfqpoint{1.793268in}{1.567964in}}%
\pgfpathlineto{\pgfqpoint{1.793858in}{1.573665in}}%
\pgfpathlineto{\pgfqpoint{1.796659in}{1.598662in}}%
\pgfpathlineto{\pgfqpoint{1.797101in}{1.597609in}}%
\pgfpathlineto{\pgfqpoint{1.799313in}{1.578158in}}%
\pgfpathlineto{\pgfqpoint{1.800197in}{1.582934in}}%
\pgfpathlineto{\pgfqpoint{1.801819in}{1.602417in}}%
\pgfpathlineto{\pgfqpoint{1.802703in}{1.596439in}}%
\pgfpathlineto{\pgfqpoint{1.805652in}{1.572792in}}%
\pgfpathlineto{\pgfqpoint{1.806094in}{1.574155in}}%
\pgfpathlineto{\pgfqpoint{1.807863in}{1.587746in}}%
\pgfpathlineto{\pgfqpoint{1.808600in}{1.583963in}}%
\pgfpathlineto{\pgfqpoint{1.810811in}{1.555649in}}%
\pgfpathlineto{\pgfqpoint{1.812285in}{1.562795in}}%
\pgfpathlineto{\pgfqpoint{1.813465in}{1.572843in}}%
\pgfpathlineto{\pgfqpoint{1.814349in}{1.579081in}}%
\pgfpathlineto{\pgfqpoint{1.815086in}{1.573888in}}%
\pgfpathlineto{\pgfqpoint{1.816560in}{1.562270in}}%
\pgfpathlineto{\pgfqpoint{1.817298in}{1.565900in}}%
\pgfpathlineto{\pgfqpoint{1.819951in}{1.595118in}}%
\pgfpathlineto{\pgfqpoint{1.821278in}{1.590673in}}%
\pgfpathlineto{\pgfqpoint{1.823047in}{1.578476in}}%
\pgfpathlineto{\pgfqpoint{1.823784in}{1.585656in}}%
\pgfpathlineto{\pgfqpoint{1.825995in}{1.600887in}}%
\pgfpathlineto{\pgfqpoint{1.826143in}{1.600687in}}%
\pgfpathlineto{\pgfqpoint{1.827027in}{1.593275in}}%
\pgfpathlineto{\pgfqpoint{1.829238in}{1.570083in}}%
\pgfpathlineto{\pgfqpoint{1.829828in}{1.571853in}}%
\pgfpathlineto{\pgfqpoint{1.831892in}{1.590557in}}%
\pgfpathlineto{\pgfqpoint{1.832629in}{1.584190in}}%
\pgfpathlineto{\pgfqpoint{1.834546in}{1.554298in}}%
\pgfpathlineto{\pgfqpoint{1.835577in}{1.558354in}}%
\pgfpathlineto{\pgfqpoint{1.837789in}{1.589597in}}%
\pgfpathlineto{\pgfqpoint{1.839115in}{1.581611in}}%
\pgfpathlineto{\pgfqpoint{1.840590in}{1.564734in}}%
\pgfpathlineto{\pgfqpoint{1.841327in}{1.571157in}}%
\pgfpathlineto{\pgfqpoint{1.843980in}{1.598277in}}%
\pgfpathlineto{\pgfqpoint{1.844275in}{1.598026in}}%
\pgfpathlineto{\pgfqpoint{1.845160in}{1.591361in}}%
\pgfpathlineto{\pgfqpoint{1.846634in}{1.578108in}}%
\pgfpathlineto{\pgfqpoint{1.847371in}{1.581761in}}%
\pgfpathlineto{\pgfqpoint{1.849877in}{1.601691in}}%
\pgfpathlineto{\pgfqpoint{1.850762in}{1.596715in}}%
\pgfpathlineto{\pgfqpoint{1.853120in}{1.575826in}}%
\pgfpathlineto{\pgfqpoint{1.853857in}{1.581309in}}%
\pgfpathlineto{\pgfqpoint{1.855479in}{1.598165in}}%
\pgfpathlineto{\pgfqpoint{1.856216in}{1.595921in}}%
\pgfpathlineto{\pgfqpoint{1.857690in}{1.584266in}}%
\pgfpathlineto{\pgfqpoint{1.858870in}{1.571929in}}%
\pgfpathlineto{\pgfqpoint{1.859607in}{1.576719in}}%
\pgfpathlineto{\pgfqpoint{1.861965in}{1.597463in}}%
\pgfpathlineto{\pgfqpoint{1.862702in}{1.590375in}}%
\pgfpathlineto{\pgfqpoint{1.865356in}{1.569961in}}%
\pgfpathlineto{\pgfqpoint{1.865651in}{1.570107in}}%
\pgfpathlineto{\pgfqpoint{1.865798in}{1.570580in}}%
\pgfpathlineto{\pgfqpoint{1.867125in}{1.587336in}}%
\pgfpathlineto{\pgfqpoint{1.868010in}{1.593635in}}%
\pgfpathlineto{\pgfqpoint{1.868894in}{1.590195in}}%
\pgfpathlineto{\pgfqpoint{1.870958in}{1.572319in}}%
\pgfpathlineto{\pgfqpoint{1.871695in}{1.582202in}}%
\pgfpathlineto{\pgfqpoint{1.874496in}{1.608582in}}%
\pgfpathlineto{\pgfqpoint{1.874643in}{1.609079in}}%
\pgfpathlineto{\pgfqpoint{1.875086in}{1.606599in}}%
\pgfpathlineto{\pgfqpoint{1.876855in}{1.583472in}}%
\pgfpathlineto{\pgfqpoint{1.877887in}{1.587734in}}%
\pgfpathlineto{\pgfqpoint{1.879656in}{1.601396in}}%
\pgfpathlineto{\pgfqpoint{1.880393in}{1.596585in}}%
\pgfpathlineto{\pgfqpoint{1.883488in}{1.566688in}}%
\pgfpathlineto{\pgfqpoint{1.884078in}{1.573592in}}%
\pgfpathlineto{\pgfqpoint{1.885405in}{1.591976in}}%
\pgfpathlineto{\pgfqpoint{1.886289in}{1.588043in}}%
\pgfpathlineto{\pgfqpoint{1.887764in}{1.579938in}}%
\pgfpathlineto{\pgfqpoint{1.888796in}{1.572661in}}%
\pgfpathlineto{\pgfqpoint{1.889533in}{1.577113in}}%
\pgfpathlineto{\pgfqpoint{1.892334in}{1.603394in}}%
\pgfpathlineto{\pgfqpoint{1.893071in}{1.596457in}}%
\pgfpathlineto{\pgfqpoint{1.894103in}{1.583873in}}%
\pgfpathlineto{\pgfqpoint{1.894987in}{1.588002in}}%
\pgfpathlineto{\pgfqpoint{1.895282in}{1.588756in}}%
\pgfpathlineto{\pgfqpoint{1.896019in}{1.586527in}}%
\pgfpathlineto{\pgfqpoint{1.896166in}{1.586376in}}%
\pgfpathlineto{\pgfqpoint{1.896461in}{1.587616in}}%
\pgfpathlineto{\pgfqpoint{1.897788in}{1.602133in}}%
\pgfpathlineto{\pgfqpoint{1.898525in}{1.595781in}}%
\pgfpathlineto{\pgfqpoint{1.901179in}{1.569515in}}%
\pgfpathlineto{\pgfqpoint{1.901621in}{1.571710in}}%
\pgfpathlineto{\pgfqpoint{1.904864in}{1.594399in}}%
\pgfpathlineto{\pgfqpoint{1.905159in}{1.595233in}}%
\pgfpathlineto{\pgfqpoint{1.905601in}{1.593179in}}%
\pgfpathlineto{\pgfqpoint{1.906781in}{1.577632in}}%
\pgfpathlineto{\pgfqpoint{1.907518in}{1.586412in}}%
\pgfpathlineto{\pgfqpoint{1.910171in}{1.612027in}}%
\pgfpathlineto{\pgfqpoint{1.910319in}{1.611928in}}%
\pgfpathlineto{\pgfqpoint{1.911203in}{1.601746in}}%
\pgfpathlineto{\pgfqpoint{1.913709in}{1.588366in}}%
\pgfpathlineto{\pgfqpoint{1.913857in}{1.588088in}}%
\pgfpathlineto{\pgfqpoint{1.914152in}{1.589664in}}%
\pgfpathlineto{\pgfqpoint{1.915626in}{1.610319in}}%
\pgfpathlineto{\pgfqpoint{1.916363in}{1.602370in}}%
\pgfpathlineto{\pgfqpoint{1.919311in}{1.567253in}}%
\pgfpathlineto{\pgfqpoint{1.919606in}{1.568680in}}%
\pgfpathlineto{\pgfqpoint{1.922554in}{1.593194in}}%
\pgfpathlineto{\pgfqpoint{1.922997in}{1.589908in}}%
\pgfpathlineto{\pgfqpoint{1.924176in}{1.566040in}}%
\pgfpathlineto{\pgfqpoint{1.925060in}{1.579604in}}%
\pgfpathlineto{\pgfqpoint{1.928156in}{1.613968in}}%
\pgfpathlineto{\pgfqpoint{1.928893in}{1.608355in}}%
\pgfpathlineto{\pgfqpoint{1.931252in}{1.588252in}}%
\pgfpathlineto{\pgfqpoint{1.931694in}{1.590244in}}%
\pgfpathlineto{\pgfqpoint{1.933168in}{1.609425in}}%
\pgfpathlineto{\pgfqpoint{1.934200in}{1.603076in}}%
\pgfpathlineto{\pgfqpoint{1.935969in}{1.592209in}}%
\pgfpathlineto{\pgfqpoint{1.937001in}{1.574319in}}%
\pgfpathlineto{\pgfqpoint{1.937738in}{1.584524in}}%
\pgfpathlineto{\pgfqpoint{1.940392in}{1.613255in}}%
\pgfpathlineto{\pgfqpoint{1.940687in}{1.612295in}}%
\pgfpathlineto{\pgfqpoint{1.943635in}{1.593603in}}%
\pgfpathlineto{\pgfqpoint{1.944225in}{1.597078in}}%
\pgfpathlineto{\pgfqpoint{1.945846in}{1.626366in}}%
\pgfpathlineto{\pgfqpoint{1.946584in}{1.614364in}}%
\pgfpathlineto{\pgfqpoint{1.949385in}{1.576458in}}%
\pgfpathlineto{\pgfqpoint{1.949532in}{1.576635in}}%
\pgfpathlineto{\pgfqpoint{1.950564in}{1.592536in}}%
\pgfpathlineto{\pgfqpoint{1.952333in}{1.606416in}}%
\pgfpathlineto{\pgfqpoint{1.952775in}{1.604188in}}%
\pgfpathlineto{\pgfqpoint{1.954544in}{1.567809in}}%
\pgfpathlineto{\pgfqpoint{1.955429in}{1.585350in}}%
\pgfpathlineto{\pgfqpoint{1.958230in}{1.626316in}}%
\pgfpathlineto{\pgfqpoint{1.958524in}{1.627475in}}%
\pgfpathlineto{\pgfqpoint{1.958967in}{1.623151in}}%
\pgfpathlineto{\pgfqpoint{1.960441in}{1.603650in}}%
\pgfpathlineto{\pgfqpoint{1.961473in}{1.605096in}}%
\pgfpathlineto{\pgfqpoint{1.962357in}{1.619364in}}%
\pgfpathlineto{\pgfqpoint{1.963389in}{1.637253in}}%
\pgfpathlineto{\pgfqpoint{1.964126in}{1.630150in}}%
\pgfpathlineto{\pgfqpoint{1.967370in}{1.584439in}}%
\pgfpathlineto{\pgfqpoint{1.967812in}{1.590125in}}%
\pgfpathlineto{\pgfqpoint{1.969139in}{1.613746in}}%
\pgfpathlineto{\pgfqpoint{1.970023in}{1.609422in}}%
\pgfpathlineto{\pgfqpoint{1.971202in}{1.596321in}}%
\pgfpathlineto{\pgfqpoint{1.972529in}{1.567937in}}%
\pgfpathlineto{\pgfqpoint{1.973414in}{1.576358in}}%
\pgfpathlineto{\pgfqpoint{1.975183in}{1.597175in}}%
\pgfpathlineto{\pgfqpoint{1.976067in}{1.614702in}}%
\pgfpathlineto{\pgfqpoint{1.976804in}{1.605589in}}%
\pgfpathlineto{\pgfqpoint{1.979163in}{1.586492in}}%
\pgfpathlineto{\pgfqpoint{1.979605in}{1.587557in}}%
\pgfpathlineto{\pgfqpoint{1.981669in}{1.615090in}}%
\pgfpathlineto{\pgfqpoint{1.983586in}{1.610032in}}%
\pgfpathlineto{\pgfqpoint{1.984912in}{1.585489in}}%
\pgfpathlineto{\pgfqpoint{1.985649in}{1.596708in}}%
\pgfpathlineto{\pgfqpoint{1.987861in}{1.611294in}}%
\pgfpathlineto{\pgfqpoint{1.988598in}{1.615615in}}%
\pgfpathlineto{\pgfqpoint{1.989188in}{1.610327in}}%
\pgfpathlineto{\pgfqpoint{1.991104in}{1.589131in}}%
\pgfpathlineto{\pgfqpoint{1.991841in}{1.591445in}}%
\pgfpathlineto{\pgfqpoint{1.993610in}{1.613068in}}%
\pgfpathlineto{\pgfqpoint{1.994642in}{1.600957in}}%
\pgfpathlineto{\pgfqpoint{1.997590in}{1.571344in}}%
\pgfpathlineto{\pgfqpoint{1.997738in}{1.572150in}}%
\pgfpathlineto{\pgfqpoint{1.999507in}{1.599768in}}%
\pgfpathlineto{\pgfqpoint{2.000686in}{1.592936in}}%
\pgfpathlineto{\pgfqpoint{2.002750in}{1.570901in}}%
\pgfpathlineto{\pgfqpoint{2.003782in}{1.578532in}}%
\pgfpathlineto{\pgfqpoint{2.006435in}{1.611322in}}%
\pgfpathlineto{\pgfqpoint{2.007173in}{1.599251in}}%
\pgfpathlineto{\pgfqpoint{2.008057in}{1.584015in}}%
\pgfpathlineto{\pgfqpoint{2.008942in}{1.591409in}}%
\pgfpathlineto{\pgfqpoint{2.011743in}{1.615869in}}%
\pgfpathlineto{\pgfqpoint{2.013069in}{1.609176in}}%
\pgfpathlineto{\pgfqpoint{2.014543in}{1.588515in}}%
\pgfpathlineto{\pgfqpoint{2.014986in}{1.584505in}}%
\pgfpathlineto{\pgfqpoint{2.015723in}{1.592138in}}%
\pgfpathlineto{\pgfqpoint{2.016902in}{1.608578in}}%
\pgfpathlineto{\pgfqpoint{2.017787in}{1.605136in}}%
\pgfpathlineto{\pgfqpoint{2.018819in}{1.606627in}}%
\pgfpathlineto{\pgfqpoint{2.019556in}{1.597965in}}%
\pgfpathlineto{\pgfqpoint{2.020735in}{1.573477in}}%
\pgfpathlineto{\pgfqpoint{2.021620in}{1.580686in}}%
\pgfpathlineto{\pgfqpoint{2.023831in}{1.602551in}}%
\pgfpathlineto{\pgfqpoint{2.024421in}{1.595752in}}%
\pgfpathlineto{\pgfqpoint{2.027074in}{1.571210in}}%
\pgfpathlineto{\pgfqpoint{2.027664in}{1.566851in}}%
\pgfpathlineto{\pgfqpoint{2.028106in}{1.572372in}}%
\pgfpathlineto{\pgfqpoint{2.029580in}{1.604231in}}%
\pgfpathlineto{\pgfqpoint{2.030465in}{1.596321in}}%
\pgfpathlineto{\pgfqpoint{2.032676in}{1.578157in}}%
\pgfpathlineto{\pgfqpoint{2.033118in}{1.581361in}}%
\pgfpathlineto{\pgfqpoint{2.036361in}{1.613327in}}%
\pgfpathlineto{\pgfqpoint{2.036509in}{1.613234in}}%
\pgfpathlineto{\pgfqpoint{2.037246in}{1.602797in}}%
\pgfpathlineto{\pgfqpoint{2.038425in}{1.583583in}}%
\pgfpathlineto{\pgfqpoint{2.039162in}{1.590451in}}%
\pgfpathlineto{\pgfqpoint{2.041816in}{1.605440in}}%
\pgfpathlineto{\pgfqpoint{2.042111in}{1.605115in}}%
\pgfpathlineto{\pgfqpoint{2.042258in}{1.604348in}}%
\pgfpathlineto{\pgfqpoint{2.045207in}{1.572680in}}%
\pgfpathlineto{\pgfqpoint{2.045944in}{1.584298in}}%
\pgfpathlineto{\pgfqpoint{2.047123in}{1.603580in}}%
\pgfpathlineto{\pgfqpoint{2.047860in}{1.596911in}}%
\pgfpathlineto{\pgfqpoint{2.050956in}{1.578481in}}%
\pgfpathlineto{\pgfqpoint{2.051103in}{1.578705in}}%
\pgfpathlineto{\pgfqpoint{2.051988in}{1.591680in}}%
\pgfpathlineto{\pgfqpoint{2.053462in}{1.609110in}}%
\pgfpathlineto{\pgfqpoint{2.054199in}{1.607569in}}%
\pgfpathlineto{\pgfqpoint{2.055231in}{1.595659in}}%
\pgfpathlineto{\pgfqpoint{2.056263in}{1.580771in}}%
\pgfpathlineto{\pgfqpoint{2.057295in}{1.583983in}}%
\pgfpathlineto{\pgfqpoint{2.058032in}{1.581555in}}%
\pgfpathlineto{\pgfqpoint{2.058474in}{1.585109in}}%
\pgfpathlineto{\pgfqpoint{2.059801in}{1.606181in}}%
\pgfpathlineto{\pgfqpoint{2.060538in}{1.595987in}}%
\pgfpathlineto{\pgfqpoint{2.062897in}{1.574168in}}%
\pgfpathlineto{\pgfqpoint{2.063044in}{1.574406in}}%
\pgfpathlineto{\pgfqpoint{2.063929in}{1.584265in}}%
\pgfpathlineto{\pgfqpoint{2.065993in}{1.610543in}}%
\pgfpathlineto{\pgfqpoint{2.066730in}{1.608703in}}%
\pgfpathlineto{\pgfqpoint{2.067762in}{1.596688in}}%
\pgfpathlineto{\pgfqpoint{2.068646in}{1.584758in}}%
\pgfpathlineto{\pgfqpoint{2.069236in}{1.592876in}}%
\pgfpathlineto{\pgfqpoint{2.071594in}{1.612672in}}%
\pgfpathlineto{\pgfqpoint{2.072037in}{1.614682in}}%
\pgfpathlineto{\pgfqpoint{2.072626in}{1.610196in}}%
\pgfpathlineto{\pgfqpoint{2.074690in}{1.577998in}}%
\pgfpathlineto{\pgfqpoint{2.075575in}{1.582466in}}%
\pgfpathlineto{\pgfqpoint{2.077491in}{1.605608in}}%
\pgfpathlineto{\pgfqpoint{2.078376in}{1.596754in}}%
\pgfpathlineto{\pgfqpoint{2.081029in}{1.565290in}}%
\pgfpathlineto{\pgfqpoint{2.081471in}{1.569812in}}%
\pgfpathlineto{\pgfqpoint{2.083388in}{1.610016in}}%
\pgfpathlineto{\pgfqpoint{2.084272in}{1.605696in}}%
\pgfpathlineto{\pgfqpoint{2.086631in}{1.576734in}}%
\pgfpathlineto{\pgfqpoint{2.087810in}{1.585990in}}%
\pgfpathlineto{\pgfqpoint{2.089874in}{1.610231in}}%
\pgfpathlineto{\pgfqpoint{2.090464in}{1.604783in}}%
\pgfpathlineto{\pgfqpoint{2.091938in}{1.573456in}}%
\pgfpathlineto{\pgfqpoint{2.093265in}{1.576638in}}%
\pgfpathlineto{\pgfqpoint{2.093412in}{1.576501in}}%
\pgfpathlineto{\pgfqpoint{2.093707in}{1.577384in}}%
\pgfpathlineto{\pgfqpoint{2.095476in}{1.604674in}}%
\pgfpathlineto{\pgfqpoint{2.096803in}{1.594061in}}%
\pgfpathlineto{\pgfqpoint{2.098425in}{1.566504in}}%
\pgfpathlineto{\pgfqpoint{2.099309in}{1.580207in}}%
\pgfpathlineto{\pgfqpoint{2.101815in}{1.609747in}}%
\pgfpathlineto{\pgfqpoint{2.101963in}{1.609656in}}%
\pgfpathlineto{\pgfqpoint{2.102552in}{1.607915in}}%
\pgfpathlineto{\pgfqpoint{2.103584in}{1.589569in}}%
\pgfpathlineto{\pgfqpoint{2.104469in}{1.574256in}}%
\pgfpathlineto{\pgfqpoint{2.105206in}{1.586835in}}%
\pgfpathlineto{\pgfqpoint{2.107417in}{1.609891in}}%
\pgfpathlineto{\pgfqpoint{2.107712in}{1.609174in}}%
\pgfpathlineto{\pgfqpoint{2.109039in}{1.582941in}}%
\pgfpathlineto{\pgfqpoint{2.111103in}{1.556066in}}%
\pgfpathlineto{\pgfqpoint{2.111692in}{1.563650in}}%
\pgfpathlineto{\pgfqpoint{2.113166in}{1.596376in}}%
\pgfpathlineto{\pgfqpoint{2.114051in}{1.587602in}}%
\pgfpathlineto{\pgfqpoint{2.116704in}{1.559701in}}%
\pgfpathlineto{\pgfqpoint{2.117147in}{1.562958in}}%
\pgfpathlineto{\pgfqpoint{2.120095in}{1.615893in}}%
\pgfpathlineto{\pgfqpoint{2.120832in}{1.604297in}}%
\pgfpathlineto{\pgfqpoint{2.122012in}{1.581930in}}%
\pgfpathlineto{\pgfqpoint{2.122896in}{1.588324in}}%
\pgfpathlineto{\pgfqpoint{2.124960in}{1.612940in}}%
\pgfpathlineto{\pgfqpoint{2.125844in}{1.623500in}}%
\pgfpathlineto{\pgfqpoint{2.126434in}{1.617086in}}%
\pgfpathlineto{\pgfqpoint{2.128940in}{1.572422in}}%
\pgfpathlineto{\pgfqpoint{2.129530in}{1.580949in}}%
\pgfpathlineto{\pgfqpoint{2.131004in}{1.607591in}}%
\pgfpathlineto{\pgfqpoint{2.131889in}{1.605874in}}%
\pgfpathlineto{\pgfqpoint{2.132478in}{1.607044in}}%
\pgfpathlineto{\pgfqpoint{2.132773in}{1.605188in}}%
\pgfpathlineto{\pgfqpoint{2.134690in}{1.559103in}}%
\pgfpathlineto{\pgfqpoint{2.135869in}{1.579700in}}%
\pgfpathlineto{\pgfqpoint{2.137638in}{1.610989in}}%
\pgfpathlineto{\pgfqpoint{2.138375in}{1.606973in}}%
\pgfpathlineto{\pgfqpoint{2.140881in}{1.572633in}}%
\pgfpathlineto{\pgfqpoint{2.142060in}{1.584058in}}%
\pgfpathlineto{\pgfqpoint{2.143535in}{1.616436in}}%
\pgfpathlineto{\pgfqpoint{2.144272in}{1.605285in}}%
\pgfpathlineto{\pgfqpoint{2.146778in}{1.573854in}}%
\pgfpathlineto{\pgfqpoint{2.147073in}{1.575063in}}%
\pgfpathlineto{\pgfqpoint{2.149874in}{1.621156in}}%
\pgfpathlineto{\pgfqpoint{2.150906in}{1.608613in}}%
\pgfpathlineto{\pgfqpoint{2.152380in}{1.585718in}}%
\pgfpathlineto{\pgfqpoint{2.152969in}{1.591362in}}%
\pgfpathlineto{\pgfqpoint{2.155918in}{1.622773in}}%
\pgfpathlineto{\pgfqpoint{2.156213in}{1.621139in}}%
\pgfpathlineto{\pgfqpoint{2.159161in}{1.580613in}}%
\pgfpathlineto{\pgfqpoint{2.160046in}{1.588155in}}%
\pgfpathlineto{\pgfqpoint{2.161225in}{1.602173in}}%
\pgfpathlineto{\pgfqpoint{2.161962in}{1.596810in}}%
\pgfpathlineto{\pgfqpoint{2.164173in}{1.561998in}}%
\pgfpathlineto{\pgfqpoint{2.164763in}{1.550910in}}%
\pgfpathlineto{\pgfqpoint{2.165500in}{1.566677in}}%
\pgfpathlineto{\pgfqpoint{2.167859in}{1.596869in}}%
\pgfpathlineto{\pgfqpoint{2.168301in}{1.598683in}}%
\pgfpathlineto{\pgfqpoint{2.168743in}{1.595090in}}%
\pgfpathlineto{\pgfqpoint{2.170070in}{1.568827in}}%
\pgfpathlineto{\pgfqpoint{2.170954in}{1.578065in}}%
\pgfpathlineto{\pgfqpoint{2.173608in}{1.621147in}}%
\pgfpathlineto{\pgfqpoint{2.174493in}{1.610747in}}%
\pgfpathlineto{\pgfqpoint{2.177146in}{1.588186in}}%
\pgfpathlineto{\pgfqpoint{2.177588in}{1.591092in}}%
\pgfpathlineto{\pgfqpoint{2.179063in}{1.618636in}}%
\pgfpathlineto{\pgfqpoint{2.180094in}{1.608152in}}%
\pgfpathlineto{\pgfqpoint{2.181716in}{1.585594in}}%
\pgfpathlineto{\pgfqpoint{2.182601in}{1.568429in}}%
\pgfpathlineto{\pgfqpoint{2.183338in}{1.580529in}}%
\pgfpathlineto{\pgfqpoint{2.185844in}{1.606080in}}%
\pgfpathlineto{\pgfqpoint{2.186139in}{1.605058in}}%
\pgfpathlineto{\pgfqpoint{2.187760in}{1.566007in}}%
\pgfpathlineto{\pgfqpoint{2.188350in}{1.561534in}}%
\pgfpathlineto{\pgfqpoint{2.189234in}{1.565679in}}%
\pgfpathlineto{\pgfqpoint{2.190856in}{1.591095in}}%
\pgfpathlineto{\pgfqpoint{2.191740in}{1.601533in}}%
\pgfpathlineto{\pgfqpoint{2.192478in}{1.594405in}}%
\pgfpathlineto{\pgfqpoint{2.194984in}{1.566081in}}%
\pgfpathlineto{\pgfqpoint{2.195426in}{1.573245in}}%
\pgfpathlineto{\pgfqpoint{2.196900in}{1.605479in}}%
\pgfpathlineto{\pgfqpoint{2.197785in}{1.601424in}}%
\pgfpathlineto{\pgfqpoint{2.198964in}{1.597241in}}%
\pgfpathlineto{\pgfqpoint{2.200438in}{1.574700in}}%
\pgfpathlineto{\pgfqpoint{2.201175in}{1.582720in}}%
\pgfpathlineto{\pgfqpoint{2.203681in}{1.610493in}}%
\pgfpathlineto{\pgfqpoint{2.204124in}{1.607847in}}%
\pgfpathlineto{\pgfqpoint{2.207367in}{1.578508in}}%
\pgfpathlineto{\pgfqpoint{2.207514in}{1.578931in}}%
\pgfpathlineto{\pgfqpoint{2.208399in}{1.593155in}}%
\pgfpathlineto{\pgfqpoint{2.209283in}{1.604705in}}%
\pgfpathlineto{\pgfqpoint{2.209873in}{1.597485in}}%
\pgfpathlineto{\pgfqpoint{2.212674in}{1.562294in}}%
\pgfpathlineto{\pgfqpoint{2.212969in}{1.563725in}}%
\pgfpathlineto{\pgfqpoint{2.214590in}{1.592963in}}%
\pgfpathlineto{\pgfqpoint{2.216212in}{1.589246in}}%
\pgfpathlineto{\pgfqpoint{2.216949in}{1.577636in}}%
\pgfpathlineto{\pgfqpoint{2.218128in}{1.551399in}}%
\pgfpathlineto{\pgfqpoint{2.218865in}{1.562635in}}%
\pgfpathlineto{\pgfqpoint{2.221519in}{1.594813in}}%
\pgfpathlineto{\pgfqpoint{2.221961in}{1.591998in}}%
\pgfpathlineto{\pgfqpoint{2.223435in}{1.570348in}}%
\pgfpathlineto{\pgfqpoint{2.224467in}{1.576052in}}%
\pgfpathlineto{\pgfqpoint{2.226089in}{1.591854in}}%
\pgfpathlineto{\pgfqpoint{2.227121in}{1.605462in}}%
\pgfpathlineto{\pgfqpoint{2.227858in}{1.599900in}}%
\pgfpathlineto{\pgfqpoint{2.230512in}{1.571337in}}%
\pgfpathlineto{\pgfqpoint{2.231101in}{1.580033in}}%
\pgfpathlineto{\pgfqpoint{2.232133in}{1.594830in}}%
\pgfpathlineto{\pgfqpoint{2.233018in}{1.591576in}}%
\pgfpathlineto{\pgfqpoint{2.233902in}{1.593932in}}%
\pgfpathlineto{\pgfqpoint{2.234197in}{1.592170in}}%
\pgfpathlineto{\pgfqpoint{2.236113in}{1.561096in}}%
\pgfpathlineto{\pgfqpoint{2.237588in}{1.567888in}}%
\pgfpathlineto{\pgfqpoint{2.239209in}{1.586758in}}%
\pgfpathlineto{\pgfqpoint{2.240094in}{1.578675in}}%
\pgfpathlineto{\pgfqpoint{2.242747in}{1.559822in}}%
\pgfpathlineto{\pgfqpoint{2.243190in}{1.562118in}}%
\pgfpathlineto{\pgfqpoint{2.244959in}{1.590226in}}%
\pgfpathlineto{\pgfqpoint{2.246138in}{1.584742in}}%
\pgfpathlineto{\pgfqpoint{2.247170in}{1.579575in}}%
\pgfpathlineto{\pgfqpoint{2.248202in}{1.567818in}}%
\pgfpathlineto{\pgfqpoint{2.248939in}{1.575397in}}%
\pgfpathlineto{\pgfqpoint{2.251592in}{1.595034in}}%
\pgfpathlineto{\pgfqpoint{2.251887in}{1.595803in}}%
\pgfpathlineto{\pgfqpoint{2.252329in}{1.591886in}}%
\pgfpathlineto{\pgfqpoint{2.253804in}{1.568983in}}%
\pgfpathlineto{\pgfqpoint{2.254836in}{1.573086in}}%
\pgfpathlineto{\pgfqpoint{2.255573in}{1.573054in}}%
\pgfpathlineto{\pgfqpoint{2.255720in}{1.573705in}}%
\pgfpathlineto{\pgfqpoint{2.257194in}{1.588188in}}%
\pgfpathlineto{\pgfqpoint{2.258079in}{1.581379in}}%
\pgfpathlineto{\pgfqpoint{2.260732in}{1.564083in}}%
\pgfpathlineto{\pgfqpoint{2.260880in}{1.564350in}}%
\pgfpathlineto{\pgfqpoint{2.261764in}{1.578650in}}%
\pgfpathlineto{\pgfqpoint{2.262796in}{1.592464in}}%
\pgfpathlineto{\pgfqpoint{2.263828in}{1.591018in}}%
\pgfpathlineto{\pgfqpoint{2.264565in}{1.593487in}}%
\pgfpathlineto{\pgfqpoint{2.265007in}{1.590377in}}%
\pgfpathlineto{\pgfqpoint{2.266039in}{1.577162in}}%
\pgfpathlineto{\pgfqpoint{2.266777in}{1.584473in}}%
\pgfpathlineto{\pgfqpoint{2.268840in}{1.591913in}}%
\pgfpathlineto{\pgfqpoint{2.269430in}{1.594930in}}%
\pgfpathlineto{\pgfqpoint{2.269872in}{1.591262in}}%
\pgfpathlineto{\pgfqpoint{2.272968in}{1.564100in}}%
\pgfpathlineto{\pgfqpoint{2.273410in}{1.565568in}}%
\pgfpathlineto{\pgfqpoint{2.274737in}{1.579785in}}%
\pgfpathlineto{\pgfqpoint{2.275474in}{1.573165in}}%
\pgfpathlineto{\pgfqpoint{2.278275in}{1.549503in}}%
\pgfpathlineto{\pgfqpoint{2.278423in}{1.548786in}}%
\pgfpathlineto{\pgfqpoint{2.278865in}{1.552376in}}%
\pgfpathlineto{\pgfqpoint{2.280192in}{1.576465in}}%
\pgfpathlineto{\pgfqpoint{2.281518in}{1.575212in}}%
\pgfpathlineto{\pgfqpoint{2.282108in}{1.578025in}}%
\pgfpathlineto{\pgfqpoint{2.282550in}{1.574704in}}%
\pgfpathlineto{\pgfqpoint{2.283582in}{1.562080in}}%
\pgfpathlineto{\pgfqpoint{2.284319in}{1.570887in}}%
\pgfpathlineto{\pgfqpoint{2.287120in}{1.591968in}}%
\pgfpathlineto{\pgfqpoint{2.287268in}{1.592388in}}%
\pgfpathlineto{\pgfqpoint{2.287710in}{1.589793in}}%
\pgfpathlineto{\pgfqpoint{2.290511in}{1.572913in}}%
\pgfpathlineto{\pgfqpoint{2.290953in}{1.571628in}}%
\pgfpathlineto{\pgfqpoint{2.291543in}{1.575189in}}%
\pgfpathlineto{\pgfqpoint{2.292427in}{1.582977in}}%
\pgfpathlineto{\pgfqpoint{2.293164in}{1.577850in}}%
\pgfpathlineto{\pgfqpoint{2.296260in}{1.552128in}}%
\pgfpathlineto{\pgfqpoint{2.296555in}{1.553070in}}%
\pgfpathlineto{\pgfqpoint{2.298029in}{1.568225in}}%
\pgfpathlineto{\pgfqpoint{2.299209in}{1.564573in}}%
\pgfpathlineto{\pgfqpoint{2.299946in}{1.566676in}}%
\pgfpathlineto{\pgfqpoint{2.300388in}{1.563840in}}%
\pgfpathlineto{\pgfqpoint{2.301567in}{1.549818in}}%
\pgfpathlineto{\pgfqpoint{2.302304in}{1.557978in}}%
\pgfpathlineto{\pgfqpoint{2.305253in}{1.575592in}}%
\pgfpathlineto{\pgfqpoint{2.305990in}{1.569320in}}%
\pgfpathlineto{\pgfqpoint{2.306579in}{1.565575in}}%
\pgfpathlineto{\pgfqpoint{2.307317in}{1.569539in}}%
\pgfpathlineto{\pgfqpoint{2.307759in}{1.571295in}}%
\pgfpathlineto{\pgfqpoint{2.308643in}{1.569031in}}%
\pgfpathlineto{\pgfqpoint{2.308938in}{1.569189in}}%
\pgfpathlineto{\pgfqpoint{2.309086in}{1.569880in}}%
\pgfpathlineto{\pgfqpoint{2.310707in}{1.585067in}}%
\pgfpathlineto{\pgfqpoint{2.311887in}{1.579782in}}%
\pgfpathlineto{\pgfqpoint{2.312771in}{1.583649in}}%
\pgfpathlineto{\pgfqpoint{2.313361in}{1.580010in}}%
\pgfpathlineto{\pgfqpoint{2.314245in}{1.570206in}}%
\pgfpathlineto{\pgfqpoint{2.314982in}{1.577170in}}%
\pgfpathlineto{\pgfqpoint{2.315572in}{1.582186in}}%
\pgfpathlineto{\pgfqpoint{2.316604in}{1.579129in}}%
\pgfpathlineto{\pgfqpoint{2.317783in}{1.589555in}}%
\pgfpathlineto{\pgfqpoint{2.318520in}{1.581914in}}%
\pgfpathlineto{\pgfqpoint{2.319405in}{1.572576in}}%
\pgfpathlineto{\pgfqpoint{2.320289in}{1.576970in}}%
\pgfpathlineto{\pgfqpoint{2.320584in}{1.576692in}}%
\pgfpathlineto{\pgfqpoint{2.320732in}{1.575991in}}%
\pgfpathlineto{\pgfqpoint{2.321469in}{1.572441in}}%
\pgfpathlineto{\pgfqpoint{2.322058in}{1.575623in}}%
\pgfpathlineto{\pgfqpoint{2.322943in}{1.581403in}}%
\pgfpathlineto{\pgfqpoint{2.323533in}{1.577668in}}%
\pgfpathlineto{\pgfqpoint{2.326776in}{1.551058in}}%
\pgfpathlineto{\pgfqpoint{2.327218in}{1.555349in}}%
\pgfpathlineto{\pgfqpoint{2.328250in}{1.566533in}}%
\pgfpathlineto{\pgfqpoint{2.328987in}{1.560799in}}%
\pgfpathlineto{\pgfqpoint{2.329282in}{1.559413in}}%
\pgfpathlineto{\pgfqpoint{2.330019in}{1.564126in}}%
\pgfpathlineto{\pgfqpoint{2.330461in}{1.566376in}}%
\pgfpathlineto{\pgfqpoint{2.331051in}{1.561125in}}%
\pgfpathlineto{\pgfqpoint{2.331935in}{1.550420in}}%
\pgfpathlineto{\pgfqpoint{2.332673in}{1.556732in}}%
\pgfpathlineto{\pgfqpoint{2.335621in}{1.581622in}}%
\pgfpathlineto{\pgfqpoint{2.336063in}{1.579509in}}%
\pgfpathlineto{\pgfqpoint{2.336948in}{1.572778in}}%
\pgfpathlineto{\pgfqpoint{2.337685in}{1.576746in}}%
\pgfpathlineto{\pgfqpoint{2.337980in}{1.577835in}}%
\pgfpathlineto{\pgfqpoint{2.338569in}{1.575237in}}%
\pgfpathlineto{\pgfqpoint{2.339159in}{1.571832in}}%
\pgfpathlineto{\pgfqpoint{2.339749in}{1.577224in}}%
\pgfpathlineto{\pgfqpoint{2.340928in}{1.591499in}}%
\pgfpathlineto{\pgfqpoint{2.341813in}{1.586259in}}%
\pgfpathlineto{\pgfqpoint{2.344613in}{1.571496in}}%
\pgfpathlineto{\pgfqpoint{2.346088in}{1.573006in}}%
\pgfpathlineto{\pgfqpoint{2.346677in}{1.571596in}}%
\pgfpathlineto{\pgfqpoint{2.347267in}{1.573728in}}%
\pgfpathlineto{\pgfqpoint{2.348004in}{1.577572in}}%
\pgfpathlineto{\pgfqpoint{2.348594in}{1.572680in}}%
\pgfpathlineto{\pgfqpoint{2.349773in}{1.553561in}}%
\pgfpathlineto{\pgfqpoint{2.350658in}{1.560994in}}%
\pgfpathlineto{\pgfqpoint{2.353606in}{1.577792in}}%
\pgfpathlineto{\pgfqpoint{2.353901in}{1.576822in}}%
\pgfpathlineto{\pgfqpoint{2.356554in}{1.566378in}}%
\pgfpathlineto{\pgfqpoint{2.356849in}{1.567321in}}%
\pgfpathlineto{\pgfqpoint{2.358471in}{1.589918in}}%
\pgfpathlineto{\pgfqpoint{2.359945in}{1.581979in}}%
\pgfpathlineto{\pgfqpoint{2.360829in}{1.585275in}}%
\pgfpathlineto{\pgfqpoint{2.361419in}{1.582131in}}%
\pgfpathlineto{\pgfqpoint{2.362304in}{1.574005in}}%
\pgfpathlineto{\pgfqpoint{2.363041in}{1.578367in}}%
\pgfpathlineto{\pgfqpoint{2.365547in}{1.589072in}}%
\pgfpathlineto{\pgfqpoint{2.365694in}{1.589404in}}%
\pgfpathlineto{\pgfqpoint{2.366137in}{1.587238in}}%
\pgfpathlineto{\pgfqpoint{2.367168in}{1.577037in}}%
\pgfpathlineto{\pgfqpoint{2.367758in}{1.583131in}}%
\pgfpathlineto{\pgfqpoint{2.368495in}{1.590124in}}%
\pgfpathlineto{\pgfqpoint{2.369232in}{1.582947in}}%
\pgfpathlineto{\pgfqpoint{2.369675in}{1.579320in}}%
\pgfpathlineto{\pgfqpoint{2.370412in}{1.586973in}}%
\pgfpathlineto{\pgfqpoint{2.371149in}{1.594964in}}%
\pgfpathlineto{\pgfqpoint{2.371886in}{1.587474in}}%
\pgfpathlineto{\pgfqpoint{2.373950in}{1.579205in}}%
\pgfpathlineto{\pgfqpoint{2.374687in}{1.574925in}}%
\pgfpathlineto{\pgfqpoint{2.375277in}{1.579055in}}%
\pgfpathlineto{\pgfqpoint{2.375866in}{1.584267in}}%
\pgfpathlineto{\pgfqpoint{2.376603in}{1.577902in}}%
\pgfpathlineto{\pgfqpoint{2.377193in}{1.571145in}}%
\pgfpathlineto{\pgfqpoint{2.377930in}{1.577598in}}%
\pgfpathlineto{\pgfqpoint{2.378372in}{1.580853in}}%
\pgfpathlineto{\pgfqpoint{2.378962in}{1.572275in}}%
\pgfpathlineto{\pgfqpoint{2.379846in}{1.553184in}}%
\pgfpathlineto{\pgfqpoint{2.380584in}{1.563838in}}%
\pgfpathlineto{\pgfqpoint{2.381321in}{1.573264in}}%
\pgfpathlineto{\pgfqpoint{2.382205in}{1.566596in}}%
\pgfpathlineto{\pgfqpoint{2.382500in}{1.565360in}}%
\pgfpathlineto{\pgfqpoint{2.383237in}{1.568985in}}%
\pgfpathlineto{\pgfqpoint{2.383679in}{1.570892in}}%
\pgfpathlineto{\pgfqpoint{2.384416in}{1.568006in}}%
\pgfpathlineto{\pgfqpoint{2.385006in}{1.566031in}}%
\pgfpathlineto{\pgfqpoint{2.385596in}{1.569124in}}%
\pgfpathlineto{\pgfqpoint{2.388692in}{1.593389in}}%
\pgfpathlineto{\pgfqpoint{2.389281in}{1.585610in}}%
\pgfpathlineto{\pgfqpoint{2.390018in}{1.574006in}}%
\pgfpathlineto{\pgfqpoint{2.390755in}{1.583143in}}%
\pgfpathlineto{\pgfqpoint{2.391345in}{1.588952in}}%
\pgfpathlineto{\pgfqpoint{2.392082in}{1.581764in}}%
\pgfpathlineto{\pgfqpoint{2.392524in}{1.579115in}}%
\pgfpathlineto{\pgfqpoint{2.393114in}{1.584587in}}%
\pgfpathlineto{\pgfqpoint{2.393851in}{1.591194in}}%
\pgfpathlineto{\pgfqpoint{2.394883in}{1.587738in}}%
\pgfpathlineto{\pgfqpoint{2.396063in}{1.585679in}}%
\pgfpathlineto{\pgfqpoint{2.397389in}{1.568019in}}%
\pgfpathlineto{\pgfqpoint{2.398274in}{1.575908in}}%
\pgfpathlineto{\pgfqpoint{2.398716in}{1.578265in}}%
\pgfpathlineto{\pgfqpoint{2.399306in}{1.573745in}}%
\pgfpathlineto{\pgfqpoint{2.400043in}{1.567136in}}%
\pgfpathlineto{\pgfqpoint{2.400632in}{1.572689in}}%
\pgfpathlineto{\pgfqpoint{2.401370in}{1.580441in}}%
\pgfpathlineto{\pgfqpoint{2.401959in}{1.574126in}}%
\pgfpathlineto{\pgfqpoint{2.402991in}{1.559453in}}%
\pgfpathlineto{\pgfqpoint{2.403876in}{1.565921in}}%
\pgfpathlineto{\pgfqpoint{2.404171in}{1.566509in}}%
\pgfpathlineto{\pgfqpoint{2.404760in}{1.563449in}}%
\pgfpathlineto{\pgfqpoint{2.405055in}{1.562354in}}%
\pgfpathlineto{\pgfqpoint{2.405645in}{1.565658in}}%
\pgfpathlineto{\pgfqpoint{2.406529in}{1.572187in}}%
\pgfpathlineto{\pgfqpoint{2.407266in}{1.567779in}}%
\pgfpathlineto{\pgfqpoint{2.407561in}{1.566821in}}%
\pgfpathlineto{\pgfqpoint{2.408151in}{1.570507in}}%
\pgfpathlineto{\pgfqpoint{2.408740in}{1.574821in}}%
\pgfpathlineto{\pgfqpoint{2.409330in}{1.569697in}}%
\pgfpathlineto{\pgfqpoint{2.410215in}{1.557582in}}%
\pgfpathlineto{\pgfqpoint{2.410804in}{1.565506in}}%
\pgfpathlineto{\pgfqpoint{2.411689in}{1.578228in}}%
\pgfpathlineto{\pgfqpoint{2.412573in}{1.570796in}}%
\pgfpathlineto{\pgfqpoint{2.412868in}{1.570124in}}%
\pgfpathlineto{\pgfqpoint{2.413310in}{1.573849in}}%
\pgfpathlineto{\pgfqpoint{2.414048in}{1.579957in}}%
\pgfpathlineto{\pgfqpoint{2.414637in}{1.574502in}}%
\pgfpathlineto{\pgfqpoint{2.415522in}{1.565161in}}%
\pgfpathlineto{\pgfqpoint{2.416554in}{1.568420in}}%
\pgfpathlineto{\pgfqpoint{2.417438in}{1.565654in}}%
\pgfpathlineto{\pgfqpoint{2.417880in}{1.568680in}}%
\pgfpathlineto{\pgfqpoint{2.418912in}{1.580603in}}%
\pgfpathlineto{\pgfqpoint{2.419649in}{1.573293in}}%
\pgfpathlineto{\pgfqpoint{2.420387in}{1.564008in}}%
\pgfpathlineto{\pgfqpoint{2.421271in}{1.571348in}}%
\pgfpathlineto{\pgfqpoint{2.421566in}{1.573105in}}%
\pgfpathlineto{\pgfqpoint{2.422156in}{1.567846in}}%
\pgfpathlineto{\pgfqpoint{2.422893in}{1.557292in}}%
\pgfpathlineto{\pgfqpoint{2.423630in}{1.569053in}}%
\pgfpathlineto{\pgfqpoint{2.424514in}{1.583127in}}%
\pgfpathlineto{\pgfqpoint{2.425251in}{1.575459in}}%
\pgfpathlineto{\pgfqpoint{2.425546in}{1.573474in}}%
\pgfpathlineto{\pgfqpoint{2.426283in}{1.578058in}}%
\pgfpathlineto{\pgfqpoint{2.426578in}{1.579122in}}%
\pgfpathlineto{\pgfqpoint{2.427020in}{1.575732in}}%
\pgfpathlineto{\pgfqpoint{2.427757in}{1.566344in}}%
\pgfpathlineto{\pgfqpoint{2.428495in}{1.574792in}}%
\pgfpathlineto{\pgfqpoint{2.429084in}{1.581671in}}%
\pgfpathlineto{\pgfqpoint{2.429821in}{1.574769in}}%
\pgfpathlineto{\pgfqpoint{2.430264in}{1.571914in}}%
\pgfpathlineto{\pgfqpoint{2.430853in}{1.579551in}}%
\pgfpathlineto{\pgfqpoint{2.431590in}{1.588027in}}%
\pgfpathlineto{\pgfqpoint{2.432180in}{1.579727in}}%
\pgfpathlineto{\pgfqpoint{2.433212in}{1.564419in}}%
\pgfpathlineto{\pgfqpoint{2.433949in}{1.571393in}}%
\pgfpathlineto{\pgfqpoint{2.434391in}{1.573266in}}%
\pgfpathlineto{\pgfqpoint{2.434981in}{1.568813in}}%
\pgfpathlineto{\pgfqpoint{2.435571in}{1.564713in}}%
\pgfpathlineto{\pgfqpoint{2.436308in}{1.570514in}}%
\pgfpathlineto{\pgfqpoint{2.436897in}{1.574774in}}%
\pgfpathlineto{\pgfqpoint{2.437635in}{1.569468in}}%
\pgfpathlineto{\pgfqpoint{2.438077in}{1.566737in}}%
\pgfpathlineto{\pgfqpoint{2.438814in}{1.571556in}}%
\pgfpathlineto{\pgfqpoint{2.439109in}{1.572933in}}%
\pgfpathlineto{\pgfqpoint{2.439698in}{1.567114in}}%
\pgfpathlineto{\pgfqpoint{2.440435in}{1.556920in}}%
\pgfpathlineto{\pgfqpoint{2.441025in}{1.566324in}}%
\pgfpathlineto{\pgfqpoint{2.441910in}{1.581636in}}%
\pgfpathlineto{\pgfqpoint{2.442647in}{1.573036in}}%
\pgfpathlineto{\pgfqpoint{2.443089in}{1.570128in}}%
\pgfpathlineto{\pgfqpoint{2.443679in}{1.577275in}}%
\pgfpathlineto{\pgfqpoint{2.444416in}{1.584602in}}%
\pgfpathlineto{\pgfqpoint{2.445153in}{1.576259in}}%
\pgfpathlineto{\pgfqpoint{2.445743in}{1.570818in}}%
\pgfpathlineto{\pgfqpoint{2.446480in}{1.576235in}}%
\pgfpathlineto{\pgfqpoint{2.449133in}{1.594106in}}%
\pgfpathlineto{\pgfqpoint{2.449428in}{1.593103in}}%
\pgfpathlineto{\pgfqpoint{2.450755in}{1.577292in}}%
\pgfpathlineto{\pgfqpoint{2.451639in}{1.585199in}}%
\pgfpathlineto{\pgfqpoint{2.451934in}{1.586078in}}%
\pgfpathlineto{\pgfqpoint{2.452376in}{1.582000in}}%
\pgfpathlineto{\pgfqpoint{2.453261in}{1.571574in}}%
\pgfpathlineto{\pgfqpoint{2.453851in}{1.579727in}}%
\pgfpathlineto{\pgfqpoint{2.454735in}{1.591120in}}%
\pgfpathlineto{\pgfqpoint{2.455472in}{1.583069in}}%
\pgfpathlineto{\pgfqpoint{2.457831in}{1.570482in}}%
\pgfpathlineto{\pgfqpoint{2.458273in}{1.568887in}}%
\pgfpathlineto{\pgfqpoint{2.458863in}{1.572209in}}%
\pgfpathlineto{\pgfqpoint{2.459600in}{1.576388in}}%
\pgfpathlineto{\pgfqpoint{2.460337in}{1.572623in}}%
\pgfpathlineto{\pgfqpoint{2.460632in}{1.571628in}}%
\pgfpathlineto{\pgfqpoint{2.461221in}{1.575915in}}%
\pgfpathlineto{\pgfqpoint{2.461959in}{1.582587in}}%
\pgfpathlineto{\pgfqpoint{2.462548in}{1.573839in}}%
\pgfpathlineto{\pgfqpoint{2.463580in}{1.550236in}}%
\pgfpathlineto{\pgfqpoint{2.464317in}{1.563336in}}%
\pgfpathlineto{\pgfqpoint{2.464907in}{1.571286in}}%
\pgfpathlineto{\pgfqpoint{2.465791in}{1.564875in}}%
\pgfpathlineto{\pgfqpoint{2.465939in}{1.564329in}}%
\pgfpathlineto{\pgfqpoint{2.466381in}{1.566639in}}%
\pgfpathlineto{\pgfqpoint{2.467266in}{1.574617in}}%
\pgfpathlineto{\pgfqpoint{2.468003in}{1.570358in}}%
\pgfpathlineto{\pgfqpoint{2.470656in}{1.555892in}}%
\pgfpathlineto{\pgfqpoint{2.470951in}{1.558338in}}%
\pgfpathlineto{\pgfqpoint{2.472278in}{1.580272in}}%
\pgfpathlineto{\pgfqpoint{2.473162in}{1.571011in}}%
\pgfpathlineto{\pgfqpoint{2.473605in}{1.567266in}}%
\pgfpathlineto{\pgfqpoint{2.474342in}{1.575195in}}%
\pgfpathlineto{\pgfqpoint{2.474931in}{1.579920in}}%
\pgfpathlineto{\pgfqpoint{2.475521in}{1.571939in}}%
\pgfpathlineto{\pgfqpoint{2.476258in}{1.561812in}}%
\pgfpathlineto{\pgfqpoint{2.476995in}{1.571996in}}%
\pgfpathlineto{\pgfqpoint{2.479207in}{1.583493in}}%
\pgfpathlineto{\pgfqpoint{2.479501in}{1.585263in}}%
\pgfpathlineto{\pgfqpoint{2.480091in}{1.580599in}}%
\pgfpathlineto{\pgfqpoint{2.480976in}{1.570585in}}%
\pgfpathlineto{\pgfqpoint{2.481713in}{1.578557in}}%
\pgfpathlineto{\pgfqpoint{2.482155in}{1.580725in}}%
\pgfpathlineto{\pgfqpoint{2.482745in}{1.573700in}}%
\pgfpathlineto{\pgfqpoint{2.483482in}{1.564881in}}%
\pgfpathlineto{\pgfqpoint{2.484219in}{1.574485in}}%
\pgfpathlineto{\pgfqpoint{2.485103in}{1.583928in}}%
\pgfpathlineto{\pgfqpoint{2.485693in}{1.578052in}}%
\pgfpathlineto{\pgfqpoint{2.488199in}{1.562969in}}%
\pgfpathlineto{\pgfqpoint{2.488641in}{1.561277in}}%
\pgfpathlineto{\pgfqpoint{2.489231in}{1.564177in}}%
\pgfpathlineto{\pgfqpoint{2.489821in}{1.567194in}}%
\pgfpathlineto{\pgfqpoint{2.490558in}{1.563729in}}%
\pgfpathlineto{\pgfqpoint{2.490853in}{1.562702in}}%
\pgfpathlineto{\pgfqpoint{2.491442in}{1.566854in}}%
\pgfpathlineto{\pgfqpoint{2.492179in}{1.573787in}}%
\pgfpathlineto{\pgfqpoint{2.492769in}{1.566723in}}%
\pgfpathlineto{\pgfqpoint{2.493801in}{1.549754in}}%
\pgfpathlineto{\pgfqpoint{2.494538in}{1.559061in}}%
\pgfpathlineto{\pgfqpoint{2.495128in}{1.564162in}}%
\pgfpathlineto{\pgfqpoint{2.496012in}{1.559210in}}%
\pgfpathlineto{\pgfqpoint{2.496160in}{1.558825in}}%
\pgfpathlineto{\pgfqpoint{2.496602in}{1.560989in}}%
\pgfpathlineto{\pgfqpoint{2.497929in}{1.572039in}}%
\pgfpathlineto{\pgfqpoint{2.498813in}{1.569669in}}%
\pgfpathlineto{\pgfqpoint{2.499698in}{1.572089in}}%
\pgfpathlineto{\pgfqpoint{2.499993in}{1.572096in}}%
\pgfpathlineto{\pgfqpoint{2.500287in}{1.571105in}}%
\pgfpathlineto{\pgfqpoint{2.500877in}{1.569463in}}%
\pgfpathlineto{\pgfqpoint{2.501319in}{1.572890in}}%
\pgfpathlineto{\pgfqpoint{2.502646in}{1.590180in}}%
\pgfpathlineto{\pgfqpoint{2.503531in}{1.583355in}}%
\pgfpathlineto{\pgfqpoint{2.503825in}{1.582810in}}%
\pgfpathlineto{\pgfqpoint{2.504268in}{1.586016in}}%
\pgfpathlineto{\pgfqpoint{2.505152in}{1.593378in}}%
\pgfpathlineto{\pgfqpoint{2.505889in}{1.588255in}}%
\pgfpathlineto{\pgfqpoint{2.506774in}{1.581350in}}%
\pgfpathlineto{\pgfqpoint{2.507658in}{1.585666in}}%
\pgfpathlineto{\pgfqpoint{2.508543in}{1.589004in}}%
\pgfpathlineto{\pgfqpoint{2.509427in}{1.587223in}}%
\pgfpathlineto{\pgfqpoint{2.510459in}{1.583987in}}%
\pgfpathlineto{\pgfqpoint{2.513997in}{1.555840in}}%
\pgfpathlineto{\pgfqpoint{2.514440in}{1.560685in}}%
\pgfpathlineto{\pgfqpoint{2.515324in}{1.573777in}}%
\pgfpathlineto{\pgfqpoint{2.516061in}{1.566700in}}%
\pgfpathlineto{\pgfqpoint{2.518125in}{1.557722in}}%
\pgfpathlineto{\pgfqpoint{2.518862in}{1.553302in}}%
\pgfpathlineto{\pgfqpoint{2.519599in}{1.557042in}}%
\pgfpathlineto{\pgfqpoint{2.522842in}{1.579045in}}%
\pgfpathlineto{\pgfqpoint{2.523285in}{1.576719in}}%
\pgfpathlineto{\pgfqpoint{2.524317in}{1.567410in}}%
\pgfpathlineto{\pgfqpoint{2.524906in}{1.573422in}}%
\pgfpathlineto{\pgfqpoint{2.525791in}{1.582948in}}%
\pgfpathlineto{\pgfqpoint{2.526970in}{1.582566in}}%
\pgfpathlineto{\pgfqpoint{2.528149in}{1.593884in}}%
\pgfpathlineto{\pgfqpoint{2.529034in}{1.588495in}}%
\pgfpathlineto{\pgfqpoint{2.531393in}{1.575588in}}%
\pgfpathlineto{\pgfqpoint{2.531982in}{1.580931in}}%
\pgfpathlineto{\pgfqpoint{2.532867in}{1.590189in}}%
\pgfpathlineto{\pgfqpoint{2.533604in}{1.583133in}}%
\pgfpathlineto{\pgfqpoint{2.534194in}{1.578035in}}%
\pgfpathlineto{\pgfqpoint{2.535078in}{1.582212in}}%
\pgfpathlineto{\pgfqpoint{2.535226in}{1.582627in}}%
\pgfpathlineto{\pgfqpoint{2.535668in}{1.580774in}}%
\pgfpathlineto{\pgfqpoint{2.537142in}{1.563170in}}%
\pgfpathlineto{\pgfqpoint{2.538321in}{1.567782in}}%
\pgfpathlineto{\pgfqpoint{2.540090in}{1.575859in}}%
\pgfpathlineto{\pgfqpoint{2.540680in}{1.570952in}}%
\pgfpathlineto{\pgfqpoint{2.541565in}{1.561752in}}%
\pgfpathlineto{\pgfqpoint{2.542449in}{1.567278in}}%
\pgfpathlineto{\pgfqpoint{2.542744in}{1.568128in}}%
\pgfpathlineto{\pgfqpoint{2.543186in}{1.565160in}}%
\pgfpathlineto{\pgfqpoint{2.544071in}{1.555545in}}%
\pgfpathlineto{\pgfqpoint{2.544808in}{1.564142in}}%
\pgfpathlineto{\pgfqpoint{2.545840in}{1.578350in}}%
\pgfpathlineto{\pgfqpoint{2.546577in}{1.573194in}}%
\pgfpathlineto{\pgfqpoint{2.548935in}{1.563989in}}%
\pgfpathlineto{\pgfqpoint{2.549083in}{1.563830in}}%
\pgfpathlineto{\pgfqpoint{2.549378in}{1.564758in}}%
\pgfpathlineto{\pgfqpoint{2.552916in}{1.592186in}}%
\pgfpathlineto{\pgfqpoint{2.553653in}{1.583839in}}%
\pgfpathlineto{\pgfqpoint{2.554537in}{1.575105in}}%
\pgfpathlineto{\pgfqpoint{2.555274in}{1.580230in}}%
\pgfpathlineto{\pgfqpoint{2.558223in}{1.593931in}}%
\pgfpathlineto{\pgfqpoint{2.558813in}{1.595371in}}%
\pgfpathlineto{\pgfqpoint{2.559402in}{1.593586in}}%
\pgfpathlineto{\pgfqpoint{2.561761in}{1.575205in}}%
\pgfpathlineto{\pgfqpoint{2.562498in}{1.581760in}}%
\pgfpathlineto{\pgfqpoint{2.563382in}{1.591543in}}%
\pgfpathlineto{\pgfqpoint{2.564267in}{1.585276in}}%
\pgfpathlineto{\pgfqpoint{2.567510in}{1.564514in}}%
\pgfpathlineto{\pgfqpoint{2.565594in}{1.585721in}}%
\pgfpathlineto{\pgfqpoint{2.567952in}{1.566217in}}%
\pgfpathlineto{\pgfqpoint{2.570311in}{1.584133in}}%
\pgfpathlineto{\pgfqpoint{2.571048in}{1.578960in}}%
\pgfpathlineto{\pgfqpoint{2.572228in}{1.569022in}}%
\pgfpathlineto{\pgfqpoint{2.572965in}{1.573365in}}%
\pgfpathlineto{\pgfqpoint{2.573407in}{1.575085in}}%
\pgfpathlineto{\pgfqpoint{2.574144in}{1.571181in}}%
\pgfpathlineto{\pgfqpoint{2.574586in}{1.569493in}}%
\pgfpathlineto{\pgfqpoint{2.575029in}{1.573261in}}%
\pgfpathlineto{\pgfqpoint{2.576355in}{1.591777in}}%
\pgfpathlineto{\pgfqpoint{2.577240in}{1.585393in}}%
\pgfpathlineto{\pgfqpoint{2.579156in}{1.581890in}}%
\pgfpathlineto{\pgfqpoint{2.579746in}{1.583588in}}%
\pgfpathlineto{\pgfqpoint{2.583137in}{1.597299in}}%
\pgfpathlineto{\pgfqpoint{2.583431in}{1.596541in}}%
\pgfpathlineto{\pgfqpoint{2.585053in}{1.573830in}}%
\pgfpathlineto{\pgfqpoint{2.586232in}{1.584131in}}%
\pgfpathlineto{\pgfqpoint{2.588296in}{1.588209in}}%
\pgfpathlineto{\pgfqpoint{2.588738in}{1.586767in}}%
\pgfpathlineto{\pgfqpoint{2.591982in}{1.569149in}}%
\pgfpathlineto{\pgfqpoint{2.592424in}{1.572023in}}%
\pgfpathlineto{\pgfqpoint{2.593898in}{1.587972in}}%
\pgfpathlineto{\pgfqpoint{2.594635in}{1.582955in}}%
\pgfpathlineto{\pgfqpoint{2.595225in}{1.580223in}}%
\pgfpathlineto{\pgfqpoint{2.596109in}{1.583294in}}%
\pgfpathlineto{\pgfqpoint{2.596552in}{1.581773in}}%
\pgfpathlineto{\pgfqpoint{2.597584in}{1.572975in}}%
\pgfpathlineto{\pgfqpoint{2.598468in}{1.578209in}}%
\pgfpathlineto{\pgfqpoint{2.600679in}{1.590032in}}%
\pgfpathlineto{\pgfqpoint{2.601122in}{1.588106in}}%
\pgfpathlineto{\pgfqpoint{2.602448in}{1.572141in}}%
\pgfpathlineto{\pgfqpoint{2.603480in}{1.577930in}}%
\pgfpathlineto{\pgfqpoint{2.603775in}{1.578104in}}%
\pgfpathlineto{\pgfqpoint{2.604217in}{1.576757in}}%
\pgfpathlineto{\pgfqpoint{2.604660in}{1.576126in}}%
\pgfpathlineto{\pgfqpoint{2.605102in}{1.578314in}}%
\pgfpathlineto{\pgfqpoint{2.606871in}{1.592491in}}%
\pgfpathlineto{\pgfqpoint{2.607608in}{1.587727in}}%
\pgfpathlineto{\pgfqpoint{2.608935in}{1.579024in}}%
\pgfpathlineto{\pgfqpoint{2.609672in}{1.581235in}}%
\pgfpathlineto{\pgfqpoint{2.613652in}{1.597550in}}%
\pgfpathlineto{\pgfqpoint{2.613947in}{1.595882in}}%
\pgfpathlineto{\pgfqpoint{2.615421in}{1.573862in}}%
\pgfpathlineto{\pgfqpoint{2.616453in}{1.584228in}}%
\pgfpathlineto{\pgfqpoint{2.617338in}{1.589035in}}%
\pgfpathlineto{\pgfqpoint{2.618222in}{1.588169in}}%
\pgfpathlineto{\pgfqpoint{2.619254in}{1.586779in}}%
\pgfpathlineto{\pgfqpoint{2.622202in}{1.570732in}}%
\pgfpathlineto{\pgfqpoint{2.622792in}{1.575623in}}%
\pgfpathlineto{\pgfqpoint{2.624119in}{1.596927in}}%
\pgfpathlineto{\pgfqpoint{2.624856in}{1.590170in}}%
\pgfpathlineto{\pgfqpoint{2.625740in}{1.583677in}}%
\pgfpathlineto{\pgfqpoint{2.626772in}{1.584739in}}%
\pgfpathlineto{\pgfqpoint{2.628247in}{1.576466in}}%
\pgfpathlineto{\pgfqpoint{2.628984in}{1.581305in}}%
\pgfpathlineto{\pgfqpoint{2.630753in}{1.589571in}}%
\pgfpathlineto{\pgfqpoint{2.631195in}{1.588598in}}%
\pgfpathlineto{\pgfqpoint{2.632227in}{1.575581in}}%
\pgfpathlineto{\pgfqpoint{2.632817in}{1.568323in}}%
\pgfpathlineto{\pgfqpoint{2.633554in}{1.576085in}}%
\pgfpathlineto{\pgfqpoint{2.634291in}{1.583545in}}%
\pgfpathlineto{\pgfqpoint{2.635028in}{1.576513in}}%
\pgfpathlineto{\pgfqpoint{2.635323in}{1.575219in}}%
\pgfpathlineto{\pgfqpoint{2.635765in}{1.580100in}}%
\pgfpathlineto{\pgfqpoint{2.636797in}{1.597659in}}%
\pgfpathlineto{\pgfqpoint{2.637681in}{1.589438in}}%
\pgfpathlineto{\pgfqpoint{2.639450in}{1.582864in}}%
\pgfpathlineto{\pgfqpoint{2.639598in}{1.583086in}}%
\pgfpathlineto{\pgfqpoint{2.640482in}{1.593240in}}%
\pgfpathlineto{\pgfqpoint{2.641514in}{1.603479in}}%
\pgfpathlineto{\pgfqpoint{2.642399in}{1.599468in}}%
\pgfpathlineto{\pgfqpoint{2.642694in}{1.599031in}}%
\pgfpathlineto{\pgfqpoint{2.643283in}{1.601022in}}%
\pgfpathlineto{\pgfqpoint{2.643873in}{1.603219in}}%
\pgfpathlineto{\pgfqpoint{2.644463in}{1.599875in}}%
\pgfpathlineto{\pgfqpoint{2.645642in}{1.583815in}}%
\pgfpathlineto{\pgfqpoint{2.646379in}{1.592414in}}%
\pgfpathlineto{\pgfqpoint{2.647264in}{1.602993in}}%
\pgfpathlineto{\pgfqpoint{2.648001in}{1.597121in}}%
\pgfpathlineto{\pgfqpoint{2.650654in}{1.575409in}}%
\pgfpathlineto{\pgfqpoint{2.651391in}{1.576929in}}%
\pgfpathlineto{\pgfqpoint{2.651834in}{1.575612in}}%
\pgfpathlineto{\pgfqpoint{2.652571in}{1.571462in}}%
\pgfpathlineto{\pgfqpoint{2.653013in}{1.575112in}}%
\pgfpathlineto{\pgfqpoint{2.654192in}{1.592023in}}%
\pgfpathlineto{\pgfqpoint{2.654929in}{1.582251in}}%
\pgfpathlineto{\pgfqpoint{2.655961in}{1.568123in}}%
\pgfpathlineto{\pgfqpoint{2.656846in}{1.572013in}}%
\pgfpathlineto{\pgfqpoint{2.657878in}{1.571622in}}%
\pgfpathlineto{\pgfqpoint{2.660089in}{1.589079in}}%
\pgfpathlineto{\pgfqpoint{2.660826in}{1.590951in}}%
\pgfpathlineto{\pgfqpoint{2.661416in}{1.589044in}}%
\pgfpathlineto{\pgfqpoint{2.662743in}{1.577235in}}%
\pgfpathlineto{\pgfqpoint{2.663480in}{1.583053in}}%
\pgfpathlineto{\pgfqpoint{2.664806in}{1.598056in}}%
\pgfpathlineto{\pgfqpoint{2.665691in}{1.595797in}}%
\pgfpathlineto{\pgfqpoint{2.666428in}{1.600878in}}%
\pgfpathlineto{\pgfqpoint{2.667165in}{1.606317in}}%
\pgfpathlineto{\pgfqpoint{2.667755in}{1.600861in}}%
\pgfpathlineto{\pgfqpoint{2.668934in}{1.583919in}}%
\pgfpathlineto{\pgfqpoint{2.669819in}{1.588674in}}%
\pgfpathlineto{\pgfqpoint{2.672030in}{1.597608in}}%
\pgfpathlineto{\pgfqpoint{2.672472in}{1.595360in}}%
\pgfpathlineto{\pgfqpoint{2.673209in}{1.590465in}}%
\pgfpathlineto{\pgfqpoint{2.674241in}{1.593473in}}%
\pgfpathlineto{\pgfqpoint{2.674978in}{1.584486in}}%
\pgfpathlineto{\pgfqpoint{2.675715in}{1.573448in}}%
\pgfpathlineto{\pgfqpoint{2.676452in}{1.581355in}}%
\pgfpathlineto{\pgfqpoint{2.677632in}{1.595860in}}%
\pgfpathlineto{\pgfqpoint{2.678369in}{1.591801in}}%
\pgfpathlineto{\pgfqpoint{2.681907in}{1.573296in}}%
\pgfpathlineto{\pgfqpoint{2.682054in}{1.573475in}}%
\pgfpathlineto{\pgfqpoint{2.683823in}{1.581811in}}%
\pgfpathlineto{\pgfqpoint{2.684708in}{1.592722in}}%
\pgfpathlineto{\pgfqpoint{2.685298in}{1.586075in}}%
\pgfpathlineto{\pgfqpoint{2.686182in}{1.572970in}}%
\pgfpathlineto{\pgfqpoint{2.686919in}{1.583545in}}%
\pgfpathlineto{\pgfqpoint{2.687361in}{1.587576in}}%
\pgfpathlineto{\pgfqpoint{2.688099in}{1.580606in}}%
\pgfpathlineto{\pgfqpoint{2.688541in}{1.577867in}}%
\pgfpathlineto{\pgfqpoint{2.689130in}{1.584583in}}%
\pgfpathlineto{\pgfqpoint{2.691489in}{1.605614in}}%
\pgfpathlineto{\pgfqpoint{2.691637in}{1.605322in}}%
\pgfpathlineto{\pgfqpoint{2.693258in}{1.591691in}}%
\pgfpathlineto{\pgfqpoint{2.694290in}{1.598484in}}%
\pgfpathlineto{\pgfqpoint{2.694880in}{1.600248in}}%
\pgfpathlineto{\pgfqpoint{2.695764in}{1.598163in}}%
\pgfpathlineto{\pgfqpoint{2.696059in}{1.598211in}}%
\pgfpathlineto{\pgfqpoint{2.696354in}{1.599642in}}%
\pgfpathlineto{\pgfqpoint{2.697533in}{1.610178in}}%
\pgfpathlineto{\pgfqpoint{2.698123in}{1.604547in}}%
\pgfpathlineto{\pgfqpoint{2.699155in}{1.589706in}}%
\pgfpathlineto{\pgfqpoint{2.700039in}{1.595071in}}%
\pgfpathlineto{\pgfqpoint{2.702103in}{1.599120in}}%
\pgfpathlineto{\pgfqpoint{2.702251in}{1.599033in}}%
\pgfpathlineto{\pgfqpoint{2.703283in}{1.591810in}}%
\pgfpathlineto{\pgfqpoint{2.706231in}{1.565816in}}%
\pgfpathlineto{\pgfqpoint{2.707115in}{1.577829in}}%
\pgfpathlineto{\pgfqpoint{2.707853in}{1.584646in}}%
\pgfpathlineto{\pgfqpoint{2.708737in}{1.579643in}}%
\pgfpathlineto{\pgfqpoint{2.709179in}{1.578401in}}%
\pgfpathlineto{\pgfqpoint{2.709916in}{1.581387in}}%
\pgfpathlineto{\pgfqpoint{2.710211in}{1.581948in}}%
\pgfpathlineto{\pgfqpoint{2.710801in}{1.579079in}}%
\pgfpathlineto{\pgfqpoint{2.711833in}{1.573563in}}%
\pgfpathlineto{\pgfqpoint{2.712570in}{1.577028in}}%
\pgfpathlineto{\pgfqpoint{2.715076in}{1.598379in}}%
\pgfpathlineto{\pgfqpoint{2.715961in}{1.592298in}}%
\pgfpathlineto{\pgfqpoint{2.716698in}{1.588124in}}%
\pgfpathlineto{\pgfqpoint{2.717287in}{1.592770in}}%
\pgfpathlineto{\pgfqpoint{2.720531in}{1.615088in}}%
\pgfpathlineto{\pgfqpoint{2.720678in}{1.615221in}}%
\pgfpathlineto{\pgfqpoint{2.720973in}{1.613683in}}%
\pgfpathlineto{\pgfqpoint{2.723774in}{1.593902in}}%
\pgfpathlineto{\pgfqpoint{2.724216in}{1.596646in}}%
\pgfpathlineto{\pgfqpoint{2.725101in}{1.605931in}}%
\pgfpathlineto{\pgfqpoint{2.725838in}{1.598979in}}%
\pgfpathlineto{\pgfqpoint{2.726575in}{1.593252in}}%
\pgfpathlineto{\pgfqpoint{2.727312in}{1.597649in}}%
\pgfpathlineto{\pgfqpoint{2.727459in}{1.597996in}}%
\pgfpathlineto{\pgfqpoint{2.727902in}{1.595486in}}%
\pgfpathlineto{\pgfqpoint{2.729523in}{1.572735in}}%
\pgfpathlineto{\pgfqpoint{2.730408in}{1.581319in}}%
\pgfpathlineto{\pgfqpoint{2.732471in}{1.587425in}}%
\pgfpathlineto{\pgfqpoint{2.733061in}{1.589377in}}%
\pgfpathlineto{\pgfqpoint{2.733651in}{1.586529in}}%
\pgfpathlineto{\pgfqpoint{2.734240in}{1.583430in}}%
\pgfpathlineto{\pgfqpoint{2.734830in}{1.587773in}}%
\pgfpathlineto{\pgfqpoint{2.735420in}{1.592666in}}%
\pgfpathlineto{\pgfqpoint{2.736304in}{1.587125in}}%
\pgfpathlineto{\pgfqpoint{2.736452in}{1.587010in}}%
\pgfpathlineto{\pgfqpoint{2.736599in}{1.587899in}}%
\pgfpathlineto{\pgfqpoint{2.738073in}{1.618615in}}%
\pgfpathlineto{\pgfqpoint{2.739253in}{1.605124in}}%
\pgfpathlineto{\pgfqpoint{2.739400in}{1.604842in}}%
\pgfpathlineto{\pgfqpoint{2.739842in}{1.607220in}}%
\pgfpathlineto{\pgfqpoint{2.740432in}{1.609920in}}%
\pgfpathlineto{\pgfqpoint{2.741169in}{1.605745in}}%
\pgfpathlineto{\pgfqpoint{2.741464in}{1.604692in}}%
\pgfpathlineto{\pgfqpoint{2.742054in}{1.607480in}}%
\pgfpathlineto{\pgfqpoint{2.744855in}{1.619629in}}%
\pgfpathlineto{\pgfqpoint{2.745297in}{1.617558in}}%
\pgfpathlineto{\pgfqpoint{2.746771in}{1.597367in}}%
\pgfpathlineto{\pgfqpoint{2.747656in}{1.606253in}}%
\pgfpathlineto{\pgfqpoint{2.748245in}{1.610639in}}%
\pgfpathlineto{\pgfqpoint{2.749130in}{1.605847in}}%
\pgfpathlineto{\pgfqpoint{2.749277in}{1.605545in}}%
\pgfpathlineto{\pgfqpoint{2.749719in}{1.608020in}}%
\pgfpathlineto{\pgfqpoint{2.750604in}{1.615702in}}%
\pgfpathlineto{\pgfqpoint{2.751341in}{1.608848in}}%
\pgfpathlineto{\pgfqpoint{2.753405in}{1.599701in}}%
\pgfpathlineto{\pgfqpoint{2.753700in}{1.598875in}}%
\pgfpathlineto{\pgfqpoint{2.754142in}{1.600781in}}%
\pgfpathlineto{\pgfqpoint{2.755616in}{1.620515in}}%
\pgfpathlineto{\pgfqpoint{2.756353in}{1.612419in}}%
\pgfpathlineto{\pgfqpoint{2.756943in}{1.607803in}}%
\pgfpathlineto{\pgfqpoint{2.757827in}{1.612706in}}%
\pgfpathlineto{\pgfqpoint{2.757975in}{1.612806in}}%
\pgfpathlineto{\pgfqpoint{2.758122in}{1.612257in}}%
\pgfpathlineto{\pgfqpoint{2.759449in}{1.595638in}}%
\pgfpathlineto{\pgfqpoint{2.760186in}{1.607007in}}%
\pgfpathlineto{\pgfqpoint{2.761218in}{1.620376in}}%
\pgfpathlineto{\pgfqpoint{2.762103in}{1.618494in}}%
\pgfpathlineto{\pgfqpoint{2.763135in}{1.613732in}}%
\pgfpathlineto{\pgfqpoint{2.764756in}{1.596823in}}%
\pgfpathlineto{\pgfqpoint{2.765788in}{1.599054in}}%
\pgfpathlineto{\pgfqpoint{2.766673in}{1.600080in}}%
\pgfpathlineto{\pgfqpoint{2.768294in}{1.616496in}}%
\pgfpathlineto{\pgfqpoint{2.769031in}{1.605799in}}%
\pgfpathlineto{\pgfqpoint{2.769916in}{1.591429in}}%
\pgfpathlineto{\pgfqpoint{2.770653in}{1.599334in}}%
\pgfpathlineto{\pgfqpoint{2.773749in}{1.626222in}}%
\pgfpathlineto{\pgfqpoint{2.774043in}{1.624781in}}%
\pgfpathlineto{\pgfqpoint{2.776844in}{1.605442in}}%
\pgfpathlineto{\pgfqpoint{2.777287in}{1.609622in}}%
\pgfpathlineto{\pgfqpoint{2.778613in}{1.630000in}}%
\pgfpathlineto{\pgfqpoint{2.779498in}{1.623730in}}%
\pgfpathlineto{\pgfqpoint{2.780530in}{1.624407in}}%
\pgfpathlineto{\pgfqpoint{2.781709in}{1.609737in}}%
\pgfpathlineto{\pgfqpoint{2.782741in}{1.595840in}}%
\pgfpathlineto{\pgfqpoint{2.783478in}{1.602217in}}%
\pgfpathlineto{\pgfqpoint{2.785690in}{1.612953in}}%
\pgfpathlineto{\pgfqpoint{2.785984in}{1.612216in}}%
\pgfpathlineto{\pgfqpoint{2.789522in}{1.583090in}}%
\pgfpathlineto{\pgfqpoint{2.789817in}{1.585903in}}%
\pgfpathlineto{\pgfqpoint{2.791291in}{1.622655in}}%
\pgfpathlineto{\pgfqpoint{2.792471in}{1.612703in}}%
\pgfpathlineto{\pgfqpoint{2.793060in}{1.613406in}}%
\pgfpathlineto{\pgfqpoint{2.793798in}{1.610076in}}%
\pgfpathlineto{\pgfqpoint{2.794682in}{1.603182in}}%
\pgfpathlineto{\pgfqpoint{2.795272in}{1.608941in}}%
\pgfpathlineto{\pgfqpoint{2.797925in}{1.625569in}}%
\pgfpathlineto{\pgfqpoint{2.798220in}{1.626669in}}%
\pgfpathlineto{\pgfqpoint{2.798662in}{1.623124in}}%
\pgfpathlineto{\pgfqpoint{2.800137in}{1.595227in}}%
\pgfpathlineto{\pgfqpoint{2.801168in}{1.605554in}}%
\pgfpathlineto{\pgfqpoint{2.801463in}{1.606416in}}%
\pgfpathlineto{\pgfqpoint{2.802200in}{1.603650in}}%
\pgfpathlineto{\pgfqpoint{2.802348in}{1.603603in}}%
\pgfpathlineto{\pgfqpoint{2.802643in}{1.605075in}}%
\pgfpathlineto{\pgfqpoint{2.803822in}{1.616709in}}%
\pgfpathlineto{\pgfqpoint{2.804412in}{1.610961in}}%
\pgfpathlineto{\pgfqpoint{2.806623in}{1.599246in}}%
\pgfpathlineto{\pgfqpoint{2.807065in}{1.597761in}}%
\pgfpathlineto{\pgfqpoint{2.807507in}{1.601886in}}%
\pgfpathlineto{\pgfqpoint{2.808982in}{1.630363in}}%
\pgfpathlineto{\pgfqpoint{2.810014in}{1.623268in}}%
\pgfpathlineto{\pgfqpoint{2.810456in}{1.624544in}}%
\pgfpathlineto{\pgfqpoint{2.811046in}{1.626994in}}%
\pgfpathlineto{\pgfqpoint{2.811488in}{1.623359in}}%
\pgfpathlineto{\pgfqpoint{2.812667in}{1.606423in}}%
\pgfpathlineto{\pgfqpoint{2.813552in}{1.613857in}}%
\pgfpathlineto{\pgfqpoint{2.815615in}{1.621510in}}%
\pgfpathlineto{\pgfqpoint{2.815910in}{1.620883in}}%
\pgfpathlineto{\pgfqpoint{2.816942in}{1.602723in}}%
\pgfpathlineto{\pgfqpoint{2.817974in}{1.585631in}}%
\pgfpathlineto{\pgfqpoint{2.818859in}{1.590198in}}%
\pgfpathlineto{\pgfqpoint{2.819154in}{1.590859in}}%
\pgfpathlineto{\pgfqpoint{2.819743in}{1.588443in}}%
\pgfpathlineto{\pgfqpoint{2.819891in}{1.588045in}}%
\pgfpathlineto{\pgfqpoint{2.820333in}{1.590518in}}%
\pgfpathlineto{\pgfqpoint{2.821807in}{1.618118in}}%
\pgfpathlineto{\pgfqpoint{2.822692in}{1.606684in}}%
\pgfpathlineto{\pgfqpoint{2.823281in}{1.602771in}}%
\pgfpathlineto{\pgfqpoint{2.824313in}{1.604576in}}%
\pgfpathlineto{\pgfqpoint{2.824755in}{1.603332in}}%
\pgfpathlineto{\pgfqpoint{2.825198in}{1.606236in}}%
\pgfpathlineto{\pgfqpoint{2.828293in}{1.633723in}}%
\pgfpathlineto{\pgfqpoint{2.828588in}{1.634905in}}%
\pgfpathlineto{\pgfqpoint{2.829031in}{1.631273in}}%
\pgfpathlineto{\pgfqpoint{2.830357in}{1.604240in}}%
\pgfpathlineto{\pgfqpoint{2.831242in}{1.617409in}}%
\pgfpathlineto{\pgfqpoint{2.832126in}{1.627111in}}%
\pgfpathlineto{\pgfqpoint{2.833158in}{1.625625in}}%
\pgfpathlineto{\pgfqpoint{2.833748in}{1.626000in}}%
\pgfpathlineto{\pgfqpoint{2.834043in}{1.625011in}}%
\pgfpathlineto{\pgfqpoint{2.835370in}{1.602602in}}%
\pgfpathlineto{\pgfqpoint{2.837433in}{1.593199in}}%
\pgfpathlineto{\pgfqpoint{2.837876in}{1.597002in}}%
\pgfpathlineto{\pgfqpoint{2.839202in}{1.624171in}}%
\pgfpathlineto{\pgfqpoint{2.839940in}{1.612339in}}%
\pgfpathlineto{\pgfqpoint{2.840824in}{1.598956in}}%
\pgfpathlineto{\pgfqpoint{2.841856in}{1.602416in}}%
\pgfpathlineto{\pgfqpoint{2.842888in}{1.589861in}}%
\pgfpathlineto{\pgfqpoint{2.843478in}{1.597965in}}%
\pgfpathlineto{\pgfqpoint{2.845836in}{1.619257in}}%
\pgfpathlineto{\pgfqpoint{2.846131in}{1.619804in}}%
\pgfpathlineto{\pgfqpoint{2.846573in}{1.616583in}}%
\pgfpathlineto{\pgfqpoint{2.847900in}{1.589348in}}%
\pgfpathlineto{\pgfqpoint{2.848932in}{1.600299in}}%
\pgfpathlineto{\pgfqpoint{2.851733in}{1.618592in}}%
\pgfpathlineto{\pgfqpoint{2.849964in}{1.599829in}}%
\pgfpathlineto{\pgfqpoint{2.852323in}{1.613652in}}%
\pgfpathlineto{\pgfqpoint{2.853797in}{1.590312in}}%
\pgfpathlineto{\pgfqpoint{2.854681in}{1.595569in}}%
\pgfpathlineto{\pgfqpoint{2.856745in}{1.620444in}}%
\pgfpathlineto{\pgfqpoint{2.857188in}{1.623116in}}%
\pgfpathlineto{\pgfqpoint{2.857925in}{1.618169in}}%
\pgfpathlineto{\pgfqpoint{2.858219in}{1.616968in}}%
\pgfpathlineto{\pgfqpoint{2.859104in}{1.620255in}}%
\pgfpathlineto{\pgfqpoint{2.859694in}{1.613511in}}%
\pgfpathlineto{\pgfqpoint{2.860578in}{1.597364in}}%
\pgfpathlineto{\pgfqpoint{2.861315in}{1.611387in}}%
\pgfpathlineto{\pgfqpoint{2.862347in}{1.627234in}}%
\pgfpathlineto{\pgfqpoint{2.863232in}{1.624718in}}%
\pgfpathlineto{\pgfqpoint{2.863821in}{1.625154in}}%
\pgfpathlineto{\pgfqpoint{2.864116in}{1.624209in}}%
\pgfpathlineto{\pgfqpoint{2.867065in}{1.594746in}}%
\pgfpathlineto{\pgfqpoint{2.868096in}{1.603949in}}%
\pgfpathlineto{\pgfqpoint{2.869276in}{1.617273in}}%
\pgfpathlineto{\pgfqpoint{2.870013in}{1.609854in}}%
\pgfpathlineto{\pgfqpoint{2.871340in}{1.583838in}}%
\pgfpathlineto{\pgfqpoint{2.872372in}{1.588123in}}%
\pgfpathlineto{\pgfqpoint{2.873109in}{1.585068in}}%
\pgfpathlineto{\pgfqpoint{2.873551in}{1.587866in}}%
\pgfpathlineto{\pgfqpoint{2.875025in}{1.615836in}}%
\pgfpathlineto{\pgfqpoint{2.876352in}{1.610204in}}%
\pgfpathlineto{\pgfqpoint{2.876942in}{1.608727in}}%
\pgfpathlineto{\pgfqpoint{2.878121in}{1.597352in}}%
\pgfpathlineto{\pgfqpoint{2.878711in}{1.605089in}}%
\pgfpathlineto{\pgfqpoint{2.879890in}{1.623806in}}%
\pgfpathlineto{\pgfqpoint{2.880922in}{1.621545in}}%
\pgfpathlineto{\pgfqpoint{2.881954in}{1.627614in}}%
\pgfpathlineto{\pgfqpoint{2.882396in}{1.623487in}}%
\pgfpathlineto{\pgfqpoint{2.883870in}{1.601053in}}%
\pgfpathlineto{\pgfqpoint{2.884755in}{1.607410in}}%
\pgfpathlineto{\pgfqpoint{2.886966in}{1.626386in}}%
\pgfpathlineto{\pgfqpoint{2.887703in}{1.620833in}}%
\pgfpathlineto{\pgfqpoint{2.890799in}{1.592551in}}%
\pgfpathlineto{\pgfqpoint{2.891094in}{1.593568in}}%
\pgfpathlineto{\pgfqpoint{2.892568in}{1.620093in}}%
\pgfpathlineto{\pgfqpoint{2.893452in}{1.606232in}}%
\pgfpathlineto{\pgfqpoint{2.895959in}{1.579892in}}%
\pgfpathlineto{\pgfqpoint{2.896253in}{1.581176in}}%
\pgfpathlineto{\pgfqpoint{2.899349in}{1.616752in}}%
\pgfpathlineto{\pgfqpoint{2.899791in}{1.613712in}}%
\pgfpathlineto{\pgfqpoint{2.901118in}{1.587077in}}%
\pgfpathlineto{\pgfqpoint{2.902150in}{1.596034in}}%
\pgfpathlineto{\pgfqpoint{2.905393in}{1.623097in}}%
\pgfpathlineto{\pgfqpoint{2.905836in}{1.620789in}}%
\pgfpathlineto{\pgfqpoint{2.908342in}{1.595669in}}%
\pgfpathlineto{\pgfqpoint{2.908784in}{1.598566in}}%
\pgfpathlineto{\pgfqpoint{2.910258in}{1.623768in}}%
\pgfpathlineto{\pgfqpoint{2.911290in}{1.615525in}}%
\pgfpathlineto{\pgfqpoint{2.911880in}{1.618703in}}%
\pgfpathlineto{\pgfqpoint{2.912175in}{1.620062in}}%
\pgfpathlineto{\pgfqpoint{2.912617in}{1.616860in}}%
\pgfpathlineto{\pgfqpoint{2.913944in}{1.593645in}}%
\pgfpathlineto{\pgfqpoint{2.914828in}{1.601645in}}%
\pgfpathlineto{\pgfqpoint{2.916892in}{1.616749in}}%
\pgfpathlineto{\pgfqpoint{2.917334in}{1.614095in}}%
\pgfpathlineto{\pgfqpoint{2.920725in}{1.592260in}}%
\pgfpathlineto{\pgfqpoint{2.921167in}{1.591649in}}%
\pgfpathlineto{\pgfqpoint{2.921609in}{1.593859in}}%
\pgfpathlineto{\pgfqpoint{2.922936in}{1.608535in}}%
\pgfpathlineto{\pgfqpoint{2.923673in}{1.602879in}}%
\pgfpathlineto{\pgfqpoint{2.926179in}{1.587543in}}%
\pgfpathlineto{\pgfqpoint{2.926622in}{1.586319in}}%
\pgfpathlineto{\pgfqpoint{2.927064in}{1.589382in}}%
\pgfpathlineto{\pgfqpoint{2.930012in}{1.613733in}}%
\pgfpathlineto{\pgfqpoint{2.930307in}{1.612180in}}%
\pgfpathlineto{\pgfqpoint{2.931486in}{1.589368in}}%
\pgfpathlineto{\pgfqpoint{2.932371in}{1.602245in}}%
\pgfpathlineto{\pgfqpoint{2.934582in}{1.614785in}}%
\pgfpathlineto{\pgfqpoint{2.935172in}{1.617688in}}%
\pgfpathlineto{\pgfqpoint{2.935909in}{1.613533in}}%
\pgfpathlineto{\pgfqpoint{2.938120in}{1.598906in}}%
\pgfpathlineto{\pgfqpoint{2.938710in}{1.603145in}}%
\pgfpathlineto{\pgfqpoint{2.940479in}{1.620072in}}%
\pgfpathlineto{\pgfqpoint{2.941069in}{1.615150in}}%
\pgfpathlineto{\pgfqpoint{2.942101in}{1.601764in}}%
\pgfpathlineto{\pgfqpoint{2.943280in}{1.604223in}}%
\pgfpathlineto{\pgfqpoint{2.944312in}{1.589603in}}%
\pgfpathlineto{\pgfqpoint{2.944902in}{1.598173in}}%
\pgfpathlineto{\pgfqpoint{2.945786in}{1.611000in}}%
\pgfpathlineto{\pgfqpoint{2.946671in}{1.604957in}}%
\pgfpathlineto{\pgfqpoint{2.947997in}{1.598853in}}%
\pgfpathlineto{\pgfqpoint{2.948882in}{1.583942in}}%
\pgfpathlineto{\pgfqpoint{2.949766in}{1.593484in}}%
\pgfpathlineto{\pgfqpoint{2.950209in}{1.596144in}}%
\pgfpathlineto{\pgfqpoint{2.951093in}{1.591857in}}%
\pgfpathlineto{\pgfqpoint{2.951683in}{1.597633in}}%
\pgfpathlineto{\pgfqpoint{2.952715in}{1.609091in}}%
\pgfpathlineto{\pgfqpoint{2.953452in}{1.604623in}}%
\pgfpathlineto{\pgfqpoint{2.955073in}{1.589989in}}%
\pgfpathlineto{\pgfqpoint{2.955810in}{1.595819in}}%
\pgfpathlineto{\pgfqpoint{2.958317in}{1.614379in}}%
\pgfpathlineto{\pgfqpoint{2.958611in}{1.614003in}}%
\pgfpathlineto{\pgfqpoint{2.960970in}{1.598644in}}%
\pgfpathlineto{\pgfqpoint{2.961707in}{1.588876in}}%
\pgfpathlineto{\pgfqpoint{2.962444in}{1.599776in}}%
\pgfpathlineto{\pgfqpoint{2.963476in}{1.615804in}}%
\pgfpathlineto{\pgfqpoint{2.964361in}{1.611963in}}%
\pgfpathlineto{\pgfqpoint{2.965982in}{1.606597in}}%
\pgfpathlineto{\pgfqpoint{2.967604in}{1.592563in}}%
\pgfpathlineto{\pgfqpoint{2.968341in}{1.593405in}}%
\pgfpathlineto{\pgfqpoint{2.969226in}{1.604480in}}%
\pgfpathlineto{\pgfqpoint{2.970257in}{1.619626in}}%
\pgfpathlineto{\pgfqpoint{2.970995in}{1.613502in}}%
\pgfpathlineto{\pgfqpoint{2.973796in}{1.592515in}}%
\pgfpathlineto{\pgfqpoint{2.974385in}{1.591523in}}%
\pgfpathlineto{\pgfqpoint{2.974827in}{1.593052in}}%
\pgfpathlineto{\pgfqpoint{2.976449in}{1.614448in}}%
\pgfpathlineto{\pgfqpoint{2.977334in}{1.602981in}}%
\pgfpathlineto{\pgfqpoint{2.979692in}{1.576964in}}%
\pgfpathlineto{\pgfqpoint{2.979987in}{1.579412in}}%
\pgfpathlineto{\pgfqpoint{2.983083in}{1.607576in}}%
\pgfpathlineto{\pgfqpoint{2.983230in}{1.607706in}}%
\pgfpathlineto{\pgfqpoint{2.983378in}{1.606599in}}%
\pgfpathlineto{\pgfqpoint{2.984852in}{1.570524in}}%
\pgfpathlineto{\pgfqpoint{2.986474in}{1.581323in}}%
\pgfpathlineto{\pgfqpoint{2.988980in}{1.605616in}}%
\pgfpathlineto{\pgfqpoint{2.989864in}{1.601485in}}%
\pgfpathlineto{\pgfqpoint{2.991928in}{1.590004in}}%
\pgfpathlineto{\pgfqpoint{2.992223in}{1.591413in}}%
\pgfpathlineto{\pgfqpoint{2.993992in}{1.627976in}}%
\pgfpathlineto{\pgfqpoint{2.995908in}{1.618774in}}%
\pgfpathlineto{\pgfqpoint{2.996351in}{1.614845in}}%
\pgfpathlineto{\pgfqpoint{2.997530in}{1.593970in}}%
\pgfpathlineto{\pgfqpoint{2.998267in}{1.602961in}}%
\pgfpathlineto{\pgfqpoint{3.000626in}{1.623171in}}%
\pgfpathlineto{\pgfqpoint{3.001068in}{1.617852in}}%
\pgfpathlineto{\pgfqpoint{3.004016in}{1.585118in}}%
\pgfpathlineto{\pgfqpoint{3.004311in}{1.584874in}}%
\pgfpathlineto{\pgfqpoint{3.004606in}{1.586578in}}%
\pgfpathlineto{\pgfqpoint{3.006522in}{1.616631in}}%
\pgfpathlineto{\pgfqpoint{3.007702in}{1.604382in}}%
\pgfpathlineto{\pgfqpoint{3.009323in}{1.588199in}}%
\pgfpathlineto{\pgfqpoint{3.009913in}{1.591564in}}%
\pgfpathlineto{\pgfqpoint{3.013304in}{1.621347in}}%
\pgfpathlineto{\pgfqpoint{3.013451in}{1.621275in}}%
\pgfpathlineto{\pgfqpoint{3.014188in}{1.611137in}}%
\pgfpathlineto{\pgfqpoint{3.015220in}{1.588969in}}%
\pgfpathlineto{\pgfqpoint{3.015957in}{1.599161in}}%
\pgfpathlineto{\pgfqpoint{3.018611in}{1.626944in}}%
\pgfpathlineto{\pgfqpoint{3.019200in}{1.621494in}}%
\pgfpathlineto{\pgfqpoint{3.022001in}{1.589700in}}%
\pgfpathlineto{\pgfqpoint{3.022296in}{1.592614in}}%
\pgfpathlineto{\pgfqpoint{3.023918in}{1.627056in}}%
\pgfpathlineto{\pgfqpoint{3.024950in}{1.619876in}}%
\pgfpathlineto{\pgfqpoint{3.027898in}{1.594348in}}%
\pgfpathlineto{\pgfqpoint{3.028193in}{1.595686in}}%
\pgfpathlineto{\pgfqpoint{3.029815in}{1.624342in}}%
\pgfpathlineto{\pgfqpoint{3.031436in}{1.619056in}}%
\pgfpathlineto{\pgfqpoint{3.032910in}{1.579583in}}%
\pgfpathlineto{\pgfqpoint{3.034090in}{1.597758in}}%
\pgfpathlineto{\pgfqpoint{3.034532in}{1.595921in}}%
\pgfpathlineto{\pgfqpoint{3.034827in}{1.594713in}}%
\pgfpathlineto{\pgfqpoint{3.035269in}{1.599521in}}%
\pgfpathlineto{\pgfqpoint{3.036301in}{1.621059in}}%
\pgfpathlineto{\pgfqpoint{3.037038in}{1.611164in}}%
\pgfpathlineto{\pgfqpoint{3.038365in}{1.581949in}}%
\pgfpathlineto{\pgfqpoint{3.038365in}{1.581949in}}%
\pgfusepath{stroke}%
\end{pgfscope}%
\begin{pgfscope}%
\pgfpathrectangle{\pgfqpoint{0.679669in}{0.526079in}}{\pgfqpoint{2.358696in}{1.661000in}} %
\pgfusepath{clip}%
\pgfsetbuttcap%
\pgfsetroundjoin%
\pgfsetlinewidth{1.003750pt}%
\definecolor{currentstroke}{rgb}{0.627451,0.321569,0.176471}%
\pgfsetstrokecolor{currentstroke}%
\pgfsetdash{{3.700000pt}{1.600000pt}}{0.000000pt}%
\pgfpathmoveto{\pgfqpoint{0.682524in}{0.512191in}}%
\pgfpathlineto{\pgfqpoint{0.686008in}{0.779359in}}%
\pgfpathlineto{\pgfqpoint{0.691168in}{0.958106in}}%
\pgfpathlineto{\pgfqpoint{0.695885in}{1.015582in}}%
\pgfpathlineto{\pgfqpoint{0.698244in}{1.027373in}}%
\pgfpathlineto{\pgfqpoint{0.698686in}{1.026829in}}%
\pgfpathlineto{\pgfqpoint{0.700603in}{1.017807in}}%
\pgfpathlineto{\pgfqpoint{0.705615in}{0.998880in}}%
\pgfpathlineto{\pgfqpoint{0.706204in}{0.998864in}}%
\pgfpathlineto{\pgfqpoint{0.706647in}{0.999802in}}%
\pgfpathlineto{\pgfqpoint{0.708268in}{1.010570in}}%
\pgfpathlineto{\pgfqpoint{0.712396in}{1.052361in}}%
\pgfpathlineto{\pgfqpoint{0.713428in}{1.048170in}}%
\pgfpathlineto{\pgfqpoint{0.718882in}{0.992966in}}%
\pgfpathlineto{\pgfqpoint{0.723158in}{0.901501in}}%
\pgfpathlineto{\pgfqpoint{0.724632in}{0.909561in}}%
\pgfpathlineto{\pgfqpoint{0.727285in}{0.933338in}}%
\pgfpathlineto{\pgfqpoint{0.731708in}{1.014855in}}%
\pgfpathlineto{\pgfqpoint{0.731855in}{1.014804in}}%
\pgfpathlineto{\pgfqpoint{0.733035in}{1.010793in}}%
\pgfpathlineto{\pgfqpoint{0.736573in}{0.991286in}}%
\pgfpathlineto{\pgfqpoint{0.739963in}{0.931697in}}%
\pgfpathlineto{\pgfqpoint{0.740848in}{0.939990in}}%
\pgfpathlineto{\pgfqpoint{0.744533in}{0.979325in}}%
\pgfpathlineto{\pgfqpoint{0.746007in}{0.988434in}}%
\pgfpathlineto{\pgfqpoint{0.752641in}{1.050703in}}%
\pgfpathlineto{\pgfqpoint{0.753526in}{1.051766in}}%
\pgfpathlineto{\pgfqpoint{0.754115in}{1.050852in}}%
\pgfpathlineto{\pgfqpoint{0.755295in}{1.042417in}}%
\pgfpathlineto{\pgfqpoint{0.757359in}{0.992134in}}%
\pgfpathlineto{\pgfqpoint{0.760160in}{0.958757in}}%
\pgfpathlineto{\pgfqpoint{0.762518in}{0.928389in}}%
\pgfpathlineto{\pgfqpoint{0.763550in}{0.915085in}}%
\pgfpathlineto{\pgfqpoint{0.764287in}{0.923104in}}%
\pgfpathlineto{\pgfqpoint{0.768857in}{1.005406in}}%
\pgfpathlineto{\pgfqpoint{0.771953in}{1.015573in}}%
\pgfpathlineto{\pgfqpoint{0.772838in}{1.011460in}}%
\pgfpathlineto{\pgfqpoint{0.775049in}{0.987493in}}%
\pgfpathlineto{\pgfqpoint{0.775933in}{0.993198in}}%
\pgfpathlineto{\pgfqpoint{0.781388in}{1.074023in}}%
\pgfpathlineto{\pgfqpoint{0.784631in}{1.119687in}}%
\pgfpathlineto{\pgfqpoint{0.784779in}{1.119573in}}%
\pgfpathlineto{\pgfqpoint{0.786695in}{1.116536in}}%
\pgfpathlineto{\pgfqpoint{0.787579in}{1.117813in}}%
\pgfpathlineto{\pgfqpoint{0.788317in}{1.118484in}}%
\pgfpathlineto{\pgfqpoint{0.788906in}{1.117209in}}%
\pgfpathlineto{\pgfqpoint{0.790233in}{1.104361in}}%
\pgfpathlineto{\pgfqpoint{0.794950in}{1.055180in}}%
\pgfpathlineto{\pgfqpoint{0.797309in}{1.044227in}}%
\pgfpathlineto{\pgfqpoint{0.798194in}{1.049204in}}%
\pgfpathlineto{\pgfqpoint{0.801879in}{1.112228in}}%
\pgfpathlineto{\pgfqpoint{0.808365in}{1.201801in}}%
\pgfpathlineto{\pgfqpoint{0.809692in}{1.198508in}}%
\pgfpathlineto{\pgfqpoint{0.811756in}{1.181654in}}%
\pgfpathlineto{\pgfqpoint{0.814557in}{1.123485in}}%
\pgfpathlineto{\pgfqpoint{0.816916in}{1.083435in}}%
\pgfpathlineto{\pgfqpoint{0.817505in}{1.085780in}}%
\pgfpathlineto{\pgfqpoint{0.819274in}{1.117308in}}%
\pgfpathlineto{\pgfqpoint{0.824876in}{1.214115in}}%
\pgfpathlineto{\pgfqpoint{0.826793in}{1.219752in}}%
\pgfpathlineto{\pgfqpoint{0.827235in}{1.219354in}}%
\pgfpathlineto{\pgfqpoint{0.828414in}{1.213324in}}%
\pgfpathlineto{\pgfqpoint{0.831215in}{1.173357in}}%
\pgfpathlineto{\pgfqpoint{0.833574in}{1.148721in}}%
\pgfpathlineto{\pgfqpoint{0.834164in}{1.150637in}}%
\pgfpathlineto{\pgfqpoint{0.835785in}{1.175618in}}%
\pgfpathlineto{\pgfqpoint{0.841977in}{1.281198in}}%
\pgfpathlineto{\pgfqpoint{0.842419in}{1.281564in}}%
\pgfpathlineto{\pgfqpoint{0.843009in}{1.280250in}}%
\pgfpathlineto{\pgfqpoint{0.844630in}{1.267648in}}%
\pgfpathlineto{\pgfqpoint{0.851264in}{1.180720in}}%
\pgfpathlineto{\pgfqpoint{0.852591in}{1.189790in}}%
\pgfpathlineto{\pgfqpoint{0.860109in}{1.289205in}}%
\pgfpathlineto{\pgfqpoint{0.861289in}{1.285064in}}%
\pgfpathlineto{\pgfqpoint{0.863500in}{1.255475in}}%
\pgfpathlineto{\pgfqpoint{0.868954in}{1.161840in}}%
\pgfpathlineto{\pgfqpoint{0.869839in}{1.167862in}}%
\pgfpathlineto{\pgfqpoint{0.878389in}{1.297626in}}%
\pgfpathlineto{\pgfqpoint{0.879716in}{1.292197in}}%
\pgfpathlineto{\pgfqpoint{0.886645in}{1.232734in}}%
\pgfpathlineto{\pgfqpoint{0.887824in}{1.241347in}}%
\pgfpathlineto{\pgfqpoint{0.895785in}{1.347586in}}%
\pgfpathlineto{\pgfqpoint{0.896374in}{1.346666in}}%
\pgfpathlineto{\pgfqpoint{0.897701in}{1.336515in}}%
\pgfpathlineto{\pgfqpoint{0.905072in}{1.236796in}}%
\pgfpathlineto{\pgfqpoint{0.906251in}{1.247672in}}%
\pgfpathlineto{\pgfqpoint{0.912443in}{1.331422in}}%
\pgfpathlineto{\pgfqpoint{0.914212in}{1.340864in}}%
\pgfpathlineto{\pgfqpoint{0.914802in}{1.339798in}}%
\pgfpathlineto{\pgfqpoint{0.916423in}{1.326403in}}%
\pgfpathlineto{\pgfqpoint{0.922320in}{1.276170in}}%
\pgfpathlineto{\pgfqpoint{0.922762in}{1.275583in}}%
\pgfpathlineto{\pgfqpoint{0.923352in}{1.276839in}}%
\pgfpathlineto{\pgfqpoint{0.925121in}{1.291235in}}%
\pgfpathlineto{\pgfqpoint{0.930723in}{1.334526in}}%
\pgfpathlineto{\pgfqpoint{0.932344in}{1.339498in}}%
\pgfpathlineto{\pgfqpoint{0.932934in}{1.338476in}}%
\pgfpathlineto{\pgfqpoint{0.934408in}{1.327281in}}%
\pgfpathlineto{\pgfqpoint{0.940158in}{1.276935in}}%
\pgfpathlineto{\pgfqpoint{0.940600in}{1.276516in}}%
\pgfpathlineto{\pgfqpoint{0.941190in}{1.277839in}}%
\pgfpathlineto{\pgfqpoint{0.942664in}{1.292544in}}%
\pgfpathlineto{\pgfqpoint{0.946497in}{1.328167in}}%
\pgfpathlineto{\pgfqpoint{0.949445in}{1.335663in}}%
\pgfpathlineto{\pgfqpoint{0.950035in}{1.334504in}}%
\pgfpathlineto{\pgfqpoint{0.951361in}{1.322661in}}%
\pgfpathlineto{\pgfqpoint{0.956374in}{1.271121in}}%
\pgfpathlineto{\pgfqpoint{0.958585in}{1.258259in}}%
\pgfpathlineto{\pgfqpoint{0.959322in}{1.262247in}}%
\pgfpathlineto{\pgfqpoint{0.967577in}{1.355860in}}%
\pgfpathlineto{\pgfqpoint{0.968462in}{1.352894in}}%
\pgfpathlineto{\pgfqpoint{0.972295in}{1.324574in}}%
\pgfpathlineto{\pgfqpoint{0.974064in}{1.325189in}}%
\pgfpathlineto{\pgfqpoint{0.976423in}{1.317841in}}%
\pgfpathlineto{\pgfqpoint{0.977307in}{1.321082in}}%
\pgfpathlineto{\pgfqpoint{0.980698in}{1.346081in}}%
\pgfpathlineto{\pgfqpoint{0.982024in}{1.342589in}}%
\pgfpathlineto{\pgfqpoint{0.988216in}{1.312048in}}%
\pgfpathlineto{\pgfqpoint{0.989248in}{1.307778in}}%
\pgfpathlineto{\pgfqpoint{0.989985in}{1.309247in}}%
\pgfpathlineto{\pgfqpoint{0.995292in}{1.337483in}}%
\pgfpathlineto{\pgfqpoint{0.998978in}{1.369798in}}%
\pgfpathlineto{\pgfqpoint{0.999567in}{1.369246in}}%
\pgfpathlineto{\pgfqpoint{1.002221in}{1.360386in}}%
\pgfpathlineto{\pgfqpoint{1.004285in}{1.349379in}}%
\pgfpathlineto{\pgfqpoint{1.009297in}{1.304907in}}%
\pgfpathlineto{\pgfqpoint{1.010034in}{1.306548in}}%
\pgfpathlineto{\pgfqpoint{1.011950in}{1.320666in}}%
\pgfpathlineto{\pgfqpoint{1.019616in}{1.408428in}}%
\pgfpathlineto{\pgfqpoint{1.020501in}{1.407383in}}%
\pgfpathlineto{\pgfqpoint{1.022122in}{1.398946in}}%
\pgfpathlineto{\pgfqpoint{1.025513in}{1.359239in}}%
\pgfpathlineto{\pgfqpoint{1.027282in}{1.349801in}}%
\pgfpathlineto{\pgfqpoint{1.027724in}{1.350264in}}%
\pgfpathlineto{\pgfqpoint{1.029198in}{1.356667in}}%
\pgfpathlineto{\pgfqpoint{1.033474in}{1.396345in}}%
\pgfpathlineto{\pgfqpoint{1.036864in}{1.415748in}}%
\pgfpathlineto{\pgfqpoint{1.037896in}{1.416485in}}%
\pgfpathlineto{\pgfqpoint{1.038486in}{1.415769in}}%
\pgfpathlineto{\pgfqpoint{1.040402in}{1.409014in}}%
\pgfpathlineto{\pgfqpoint{1.044382in}{1.385304in}}%
\pgfpathlineto{\pgfqpoint{1.045562in}{1.388761in}}%
\pgfpathlineto{\pgfqpoint{1.053817in}{1.434913in}}%
\pgfpathlineto{\pgfqpoint{1.054702in}{1.433907in}}%
\pgfpathlineto{\pgfqpoint{1.058682in}{1.422441in}}%
\pgfpathlineto{\pgfqpoint{1.063399in}{1.391210in}}%
\pgfpathlineto{\pgfqpoint{1.064579in}{1.394022in}}%
\pgfpathlineto{\pgfqpoint{1.067380in}{1.409831in}}%
\pgfpathlineto{\pgfqpoint{1.071655in}{1.438828in}}%
\pgfpathlineto{\pgfqpoint{1.072097in}{1.438196in}}%
\pgfpathlineto{\pgfqpoint{1.076962in}{1.421540in}}%
\pgfpathlineto{\pgfqpoint{1.079615in}{1.413201in}}%
\pgfpathlineto{\pgfqpoint{1.080353in}{1.414890in}}%
\pgfpathlineto{\pgfqpoint{1.086102in}{1.446341in}}%
\pgfpathlineto{\pgfqpoint{1.090230in}{1.467256in}}%
\pgfpathlineto{\pgfqpoint{1.090524in}{1.467144in}}%
\pgfpathlineto{\pgfqpoint{1.092588in}{1.463870in}}%
\pgfpathlineto{\pgfqpoint{1.094800in}{1.456749in}}%
\pgfpathlineto{\pgfqpoint{1.098338in}{1.432336in}}%
\pgfpathlineto{\pgfqpoint{1.099370in}{1.435903in}}%
\pgfpathlineto{\pgfqpoint{1.103055in}{1.471018in}}%
\pgfpathlineto{\pgfqpoint{1.107920in}{1.513798in}}%
\pgfpathlineto{\pgfqpoint{1.108215in}{1.513670in}}%
\pgfpathlineto{\pgfqpoint{1.109689in}{1.509516in}}%
\pgfpathlineto{\pgfqpoint{1.112932in}{1.491712in}}%
\pgfpathlineto{\pgfqpoint{1.116912in}{1.458878in}}%
\pgfpathlineto{\pgfqpoint{1.117649in}{1.460333in}}%
\pgfpathlineto{\pgfqpoint{1.120745in}{1.476366in}}%
\pgfpathlineto{\pgfqpoint{1.126789in}{1.518958in}}%
\pgfpathlineto{\pgfqpoint{1.127527in}{1.518287in}}%
\pgfpathlineto{\pgfqpoint{1.129296in}{1.511940in}}%
\pgfpathlineto{\pgfqpoint{1.132539in}{1.486163in}}%
\pgfpathlineto{\pgfqpoint{1.134897in}{1.474439in}}%
\pgfpathlineto{\pgfqpoint{1.135340in}{1.474809in}}%
\pgfpathlineto{\pgfqpoint{1.136814in}{1.480681in}}%
\pgfpathlineto{\pgfqpoint{1.143595in}{1.512923in}}%
\pgfpathlineto{\pgfqpoint{1.143890in}{1.512745in}}%
\pgfpathlineto{\pgfqpoint{1.145217in}{1.508986in}}%
\pgfpathlineto{\pgfqpoint{1.152293in}{1.481439in}}%
\pgfpathlineto{\pgfqpoint{1.152735in}{1.482067in}}%
\pgfpathlineto{\pgfqpoint{1.154357in}{1.490163in}}%
\pgfpathlineto{\pgfqpoint{1.161580in}{1.528727in}}%
\pgfpathlineto{\pgfqpoint{1.162760in}{1.527551in}}%
\pgfpathlineto{\pgfqpoint{1.167182in}{1.513347in}}%
\pgfpathlineto{\pgfqpoint{1.172489in}{1.493851in}}%
\pgfpathlineto{\pgfqpoint{1.173079in}{1.494238in}}%
\pgfpathlineto{\pgfqpoint{1.174995in}{1.498913in}}%
\pgfpathlineto{\pgfqpoint{1.177649in}{1.515073in}}%
\pgfpathlineto{\pgfqpoint{1.182366in}{1.540932in}}%
\pgfpathlineto{\pgfqpoint{1.185020in}{1.544833in}}%
\pgfpathlineto{\pgfqpoint{1.185315in}{1.544618in}}%
\pgfpathlineto{\pgfqpoint{1.187673in}{1.539777in}}%
\pgfpathlineto{\pgfqpoint{1.194012in}{1.527904in}}%
\pgfpathlineto{\pgfqpoint{1.194160in}{1.527969in}}%
\pgfpathlineto{\pgfqpoint{1.196518in}{1.531208in}}%
\pgfpathlineto{\pgfqpoint{1.201973in}{1.537800in}}%
\pgfpathlineto{\pgfqpoint{1.206690in}{1.541389in}}%
\pgfpathlineto{\pgfqpoint{1.215683in}{1.560231in}}%
\pgfpathlineto{\pgfqpoint{1.216125in}{1.559988in}}%
\pgfpathlineto{\pgfqpoint{1.224528in}{1.556207in}}%
\pgfpathlineto{\pgfqpoint{1.226887in}{1.561984in}}%
\pgfpathlineto{\pgfqpoint{1.235289in}{1.584834in}}%
\pgfpathlineto{\pgfqpoint{1.253717in}{1.602954in}}%
\pgfpathlineto{\pgfqpoint{1.256813in}{1.601143in}}%
\pgfpathlineto{\pgfqpoint{1.259466in}{1.600658in}}%
\pgfpathlineto{\pgfqpoint{1.261972in}{1.603733in}}%
\pgfpathlineto{\pgfqpoint{1.275682in}{1.624163in}}%
\pgfpathlineto{\pgfqpoint{1.281137in}{1.625207in}}%
\pgfpathlineto{\pgfqpoint{1.285117in}{1.629635in}}%
\pgfpathlineto{\pgfqpoint{1.292193in}{1.645722in}}%
\pgfpathlineto{\pgfqpoint{1.296615in}{1.654718in}}%
\pgfpathlineto{\pgfqpoint{1.300891in}{1.656968in}}%
\pgfpathlineto{\pgfqpoint{1.307672in}{1.654068in}}%
\pgfpathlineto{\pgfqpoint{1.311357in}{1.652027in}}%
\pgfpathlineto{\pgfqpoint{1.314306in}{1.654774in}}%
\pgfpathlineto{\pgfqpoint{1.320792in}{1.669131in}}%
\pgfpathlineto{\pgfqpoint{1.326836in}{1.678929in}}%
\pgfpathlineto{\pgfqpoint{1.329195in}{1.677339in}}%
\pgfpathlineto{\pgfqpoint{1.333470in}{1.674336in}}%
\pgfpathlineto{\pgfqpoint{1.335976in}{1.677476in}}%
\pgfpathlineto{\pgfqpoint{1.339219in}{1.687591in}}%
\pgfpathlineto{\pgfqpoint{1.346443in}{1.708892in}}%
\pgfpathlineto{\pgfqpoint{1.350718in}{1.712808in}}%
\pgfpathlineto{\pgfqpoint{1.356173in}{1.715112in}}%
\pgfpathlineto{\pgfqpoint{1.359416in}{1.714377in}}%
\pgfpathlineto{\pgfqpoint{1.364870in}{1.711167in}}%
\pgfpathlineto{\pgfqpoint{1.367082in}{1.712842in}}%
\pgfpathlineto{\pgfqpoint{1.370914in}{1.722249in}}%
\pgfpathlineto{\pgfqpoint{1.377548in}{1.734808in}}%
\pgfpathlineto{\pgfqpoint{1.379612in}{1.733684in}}%
\pgfpathlineto{\pgfqpoint{1.386393in}{1.726520in}}%
\pgfpathlineto{\pgfqpoint{1.387130in}{1.727484in}}%
\pgfpathlineto{\pgfqpoint{1.393322in}{1.741061in}}%
\pgfpathlineto{\pgfqpoint{1.399808in}{1.752195in}}%
\pgfpathlineto{\pgfqpoint{1.405705in}{1.758467in}}%
\pgfpathlineto{\pgfqpoint{1.413813in}{1.763862in}}%
\pgfpathlineto{\pgfqpoint{1.419268in}{1.767688in}}%
\pgfpathlineto{\pgfqpoint{1.436516in}{1.783572in}}%
\pgfpathlineto{\pgfqpoint{1.447719in}{1.787317in}}%
\pgfpathlineto{\pgfqpoint{1.453321in}{1.794216in}}%
\pgfpathlineto{\pgfqpoint{1.462314in}{1.803247in}}%
\pgfpathlineto{\pgfqpoint{1.466442in}{1.803604in}}%
\pgfpathlineto{\pgfqpoint{1.471012in}{1.803568in}}%
\pgfpathlineto{\pgfqpoint{1.476761in}{1.806762in}}%
\pgfpathlineto{\pgfqpoint{1.482363in}{1.813479in}}%
\pgfpathlineto{\pgfqpoint{1.488260in}{1.819407in}}%
\pgfpathlineto{\pgfqpoint{1.492240in}{1.819219in}}%
\pgfpathlineto{\pgfqpoint{1.502854in}{1.817480in}}%
\pgfpathlineto{\pgfqpoint{1.506982in}{1.824220in}}%
\pgfpathlineto{\pgfqpoint{1.513173in}{1.832530in}}%
\pgfpathlineto{\pgfqpoint{1.516859in}{1.832662in}}%
\pgfpathlineto{\pgfqpoint{1.525556in}{1.831640in}}%
\pgfpathlineto{\pgfqpoint{1.529979in}{1.835765in}}%
\pgfpathlineto{\pgfqpoint{1.544279in}{1.852312in}}%
\pgfpathlineto{\pgfqpoint{1.549586in}{1.851771in}}%
\pgfpathlineto{\pgfqpoint{1.553713in}{1.851726in}}%
\pgfpathlineto{\pgfqpoint{1.557841in}{1.856115in}}%
\pgfpathlineto{\pgfqpoint{1.566244in}{1.866533in}}%
\pgfpathlineto{\pgfqpoint{1.569635in}{1.865867in}}%
\pgfpathlineto{\pgfqpoint{1.576416in}{1.864310in}}%
\pgfpathlineto{\pgfqpoint{1.582165in}{1.867960in}}%
\pgfpathlineto{\pgfqpoint{1.592190in}{1.874868in}}%
\pgfpathlineto{\pgfqpoint{1.599855in}{1.878915in}}%
\pgfpathlineto{\pgfqpoint{1.605899in}{1.880584in}}%
\pgfpathlineto{\pgfqpoint{1.612238in}{1.883936in}}%
\pgfpathlineto{\pgfqpoint{1.624474in}{1.893464in}}%
\pgfpathlineto{\pgfqpoint{1.641575in}{1.899979in}}%
\pgfpathlineto{\pgfqpoint{1.655874in}{1.909856in}}%
\pgfpathlineto{\pgfqpoint{1.660444in}{1.907926in}}%
\pgfpathlineto{\pgfqpoint{1.666046in}{1.907074in}}%
\pgfpathlineto{\pgfqpoint{1.671353in}{1.911617in}}%
\pgfpathlineto{\pgfqpoint{1.676808in}{1.915711in}}%
\pgfpathlineto{\pgfqpoint{1.680641in}{1.914855in}}%
\pgfpathlineto{\pgfqpoint{1.688306in}{1.911073in}}%
\pgfpathlineto{\pgfqpoint{1.691844in}{1.913032in}}%
\pgfpathlineto{\pgfqpoint{1.698773in}{1.920278in}}%
\pgfpathlineto{\pgfqpoint{1.705849in}{1.925241in}}%
\pgfpathlineto{\pgfqpoint{1.711009in}{1.924187in}}%
\pgfpathlineto{\pgfqpoint{1.716906in}{1.922706in}}%
\pgfpathlineto{\pgfqpoint{1.720296in}{1.924290in}}%
\pgfpathlineto{\pgfqpoint{1.728109in}{1.929049in}}%
\pgfpathlineto{\pgfqpoint{1.731942in}{1.927335in}}%
\pgfpathlineto{\pgfqpoint{1.745063in}{1.920642in}}%
\pgfpathlineto{\pgfqpoint{1.763342in}{1.925814in}}%
\pgfpathlineto{\pgfqpoint{1.769829in}{1.926657in}}%
\pgfpathlineto{\pgfqpoint{1.774988in}{1.930148in}}%
\pgfpathlineto{\pgfqpoint{1.782065in}{1.934309in}}%
\pgfpathlineto{\pgfqpoint{1.788698in}{1.933650in}}%
\pgfpathlineto{\pgfqpoint{1.796954in}{1.933225in}}%
\pgfpathlineto{\pgfqpoint{1.807421in}{1.935033in}}%
\pgfpathlineto{\pgfqpoint{1.819214in}{1.938900in}}%
\pgfpathlineto{\pgfqpoint{1.828354in}{1.939263in}}%
\pgfpathlineto{\pgfqpoint{1.844570in}{1.944090in}}%
\pgfpathlineto{\pgfqpoint{1.851646in}{1.940866in}}%
\pgfpathlineto{\pgfqpoint{1.857395in}{1.940496in}}%
\pgfpathlineto{\pgfqpoint{1.869631in}{1.945939in}}%
\pgfpathlineto{\pgfqpoint{1.874201in}{1.942132in}}%
\pgfpathlineto{\pgfqpoint{1.879950in}{1.939084in}}%
\pgfpathlineto{\pgfqpoint{1.884963in}{1.940577in}}%
\pgfpathlineto{\pgfqpoint{1.892481in}{1.941567in}}%
\pgfpathlineto{\pgfqpoint{1.897051in}{1.938851in}}%
\pgfpathlineto{\pgfqpoint{1.906043in}{1.932448in}}%
\pgfpathlineto{\pgfqpoint{1.910908in}{1.933230in}}%
\pgfpathlineto{\pgfqpoint{1.921522in}{1.939410in}}%
\pgfpathlineto{\pgfqpoint{1.927714in}{1.935912in}}%
\pgfpathlineto{\pgfqpoint{1.931105in}{1.936205in}}%
\pgfpathlineto{\pgfqpoint{1.936707in}{1.940037in}}%
\pgfpathlineto{\pgfqpoint{1.942014in}{1.941796in}}%
\pgfpathlineto{\pgfqpoint{1.949679in}{1.942037in}}%
\pgfpathlineto{\pgfqpoint{1.963389in}{1.940452in}}%
\pgfpathlineto{\pgfqpoint{1.976952in}{1.945357in}}%
\pgfpathlineto{\pgfqpoint{2.009384in}{1.942984in}}%
\pgfpathlineto{\pgfqpoint{2.018082in}{1.938822in}}%
\pgfpathlineto{\pgfqpoint{2.023094in}{1.940035in}}%
\pgfpathlineto{\pgfqpoint{2.032086in}{1.942033in}}%
\pgfpathlineto{\pgfqpoint{2.038425in}{1.938383in}}%
\pgfpathlineto{\pgfqpoint{2.044617in}{1.937004in}}%
\pgfpathlineto{\pgfqpoint{2.052872in}{1.939299in}}%
\pgfpathlineto{\pgfqpoint{2.058769in}{1.939659in}}%
\pgfpathlineto{\pgfqpoint{2.063929in}{1.934312in}}%
\pgfpathlineto{\pgfqpoint{2.069678in}{1.928241in}}%
\pgfpathlineto{\pgfqpoint{2.072774in}{1.928252in}}%
\pgfpathlineto{\pgfqpoint{2.082503in}{1.932432in}}%
\pgfpathlineto{\pgfqpoint{2.095624in}{1.928933in}}%
\pgfpathlineto{\pgfqpoint{2.103584in}{1.934457in}}%
\pgfpathlineto{\pgfqpoint{2.109628in}{1.937950in}}%
\pgfpathlineto{\pgfqpoint{2.112872in}{1.937513in}}%
\pgfpathlineto{\pgfqpoint{2.125550in}{1.933083in}}%
\pgfpathlineto{\pgfqpoint{2.135721in}{1.935002in}}%
\pgfpathlineto{\pgfqpoint{2.146336in}{1.930523in}}%
\pgfpathlineto{\pgfqpoint{2.156950in}{1.929644in}}%
\pgfpathlineto{\pgfqpoint{2.165058in}{1.931037in}}%
\pgfpathlineto{\pgfqpoint{2.173018in}{1.928600in}}%
\pgfpathlineto{\pgfqpoint{2.178178in}{1.927459in}}%
\pgfpathlineto{\pgfqpoint{2.182011in}{1.928188in}}%
\pgfpathlineto{\pgfqpoint{2.194689in}{1.930912in}}%
\pgfpathlineto{\pgfqpoint{2.201175in}{1.928830in}}%
\pgfpathlineto{\pgfqpoint{2.208399in}{1.926256in}}%
\pgfpathlineto{\pgfqpoint{2.212084in}{1.927645in}}%
\pgfpathlineto{\pgfqpoint{2.220192in}{1.930034in}}%
\pgfpathlineto{\pgfqpoint{2.225352in}{1.927055in}}%
\pgfpathlineto{\pgfqpoint{2.232133in}{1.921668in}}%
\pgfpathlineto{\pgfqpoint{2.235671in}{1.922066in}}%
\pgfpathlineto{\pgfqpoint{2.245253in}{1.923615in}}%
\pgfpathlineto{\pgfqpoint{2.251003in}{1.920717in}}%
\pgfpathlineto{\pgfqpoint{2.259995in}{1.916418in}}%
\pgfpathlineto{\pgfqpoint{2.264565in}{1.917451in}}%
\pgfpathlineto{\pgfqpoint{2.275916in}{1.923569in}}%
\pgfpathlineto{\pgfqpoint{2.279749in}{1.922079in}}%
\pgfpathlineto{\pgfqpoint{2.285499in}{1.920036in}}%
\pgfpathlineto{\pgfqpoint{2.289037in}{1.921803in}}%
\pgfpathlineto{\pgfqpoint{2.297734in}{1.927108in}}%
\pgfpathlineto{\pgfqpoint{2.303041in}{1.924825in}}%
\pgfpathlineto{\pgfqpoint{2.310560in}{1.921445in}}%
\pgfpathlineto{\pgfqpoint{2.315719in}{1.922314in}}%
\pgfpathlineto{\pgfqpoint{2.334589in}{1.927693in}}%
\pgfpathlineto{\pgfqpoint{2.342844in}{1.927609in}}%
\pgfpathlineto{\pgfqpoint{2.354048in}{1.928909in}}%
\pgfpathlineto{\pgfqpoint{2.367758in}{1.924172in}}%
\pgfpathlineto{\pgfqpoint{2.372770in}{1.922813in}}%
\pgfpathlineto{\pgfqpoint{2.378077in}{1.925010in}}%
\pgfpathlineto{\pgfqpoint{2.385891in}{1.928082in}}%
\pgfpathlineto{\pgfqpoint{2.400338in}{1.927212in}}%
\pgfpathlineto{\pgfqpoint{2.411836in}{1.933605in}}%
\pgfpathlineto{\pgfqpoint{2.417733in}{1.931372in}}%
\pgfpathlineto{\pgfqpoint{2.425251in}{1.928010in}}%
\pgfpathlineto{\pgfqpoint{2.431738in}{1.929278in}}%
\pgfpathlineto{\pgfqpoint{2.436897in}{1.929506in}}%
\pgfpathlineto{\pgfqpoint{2.441467in}{1.926096in}}%
\pgfpathlineto{\pgfqpoint{2.448691in}{1.920687in}}%
\pgfpathlineto{\pgfqpoint{2.452082in}{1.921704in}}%
\pgfpathlineto{\pgfqpoint{2.465497in}{1.929009in}}%
\pgfpathlineto{\pgfqpoint{2.477585in}{1.928199in}}%
\pgfpathlineto{\pgfqpoint{2.491295in}{1.931231in}}%
\pgfpathlineto{\pgfqpoint{2.510312in}{1.926153in}}%
\pgfpathlineto{\pgfqpoint{2.520041in}{1.927975in}}%
\pgfpathlineto{\pgfqpoint{2.531688in}{1.927602in}}%
\pgfpathlineto{\pgfqpoint{2.541270in}{1.931833in}}%
\pgfpathlineto{\pgfqpoint{2.547609in}{1.933537in}}%
\pgfpathlineto{\pgfqpoint{2.552916in}{1.932531in}}%
\pgfpathlineto{\pgfqpoint{2.564267in}{1.929598in}}%
\pgfpathlineto{\pgfqpoint{2.577977in}{1.931865in}}%
\pgfpathlineto{\pgfqpoint{2.589181in}{1.927872in}}%
\pgfpathlineto{\pgfqpoint{2.602743in}{1.929204in}}%
\pgfpathlineto{\pgfqpoint{2.608787in}{1.924347in}}%
\pgfpathlineto{\pgfqpoint{2.613947in}{1.922315in}}%
\pgfpathlineto{\pgfqpoint{2.621318in}{1.925310in}}%
\pgfpathlineto{\pgfqpoint{2.626478in}{1.925976in}}%
\pgfpathlineto{\pgfqpoint{2.631342in}{1.923552in}}%
\pgfpathlineto{\pgfqpoint{2.642104in}{1.918960in}}%
\pgfpathlineto{\pgfqpoint{2.647706in}{1.922886in}}%
\pgfpathlineto{\pgfqpoint{2.653455in}{1.925501in}}%
\pgfpathlineto{\pgfqpoint{2.659499in}{1.924034in}}%
\pgfpathlineto{\pgfqpoint{2.664512in}{1.923530in}}%
\pgfpathlineto{\pgfqpoint{2.668934in}{1.926029in}}%
\pgfpathlineto{\pgfqpoint{2.676010in}{1.929222in}}%
\pgfpathlineto{\pgfqpoint{2.683381in}{1.928306in}}%
\pgfpathlineto{\pgfqpoint{2.697091in}{1.924886in}}%
\pgfpathlineto{\pgfqpoint{2.705494in}{1.927952in}}%
\pgfpathlineto{\pgfqpoint{2.711980in}{1.929650in}}%
\pgfpathlineto{\pgfqpoint{2.718172in}{1.927753in}}%
\pgfpathlineto{\pgfqpoint{2.724069in}{1.926979in}}%
\pgfpathlineto{\pgfqpoint{2.738516in}{1.926832in}}%
\pgfpathlineto{\pgfqpoint{2.749572in}{1.923036in}}%
\pgfpathlineto{\pgfqpoint{2.756058in}{1.923915in}}%
\pgfpathlineto{\pgfqpoint{2.770505in}{1.929801in}}%
\pgfpathlineto{\pgfqpoint{2.778171in}{1.928353in}}%
\pgfpathlineto{\pgfqpoint{2.781709in}{1.931332in}}%
\pgfpathlineto{\pgfqpoint{2.788490in}{1.937123in}}%
\pgfpathlineto{\pgfqpoint{2.791734in}{1.936192in}}%
\pgfpathlineto{\pgfqpoint{2.803675in}{1.932473in}}%
\pgfpathlineto{\pgfqpoint{2.814584in}{1.934797in}}%
\pgfpathlineto{\pgfqpoint{2.819891in}{1.935653in}}%
\pgfpathlineto{\pgfqpoint{2.829620in}{1.931591in}}%
\pgfpathlineto{\pgfqpoint{2.835517in}{1.936493in}}%
\pgfpathlineto{\pgfqpoint{2.840234in}{1.939222in}}%
\pgfpathlineto{\pgfqpoint{2.844362in}{1.939082in}}%
\pgfpathlineto{\pgfqpoint{2.870603in}{1.933051in}}%
\pgfpathlineto{\pgfqpoint{2.874435in}{1.931667in}}%
\pgfpathlineto{\pgfqpoint{2.881659in}{1.928939in}}%
\pgfpathlineto{\pgfqpoint{2.887408in}{1.930891in}}%
\pgfpathlineto{\pgfqpoint{2.902003in}{1.940324in}}%
\pgfpathlineto{\pgfqpoint{2.909226in}{1.939238in}}%
\pgfpathlineto{\pgfqpoint{2.915860in}{1.938552in}}%
\pgfpathlineto{\pgfqpoint{2.920725in}{1.939974in}}%
\pgfpathlineto{\pgfqpoint{2.927654in}{1.940589in}}%
\pgfpathlineto{\pgfqpoint{2.933698in}{1.935949in}}%
\pgfpathlineto{\pgfqpoint{2.940037in}{1.931345in}}%
\pgfpathlineto{\pgfqpoint{2.944459in}{1.931975in}}%
\pgfpathlineto{\pgfqpoint{2.957579in}{1.934038in}}%
\pgfpathlineto{\pgfqpoint{2.963771in}{1.932112in}}%
\pgfpathlineto{\pgfqpoint{2.967899in}{1.933981in}}%
\pgfpathlineto{\pgfqpoint{2.975859in}{1.940816in}}%
\pgfpathlineto{\pgfqpoint{2.982788in}{1.944179in}}%
\pgfpathlineto{\pgfqpoint{2.987063in}{1.942774in}}%
\pgfpathlineto{\pgfqpoint{2.994287in}{1.938722in}}%
\pgfpathlineto{\pgfqpoint{3.001215in}{1.940137in}}%
\pgfpathlineto{\pgfqpoint{3.005343in}{1.939824in}}%
\pgfpathlineto{\pgfqpoint{3.012567in}{1.934143in}}%
\pgfpathlineto{\pgfqpoint{3.018316in}{1.931365in}}%
\pgfpathlineto{\pgfqpoint{3.024213in}{1.932539in}}%
\pgfpathlineto{\pgfqpoint{3.038365in}{1.936246in}}%
\pgfpathlineto{\pgfqpoint{3.038365in}{1.936246in}}%
\pgfusepath{stroke}%
\end{pgfscope}%
\begin{pgfscope}%
\pgfpathrectangle{\pgfqpoint{0.679669in}{0.526079in}}{\pgfqpoint{2.358696in}{1.661000in}} %
\pgfusepath{clip}%
\pgfsetbuttcap%
\pgfsetroundjoin%
\pgfsetlinewidth{1.003750pt}%
\definecolor{currentstroke}{rgb}{0.000000,0.000000,0.000000}%
\pgfsetstrokecolor{currentstroke}%
\pgfsetdash{{3.700000pt}{1.600000pt}}{0.000000pt}%
\pgfpathmoveto{\pgfqpoint{0.679669in}{0.858279in}}%
\pgfpathlineto{\pgfqpoint{0.693964in}{0.874619in}}%
\pgfpathlineto{\pgfqpoint{0.708259in}{0.890958in}}%
\pgfpathlineto{\pgfqpoint{0.722555in}{0.907297in}}%
\pgfpathlineto{\pgfqpoint{0.736850in}{0.923637in}}%
\pgfpathlineto{\pgfqpoint{0.751145in}{0.939976in}}%
\pgfpathlineto{\pgfqpoint{0.765440in}{0.956316in}}%
\pgfpathlineto{\pgfqpoint{0.779735in}{0.972655in}}%
\pgfpathlineto{\pgfqpoint{0.794030in}{0.988994in}}%
\pgfpathlineto{\pgfqpoint{0.808325in}{1.005334in}}%
\pgfpathlineto{\pgfqpoint{0.822620in}{1.021673in}}%
\pgfpathlineto{\pgfqpoint{0.836916in}{1.038012in}}%
\pgfpathlineto{\pgfqpoint{0.851211in}{1.054352in}}%
\pgfpathlineto{\pgfqpoint{0.865506in}{1.070691in}}%
\pgfpathlineto{\pgfqpoint{0.879801in}{1.087030in}}%
\pgfpathlineto{\pgfqpoint{0.894096in}{1.103370in}}%
\pgfpathlineto{\pgfqpoint{0.908391in}{1.119709in}}%
\pgfpathlineto{\pgfqpoint{0.922686in}{1.136048in}}%
\pgfpathlineto{\pgfqpoint{0.936981in}{1.152388in}}%
\pgfpathlineto{\pgfqpoint{0.951277in}{1.168727in}}%
\pgfpathlineto{\pgfqpoint{0.965572in}{1.185066in}}%
\pgfpathlineto{\pgfqpoint{0.979867in}{1.201406in}}%
\pgfpathlineto{\pgfqpoint{0.994162in}{1.217745in}}%
\pgfpathlineto{\pgfqpoint{1.008457in}{1.234084in}}%
\pgfpathlineto{\pgfqpoint{1.022752in}{1.250424in}}%
\pgfpathlineto{\pgfqpoint{1.037047in}{1.266763in}}%
\pgfpathlineto{\pgfqpoint{1.051342in}{1.283103in}}%
\pgfpathlineto{\pgfqpoint{1.065638in}{1.299442in}}%
\pgfpathlineto{\pgfqpoint{1.079933in}{1.315781in}}%
\pgfpathlineto{\pgfqpoint{1.094228in}{1.332121in}}%
\pgfpathlineto{\pgfqpoint{1.108523in}{1.348460in}}%
\pgfpathlineto{\pgfqpoint{1.122818in}{1.364799in}}%
\pgfpathlineto{\pgfqpoint{1.137113in}{1.381139in}}%
\pgfpathlineto{\pgfqpoint{1.151408in}{1.397478in}}%
\pgfpathlineto{\pgfqpoint{1.165703in}{1.413817in}}%
\pgfpathlineto{\pgfqpoint{1.179999in}{1.430157in}}%
\pgfpathlineto{\pgfqpoint{1.194294in}{1.446496in}}%
\pgfpathlineto{\pgfqpoint{1.208589in}{1.462835in}}%
\pgfpathlineto{\pgfqpoint{1.222884in}{1.479175in}}%
\pgfpathlineto{\pgfqpoint{1.237179in}{1.495514in}}%
\pgfpathlineto{\pgfqpoint{1.251474in}{1.511853in}}%
\pgfpathlineto{\pgfqpoint{1.265769in}{1.528193in}}%
\pgfpathlineto{\pgfqpoint{1.280064in}{1.544532in}}%
\pgfpathlineto{\pgfqpoint{1.294360in}{1.560871in}}%
\pgfpathlineto{\pgfqpoint{1.308655in}{1.577211in}}%
\pgfpathlineto{\pgfqpoint{1.322950in}{1.593550in}}%
\pgfpathlineto{\pgfqpoint{1.337245in}{1.609890in}}%
\pgfpathlineto{\pgfqpoint{1.351540in}{1.626229in}}%
\pgfpathlineto{\pgfqpoint{1.365835in}{1.642568in}}%
\pgfpathlineto{\pgfqpoint{1.380130in}{1.658908in}}%
\pgfpathlineto{\pgfqpoint{1.394425in}{1.675247in}}%
\pgfpathlineto{\pgfqpoint{1.408721in}{1.691586in}}%
\pgfpathlineto{\pgfqpoint{1.423016in}{1.707926in}}%
\pgfpathlineto{\pgfqpoint{1.437311in}{1.724265in}}%
\pgfpathlineto{\pgfqpoint{1.451606in}{1.740604in}}%
\pgfpathlineto{\pgfqpoint{1.465901in}{1.756944in}}%
\pgfpathlineto{\pgfqpoint{1.480196in}{1.773283in}}%
\pgfpathlineto{\pgfqpoint{1.494491in}{1.789622in}}%
\pgfpathlineto{\pgfqpoint{1.508786in}{1.805962in}}%
\pgfpathlineto{\pgfqpoint{1.523082in}{1.822301in}}%
\pgfpathlineto{\pgfqpoint{1.537377in}{1.838640in}}%
\pgfpathlineto{\pgfqpoint{1.551672in}{1.854980in}}%
\pgfpathlineto{\pgfqpoint{1.565967in}{1.871319in}}%
\pgfpathlineto{\pgfqpoint{1.580262in}{1.887658in}}%
\pgfpathlineto{\pgfqpoint{1.594557in}{1.903998in}}%
\pgfpathlineto{\pgfqpoint{1.608852in}{1.920337in}}%
\pgfpathlineto{\pgfqpoint{1.623147in}{1.936677in}}%
\pgfpathlineto{\pgfqpoint{1.637443in}{1.953016in}}%
\pgfpathlineto{\pgfqpoint{1.651738in}{1.969355in}}%
\pgfpathlineto{\pgfqpoint{1.666033in}{1.985695in}}%
\pgfpathlineto{\pgfqpoint{1.680328in}{2.002034in}}%
\pgfpathlineto{\pgfqpoint{1.694623in}{2.018373in}}%
\pgfpathlineto{\pgfqpoint{1.708918in}{2.034713in}}%
\pgfpathlineto{\pgfqpoint{1.723213in}{2.051052in}}%
\pgfpathlineto{\pgfqpoint{1.737508in}{2.067391in}}%
\pgfpathlineto{\pgfqpoint{1.751804in}{2.083731in}}%
\pgfpathlineto{\pgfqpoint{1.766099in}{2.100070in}}%
\pgfpathlineto{\pgfqpoint{1.780394in}{2.116409in}}%
\pgfpathlineto{\pgfqpoint{1.794689in}{2.132749in}}%
\pgfpathlineto{\pgfqpoint{1.808984in}{2.149088in}}%
\pgfpathlineto{\pgfqpoint{1.823279in}{2.165427in}}%
\pgfpathlineto{\pgfqpoint{1.837574in}{2.181767in}}%
\pgfpathlineto{\pgfqpoint{1.851869in}{2.198106in}}%
\pgfpathlineto{\pgfqpoint{1.854374in}{2.200968in}}%
\pgfusepath{stroke}%
\end{pgfscope}%
\begin{pgfscope}%
\pgfsetrectcap%
\pgfsetmiterjoin%
\pgfsetlinewidth{0.803000pt}%
\definecolor{currentstroke}{rgb}{0.000000,0.000000,0.000000}%
\pgfsetstrokecolor{currentstroke}%
\pgfsetdash{}{0pt}%
\pgfpathmoveto{\pgfqpoint{0.679669in}{0.526079in}}%
\pgfpathlineto{\pgfqpoint{0.679669in}{2.187079in}}%
\pgfusepath{stroke}%
\end{pgfscope}%
\begin{pgfscope}%
\pgfsetrectcap%
\pgfsetmiterjoin%
\pgfsetlinewidth{0.803000pt}%
\definecolor{currentstroke}{rgb}{0.000000,0.000000,0.000000}%
\pgfsetstrokecolor{currentstroke}%
\pgfsetdash{}{0pt}%
\pgfpathmoveto{\pgfqpoint{3.038365in}{0.526079in}}%
\pgfpathlineto{\pgfqpoint{3.038365in}{2.187079in}}%
\pgfusepath{stroke}%
\end{pgfscope}%
\begin{pgfscope}%
\pgfsetrectcap%
\pgfsetmiterjoin%
\pgfsetlinewidth{0.803000pt}%
\definecolor{currentstroke}{rgb}{0.000000,0.000000,0.000000}%
\pgfsetstrokecolor{currentstroke}%
\pgfsetdash{}{0pt}%
\pgfpathmoveto{\pgfqpoint{0.679669in}{0.526079in}}%
\pgfpathlineto{\pgfqpoint{3.038365in}{0.526079in}}%
\pgfusepath{stroke}%
\end{pgfscope}%
\begin{pgfscope}%
\pgfsetrectcap%
\pgfsetmiterjoin%
\pgfsetlinewidth{0.803000pt}%
\definecolor{currentstroke}{rgb}{0.000000,0.000000,0.000000}%
\pgfsetstrokecolor{currentstroke}%
\pgfsetdash{}{0pt}%
\pgfpathmoveto{\pgfqpoint{0.679669in}{2.187079in}}%
\pgfpathlineto{\pgfqpoint{3.038365in}{2.187079in}}%
\pgfusepath{stroke}%
\end{pgfscope}%
\begin{pgfscope}%
\pgftext[x=0.738637in,y=2.013379in,left,base]{\rmfamily\fontsize{10.000000}{12.000000}\selectfont (a)}%
\end{pgfscope}%
\begin{pgfscope}%
\pgftext[x=1.859017in,y=2.270413in,,base]{\rmfamily\fontsize{12.000000}{14.400000}\selectfont Standard}%
\end{pgfscope}%
\begin{pgfscope}%
\pgfsetbuttcap%
\pgfsetmiterjoin%
\definecolor{currentfill}{rgb}{1.000000,1.000000,1.000000}%
\pgfsetfillcolor{currentfill}%
\pgfsetfillopacity{0.800000}%
\pgfsetlinewidth{1.003750pt}%
\definecolor{currentstroke}{rgb}{0.800000,0.800000,0.800000}%
\pgfsetstrokecolor{currentstroke}%
\pgfsetstrokeopacity{0.800000}%
\pgfsetdash{}{0pt}%
\pgfpathmoveto{\pgfqpoint{1.158066in}{0.595524in}}%
\pgfpathlineto{\pgfqpoint{2.941143in}{0.595524in}}%
\pgfpathquadraticcurveto{\pgfqpoint{2.968920in}{0.595524in}}{\pgfqpoint{2.968920in}{0.623302in}}%
\pgfpathlineto{\pgfqpoint{2.968920in}{1.019094in}}%
\pgfpathquadraticcurveto{\pgfqpoint{2.968920in}{1.046872in}}{\pgfqpoint{2.941143in}{1.046872in}}%
\pgfpathlineto{\pgfqpoint{1.158066in}{1.046872in}}%
\pgfpathquadraticcurveto{\pgfqpoint{1.130288in}{1.046872in}}{\pgfqpoint{1.130288in}{1.019094in}}%
\pgfpathlineto{\pgfqpoint{1.130288in}{0.623302in}}%
\pgfpathquadraticcurveto{\pgfqpoint{1.130288in}{0.595524in}}{\pgfqpoint{1.158066in}{0.595524in}}%
\pgfpathclose%
\pgfusepath{stroke,fill}%
\end{pgfscope}%
\begin{pgfscope}%
\pgfsetrectcap%
\pgfsetroundjoin%
\pgfsetlinewidth{1.003750pt}%
\definecolor{currentstroke}{rgb}{1.000000,0.549020,0.000000}%
\pgfsetstrokecolor{currentstroke}%
\pgfsetdash{}{0pt}%
\pgfpathmoveto{\pgfqpoint{1.185844in}{0.934404in}}%
\pgfpathlineto{\pgfqpoint{1.463622in}{0.934404in}}%
\pgfusepath{stroke}%
\end{pgfscope}%
\begin{pgfscope}%
\pgftext[x=1.574733in,y=0.885793in,left,base]{\rmfamily\fontsize{10.000000}{12.000000}\selectfont \(\displaystyle \mathcal{E}_{B}\)}%
\end{pgfscope}%
\begin{pgfscope}%
\pgfsetrectcap%
\pgfsetroundjoin%
\pgfsetlinewidth{1.003750pt}%
\definecolor{currentstroke}{rgb}{0.501961,0.000000,0.501961}%
\pgfsetstrokecolor{currentstroke}%
\pgfsetdash{}{0pt}%
\pgfpathmoveto{\pgfqpoint{1.185844in}{0.730547in}}%
\pgfpathlineto{\pgfqpoint{1.463622in}{0.730547in}}%
\pgfusepath{stroke}%
\end{pgfscope}%
\begin{pgfscope}%
\pgftext[x=1.574733in,y=0.681936in,left,base]{\rmfamily\fontsize{10.000000}{12.000000}\selectfont \(\displaystyle \mathcal{E}_{E}\)}%
\end{pgfscope}%
\begin{pgfscope}%
\pgfsetbuttcap%
\pgfsetroundjoin%
\pgfsetlinewidth{1.003750pt}%
\definecolor{currentstroke}{rgb}{0.627451,0.321569,0.176471}%
\pgfsetstrokecolor{currentstroke}%
\pgfsetdash{{3.700000pt}{1.600000pt}}{0.000000pt}%
\pgfpathmoveto{\pgfqpoint{2.021411in}{0.934404in}}%
\pgfpathlineto{\pgfqpoint{2.299188in}{0.934404in}}%
\pgfusepath{stroke}%
\end{pgfscope}%
\begin{pgfscope}%
\pgftext[x=2.410300in,y=0.885793in,left,base]{\rmfamily\fontsize{10.000000}{12.000000}\selectfont \(\displaystyle \mathcal{E}_\mathrm{c}\)}%
\end{pgfscope}%
\begin{pgfscope}%
\pgfsetbuttcap%
\pgfsetroundjoin%
\pgfsetlinewidth{1.003750pt}%
\definecolor{currentstroke}{rgb}{0.000000,0.000000,0.000000}%
\pgfsetstrokecolor{currentstroke}%
\pgfsetdash{{3.700000pt}{1.600000pt}}{0.000000pt}%
\pgfpathmoveto{\pgfqpoint{2.021411in}{0.730547in}}%
\pgfpathlineto{\pgfqpoint{2.299188in}{0.730547in}}%
\pgfusepath{stroke}%
\end{pgfscope}%
\begin{pgfscope}%
\pgftext[x=2.410300in,y=0.681936in,left,base]{\rmfamily\fontsize{10.000000}{12.000000}\selectfont growth}%
\end{pgfscope}%
\begin{pgfscope}%
\pgfsetbuttcap%
\pgfsetmiterjoin%
\definecolor{currentfill}{rgb}{1.000000,1.000000,1.000000}%
\pgfsetfillcolor{currentfill}%
\pgfsetlinewidth{0.000000pt}%
\definecolor{currentstroke}{rgb}{0.000000,0.000000,0.000000}%
\pgfsetstrokecolor{currentstroke}%
\pgfsetstrokeopacity{0.000000}%
\pgfsetdash{}{0pt}%
\pgfpathmoveto{\pgfqpoint{3.745974in}{0.526079in}}%
\pgfpathlineto{\pgfqpoint{6.104669in}{0.526079in}}%
\pgfpathlineto{\pgfqpoint{6.104669in}{2.187079in}}%
\pgfpathlineto{\pgfqpoint{3.745974in}{2.187079in}}%
\pgfpathclose%
\pgfusepath{fill}%
\end{pgfscope}%
\begin{pgfscope}%
\pgfsetbuttcap%
\pgfsetroundjoin%
\definecolor{currentfill}{rgb}{0.000000,0.000000,0.000000}%
\pgfsetfillcolor{currentfill}%
\pgfsetlinewidth{0.803000pt}%
\definecolor{currentstroke}{rgb}{0.000000,0.000000,0.000000}%
\pgfsetstrokecolor{currentstroke}%
\pgfsetdash{}{0pt}%
\pgfsys@defobject{currentmarker}{\pgfqpoint{0.000000in}{-0.048611in}}{\pgfqpoint{0.000000in}{0.000000in}}{%
\pgfpathmoveto{\pgfqpoint{0.000000in}{0.000000in}}%
\pgfpathlineto{\pgfqpoint{0.000000in}{-0.048611in}}%
\pgfusepath{stroke,fill}%
}%
\begin{pgfscope}%
\pgfsys@transformshift{3.745974in}{0.526079in}%
\pgfsys@useobject{currentmarker}{}%
\end{pgfscope}%
\end{pgfscope}%
\begin{pgfscope}%
\pgftext[x=3.745974in,y=0.428857in,,top]{\rmfamily\fontsize{10.000000}{12.000000}\selectfont \(\displaystyle 0\)}%
\end{pgfscope}%
\begin{pgfscope}%
\pgfsetbuttcap%
\pgfsetroundjoin%
\definecolor{currentfill}{rgb}{0.000000,0.000000,0.000000}%
\pgfsetfillcolor{currentfill}%
\pgfsetlinewidth{0.803000pt}%
\definecolor{currentstroke}{rgb}{0.000000,0.000000,0.000000}%
\pgfsetstrokecolor{currentstroke}%
\pgfsetdash{}{0pt}%
\pgfsys@defobject{currentmarker}{\pgfqpoint{0.000000in}{-0.048611in}}{\pgfqpoint{0.000000in}{0.000000in}}{%
\pgfpathmoveto{\pgfqpoint{0.000000in}{0.000000in}}%
\pgfpathlineto{\pgfqpoint{0.000000in}{-0.048611in}}%
\pgfusepath{stroke,fill}%
}%
\begin{pgfscope}%
\pgfsys@transformshift{4.335647in}{0.526079in}%
\pgfsys@useobject{currentmarker}{}%
\end{pgfscope}%
\end{pgfscope}%
\begin{pgfscope}%
\pgftext[x=4.335647in,y=0.428857in,,top]{\rmfamily\fontsize{10.000000}{12.000000}\selectfont \(\displaystyle 50\)}%
\end{pgfscope}%
\begin{pgfscope}%
\pgfsetbuttcap%
\pgfsetroundjoin%
\definecolor{currentfill}{rgb}{0.000000,0.000000,0.000000}%
\pgfsetfillcolor{currentfill}%
\pgfsetlinewidth{0.803000pt}%
\definecolor{currentstroke}{rgb}{0.000000,0.000000,0.000000}%
\pgfsetstrokecolor{currentstroke}%
\pgfsetdash{}{0pt}%
\pgfsys@defobject{currentmarker}{\pgfqpoint{0.000000in}{-0.048611in}}{\pgfqpoint{0.000000in}{0.000000in}}{%
\pgfpathmoveto{\pgfqpoint{0.000000in}{0.000000in}}%
\pgfpathlineto{\pgfqpoint{0.000000in}{-0.048611in}}%
\pgfusepath{stroke,fill}%
}%
\begin{pgfscope}%
\pgfsys@transformshift{4.925321in}{0.526079in}%
\pgfsys@useobject{currentmarker}{}%
\end{pgfscope}%
\end{pgfscope}%
\begin{pgfscope}%
\pgftext[x=4.925321in,y=0.428857in,,top]{\rmfamily\fontsize{10.000000}{12.000000}\selectfont \(\displaystyle 100\)}%
\end{pgfscope}%
\begin{pgfscope}%
\pgfsetbuttcap%
\pgfsetroundjoin%
\definecolor{currentfill}{rgb}{0.000000,0.000000,0.000000}%
\pgfsetfillcolor{currentfill}%
\pgfsetlinewidth{0.803000pt}%
\definecolor{currentstroke}{rgb}{0.000000,0.000000,0.000000}%
\pgfsetstrokecolor{currentstroke}%
\pgfsetdash{}{0pt}%
\pgfsys@defobject{currentmarker}{\pgfqpoint{0.000000in}{-0.048611in}}{\pgfqpoint{0.000000in}{0.000000in}}{%
\pgfpathmoveto{\pgfqpoint{0.000000in}{0.000000in}}%
\pgfpathlineto{\pgfqpoint{0.000000in}{-0.048611in}}%
\pgfusepath{stroke,fill}%
}%
\begin{pgfscope}%
\pgfsys@transformshift{5.514995in}{0.526079in}%
\pgfsys@useobject{currentmarker}{}%
\end{pgfscope}%
\end{pgfscope}%
\begin{pgfscope}%
\pgftext[x=5.514995in,y=0.428857in,,top]{\rmfamily\fontsize{10.000000}{12.000000}\selectfont \(\displaystyle 150\)}%
\end{pgfscope}%
\begin{pgfscope}%
\pgfsetbuttcap%
\pgfsetroundjoin%
\definecolor{currentfill}{rgb}{0.000000,0.000000,0.000000}%
\pgfsetfillcolor{currentfill}%
\pgfsetlinewidth{0.803000pt}%
\definecolor{currentstroke}{rgb}{0.000000,0.000000,0.000000}%
\pgfsetstrokecolor{currentstroke}%
\pgfsetdash{}{0pt}%
\pgfsys@defobject{currentmarker}{\pgfqpoint{0.000000in}{-0.048611in}}{\pgfqpoint{0.000000in}{0.000000in}}{%
\pgfpathmoveto{\pgfqpoint{0.000000in}{0.000000in}}%
\pgfpathlineto{\pgfqpoint{0.000000in}{-0.048611in}}%
\pgfusepath{stroke,fill}%
}%
\begin{pgfscope}%
\pgfsys@transformshift{6.104669in}{0.526079in}%
\pgfsys@useobject{currentmarker}{}%
\end{pgfscope}%
\end{pgfscope}%
\begin{pgfscope}%
\pgftext[x=6.104669in,y=0.428857in,,top]{\rmfamily\fontsize{10.000000}{12.000000}\selectfont \(\displaystyle 200\)}%
\end{pgfscope}%
\begin{pgfscope}%
\pgftext[x=4.925321in,y=0.238889in,,top]{\rmfamily\fontsize{10.000000}{12.000000}\selectfont \(\displaystyle t|\Omega_\mathrm{ce}|\)}%
\end{pgfscope}%
\begin{pgfscope}%
\pgfsetbuttcap%
\pgfsetroundjoin%
\definecolor{currentfill}{rgb}{0.000000,0.000000,0.000000}%
\pgfsetfillcolor{currentfill}%
\pgfsetlinewidth{0.803000pt}%
\definecolor{currentstroke}{rgb}{0.000000,0.000000,0.000000}%
\pgfsetstrokecolor{currentstroke}%
\pgfsetdash{}{0pt}%
\pgfsys@defobject{currentmarker}{\pgfqpoint{-0.048611in}{0.000000in}}{\pgfqpoint{0.000000in}{0.000000in}}{%
\pgfpathmoveto{\pgfqpoint{0.000000in}{0.000000in}}%
\pgfpathlineto{\pgfqpoint{-0.048611in}{0.000000in}}%
\pgfusepath{stroke,fill}%
}%
\begin{pgfscope}%
\pgfsys@transformshift{3.745974in}{0.526079in}%
\pgfsys@useobject{currentmarker}{}%
\end{pgfscope}%
\end{pgfscope}%
\begin{pgfscope}%
\pgfsetbuttcap%
\pgfsetroundjoin%
\definecolor{currentfill}{rgb}{0.000000,0.000000,0.000000}%
\pgfsetfillcolor{currentfill}%
\pgfsetlinewidth{0.803000pt}%
\definecolor{currentstroke}{rgb}{0.000000,0.000000,0.000000}%
\pgfsetstrokecolor{currentstroke}%
\pgfsetdash{}{0pt}%
\pgfsys@defobject{currentmarker}{\pgfqpoint{-0.048611in}{0.000000in}}{\pgfqpoint{0.000000in}{0.000000in}}{%
\pgfpathmoveto{\pgfqpoint{0.000000in}{0.000000in}}%
\pgfpathlineto{\pgfqpoint{-0.048611in}{0.000000in}}%
\pgfusepath{stroke,fill}%
}%
\begin{pgfscope}%
\pgfsys@transformshift{3.745974in}{1.190479in}%
\pgfsys@useobject{currentmarker}{}%
\end{pgfscope}%
\end{pgfscope}%
\begin{pgfscope}%
\pgfsetbuttcap%
\pgfsetroundjoin%
\definecolor{currentfill}{rgb}{0.000000,0.000000,0.000000}%
\pgfsetfillcolor{currentfill}%
\pgfsetlinewidth{0.803000pt}%
\definecolor{currentstroke}{rgb}{0.000000,0.000000,0.000000}%
\pgfsetstrokecolor{currentstroke}%
\pgfsetdash{}{0pt}%
\pgfsys@defobject{currentmarker}{\pgfqpoint{-0.048611in}{0.000000in}}{\pgfqpoint{0.000000in}{0.000000in}}{%
\pgfpathmoveto{\pgfqpoint{0.000000in}{0.000000in}}%
\pgfpathlineto{\pgfqpoint{-0.048611in}{0.000000in}}%
\pgfusepath{stroke,fill}%
}%
\begin{pgfscope}%
\pgfsys@transformshift{3.745974in}{1.854879in}%
\pgfsys@useobject{currentmarker}{}%
\end{pgfscope}%
\end{pgfscope}%
\begin{pgfscope}%
\pgfpathrectangle{\pgfqpoint{3.745974in}{0.526079in}}{\pgfqpoint{2.358696in}{1.661000in}} %
\pgfusepath{clip}%
\pgfsetrectcap%
\pgfsetroundjoin%
\pgfsetlinewidth{1.003750pt}%
\definecolor{currentstroke}{rgb}{1.000000,0.549020,0.000000}%
\pgfsetstrokecolor{currentstroke}%
\pgfsetdash{}{0pt}%
\pgfpathmoveto{\pgfqpoint{3.746727in}{0.512191in}}%
\pgfpathlineto{\pgfqpoint{3.749069in}{0.775916in}}%
\pgfpathlineto{\pgfqpoint{3.752313in}{0.875970in}}%
\pgfpathlineto{\pgfqpoint{3.758652in}{0.985874in}}%
\pgfpathlineto{\pgfqpoint{3.759389in}{0.982352in}}%
\pgfpathlineto{\pgfqpoint{3.761452in}{0.961758in}}%
\pgfpathlineto{\pgfqpoint{3.764696in}{0.892874in}}%
\pgfpathlineto{\pgfqpoint{3.769266in}{0.798795in}}%
\pgfpathlineto{\pgfqpoint{3.769855in}{0.791208in}}%
\pgfpathlineto{\pgfqpoint{3.770592in}{0.799454in}}%
\pgfpathlineto{\pgfqpoint{3.772361in}{0.835163in}}%
\pgfpathlineto{\pgfqpoint{3.773246in}{0.823464in}}%
\pgfpathlineto{\pgfqpoint{3.774425in}{0.809979in}}%
\pgfpathlineto{\pgfqpoint{3.775015in}{0.814807in}}%
\pgfpathlineto{\pgfqpoint{3.779290in}{0.926657in}}%
\pgfpathlineto{\pgfqpoint{3.782828in}{0.991385in}}%
\pgfpathlineto{\pgfqpoint{3.785777in}{1.006183in}}%
\pgfpathlineto{\pgfqpoint{3.786219in}{1.003658in}}%
\pgfpathlineto{\pgfqpoint{3.788283in}{0.969207in}}%
\pgfpathlineto{\pgfqpoint{3.796243in}{0.778966in}}%
\pgfpathlineto{\pgfqpoint{3.797423in}{0.805576in}}%
\pgfpathlineto{\pgfqpoint{3.803909in}{1.006928in}}%
\pgfpathlineto{\pgfqpoint{3.810101in}{1.068899in}}%
\pgfpathlineto{\pgfqpoint{3.810543in}{1.068247in}}%
\pgfpathlineto{\pgfqpoint{3.812164in}{1.058022in}}%
\pgfpathlineto{\pgfqpoint{3.819683in}{0.975354in}}%
\pgfpathlineto{\pgfqpoint{3.820272in}{0.977110in}}%
\pgfpathlineto{\pgfqpoint{3.820420in}{0.977177in}}%
\pgfpathlineto{\pgfqpoint{3.820715in}{0.975763in}}%
\pgfpathlineto{\pgfqpoint{3.822779in}{0.957319in}}%
\pgfpathlineto{\pgfqpoint{3.823663in}{0.961536in}}%
\pgfpathlineto{\pgfqpoint{3.826022in}{0.991463in}}%
\pgfpathlineto{\pgfqpoint{3.830002in}{1.031002in}}%
\pgfpathlineto{\pgfqpoint{3.834130in}{1.061908in}}%
\pgfpathlineto{\pgfqpoint{3.835014in}{1.057799in}}%
\pgfpathlineto{\pgfqpoint{3.835457in}{1.056764in}}%
\pgfpathlineto{\pgfqpoint{3.836194in}{1.058304in}}%
\pgfpathlineto{\pgfqpoint{3.836488in}{1.057920in}}%
\pgfpathlineto{\pgfqpoint{3.837815in}{1.055864in}}%
\pgfpathlineto{\pgfqpoint{3.840469in}{1.040449in}}%
\pgfpathlineto{\pgfqpoint{3.840764in}{1.040882in}}%
\pgfpathlineto{\pgfqpoint{3.844596in}{1.059129in}}%
\pgfpathlineto{\pgfqpoint{3.844744in}{1.058724in}}%
\pgfpathlineto{\pgfqpoint{3.846365in}{1.057413in}}%
\pgfpathlineto{\pgfqpoint{3.846513in}{1.057596in}}%
\pgfpathlineto{\pgfqpoint{3.846955in}{1.057928in}}%
\pgfpathlineto{\pgfqpoint{3.847250in}{1.056966in}}%
\pgfpathlineto{\pgfqpoint{3.852115in}{1.036980in}}%
\pgfpathlineto{\pgfqpoint{3.853147in}{1.035604in}}%
\pgfpathlineto{\pgfqpoint{3.853589in}{1.037273in}}%
\pgfpathlineto{\pgfqpoint{3.860223in}{1.102251in}}%
\pgfpathlineto{\pgfqpoint{3.860960in}{1.096330in}}%
\pgfpathlineto{\pgfqpoint{3.864940in}{1.066957in}}%
\pgfpathlineto{\pgfqpoint{3.865382in}{1.068215in}}%
\pgfpathlineto{\pgfqpoint{3.871279in}{1.105339in}}%
\pgfpathlineto{\pgfqpoint{3.875260in}{1.131606in}}%
\pgfpathlineto{\pgfqpoint{3.876439in}{1.136653in}}%
\pgfpathlineto{\pgfqpoint{3.877176in}{1.133942in}}%
\pgfpathlineto{\pgfqpoint{3.883368in}{1.079149in}}%
\pgfpathlineto{\pgfqpoint{3.883957in}{1.081768in}}%
\pgfpathlineto{\pgfqpoint{3.893982in}{1.164827in}}%
\pgfpathlineto{\pgfqpoint{3.896930in}{1.176843in}}%
\pgfpathlineto{\pgfqpoint{3.897225in}{1.176599in}}%
\pgfpathlineto{\pgfqpoint{3.900321in}{1.170419in}}%
\pgfpathlineto{\pgfqpoint{3.905038in}{1.117348in}}%
\pgfpathlineto{\pgfqpoint{3.909166in}{1.071966in}}%
\pgfpathlineto{\pgfqpoint{3.909608in}{1.073692in}}%
\pgfpathlineto{\pgfqpoint{3.912704in}{1.097579in}}%
\pgfpathlineto{\pgfqpoint{3.919190in}{1.155850in}}%
\pgfpathlineto{\pgfqpoint{3.921696in}{1.161272in}}%
\pgfpathlineto{\pgfqpoint{3.922286in}{1.163079in}}%
\pgfpathlineto{\pgfqpoint{3.922876in}{1.160650in}}%
\pgfpathlineto{\pgfqpoint{3.926119in}{1.138557in}}%
\pgfpathlineto{\pgfqpoint{3.934522in}{1.068460in}}%
\pgfpathlineto{\pgfqpoint{3.934964in}{1.069440in}}%
\pgfpathlineto{\pgfqpoint{3.935259in}{1.067949in}}%
\pgfpathlineto{\pgfqpoint{3.935849in}{1.061763in}}%
\pgfpathlineto{\pgfqpoint{3.936880in}{1.065647in}}%
\pgfpathlineto{\pgfqpoint{3.938060in}{1.071122in}}%
\pgfpathlineto{\pgfqpoint{3.948969in}{1.199306in}}%
\pgfpathlineto{\pgfqpoint{3.949411in}{1.197919in}}%
\pgfpathlineto{\pgfqpoint{3.953539in}{1.159351in}}%
\pgfpathlineto{\pgfqpoint{3.956635in}{1.136529in}}%
\pgfpathlineto{\pgfqpoint{3.957077in}{1.137685in}}%
\pgfpathlineto{\pgfqpoint{3.957372in}{1.138249in}}%
\pgfpathlineto{\pgfqpoint{3.958109in}{1.135913in}}%
\pgfpathlineto{\pgfqpoint{3.960467in}{1.133266in}}%
\pgfpathlineto{\pgfqpoint{3.960910in}{1.132507in}}%
\pgfpathlineto{\pgfqpoint{3.961499in}{1.134293in}}%
\pgfpathlineto{\pgfqpoint{3.964448in}{1.146305in}}%
\pgfpathlineto{\pgfqpoint{3.965922in}{1.151314in}}%
\pgfpathlineto{\pgfqpoint{3.966364in}{1.150161in}}%
\pgfpathlineto{\pgfqpoint{3.968133in}{1.157025in}}%
\pgfpathlineto{\pgfqpoint{3.973293in}{1.188982in}}%
\pgfpathlineto{\pgfqpoint{3.974472in}{1.187217in}}%
\pgfpathlineto{\pgfqpoint{3.977715in}{1.171741in}}%
\pgfpathlineto{\pgfqpoint{3.980369in}{1.159088in}}%
\pgfpathlineto{\pgfqpoint{3.981843in}{1.155277in}}%
\pgfpathlineto{\pgfqpoint{3.982433in}{1.157720in}}%
\pgfpathlineto{\pgfqpoint{3.982580in}{1.157940in}}%
\pgfpathlineto{\pgfqpoint{3.983022in}{1.156220in}}%
\pgfpathlineto{\pgfqpoint{3.984054in}{1.155208in}}%
\pgfpathlineto{\pgfqpoint{3.984349in}{1.156134in}}%
\pgfpathlineto{\pgfqpoint{3.987003in}{1.166331in}}%
\pgfpathlineto{\pgfqpoint{3.987740in}{1.167678in}}%
\pgfpathlineto{\pgfqpoint{3.990099in}{1.175399in}}%
\pgfpathlineto{\pgfqpoint{3.990541in}{1.174199in}}%
\pgfpathlineto{\pgfqpoint{3.992310in}{1.173763in}}%
\pgfpathlineto{\pgfqpoint{3.995700in}{1.174132in}}%
\pgfpathlineto{\pgfqpoint{4.004398in}{1.191695in}}%
\pgfpathlineto{\pgfqpoint{4.004546in}{1.191576in}}%
\pgfpathlineto{\pgfqpoint{4.006315in}{1.188705in}}%
\pgfpathlineto{\pgfqpoint{4.006609in}{1.189326in}}%
\pgfpathlineto{\pgfqpoint{4.007936in}{1.191078in}}%
\pgfpathlineto{\pgfqpoint{4.008231in}{1.190441in}}%
\pgfpathlineto{\pgfqpoint{4.008526in}{1.189496in}}%
\pgfpathlineto{\pgfqpoint{4.009410in}{1.191906in}}%
\pgfpathlineto{\pgfqpoint{4.011327in}{1.199575in}}%
\pgfpathlineto{\pgfqpoint{4.013096in}{1.197062in}}%
\pgfpathlineto{\pgfqpoint{4.013685in}{1.195523in}}%
\pgfpathlineto{\pgfqpoint{4.019140in}{1.177392in}}%
\pgfpathlineto{\pgfqpoint{4.019435in}{1.177814in}}%
\pgfpathlineto{\pgfqpoint{4.021499in}{1.179337in}}%
\pgfpathlineto{\pgfqpoint{4.023710in}{1.179339in}}%
\pgfpathlineto{\pgfqpoint{4.024594in}{1.180445in}}%
\pgfpathlineto{\pgfqpoint{4.027543in}{1.202107in}}%
\pgfpathlineto{\pgfqpoint{4.031818in}{1.233746in}}%
\pgfpathlineto{\pgfqpoint{4.034471in}{1.243318in}}%
\pgfpathlineto{\pgfqpoint{4.035503in}{1.241482in}}%
\pgfpathlineto{\pgfqpoint{4.037862in}{1.238666in}}%
\pgfpathlineto{\pgfqpoint{4.039041in}{1.237515in}}%
\pgfpathlineto{\pgfqpoint{4.044791in}{1.210149in}}%
\pgfpathlineto{\pgfqpoint{4.045528in}{1.208841in}}%
\pgfpathlineto{\pgfqpoint{4.047149in}{1.206430in}}%
\pgfpathlineto{\pgfqpoint{4.047444in}{1.206685in}}%
\pgfpathlineto{\pgfqpoint{4.050982in}{1.215972in}}%
\pgfpathlineto{\pgfqpoint{4.057027in}{1.249567in}}%
\pgfpathlineto{\pgfqpoint{4.061007in}{1.270174in}}%
\pgfpathlineto{\pgfqpoint{4.062628in}{1.269764in}}%
\pgfpathlineto{\pgfqpoint{4.062776in}{1.269243in}}%
\pgfpathlineto{\pgfqpoint{4.067493in}{1.239979in}}%
\pgfpathlineto{\pgfqpoint{4.072358in}{1.211869in}}%
\pgfpathlineto{\pgfqpoint{4.073980in}{1.210043in}}%
\pgfpathlineto{\pgfqpoint{4.074274in}{1.210902in}}%
\pgfpathlineto{\pgfqpoint{4.078992in}{1.236090in}}%
\pgfpathlineto{\pgfqpoint{4.084594in}{1.278863in}}%
\pgfpathlineto{\pgfqpoint{4.087542in}{1.292488in}}%
\pgfpathlineto{\pgfqpoint{4.089164in}{1.292550in}}%
\pgfpathlineto{\pgfqpoint{4.089311in}{1.292308in}}%
\pgfpathlineto{\pgfqpoint{4.093439in}{1.272096in}}%
\pgfpathlineto{\pgfqpoint{4.096977in}{1.261629in}}%
\pgfpathlineto{\pgfqpoint{4.097272in}{1.261213in}}%
\pgfpathlineto{\pgfqpoint{4.097861in}{1.262895in}}%
\pgfpathlineto{\pgfqpoint{4.102431in}{1.284656in}}%
\pgfpathlineto{\pgfqpoint{4.109950in}{1.317428in}}%
\pgfpathlineto{\pgfqpoint{4.113340in}{1.323698in}}%
\pgfpathlineto{\pgfqpoint{4.113488in}{1.323635in}}%
\pgfpathlineto{\pgfqpoint{4.116878in}{1.314530in}}%
\pgfpathlineto{\pgfqpoint{4.124544in}{1.294336in}}%
\pgfpathlineto{\pgfqpoint{4.126755in}{1.296797in}}%
\pgfpathlineto{\pgfqpoint{4.131031in}{1.315386in}}%
\pgfpathlineto{\pgfqpoint{4.133832in}{1.323022in}}%
\pgfpathlineto{\pgfqpoint{4.137812in}{1.329109in}}%
\pgfpathlineto{\pgfqpoint{4.137959in}{1.328940in}}%
\pgfpathlineto{\pgfqpoint{4.145625in}{1.316758in}}%
\pgfpathlineto{\pgfqpoint{4.148426in}{1.312402in}}%
\pgfpathlineto{\pgfqpoint{4.152406in}{1.315220in}}%
\pgfpathlineto{\pgfqpoint{4.156534in}{1.330293in}}%
\pgfpathlineto{\pgfqpoint{4.162136in}{1.352377in}}%
\pgfpathlineto{\pgfqpoint{4.162726in}{1.351883in}}%
\pgfpathlineto{\pgfqpoint{4.164052in}{1.352318in}}%
\pgfpathlineto{\pgfqpoint{4.164347in}{1.351568in}}%
\pgfpathlineto{\pgfqpoint{4.166853in}{1.347874in}}%
\pgfpathlineto{\pgfqpoint{4.167148in}{1.348070in}}%
\pgfpathlineto{\pgfqpoint{4.168917in}{1.347492in}}%
\pgfpathlineto{\pgfqpoint{4.169212in}{1.348119in}}%
\pgfpathlineto{\pgfqpoint{4.172603in}{1.352481in}}%
\pgfpathlineto{\pgfqpoint{4.172897in}{1.352336in}}%
\pgfpathlineto{\pgfqpoint{4.177467in}{1.354382in}}%
\pgfpathlineto{\pgfqpoint{4.178794in}{1.352744in}}%
\pgfpathlineto{\pgfqpoint{4.181153in}{1.349928in}}%
\pgfpathlineto{\pgfqpoint{4.182332in}{1.350262in}}%
\pgfpathlineto{\pgfqpoint{4.182480in}{1.350599in}}%
\pgfpathlineto{\pgfqpoint{4.182922in}{1.351305in}}%
\pgfpathlineto{\pgfqpoint{4.183512in}{1.349744in}}%
\pgfpathlineto{\pgfqpoint{4.184101in}{1.350579in}}%
\pgfpathlineto{\pgfqpoint{4.184691in}{1.351580in}}%
\pgfpathlineto{\pgfqpoint{4.185133in}{1.353027in}}%
\pgfpathlineto{\pgfqpoint{4.186313in}{1.352061in}}%
\pgfpathlineto{\pgfqpoint{4.186755in}{1.351507in}}%
\pgfpathlineto{\pgfqpoint{4.187639in}{1.352745in}}%
\pgfpathlineto{\pgfqpoint{4.196632in}{1.368774in}}%
\pgfpathlineto{\pgfqpoint{4.201791in}{1.380502in}}%
\pgfpathlineto{\pgfqpoint{4.202823in}{1.378966in}}%
\pgfpathlineto{\pgfqpoint{4.205477in}{1.374468in}}%
\pgfpathlineto{\pgfqpoint{4.209752in}{1.367736in}}%
\pgfpathlineto{\pgfqpoint{4.211963in}{1.368898in}}%
\pgfpathlineto{\pgfqpoint{4.215944in}{1.385196in}}%
\pgfpathlineto{\pgfqpoint{4.226705in}{1.425223in}}%
\pgfpathlineto{\pgfqpoint{4.227295in}{1.425825in}}%
\pgfpathlineto{\pgfqpoint{4.228032in}{1.424558in}}%
\pgfpathlineto{\pgfqpoint{4.230243in}{1.421679in}}%
\pgfpathlineto{\pgfqpoint{4.232897in}{1.413463in}}%
\pgfpathlineto{\pgfqpoint{4.237467in}{1.400821in}}%
\pgfpathlineto{\pgfqpoint{4.237614in}{1.400893in}}%
\pgfpathlineto{\pgfqpoint{4.240268in}{1.403155in}}%
\pgfpathlineto{\pgfqpoint{4.242774in}{1.414341in}}%
\pgfpathlineto{\pgfqpoint{4.247639in}{1.438356in}}%
\pgfpathlineto{\pgfqpoint{4.251177in}{1.446425in}}%
\pgfpathlineto{\pgfqpoint{4.252798in}{1.445398in}}%
\pgfpathlineto{\pgfqpoint{4.254862in}{1.440350in}}%
\pgfpathlineto{\pgfqpoint{4.260759in}{1.422238in}}%
\pgfpathlineto{\pgfqpoint{4.263560in}{1.419273in}}%
\pgfpathlineto{\pgfqpoint{4.263707in}{1.419348in}}%
\pgfpathlineto{\pgfqpoint{4.265918in}{1.424048in}}%
\pgfpathlineto{\pgfqpoint{4.271078in}{1.448853in}}%
\pgfpathlineto{\pgfqpoint{4.275796in}{1.467666in}}%
\pgfpathlineto{\pgfqpoint{4.275943in}{1.467631in}}%
\pgfpathlineto{\pgfqpoint{4.278449in}{1.465296in}}%
\pgfpathlineto{\pgfqpoint{4.288031in}{1.442872in}}%
\pgfpathlineto{\pgfqpoint{4.288768in}{1.443255in}}%
\pgfpathlineto{\pgfqpoint{4.289948in}{1.445308in}}%
\pgfpathlineto{\pgfqpoint{4.292601in}{1.454512in}}%
\pgfpathlineto{\pgfqpoint{4.302036in}{1.492862in}}%
\pgfpathlineto{\pgfqpoint{4.304247in}{1.493774in}}%
\pgfpathlineto{\pgfqpoint{4.304395in}{1.493630in}}%
\pgfpathlineto{\pgfqpoint{4.311176in}{1.483922in}}%
\pgfpathlineto{\pgfqpoint{4.311913in}{1.484734in}}%
\pgfpathlineto{\pgfqpoint{4.313682in}{1.486873in}}%
\pgfpathlineto{\pgfqpoint{4.331520in}{1.521920in}}%
\pgfpathlineto{\pgfqpoint{4.339775in}{1.526203in}}%
\pgfpathlineto{\pgfqpoint{4.363362in}{1.567938in}}%
\pgfpathlineto{\pgfqpoint{4.371618in}{1.571467in}}%
\pgfpathlineto{\pgfqpoint{4.373534in}{1.570717in}}%
\pgfpathlineto{\pgfqpoint{4.378399in}{1.570212in}}%
\pgfpathlineto{\pgfqpoint{4.381937in}{1.575070in}}%
\pgfpathlineto{\pgfqpoint{4.391961in}{1.594517in}}%
\pgfpathlineto{\pgfqpoint{4.392404in}{1.594152in}}%
\pgfpathlineto{\pgfqpoint{4.396089in}{1.591033in}}%
\pgfpathlineto{\pgfqpoint{4.399774in}{1.588543in}}%
\pgfpathlineto{\pgfqpoint{4.401986in}{1.589909in}}%
\pgfpathlineto{\pgfqpoint{4.405524in}{1.596882in}}%
\pgfpathlineto{\pgfqpoint{4.416433in}{1.626762in}}%
\pgfpathlineto{\pgfqpoint{4.418939in}{1.625667in}}%
\pgfpathlineto{\pgfqpoint{4.422772in}{1.619962in}}%
\pgfpathlineto{\pgfqpoint{4.426015in}{1.617412in}}%
\pgfpathlineto{\pgfqpoint{4.428374in}{1.619710in}}%
\pgfpathlineto{\pgfqpoint{4.431175in}{1.626203in}}%
\pgfpathlineto{\pgfqpoint{4.436482in}{1.643669in}}%
\pgfpathlineto{\pgfqpoint{4.441199in}{1.653754in}}%
\pgfpathlineto{\pgfqpoint{4.443705in}{1.654158in}}%
\pgfpathlineto{\pgfqpoint{4.447685in}{1.648869in}}%
\pgfpathlineto{\pgfqpoint{4.451813in}{1.643856in}}%
\pgfpathlineto{\pgfqpoint{4.454024in}{1.645851in}}%
\pgfpathlineto{\pgfqpoint{4.456825in}{1.652307in}}%
\pgfpathlineto{\pgfqpoint{4.466260in}{1.675050in}}%
\pgfpathlineto{\pgfqpoint{4.468471in}{1.675007in}}%
\pgfpathlineto{\pgfqpoint{4.470978in}{1.673143in}}%
\pgfpathlineto{\pgfqpoint{4.476727in}{1.667363in}}%
\pgfpathlineto{\pgfqpoint{4.478938in}{1.668646in}}%
\pgfpathlineto{\pgfqpoint{4.482476in}{1.675264in}}%
\pgfpathlineto{\pgfqpoint{4.492058in}{1.694265in}}%
\pgfpathlineto{\pgfqpoint{4.502525in}{1.696543in}}%
\pgfpathlineto{\pgfqpoint{4.507537in}{1.703159in}}%
\pgfpathlineto{\pgfqpoint{4.516235in}{1.717747in}}%
\pgfpathlineto{\pgfqpoint{4.522279in}{1.721399in}}%
\pgfpathlineto{\pgfqpoint{4.545129in}{1.741807in}}%
\pgfpathlineto{\pgfqpoint{4.553532in}{1.753763in}}%
\pgfpathlineto{\pgfqpoint{4.559134in}{1.758978in}}%
\pgfpathlineto{\pgfqpoint{4.569453in}{1.765434in}}%
\pgfpathlineto{\pgfqpoint{4.581541in}{1.781714in}}%
\pgfpathlineto{\pgfqpoint{4.585817in}{1.780226in}}%
\pgfpathlineto{\pgfqpoint{4.590681in}{1.778408in}}%
\pgfpathlineto{\pgfqpoint{4.594072in}{1.781324in}}%
\pgfpathlineto{\pgfqpoint{4.600116in}{1.794501in}}%
\pgfpathlineto{\pgfqpoint{4.605276in}{1.801839in}}%
\pgfpathlineto{\pgfqpoint{4.608814in}{1.801720in}}%
\pgfpathlineto{\pgfqpoint{4.617954in}{1.800295in}}%
\pgfpathlineto{\pgfqpoint{4.621639in}{1.806036in}}%
\pgfpathlineto{\pgfqpoint{4.630632in}{1.820922in}}%
\pgfpathlineto{\pgfqpoint{4.633580in}{1.820495in}}%
\pgfpathlineto{\pgfqpoint{4.642130in}{1.817027in}}%
\pgfpathlineto{\pgfqpoint{4.645374in}{1.820413in}}%
\pgfpathlineto{\pgfqpoint{4.657609in}{1.837830in}}%
\pgfpathlineto{\pgfqpoint{4.661147in}{1.836983in}}%
\pgfpathlineto{\pgfqpoint{4.666602in}{1.836006in}}%
\pgfpathlineto{\pgfqpoint{4.670140in}{1.839104in}}%
\pgfpathlineto{\pgfqpoint{4.681491in}{1.852284in}}%
\pgfpathlineto{\pgfqpoint{4.685914in}{1.852734in}}%
\pgfpathlineto{\pgfqpoint{4.693137in}{1.854133in}}%
\pgfpathlineto{\pgfqpoint{4.715840in}{1.869562in}}%
\pgfpathlineto{\pgfqpoint{4.727043in}{1.874677in}}%
\pgfpathlineto{\pgfqpoint{4.736478in}{1.880449in}}%
\pgfpathlineto{\pgfqpoint{4.746503in}{1.887750in}}%
\pgfpathlineto{\pgfqpoint{4.757707in}{1.889231in}}%
\pgfpathlineto{\pgfqpoint{4.763898in}{1.896100in}}%
\pgfpathlineto{\pgfqpoint{4.769647in}{1.901492in}}%
\pgfpathlineto{\pgfqpoint{4.773923in}{1.901405in}}%
\pgfpathlineto{\pgfqpoint{4.781146in}{1.900883in}}%
\pgfpathlineto{\pgfqpoint{4.784979in}{1.904847in}}%
\pgfpathlineto{\pgfqpoint{4.795888in}{1.917294in}}%
\pgfpathlineto{\pgfqpoint{4.800016in}{1.915767in}}%
\pgfpathlineto{\pgfqpoint{4.806207in}{1.913300in}}%
\pgfpathlineto{\pgfqpoint{4.809745in}{1.915584in}}%
\pgfpathlineto{\pgfqpoint{4.821244in}{1.925564in}}%
\pgfpathlineto{\pgfqpoint{4.827878in}{1.922387in}}%
\pgfpathlineto{\pgfqpoint{4.832300in}{1.922155in}}%
\pgfpathlineto{\pgfqpoint{4.836870in}{1.926138in}}%
\pgfpathlineto{\pgfqpoint{4.844683in}{1.933431in}}%
\pgfpathlineto{\pgfqpoint{4.849253in}{1.933273in}}%
\pgfpathlineto{\pgfqpoint{4.858541in}{1.933328in}}%
\pgfpathlineto{\pgfqpoint{4.874020in}{1.940865in}}%
\pgfpathlineto{\pgfqpoint{4.883160in}{1.940792in}}%
\pgfpathlineto{\pgfqpoint{4.914118in}{1.950571in}}%
\pgfpathlineto{\pgfqpoint{4.923700in}{1.951864in}}%
\pgfpathlineto{\pgfqpoint{4.935641in}{1.956153in}}%
\pgfpathlineto{\pgfqpoint{4.948613in}{1.954641in}}%
\pgfpathlineto{\pgfqpoint{4.960554in}{1.962478in}}%
\pgfpathlineto{\pgfqpoint{4.973822in}{1.960714in}}%
\pgfpathlineto{\pgfqpoint{4.984289in}{1.968205in}}%
\pgfpathlineto{\pgfqpoint{4.987974in}{1.966760in}}%
\pgfpathlineto{\pgfqpoint{4.995935in}{1.962956in}}%
\pgfpathlineto{\pgfqpoint{5.000063in}{1.965231in}}%
\pgfpathlineto{\pgfqpoint{5.009497in}{1.971637in}}%
\pgfpathlineto{\pgfqpoint{5.013772in}{1.970092in}}%
\pgfpathlineto{\pgfqpoint{5.022175in}{1.967221in}}%
\pgfpathlineto{\pgfqpoint{5.027777in}{1.970182in}}%
\pgfpathlineto{\pgfqpoint{5.033527in}{1.972280in}}%
\pgfpathlineto{\pgfqpoint{5.039128in}{1.970413in}}%
\pgfpathlineto{\pgfqpoint{5.048121in}{1.968037in}}%
\pgfpathlineto{\pgfqpoint{5.080700in}{1.972765in}}%
\pgfpathlineto{\pgfqpoint{5.092052in}{1.972822in}}%
\pgfpathlineto{\pgfqpoint{5.099717in}{1.972997in}}%
\pgfpathlineto{\pgfqpoint{5.111806in}{1.970842in}}%
\pgfpathlineto{\pgfqpoint{5.126695in}{1.974531in}}%
\pgfpathlineto{\pgfqpoint{5.136719in}{1.971319in}}%
\pgfpathlineto{\pgfqpoint{5.141584in}{1.974275in}}%
\pgfpathlineto{\pgfqpoint{5.147776in}{1.976990in}}%
\pgfpathlineto{\pgfqpoint{5.151461in}{1.975563in}}%
\pgfpathlineto{\pgfqpoint{5.161043in}{1.970720in}}%
\pgfpathlineto{\pgfqpoint{5.165171in}{1.972968in}}%
\pgfpathlineto{\pgfqpoint{5.173279in}{1.976994in}}%
\pgfpathlineto{\pgfqpoint{5.177260in}{1.974954in}}%
\pgfpathlineto{\pgfqpoint{5.186694in}{1.970105in}}%
\pgfpathlineto{\pgfqpoint{5.192149in}{1.972836in}}%
\pgfpathlineto{\pgfqpoint{5.198046in}{1.974553in}}%
\pgfpathlineto{\pgfqpoint{5.202615in}{1.972258in}}%
\pgfpathlineto{\pgfqpoint{5.211313in}{1.967497in}}%
\pgfpathlineto{\pgfqpoint{5.219274in}{1.969914in}}%
\pgfpathlineto{\pgfqpoint{5.225760in}{1.969600in}}%
\pgfpathlineto{\pgfqpoint{5.238733in}{1.966535in}}%
\pgfpathlineto{\pgfqpoint{5.249052in}{1.967506in}}%
\pgfpathlineto{\pgfqpoint{5.264089in}{1.966039in}}%
\pgfpathlineto{\pgfqpoint{5.295194in}{1.960247in}}%
\pgfpathlineto{\pgfqpoint{5.300059in}{1.959064in}}%
\pgfpathlineto{\pgfqpoint{5.305808in}{1.961228in}}%
\pgfpathlineto{\pgfqpoint{5.311705in}{1.963018in}}%
\pgfpathlineto{\pgfqpoint{5.318339in}{1.958873in}}%
\pgfpathlineto{\pgfqpoint{5.324531in}{1.955244in}}%
\pgfpathlineto{\pgfqpoint{5.327921in}{1.955995in}}%
\pgfpathlineto{\pgfqpoint{5.338830in}{1.959193in}}%
\pgfpathlineto{\pgfqpoint{5.345317in}{1.953649in}}%
\pgfpathlineto{\pgfqpoint{5.350181in}{1.951663in}}%
\pgfpathlineto{\pgfqpoint{5.354309in}{1.953784in}}%
\pgfpathlineto{\pgfqpoint{5.361385in}{1.958007in}}%
\pgfpathlineto{\pgfqpoint{5.365513in}{1.956530in}}%
\pgfpathlineto{\pgfqpoint{5.375980in}{1.950246in}}%
\pgfpathlineto{\pgfqpoint{5.381582in}{1.952688in}}%
\pgfpathlineto{\pgfqpoint{5.386741in}{1.954181in}}%
\pgfpathlineto{\pgfqpoint{5.391753in}{1.951992in}}%
\pgfpathlineto{\pgfqpoint{5.400304in}{1.947782in}}%
\pgfpathlineto{\pgfqpoint{5.405168in}{1.948628in}}%
\pgfpathlineto{\pgfqpoint{5.412245in}{1.949729in}}%
\pgfpathlineto{\pgfqpoint{5.417109in}{1.948118in}}%
\pgfpathlineto{\pgfqpoint{5.426249in}{1.945603in}}%
\pgfpathlineto{\pgfqpoint{5.438632in}{1.945646in}}%
\pgfpathlineto{\pgfqpoint{5.446593in}{1.945802in}}%
\pgfpathlineto{\pgfqpoint{5.453079in}{1.945268in}}%
\pgfpathlineto{\pgfqpoint{5.467527in}{1.941274in}}%
\pgfpathlineto{\pgfqpoint{5.477256in}{1.943346in}}%
\pgfpathlineto{\pgfqpoint{5.481974in}{1.941023in}}%
\pgfpathlineto{\pgfqpoint{5.488755in}{1.938296in}}%
\pgfpathlineto{\pgfqpoint{5.493030in}{1.939907in}}%
\pgfpathlineto{\pgfqpoint{5.501875in}{1.943413in}}%
\pgfpathlineto{\pgfqpoint{5.506592in}{1.939948in}}%
\pgfpathlineto{\pgfqpoint{5.514258in}{1.934703in}}%
\pgfpathlineto{\pgfqpoint{5.517207in}{1.935567in}}%
\pgfpathlineto{\pgfqpoint{5.525904in}{1.940498in}}%
\pgfpathlineto{\pgfqpoint{5.530327in}{1.938385in}}%
\pgfpathlineto{\pgfqpoint{5.538877in}{1.932318in}}%
\pgfpathlineto{\pgfqpoint{5.542857in}{1.934082in}}%
\pgfpathlineto{\pgfqpoint{5.552440in}{1.939206in}}%
\pgfpathlineto{\pgfqpoint{5.557599in}{1.935956in}}%
\pgfpathlineto{\pgfqpoint{5.562906in}{1.932665in}}%
\pgfpathlineto{\pgfqpoint{5.566592in}{1.933255in}}%
\pgfpathlineto{\pgfqpoint{5.577648in}{1.937978in}}%
\pgfpathlineto{\pgfqpoint{5.585904in}{1.934514in}}%
\pgfpathlineto{\pgfqpoint{5.590768in}{1.933889in}}%
\pgfpathlineto{\pgfqpoint{5.627181in}{1.938108in}}%
\pgfpathlineto{\pgfqpoint{5.633077in}{1.939492in}}%
\pgfpathlineto{\pgfqpoint{5.640891in}{1.941981in}}%
\pgfpathlineto{\pgfqpoint{5.657402in}{1.941451in}}%
\pgfpathlineto{\pgfqpoint{5.666394in}{1.946075in}}%
\pgfpathlineto{\pgfqpoint{5.670964in}{1.944250in}}%
\pgfpathlineto{\pgfqpoint{5.680104in}{1.942045in}}%
\pgfpathlineto{\pgfqpoint{5.685116in}{1.945939in}}%
\pgfpathlineto{\pgfqpoint{5.691308in}{1.948820in}}%
\pgfpathlineto{\pgfqpoint{5.695435in}{1.947042in}}%
\pgfpathlineto{\pgfqpoint{5.702806in}{1.943581in}}%
\pgfpathlineto{\pgfqpoint{5.706197in}{1.945316in}}%
\pgfpathlineto{\pgfqpoint{5.718285in}{1.952808in}}%
\pgfpathlineto{\pgfqpoint{5.722266in}{1.950184in}}%
\pgfpathlineto{\pgfqpoint{5.727573in}{1.947599in}}%
\pgfpathlineto{\pgfqpoint{5.732290in}{1.949566in}}%
\pgfpathlineto{\pgfqpoint{5.741430in}{1.956076in}}%
\pgfpathlineto{\pgfqpoint{5.745410in}{1.955369in}}%
\pgfpathlineto{\pgfqpoint{5.752929in}{1.953786in}}%
\pgfpathlineto{\pgfqpoint{5.756909in}{1.955752in}}%
\pgfpathlineto{\pgfqpoint{5.768113in}{1.960954in}}%
\pgfpathlineto{\pgfqpoint{5.780054in}{1.959151in}}%
\pgfpathlineto{\pgfqpoint{5.790815in}{1.961851in}}%
\pgfpathlineto{\pgfqpoint{5.833124in}{1.970921in}}%
\pgfpathlineto{\pgfqpoint{5.843149in}{1.968320in}}%
\pgfpathlineto{\pgfqpoint{5.848751in}{1.971535in}}%
\pgfpathlineto{\pgfqpoint{5.854795in}{1.974944in}}%
\pgfpathlineto{\pgfqpoint{5.859365in}{1.973761in}}%
\pgfpathlineto{\pgfqpoint{5.868505in}{1.971232in}}%
\pgfpathlineto{\pgfqpoint{5.882657in}{1.977797in}}%
\pgfpathlineto{\pgfqpoint{5.886932in}{1.974882in}}%
\pgfpathlineto{\pgfqpoint{5.892239in}{1.972457in}}%
\pgfpathlineto{\pgfqpoint{5.896514in}{1.974231in}}%
\pgfpathlineto{\pgfqpoint{5.907423in}{1.978959in}}%
\pgfpathlineto{\pgfqpoint{5.912878in}{1.975664in}}%
\pgfpathlineto{\pgfqpoint{5.917300in}{1.973886in}}%
\pgfpathlineto{\pgfqpoint{5.922755in}{1.976035in}}%
\pgfpathlineto{\pgfqpoint{5.929831in}{1.979662in}}%
\pgfpathlineto{\pgfqpoint{5.934106in}{1.978932in}}%
\pgfpathlineto{\pgfqpoint{5.944425in}{1.976448in}}%
\pgfpathlineto{\pgfqpoint{5.959609in}{1.978454in}}%
\pgfpathlineto{\pgfqpoint{5.968160in}{1.977030in}}%
\pgfpathlineto{\pgfqpoint{6.005457in}{1.976119in}}%
\pgfpathlineto{\pgfqpoint{6.010174in}{1.976858in}}%
\pgfpathlineto{\pgfqpoint{6.023736in}{1.977843in}}%
\pgfpathlineto{\pgfqpoint{6.034351in}{1.975111in}}%
\pgfpathlineto{\pgfqpoint{6.047029in}{1.979118in}}%
\pgfpathlineto{\pgfqpoint{6.058232in}{1.973329in}}%
\pgfpathlineto{\pgfqpoint{6.065456in}{1.977874in}}%
\pgfpathlineto{\pgfqpoint{6.070321in}{1.978590in}}%
\pgfpathlineto{\pgfqpoint{6.075923in}{1.974603in}}%
\pgfpathlineto{\pgfqpoint{6.081524in}{1.971266in}}%
\pgfpathlineto{\pgfqpoint{6.087863in}{1.973281in}}%
\pgfpathlineto{\pgfqpoint{6.094792in}{1.975500in}}%
\pgfpathlineto{\pgfqpoint{6.099657in}{1.972898in}}%
\pgfpathlineto{\pgfqpoint{6.104669in}{1.969622in}}%
\pgfpathlineto{\pgfqpoint{6.104669in}{1.969622in}}%
\pgfusepath{stroke}%
\end{pgfscope}%
\begin{pgfscope}%
\pgfpathrectangle{\pgfqpoint{3.745974in}{0.526079in}}{\pgfqpoint{2.358696in}{1.661000in}} %
\pgfusepath{clip}%
\pgfsetrectcap%
\pgfsetroundjoin%
\pgfsetlinewidth{1.003750pt}%
\definecolor{currentstroke}{rgb}{0.501961,0.000000,0.501961}%
\pgfsetstrokecolor{currentstroke}%
\pgfsetdash{}{0pt}%
\pgfpathmoveto{\pgfqpoint{3.746485in}{0.512191in}}%
\pgfpathlineto{\pgfqpoint{3.748774in}{0.871181in}}%
\pgfpathlineto{\pgfqpoint{3.752165in}{1.001831in}}%
\pgfpathlineto{\pgfqpoint{3.753934in}{1.015847in}}%
\pgfpathlineto{\pgfqpoint{3.754376in}{1.014840in}}%
\pgfpathlineto{\pgfqpoint{3.756293in}{0.995489in}}%
\pgfpathlineto{\pgfqpoint{3.758946in}{0.911782in}}%
\pgfpathlineto{\pgfqpoint{3.762779in}{0.798910in}}%
\pgfpathlineto{\pgfqpoint{3.763369in}{0.810510in}}%
\pgfpathlineto{\pgfqpoint{3.768381in}{0.943420in}}%
\pgfpathlineto{\pgfqpoint{3.773541in}{0.987833in}}%
\pgfpathlineto{\pgfqpoint{3.774278in}{0.985032in}}%
\pgfpathlineto{\pgfqpoint{3.776637in}{0.958568in}}%
\pgfpathlineto{\pgfqpoint{3.779438in}{0.876313in}}%
\pgfpathlineto{\pgfqpoint{3.781796in}{0.793158in}}%
\pgfpathlineto{\pgfqpoint{3.782238in}{0.798414in}}%
\pgfpathlineto{\pgfqpoint{3.786808in}{0.961843in}}%
\pgfpathlineto{\pgfqpoint{3.789609in}{1.015097in}}%
\pgfpathlineto{\pgfqpoint{3.790199in}{1.016361in}}%
\pgfpathlineto{\pgfqpoint{3.790641in}{1.013881in}}%
\pgfpathlineto{\pgfqpoint{3.792263in}{0.982336in}}%
\pgfpathlineto{\pgfqpoint{3.794916in}{0.856388in}}%
\pgfpathlineto{\pgfqpoint{3.797570in}{0.755206in}}%
\pgfpathlineto{\pgfqpoint{3.798012in}{0.765563in}}%
\pgfpathlineto{\pgfqpoint{3.804793in}{1.014938in}}%
\pgfpathlineto{\pgfqpoint{3.805825in}{1.020977in}}%
\pgfpathlineto{\pgfqpoint{3.806563in}{1.018267in}}%
\pgfpathlineto{\pgfqpoint{3.808479in}{0.995084in}}%
\pgfpathlineto{\pgfqpoint{3.812607in}{0.934441in}}%
\pgfpathlineto{\pgfqpoint{3.815113in}{0.928264in}}%
\pgfpathlineto{\pgfqpoint{3.813639in}{0.936137in}}%
\pgfpathlineto{\pgfqpoint{3.815408in}{0.929895in}}%
\pgfpathlineto{\pgfqpoint{3.818061in}{0.938930in}}%
\pgfpathlineto{\pgfqpoint{3.818209in}{0.938479in}}%
\pgfpathlineto{\pgfqpoint{3.818946in}{0.934908in}}%
\pgfpathlineto{\pgfqpoint{3.819978in}{0.936329in}}%
\pgfpathlineto{\pgfqpoint{3.820420in}{0.937945in}}%
\pgfpathlineto{\pgfqpoint{3.824105in}{0.984072in}}%
\pgfpathlineto{\pgfqpoint{3.825432in}{0.980060in}}%
\pgfpathlineto{\pgfqpoint{3.826906in}{0.968197in}}%
\pgfpathlineto{\pgfqpoint{3.829118in}{0.913386in}}%
\pgfpathlineto{\pgfqpoint{3.831181in}{0.808138in}}%
\pgfpathlineto{\pgfqpoint{3.833098in}{0.646977in}}%
\pgfpathlineto{\pgfqpoint{3.833835in}{0.701607in}}%
\pgfpathlineto{\pgfqpoint{3.840321in}{1.052060in}}%
\pgfpathlineto{\pgfqpoint{3.840911in}{1.055187in}}%
\pgfpathlineto{\pgfqpoint{3.841648in}{1.050651in}}%
\pgfpathlineto{\pgfqpoint{3.842975in}{1.033554in}}%
\pgfpathlineto{\pgfqpoint{3.845334in}{0.970961in}}%
\pgfpathlineto{\pgfqpoint{3.850198in}{0.690006in}}%
\pgfpathlineto{\pgfqpoint{3.851525in}{0.754689in}}%
\pgfpathlineto{\pgfqpoint{3.855358in}{0.923517in}}%
\pgfpathlineto{\pgfqpoint{3.859928in}{0.985352in}}%
\pgfpathlineto{\pgfqpoint{3.861107in}{0.998174in}}%
\pgfpathlineto{\pgfqpoint{3.863024in}{1.010879in}}%
\pgfpathlineto{\pgfqpoint{3.863761in}{1.010048in}}%
\pgfpathlineto{\pgfqpoint{3.864498in}{1.005023in}}%
\pgfpathlineto{\pgfqpoint{3.866414in}{0.975245in}}%
\pgfpathlineto{\pgfqpoint{3.869952in}{0.905710in}}%
\pgfpathlineto{\pgfqpoint{3.870247in}{0.906943in}}%
\pgfpathlineto{\pgfqpoint{3.874080in}{0.957542in}}%
\pgfpathlineto{\pgfqpoint{3.876586in}{0.985873in}}%
\pgfpathlineto{\pgfqpoint{3.877029in}{0.985699in}}%
\pgfpathlineto{\pgfqpoint{3.877176in}{0.986289in}}%
\pgfpathlineto{\pgfqpoint{3.877766in}{0.989020in}}%
\pgfpathlineto{\pgfqpoint{3.878503in}{0.985810in}}%
\pgfpathlineto{\pgfqpoint{3.879682in}{0.976647in}}%
\pgfpathlineto{\pgfqpoint{3.883220in}{0.843942in}}%
\pgfpathlineto{\pgfqpoint{3.884547in}{0.777885in}}%
\pgfpathlineto{\pgfqpoint{3.885579in}{0.803079in}}%
\pgfpathlineto{\pgfqpoint{3.890738in}{1.035019in}}%
\pgfpathlineto{\pgfqpoint{3.893539in}{1.064852in}}%
\pgfpathlineto{\pgfqpoint{3.893687in}{1.064055in}}%
\pgfpathlineto{\pgfqpoint{3.896930in}{1.018659in}}%
\pgfpathlineto{\pgfqpoint{3.900615in}{0.897740in}}%
\pgfpathlineto{\pgfqpoint{3.902532in}{0.851823in}}%
\pgfpathlineto{\pgfqpoint{3.903122in}{0.858333in}}%
\pgfpathlineto{\pgfqpoint{3.903416in}{0.855969in}}%
\pgfpathlineto{\pgfqpoint{3.903859in}{0.864148in}}%
\pgfpathlineto{\pgfqpoint{3.906365in}{0.914615in}}%
\pgfpathlineto{\pgfqpoint{3.906660in}{0.913348in}}%
\pgfpathlineto{\pgfqpoint{3.907249in}{0.923788in}}%
\pgfpathlineto{\pgfqpoint{3.913588in}{1.022857in}}%
\pgfpathlineto{\pgfqpoint{3.914620in}{1.023800in}}%
\pgfpathlineto{\pgfqpoint{3.915210in}{1.014133in}}%
\pgfpathlineto{\pgfqpoint{3.918158in}{0.940535in}}%
\pgfpathlineto{\pgfqpoint{3.921844in}{0.856953in}}%
\pgfpathlineto{\pgfqpoint{3.922286in}{0.854531in}}%
\pgfpathlineto{\pgfqpoint{3.922581in}{0.861162in}}%
\pgfpathlineto{\pgfqpoint{3.928035in}{1.027057in}}%
\pgfpathlineto{\pgfqpoint{3.929215in}{1.033414in}}%
\pgfpathlineto{\pgfqpoint{3.929952in}{1.030657in}}%
\pgfpathlineto{\pgfqpoint{3.930541in}{1.025675in}}%
\pgfpathlineto{\pgfqpoint{3.933342in}{0.954683in}}%
\pgfpathlineto{\pgfqpoint{3.936586in}{0.819605in}}%
\pgfpathlineto{\pgfqpoint{3.938355in}{0.837823in}}%
\pgfpathlineto{\pgfqpoint{3.944399in}{1.019063in}}%
\pgfpathlineto{\pgfqpoint{3.946168in}{1.022496in}}%
\pgfpathlineto{\pgfqpoint{3.946463in}{1.019761in}}%
\pgfpathlineto{\pgfqpoint{3.947789in}{1.012410in}}%
\pgfpathlineto{\pgfqpoint{3.948232in}{1.014896in}}%
\pgfpathlineto{\pgfqpoint{3.949853in}{1.017734in}}%
\pgfpathlineto{\pgfqpoint{3.949116in}{1.014444in}}%
\pgfpathlineto{\pgfqpoint{3.950001in}{1.017696in}}%
\pgfpathlineto{\pgfqpoint{3.953981in}{0.999134in}}%
\pgfpathlineto{\pgfqpoint{3.959435in}{0.929395in}}%
\pgfpathlineto{\pgfqpoint{3.960025in}{0.939789in}}%
\pgfpathlineto{\pgfqpoint{3.964153in}{0.987169in}}%
\pgfpathlineto{\pgfqpoint{3.964300in}{0.986390in}}%
\pgfpathlineto{\pgfqpoint{3.964595in}{0.984358in}}%
\pgfpathlineto{\pgfqpoint{3.965037in}{0.992025in}}%
\pgfpathlineto{\pgfqpoint{3.966364in}{0.996817in}}%
\pgfpathlineto{\pgfqpoint{3.965774in}{0.991082in}}%
\pgfpathlineto{\pgfqpoint{3.966659in}{0.993881in}}%
\pgfpathlineto{\pgfqpoint{3.969460in}{0.931727in}}%
\pgfpathlineto{\pgfqpoint{3.972261in}{0.840969in}}%
\pgfpathlineto{\pgfqpoint{3.973145in}{0.850152in}}%
\pgfpathlineto{\pgfqpoint{3.979927in}{1.067718in}}%
\pgfpathlineto{\pgfqpoint{3.980811in}{1.066087in}}%
\pgfpathlineto{\pgfqpoint{3.981990in}{1.056302in}}%
\pgfpathlineto{\pgfqpoint{3.985676in}{0.973070in}}%
\pgfpathlineto{\pgfqpoint{3.990246in}{0.802024in}}%
\pgfpathlineto{\pgfqpoint{3.990541in}{0.812456in}}%
\pgfpathlineto{\pgfqpoint{3.997764in}{1.002428in}}%
\pgfpathlineto{\pgfqpoint{4.002187in}{1.055262in}}%
\pgfpathlineto{\pgfqpoint{4.002777in}{1.051797in}}%
\pgfpathlineto{\pgfqpoint{4.005283in}{1.026415in}}%
\pgfpathlineto{\pgfqpoint{4.010295in}{0.923226in}}%
\pgfpathlineto{\pgfqpoint{4.010590in}{0.923741in}}%
\pgfpathlineto{\pgfqpoint{4.011179in}{0.914548in}}%
\pgfpathlineto{\pgfqpoint{4.011916in}{0.923937in}}%
\pgfpathlineto{\pgfqpoint{4.013538in}{0.952064in}}%
\pgfpathlineto{\pgfqpoint{4.016192in}{0.990669in}}%
\pgfpathlineto{\pgfqpoint{4.017813in}{0.997237in}}%
\pgfpathlineto{\pgfqpoint{4.018108in}{0.996126in}}%
\pgfpathlineto{\pgfqpoint{4.020172in}{0.956995in}}%
\pgfpathlineto{\pgfqpoint{4.022531in}{0.884972in}}%
\pgfpathlineto{\pgfqpoint{4.023710in}{0.894712in}}%
\pgfpathlineto{\pgfqpoint{4.024742in}{0.916388in}}%
\pgfpathlineto{\pgfqpoint{4.030639in}{1.054311in}}%
\pgfpathlineto{\pgfqpoint{4.031818in}{1.063550in}}%
\pgfpathlineto{\pgfqpoint{4.032997in}{1.060889in}}%
\pgfpathlineto{\pgfqpoint{4.036535in}{1.015068in}}%
\pgfpathlineto{\pgfqpoint{4.044054in}{0.889977in}}%
\pgfpathlineto{\pgfqpoint{4.044201in}{0.891338in}}%
\pgfpathlineto{\pgfqpoint{4.053783in}{1.064985in}}%
\pgfpathlineto{\pgfqpoint{4.053931in}{1.064572in}}%
\pgfpathlineto{\pgfqpoint{4.055995in}{1.039441in}}%
\pgfpathlineto{\pgfqpoint{4.061744in}{0.885996in}}%
\pgfpathlineto{\pgfqpoint{4.062628in}{0.912409in}}%
\pgfpathlineto{\pgfqpoint{4.065724in}{1.012965in}}%
\pgfpathlineto{\pgfqpoint{4.068673in}{1.046844in}}%
\pgfpathlineto{\pgfqpoint{4.068820in}{1.046977in}}%
\pgfpathlineto{\pgfqpoint{4.069115in}{1.045175in}}%
\pgfpathlineto{\pgfqpoint{4.071916in}{1.000818in}}%
\pgfpathlineto{\pgfqpoint{4.076043in}{0.904636in}}%
\pgfpathlineto{\pgfqpoint{4.076338in}{0.897555in}}%
\pgfpathlineto{\pgfqpoint{4.077370in}{0.909421in}}%
\pgfpathlineto{\pgfqpoint{4.080024in}{0.970082in}}%
\pgfpathlineto{\pgfqpoint{4.083267in}{1.040355in}}%
\pgfpathlineto{\pgfqpoint{4.083709in}{1.036730in}}%
\pgfpathlineto{\pgfqpoint{4.083857in}{1.036083in}}%
\pgfpathlineto{\pgfqpoint{4.084152in}{1.040291in}}%
\pgfpathlineto{\pgfqpoint{4.084446in}{1.044793in}}%
\pgfpathlineto{\pgfqpoint{4.085626in}{1.041328in}}%
\pgfpathlineto{\pgfqpoint{4.086068in}{1.037535in}}%
\pgfpathlineto{\pgfqpoint{4.087247in}{1.039885in}}%
\pgfpathlineto{\pgfqpoint{4.087837in}{1.038433in}}%
\pgfpathlineto{\pgfqpoint{4.088427in}{1.039109in}}%
\pgfpathlineto{\pgfqpoint{4.090196in}{1.024801in}}%
\pgfpathlineto{\pgfqpoint{4.096682in}{0.924619in}}%
\pgfpathlineto{\pgfqpoint{4.097861in}{0.937842in}}%
\pgfpathlineto{\pgfqpoint{4.099336in}{0.970187in}}%
\pgfpathlineto{\pgfqpoint{4.103168in}{1.045419in}}%
\pgfpathlineto{\pgfqpoint{4.106264in}{1.064419in}}%
\pgfpathlineto{\pgfqpoint{4.106412in}{1.064328in}}%
\pgfpathlineto{\pgfqpoint{4.108770in}{1.034888in}}%
\pgfpathlineto{\pgfqpoint{4.112161in}{0.964817in}}%
\pgfpathlineto{\pgfqpoint{4.113193in}{0.962298in}}%
\pgfpathlineto{\pgfqpoint{4.112751in}{0.966507in}}%
\pgfpathlineto{\pgfqpoint{4.113340in}{0.963063in}}%
\pgfpathlineto{\pgfqpoint{4.118500in}{1.069263in}}%
\pgfpathlineto{\pgfqpoint{4.119237in}{1.074367in}}%
\pgfpathlineto{\pgfqpoint{4.120416in}{1.073411in}}%
\pgfpathlineto{\pgfqpoint{4.121891in}{1.066673in}}%
\pgfpathlineto{\pgfqpoint{4.126903in}{0.972337in}}%
\pgfpathlineto{\pgfqpoint{4.129999in}{0.913899in}}%
\pgfpathlineto{\pgfqpoint{4.131031in}{0.910566in}}%
\pgfpathlineto{\pgfqpoint{4.130588in}{0.915465in}}%
\pgfpathlineto{\pgfqpoint{4.131325in}{0.914864in}}%
\pgfpathlineto{\pgfqpoint{4.134421in}{0.984124in}}%
\pgfpathlineto{\pgfqpoint{4.141202in}{1.090458in}}%
\pgfpathlineto{\pgfqpoint{4.142382in}{1.087893in}}%
\pgfpathlineto{\pgfqpoint{4.143561in}{1.076464in}}%
\pgfpathlineto{\pgfqpoint{4.149458in}{0.951431in}}%
\pgfpathlineto{\pgfqpoint{4.151669in}{0.959977in}}%
\pgfpathlineto{\pgfqpoint{4.157271in}{1.055599in}}%
\pgfpathlineto{\pgfqpoint{4.158156in}{1.051443in}}%
\pgfpathlineto{\pgfqpoint{4.161988in}{1.008554in}}%
\pgfpathlineto{\pgfqpoint{4.162578in}{1.022500in}}%
\pgfpathlineto{\pgfqpoint{4.163020in}{1.021126in}}%
\pgfpathlineto{\pgfqpoint{4.163168in}{1.022274in}}%
\pgfpathlineto{\pgfqpoint{4.169949in}{1.106110in}}%
\pgfpathlineto{\pgfqpoint{4.170981in}{1.105434in}}%
\pgfpathlineto{\pgfqpoint{4.171128in}{1.105086in}}%
\pgfpathlineto{\pgfqpoint{4.171571in}{1.107223in}}%
\pgfpathlineto{\pgfqpoint{4.171718in}{1.108163in}}%
\pgfpathlineto{\pgfqpoint{4.172160in}{1.103383in}}%
\pgfpathlineto{\pgfqpoint{4.179974in}{0.979591in}}%
\pgfpathlineto{\pgfqpoint{4.180121in}{0.979971in}}%
\pgfpathlineto{\pgfqpoint{4.180416in}{0.977710in}}%
\pgfpathlineto{\pgfqpoint{4.182922in}{0.924958in}}%
\pgfpathlineto{\pgfqpoint{4.183364in}{0.941422in}}%
\pgfpathlineto{\pgfqpoint{4.183954in}{0.936354in}}%
\pgfpathlineto{\pgfqpoint{4.186165in}{0.983769in}}%
\pgfpathlineto{\pgfqpoint{4.190735in}{1.107544in}}%
\pgfpathlineto{\pgfqpoint{4.193094in}{1.125779in}}%
\pgfpathlineto{\pgfqpoint{4.193389in}{1.125665in}}%
\pgfpathlineto{\pgfqpoint{4.193978in}{1.127135in}}%
\pgfpathlineto{\pgfqpoint{4.194273in}{1.124973in}}%
\pgfpathlineto{\pgfqpoint{4.197811in}{1.073885in}}%
\pgfpathlineto{\pgfqpoint{4.201644in}{0.956374in}}%
\pgfpathlineto{\pgfqpoint{4.201791in}{0.952582in}}%
\pgfpathlineto{\pgfqpoint{4.202381in}{0.965016in}}%
\pgfpathlineto{\pgfqpoint{4.202971in}{0.956444in}}%
\pgfpathlineto{\pgfqpoint{4.209752in}{1.042817in}}%
\pgfpathlineto{\pgfqpoint{4.212553in}{1.054691in}}%
\pgfpathlineto{\pgfqpoint{4.215207in}{1.083799in}}%
\pgfpathlineto{\pgfqpoint{4.215649in}{1.083319in}}%
\pgfpathlineto{\pgfqpoint{4.218892in}{1.111340in}}%
\pgfpathlineto{\pgfqpoint{4.221546in}{1.126186in}}%
\pgfpathlineto{\pgfqpoint{4.221693in}{1.125963in}}%
\pgfpathlineto{\pgfqpoint{4.224936in}{1.112658in}}%
\pgfpathlineto{\pgfqpoint{4.230391in}{1.063878in}}%
\pgfpathlineto{\pgfqpoint{4.234666in}{0.998982in}}%
\pgfpathlineto{\pgfqpoint{4.235993in}{1.013921in}}%
\pgfpathlineto{\pgfqpoint{4.241300in}{1.098803in}}%
\pgfpathlineto{\pgfqpoint{4.244690in}{1.135815in}}%
\pgfpathlineto{\pgfqpoint{4.244838in}{1.135340in}}%
\pgfpathlineto{\pgfqpoint{4.251029in}{1.062434in}}%
\pgfpathlineto{\pgfqpoint{4.252356in}{1.051415in}}%
\pgfpathlineto{\pgfqpoint{4.252798in}{1.055484in}}%
\pgfpathlineto{\pgfqpoint{4.259432in}{1.104669in}}%
\pgfpathlineto{\pgfqpoint{4.259874in}{1.101713in}}%
\pgfpathlineto{\pgfqpoint{4.261496in}{1.105508in}}%
\pgfpathlineto{\pgfqpoint{4.261791in}{1.103636in}}%
\pgfpathlineto{\pgfqpoint{4.264149in}{1.092684in}}%
\pgfpathlineto{\pgfqpoint{4.264297in}{1.093269in}}%
\pgfpathlineto{\pgfqpoint{4.264444in}{1.094010in}}%
\pgfpathlineto{\pgfqpoint{4.264887in}{1.090085in}}%
\pgfpathlineto{\pgfqpoint{4.266213in}{1.081653in}}%
\pgfpathlineto{\pgfqpoint{4.266656in}{1.084650in}}%
\pgfpathlineto{\pgfqpoint{4.267835in}{1.088932in}}%
\pgfpathlineto{\pgfqpoint{4.267245in}{1.084408in}}%
\pgfpathlineto{\pgfqpoint{4.268130in}{1.084816in}}%
\pgfpathlineto{\pgfqpoint{4.268425in}{1.080323in}}%
\pgfpathlineto{\pgfqpoint{4.269309in}{1.089191in}}%
\pgfpathlineto{\pgfqpoint{4.269751in}{1.088820in}}%
\pgfpathlineto{\pgfqpoint{4.269899in}{1.088188in}}%
\pgfpathlineto{\pgfqpoint{4.270341in}{1.085145in}}%
\pgfpathlineto{\pgfqpoint{4.270931in}{1.090564in}}%
\pgfpathlineto{\pgfqpoint{4.271815in}{1.092813in}}%
\pgfpathlineto{\pgfqpoint{4.272405in}{1.090026in}}%
\pgfpathlineto{\pgfqpoint{4.272552in}{1.089478in}}%
\pgfpathlineto{\pgfqpoint{4.272995in}{1.092235in}}%
\pgfpathlineto{\pgfqpoint{4.280365in}{1.148913in}}%
\pgfpathlineto{\pgfqpoint{4.280660in}{1.148301in}}%
\pgfpathlineto{\pgfqpoint{4.283756in}{1.126416in}}%
\pgfpathlineto{\pgfqpoint{4.289653in}{1.047754in}}%
\pgfpathlineto{\pgfqpoint{4.289948in}{1.050991in}}%
\pgfpathlineto{\pgfqpoint{4.296434in}{1.130943in}}%
\pgfpathlineto{\pgfqpoint{4.296876in}{1.126351in}}%
\pgfpathlineto{\pgfqpoint{4.297171in}{1.124322in}}%
\pgfpathlineto{\pgfqpoint{4.298203in}{1.126660in}}%
\pgfpathlineto{\pgfqpoint{4.298645in}{1.129061in}}%
\pgfpathlineto{\pgfqpoint{4.299235in}{1.122514in}}%
\pgfpathlineto{\pgfqpoint{4.299382in}{1.122073in}}%
\pgfpathlineto{\pgfqpoint{4.300267in}{1.123497in}}%
\pgfpathlineto{\pgfqpoint{4.301299in}{1.122360in}}%
\pgfpathlineto{\pgfqpoint{4.307343in}{1.165176in}}%
\pgfpathlineto{\pgfqpoint{4.309112in}{1.173386in}}%
\pgfpathlineto{\pgfqpoint{4.309554in}{1.171637in}}%
\pgfpathlineto{\pgfqpoint{4.309849in}{1.172145in}}%
\pgfpathlineto{\pgfqpoint{4.309997in}{1.172923in}}%
\pgfpathlineto{\pgfqpoint{4.311029in}{1.175112in}}%
\pgfpathlineto{\pgfqpoint{4.311323in}{1.172831in}}%
\pgfpathlineto{\pgfqpoint{4.321200in}{1.046664in}}%
\pgfpathlineto{\pgfqpoint{4.321790in}{1.054552in}}%
\pgfpathlineto{\pgfqpoint{4.325770in}{1.111688in}}%
\pgfpathlineto{\pgfqpoint{4.333141in}{1.212242in}}%
\pgfpathlineto{\pgfqpoint{4.334468in}{1.209829in}}%
\pgfpathlineto{\pgfqpoint{4.337416in}{1.169186in}}%
\pgfpathlineto{\pgfqpoint{4.343313in}{1.070290in}}%
\pgfpathlineto{\pgfqpoint{4.345672in}{1.093141in}}%
\pgfpathlineto{\pgfqpoint{4.349652in}{1.147911in}}%
\pgfpathlineto{\pgfqpoint{4.350979in}{1.171568in}}%
\pgfpathlineto{\pgfqpoint{4.354812in}{1.213962in}}%
\pgfpathlineto{\pgfqpoint{4.356139in}{1.223650in}}%
\pgfpathlineto{\pgfqpoint{4.358645in}{1.233665in}}%
\pgfpathlineto{\pgfqpoint{4.359529in}{1.233646in}}%
\pgfpathlineto{\pgfqpoint{4.359824in}{1.232787in}}%
\pgfpathlineto{\pgfqpoint{4.365279in}{1.194056in}}%
\pgfpathlineto{\pgfqpoint{4.369554in}{1.140219in}}%
\pgfpathlineto{\pgfqpoint{4.372649in}{1.107917in}}%
\pgfpathlineto{\pgfqpoint{4.372797in}{1.107147in}}%
\pgfpathlineto{\pgfqpoint{4.373239in}{1.110456in}}%
\pgfpathlineto{\pgfqpoint{4.383853in}{1.252461in}}%
\pgfpathlineto{\pgfqpoint{4.385033in}{1.250154in}}%
\pgfpathlineto{\pgfqpoint{4.386359in}{1.240256in}}%
\pgfpathlineto{\pgfqpoint{4.395204in}{1.131289in}}%
\pgfpathlineto{\pgfqpoint{4.395499in}{1.132100in}}%
\pgfpathlineto{\pgfqpoint{4.396089in}{1.133643in}}%
\pgfpathlineto{\pgfqpoint{4.402870in}{1.211599in}}%
\pgfpathlineto{\pgfqpoint{4.407293in}{1.248487in}}%
\pgfpathlineto{\pgfqpoint{4.407735in}{1.247796in}}%
\pgfpathlineto{\pgfqpoint{4.409062in}{1.253660in}}%
\pgfpathlineto{\pgfqpoint{4.410241in}{1.250252in}}%
\pgfpathlineto{\pgfqpoint{4.420560in}{1.214995in}}%
\pgfpathlineto{\pgfqpoint{4.421150in}{1.217608in}}%
\pgfpathlineto{\pgfqpoint{4.421592in}{1.218596in}}%
\pgfpathlineto{\pgfqpoint{4.422035in}{1.216762in}}%
\pgfpathlineto{\pgfqpoint{4.424541in}{1.206246in}}%
\pgfpathlineto{\pgfqpoint{4.424836in}{1.205721in}}%
\pgfpathlineto{\pgfqpoint{4.425425in}{1.207039in}}%
\pgfpathlineto{\pgfqpoint{4.425868in}{1.207064in}}%
\pgfpathlineto{\pgfqpoint{4.428374in}{1.214257in}}%
\pgfpathlineto{\pgfqpoint{4.435597in}{1.270183in}}%
\pgfpathlineto{\pgfqpoint{4.436187in}{1.271647in}}%
\pgfpathlineto{\pgfqpoint{4.436924in}{1.269314in}}%
\pgfpathlineto{\pgfqpoint{4.441789in}{1.253299in}}%
\pgfpathlineto{\pgfqpoint{4.444737in}{1.247154in}}%
\pgfpathlineto{\pgfqpoint{4.445474in}{1.247597in}}%
\pgfpathlineto{\pgfqpoint{4.445622in}{1.247285in}}%
\pgfpathlineto{\pgfqpoint{4.446064in}{1.249083in}}%
\pgfpathlineto{\pgfqpoint{4.448570in}{1.257567in}}%
\pgfpathlineto{\pgfqpoint{4.449012in}{1.257469in}}%
\pgfpathlineto{\pgfqpoint{4.451371in}{1.261304in}}%
\pgfpathlineto{\pgfqpoint{4.449749in}{1.256448in}}%
\pgfpathlineto{\pgfqpoint{4.451813in}{1.259759in}}%
\pgfpathlineto{\pgfqpoint{4.451961in}{1.259314in}}%
\pgfpathlineto{\pgfqpoint{4.452550in}{1.262292in}}%
\pgfpathlineto{\pgfqpoint{4.452698in}{1.262588in}}%
\pgfpathlineto{\pgfqpoint{4.453287in}{1.260527in}}%
\pgfpathlineto{\pgfqpoint{4.455351in}{1.259021in}}%
\pgfpathlineto{\pgfqpoint{4.455793in}{1.259462in}}%
\pgfpathlineto{\pgfqpoint{4.456383in}{1.257964in}}%
\pgfpathlineto{\pgfqpoint{4.456531in}{1.258014in}}%
\pgfpathlineto{\pgfqpoint{4.457710in}{1.256990in}}%
\pgfpathlineto{\pgfqpoint{4.460216in}{1.252268in}}%
\pgfpathlineto{\pgfqpoint{4.460363in}{1.252393in}}%
\pgfpathlineto{\pgfqpoint{4.460953in}{1.256074in}}%
\pgfpathlineto{\pgfqpoint{4.462132in}{1.253920in}}%
\pgfpathlineto{\pgfqpoint{4.462427in}{1.254490in}}%
\pgfpathlineto{\pgfqpoint{4.469356in}{1.300210in}}%
\pgfpathlineto{\pgfqpoint{4.472452in}{1.305779in}}%
\pgfpathlineto{\pgfqpoint{4.472747in}{1.304417in}}%
\pgfpathlineto{\pgfqpoint{4.481297in}{1.244514in}}%
\pgfpathlineto{\pgfqpoint{4.481444in}{1.244752in}}%
\pgfpathlineto{\pgfqpoint{4.487783in}{1.267652in}}%
\pgfpathlineto{\pgfqpoint{4.491616in}{1.298225in}}%
\pgfpathlineto{\pgfqpoint{4.497808in}{1.322278in}}%
\pgfpathlineto{\pgfqpoint{4.497955in}{1.322107in}}%
\pgfpathlineto{\pgfqpoint{4.502673in}{1.301706in}}%
\pgfpathlineto{\pgfqpoint{4.510781in}{1.264003in}}%
\pgfpathlineto{\pgfqpoint{4.511813in}{1.266226in}}%
\pgfpathlineto{\pgfqpoint{4.512255in}{1.267845in}}%
\pgfpathlineto{\pgfqpoint{4.516530in}{1.309958in}}%
\pgfpathlineto{\pgfqpoint{4.521100in}{1.360737in}}%
\pgfpathlineto{\pgfqpoint{4.523311in}{1.366253in}}%
\pgfpathlineto{\pgfqpoint{4.523459in}{1.365852in}}%
\pgfpathlineto{\pgfqpoint{4.526554in}{1.346530in}}%
\pgfpathlineto{\pgfqpoint{4.533188in}{1.268836in}}%
\pgfpathlineto{\pgfqpoint{4.534515in}{1.265822in}}%
\pgfpathlineto{\pgfqpoint{4.535252in}{1.267404in}}%
\pgfpathlineto{\pgfqpoint{4.546456in}{1.370552in}}%
\pgfpathlineto{\pgfqpoint{4.548372in}{1.374078in}}%
\pgfpathlineto{\pgfqpoint{4.548815in}{1.371830in}}%
\pgfpathlineto{\pgfqpoint{4.560313in}{1.316375in}}%
\pgfpathlineto{\pgfqpoint{4.560608in}{1.317267in}}%
\pgfpathlineto{\pgfqpoint{4.573728in}{1.400231in}}%
\pgfpathlineto{\pgfqpoint{4.575055in}{1.398774in}}%
\pgfpathlineto{\pgfqpoint{4.578593in}{1.380411in}}%
\pgfpathlineto{\pgfqpoint{4.582131in}{1.353916in}}%
\pgfpathlineto{\pgfqpoint{4.585374in}{1.337044in}}%
\pgfpathlineto{\pgfqpoint{4.585669in}{1.336513in}}%
\pgfpathlineto{\pgfqpoint{4.586259in}{1.338464in}}%
\pgfpathlineto{\pgfqpoint{4.589944in}{1.354575in}}%
\pgfpathlineto{\pgfqpoint{4.597020in}{1.397694in}}%
\pgfpathlineto{\pgfqpoint{4.597315in}{1.397262in}}%
\pgfpathlineto{\pgfqpoint{4.597610in}{1.397543in}}%
\pgfpathlineto{\pgfqpoint{4.597757in}{1.398194in}}%
\pgfpathlineto{\pgfqpoint{4.599084in}{1.399785in}}%
\pgfpathlineto{\pgfqpoint{4.599379in}{1.398846in}}%
\pgfpathlineto{\pgfqpoint{4.601885in}{1.393157in}}%
\pgfpathlineto{\pgfqpoint{4.602622in}{1.391743in}}%
\pgfpathlineto{\pgfqpoint{4.605718in}{1.386332in}}%
\pgfpathlineto{\pgfqpoint{4.608077in}{1.383830in}}%
\pgfpathlineto{\pgfqpoint{4.610730in}{1.386904in}}%
\pgfpathlineto{\pgfqpoint{4.611615in}{1.386245in}}%
\pgfpathlineto{\pgfqpoint{4.613826in}{1.387483in}}%
\pgfpathlineto{\pgfqpoint{4.614121in}{1.386002in}}%
\pgfpathlineto{\pgfqpoint{4.614416in}{1.385522in}}%
\pgfpathlineto{\pgfqpoint{4.614858in}{1.386968in}}%
\pgfpathlineto{\pgfqpoint{4.615448in}{1.386596in}}%
\pgfpathlineto{\pgfqpoint{4.616332in}{1.386023in}}%
\pgfpathlineto{\pgfqpoint{4.618101in}{1.389694in}}%
\pgfpathlineto{\pgfqpoint{4.618543in}{1.389743in}}%
\pgfpathlineto{\pgfqpoint{4.618691in}{1.390365in}}%
\pgfpathlineto{\pgfqpoint{4.621344in}{1.398271in}}%
\pgfpathlineto{\pgfqpoint{4.621639in}{1.398151in}}%
\pgfpathlineto{\pgfqpoint{4.625472in}{1.406188in}}%
\pgfpathlineto{\pgfqpoint{4.625767in}{1.405474in}}%
\pgfpathlineto{\pgfqpoint{4.626946in}{1.405804in}}%
\pgfpathlineto{\pgfqpoint{4.626357in}{1.406635in}}%
\pgfpathlineto{\pgfqpoint{4.627094in}{1.406117in}}%
\pgfpathlineto{\pgfqpoint{4.627683in}{1.408380in}}%
\pgfpathlineto{\pgfqpoint{4.628863in}{1.407286in}}%
\pgfpathlineto{\pgfqpoint{4.629158in}{1.406913in}}%
\pgfpathlineto{\pgfqpoint{4.629747in}{1.408692in}}%
\pgfpathlineto{\pgfqpoint{4.629895in}{1.408656in}}%
\pgfpathlineto{\pgfqpoint{4.631221in}{1.408198in}}%
\pgfpathlineto{\pgfqpoint{4.631664in}{1.409033in}}%
\pgfpathlineto{\pgfqpoint{4.633580in}{1.409746in}}%
\pgfpathlineto{\pgfqpoint{4.634760in}{1.410234in}}%
\pgfpathlineto{\pgfqpoint{4.635202in}{1.409440in}}%
\pgfpathlineto{\pgfqpoint{4.640951in}{1.403810in}}%
\pgfpathlineto{\pgfqpoint{4.643752in}{1.400756in}}%
\pgfpathlineto{\pgfqpoint{4.643899in}{1.401242in}}%
\pgfpathlineto{\pgfqpoint{4.645374in}{1.403171in}}%
\pgfpathlineto{\pgfqpoint{4.645668in}{1.402460in}}%
\pgfpathlineto{\pgfqpoint{4.645963in}{1.401923in}}%
\pgfpathlineto{\pgfqpoint{4.646406in}{1.402901in}}%
\pgfpathlineto{\pgfqpoint{4.647143in}{1.402684in}}%
\pgfpathlineto{\pgfqpoint{4.648617in}{1.403636in}}%
\pgfpathlineto{\pgfqpoint{4.648912in}{1.403352in}}%
\pgfpathlineto{\pgfqpoint{4.649207in}{1.403103in}}%
\pgfpathlineto{\pgfqpoint{4.649944in}{1.404484in}}%
\pgfpathlineto{\pgfqpoint{4.652892in}{1.410495in}}%
\pgfpathlineto{\pgfqpoint{4.655988in}{1.421806in}}%
\pgfpathlineto{\pgfqpoint{4.661442in}{1.443497in}}%
\pgfpathlineto{\pgfqpoint{4.662769in}{1.443936in}}%
\pgfpathlineto{\pgfqpoint{4.663064in}{1.443564in}}%
\pgfpathlineto{\pgfqpoint{4.665717in}{1.434726in}}%
\pgfpathlineto{\pgfqpoint{4.672499in}{1.403310in}}%
\pgfpathlineto{\pgfqpoint{4.672646in}{1.403580in}}%
\pgfpathlineto{\pgfqpoint{4.677953in}{1.422822in}}%
\pgfpathlineto{\pgfqpoint{4.682523in}{1.452508in}}%
\pgfpathlineto{\pgfqpoint{4.686503in}{1.464232in}}%
\pgfpathlineto{\pgfqpoint{4.686946in}{1.463346in}}%
\pgfpathlineto{\pgfqpoint{4.690926in}{1.450631in}}%
\pgfpathlineto{\pgfqpoint{4.696970in}{1.421551in}}%
\pgfpathlineto{\pgfqpoint{4.699624in}{1.417211in}}%
\pgfpathlineto{\pgfqpoint{4.699771in}{1.417405in}}%
\pgfpathlineto{\pgfqpoint{4.705373in}{1.446993in}}%
\pgfpathlineto{\pgfqpoint{4.710975in}{1.481004in}}%
\pgfpathlineto{\pgfqpoint{4.712744in}{1.483364in}}%
\pgfpathlineto{\pgfqpoint{4.713334in}{1.482345in}}%
\pgfpathlineto{\pgfqpoint{4.714955in}{1.478944in}}%
\pgfpathlineto{\pgfqpoint{4.719230in}{1.452568in}}%
\pgfpathlineto{\pgfqpoint{4.724095in}{1.425887in}}%
\pgfpathlineto{\pgfqpoint{4.725127in}{1.426631in}}%
\pgfpathlineto{\pgfqpoint{4.728223in}{1.440590in}}%
\pgfpathlineto{\pgfqpoint{4.731761in}{1.469861in}}%
\pgfpathlineto{\pgfqpoint{4.735741in}{1.492221in}}%
\pgfpathlineto{\pgfqpoint{4.737363in}{1.494494in}}%
\pgfpathlineto{\pgfqpoint{4.738100in}{1.493774in}}%
\pgfpathlineto{\pgfqpoint{4.740164in}{1.487919in}}%
\pgfpathlineto{\pgfqpoint{4.748419in}{1.451682in}}%
\pgfpathlineto{\pgfqpoint{4.749599in}{1.452201in}}%
\pgfpathlineto{\pgfqpoint{4.754758in}{1.466246in}}%
\pgfpathlineto{\pgfqpoint{4.761245in}{1.495968in}}%
\pgfpathlineto{\pgfqpoint{4.763603in}{1.499013in}}%
\pgfpathlineto{\pgfqpoint{4.763751in}{1.498804in}}%
\pgfpathlineto{\pgfqpoint{4.770237in}{1.479858in}}%
\pgfpathlineto{\pgfqpoint{4.773775in}{1.470871in}}%
\pgfpathlineto{\pgfqpoint{4.777166in}{1.471749in}}%
\pgfpathlineto{\pgfqpoint{4.783063in}{1.495109in}}%
\pgfpathlineto{\pgfqpoint{4.786895in}{1.508004in}}%
\pgfpathlineto{\pgfqpoint{4.790433in}{1.509802in}}%
\pgfpathlineto{\pgfqpoint{4.792202in}{1.507379in}}%
\pgfpathlineto{\pgfqpoint{4.797215in}{1.501930in}}%
\pgfpathlineto{\pgfqpoint{4.805175in}{1.503599in}}%
\pgfpathlineto{\pgfqpoint{4.807387in}{1.502791in}}%
\pgfpathlineto{\pgfqpoint{4.809893in}{1.502212in}}%
\pgfpathlineto{\pgfqpoint{4.816969in}{1.506752in}}%
\pgfpathlineto{\pgfqpoint{4.821096in}{1.513455in}}%
\pgfpathlineto{\pgfqpoint{4.824635in}{1.517171in}}%
\pgfpathlineto{\pgfqpoint{4.825814in}{1.516739in}}%
\pgfpathlineto{\pgfqpoint{4.825961in}{1.516527in}}%
\pgfpathlineto{\pgfqpoint{4.831121in}{1.507485in}}%
\pgfpathlineto{\pgfqpoint{4.835838in}{1.501376in}}%
\pgfpathlineto{\pgfqpoint{4.835986in}{1.501532in}}%
\pgfpathlineto{\pgfqpoint{4.837607in}{1.501356in}}%
\pgfpathlineto{\pgfqpoint{4.839229in}{1.502544in}}%
\pgfpathlineto{\pgfqpoint{4.851022in}{1.533523in}}%
\pgfpathlineto{\pgfqpoint{4.851317in}{1.533314in}}%
\pgfpathlineto{\pgfqpoint{4.855003in}{1.527243in}}%
\pgfpathlineto{\pgfqpoint{4.863995in}{1.502368in}}%
\pgfpathlineto{\pgfqpoint{4.865469in}{1.505675in}}%
\pgfpathlineto{\pgfqpoint{4.867976in}{1.516074in}}%
\pgfpathlineto{\pgfqpoint{4.875494in}{1.547673in}}%
\pgfpathlineto{\pgfqpoint{4.876673in}{1.546476in}}%
\pgfpathlineto{\pgfqpoint{4.879769in}{1.537356in}}%
\pgfpathlineto{\pgfqpoint{4.887582in}{1.502721in}}%
\pgfpathlineto{\pgfqpoint{4.889499in}{1.503786in}}%
\pgfpathlineto{\pgfqpoint{4.892005in}{1.512605in}}%
\pgfpathlineto{\pgfqpoint{4.899818in}{1.550561in}}%
\pgfpathlineto{\pgfqpoint{4.902914in}{1.553017in}}%
\pgfpathlineto{\pgfqpoint{4.903503in}{1.551692in}}%
\pgfpathlineto{\pgfqpoint{4.905272in}{1.547452in}}%
\pgfpathlineto{\pgfqpoint{4.913086in}{1.516137in}}%
\pgfpathlineto{\pgfqpoint{4.915297in}{1.516798in}}%
\pgfpathlineto{\pgfqpoint{4.917656in}{1.524363in}}%
\pgfpathlineto{\pgfqpoint{4.919130in}{1.532090in}}%
\pgfpathlineto{\pgfqpoint{4.924437in}{1.557683in}}%
\pgfpathlineto{\pgfqpoint{4.926796in}{1.560825in}}%
\pgfpathlineto{\pgfqpoint{4.928122in}{1.559375in}}%
\pgfpathlineto{\pgfqpoint{4.931365in}{1.548258in}}%
\pgfpathlineto{\pgfqpoint{4.938736in}{1.522458in}}%
\pgfpathlineto{\pgfqpoint{4.940505in}{1.523380in}}%
\pgfpathlineto{\pgfqpoint{4.943601in}{1.531016in}}%
\pgfpathlineto{\pgfqpoint{4.950677in}{1.551605in}}%
\pgfpathlineto{\pgfqpoint{4.956721in}{1.548656in}}%
\pgfpathlineto{\pgfqpoint{4.964535in}{1.538654in}}%
\pgfpathlineto{\pgfqpoint{4.967778in}{1.539472in}}%
\pgfpathlineto{\pgfqpoint{4.979866in}{1.554459in}}%
\pgfpathlineto{\pgfqpoint{4.984878in}{1.549972in}}%
\pgfpathlineto{\pgfqpoint{4.990628in}{1.549439in}}%
\pgfpathlineto{\pgfqpoint{4.993281in}{1.548882in}}%
\pgfpathlineto{\pgfqpoint{4.994903in}{1.547954in}}%
\pgfpathlineto{\pgfqpoint{5.001537in}{1.544212in}}%
\pgfpathlineto{\pgfqpoint{5.001684in}{1.544395in}}%
\pgfpathlineto{\pgfqpoint{5.004780in}{1.546254in}}%
\pgfpathlineto{\pgfqpoint{5.007139in}{1.550490in}}%
\pgfpathlineto{\pgfqpoint{5.014952in}{1.565395in}}%
\pgfpathlineto{\pgfqpoint{5.018342in}{1.560383in}}%
\pgfpathlineto{\pgfqpoint{5.020996in}{1.552914in}}%
\pgfpathlineto{\pgfqpoint{5.025566in}{1.541915in}}%
\pgfpathlineto{\pgfqpoint{5.026893in}{1.540724in}}%
\pgfpathlineto{\pgfqpoint{5.027335in}{1.541760in}}%
\pgfpathlineto{\pgfqpoint{5.032347in}{1.551341in}}%
\pgfpathlineto{\pgfqpoint{5.036327in}{1.563521in}}%
\pgfpathlineto{\pgfqpoint{5.039128in}{1.568044in}}%
\pgfpathlineto{\pgfqpoint{5.039571in}{1.567734in}}%
\pgfpathlineto{\pgfqpoint{5.040308in}{1.568103in}}%
\pgfpathlineto{\pgfqpoint{5.041045in}{1.567279in}}%
\pgfpathlineto{\pgfqpoint{5.044583in}{1.558004in}}%
\pgfpathlineto{\pgfqpoint{5.050922in}{1.533251in}}%
\pgfpathlineto{\pgfqpoint{5.053133in}{1.531285in}}%
\pgfpathlineto{\pgfqpoint{5.054460in}{1.532940in}}%
\pgfpathlineto{\pgfqpoint{5.066106in}{1.576862in}}%
\pgfpathlineto{\pgfqpoint{5.067433in}{1.575482in}}%
\pgfpathlineto{\pgfqpoint{5.071118in}{1.561150in}}%
\pgfpathlineto{\pgfqpoint{5.077605in}{1.538389in}}%
\pgfpathlineto{\pgfqpoint{5.077752in}{1.538425in}}%
\pgfpathlineto{\pgfqpoint{5.080111in}{1.542500in}}%
\pgfpathlineto{\pgfqpoint{5.083207in}{1.555080in}}%
\pgfpathlineto{\pgfqpoint{5.087777in}{1.573640in}}%
\pgfpathlineto{\pgfqpoint{5.091609in}{1.575726in}}%
\pgfpathlineto{\pgfqpoint{5.091757in}{1.575455in}}%
\pgfpathlineto{\pgfqpoint{5.097064in}{1.556355in}}%
\pgfpathlineto{\pgfqpoint{5.101044in}{1.540602in}}%
\pgfpathlineto{\pgfqpoint{5.103845in}{1.536964in}}%
\pgfpathlineto{\pgfqpoint{5.104140in}{1.537369in}}%
\pgfpathlineto{\pgfqpoint{5.107383in}{1.544234in}}%
\pgfpathlineto{\pgfqpoint{5.113427in}{1.568678in}}%
\pgfpathlineto{\pgfqpoint{5.116965in}{1.574037in}}%
\pgfpathlineto{\pgfqpoint{5.117113in}{1.573889in}}%
\pgfpathlineto{\pgfqpoint{5.122272in}{1.560170in}}%
\pgfpathlineto{\pgfqpoint{5.126990in}{1.545793in}}%
\pgfpathlineto{\pgfqpoint{5.127579in}{1.546071in}}%
\pgfpathlineto{\pgfqpoint{5.128022in}{1.544650in}}%
\pgfpathlineto{\pgfqpoint{5.128317in}{1.544549in}}%
\pgfpathlineto{\pgfqpoint{5.129201in}{1.545242in}}%
\pgfpathlineto{\pgfqpoint{5.141732in}{1.567408in}}%
\pgfpathlineto{\pgfqpoint{5.142027in}{1.567164in}}%
\pgfpathlineto{\pgfqpoint{5.147776in}{1.559338in}}%
\pgfpathlineto{\pgfqpoint{5.150282in}{1.556171in}}%
\pgfpathlineto{\pgfqpoint{5.154999in}{1.554484in}}%
\pgfpathlineto{\pgfqpoint{5.156768in}{1.554208in}}%
\pgfpathlineto{\pgfqpoint{5.160601in}{1.554344in}}%
\pgfpathlineto{\pgfqpoint{5.161043in}{1.554142in}}%
\pgfpathlineto{\pgfqpoint{5.161781in}{1.555185in}}%
\pgfpathlineto{\pgfqpoint{5.168414in}{1.560245in}}%
\pgfpathlineto{\pgfqpoint{5.169741in}{1.560801in}}%
\pgfpathlineto{\pgfqpoint{5.170036in}{1.560532in}}%
\pgfpathlineto{\pgfqpoint{5.170331in}{1.560337in}}%
\pgfpathlineto{\pgfqpoint{5.170921in}{1.561587in}}%
\pgfpathlineto{\pgfqpoint{5.171068in}{1.561660in}}%
\pgfpathlineto{\pgfqpoint{5.172984in}{1.562022in}}%
\pgfpathlineto{\pgfqpoint{5.175933in}{1.562692in}}%
\pgfpathlineto{\pgfqpoint{5.177112in}{1.562662in}}%
\pgfpathlineto{\pgfqpoint{5.177407in}{1.562318in}}%
\pgfpathlineto{\pgfqpoint{5.179913in}{1.561883in}}%
\pgfpathlineto{\pgfqpoint{5.180798in}{1.561161in}}%
\pgfpathlineto{\pgfqpoint{5.181387in}{1.561330in}}%
\pgfpathlineto{\pgfqpoint{5.188463in}{1.552283in}}%
\pgfpathlineto{\pgfqpoint{5.188758in}{1.552717in}}%
\pgfpathlineto{\pgfqpoint{5.190085in}{1.551179in}}%
\pgfpathlineto{\pgfqpoint{5.192444in}{1.550577in}}%
\pgfpathlineto{\pgfqpoint{5.205564in}{1.569010in}}%
\pgfpathlineto{\pgfqpoint{5.209692in}{1.556394in}}%
\pgfpathlineto{\pgfqpoint{5.215441in}{1.537041in}}%
\pgfpathlineto{\pgfqpoint{5.216915in}{1.536880in}}%
\pgfpathlineto{\pgfqpoint{5.217210in}{1.537199in}}%
\pgfpathlineto{\pgfqpoint{5.219126in}{1.539776in}}%
\pgfpathlineto{\pgfqpoint{5.225908in}{1.565358in}}%
\pgfpathlineto{\pgfqpoint{5.229003in}{1.570696in}}%
\pgfpathlineto{\pgfqpoint{5.229593in}{1.569878in}}%
\pgfpathlineto{\pgfqpoint{5.231215in}{1.568140in}}%
\pgfpathlineto{\pgfqpoint{5.240944in}{1.534077in}}%
\pgfpathlineto{\pgfqpoint{5.241239in}{1.534528in}}%
\pgfpathlineto{\pgfqpoint{5.241534in}{1.535259in}}%
\pgfpathlineto{\pgfqpoint{5.242271in}{1.533164in}}%
\pgfpathlineto{\pgfqpoint{5.242566in}{1.533646in}}%
\pgfpathlineto{\pgfqpoint{5.246546in}{1.544581in}}%
\pgfpathlineto{\pgfqpoint{5.252738in}{1.571036in}}%
\pgfpathlineto{\pgfqpoint{5.254359in}{1.574398in}}%
\pgfpathlineto{\pgfqpoint{5.255244in}{1.573967in}}%
\pgfpathlineto{\pgfqpoint{5.256128in}{1.574100in}}%
\pgfpathlineto{\pgfqpoint{5.256423in}{1.573220in}}%
\pgfpathlineto{\pgfqpoint{5.267627in}{1.529580in}}%
\pgfpathlineto{\pgfqpoint{5.268364in}{1.530815in}}%
\pgfpathlineto{\pgfqpoint{5.273966in}{1.551914in}}%
\pgfpathlineto{\pgfqpoint{5.278241in}{1.566999in}}%
\pgfpathlineto{\pgfqpoint{5.280305in}{1.567519in}}%
\pgfpathlineto{\pgfqpoint{5.280600in}{1.567027in}}%
\pgfpathlineto{\pgfqpoint{5.283696in}{1.559104in}}%
\pgfpathlineto{\pgfqpoint{5.288708in}{1.538764in}}%
\pgfpathlineto{\pgfqpoint{5.293425in}{1.531393in}}%
\pgfpathlineto{\pgfqpoint{5.294015in}{1.532750in}}%
\pgfpathlineto{\pgfqpoint{5.305808in}{1.561664in}}%
\pgfpathlineto{\pgfqpoint{5.305956in}{1.561577in}}%
\pgfpathlineto{\pgfqpoint{5.309052in}{1.557943in}}%
\pgfpathlineto{\pgfqpoint{5.315833in}{1.539192in}}%
\pgfpathlineto{\pgfqpoint{5.320255in}{1.535867in}}%
\pgfpathlineto{\pgfqpoint{5.320403in}{1.536150in}}%
\pgfpathlineto{\pgfqpoint{5.324973in}{1.544963in}}%
\pgfpathlineto{\pgfqpoint{5.325120in}{1.544853in}}%
\pgfpathlineto{\pgfqpoint{5.325415in}{1.544641in}}%
\pgfpathlineto{\pgfqpoint{5.326005in}{1.546164in}}%
\pgfpathlineto{\pgfqpoint{5.328953in}{1.550592in}}%
\pgfpathlineto{\pgfqpoint{5.329101in}{1.550294in}}%
\pgfpathlineto{\pgfqpoint{5.330722in}{1.549906in}}%
\pgfpathlineto{\pgfqpoint{5.332196in}{1.549243in}}%
\pgfpathlineto{\pgfqpoint{5.338240in}{1.541115in}}%
\pgfpathlineto{\pgfqpoint{5.342073in}{1.539561in}}%
\pgfpathlineto{\pgfqpoint{5.342221in}{1.539368in}}%
\pgfpathlineto{\pgfqpoint{5.342663in}{1.540773in}}%
\pgfpathlineto{\pgfqpoint{5.342958in}{1.541194in}}%
\pgfpathlineto{\pgfqpoint{5.343990in}{1.540373in}}%
\pgfpathlineto{\pgfqpoint{5.346201in}{1.539662in}}%
\pgfpathlineto{\pgfqpoint{5.347233in}{1.538531in}}%
\pgfpathlineto{\pgfqpoint{5.349739in}{1.536831in}}%
\pgfpathlineto{\pgfqpoint{5.350329in}{1.537281in}}%
\pgfpathlineto{\pgfqpoint{5.351066in}{1.536256in}}%
\pgfpathlineto{\pgfqpoint{5.352835in}{1.536124in}}%
\pgfpathlineto{\pgfqpoint{5.354309in}{1.536896in}}%
\pgfpathlineto{\pgfqpoint{5.354604in}{1.536176in}}%
\pgfpathlineto{\pgfqpoint{5.354899in}{1.535711in}}%
\pgfpathlineto{\pgfqpoint{5.355488in}{1.537652in}}%
\pgfpathlineto{\pgfqpoint{5.358289in}{1.539601in}}%
\pgfpathlineto{\pgfqpoint{5.360353in}{1.542219in}}%
\pgfpathlineto{\pgfqpoint{5.363007in}{1.546901in}}%
\pgfpathlineto{\pgfqpoint{5.364776in}{1.549305in}}%
\pgfpathlineto{\pgfqpoint{5.367135in}{1.550030in}}%
\pgfpathlineto{\pgfqpoint{5.367282in}{1.549771in}}%
\pgfpathlineto{\pgfqpoint{5.369493in}{1.547693in}}%
\pgfpathlineto{\pgfqpoint{5.369641in}{1.547783in}}%
\pgfpathlineto{\pgfqpoint{5.370083in}{1.548117in}}%
\pgfpathlineto{\pgfqpoint{5.370525in}{1.546948in}}%
\pgfpathlineto{\pgfqpoint{5.375243in}{1.535665in}}%
\pgfpathlineto{\pgfqpoint{5.377896in}{1.530543in}}%
\pgfpathlineto{\pgfqpoint{5.380107in}{1.527880in}}%
\pgfpathlineto{\pgfqpoint{5.382024in}{1.527531in}}%
\pgfpathlineto{\pgfqpoint{5.382613in}{1.528868in}}%
\pgfpathlineto{\pgfqpoint{5.394260in}{1.552056in}}%
\pgfpathlineto{\pgfqpoint{5.383351in}{1.528470in}}%
\pgfpathlineto{\pgfqpoint{5.394702in}{1.551048in}}%
\pgfpathlineto{\pgfqpoint{5.399567in}{1.536952in}}%
\pgfpathlineto{\pgfqpoint{5.403842in}{1.518861in}}%
\pgfpathlineto{\pgfqpoint{5.405021in}{1.516227in}}%
\pgfpathlineto{\pgfqpoint{5.405611in}{1.514958in}}%
\pgfpathlineto{\pgfqpoint{5.406643in}{1.515460in}}%
\pgfpathlineto{\pgfqpoint{5.409444in}{1.521623in}}%
\pgfpathlineto{\pgfqpoint{5.418731in}{1.553604in}}%
\pgfpathlineto{\pgfqpoint{5.418878in}{1.553423in}}%
\pgfpathlineto{\pgfqpoint{5.421974in}{1.544675in}}%
\pgfpathlineto{\pgfqpoint{5.425217in}{1.528151in}}%
\pgfpathlineto{\pgfqpoint{5.428461in}{1.514147in}}%
\pgfpathlineto{\pgfqpoint{5.430672in}{1.509244in}}%
\pgfpathlineto{\pgfqpoint{5.431114in}{1.509877in}}%
\pgfpathlineto{\pgfqpoint{5.433325in}{1.513845in}}%
\pgfpathlineto{\pgfqpoint{5.443055in}{1.553531in}}%
\pgfpathlineto{\pgfqpoint{5.443202in}{1.553352in}}%
\pgfpathlineto{\pgfqpoint{5.443497in}{1.553330in}}%
\pgfpathlineto{\pgfqpoint{5.443940in}{1.554615in}}%
\pgfpathlineto{\pgfqpoint{5.445119in}{1.554453in}}%
\pgfpathlineto{\pgfqpoint{5.445266in}{1.554220in}}%
\pgfpathlineto{\pgfqpoint{5.450279in}{1.537419in}}%
\pgfpathlineto{\pgfqpoint{5.456028in}{1.510763in}}%
\pgfpathlineto{\pgfqpoint{5.457502in}{1.510102in}}%
\pgfpathlineto{\pgfqpoint{5.457649in}{1.510220in}}%
\pgfpathlineto{\pgfqpoint{5.459418in}{1.512938in}}%
\pgfpathlineto{\pgfqpoint{5.462072in}{1.523624in}}%
\pgfpathlineto{\pgfqpoint{5.468411in}{1.547164in}}%
\pgfpathlineto{\pgfqpoint{5.469001in}{1.546344in}}%
\pgfpathlineto{\pgfqpoint{5.469590in}{1.547228in}}%
\pgfpathlineto{\pgfqpoint{5.470327in}{1.545926in}}%
\pgfpathlineto{\pgfqpoint{5.473865in}{1.535476in}}%
\pgfpathlineto{\pgfqpoint{5.477551in}{1.519507in}}%
\pgfpathlineto{\pgfqpoint{5.480352in}{1.512853in}}%
\pgfpathlineto{\pgfqpoint{5.482416in}{1.513537in}}%
\pgfpathlineto{\pgfqpoint{5.488902in}{1.530584in}}%
\pgfpathlineto{\pgfqpoint{5.491556in}{1.539374in}}%
\pgfpathlineto{\pgfqpoint{5.491998in}{1.538972in}}%
\pgfpathlineto{\pgfqpoint{5.495536in}{1.542613in}}%
\pgfpathlineto{\pgfqpoint{5.495978in}{1.541650in}}%
\pgfpathlineto{\pgfqpoint{5.505855in}{1.524440in}}%
\pgfpathlineto{\pgfqpoint{5.511310in}{1.520187in}}%
\pgfpathlineto{\pgfqpoint{5.513226in}{1.522212in}}%
\pgfpathlineto{\pgfqpoint{5.513963in}{1.521721in}}%
\pgfpathlineto{\pgfqpoint{5.515880in}{1.525200in}}%
\pgfpathlineto{\pgfqpoint{5.516322in}{1.525452in}}%
\pgfpathlineto{\pgfqpoint{5.516469in}{1.526024in}}%
\pgfpathlineto{\pgfqpoint{5.518533in}{1.528908in}}%
\pgfpathlineto{\pgfqpoint{5.521039in}{1.529654in}}%
\pgfpathlineto{\pgfqpoint{5.521187in}{1.529354in}}%
\pgfpathlineto{\pgfqpoint{5.523251in}{1.528086in}}%
\pgfpathlineto{\pgfqpoint{5.529590in}{1.525623in}}%
\pgfpathlineto{\pgfqpoint{5.536961in}{1.525058in}}%
\pgfpathlineto{\pgfqpoint{5.539614in}{1.520784in}}%
\pgfpathlineto{\pgfqpoint{5.539762in}{1.520964in}}%
\pgfpathlineto{\pgfqpoint{5.540056in}{1.521484in}}%
\pgfpathlineto{\pgfqpoint{5.540793in}{1.519810in}}%
\pgfpathlineto{\pgfqpoint{5.543594in}{1.517065in}}%
\pgfpathlineto{\pgfqpoint{5.543889in}{1.517633in}}%
\pgfpathlineto{\pgfqpoint{5.544037in}{1.517825in}}%
\pgfpathlineto{\pgfqpoint{5.544626in}{1.516195in}}%
\pgfpathlineto{\pgfqpoint{5.545069in}{1.517400in}}%
\pgfpathlineto{\pgfqpoint{5.545363in}{1.518165in}}%
\pgfpathlineto{\pgfqpoint{5.546543in}{1.517596in}}%
\pgfpathlineto{\pgfqpoint{5.546838in}{1.517497in}}%
\pgfpathlineto{\pgfqpoint{5.547132in}{1.518549in}}%
\pgfpathlineto{\pgfqpoint{5.549639in}{1.524089in}}%
\pgfpathlineto{\pgfqpoint{5.550523in}{1.525500in}}%
\pgfpathlineto{\pgfqpoint{5.555978in}{1.537422in}}%
\pgfpathlineto{\pgfqpoint{5.559073in}{1.536141in}}%
\pgfpathlineto{\pgfqpoint{5.560990in}{1.531267in}}%
\pgfpathlineto{\pgfqpoint{5.566149in}{1.515211in}}%
\pgfpathlineto{\pgfqpoint{5.568950in}{1.510003in}}%
\pgfpathlineto{\pgfqpoint{5.569098in}{1.510061in}}%
\pgfpathlineto{\pgfqpoint{5.570130in}{1.509422in}}%
\pgfpathlineto{\pgfqpoint{5.571604in}{1.511857in}}%
\pgfpathlineto{\pgfqpoint{5.574257in}{1.518618in}}%
\pgfpathlineto{\pgfqpoint{5.577943in}{1.531249in}}%
\pgfpathlineto{\pgfqpoint{5.581481in}{1.540438in}}%
\pgfpathlineto{\pgfqpoint{5.583840in}{1.541474in}}%
\pgfpathlineto{\pgfqpoint{5.583987in}{1.541235in}}%
\pgfpathlineto{\pgfqpoint{5.591505in}{1.512235in}}%
\pgfpathlineto{\pgfqpoint{5.595191in}{1.498940in}}%
\pgfpathlineto{\pgfqpoint{5.595486in}{1.498442in}}%
\pgfpathlineto{\pgfqpoint{5.596518in}{1.499399in}}%
\pgfpathlineto{\pgfqpoint{5.597255in}{1.499003in}}%
\pgfpathlineto{\pgfqpoint{5.599908in}{1.509263in}}%
\pgfpathlineto{\pgfqpoint{5.604626in}{1.535222in}}%
\pgfpathlineto{\pgfqpoint{5.608459in}{1.545462in}}%
\pgfpathlineto{\pgfqpoint{5.608753in}{1.545098in}}%
\pgfpathlineto{\pgfqpoint{5.612881in}{1.531807in}}%
\pgfpathlineto{\pgfqpoint{5.619220in}{1.501401in}}%
\pgfpathlineto{\pgfqpoint{5.620105in}{1.502346in}}%
\pgfpathlineto{\pgfqpoint{5.620842in}{1.501069in}}%
\pgfpathlineto{\pgfqpoint{5.624527in}{1.512705in}}%
\pgfpathlineto{\pgfqpoint{5.630129in}{1.540935in}}%
\pgfpathlineto{\pgfqpoint{5.632783in}{1.548354in}}%
\pgfpathlineto{\pgfqpoint{5.633815in}{1.547572in}}%
\pgfpathlineto{\pgfqpoint{5.635141in}{1.546801in}}%
\pgfpathlineto{\pgfqpoint{5.646050in}{1.507370in}}%
\pgfpathlineto{\pgfqpoint{5.646198in}{1.507528in}}%
\pgfpathlineto{\pgfqpoint{5.647524in}{1.508873in}}%
\pgfpathlineto{\pgfqpoint{5.647819in}{1.508260in}}%
\pgfpathlineto{\pgfqpoint{5.648114in}{1.507647in}}%
\pgfpathlineto{\pgfqpoint{5.648704in}{1.509994in}}%
\pgfpathlineto{\pgfqpoint{5.659318in}{1.546841in}}%
\pgfpathlineto{\pgfqpoint{5.660350in}{1.545930in}}%
\pgfpathlineto{\pgfqpoint{5.665510in}{1.529872in}}%
\pgfpathlineto{\pgfqpoint{5.670227in}{1.516028in}}%
\pgfpathlineto{\pgfqpoint{5.674060in}{1.520095in}}%
\pgfpathlineto{\pgfqpoint{5.683642in}{1.539491in}}%
\pgfpathlineto{\pgfqpoint{5.683937in}{1.538865in}}%
\pgfpathlineto{\pgfqpoint{5.686001in}{1.537221in}}%
\pgfpathlineto{\pgfqpoint{5.687475in}{1.536302in}}%
\pgfpathlineto{\pgfqpoint{5.692487in}{1.529929in}}%
\pgfpathlineto{\pgfqpoint{5.692635in}{1.530014in}}%
\pgfpathlineto{\pgfqpoint{5.694109in}{1.529214in}}%
\pgfpathlineto{\pgfqpoint{5.697204in}{1.527430in}}%
\pgfpathlineto{\pgfqpoint{5.700448in}{1.528314in}}%
\pgfpathlineto{\pgfqpoint{5.700743in}{1.527506in}}%
\pgfpathlineto{\pgfqpoint{5.700890in}{1.527314in}}%
\pgfpathlineto{\pgfqpoint{5.701480in}{1.528891in}}%
\pgfpathlineto{\pgfqpoint{5.706492in}{1.533629in}}%
\pgfpathlineto{\pgfqpoint{5.709882in}{1.538376in}}%
\pgfpathlineto{\pgfqpoint{5.711652in}{1.538997in}}%
\pgfpathlineto{\pgfqpoint{5.712241in}{1.540869in}}%
\pgfpathlineto{\pgfqpoint{5.713421in}{1.540288in}}%
\pgfpathlineto{\pgfqpoint{5.714305in}{1.541534in}}%
\pgfpathlineto{\pgfqpoint{5.718138in}{1.543485in}}%
\pgfpathlineto{\pgfqpoint{5.718433in}{1.542947in}}%
\pgfpathlineto{\pgfqpoint{5.718728in}{1.542695in}}%
\pgfpathlineto{\pgfqpoint{5.719317in}{1.544479in}}%
\pgfpathlineto{\pgfqpoint{5.721529in}{1.544416in}}%
\pgfpathlineto{\pgfqpoint{5.721676in}{1.544133in}}%
\pgfpathlineto{\pgfqpoint{5.724329in}{1.541602in}}%
\pgfpathlineto{\pgfqpoint{5.724477in}{1.541893in}}%
\pgfpathlineto{\pgfqpoint{5.724772in}{1.542385in}}%
\pgfpathlineto{\pgfqpoint{5.725361in}{1.540516in}}%
\pgfpathlineto{\pgfqpoint{5.728162in}{1.535641in}}%
\pgfpathlineto{\pgfqpoint{5.728605in}{1.536454in}}%
\pgfpathlineto{\pgfqpoint{5.729047in}{1.535082in}}%
\pgfpathlineto{\pgfqpoint{5.731406in}{1.530681in}}%
\pgfpathlineto{\pgfqpoint{5.736860in}{1.530148in}}%
\pgfpathlineto{\pgfqpoint{5.740840in}{1.539280in}}%
\pgfpathlineto{\pgfqpoint{5.744821in}{1.553420in}}%
\pgfpathlineto{\pgfqpoint{5.748801in}{1.555764in}}%
\pgfpathlineto{\pgfqpoint{5.748948in}{1.555472in}}%
\pgfpathlineto{\pgfqpoint{5.758531in}{1.529695in}}%
\pgfpathlineto{\pgfqpoint{5.759857in}{1.529602in}}%
\pgfpathlineto{\pgfqpoint{5.759268in}{1.530776in}}%
\pgfpathlineto{\pgfqpoint{5.760005in}{1.529827in}}%
\pgfpathlineto{\pgfqpoint{5.770177in}{1.564322in}}%
\pgfpathlineto{\pgfqpoint{5.771798in}{1.565771in}}%
\pgfpathlineto{\pgfqpoint{5.772093in}{1.566344in}}%
\pgfpathlineto{\pgfqpoint{5.772683in}{1.564832in}}%
\pgfpathlineto{\pgfqpoint{5.773125in}{1.565289in}}%
\pgfpathlineto{\pgfqpoint{5.773862in}{1.563754in}}%
\pgfpathlineto{\pgfqpoint{5.776663in}{1.554027in}}%
\pgfpathlineto{\pgfqpoint{5.784329in}{1.522962in}}%
\pgfpathlineto{\pgfqpoint{5.784624in}{1.523678in}}%
\pgfpathlineto{\pgfqpoint{5.785508in}{1.522342in}}%
\pgfpathlineto{\pgfqpoint{5.786688in}{1.524165in}}%
\pgfpathlineto{\pgfqpoint{5.787425in}{1.525466in}}%
\pgfpathlineto{\pgfqpoint{5.797596in}{1.569069in}}%
\pgfpathlineto{\pgfqpoint{5.798186in}{1.568410in}}%
\pgfpathlineto{\pgfqpoint{5.800987in}{1.564709in}}%
\pgfpathlineto{\pgfqpoint{5.804967in}{1.547570in}}%
\pgfpathlineto{\pgfqpoint{5.808211in}{1.534379in}}%
\pgfpathlineto{\pgfqpoint{5.810569in}{1.531336in}}%
\pgfpathlineto{\pgfqpoint{5.810717in}{1.531402in}}%
\pgfpathlineto{\pgfqpoint{5.811749in}{1.534611in}}%
\pgfpathlineto{\pgfqpoint{5.819414in}{1.568622in}}%
\pgfpathlineto{\pgfqpoint{5.822658in}{1.574713in}}%
\pgfpathlineto{\pgfqpoint{5.822805in}{1.574462in}}%
\pgfpathlineto{\pgfqpoint{5.825606in}{1.569067in}}%
\pgfpathlineto{\pgfqpoint{5.826048in}{1.569392in}}%
\pgfpathlineto{\pgfqpoint{5.827522in}{1.563365in}}%
\pgfpathlineto{\pgfqpoint{5.834009in}{1.543167in}}%
\pgfpathlineto{\pgfqpoint{5.836073in}{1.540502in}}%
\pgfpathlineto{\pgfqpoint{5.836515in}{1.541200in}}%
\pgfpathlineto{\pgfqpoint{5.840200in}{1.546635in}}%
\pgfpathlineto{\pgfqpoint{5.844033in}{1.559999in}}%
\pgfpathlineto{\pgfqpoint{5.846097in}{1.565756in}}%
\pgfpathlineto{\pgfqpoint{5.848308in}{1.569722in}}%
\pgfpathlineto{\pgfqpoint{5.849046in}{1.568807in}}%
\pgfpathlineto{\pgfqpoint{5.850962in}{1.567984in}}%
\pgfpathlineto{\pgfqpoint{5.852289in}{1.565083in}}%
\pgfpathlineto{\pgfqpoint{5.854942in}{1.558490in}}%
\pgfpathlineto{\pgfqpoint{5.855385in}{1.558824in}}%
\pgfpathlineto{\pgfqpoint{5.857006in}{1.554036in}}%
\pgfpathlineto{\pgfqpoint{5.859954in}{1.549099in}}%
\pgfpathlineto{\pgfqpoint{5.865262in}{1.554082in}}%
\pgfpathlineto{\pgfqpoint{5.870569in}{1.565576in}}%
\pgfpathlineto{\pgfqpoint{5.873222in}{1.568406in}}%
\pgfpathlineto{\pgfqpoint{5.873517in}{1.568427in}}%
\pgfpathlineto{\pgfqpoint{5.873959in}{1.567515in}}%
\pgfpathlineto{\pgfqpoint{5.874402in}{1.567406in}}%
\pgfpathlineto{\pgfqpoint{5.875581in}{1.567822in}}%
\pgfpathlineto{\pgfqpoint{5.876023in}{1.566417in}}%
\pgfpathlineto{\pgfqpoint{5.879561in}{1.563307in}}%
\pgfpathlineto{\pgfqpoint{5.879856in}{1.563625in}}%
\pgfpathlineto{\pgfqpoint{5.880298in}{1.563840in}}%
\pgfpathlineto{\pgfqpoint{5.881183in}{1.562901in}}%
\pgfpathlineto{\pgfqpoint{5.881920in}{1.562642in}}%
\pgfpathlineto{\pgfqpoint{5.882362in}{1.563888in}}%
\pgfpathlineto{\pgfqpoint{5.882510in}{1.564129in}}%
\pgfpathlineto{\pgfqpoint{5.883099in}{1.562497in}}%
\pgfpathlineto{\pgfqpoint{5.883247in}{1.562243in}}%
\pgfpathlineto{\pgfqpoint{5.883836in}{1.563374in}}%
\pgfpathlineto{\pgfqpoint{5.884426in}{1.563072in}}%
\pgfpathlineto{\pgfqpoint{5.886342in}{1.563252in}}%
\pgfpathlineto{\pgfqpoint{5.888996in}{1.561718in}}%
\pgfpathlineto{\pgfqpoint{5.889291in}{1.562147in}}%
\pgfpathlineto{\pgfqpoint{5.889733in}{1.562482in}}%
\pgfpathlineto{\pgfqpoint{5.890323in}{1.561030in}}%
\pgfpathlineto{\pgfqpoint{5.892681in}{1.558497in}}%
\pgfpathlineto{\pgfqpoint{5.894303in}{1.557289in}}%
\pgfpathlineto{\pgfqpoint{5.898431in}{1.555796in}}%
\pgfpathlineto{\pgfqpoint{5.901969in}{1.560122in}}%
\pgfpathlineto{\pgfqpoint{5.911256in}{1.569645in}}%
\pgfpathlineto{\pgfqpoint{5.912435in}{1.568198in}}%
\pgfpathlineto{\pgfqpoint{5.915236in}{1.564426in}}%
\pgfpathlineto{\pgfqpoint{5.917153in}{1.560372in}}%
\pgfpathlineto{\pgfqpoint{5.920101in}{1.553602in}}%
\pgfpathlineto{\pgfqpoint{5.924524in}{1.550538in}}%
\pgfpathlineto{\pgfqpoint{5.924671in}{1.550802in}}%
\pgfpathlineto{\pgfqpoint{5.930126in}{1.562128in}}%
\pgfpathlineto{\pgfqpoint{5.936612in}{1.579144in}}%
\pgfpathlineto{\pgfqpoint{5.936760in}{1.578928in}}%
\pgfpathlineto{\pgfqpoint{5.946047in}{1.551781in}}%
\pgfpathlineto{\pgfqpoint{5.948848in}{1.545413in}}%
\pgfpathlineto{\pgfqpoint{5.948995in}{1.545627in}}%
\pgfpathlineto{\pgfqpoint{5.954597in}{1.561421in}}%
\pgfpathlineto{\pgfqpoint{5.960494in}{1.582628in}}%
\pgfpathlineto{\pgfqpoint{5.964179in}{1.578196in}}%
\pgfpathlineto{\pgfqpoint{5.966243in}{1.570288in}}%
\pgfpathlineto{\pgfqpoint{5.970371in}{1.551019in}}%
\pgfpathlineto{\pgfqpoint{5.973909in}{1.542202in}}%
\pgfpathlineto{\pgfqpoint{5.974499in}{1.543735in}}%
\pgfpathlineto{\pgfqpoint{5.977300in}{1.547218in}}%
\pgfpathlineto{\pgfqpoint{5.987471in}{1.582560in}}%
\pgfpathlineto{\pgfqpoint{5.987619in}{1.582321in}}%
\pgfpathlineto{\pgfqpoint{5.994400in}{1.561721in}}%
\pgfpathlineto{\pgfqpoint{5.998380in}{1.545989in}}%
\pgfpathlineto{\pgfqpoint{5.998528in}{1.546026in}}%
\pgfpathlineto{\pgfqpoint{5.998970in}{1.546354in}}%
\pgfpathlineto{\pgfqpoint{5.999412in}{1.544776in}}%
\pgfpathlineto{\pgfqpoint{6.000739in}{1.544508in}}%
\pgfpathlineto{\pgfqpoint{6.000887in}{1.544630in}}%
\pgfpathlineto{\pgfqpoint{6.005162in}{1.558738in}}%
\pgfpathlineto{\pgfqpoint{6.009584in}{1.576727in}}%
\pgfpathlineto{\pgfqpoint{6.009732in}{1.576663in}}%
\pgfpathlineto{\pgfqpoint{6.010027in}{1.576670in}}%
\pgfpathlineto{\pgfqpoint{6.010321in}{1.577799in}}%
\pgfpathlineto{\pgfqpoint{6.011796in}{1.580382in}}%
\pgfpathlineto{\pgfqpoint{6.012090in}{1.580011in}}%
\pgfpathlineto{\pgfqpoint{6.020788in}{1.556851in}}%
\pgfpathlineto{\pgfqpoint{6.024326in}{1.550144in}}%
\pgfpathlineto{\pgfqpoint{6.024621in}{1.550554in}}%
\pgfpathlineto{\pgfqpoint{6.028601in}{1.556445in}}%
\pgfpathlineto{\pgfqpoint{6.028896in}{1.556192in}}%
\pgfpathlineto{\pgfqpoint{6.029486in}{1.557444in}}%
\pgfpathlineto{\pgfqpoint{6.034351in}{1.568139in}}%
\pgfpathlineto{\pgfqpoint{6.037741in}{1.572869in}}%
\pgfpathlineto{\pgfqpoint{6.040395in}{1.571760in}}%
\pgfpathlineto{\pgfqpoint{6.042164in}{1.569176in}}%
\pgfpathlineto{\pgfqpoint{6.042311in}{1.569355in}}%
\pgfpathlineto{\pgfqpoint{6.042606in}{1.569529in}}%
\pgfpathlineto{\pgfqpoint{6.043048in}{1.568389in}}%
\pgfpathlineto{\pgfqpoint{6.046586in}{1.562113in}}%
\pgfpathlineto{\pgfqpoint{6.048060in}{1.560636in}}%
\pgfpathlineto{\pgfqpoint{6.051304in}{1.557023in}}%
\pgfpathlineto{\pgfqpoint{6.051451in}{1.557229in}}%
\pgfpathlineto{\pgfqpoint{6.052925in}{1.557772in}}%
\pgfpathlineto{\pgfqpoint{6.053073in}{1.557447in}}%
\pgfpathlineto{\pgfqpoint{6.053515in}{1.556095in}}%
\pgfpathlineto{\pgfqpoint{6.054399in}{1.558315in}}%
\pgfpathlineto{\pgfqpoint{6.056168in}{1.558210in}}%
\pgfpathlineto{\pgfqpoint{6.056316in}{1.558432in}}%
\pgfpathlineto{\pgfqpoint{6.058969in}{1.561967in}}%
\pgfpathlineto{\pgfqpoint{6.061918in}{1.565578in}}%
\pgfpathlineto{\pgfqpoint{6.062802in}{1.564708in}}%
\pgfpathlineto{\pgfqpoint{6.063687in}{1.565279in}}%
\pgfpathlineto{\pgfqpoint{6.066488in}{1.563731in}}%
\pgfpathlineto{\pgfqpoint{6.066783in}{1.564524in}}%
\pgfpathlineto{\pgfqpoint{6.067077in}{1.565197in}}%
\pgfpathlineto{\pgfqpoint{6.067815in}{1.562956in}}%
\pgfpathlineto{\pgfqpoint{6.071058in}{1.564237in}}%
\pgfpathlineto{\pgfqpoint{6.073711in}{1.565284in}}%
\pgfpathlineto{\pgfqpoint{6.076217in}{1.566372in}}%
\pgfpathlineto{\pgfqpoint{6.076660in}{1.566516in}}%
\pgfpathlineto{\pgfqpoint{6.077397in}{1.565454in}}%
\pgfpathlineto{\pgfqpoint{6.079018in}{1.564482in}}%
\pgfpathlineto{\pgfqpoint{6.082262in}{1.559116in}}%
\pgfpathlineto{\pgfqpoint{6.084178in}{1.555820in}}%
\pgfpathlineto{\pgfqpoint{6.086979in}{1.552405in}}%
\pgfpathlineto{\pgfqpoint{6.090959in}{1.553984in}}%
\pgfpathlineto{\pgfqpoint{6.093760in}{1.559369in}}%
\pgfpathlineto{\pgfqpoint{6.096414in}{1.566079in}}%
\pgfpathlineto{\pgfqpoint{6.099067in}{1.570034in}}%
\pgfpathlineto{\pgfqpoint{6.099215in}{1.569868in}}%
\pgfpathlineto{\pgfqpoint{6.100099in}{1.570631in}}%
\pgfpathlineto{\pgfqpoint{6.100836in}{1.570134in}}%
\pgfpathlineto{\pgfqpoint{6.101426in}{1.570166in}}%
\pgfpathlineto{\pgfqpoint{6.101868in}{1.569075in}}%
\pgfpathlineto{\pgfqpoint{6.104669in}{1.562020in}}%
\pgfpathlineto{\pgfqpoint{6.104669in}{1.562020in}}%
\pgfusepath{stroke}%
\end{pgfscope}%
\begin{pgfscope}%
\pgfpathrectangle{\pgfqpoint{3.745974in}{0.526079in}}{\pgfqpoint{2.358696in}{1.661000in}} %
\pgfusepath{clip}%
\pgfsetbuttcap%
\pgfsetroundjoin%
\pgfsetlinewidth{1.003750pt}%
\definecolor{currentstroke}{rgb}{0.627451,0.321569,0.176471}%
\pgfsetstrokecolor{currentstroke}%
\pgfsetdash{{3.700000pt}{1.600000pt}}{0.000000pt}%
\pgfpathmoveto{\pgfqpoint{3.748930in}{0.512191in}}%
\pgfpathlineto{\pgfqpoint{3.752902in}{0.914934in}}%
\pgfpathlineto{\pgfqpoint{3.757472in}{1.108429in}}%
\pgfpathlineto{\pgfqpoint{3.761305in}{1.166900in}}%
\pgfpathlineto{\pgfqpoint{3.763074in}{1.172315in}}%
\pgfpathlineto{\pgfqpoint{3.763516in}{1.171865in}}%
\pgfpathlineto{\pgfqpoint{3.764990in}{1.165877in}}%
\pgfpathlineto{\pgfqpoint{3.768381in}{1.135536in}}%
\pgfpathlineto{\pgfqpoint{3.775162in}{1.078465in}}%
\pgfpathlineto{\pgfqpoint{3.775899in}{1.078136in}}%
\pgfpathlineto{\pgfqpoint{3.776489in}{1.078926in}}%
\pgfpathlineto{\pgfqpoint{3.780617in}{1.089867in}}%
\pgfpathlineto{\pgfqpoint{3.781796in}{1.087431in}}%
\pgfpathlineto{\pgfqpoint{3.784007in}{1.072021in}}%
\pgfpathlineto{\pgfqpoint{3.788725in}{1.030454in}}%
\pgfpathlineto{\pgfqpoint{3.789462in}{1.032182in}}%
\pgfpathlineto{\pgfqpoint{3.791231in}{1.049826in}}%
\pgfpathlineto{\pgfqpoint{3.796685in}{1.104949in}}%
\pgfpathlineto{\pgfqpoint{3.797570in}{1.103138in}}%
\pgfpathlineto{\pgfqpoint{3.799339in}{1.089657in}}%
\pgfpathlineto{\pgfqpoint{3.805088in}{1.018233in}}%
\pgfpathlineto{\pgfqpoint{3.806268in}{1.027288in}}%
\pgfpathlineto{\pgfqpoint{3.809216in}{1.094851in}}%
\pgfpathlineto{\pgfqpoint{3.814376in}{1.186878in}}%
\pgfpathlineto{\pgfqpoint{3.818209in}{1.210951in}}%
\pgfpathlineto{\pgfqpoint{3.819240in}{1.211700in}}%
\pgfpathlineto{\pgfqpoint{3.819830in}{1.210868in}}%
\pgfpathlineto{\pgfqpoint{3.821599in}{1.202543in}}%
\pgfpathlineto{\pgfqpoint{3.824990in}{1.166523in}}%
\pgfpathlineto{\pgfqpoint{3.832656in}{1.082825in}}%
\pgfpathlineto{\pgfqpoint{3.834425in}{1.079754in}}%
\pgfpathlineto{\pgfqpoint{3.835014in}{1.080323in}}%
\pgfpathlineto{\pgfqpoint{3.836341in}{1.086075in}}%
\pgfpathlineto{\pgfqpoint{3.838700in}{1.115182in}}%
\pgfpathlineto{\pgfqpoint{3.848282in}{1.255731in}}%
\pgfpathlineto{\pgfqpoint{3.850198in}{1.259502in}}%
\pgfpathlineto{\pgfqpoint{3.850641in}{1.259116in}}%
\pgfpathlineto{\pgfqpoint{3.852410in}{1.253544in}}%
\pgfpathlineto{\pgfqpoint{3.860813in}{1.220850in}}%
\pgfpathlineto{\pgfqpoint{3.862582in}{1.222710in}}%
\pgfpathlineto{\pgfqpoint{3.867741in}{1.232731in}}%
\pgfpathlineto{\pgfqpoint{3.868626in}{1.231685in}}%
\pgfpathlineto{\pgfqpoint{3.870690in}{1.223759in}}%
\pgfpathlineto{\pgfqpoint{3.874817in}{1.191128in}}%
\pgfpathlineto{\pgfqpoint{3.878208in}{1.175999in}}%
\pgfpathlineto{\pgfqpoint{3.879387in}{1.177419in}}%
\pgfpathlineto{\pgfqpoint{3.883368in}{1.186876in}}%
\pgfpathlineto{\pgfqpoint{3.884252in}{1.185205in}}%
\pgfpathlineto{\pgfqpoint{3.886021in}{1.173870in}}%
\pgfpathlineto{\pgfqpoint{3.889264in}{1.127923in}}%
\pgfpathlineto{\pgfqpoint{3.892950in}{1.084126in}}%
\pgfpathlineto{\pgfqpoint{3.893245in}{1.084726in}}%
\pgfpathlineto{\pgfqpoint{3.894424in}{1.095182in}}%
\pgfpathlineto{\pgfqpoint{3.898552in}{1.175206in}}%
\pgfpathlineto{\pgfqpoint{3.902974in}{1.229020in}}%
\pgfpathlineto{\pgfqpoint{3.906070in}{1.239103in}}%
\pgfpathlineto{\pgfqpoint{3.907397in}{1.237842in}}%
\pgfpathlineto{\pgfqpoint{3.909608in}{1.228914in}}%
\pgfpathlineto{\pgfqpoint{3.913883in}{1.195114in}}%
\pgfpathlineto{\pgfqpoint{3.918158in}{1.171186in}}%
\pgfpathlineto{\pgfqpoint{3.919338in}{1.170967in}}%
\pgfpathlineto{\pgfqpoint{3.919780in}{1.171518in}}%
\pgfpathlineto{\pgfqpoint{3.922581in}{1.179011in}}%
\pgfpathlineto{\pgfqpoint{3.927888in}{1.201046in}}%
\pgfpathlineto{\pgfqpoint{3.936880in}{1.257635in}}%
\pgfpathlineto{\pgfqpoint{3.937912in}{1.256114in}}%
\pgfpathlineto{\pgfqpoint{3.939829in}{1.245911in}}%
\pgfpathlineto{\pgfqpoint{3.946315in}{1.197830in}}%
\pgfpathlineto{\pgfqpoint{3.947347in}{1.199975in}}%
\pgfpathlineto{\pgfqpoint{3.949411in}{1.214349in}}%
\pgfpathlineto{\pgfqpoint{3.958698in}{1.290084in}}%
\pgfpathlineto{\pgfqpoint{3.959878in}{1.288993in}}%
\pgfpathlineto{\pgfqpoint{3.961942in}{1.280246in}}%
\pgfpathlineto{\pgfqpoint{3.965922in}{1.246130in}}%
\pgfpathlineto{\pgfqpoint{3.972408in}{1.195196in}}%
\pgfpathlineto{\pgfqpoint{3.974030in}{1.192964in}}%
\pgfpathlineto{\pgfqpoint{3.974472in}{1.193308in}}%
\pgfpathlineto{\pgfqpoint{3.976094in}{1.197870in}}%
\pgfpathlineto{\pgfqpoint{3.978895in}{1.217115in}}%
\pgfpathlineto{\pgfqpoint{3.988772in}{1.294390in}}%
\pgfpathlineto{\pgfqpoint{3.990688in}{1.295298in}}%
\pgfpathlineto{\pgfqpoint{3.990983in}{1.295069in}}%
\pgfpathlineto{\pgfqpoint{3.993194in}{1.290871in}}%
\pgfpathlineto{\pgfqpoint{3.997617in}{1.273614in}}%
\pgfpathlineto{\pgfqpoint{4.003071in}{1.256060in}}%
\pgfpathlineto{\pgfqpoint{4.004840in}{1.256682in}}%
\pgfpathlineto{\pgfqpoint{4.010590in}{1.261613in}}%
\pgfpathlineto{\pgfqpoint{4.010737in}{1.261474in}}%
\pgfpathlineto{\pgfqpoint{4.014423in}{1.255425in}}%
\pgfpathlineto{\pgfqpoint{4.017518in}{1.253608in}}%
\pgfpathlineto{\pgfqpoint{4.020172in}{1.256085in}}%
\pgfpathlineto{\pgfqpoint{4.022088in}{1.256715in}}%
\pgfpathlineto{\pgfqpoint{4.022383in}{1.256383in}}%
\pgfpathlineto{\pgfqpoint{4.024005in}{1.251469in}}%
\pgfpathlineto{\pgfqpoint{4.026658in}{1.230874in}}%
\pgfpathlineto{\pgfqpoint{4.032113in}{1.185177in}}%
\pgfpathlineto{\pgfqpoint{4.032555in}{1.185565in}}%
\pgfpathlineto{\pgfqpoint{4.033882in}{1.191903in}}%
\pgfpathlineto{\pgfqpoint{4.037272in}{1.229675in}}%
\pgfpathlineto{\pgfqpoint{4.042285in}{1.273250in}}%
\pgfpathlineto{\pgfqpoint{4.044643in}{1.278010in}}%
\pgfpathlineto{\pgfqpoint{4.044938in}{1.277864in}}%
\pgfpathlineto{\pgfqpoint{4.046412in}{1.274415in}}%
\pgfpathlineto{\pgfqpoint{4.048918in}{1.257821in}}%
\pgfpathlineto{\pgfqpoint{4.054226in}{1.194451in}}%
\pgfpathlineto{\pgfqpoint{4.058796in}{1.158134in}}%
\pgfpathlineto{\pgfqpoint{4.059827in}{1.156955in}}%
\pgfpathlineto{\pgfqpoint{4.060417in}{1.157617in}}%
\pgfpathlineto{\pgfqpoint{4.062186in}{1.164448in}}%
\pgfpathlineto{\pgfqpoint{4.065724in}{1.192889in}}%
\pgfpathlineto{\pgfqpoint{4.075306in}{1.275537in}}%
\pgfpathlineto{\pgfqpoint{4.076338in}{1.275336in}}%
\pgfpathlineto{\pgfqpoint{4.076633in}{1.274830in}}%
\pgfpathlineto{\pgfqpoint{4.078402in}{1.267584in}}%
\pgfpathlineto{\pgfqpoint{4.082235in}{1.234632in}}%
\pgfpathlineto{\pgfqpoint{4.086068in}{1.213447in}}%
\pgfpathlineto{\pgfqpoint{4.087100in}{1.214602in}}%
\pgfpathlineto{\pgfqpoint{4.089016in}{1.224747in}}%
\pgfpathlineto{\pgfqpoint{4.097124in}{1.283333in}}%
\pgfpathlineto{\pgfqpoint{4.097861in}{1.282377in}}%
\pgfpathlineto{\pgfqpoint{4.099630in}{1.274333in}}%
\pgfpathlineto{\pgfqpoint{4.103906in}{1.235038in}}%
\pgfpathlineto{\pgfqpoint{4.107738in}{1.212580in}}%
\pgfpathlineto{\pgfqpoint{4.108623in}{1.212697in}}%
\pgfpathlineto{\pgfqpoint{4.109065in}{1.213469in}}%
\pgfpathlineto{\pgfqpoint{4.111277in}{1.222398in}}%
\pgfpathlineto{\pgfqpoint{4.116289in}{1.257676in}}%
\pgfpathlineto{\pgfqpoint{4.129409in}{1.367140in}}%
\pgfpathlineto{\pgfqpoint{4.131031in}{1.365746in}}%
\pgfpathlineto{\pgfqpoint{4.134274in}{1.356749in}}%
\pgfpathlineto{\pgfqpoint{4.138696in}{1.347676in}}%
\pgfpathlineto{\pgfqpoint{4.140171in}{1.349530in}}%
\pgfpathlineto{\pgfqpoint{4.142677in}{1.359935in}}%
\pgfpathlineto{\pgfqpoint{4.150637in}{1.396473in}}%
\pgfpathlineto{\pgfqpoint{4.152111in}{1.395588in}}%
\pgfpathlineto{\pgfqpoint{4.154470in}{1.388458in}}%
\pgfpathlineto{\pgfqpoint{4.163168in}{1.357780in}}%
\pgfpathlineto{\pgfqpoint{4.165527in}{1.358668in}}%
\pgfpathlineto{\pgfqpoint{4.168770in}{1.364354in}}%
\pgfpathlineto{\pgfqpoint{4.172897in}{1.378258in}}%
\pgfpathlineto{\pgfqpoint{4.182774in}{1.413197in}}%
\pgfpathlineto{\pgfqpoint{4.184691in}{1.412842in}}%
\pgfpathlineto{\pgfqpoint{4.186902in}{1.407576in}}%
\pgfpathlineto{\pgfqpoint{4.191177in}{1.386982in}}%
\pgfpathlineto{\pgfqpoint{4.196484in}{1.366369in}}%
\pgfpathlineto{\pgfqpoint{4.199138in}{1.364788in}}%
\pgfpathlineto{\pgfqpoint{4.201791in}{1.367210in}}%
\pgfpathlineto{\pgfqpoint{4.205329in}{1.375491in}}%
\pgfpathlineto{\pgfqpoint{4.211816in}{1.390217in}}%
\pgfpathlineto{\pgfqpoint{4.213438in}{1.388704in}}%
\pgfpathlineto{\pgfqpoint{4.216091in}{1.380207in}}%
\pgfpathlineto{\pgfqpoint{4.224052in}{1.352178in}}%
\pgfpathlineto{\pgfqpoint{4.225526in}{1.353719in}}%
\pgfpathlineto{\pgfqpoint{4.228032in}{1.362312in}}%
\pgfpathlineto{\pgfqpoint{4.238941in}{1.409218in}}%
\pgfpathlineto{\pgfqpoint{4.239531in}{1.408844in}}%
\pgfpathlineto{\pgfqpoint{4.241594in}{1.404518in}}%
\pgfpathlineto{\pgfqpoint{4.248228in}{1.386920in}}%
\pgfpathlineto{\pgfqpoint{4.248671in}{1.387182in}}%
\pgfpathlineto{\pgfqpoint{4.250440in}{1.390769in}}%
\pgfpathlineto{\pgfqpoint{4.253830in}{1.406602in}}%
\pgfpathlineto{\pgfqpoint{4.266950in}{1.475423in}}%
\pgfpathlineto{\pgfqpoint{4.268719in}{1.474699in}}%
\pgfpathlineto{\pgfqpoint{4.271668in}{1.468398in}}%
\pgfpathlineto{\pgfqpoint{4.275796in}{1.461756in}}%
\pgfpathlineto{\pgfqpoint{4.277417in}{1.463797in}}%
\pgfpathlineto{\pgfqpoint{4.280218in}{1.474343in}}%
\pgfpathlineto{\pgfqpoint{4.288621in}{1.508836in}}%
\pgfpathlineto{\pgfqpoint{4.290832in}{1.508823in}}%
\pgfpathlineto{\pgfqpoint{4.293338in}{1.504143in}}%
\pgfpathlineto{\pgfqpoint{4.301889in}{1.485932in}}%
\pgfpathlineto{\pgfqpoint{4.303952in}{1.487317in}}%
\pgfpathlineto{\pgfqpoint{4.307048in}{1.494363in}}%
\pgfpathlineto{\pgfqpoint{4.315156in}{1.513601in}}%
\pgfpathlineto{\pgfqpoint{4.318399in}{1.514713in}}%
\pgfpathlineto{\pgfqpoint{4.321200in}{1.512229in}}%
\pgfpathlineto{\pgfqpoint{4.324886in}{1.504006in}}%
\pgfpathlineto{\pgfqpoint{4.331667in}{1.488934in}}%
\pgfpathlineto{\pgfqpoint{4.333584in}{1.490097in}}%
\pgfpathlineto{\pgfqpoint{4.337122in}{1.497701in}}%
\pgfpathlineto{\pgfqpoint{4.341102in}{1.503400in}}%
\pgfpathlineto{\pgfqpoint{4.343461in}{1.502097in}}%
\pgfpathlineto{\pgfqpoint{4.350242in}{1.496899in}}%
\pgfpathlineto{\pgfqpoint{4.353485in}{1.499104in}}%
\pgfpathlineto{\pgfqpoint{4.357908in}{1.506650in}}%
\pgfpathlineto{\pgfqpoint{4.363510in}{1.523510in}}%
\pgfpathlineto{\pgfqpoint{4.379283in}{1.572791in}}%
\pgfpathlineto{\pgfqpoint{4.392109in}{1.595805in}}%
\pgfpathlineto{\pgfqpoint{4.400069in}{1.615686in}}%
\pgfpathlineto{\pgfqpoint{4.403607in}{1.617891in}}%
\pgfpathlineto{\pgfqpoint{4.406556in}{1.616029in}}%
\pgfpathlineto{\pgfqpoint{4.416728in}{1.605740in}}%
\pgfpathlineto{\pgfqpoint{4.417170in}{1.606187in}}%
\pgfpathlineto{\pgfqpoint{4.420413in}{1.612414in}}%
\pgfpathlineto{\pgfqpoint{4.426605in}{1.623875in}}%
\pgfpathlineto{\pgfqpoint{4.428816in}{1.623078in}}%
\pgfpathlineto{\pgfqpoint{4.431764in}{1.617619in}}%
\pgfpathlineto{\pgfqpoint{4.439872in}{1.601478in}}%
\pgfpathlineto{\pgfqpoint{4.441936in}{1.602756in}}%
\pgfpathlineto{\pgfqpoint{4.445032in}{1.609593in}}%
\pgfpathlineto{\pgfqpoint{4.453730in}{1.630649in}}%
\pgfpathlineto{\pgfqpoint{4.456088in}{1.629944in}}%
\pgfpathlineto{\pgfqpoint{4.460216in}{1.624095in}}%
\pgfpathlineto{\pgfqpoint{4.464491in}{1.620171in}}%
\pgfpathlineto{\pgfqpoint{4.466702in}{1.621730in}}%
\pgfpathlineto{\pgfqpoint{4.469651in}{1.628596in}}%
\pgfpathlineto{\pgfqpoint{4.481444in}{1.661428in}}%
\pgfpathlineto{\pgfqpoint{4.484393in}{1.659642in}}%
\pgfpathlineto{\pgfqpoint{4.489110in}{1.657226in}}%
\pgfpathlineto{\pgfqpoint{4.491764in}{1.659973in}}%
\pgfpathlineto{\pgfqpoint{4.496334in}{1.670515in}}%
\pgfpathlineto{\pgfqpoint{4.503704in}{1.685729in}}%
\pgfpathlineto{\pgfqpoint{4.507685in}{1.687753in}}%
\pgfpathlineto{\pgfqpoint{4.512550in}{1.686185in}}%
\pgfpathlineto{\pgfqpoint{4.517267in}{1.685689in}}%
\pgfpathlineto{\pgfqpoint{4.520658in}{1.688928in}}%
\pgfpathlineto{\pgfqpoint{4.526407in}{1.700149in}}%
\pgfpathlineto{\pgfqpoint{4.533630in}{1.712041in}}%
\pgfpathlineto{\pgfqpoint{4.538495in}{1.714454in}}%
\pgfpathlineto{\pgfqpoint{4.548815in}{1.717505in}}%
\pgfpathlineto{\pgfqpoint{4.562230in}{1.727977in}}%
\pgfpathlineto{\pgfqpoint{4.567979in}{1.731739in}}%
\pgfpathlineto{\pgfqpoint{4.575645in}{1.734645in}}%
\pgfpathlineto{\pgfqpoint{4.579625in}{1.741879in}}%
\pgfpathlineto{\pgfqpoint{4.593335in}{1.770781in}}%
\pgfpathlineto{\pgfqpoint{4.596726in}{1.770087in}}%
\pgfpathlineto{\pgfqpoint{4.602770in}{1.768489in}}%
\pgfpathlineto{\pgfqpoint{4.605865in}{1.771741in}}%
\pgfpathlineto{\pgfqpoint{4.611910in}{1.784259in}}%
\pgfpathlineto{\pgfqpoint{4.617364in}{1.792389in}}%
\pgfpathlineto{\pgfqpoint{4.620460in}{1.792359in}}%
\pgfpathlineto{\pgfqpoint{4.625325in}{1.787902in}}%
\pgfpathlineto{\pgfqpoint{4.629010in}{1.786514in}}%
\pgfpathlineto{\pgfqpoint{4.631811in}{1.789150in}}%
\pgfpathlineto{\pgfqpoint{4.638298in}{1.801859in}}%
\pgfpathlineto{\pgfqpoint{4.642425in}{1.805960in}}%
\pgfpathlineto{\pgfqpoint{4.645226in}{1.805000in}}%
\pgfpathlineto{\pgfqpoint{4.649354in}{1.799472in}}%
\pgfpathlineto{\pgfqpoint{4.655693in}{1.791973in}}%
\pgfpathlineto{\pgfqpoint{4.658641in}{1.792844in}}%
\pgfpathlineto{\pgfqpoint{4.663064in}{1.798582in}}%
\pgfpathlineto{\pgfqpoint{4.668371in}{1.803804in}}%
\pgfpathlineto{\pgfqpoint{4.671762in}{1.803109in}}%
\pgfpathlineto{\pgfqpoint{4.682228in}{1.798336in}}%
\pgfpathlineto{\pgfqpoint{4.685914in}{1.802872in}}%
\pgfpathlineto{\pgfqpoint{4.696086in}{1.817069in}}%
\pgfpathlineto{\pgfqpoint{4.711565in}{1.826545in}}%
\pgfpathlineto{\pgfqpoint{4.719673in}{1.841011in}}%
\pgfpathlineto{\pgfqpoint{4.726749in}{1.850427in}}%
\pgfpathlineto{\pgfqpoint{4.734267in}{1.854956in}}%
\pgfpathlineto{\pgfqpoint{4.741933in}{1.860390in}}%
\pgfpathlineto{\pgfqpoint{4.758296in}{1.875463in}}%
\pgfpathlineto{\pgfqpoint{4.764193in}{1.873078in}}%
\pgfpathlineto{\pgfqpoint{4.768763in}{1.872818in}}%
\pgfpathlineto{\pgfqpoint{4.773038in}{1.876270in}}%
\pgfpathlineto{\pgfqpoint{4.779819in}{1.881467in}}%
\pgfpathlineto{\pgfqpoint{4.782915in}{1.880002in}}%
\pgfpathlineto{\pgfqpoint{4.788370in}{1.872749in}}%
\pgfpathlineto{\pgfqpoint{4.793234in}{1.868677in}}%
\pgfpathlineto{\pgfqpoint{4.796183in}{1.869980in}}%
\pgfpathlineto{\pgfqpoint{4.800458in}{1.876223in}}%
\pgfpathlineto{\pgfqpoint{4.807387in}{1.885408in}}%
\pgfpathlineto{\pgfqpoint{4.811072in}{1.885475in}}%
\pgfpathlineto{\pgfqpoint{4.819180in}{1.884120in}}%
\pgfpathlineto{\pgfqpoint{4.822865in}{1.888409in}}%
\pgfpathlineto{\pgfqpoint{4.833922in}{1.903359in}}%
\pgfpathlineto{\pgfqpoint{4.837755in}{1.902288in}}%
\pgfpathlineto{\pgfqpoint{4.844978in}{1.899822in}}%
\pgfpathlineto{\pgfqpoint{4.849106in}{1.902585in}}%
\pgfpathlineto{\pgfqpoint{4.859720in}{1.911202in}}%
\pgfpathlineto{\pgfqpoint{4.864438in}{1.909985in}}%
\pgfpathlineto{\pgfqpoint{4.871808in}{1.908141in}}%
\pgfpathlineto{\pgfqpoint{4.876673in}{1.910773in}}%
\pgfpathlineto{\pgfqpoint{4.885813in}{1.915938in}}%
\pgfpathlineto{\pgfqpoint{4.891857in}{1.915217in}}%
\pgfpathlineto{\pgfqpoint{4.907336in}{1.911548in}}%
\pgfpathlineto{\pgfqpoint{4.926501in}{1.914497in}}%
\pgfpathlineto{\pgfqpoint{4.931218in}{1.915792in}}%
\pgfpathlineto{\pgfqpoint{4.936378in}{1.920982in}}%
\pgfpathlineto{\pgfqpoint{4.945370in}{1.930053in}}%
\pgfpathlineto{\pgfqpoint{4.949498in}{1.929740in}}%
\pgfpathlineto{\pgfqpoint{4.959080in}{1.927190in}}%
\pgfpathlineto{\pgfqpoint{4.963650in}{1.931141in}}%
\pgfpathlineto{\pgfqpoint{4.970579in}{1.936513in}}%
\pgfpathlineto{\pgfqpoint{4.974264in}{1.935367in}}%
\pgfpathlineto{\pgfqpoint{4.984141in}{1.930449in}}%
\pgfpathlineto{\pgfqpoint{4.988269in}{1.933855in}}%
\pgfpathlineto{\pgfqpoint{4.995935in}{1.940181in}}%
\pgfpathlineto{\pgfqpoint{4.999620in}{1.939025in}}%
\pgfpathlineto{\pgfqpoint{5.012151in}{1.931958in}}%
\pgfpathlineto{\pgfqpoint{5.018932in}{1.936927in}}%
\pgfpathlineto{\pgfqpoint{5.023502in}{1.937854in}}%
\pgfpathlineto{\pgfqpoint{5.028957in}{1.935028in}}%
\pgfpathlineto{\pgfqpoint{5.034558in}{1.933348in}}%
\pgfpathlineto{\pgfqpoint{5.039128in}{1.935526in}}%
\pgfpathlineto{\pgfqpoint{5.048416in}{1.940495in}}%
\pgfpathlineto{\pgfqpoint{5.053281in}{1.938922in}}%
\pgfpathlineto{\pgfqpoint{5.067580in}{1.932468in}}%
\pgfpathlineto{\pgfqpoint{5.080700in}{1.933942in}}%
\pgfpathlineto{\pgfqpoint{5.094410in}{1.936501in}}%
\pgfpathlineto{\pgfqpoint{5.102371in}{1.942092in}}%
\pgfpathlineto{\pgfqpoint{5.108710in}{1.945172in}}%
\pgfpathlineto{\pgfqpoint{5.113427in}{1.943907in}}%
\pgfpathlineto{\pgfqpoint{5.121830in}{1.941189in}}%
\pgfpathlineto{\pgfqpoint{5.126548in}{1.943784in}}%
\pgfpathlineto{\pgfqpoint{5.134361in}{1.948019in}}%
\pgfpathlineto{\pgfqpoint{5.138488in}{1.946370in}}%
\pgfpathlineto{\pgfqpoint{5.148513in}{1.940999in}}%
\pgfpathlineto{\pgfqpoint{5.153083in}{1.943509in}}%
\pgfpathlineto{\pgfqpoint{5.159569in}{1.946625in}}%
\pgfpathlineto{\pgfqpoint{5.163255in}{1.944666in}}%
\pgfpathlineto{\pgfqpoint{5.174753in}{1.936024in}}%
\pgfpathlineto{\pgfqpoint{5.179471in}{1.939089in}}%
\pgfpathlineto{\pgfqpoint{5.185220in}{1.941824in}}%
\pgfpathlineto{\pgfqpoint{5.189200in}{1.940133in}}%
\pgfpathlineto{\pgfqpoint{5.199520in}{1.934128in}}%
\pgfpathlineto{\pgfqpoint{5.203942in}{1.936555in}}%
\pgfpathlineto{\pgfqpoint{5.211903in}{1.941029in}}%
\pgfpathlineto{\pgfqpoint{5.216473in}{1.939423in}}%
\pgfpathlineto{\pgfqpoint{5.225760in}{1.935408in}}%
\pgfpathlineto{\pgfqpoint{5.232689in}{1.937364in}}%
\pgfpathlineto{\pgfqpoint{5.238586in}{1.937468in}}%
\pgfpathlineto{\pgfqpoint{5.246841in}{1.933488in}}%
\pgfpathlineto{\pgfqpoint{5.255244in}{1.930783in}}%
\pgfpathlineto{\pgfqpoint{5.261583in}{1.932160in}}%
\pgfpathlineto{\pgfqpoint{5.273082in}{1.935120in}}%
\pgfpathlineto{\pgfqpoint{5.290624in}{1.935042in}}%
\pgfpathlineto{\pgfqpoint{5.297995in}{1.937006in}}%
\pgfpathlineto{\pgfqpoint{5.302418in}{1.934512in}}%
\pgfpathlineto{\pgfqpoint{5.311558in}{1.928635in}}%
\pgfpathlineto{\pgfqpoint{5.315685in}{1.930563in}}%
\pgfpathlineto{\pgfqpoint{5.323499in}{1.934679in}}%
\pgfpathlineto{\pgfqpoint{5.327184in}{1.932656in}}%
\pgfpathlineto{\pgfqpoint{5.338683in}{1.924018in}}%
\pgfpathlineto{\pgfqpoint{5.343842in}{1.926616in}}%
\pgfpathlineto{\pgfqpoint{5.348855in}{1.927829in}}%
\pgfpathlineto{\pgfqpoint{5.352393in}{1.925186in}}%
\pgfpathlineto{\pgfqpoint{5.364776in}{1.912797in}}%
\pgfpathlineto{\pgfqpoint{5.369493in}{1.915588in}}%
\pgfpathlineto{\pgfqpoint{5.375095in}{1.917814in}}%
\pgfpathlineto{\pgfqpoint{5.379223in}{1.915744in}}%
\pgfpathlineto{\pgfqpoint{5.388363in}{1.910105in}}%
\pgfpathlineto{\pgfqpoint{5.392490in}{1.911986in}}%
\pgfpathlineto{\pgfqpoint{5.403252in}{1.918291in}}%
\pgfpathlineto{\pgfqpoint{5.409296in}{1.916486in}}%
\pgfpathlineto{\pgfqpoint{5.415340in}{1.915765in}}%
\pgfpathlineto{\pgfqpoint{5.421532in}{1.918802in}}%
\pgfpathlineto{\pgfqpoint{5.428313in}{1.921008in}}%
\pgfpathlineto{\pgfqpoint{5.434505in}{1.919459in}}%
\pgfpathlineto{\pgfqpoint{5.449541in}{1.914590in}}%
\pgfpathlineto{\pgfqpoint{5.457207in}{1.917872in}}%
\pgfpathlineto{\pgfqpoint{5.462662in}{1.918502in}}%
\pgfpathlineto{\pgfqpoint{5.467674in}{1.915518in}}%
\pgfpathlineto{\pgfqpoint{5.475340in}{1.911236in}}%
\pgfpathlineto{\pgfqpoint{5.480794in}{1.912407in}}%
\pgfpathlineto{\pgfqpoint{5.487133in}{1.913129in}}%
\pgfpathlineto{\pgfqpoint{5.491408in}{1.910084in}}%
\pgfpathlineto{\pgfqpoint{5.500696in}{1.902362in}}%
\pgfpathlineto{\pgfqpoint{5.504234in}{1.904325in}}%
\pgfpathlineto{\pgfqpoint{5.514700in}{1.912378in}}%
\pgfpathlineto{\pgfqpoint{5.518533in}{1.909838in}}%
\pgfpathlineto{\pgfqpoint{5.526346in}{1.904408in}}%
\pgfpathlineto{\pgfqpoint{5.529885in}{1.906217in}}%
\pgfpathlineto{\pgfqpoint{5.539762in}{1.913115in}}%
\pgfpathlineto{\pgfqpoint{5.543594in}{1.910665in}}%
\pgfpathlineto{\pgfqpoint{5.552882in}{1.903537in}}%
\pgfpathlineto{\pgfqpoint{5.556567in}{1.905346in}}%
\pgfpathlineto{\pgfqpoint{5.565412in}{1.910801in}}%
\pgfpathlineto{\pgfqpoint{5.569688in}{1.908508in}}%
\pgfpathlineto{\pgfqpoint{5.577943in}{1.903743in}}%
\pgfpathlineto{\pgfqpoint{5.582513in}{1.905361in}}%
\pgfpathlineto{\pgfqpoint{5.593274in}{1.910418in}}%
\pgfpathlineto{\pgfqpoint{5.599024in}{1.908298in}}%
\pgfpathlineto{\pgfqpoint{5.609638in}{1.904344in}}%
\pgfpathlineto{\pgfqpoint{5.626149in}{1.901983in}}%
\pgfpathlineto{\pgfqpoint{5.637353in}{1.898156in}}%
\pgfpathlineto{\pgfqpoint{5.642217in}{1.900252in}}%
\pgfpathlineto{\pgfqpoint{5.653716in}{1.906852in}}%
\pgfpathlineto{\pgfqpoint{5.666541in}{1.906035in}}%
\pgfpathlineto{\pgfqpoint{5.673028in}{1.913911in}}%
\pgfpathlineto{\pgfqpoint{5.678335in}{1.917998in}}%
\pgfpathlineto{\pgfqpoint{5.682610in}{1.917277in}}%
\pgfpathlineto{\pgfqpoint{5.690276in}{1.915394in}}%
\pgfpathlineto{\pgfqpoint{5.694551in}{1.918417in}}%
\pgfpathlineto{\pgfqpoint{5.702954in}{1.924801in}}%
\pgfpathlineto{\pgfqpoint{5.706344in}{1.923112in}}%
\pgfpathlineto{\pgfqpoint{5.717253in}{1.915154in}}%
\pgfpathlineto{\pgfqpoint{5.721381in}{1.918422in}}%
\pgfpathlineto{\pgfqpoint{5.728752in}{1.924059in}}%
\pgfpathlineto{\pgfqpoint{5.732438in}{1.922907in}}%
\pgfpathlineto{\pgfqpoint{5.744231in}{1.916722in}}%
\pgfpathlineto{\pgfqpoint{5.758678in}{1.920536in}}%
\pgfpathlineto{\pgfqpoint{5.766491in}{1.918380in}}%
\pgfpathlineto{\pgfqpoint{5.770619in}{1.921049in}}%
\pgfpathlineto{\pgfqpoint{5.785508in}{1.933640in}}%
\pgfpathlineto{\pgfqpoint{5.793321in}{1.933652in}}%
\pgfpathlineto{\pgfqpoint{5.802756in}{1.934027in}}%
\pgfpathlineto{\pgfqpoint{5.817498in}{1.936443in}}%
\pgfpathlineto{\pgfqpoint{5.828997in}{1.936177in}}%
\pgfpathlineto{\pgfqpoint{5.835630in}{1.941104in}}%
\pgfpathlineto{\pgfqpoint{5.841822in}{1.944235in}}%
\pgfpathlineto{\pgfqpoint{5.846097in}{1.942856in}}%
\pgfpathlineto{\pgfqpoint{5.855090in}{1.938921in}}%
\pgfpathlineto{\pgfqpoint{5.859512in}{1.941648in}}%
\pgfpathlineto{\pgfqpoint{5.866293in}{1.945432in}}%
\pgfpathlineto{\pgfqpoint{5.870274in}{1.943881in}}%
\pgfpathlineto{\pgfqpoint{5.880446in}{1.938320in}}%
\pgfpathlineto{\pgfqpoint{5.884426in}{1.941079in}}%
\pgfpathlineto{\pgfqpoint{5.892976in}{1.947631in}}%
\pgfpathlineto{\pgfqpoint{5.896809in}{1.946202in}}%
\pgfpathlineto{\pgfqpoint{5.906244in}{1.941452in}}%
\pgfpathlineto{\pgfqpoint{5.910519in}{1.944267in}}%
\pgfpathlineto{\pgfqpoint{5.918185in}{1.949439in}}%
\pgfpathlineto{\pgfqpoint{5.922165in}{1.947919in}}%
\pgfpathlineto{\pgfqpoint{5.933074in}{1.941868in}}%
\pgfpathlineto{\pgfqpoint{5.938823in}{1.944134in}}%
\pgfpathlineto{\pgfqpoint{5.945310in}{1.945748in}}%
\pgfpathlineto{\pgfqpoint{5.951501in}{1.943420in}}%
\pgfpathlineto{\pgfqpoint{5.957988in}{1.942066in}}%
\pgfpathlineto{\pgfqpoint{5.964474in}{1.944511in}}%
\pgfpathlineto{\pgfqpoint{5.973909in}{1.947476in}}%
\pgfpathlineto{\pgfqpoint{5.980985in}{1.946308in}}%
\pgfpathlineto{\pgfqpoint{5.992779in}{1.943900in}}%
\pgfpathlineto{\pgfqpoint{6.009584in}{1.945735in}}%
\pgfpathlineto{\pgfqpoint{6.017545in}{1.943240in}}%
\pgfpathlineto{\pgfqpoint{6.022852in}{1.945437in}}%
\pgfpathlineto{\pgfqpoint{6.031402in}{1.949142in}}%
\pgfpathlineto{\pgfqpoint{6.035677in}{1.946831in}}%
\pgfpathlineto{\pgfqpoint{6.043638in}{1.942086in}}%
\pgfpathlineto{\pgfqpoint{6.047471in}{1.944021in}}%
\pgfpathlineto{\pgfqpoint{6.057200in}{1.950446in}}%
\pgfpathlineto{\pgfqpoint{6.061033in}{1.948147in}}%
\pgfpathlineto{\pgfqpoint{6.069436in}{1.942386in}}%
\pgfpathlineto{\pgfqpoint{6.073416in}{1.944281in}}%
\pgfpathlineto{\pgfqpoint{6.082704in}{1.949817in}}%
\pgfpathlineto{\pgfqpoint{6.086832in}{1.947889in}}%
\pgfpathlineto{\pgfqpoint{6.096561in}{1.942244in}}%
\pgfpathlineto{\pgfqpoint{6.101573in}{1.944579in}}%
\pgfpathlineto{\pgfqpoint{6.104669in}{1.946092in}}%
\pgfpathlineto{\pgfqpoint{6.104669in}{1.946092in}}%
\pgfusepath{stroke}%
\end{pgfscope}%
\begin{pgfscope}%
\pgfpathrectangle{\pgfqpoint{3.745974in}{0.526079in}}{\pgfqpoint{2.358696in}{1.661000in}} %
\pgfusepath{clip}%
\pgfsetbuttcap%
\pgfsetroundjoin%
\pgfsetlinewidth{1.003750pt}%
\definecolor{currentstroke}{rgb}{0.000000,0.000000,0.000000}%
\pgfsetstrokecolor{currentstroke}%
\pgfsetdash{{3.700000pt}{1.600000pt}}{0.000000pt}%
\pgfpathmoveto{\pgfqpoint{3.745974in}{0.858279in}}%
\pgfpathlineto{\pgfqpoint{3.760269in}{0.874619in}}%
\pgfpathlineto{\pgfqpoint{3.774564in}{0.890958in}}%
\pgfpathlineto{\pgfqpoint{3.788859in}{0.907297in}}%
\pgfpathlineto{\pgfqpoint{3.803154in}{0.923637in}}%
\pgfpathlineto{\pgfqpoint{3.817449in}{0.939976in}}%
\pgfpathlineto{\pgfqpoint{3.831744in}{0.956316in}}%
\pgfpathlineto{\pgfqpoint{3.846039in}{0.972655in}}%
\pgfpathlineto{\pgfqpoint{3.860335in}{0.988994in}}%
\pgfpathlineto{\pgfqpoint{3.874630in}{1.005334in}}%
\pgfpathlineto{\pgfqpoint{3.888925in}{1.021673in}}%
\pgfpathlineto{\pgfqpoint{3.903220in}{1.038012in}}%
\pgfpathlineto{\pgfqpoint{3.917515in}{1.054352in}}%
\pgfpathlineto{\pgfqpoint{3.931810in}{1.070691in}}%
\pgfpathlineto{\pgfqpoint{3.946105in}{1.087030in}}%
\pgfpathlineto{\pgfqpoint{3.960400in}{1.103370in}}%
\pgfpathlineto{\pgfqpoint{3.974696in}{1.119709in}}%
\pgfpathlineto{\pgfqpoint{3.988991in}{1.136048in}}%
\pgfpathlineto{\pgfqpoint{4.003286in}{1.152388in}}%
\pgfpathlineto{\pgfqpoint{4.017581in}{1.168727in}}%
\pgfpathlineto{\pgfqpoint{4.031876in}{1.185066in}}%
\pgfpathlineto{\pgfqpoint{4.046171in}{1.201406in}}%
\pgfpathlineto{\pgfqpoint{4.060466in}{1.217745in}}%
\pgfpathlineto{\pgfqpoint{4.074761in}{1.234084in}}%
\pgfpathlineto{\pgfqpoint{4.089057in}{1.250424in}}%
\pgfpathlineto{\pgfqpoint{4.103352in}{1.266763in}}%
\pgfpathlineto{\pgfqpoint{4.117647in}{1.283103in}}%
\pgfpathlineto{\pgfqpoint{4.131942in}{1.299442in}}%
\pgfpathlineto{\pgfqpoint{4.146237in}{1.315781in}}%
\pgfpathlineto{\pgfqpoint{4.160532in}{1.332121in}}%
\pgfpathlineto{\pgfqpoint{4.174827in}{1.348460in}}%
\pgfpathlineto{\pgfqpoint{4.189122in}{1.364799in}}%
\pgfpathlineto{\pgfqpoint{4.203418in}{1.381139in}}%
\pgfpathlineto{\pgfqpoint{4.217713in}{1.397478in}}%
\pgfpathlineto{\pgfqpoint{4.232008in}{1.413817in}}%
\pgfpathlineto{\pgfqpoint{4.246303in}{1.430157in}}%
\pgfpathlineto{\pgfqpoint{4.260598in}{1.446496in}}%
\pgfpathlineto{\pgfqpoint{4.274893in}{1.462835in}}%
\pgfpathlineto{\pgfqpoint{4.289188in}{1.479175in}}%
\pgfpathlineto{\pgfqpoint{4.303483in}{1.495514in}}%
\pgfpathlineto{\pgfqpoint{4.317779in}{1.511853in}}%
\pgfpathlineto{\pgfqpoint{4.332074in}{1.528193in}}%
\pgfpathlineto{\pgfqpoint{4.346369in}{1.544532in}}%
\pgfpathlineto{\pgfqpoint{4.360664in}{1.560871in}}%
\pgfpathlineto{\pgfqpoint{4.374959in}{1.577211in}}%
\pgfpathlineto{\pgfqpoint{4.389254in}{1.593550in}}%
\pgfpathlineto{\pgfqpoint{4.403549in}{1.609890in}}%
\pgfpathlineto{\pgfqpoint{4.417844in}{1.626229in}}%
\pgfpathlineto{\pgfqpoint{4.432140in}{1.642568in}}%
\pgfpathlineto{\pgfqpoint{4.446435in}{1.658908in}}%
\pgfpathlineto{\pgfqpoint{4.460730in}{1.675247in}}%
\pgfpathlineto{\pgfqpoint{4.475025in}{1.691586in}}%
\pgfpathlineto{\pgfqpoint{4.489320in}{1.707926in}}%
\pgfpathlineto{\pgfqpoint{4.503615in}{1.724265in}}%
\pgfpathlineto{\pgfqpoint{4.517910in}{1.740604in}}%
\pgfpathlineto{\pgfqpoint{4.532205in}{1.756944in}}%
\pgfpathlineto{\pgfqpoint{4.546501in}{1.773283in}}%
\pgfpathlineto{\pgfqpoint{4.560796in}{1.789622in}}%
\pgfpathlineto{\pgfqpoint{4.575091in}{1.805962in}}%
\pgfpathlineto{\pgfqpoint{4.589386in}{1.822301in}}%
\pgfpathlineto{\pgfqpoint{4.603681in}{1.838640in}}%
\pgfpathlineto{\pgfqpoint{4.617976in}{1.854980in}}%
\pgfpathlineto{\pgfqpoint{4.632271in}{1.871319in}}%
\pgfpathlineto{\pgfqpoint{4.646566in}{1.887658in}}%
\pgfpathlineto{\pgfqpoint{4.660862in}{1.903998in}}%
\pgfpathlineto{\pgfqpoint{4.675157in}{1.920337in}}%
\pgfpathlineto{\pgfqpoint{4.689452in}{1.936677in}}%
\pgfpathlineto{\pgfqpoint{4.703747in}{1.953016in}}%
\pgfpathlineto{\pgfqpoint{4.718042in}{1.969355in}}%
\pgfpathlineto{\pgfqpoint{4.732337in}{1.985695in}}%
\pgfpathlineto{\pgfqpoint{4.746632in}{2.002034in}}%
\pgfpathlineto{\pgfqpoint{4.760927in}{2.018373in}}%
\pgfpathlineto{\pgfqpoint{4.775223in}{2.034713in}}%
\pgfpathlineto{\pgfqpoint{4.789518in}{2.051052in}}%
\pgfpathlineto{\pgfqpoint{4.803813in}{2.067391in}}%
\pgfpathlineto{\pgfqpoint{4.818108in}{2.083731in}}%
\pgfpathlineto{\pgfqpoint{4.832403in}{2.100070in}}%
\pgfpathlineto{\pgfqpoint{4.846698in}{2.116409in}}%
\pgfpathlineto{\pgfqpoint{4.860993in}{2.132749in}}%
\pgfpathlineto{\pgfqpoint{4.875288in}{2.149088in}}%
\pgfpathlineto{\pgfqpoint{4.889584in}{2.165427in}}%
\pgfpathlineto{\pgfqpoint{4.903879in}{2.181767in}}%
\pgfpathlineto{\pgfqpoint{4.918174in}{2.198106in}}%
\pgfpathlineto{\pgfqpoint{4.920678in}{2.200968in}}%
\pgfusepath{stroke}%
\end{pgfscope}%
\begin{pgfscope}%
\pgfsetrectcap%
\pgfsetmiterjoin%
\pgfsetlinewidth{0.803000pt}%
\definecolor{currentstroke}{rgb}{0.000000,0.000000,0.000000}%
\pgfsetstrokecolor{currentstroke}%
\pgfsetdash{}{0pt}%
\pgfpathmoveto{\pgfqpoint{3.745974in}{0.526079in}}%
\pgfpathlineto{\pgfqpoint{3.745974in}{2.187079in}}%
\pgfusepath{stroke}%
\end{pgfscope}%
\begin{pgfscope}%
\pgfsetrectcap%
\pgfsetmiterjoin%
\pgfsetlinewidth{0.803000pt}%
\definecolor{currentstroke}{rgb}{0.000000,0.000000,0.000000}%
\pgfsetstrokecolor{currentstroke}%
\pgfsetdash{}{0pt}%
\pgfpathmoveto{\pgfqpoint{6.104669in}{0.526079in}}%
\pgfpathlineto{\pgfqpoint{6.104669in}{2.187079in}}%
\pgfusepath{stroke}%
\end{pgfscope}%
\begin{pgfscope}%
\pgfsetrectcap%
\pgfsetmiterjoin%
\pgfsetlinewidth{0.803000pt}%
\definecolor{currentstroke}{rgb}{0.000000,0.000000,0.000000}%
\pgfsetstrokecolor{currentstroke}%
\pgfsetdash{}{0pt}%
\pgfpathmoveto{\pgfqpoint{3.745974in}{0.526079in}}%
\pgfpathlineto{\pgfqpoint{6.104669in}{0.526079in}}%
\pgfusepath{stroke}%
\end{pgfscope}%
\begin{pgfscope}%
\pgfsetrectcap%
\pgfsetmiterjoin%
\pgfsetlinewidth{0.803000pt}%
\definecolor{currentstroke}{rgb}{0.000000,0.000000,0.000000}%
\pgfsetstrokecolor{currentstroke}%
\pgfsetdash{}{0pt}%
\pgfpathmoveto{\pgfqpoint{3.745974in}{2.187079in}}%
\pgfpathlineto{\pgfqpoint{6.104669in}{2.187079in}}%
\pgfusepath{stroke}%
\end{pgfscope}%
\begin{pgfscope}%
\pgftext[x=3.804941in,y=2.013379in,left,base]{\rmfamily\fontsize{10.000000}{12.000000}\selectfont (b)}%
\end{pgfscope}%
\begin{pgfscope}%
\pgftext[x=4.925321in,y=2.270413in,,base]{\rmfamily\fontsize{12.000000}{14.400000}\selectfont Geometric (Strang)}%
\end{pgfscope}%
\begin{pgfscope}%
\pgfsetbuttcap%
\pgfsetmiterjoin%
\definecolor{currentfill}{rgb}{1.000000,1.000000,1.000000}%
\pgfsetfillcolor{currentfill}%
\pgfsetfillopacity{0.800000}%
\pgfsetlinewidth{1.003750pt}%
\definecolor{currentstroke}{rgb}{0.800000,0.800000,0.800000}%
\pgfsetstrokecolor{currentstroke}%
\pgfsetstrokeopacity{0.800000}%
\pgfsetdash{}{0pt}%
\pgfpathmoveto{\pgfqpoint{4.224371in}{0.595524in}}%
\pgfpathlineto{\pgfqpoint{6.007447in}{0.595524in}}%
\pgfpathquadraticcurveto{\pgfqpoint{6.035225in}{0.595524in}}{\pgfqpoint{6.035225in}{0.623302in}}%
\pgfpathlineto{\pgfqpoint{6.035225in}{1.019094in}}%
\pgfpathquadraticcurveto{\pgfqpoint{6.035225in}{1.046872in}}{\pgfqpoint{6.007447in}{1.046872in}}%
\pgfpathlineto{\pgfqpoint{4.224371in}{1.046872in}}%
\pgfpathquadraticcurveto{\pgfqpoint{4.196593in}{1.046872in}}{\pgfqpoint{4.196593in}{1.019094in}}%
\pgfpathlineto{\pgfqpoint{4.196593in}{0.623302in}}%
\pgfpathquadraticcurveto{\pgfqpoint{4.196593in}{0.595524in}}{\pgfqpoint{4.224371in}{0.595524in}}%
\pgfpathclose%
\pgfusepath{stroke,fill}%
\end{pgfscope}%
\begin{pgfscope}%
\pgfsetrectcap%
\pgfsetroundjoin%
\pgfsetlinewidth{1.003750pt}%
\definecolor{currentstroke}{rgb}{1.000000,0.549020,0.000000}%
\pgfsetstrokecolor{currentstroke}%
\pgfsetdash{}{0pt}%
\pgfpathmoveto{\pgfqpoint{4.252148in}{0.934404in}}%
\pgfpathlineto{\pgfqpoint{4.529926in}{0.934404in}}%
\pgfusepath{stroke}%
\end{pgfscope}%
\begin{pgfscope}%
\pgftext[x=4.641037in,y=0.885793in,left,base]{\rmfamily\fontsize{10.000000}{12.000000}\selectfont \(\displaystyle \mathcal{E}_{B}\)}%
\end{pgfscope}%
\begin{pgfscope}%
\pgfsetrectcap%
\pgfsetroundjoin%
\pgfsetlinewidth{1.003750pt}%
\definecolor{currentstroke}{rgb}{0.501961,0.000000,0.501961}%
\pgfsetstrokecolor{currentstroke}%
\pgfsetdash{}{0pt}%
\pgfpathmoveto{\pgfqpoint{4.252148in}{0.730547in}}%
\pgfpathlineto{\pgfqpoint{4.529926in}{0.730547in}}%
\pgfusepath{stroke}%
\end{pgfscope}%
\begin{pgfscope}%
\pgftext[x=4.641037in,y=0.681936in,left,base]{\rmfamily\fontsize{10.000000}{12.000000}\selectfont \(\displaystyle \mathcal{E}_{E}\)}%
\end{pgfscope}%
\begin{pgfscope}%
\pgfsetbuttcap%
\pgfsetroundjoin%
\pgfsetlinewidth{1.003750pt}%
\definecolor{currentstroke}{rgb}{0.627451,0.321569,0.176471}%
\pgfsetstrokecolor{currentstroke}%
\pgfsetdash{{3.700000pt}{1.600000pt}}{0.000000pt}%
\pgfpathmoveto{\pgfqpoint{5.087715in}{0.934404in}}%
\pgfpathlineto{\pgfqpoint{5.365493in}{0.934404in}}%
\pgfusepath{stroke}%
\end{pgfscope}%
\begin{pgfscope}%
\pgftext[x=5.476604in,y=0.885793in,left,base]{\rmfamily\fontsize{10.000000}{12.000000}\selectfont \(\displaystyle \mathcal{E}_\mathrm{c}\)}%
\end{pgfscope}%
\begin{pgfscope}%
\pgfsetbuttcap%
\pgfsetroundjoin%
\pgfsetlinewidth{1.003750pt}%
\definecolor{currentstroke}{rgb}{0.000000,0.000000,0.000000}%
\pgfsetstrokecolor{currentstroke}%
\pgfsetdash{{3.700000pt}{1.600000pt}}{0.000000pt}%
\pgfpathmoveto{\pgfqpoint{5.087715in}{0.730547in}}%
\pgfpathlineto{\pgfqpoint{5.365493in}{0.730547in}}%
\pgfusepath{stroke}%
\end{pgfscope}%
\begin{pgfscope}%
\pgftext[x=5.476604in,y=0.681936in,left,base]{\rmfamily\fontsize{10.000000}{12.000000}\selectfont growth}%
\end{pgfscope}%
\end{pgfpicture}%
\makeatother%
\endgroup%

\caption{Run 1 with parameters listed in Tab. \ref{tab_parameters}: (a) Time evolution of the magnetic field energy $\mathcal{E}_B$, electric field energy $\mathcal{E}_E$ and cold plasma energy  $\mathcal{E}_\mr{c}$ obtained with standard finite element PIC methods from section \ref{sec_standard} together with the expected growth rate from the analytical dispersion relation (\ref{eq_dispersion_relation}). (b) Same as (a) for structure-preserving finite element PIC methods from section \ref{sec_geometric} with the Strang splitting scheme (\ref{eq_Strang}).\label{fig_energies}}
\end{figure}

\begin{wraptable}{r}{7.4cm}
%\vspace{-0.35cm}
\caption{\label{tab_parameters}Parameters for Run 1. In case of the structure-preserving code, the polynomial degree refers to the Lagrange polynomials that span the space $V_0$.}
\centering
\begin{tabular}{|l|l|}
\hline
\textbf{Parameter} & \textbf{Value} \\
\hline
Parallel thermal velocity $v_{\mr{th}\parallel}$ & $0.2c$ \\
\hline
Perpendicular thermal velocity $v_{\mr{th}\perp}$ & $0.53c$ \\
\hline
Density ratio $\nu_\mr{h}=n_{\mr{h}0}/n_{\mr{c}0}$ & $0.06$ \\
\hline
Cold plasma frequency $\Omega_\mr{pe}$ & $2|\Omega_\mr{ce}|$ \\
\hline 
Wavenumber of perturbation $k$ & $2|\Omega_\mr{ce}|/c$ \\
\hline
Amplitude of perturbation $a$ & $10^{-4}B_0$ \\
\hline
Length of computational domain $L$ & $2\pi/k$ \\
\hline
Number of elements $N_\mr{el}$ & 32 \\
\hline
Polynomial degree $p$ & 1 \\
\hline
Number of particles $N_\mr{p}$ & $10^5$ \\
\hline
Time step & $0.0125|\Omega_\mr{ce}|$ \\
\hline
\end{tabular}
\end{wraptable} 
With the choice of parameters in Tab. \ref{tab_parameters}, the numerical solution of the dispersion relation (\ref{eq_dispersion_relation}) yields an expected growth rate of $\gamma\approx0.0447|\Omega_\mr{ce}|$. In Fig. \ref{fig_energies}, we plot the resulting time evolution of the magnetic field energy $\mathcal{E}_B$, the electric field energy $\mathcal{E}_E$ and the cold plasma energy $\mathcal{E}_\mr{c}$ (see (\ref{eq_discrete_Hamiltonian})) normalized to the total energy $\mathcal{E}=\mathcal{E}_B+\mathcal{E}_E+\mathcal{E}_\mr{c}+\mathcal{E}_\mr{h}$ together with the expected growth rate (which is $2\gamma$ in the case of energies). Note that most of the energy is carried by the energetic electrons which is why $\mathcal{E}_\mr{h}$ would be orders of magnitude above the other curves in Fig. \ref{fig_energies}. Therefore, we do not show its evolution. Qualitatively, we observe a similar behavior for the two codes: First, as expected, all quantities grow exponentially, i.e. energy is transfered from the fast electrons to the electromagnetic field and the cold plasma. After this, the wave fields saturate, when nonlinear terms start to play a role and the linear theory thus breaks down. In both cases, the numerical growth matches the analytical one very well and the curves end up at the same saturation level. However, the standard PIC code seems to be more sensitive to the noise induced by the random particle initialization, since it takes some time in the beginning until the linear growth phase is reached (obvious for the electric field energy). 


\begin{figure}[!t]
\centering
\includegraphics[scale=0.95]{01_Figures/distribution_functions_1e5.pdf}
%%% Creator: Matplotlib, PGF backend
%%
%% To include the figure in your LaTeX document, write
%%   \input{<filename>.pgf}
%%
%% Make sure the required packages are loaded in your preamble
%%   \usepackage{pgf}
%%
%% Figures using additional raster images can only be included by \input if
%% they are in the same directory as the main LaTeX file. For loading figures
%% from other directories you can use the `import` package
%%   \usepackage{import}
%% and then include the figures with
%%   \import{<path to file>}{<filename>.pgf}
%%
%% Matplotlib used the following preamble
%%   \usepackage{fontspec}
%%   \setmainfont{DejaVu Serif}
%%   \setsansfont{DejaVu Sans}
%%   \setmonofont{DejaVu Sans Mono}
%%
\begingroup%
\makeatletter%
\begin{pgfpicture}%
\pgfpathrectangle{\pgfpointorigin}{\pgfqpoint{6.461229in}{3.870701in}}%
\pgfusepath{use as bounding box, clip}%
\begin{pgfscope}%
\pgfsetbuttcap%
\pgfsetmiterjoin%
\definecolor{currentfill}{rgb}{1.000000,1.000000,1.000000}%
\pgfsetfillcolor{currentfill}%
\pgfsetlinewidth{0.000000pt}%
\definecolor{currentstroke}{rgb}{1.000000,1.000000,1.000000}%
\pgfsetstrokecolor{currentstroke}%
\pgfsetdash{}{0pt}%
\pgfpathmoveto{\pgfqpoint{-0.000000in}{0.000000in}}%
\pgfpathlineto{\pgfqpoint{6.461229in}{0.000000in}}%
\pgfpathlineto{\pgfqpoint{6.461229in}{3.870701in}}%
\pgfpathlineto{\pgfqpoint{-0.000000in}{3.870701in}}%
\pgfpathclose%
\pgfusepath{fill}%
\end{pgfscope}%
\begin{pgfscope}%
\pgfsetbuttcap%
\pgfsetmiterjoin%
\definecolor{currentfill}{rgb}{1.000000,1.000000,1.000000}%
\pgfsetfillcolor{currentfill}%
\pgfsetlinewidth{0.000000pt}%
\definecolor{currentstroke}{rgb}{0.000000,0.000000,0.000000}%
\pgfsetstrokecolor{currentstroke}%
\pgfsetstrokeopacity{0.000000}%
\pgfsetdash{}{0pt}%
\pgfpathmoveto{\pgfqpoint{0.847495in}{2.188013in}}%
\pgfpathlineto{\pgfqpoint{3.313404in}{2.188013in}}%
\pgfpathlineto{\pgfqpoint{3.313404in}{3.560740in}}%
\pgfpathlineto{\pgfqpoint{0.847495in}{3.560740in}}%
\pgfpathclose%
\pgfusepath{fill}%
\end{pgfscope}%
\begin{pgfscope}%
\pgfsetbuttcap%
\pgfsetroundjoin%
\definecolor{currentfill}{rgb}{0.000000,0.000000,0.000000}%
\pgfsetfillcolor{currentfill}%
\pgfsetlinewidth{0.803000pt}%
\definecolor{currentstroke}{rgb}{0.000000,0.000000,0.000000}%
\pgfsetstrokecolor{currentstroke}%
\pgfsetdash{}{0pt}%
\pgfsys@defobject{currentmarker}{\pgfqpoint{0.000000in}{-0.048611in}}{\pgfqpoint{0.000000in}{0.000000in}}{%
\pgfpathmoveto{\pgfqpoint{0.000000in}{0.000000in}}%
\pgfpathlineto{\pgfqpoint{0.000000in}{-0.048611in}}%
\pgfusepath{stroke,fill}%
}%
\begin{pgfscope}%
\pgfsys@transformshift{0.847495in}{2.188013in}%
\pgfsys@useobject{currentmarker}{}%
\end{pgfscope}%
\end{pgfscope}%
\begin{pgfscope}%
\pgfsetbuttcap%
\pgfsetroundjoin%
\definecolor{currentfill}{rgb}{0.000000,0.000000,0.000000}%
\pgfsetfillcolor{currentfill}%
\pgfsetlinewidth{0.803000pt}%
\definecolor{currentstroke}{rgb}{0.000000,0.000000,0.000000}%
\pgfsetstrokecolor{currentstroke}%
\pgfsetdash{}{0pt}%
\pgfsys@defobject{currentmarker}{\pgfqpoint{0.000000in}{-0.048611in}}{\pgfqpoint{0.000000in}{0.000000in}}{%
\pgfpathmoveto{\pgfqpoint{0.000000in}{0.000000in}}%
\pgfpathlineto{\pgfqpoint{0.000000in}{-0.048611in}}%
\pgfusepath{stroke,fill}%
}%
\begin{pgfscope}%
\pgfsys@transformshift{1.463972in}{2.188013in}%
\pgfsys@useobject{currentmarker}{}%
\end{pgfscope}%
\end{pgfscope}%
\begin{pgfscope}%
\pgfsetbuttcap%
\pgfsetroundjoin%
\definecolor{currentfill}{rgb}{0.000000,0.000000,0.000000}%
\pgfsetfillcolor{currentfill}%
\pgfsetlinewidth{0.803000pt}%
\definecolor{currentstroke}{rgb}{0.000000,0.000000,0.000000}%
\pgfsetstrokecolor{currentstroke}%
\pgfsetdash{}{0pt}%
\pgfsys@defobject{currentmarker}{\pgfqpoint{0.000000in}{-0.048611in}}{\pgfqpoint{0.000000in}{0.000000in}}{%
\pgfpathmoveto{\pgfqpoint{0.000000in}{0.000000in}}%
\pgfpathlineto{\pgfqpoint{0.000000in}{-0.048611in}}%
\pgfusepath{stroke,fill}%
}%
\begin{pgfscope}%
\pgfsys@transformshift{2.080449in}{2.188013in}%
\pgfsys@useobject{currentmarker}{}%
\end{pgfscope}%
\end{pgfscope}%
\begin{pgfscope}%
\pgfsetbuttcap%
\pgfsetroundjoin%
\definecolor{currentfill}{rgb}{0.000000,0.000000,0.000000}%
\pgfsetfillcolor{currentfill}%
\pgfsetlinewidth{0.803000pt}%
\definecolor{currentstroke}{rgb}{0.000000,0.000000,0.000000}%
\pgfsetstrokecolor{currentstroke}%
\pgfsetdash{}{0pt}%
\pgfsys@defobject{currentmarker}{\pgfqpoint{0.000000in}{-0.048611in}}{\pgfqpoint{0.000000in}{0.000000in}}{%
\pgfpathmoveto{\pgfqpoint{0.000000in}{0.000000in}}%
\pgfpathlineto{\pgfqpoint{0.000000in}{-0.048611in}}%
\pgfusepath{stroke,fill}%
}%
\begin{pgfscope}%
\pgfsys@transformshift{2.696926in}{2.188013in}%
\pgfsys@useobject{currentmarker}{}%
\end{pgfscope}%
\end{pgfscope}%
\begin{pgfscope}%
\pgfsetbuttcap%
\pgfsetroundjoin%
\definecolor{currentfill}{rgb}{0.000000,0.000000,0.000000}%
\pgfsetfillcolor{currentfill}%
\pgfsetlinewidth{0.803000pt}%
\definecolor{currentstroke}{rgb}{0.000000,0.000000,0.000000}%
\pgfsetstrokecolor{currentstroke}%
\pgfsetdash{}{0pt}%
\pgfsys@defobject{currentmarker}{\pgfqpoint{0.000000in}{-0.048611in}}{\pgfqpoint{0.000000in}{0.000000in}}{%
\pgfpathmoveto{\pgfqpoint{0.000000in}{0.000000in}}%
\pgfpathlineto{\pgfqpoint{0.000000in}{-0.048611in}}%
\pgfusepath{stroke,fill}%
}%
\begin{pgfscope}%
\pgfsys@transformshift{3.313404in}{2.188013in}%
\pgfsys@useobject{currentmarker}{}%
\end{pgfscope}%
\end{pgfscope}%
\begin{pgfscope}%
\pgfsetbuttcap%
\pgfsetroundjoin%
\definecolor{currentfill}{rgb}{0.000000,0.000000,0.000000}%
\pgfsetfillcolor{currentfill}%
\pgfsetlinewidth{0.803000pt}%
\definecolor{currentstroke}{rgb}{0.000000,0.000000,0.000000}%
\pgfsetstrokecolor{currentstroke}%
\pgfsetdash{}{0pt}%
\pgfsys@defobject{currentmarker}{\pgfqpoint{-0.048611in}{0.000000in}}{\pgfqpoint{0.000000in}{0.000000in}}{%
\pgfpathmoveto{\pgfqpoint{0.000000in}{0.000000in}}%
\pgfpathlineto{\pgfqpoint{-0.048611in}{0.000000in}}%
\pgfusepath{stroke,fill}%
}%
\begin{pgfscope}%
\pgfsys@transformshift{0.847495in}{2.188013in}%
\pgfsys@useobject{currentmarker}{}%
\end{pgfscope}%
\end{pgfscope}%
\begin{pgfscope}%
\pgftext[x=0.441027in,y=2.135251in,left,base]{\rmfamily\fontsize{10.000000}{12.000000}\selectfont 0.00}%
\end{pgfscope}%
\begin{pgfscope}%
\pgfsetbuttcap%
\pgfsetroundjoin%
\definecolor{currentfill}{rgb}{0.000000,0.000000,0.000000}%
\pgfsetfillcolor{currentfill}%
\pgfsetlinewidth{0.803000pt}%
\definecolor{currentstroke}{rgb}{0.000000,0.000000,0.000000}%
\pgfsetstrokecolor{currentstroke}%
\pgfsetdash{}{0pt}%
\pgfsys@defobject{currentmarker}{\pgfqpoint{-0.048611in}{0.000000in}}{\pgfqpoint{0.000000in}{0.000000in}}{%
\pgfpathmoveto{\pgfqpoint{0.000000in}{0.000000in}}%
\pgfpathlineto{\pgfqpoint{-0.048611in}{0.000000in}}%
\pgfusepath{stroke,fill}%
}%
\begin{pgfscope}%
\pgfsys@transformshift{0.847495in}{2.645589in}%
\pgfsys@useobject{currentmarker}{}%
\end{pgfscope}%
\end{pgfscope}%
\begin{pgfscope}%
\pgftext[x=0.441027in,y=2.592827in,left,base]{\rmfamily\fontsize{10.000000}{12.000000}\selectfont 0.20}%
\end{pgfscope}%
\begin{pgfscope}%
\pgfsetbuttcap%
\pgfsetroundjoin%
\definecolor{currentfill}{rgb}{0.000000,0.000000,0.000000}%
\pgfsetfillcolor{currentfill}%
\pgfsetlinewidth{0.803000pt}%
\definecolor{currentstroke}{rgb}{0.000000,0.000000,0.000000}%
\pgfsetstrokecolor{currentstroke}%
\pgfsetdash{}{0pt}%
\pgfsys@defobject{currentmarker}{\pgfqpoint{-0.048611in}{0.000000in}}{\pgfqpoint{0.000000in}{0.000000in}}{%
\pgfpathmoveto{\pgfqpoint{0.000000in}{0.000000in}}%
\pgfpathlineto{\pgfqpoint{-0.048611in}{0.000000in}}%
\pgfusepath{stroke,fill}%
}%
\begin{pgfscope}%
\pgfsys@transformshift{0.847495in}{3.103164in}%
\pgfsys@useobject{currentmarker}{}%
\end{pgfscope}%
\end{pgfscope}%
\begin{pgfscope}%
\pgftext[x=0.441027in,y=3.050403in,left,base]{\rmfamily\fontsize{10.000000}{12.000000}\selectfont 0.40}%
\end{pgfscope}%
\begin{pgfscope}%
\pgfsetbuttcap%
\pgfsetroundjoin%
\definecolor{currentfill}{rgb}{0.000000,0.000000,0.000000}%
\pgfsetfillcolor{currentfill}%
\pgfsetlinewidth{0.803000pt}%
\definecolor{currentstroke}{rgb}{0.000000,0.000000,0.000000}%
\pgfsetstrokecolor{currentstroke}%
\pgfsetdash{}{0pt}%
\pgfsys@defobject{currentmarker}{\pgfqpoint{-0.048611in}{0.000000in}}{\pgfqpoint{0.000000in}{0.000000in}}{%
\pgfpathmoveto{\pgfqpoint{0.000000in}{0.000000in}}%
\pgfpathlineto{\pgfqpoint{-0.048611in}{0.000000in}}%
\pgfusepath{stroke,fill}%
}%
\begin{pgfscope}%
\pgfsys@transformshift{0.847495in}{3.560740in}%
\pgfsys@useobject{currentmarker}{}%
\end{pgfscope}%
\end{pgfscope}%
\begin{pgfscope}%
\pgftext[x=0.441027in,y=3.507979in,left,base]{\rmfamily\fontsize{10.000000}{12.000000}\selectfont 0.60}%
\end{pgfscope}%
\begin{pgfscope}%
\pgftext[x=0.385472in,y=2.874377in,,bottom,rotate=90.000000]{\rmfamily\fontsize{10.000000}{12.000000}\selectfont \(\displaystyle f_{\mathrm{h}\parallel}c^2/|\Omega_\mathrm{ce}|\)}%
\end{pgfscope}%
\begin{pgfscope}%
\pgfpathrectangle{\pgfqpoint{0.847495in}{2.188013in}}{\pgfqpoint{2.465909in}{1.372727in}} %
\pgfusepath{clip}%
\pgfsetrectcap%
\pgfsetroundjoin%
\pgfsetlinewidth{1.003750pt}%
\definecolor{currentstroke}{rgb}{1.000000,0.549020,0.000000}%
\pgfsetstrokecolor{currentstroke}%
\pgfsetdash{}{0pt}%
\pgfpathmoveto{\pgfqpoint{0.833606in}{2.188016in}}%
\pgfpathlineto{\pgfqpoint{1.184631in}{2.189505in}}%
\pgfpathlineto{\pgfqpoint{1.242425in}{2.191414in}}%
\pgfpathlineto{\pgfqpoint{1.300220in}{2.195351in}}%
\pgfpathlineto{\pgfqpoint{1.358015in}{2.203001in}}%
\pgfpathlineto{\pgfqpoint{1.415810in}{2.216987in}}%
\pgfpathlineto{\pgfqpoint{1.473604in}{2.241032in}}%
\pgfpathlineto{\pgfqpoint{1.531399in}{2.279846in}}%
\pgfpathlineto{\pgfqpoint{1.589194in}{2.338572in}}%
\pgfpathlineto{\pgfqpoint{1.646989in}{2.421660in}}%
\pgfpathlineto{\pgfqpoint{1.704783in}{2.531219in}}%
\pgfpathlineto{\pgfqpoint{1.762578in}{2.665206in}}%
\pgfpathlineto{\pgfqpoint{1.878168in}{2.970367in}}%
\pgfpathlineto{\pgfqpoint{1.935962in}{3.110526in}}%
\pgfpathlineto{\pgfqpoint{1.993757in}{3.217654in}}%
\pgfpathlineto{\pgfqpoint{2.051552in}{3.275796in}}%
\pgfpathlineto{\pgfqpoint{2.109347in}{3.275796in}}%
\pgfpathlineto{\pgfqpoint{2.167141in}{3.217654in}}%
\pgfpathlineto{\pgfqpoint{2.224936in}{3.110526in}}%
\pgfpathlineto{\pgfqpoint{2.282731in}{2.970367in}}%
\pgfpathlineto{\pgfqpoint{2.398320in}{2.665206in}}%
\pgfpathlineto{\pgfqpoint{2.456115in}{2.531219in}}%
\pgfpathlineto{\pgfqpoint{2.513910in}{2.421660in}}%
\pgfpathlineto{\pgfqpoint{2.571704in}{2.338572in}}%
\pgfpathlineto{\pgfqpoint{2.629499in}{2.279846in}}%
\pgfpathlineto{\pgfqpoint{2.687294in}{2.241032in}}%
\pgfpathlineto{\pgfqpoint{2.745089in}{2.216987in}}%
\pgfpathlineto{\pgfqpoint{2.802883in}{2.203001in}}%
\pgfpathlineto{\pgfqpoint{2.860678in}{2.195351in}}%
\pgfpathlineto{\pgfqpoint{2.918473in}{2.191414in}}%
\pgfpathlineto{\pgfqpoint{3.034062in}{2.188632in}}%
\pgfpathlineto{\pgfqpoint{3.327293in}{2.188016in}}%
\pgfpathlineto{\pgfqpoint{3.327293in}{2.188016in}}%
\pgfusepath{stroke}%
\end{pgfscope}%
\begin{pgfscope}%
\pgfpathrectangle{\pgfqpoint{0.847495in}{2.188013in}}{\pgfqpoint{2.465909in}{1.372727in}} %
\pgfusepath{clip}%
\pgfsetrectcap%
\pgfsetroundjoin%
\pgfsetlinewidth{1.003750pt}%
\definecolor{currentstroke}{rgb}{0.501961,0.000000,0.501961}%
\pgfsetstrokecolor{currentstroke}%
\pgfsetdash{}{0pt}%
\pgfpathmoveto{\pgfqpoint{0.833606in}{2.188013in}}%
\pgfpathlineto{\pgfqpoint{1.126836in}{2.188481in}}%
\pgfpathlineto{\pgfqpoint{1.184631in}{2.189653in}}%
\pgfpathlineto{\pgfqpoint{1.300220in}{2.196798in}}%
\pgfpathlineto{\pgfqpoint{1.358015in}{2.206287in}}%
\pgfpathlineto{\pgfqpoint{1.415810in}{2.220109in}}%
\pgfpathlineto{\pgfqpoint{1.473604in}{2.246583in}}%
\pgfpathlineto{\pgfqpoint{1.531399in}{2.290861in}}%
\pgfpathlineto{\pgfqpoint{1.589194in}{2.374499in}}%
\pgfpathlineto{\pgfqpoint{1.646989in}{2.476410in}}%
\pgfpathlineto{\pgfqpoint{1.704783in}{2.584413in}}%
\pgfpathlineto{\pgfqpoint{1.762578in}{2.659265in}}%
\pgfpathlineto{\pgfqpoint{1.820373in}{2.746768in}}%
\pgfpathlineto{\pgfqpoint{1.935962in}{3.080849in}}%
\pgfpathlineto{\pgfqpoint{1.993757in}{3.195646in}}%
\pgfpathlineto{\pgfqpoint{2.051552in}{3.287132in}}%
\pgfpathlineto{\pgfqpoint{2.109347in}{3.274481in}}%
\pgfpathlineto{\pgfqpoint{2.167141in}{3.209586in}}%
\pgfpathlineto{\pgfqpoint{2.224936in}{3.078975in}}%
\pgfpathlineto{\pgfqpoint{2.282731in}{2.905609in}}%
\pgfpathlineto{\pgfqpoint{2.340525in}{2.739974in}}%
\pgfpathlineto{\pgfqpoint{2.456115in}{2.589684in}}%
\pgfpathlineto{\pgfqpoint{2.513910in}{2.495387in}}%
\pgfpathlineto{\pgfqpoint{2.571704in}{2.374147in}}%
\pgfpathlineto{\pgfqpoint{2.629499in}{2.296367in}}%
\pgfpathlineto{\pgfqpoint{2.687294in}{2.241897in}}%
\pgfpathlineto{\pgfqpoint{2.745089in}{2.219758in}}%
\pgfpathlineto{\pgfqpoint{2.802883in}{2.200547in}}%
\pgfpathlineto{\pgfqpoint{2.918473in}{2.190941in}}%
\pgfpathlineto{\pgfqpoint{3.265241in}{2.188013in}}%
\pgfpathlineto{\pgfqpoint{3.327293in}{2.188013in}}%
\pgfpathlineto{\pgfqpoint{3.327293in}{2.188013in}}%
\pgfusepath{stroke}%
\end{pgfscope}%
\begin{pgfscope}%
\pgfpathrectangle{\pgfqpoint{0.847495in}{2.188013in}}{\pgfqpoint{2.465909in}{1.372727in}} %
\pgfusepath{clip}%
\pgfsetbuttcap%
\pgfsetroundjoin%
\pgfsetlinewidth{0.501875pt}%
\definecolor{currentstroke}{rgb}{0.000000,0.000000,0.000000}%
\pgfsetstrokecolor{currentstroke}%
\pgfsetdash{{1.850000pt}{0.800000pt}}{0.000000pt}%
\pgfpathmoveto{\pgfqpoint{2.404569in}{2.188013in}}%
\pgfpathlineto{\pgfqpoint{2.404569in}{2.239314in}}%
\pgfpathlineto{\pgfqpoint{2.404569in}{2.290615in}}%
\pgfpathlineto{\pgfqpoint{2.404569in}{2.341916in}}%
\pgfpathlineto{\pgfqpoint{2.404569in}{2.393217in}}%
\pgfpathlineto{\pgfqpoint{2.404569in}{2.444518in}}%
\pgfpathlineto{\pgfqpoint{2.404569in}{2.495819in}}%
\pgfpathlineto{\pgfqpoint{2.404569in}{2.547121in}}%
\pgfpathlineto{\pgfqpoint{2.404569in}{2.598422in}}%
\pgfpathlineto{\pgfqpoint{2.404569in}{2.649723in}}%
\pgfusepath{stroke}%
\end{pgfscope}%
\begin{pgfscope}%
\pgfpathrectangle{\pgfqpoint{0.847495in}{2.188013in}}{\pgfqpoint{2.465909in}{1.372727in}} %
\pgfusepath{clip}%
\pgfsetbuttcap%
\pgfsetroundjoin%
\pgfsetlinewidth{0.501875pt}%
\definecolor{currentstroke}{rgb}{0.000000,0.000000,0.000000}%
\pgfsetstrokecolor{currentstroke}%
\pgfsetdash{{1.850000pt}{0.800000pt}}{0.000000pt}%
\pgfpathmoveto{\pgfqpoint{1.756329in}{2.188013in}}%
\pgfpathlineto{\pgfqpoint{1.756329in}{2.239314in}}%
\pgfpathlineto{\pgfqpoint{1.756329in}{2.290615in}}%
\pgfpathlineto{\pgfqpoint{1.756329in}{2.341916in}}%
\pgfpathlineto{\pgfqpoint{1.756329in}{2.393217in}}%
\pgfpathlineto{\pgfqpoint{1.756329in}{2.444518in}}%
\pgfpathlineto{\pgfqpoint{1.756329in}{2.495819in}}%
\pgfpathlineto{\pgfqpoint{1.756329in}{2.547121in}}%
\pgfpathlineto{\pgfqpoint{1.756329in}{2.598422in}}%
\pgfpathlineto{\pgfqpoint{1.756329in}{2.649723in}}%
\pgfusepath{stroke}%
\end{pgfscope}%
\begin{pgfscope}%
\pgfsetrectcap%
\pgfsetmiterjoin%
\pgfsetlinewidth{0.803000pt}%
\definecolor{currentstroke}{rgb}{0.000000,0.000000,0.000000}%
\pgfsetstrokecolor{currentstroke}%
\pgfsetdash{}{0pt}%
\pgfpathmoveto{\pgfqpoint{0.847495in}{2.188013in}}%
\pgfpathlineto{\pgfqpoint{0.847495in}{3.560740in}}%
\pgfusepath{stroke}%
\end{pgfscope}%
\begin{pgfscope}%
\pgfsetrectcap%
\pgfsetmiterjoin%
\pgfsetlinewidth{0.803000pt}%
\definecolor{currentstroke}{rgb}{0.000000,0.000000,0.000000}%
\pgfsetstrokecolor{currentstroke}%
\pgfsetdash{}{0pt}%
\pgfpathmoveto{\pgfqpoint{3.313404in}{2.188013in}}%
\pgfpathlineto{\pgfqpoint{3.313404in}{3.560740in}}%
\pgfusepath{stroke}%
\end{pgfscope}%
\begin{pgfscope}%
\pgfsetrectcap%
\pgfsetmiterjoin%
\pgfsetlinewidth{0.803000pt}%
\definecolor{currentstroke}{rgb}{0.000000,0.000000,0.000000}%
\pgfsetstrokecolor{currentstroke}%
\pgfsetdash{}{0pt}%
\pgfpathmoveto{\pgfqpoint{0.847495in}{2.188013in}}%
\pgfpathlineto{\pgfqpoint{3.313404in}{2.188013in}}%
\pgfusepath{stroke}%
\end{pgfscope}%
\begin{pgfscope}%
\pgfsetrectcap%
\pgfsetmiterjoin%
\pgfsetlinewidth{0.803000pt}%
\definecolor{currentstroke}{rgb}{0.000000,0.000000,0.000000}%
\pgfsetstrokecolor{currentstroke}%
\pgfsetdash{}{0pt}%
\pgfpathmoveto{\pgfqpoint{0.847495in}{3.560740in}}%
\pgfpathlineto{\pgfqpoint{3.313404in}{3.560740in}}%
\pgfusepath{stroke}%
\end{pgfscope}%
\begin{pgfscope}%
\pgftext[x=0.909142in,y=3.377710in,left,base]{\rmfamily\fontsize{10.000000}{12.000000}\selectfont (a)}%
\end{pgfscope}%
\begin{pgfscope}%
\pgftext[x=2.080449in,y=3.644073in,,base]{\rmfamily\fontsize{12.000000}{14.400000}\selectfont Standard}%
\end{pgfscope}%
\begin{pgfscope}%
\pgfsetbuttcap%
\pgfsetmiterjoin%
\definecolor{currentfill}{rgb}{1.000000,1.000000,1.000000}%
\pgfsetfillcolor{currentfill}%
\pgfsetfillopacity{0.800000}%
\pgfsetlinewidth{1.003750pt}%
\definecolor{currentstroke}{rgb}{0.800000,0.800000,0.800000}%
\pgfsetstrokecolor{currentstroke}%
\pgfsetstrokeopacity{0.800000}%
\pgfsetdash{}{0pt}%
\pgfpathmoveto{\pgfqpoint{2.624268in}{2.838057in}}%
\pgfpathlineto{\pgfqpoint{3.216181in}{2.838057in}}%
\pgfpathquadraticcurveto{\pgfqpoint{3.243959in}{2.838057in}}{\pgfqpoint{3.243959in}{2.865835in}}%
\pgfpathlineto{\pgfqpoint{3.243959in}{3.463518in}}%
\pgfpathquadraticcurveto{\pgfqpoint{3.243959in}{3.491296in}}{\pgfqpoint{3.216181in}{3.491296in}}%
\pgfpathlineto{\pgfqpoint{2.624268in}{3.491296in}}%
\pgfpathquadraticcurveto{\pgfqpoint{2.596491in}{3.491296in}}{\pgfqpoint{2.596491in}{3.463518in}}%
\pgfpathlineto{\pgfqpoint{2.596491in}{2.865835in}}%
\pgfpathquadraticcurveto{\pgfqpoint{2.596491in}{2.838057in}}{\pgfqpoint{2.624268in}{2.838057in}}%
\pgfpathclose%
\pgfusepath{stroke,fill}%
\end{pgfscope}%
\begin{pgfscope}%
\pgfsetrectcap%
\pgfsetroundjoin%
\pgfsetlinewidth{1.003750pt}%
\definecolor{currentstroke}{rgb}{1.000000,0.549020,0.000000}%
\pgfsetstrokecolor{currentstroke}%
\pgfsetdash{}{0pt}%
\pgfpathmoveto{\pgfqpoint{2.652046in}{3.378828in}}%
\pgfpathlineto{\pgfqpoint{2.929824in}{3.378828in}}%
\pgfusepath{stroke}%
\end{pgfscope}%
\begin{pgfscope}%
\pgftext[x=3.040935in,y=3.330217in,left,base]{\rmfamily\fontsize{10.000000}{12.000000}\selectfont \(\displaystyle t_0\)}%
\end{pgfscope}%
\begin{pgfscope}%
\pgfsetrectcap%
\pgfsetroundjoin%
\pgfsetlinewidth{1.003750pt}%
\definecolor{currentstroke}{rgb}{0.501961,0.000000,0.501961}%
\pgfsetstrokecolor{currentstroke}%
\pgfsetdash{}{0pt}%
\pgfpathmoveto{\pgfqpoint{2.652046in}{3.174971in}}%
\pgfpathlineto{\pgfqpoint{2.929824in}{3.174971in}}%
\pgfusepath{stroke}%
\end{pgfscope}%
\begin{pgfscope}%
\pgftext[x=3.040935in,y=3.126360in,left,base]{\rmfamily\fontsize{10.000000}{12.000000}\selectfont \(\displaystyle t_\mathrm{f}\)}%
\end{pgfscope}%
\begin{pgfscope}%
\pgfsetbuttcap%
\pgfsetroundjoin%
\pgfsetlinewidth{0.501875pt}%
\definecolor{currentstroke}{rgb}{0.000000,0.000000,0.000000}%
\pgfsetstrokecolor{currentstroke}%
\pgfsetdash{{1.850000pt}{0.800000pt}}{0.000000pt}%
\pgfpathmoveto{\pgfqpoint{2.652046in}{2.971114in}}%
\pgfpathlineto{\pgfqpoint{2.929824in}{2.971114in}}%
\pgfusepath{stroke}%
\end{pgfscope}%
\begin{pgfscope}%
\pgftext[x=3.040935in,y=2.922503in,left,base]{\rmfamily\fontsize{10.000000}{12.000000}\selectfont \(\displaystyle v_\mathrm{R}\)}%
\end{pgfscope}%
\begin{pgfscope}%
\pgfsetbuttcap%
\pgfsetmiterjoin%
\definecolor{currentfill}{rgb}{1.000000,1.000000,1.000000}%
\pgfsetfillcolor{currentfill}%
\pgfsetlinewidth{0.000000pt}%
\definecolor{currentstroke}{rgb}{0.000000,0.000000,0.000000}%
\pgfsetstrokecolor{currentstroke}%
\pgfsetstrokeopacity{0.000000}%
\pgfsetdash{}{0pt}%
\pgfpathmoveto{\pgfqpoint{3.806585in}{2.188013in}}%
\pgfpathlineto{\pgfqpoint{6.272495in}{2.188013in}}%
\pgfpathlineto{\pgfqpoint{6.272495in}{3.560740in}}%
\pgfpathlineto{\pgfqpoint{3.806585in}{3.560740in}}%
\pgfpathclose%
\pgfusepath{fill}%
\end{pgfscope}%
\begin{pgfscope}%
\pgfsetbuttcap%
\pgfsetroundjoin%
\definecolor{currentfill}{rgb}{0.000000,0.000000,0.000000}%
\pgfsetfillcolor{currentfill}%
\pgfsetlinewidth{0.803000pt}%
\definecolor{currentstroke}{rgb}{0.000000,0.000000,0.000000}%
\pgfsetstrokecolor{currentstroke}%
\pgfsetdash{}{0pt}%
\pgfsys@defobject{currentmarker}{\pgfqpoint{0.000000in}{-0.048611in}}{\pgfqpoint{0.000000in}{0.000000in}}{%
\pgfpathmoveto{\pgfqpoint{0.000000in}{0.000000in}}%
\pgfpathlineto{\pgfqpoint{0.000000in}{-0.048611in}}%
\pgfusepath{stroke,fill}%
}%
\begin{pgfscope}%
\pgfsys@transformshift{3.806585in}{2.188013in}%
\pgfsys@useobject{currentmarker}{}%
\end{pgfscope}%
\end{pgfscope}%
\begin{pgfscope}%
\pgfsetbuttcap%
\pgfsetroundjoin%
\definecolor{currentfill}{rgb}{0.000000,0.000000,0.000000}%
\pgfsetfillcolor{currentfill}%
\pgfsetlinewidth{0.803000pt}%
\definecolor{currentstroke}{rgb}{0.000000,0.000000,0.000000}%
\pgfsetstrokecolor{currentstroke}%
\pgfsetdash{}{0pt}%
\pgfsys@defobject{currentmarker}{\pgfqpoint{0.000000in}{-0.048611in}}{\pgfqpoint{0.000000in}{0.000000in}}{%
\pgfpathmoveto{\pgfqpoint{0.000000in}{0.000000in}}%
\pgfpathlineto{\pgfqpoint{0.000000in}{-0.048611in}}%
\pgfusepath{stroke,fill}%
}%
\begin{pgfscope}%
\pgfsys@transformshift{4.423063in}{2.188013in}%
\pgfsys@useobject{currentmarker}{}%
\end{pgfscope}%
\end{pgfscope}%
\begin{pgfscope}%
\pgfsetbuttcap%
\pgfsetroundjoin%
\definecolor{currentfill}{rgb}{0.000000,0.000000,0.000000}%
\pgfsetfillcolor{currentfill}%
\pgfsetlinewidth{0.803000pt}%
\definecolor{currentstroke}{rgb}{0.000000,0.000000,0.000000}%
\pgfsetstrokecolor{currentstroke}%
\pgfsetdash{}{0pt}%
\pgfsys@defobject{currentmarker}{\pgfqpoint{0.000000in}{-0.048611in}}{\pgfqpoint{0.000000in}{0.000000in}}{%
\pgfpathmoveto{\pgfqpoint{0.000000in}{0.000000in}}%
\pgfpathlineto{\pgfqpoint{0.000000in}{-0.048611in}}%
\pgfusepath{stroke,fill}%
}%
\begin{pgfscope}%
\pgfsys@transformshift{5.039540in}{2.188013in}%
\pgfsys@useobject{currentmarker}{}%
\end{pgfscope}%
\end{pgfscope}%
\begin{pgfscope}%
\pgfsetbuttcap%
\pgfsetroundjoin%
\definecolor{currentfill}{rgb}{0.000000,0.000000,0.000000}%
\pgfsetfillcolor{currentfill}%
\pgfsetlinewidth{0.803000pt}%
\definecolor{currentstroke}{rgb}{0.000000,0.000000,0.000000}%
\pgfsetstrokecolor{currentstroke}%
\pgfsetdash{}{0pt}%
\pgfsys@defobject{currentmarker}{\pgfqpoint{0.000000in}{-0.048611in}}{\pgfqpoint{0.000000in}{0.000000in}}{%
\pgfpathmoveto{\pgfqpoint{0.000000in}{0.000000in}}%
\pgfpathlineto{\pgfqpoint{0.000000in}{-0.048611in}}%
\pgfusepath{stroke,fill}%
}%
\begin{pgfscope}%
\pgfsys@transformshift{5.656017in}{2.188013in}%
\pgfsys@useobject{currentmarker}{}%
\end{pgfscope}%
\end{pgfscope}%
\begin{pgfscope}%
\pgfsetbuttcap%
\pgfsetroundjoin%
\definecolor{currentfill}{rgb}{0.000000,0.000000,0.000000}%
\pgfsetfillcolor{currentfill}%
\pgfsetlinewidth{0.803000pt}%
\definecolor{currentstroke}{rgb}{0.000000,0.000000,0.000000}%
\pgfsetstrokecolor{currentstroke}%
\pgfsetdash{}{0pt}%
\pgfsys@defobject{currentmarker}{\pgfqpoint{0.000000in}{-0.048611in}}{\pgfqpoint{0.000000in}{0.000000in}}{%
\pgfpathmoveto{\pgfqpoint{0.000000in}{0.000000in}}%
\pgfpathlineto{\pgfqpoint{0.000000in}{-0.048611in}}%
\pgfusepath{stroke,fill}%
}%
\begin{pgfscope}%
\pgfsys@transformshift{6.272495in}{2.188013in}%
\pgfsys@useobject{currentmarker}{}%
\end{pgfscope}%
\end{pgfscope}%
\begin{pgfscope}%
\pgfsetbuttcap%
\pgfsetroundjoin%
\definecolor{currentfill}{rgb}{0.000000,0.000000,0.000000}%
\pgfsetfillcolor{currentfill}%
\pgfsetlinewidth{0.803000pt}%
\definecolor{currentstroke}{rgb}{0.000000,0.000000,0.000000}%
\pgfsetstrokecolor{currentstroke}%
\pgfsetdash{}{0pt}%
\pgfsys@defobject{currentmarker}{\pgfqpoint{-0.048611in}{0.000000in}}{\pgfqpoint{0.000000in}{0.000000in}}{%
\pgfpathmoveto{\pgfqpoint{0.000000in}{0.000000in}}%
\pgfpathlineto{\pgfqpoint{-0.048611in}{0.000000in}}%
\pgfusepath{stroke,fill}%
}%
\begin{pgfscope}%
\pgfsys@transformshift{3.806585in}{2.188013in}%
\pgfsys@useobject{currentmarker}{}%
\end{pgfscope}%
\end{pgfscope}%
\begin{pgfscope}%
\pgfsetbuttcap%
\pgfsetroundjoin%
\definecolor{currentfill}{rgb}{0.000000,0.000000,0.000000}%
\pgfsetfillcolor{currentfill}%
\pgfsetlinewidth{0.803000pt}%
\definecolor{currentstroke}{rgb}{0.000000,0.000000,0.000000}%
\pgfsetstrokecolor{currentstroke}%
\pgfsetdash{}{0pt}%
\pgfsys@defobject{currentmarker}{\pgfqpoint{-0.048611in}{0.000000in}}{\pgfqpoint{0.000000in}{0.000000in}}{%
\pgfpathmoveto{\pgfqpoint{0.000000in}{0.000000in}}%
\pgfpathlineto{\pgfqpoint{-0.048611in}{0.000000in}}%
\pgfusepath{stroke,fill}%
}%
\begin{pgfscope}%
\pgfsys@transformshift{3.806585in}{2.645589in}%
\pgfsys@useobject{currentmarker}{}%
\end{pgfscope}%
\end{pgfscope}%
\begin{pgfscope}%
\pgfsetbuttcap%
\pgfsetroundjoin%
\definecolor{currentfill}{rgb}{0.000000,0.000000,0.000000}%
\pgfsetfillcolor{currentfill}%
\pgfsetlinewidth{0.803000pt}%
\definecolor{currentstroke}{rgb}{0.000000,0.000000,0.000000}%
\pgfsetstrokecolor{currentstroke}%
\pgfsetdash{}{0pt}%
\pgfsys@defobject{currentmarker}{\pgfqpoint{-0.048611in}{0.000000in}}{\pgfqpoint{0.000000in}{0.000000in}}{%
\pgfpathmoveto{\pgfqpoint{0.000000in}{0.000000in}}%
\pgfpathlineto{\pgfqpoint{-0.048611in}{0.000000in}}%
\pgfusepath{stroke,fill}%
}%
\begin{pgfscope}%
\pgfsys@transformshift{3.806585in}{3.103164in}%
\pgfsys@useobject{currentmarker}{}%
\end{pgfscope}%
\end{pgfscope}%
\begin{pgfscope}%
\pgfsetbuttcap%
\pgfsetroundjoin%
\definecolor{currentfill}{rgb}{0.000000,0.000000,0.000000}%
\pgfsetfillcolor{currentfill}%
\pgfsetlinewidth{0.803000pt}%
\definecolor{currentstroke}{rgb}{0.000000,0.000000,0.000000}%
\pgfsetstrokecolor{currentstroke}%
\pgfsetdash{}{0pt}%
\pgfsys@defobject{currentmarker}{\pgfqpoint{-0.048611in}{0.000000in}}{\pgfqpoint{0.000000in}{0.000000in}}{%
\pgfpathmoveto{\pgfqpoint{0.000000in}{0.000000in}}%
\pgfpathlineto{\pgfqpoint{-0.048611in}{0.000000in}}%
\pgfusepath{stroke,fill}%
}%
\begin{pgfscope}%
\pgfsys@transformshift{3.806585in}{3.560740in}%
\pgfsys@useobject{currentmarker}{}%
\end{pgfscope}%
\end{pgfscope}%
\begin{pgfscope}%
\pgfpathrectangle{\pgfqpoint{3.806585in}{2.188013in}}{\pgfqpoint{2.465909in}{1.372727in}} %
\pgfusepath{clip}%
\pgfsetrectcap%
\pgfsetroundjoin%
\pgfsetlinewidth{1.003750pt}%
\definecolor{currentstroke}{rgb}{1.000000,0.549020,0.000000}%
\pgfsetstrokecolor{currentstroke}%
\pgfsetdash{}{0pt}%
\pgfpathmoveto{\pgfqpoint{3.792697in}{2.188016in}}%
\pgfpathlineto{\pgfqpoint{4.143722in}{2.189505in}}%
\pgfpathlineto{\pgfqpoint{4.201516in}{2.191414in}}%
\pgfpathlineto{\pgfqpoint{4.259311in}{2.195351in}}%
\pgfpathlineto{\pgfqpoint{4.317106in}{2.203001in}}%
\pgfpathlineto{\pgfqpoint{4.374900in}{2.216987in}}%
\pgfpathlineto{\pgfqpoint{4.432695in}{2.241032in}}%
\pgfpathlineto{\pgfqpoint{4.490490in}{2.279846in}}%
\pgfpathlineto{\pgfqpoint{4.548285in}{2.338572in}}%
\pgfpathlineto{\pgfqpoint{4.606079in}{2.421660in}}%
\pgfpathlineto{\pgfqpoint{4.663874in}{2.531219in}}%
\pgfpathlineto{\pgfqpoint{4.721669in}{2.665206in}}%
\pgfpathlineto{\pgfqpoint{4.837258in}{2.970367in}}%
\pgfpathlineto{\pgfqpoint{4.895053in}{3.110526in}}%
\pgfpathlineto{\pgfqpoint{4.952848in}{3.217654in}}%
\pgfpathlineto{\pgfqpoint{5.010643in}{3.275796in}}%
\pgfpathlineto{\pgfqpoint{5.068437in}{3.275796in}}%
\pgfpathlineto{\pgfqpoint{5.126232in}{3.217654in}}%
\pgfpathlineto{\pgfqpoint{5.184027in}{3.110526in}}%
\pgfpathlineto{\pgfqpoint{5.241822in}{2.970367in}}%
\pgfpathlineto{\pgfqpoint{5.357411in}{2.665206in}}%
\pgfpathlineto{\pgfqpoint{5.415206in}{2.531219in}}%
\pgfpathlineto{\pgfqpoint{5.473001in}{2.421660in}}%
\pgfpathlineto{\pgfqpoint{5.530795in}{2.338572in}}%
\pgfpathlineto{\pgfqpoint{5.588590in}{2.279846in}}%
\pgfpathlineto{\pgfqpoint{5.646385in}{2.241032in}}%
\pgfpathlineto{\pgfqpoint{5.704180in}{2.216987in}}%
\pgfpathlineto{\pgfqpoint{5.761974in}{2.203001in}}%
\pgfpathlineto{\pgfqpoint{5.819769in}{2.195351in}}%
\pgfpathlineto{\pgfqpoint{5.877564in}{2.191414in}}%
\pgfpathlineto{\pgfqpoint{5.993153in}{2.188632in}}%
\pgfpathlineto{\pgfqpoint{6.286383in}{2.188016in}}%
\pgfpathlineto{\pgfqpoint{6.286383in}{2.188016in}}%
\pgfusepath{stroke}%
\end{pgfscope}%
\begin{pgfscope}%
\pgfpathrectangle{\pgfqpoint{3.806585in}{2.188013in}}{\pgfqpoint{2.465909in}{1.372727in}} %
\pgfusepath{clip}%
\pgfsetrectcap%
\pgfsetroundjoin%
\pgfsetlinewidth{1.003750pt}%
\definecolor{currentstroke}{rgb}{0.501961,0.000000,0.501961}%
\pgfsetstrokecolor{currentstroke}%
\pgfsetdash{}{0pt}%
\pgfpathmoveto{\pgfqpoint{3.792697in}{2.188013in}}%
\pgfpathlineto{\pgfqpoint{4.143722in}{2.189536in}}%
\pgfpathlineto{\pgfqpoint{4.259311in}{2.194573in}}%
\pgfpathlineto{\pgfqpoint{4.317106in}{2.201484in}}%
\pgfpathlineto{\pgfqpoint{4.374900in}{2.221398in}}%
\pgfpathlineto{\pgfqpoint{4.432695in}{2.244357in}}%
\pgfpathlineto{\pgfqpoint{4.490490in}{2.286527in}}%
\pgfpathlineto{\pgfqpoint{4.548285in}{2.368876in}}%
\pgfpathlineto{\pgfqpoint{4.606079in}{2.490115in}}%
\pgfpathlineto{\pgfqpoint{4.663874in}{2.599758in}}%
\pgfpathlineto{\pgfqpoint{4.721669in}{2.660202in}}%
\pgfpathlineto{\pgfqpoint{4.779464in}{2.745479in}}%
\pgfpathlineto{\pgfqpoint{4.837258in}{2.912403in}}%
\pgfpathlineto{\pgfqpoint{4.895053in}{3.086941in}}%
\pgfpathlineto{\pgfqpoint{4.952848in}{3.207126in}}%
\pgfpathlineto{\pgfqpoint{5.010643in}{3.264407in}}%
\pgfpathlineto{\pgfqpoint{5.068437in}{3.279635in}}%
\pgfpathlineto{\pgfqpoint{5.126232in}{3.199394in}}%
\pgfpathlineto{\pgfqpoint{5.184027in}{3.075461in}}%
\pgfpathlineto{\pgfqpoint{5.241822in}{2.899869in}}%
\pgfpathlineto{\pgfqpoint{5.299616in}{2.764222in}}%
\pgfpathlineto{\pgfqpoint{5.357411in}{2.661725in}}%
\pgfpathlineto{\pgfqpoint{5.415206in}{2.582187in}}%
\pgfpathlineto{\pgfqpoint{5.473001in}{2.496207in}}%
\pgfpathlineto{\pgfqpoint{5.530795in}{2.375319in}}%
\pgfpathlineto{\pgfqpoint{5.588590in}{2.290510in}}%
\pgfpathlineto{\pgfqpoint{5.646385in}{2.241546in}}%
\pgfpathlineto{\pgfqpoint{5.704180in}{2.219523in}}%
\pgfpathlineto{\pgfqpoint{5.761974in}{2.201601in}}%
\pgfpathlineto{\pgfqpoint{5.819769in}{2.194573in}}%
\pgfpathlineto{\pgfqpoint{5.935359in}{2.189887in}}%
\pgfpathlineto{\pgfqpoint{6.050948in}{2.188364in}}%
\pgfpathlineto{\pgfqpoint{6.286383in}{2.188013in}}%
\pgfpathlineto{\pgfqpoint{6.286383in}{2.188013in}}%
\pgfusepath{stroke}%
\end{pgfscope}%
\begin{pgfscope}%
\pgfpathrectangle{\pgfqpoint{3.806585in}{2.188013in}}{\pgfqpoint{2.465909in}{1.372727in}} %
\pgfusepath{clip}%
\pgfsetbuttcap%
\pgfsetroundjoin%
\pgfsetlinewidth{0.501875pt}%
\definecolor{currentstroke}{rgb}{0.000000,0.000000,0.000000}%
\pgfsetstrokecolor{currentstroke}%
\pgfsetdash{{1.850000pt}{0.800000pt}}{0.000000pt}%
\pgfpathmoveto{\pgfqpoint{5.363660in}{2.188013in}}%
\pgfpathlineto{\pgfqpoint{5.363660in}{2.239314in}}%
\pgfpathlineto{\pgfqpoint{5.363660in}{2.290615in}}%
\pgfpathlineto{\pgfqpoint{5.363660in}{2.341916in}}%
\pgfpathlineto{\pgfqpoint{5.363660in}{2.393217in}}%
\pgfpathlineto{\pgfqpoint{5.363660in}{2.444518in}}%
\pgfpathlineto{\pgfqpoint{5.363660in}{2.495819in}}%
\pgfpathlineto{\pgfqpoint{5.363660in}{2.547121in}}%
\pgfpathlineto{\pgfqpoint{5.363660in}{2.598422in}}%
\pgfpathlineto{\pgfqpoint{5.363660in}{2.649723in}}%
\pgfusepath{stroke}%
\end{pgfscope}%
\begin{pgfscope}%
\pgfpathrectangle{\pgfqpoint{3.806585in}{2.188013in}}{\pgfqpoint{2.465909in}{1.372727in}} %
\pgfusepath{clip}%
\pgfsetbuttcap%
\pgfsetroundjoin%
\pgfsetlinewidth{0.501875pt}%
\definecolor{currentstroke}{rgb}{0.000000,0.000000,0.000000}%
\pgfsetstrokecolor{currentstroke}%
\pgfsetdash{{1.850000pt}{0.800000pt}}{0.000000pt}%
\pgfpathmoveto{\pgfqpoint{4.715420in}{2.188013in}}%
\pgfpathlineto{\pgfqpoint{4.715420in}{2.239314in}}%
\pgfpathlineto{\pgfqpoint{4.715420in}{2.290615in}}%
\pgfpathlineto{\pgfqpoint{4.715420in}{2.341916in}}%
\pgfpathlineto{\pgfqpoint{4.715420in}{2.393217in}}%
\pgfpathlineto{\pgfqpoint{4.715420in}{2.444518in}}%
\pgfpathlineto{\pgfqpoint{4.715420in}{2.495819in}}%
\pgfpathlineto{\pgfqpoint{4.715420in}{2.547121in}}%
\pgfpathlineto{\pgfqpoint{4.715420in}{2.598422in}}%
\pgfpathlineto{\pgfqpoint{4.715420in}{2.649723in}}%
\pgfusepath{stroke}%
\end{pgfscope}%
\begin{pgfscope}%
\pgfsetrectcap%
\pgfsetmiterjoin%
\pgfsetlinewidth{0.803000pt}%
\definecolor{currentstroke}{rgb}{0.000000,0.000000,0.000000}%
\pgfsetstrokecolor{currentstroke}%
\pgfsetdash{}{0pt}%
\pgfpathmoveto{\pgfqpoint{3.806585in}{2.188013in}}%
\pgfpathlineto{\pgfqpoint{3.806585in}{3.560740in}}%
\pgfusepath{stroke}%
\end{pgfscope}%
\begin{pgfscope}%
\pgfsetrectcap%
\pgfsetmiterjoin%
\pgfsetlinewidth{0.803000pt}%
\definecolor{currentstroke}{rgb}{0.000000,0.000000,0.000000}%
\pgfsetstrokecolor{currentstroke}%
\pgfsetdash{}{0pt}%
\pgfpathmoveto{\pgfqpoint{6.272495in}{2.188013in}}%
\pgfpathlineto{\pgfqpoint{6.272495in}{3.560740in}}%
\pgfusepath{stroke}%
\end{pgfscope}%
\begin{pgfscope}%
\pgfsetrectcap%
\pgfsetmiterjoin%
\pgfsetlinewidth{0.803000pt}%
\definecolor{currentstroke}{rgb}{0.000000,0.000000,0.000000}%
\pgfsetstrokecolor{currentstroke}%
\pgfsetdash{}{0pt}%
\pgfpathmoveto{\pgfqpoint{3.806585in}{2.188013in}}%
\pgfpathlineto{\pgfqpoint{6.272495in}{2.188013in}}%
\pgfusepath{stroke}%
\end{pgfscope}%
\begin{pgfscope}%
\pgfsetrectcap%
\pgfsetmiterjoin%
\pgfsetlinewidth{0.803000pt}%
\definecolor{currentstroke}{rgb}{0.000000,0.000000,0.000000}%
\pgfsetstrokecolor{currentstroke}%
\pgfsetdash{}{0pt}%
\pgfpathmoveto{\pgfqpoint{3.806585in}{3.560740in}}%
\pgfpathlineto{\pgfqpoint{6.272495in}{3.560740in}}%
\pgfusepath{stroke}%
\end{pgfscope}%
\begin{pgfscope}%
\pgftext[x=3.868233in,y=3.377710in,left,base]{\rmfamily\fontsize{10.000000}{12.000000}\selectfont (b)}%
\end{pgfscope}%
\begin{pgfscope}%
\pgftext[x=5.039540in,y=3.644073in,,base]{\rmfamily\fontsize{12.000000}{14.400000}\selectfont Geometric (Strang)}%
\end{pgfscope}%
\begin{pgfscope}%
\pgfsetbuttcap%
\pgfsetmiterjoin%
\definecolor{currentfill}{rgb}{1.000000,1.000000,1.000000}%
\pgfsetfillcolor{currentfill}%
\pgfsetfillopacity{0.800000}%
\pgfsetlinewidth{1.003750pt}%
\definecolor{currentstroke}{rgb}{0.800000,0.800000,0.800000}%
\pgfsetstrokecolor{currentstroke}%
\pgfsetstrokeopacity{0.800000}%
\pgfsetdash{}{0pt}%
\pgfpathmoveto{\pgfqpoint{5.583359in}{2.838057in}}%
\pgfpathlineto{\pgfqpoint{6.175272in}{2.838057in}}%
\pgfpathquadraticcurveto{\pgfqpoint{6.203050in}{2.838057in}}{\pgfqpoint{6.203050in}{2.865835in}}%
\pgfpathlineto{\pgfqpoint{6.203050in}{3.463518in}}%
\pgfpathquadraticcurveto{\pgfqpoint{6.203050in}{3.491296in}}{\pgfqpoint{6.175272in}{3.491296in}}%
\pgfpathlineto{\pgfqpoint{5.583359in}{3.491296in}}%
\pgfpathquadraticcurveto{\pgfqpoint{5.555582in}{3.491296in}}{\pgfqpoint{5.555582in}{3.463518in}}%
\pgfpathlineto{\pgfqpoint{5.555582in}{2.865835in}}%
\pgfpathquadraticcurveto{\pgfqpoint{5.555582in}{2.838057in}}{\pgfqpoint{5.583359in}{2.838057in}}%
\pgfpathclose%
\pgfusepath{stroke,fill}%
\end{pgfscope}%
\begin{pgfscope}%
\pgfsetrectcap%
\pgfsetroundjoin%
\pgfsetlinewidth{1.003750pt}%
\definecolor{currentstroke}{rgb}{1.000000,0.549020,0.000000}%
\pgfsetstrokecolor{currentstroke}%
\pgfsetdash{}{0pt}%
\pgfpathmoveto{\pgfqpoint{5.611137in}{3.378828in}}%
\pgfpathlineto{\pgfqpoint{5.888915in}{3.378828in}}%
\pgfusepath{stroke}%
\end{pgfscope}%
\begin{pgfscope}%
\pgftext[x=6.000026in,y=3.330217in,left,base]{\rmfamily\fontsize{10.000000}{12.000000}\selectfont \(\displaystyle t_0\)}%
\end{pgfscope}%
\begin{pgfscope}%
\pgfsetrectcap%
\pgfsetroundjoin%
\pgfsetlinewidth{1.003750pt}%
\definecolor{currentstroke}{rgb}{0.501961,0.000000,0.501961}%
\pgfsetstrokecolor{currentstroke}%
\pgfsetdash{}{0pt}%
\pgfpathmoveto{\pgfqpoint{5.611137in}{3.174971in}}%
\pgfpathlineto{\pgfqpoint{5.888915in}{3.174971in}}%
\pgfusepath{stroke}%
\end{pgfscope}%
\begin{pgfscope}%
\pgftext[x=6.000026in,y=3.126360in,left,base]{\rmfamily\fontsize{10.000000}{12.000000}\selectfont \(\displaystyle t_\mathrm{f}\)}%
\end{pgfscope}%
\begin{pgfscope}%
\pgfsetbuttcap%
\pgfsetroundjoin%
\pgfsetlinewidth{0.501875pt}%
\definecolor{currentstroke}{rgb}{0.000000,0.000000,0.000000}%
\pgfsetstrokecolor{currentstroke}%
\pgfsetdash{{1.850000pt}{0.800000pt}}{0.000000pt}%
\pgfpathmoveto{\pgfqpoint{5.611137in}{2.971114in}}%
\pgfpathlineto{\pgfqpoint{5.888915in}{2.971114in}}%
\pgfusepath{stroke}%
\end{pgfscope}%
\begin{pgfscope}%
\pgftext[x=6.000026in,y=2.922503in,left,base]{\rmfamily\fontsize{10.000000}{12.000000}\selectfont \(\displaystyle v_\mathrm{R}\)}%
\end{pgfscope}%
\begin{pgfscope}%
\pgfsetbuttcap%
\pgfsetmiterjoin%
\definecolor{currentfill}{rgb}{1.000000,1.000000,1.000000}%
\pgfsetfillcolor{currentfill}%
\pgfsetlinewidth{0.000000pt}%
\definecolor{currentstroke}{rgb}{0.000000,0.000000,0.000000}%
\pgfsetstrokecolor{currentstroke}%
\pgfsetstrokeopacity{0.000000}%
\pgfsetdash{}{0pt}%
\pgfpathmoveto{\pgfqpoint{0.847495in}{0.540740in}}%
\pgfpathlineto{\pgfqpoint{3.313404in}{0.540740in}}%
\pgfpathlineto{\pgfqpoint{3.313404in}{1.913467in}}%
\pgfpathlineto{\pgfqpoint{0.847495in}{1.913467in}}%
\pgfpathclose%
\pgfusepath{fill}%
\end{pgfscope}%
\begin{pgfscope}%
\pgfsetbuttcap%
\pgfsetroundjoin%
\definecolor{currentfill}{rgb}{0.000000,0.000000,0.000000}%
\pgfsetfillcolor{currentfill}%
\pgfsetlinewidth{0.803000pt}%
\definecolor{currentstroke}{rgb}{0.000000,0.000000,0.000000}%
\pgfsetstrokecolor{currentstroke}%
\pgfsetdash{}{0pt}%
\pgfsys@defobject{currentmarker}{\pgfqpoint{0.000000in}{-0.048611in}}{\pgfqpoint{0.000000in}{0.000000in}}{%
\pgfpathmoveto{\pgfqpoint{0.000000in}{0.000000in}}%
\pgfpathlineto{\pgfqpoint{0.000000in}{-0.048611in}}%
\pgfusepath{stroke,fill}%
}%
\begin{pgfscope}%
\pgfsys@transformshift{0.847495in}{0.540740in}%
\pgfsys@useobject{currentmarker}{}%
\end{pgfscope}%
\end{pgfscope}%
\begin{pgfscope}%
\pgftext[x=0.847495in,y=0.443518in,,top]{\rmfamily\fontsize{10.000000}{12.000000}\selectfont \(\displaystyle -1.0\)}%
\end{pgfscope}%
\begin{pgfscope}%
\pgfsetbuttcap%
\pgfsetroundjoin%
\definecolor{currentfill}{rgb}{0.000000,0.000000,0.000000}%
\pgfsetfillcolor{currentfill}%
\pgfsetlinewidth{0.803000pt}%
\definecolor{currentstroke}{rgb}{0.000000,0.000000,0.000000}%
\pgfsetstrokecolor{currentstroke}%
\pgfsetdash{}{0pt}%
\pgfsys@defobject{currentmarker}{\pgfqpoint{0.000000in}{-0.048611in}}{\pgfqpoint{0.000000in}{0.000000in}}{%
\pgfpathmoveto{\pgfqpoint{0.000000in}{0.000000in}}%
\pgfpathlineto{\pgfqpoint{0.000000in}{-0.048611in}}%
\pgfusepath{stroke,fill}%
}%
\begin{pgfscope}%
\pgfsys@transformshift{1.463972in}{0.540740in}%
\pgfsys@useobject{currentmarker}{}%
\end{pgfscope}%
\end{pgfscope}%
\begin{pgfscope}%
\pgftext[x=1.463972in,y=0.443518in,,top]{\rmfamily\fontsize{10.000000}{12.000000}\selectfont \(\displaystyle -0.5\)}%
\end{pgfscope}%
\begin{pgfscope}%
\pgfsetbuttcap%
\pgfsetroundjoin%
\definecolor{currentfill}{rgb}{0.000000,0.000000,0.000000}%
\pgfsetfillcolor{currentfill}%
\pgfsetlinewidth{0.803000pt}%
\definecolor{currentstroke}{rgb}{0.000000,0.000000,0.000000}%
\pgfsetstrokecolor{currentstroke}%
\pgfsetdash{}{0pt}%
\pgfsys@defobject{currentmarker}{\pgfqpoint{0.000000in}{-0.048611in}}{\pgfqpoint{0.000000in}{0.000000in}}{%
\pgfpathmoveto{\pgfqpoint{0.000000in}{0.000000in}}%
\pgfpathlineto{\pgfqpoint{0.000000in}{-0.048611in}}%
\pgfusepath{stroke,fill}%
}%
\begin{pgfscope}%
\pgfsys@transformshift{2.080449in}{0.540740in}%
\pgfsys@useobject{currentmarker}{}%
\end{pgfscope}%
\end{pgfscope}%
\begin{pgfscope}%
\pgftext[x=2.080449in,y=0.443518in,,top]{\rmfamily\fontsize{10.000000}{12.000000}\selectfont \(\displaystyle 0.0\)}%
\end{pgfscope}%
\begin{pgfscope}%
\pgfsetbuttcap%
\pgfsetroundjoin%
\definecolor{currentfill}{rgb}{0.000000,0.000000,0.000000}%
\pgfsetfillcolor{currentfill}%
\pgfsetlinewidth{0.803000pt}%
\definecolor{currentstroke}{rgb}{0.000000,0.000000,0.000000}%
\pgfsetstrokecolor{currentstroke}%
\pgfsetdash{}{0pt}%
\pgfsys@defobject{currentmarker}{\pgfqpoint{0.000000in}{-0.048611in}}{\pgfqpoint{0.000000in}{0.000000in}}{%
\pgfpathmoveto{\pgfqpoint{0.000000in}{0.000000in}}%
\pgfpathlineto{\pgfqpoint{0.000000in}{-0.048611in}}%
\pgfusepath{stroke,fill}%
}%
\begin{pgfscope}%
\pgfsys@transformshift{2.696926in}{0.540740in}%
\pgfsys@useobject{currentmarker}{}%
\end{pgfscope}%
\end{pgfscope}%
\begin{pgfscope}%
\pgftext[x=2.696926in,y=0.443518in,,top]{\rmfamily\fontsize{10.000000}{12.000000}\selectfont \(\displaystyle 0.5\)}%
\end{pgfscope}%
\begin{pgfscope}%
\pgfsetbuttcap%
\pgfsetroundjoin%
\definecolor{currentfill}{rgb}{0.000000,0.000000,0.000000}%
\pgfsetfillcolor{currentfill}%
\pgfsetlinewidth{0.803000pt}%
\definecolor{currentstroke}{rgb}{0.000000,0.000000,0.000000}%
\pgfsetstrokecolor{currentstroke}%
\pgfsetdash{}{0pt}%
\pgfsys@defobject{currentmarker}{\pgfqpoint{0.000000in}{-0.048611in}}{\pgfqpoint{0.000000in}{0.000000in}}{%
\pgfpathmoveto{\pgfqpoint{0.000000in}{0.000000in}}%
\pgfpathlineto{\pgfqpoint{0.000000in}{-0.048611in}}%
\pgfusepath{stroke,fill}%
}%
\begin{pgfscope}%
\pgfsys@transformshift{3.313404in}{0.540740in}%
\pgfsys@useobject{currentmarker}{}%
\end{pgfscope}%
\end{pgfscope}%
\begin{pgfscope}%
\pgftext[x=3.313404in,y=0.443518in,,top]{\rmfamily\fontsize{10.000000}{12.000000}\selectfont \(\displaystyle 1.0\)}%
\end{pgfscope}%
\begin{pgfscope}%
\pgftext[x=2.080449in,y=0.253550in,,top]{\rmfamily\fontsize{10.000000}{12.000000}\selectfont \(\displaystyle v_\parallel/c\)}%
\end{pgfscope}%
\begin{pgfscope}%
\pgfsetbuttcap%
\pgfsetroundjoin%
\definecolor{currentfill}{rgb}{0.000000,0.000000,0.000000}%
\pgfsetfillcolor{currentfill}%
\pgfsetlinewidth{0.803000pt}%
\definecolor{currentstroke}{rgb}{0.000000,0.000000,0.000000}%
\pgfsetstrokecolor{currentstroke}%
\pgfsetdash{}{0pt}%
\pgfsys@defobject{currentmarker}{\pgfqpoint{-0.048611in}{0.000000in}}{\pgfqpoint{0.000000in}{0.000000in}}{%
\pgfpathmoveto{\pgfqpoint{0.000000in}{0.000000in}}%
\pgfpathlineto{\pgfqpoint{-0.048611in}{0.000000in}}%
\pgfusepath{stroke,fill}%
}%
\begin{pgfscope}%
\pgfsys@transformshift{0.847495in}{0.540740in}%
\pgfsys@useobject{currentmarker}{}%
\end{pgfscope}%
\end{pgfscope}%
\begin{pgfscope}%
\pgftext[x=0.325888in,y=0.487979in,left,base]{\rmfamily\fontsize{10.000000}{12.000000}\selectfont \(\displaystyle -0.050\)}%
\end{pgfscope}%
\begin{pgfscope}%
\pgfsetbuttcap%
\pgfsetroundjoin%
\definecolor{currentfill}{rgb}{0.000000,0.000000,0.000000}%
\pgfsetfillcolor{currentfill}%
\pgfsetlinewidth{0.803000pt}%
\definecolor{currentstroke}{rgb}{0.000000,0.000000,0.000000}%
\pgfsetstrokecolor{currentstroke}%
\pgfsetdash{}{0pt}%
\pgfsys@defobject{currentmarker}{\pgfqpoint{-0.048611in}{0.000000in}}{\pgfqpoint{0.000000in}{0.000000in}}{%
\pgfpathmoveto{\pgfqpoint{0.000000in}{0.000000in}}%
\pgfpathlineto{\pgfqpoint{-0.048611in}{0.000000in}}%
\pgfusepath{stroke,fill}%
}%
\begin{pgfscope}%
\pgfsys@transformshift{0.847495in}{0.883922in}%
\pgfsys@useobject{currentmarker}{}%
\end{pgfscope}%
\end{pgfscope}%
\begin{pgfscope}%
\pgftext[x=0.325888in,y=0.831160in,left,base]{\rmfamily\fontsize{10.000000}{12.000000}\selectfont \(\displaystyle -0.025\)}%
\end{pgfscope}%
\begin{pgfscope}%
\pgfsetbuttcap%
\pgfsetroundjoin%
\definecolor{currentfill}{rgb}{0.000000,0.000000,0.000000}%
\pgfsetfillcolor{currentfill}%
\pgfsetlinewidth{0.803000pt}%
\definecolor{currentstroke}{rgb}{0.000000,0.000000,0.000000}%
\pgfsetstrokecolor{currentstroke}%
\pgfsetdash{}{0pt}%
\pgfsys@defobject{currentmarker}{\pgfqpoint{-0.048611in}{0.000000in}}{\pgfqpoint{0.000000in}{0.000000in}}{%
\pgfpathmoveto{\pgfqpoint{0.000000in}{0.000000in}}%
\pgfpathlineto{\pgfqpoint{-0.048611in}{0.000000in}}%
\pgfusepath{stroke,fill}%
}%
\begin{pgfscope}%
\pgfsys@transformshift{0.847495in}{1.227104in}%
\pgfsys@useobject{currentmarker}{}%
\end{pgfscope}%
\end{pgfscope}%
\begin{pgfscope}%
\pgftext[x=0.433913in,y=1.174342in,left,base]{\rmfamily\fontsize{10.000000}{12.000000}\selectfont \(\displaystyle 0.000\)}%
\end{pgfscope}%
\begin{pgfscope}%
\pgfsetbuttcap%
\pgfsetroundjoin%
\definecolor{currentfill}{rgb}{0.000000,0.000000,0.000000}%
\pgfsetfillcolor{currentfill}%
\pgfsetlinewidth{0.803000pt}%
\definecolor{currentstroke}{rgb}{0.000000,0.000000,0.000000}%
\pgfsetstrokecolor{currentstroke}%
\pgfsetdash{}{0pt}%
\pgfsys@defobject{currentmarker}{\pgfqpoint{-0.048611in}{0.000000in}}{\pgfqpoint{0.000000in}{0.000000in}}{%
\pgfpathmoveto{\pgfqpoint{0.000000in}{0.000000in}}%
\pgfpathlineto{\pgfqpoint{-0.048611in}{0.000000in}}%
\pgfusepath{stroke,fill}%
}%
\begin{pgfscope}%
\pgfsys@transformshift{0.847495in}{1.570286in}%
\pgfsys@useobject{currentmarker}{}%
\end{pgfscope}%
\end{pgfscope}%
\begin{pgfscope}%
\pgftext[x=0.433913in,y=1.517524in,left,base]{\rmfamily\fontsize{10.000000}{12.000000}\selectfont \(\displaystyle 0.025\)}%
\end{pgfscope}%
\begin{pgfscope}%
\pgfsetbuttcap%
\pgfsetroundjoin%
\definecolor{currentfill}{rgb}{0.000000,0.000000,0.000000}%
\pgfsetfillcolor{currentfill}%
\pgfsetlinewidth{0.803000pt}%
\definecolor{currentstroke}{rgb}{0.000000,0.000000,0.000000}%
\pgfsetstrokecolor{currentstroke}%
\pgfsetdash{}{0pt}%
\pgfsys@defobject{currentmarker}{\pgfqpoint{-0.048611in}{0.000000in}}{\pgfqpoint{0.000000in}{0.000000in}}{%
\pgfpathmoveto{\pgfqpoint{0.000000in}{0.000000in}}%
\pgfpathlineto{\pgfqpoint{-0.048611in}{0.000000in}}%
\pgfusepath{stroke,fill}%
}%
\begin{pgfscope}%
\pgfsys@transformshift{0.847495in}{1.913467in}%
\pgfsys@useobject{currentmarker}{}%
\end{pgfscope}%
\end{pgfscope}%
\begin{pgfscope}%
\pgftext[x=0.433913in,y=1.860706in,left,base]{\rmfamily\fontsize{10.000000}{12.000000}\selectfont \(\displaystyle 0.050\)}%
\end{pgfscope}%
\begin{pgfscope}%
\pgftext[x=0.270333in,y=1.227104in,,bottom,rotate=90.000000]{\rmfamily\fontsize{10.000000}{12.000000}\selectfont \(\displaystyle \Delta f_{\mathrm{h}\parallel}c^2/|\Omega_\mathrm{ce}|\)}%
\end{pgfscope}%
\begin{pgfscope}%
\pgfpathrectangle{\pgfqpoint{0.847495in}{0.540740in}}{\pgfqpoint{2.465909in}{1.372727in}} %
\pgfusepath{clip}%
\pgfsetbuttcap%
\pgfsetroundjoin%
\pgfsetlinewidth{0.501875pt}%
\definecolor{currentstroke}{rgb}{0.000000,0.000000,0.000000}%
\pgfsetstrokecolor{currentstroke}%
\pgfsetdash{{1.850000pt}{0.800000pt}}{0.000000pt}%
\pgfpathmoveto{\pgfqpoint{0.847495in}{1.227104in}}%
\pgfpathlineto{\pgfqpoint{1.121484in}{1.227104in}}%
\pgfpathlineto{\pgfqpoint{1.395474in}{1.227104in}}%
\pgfpathlineto{\pgfqpoint{1.669464in}{1.227104in}}%
\pgfpathlineto{\pgfqpoint{1.943454in}{1.227104in}}%
\pgfpathlineto{\pgfqpoint{2.217444in}{1.227104in}}%
\pgfpathlineto{\pgfqpoint{2.491434in}{1.227104in}}%
\pgfpathlineto{\pgfqpoint{2.765424in}{1.227104in}}%
\pgfpathlineto{\pgfqpoint{3.039414in}{1.227104in}}%
\pgfpathlineto{\pgfqpoint{3.313404in}{1.227104in}}%
\pgfusepath{stroke}%
\end{pgfscope}%
\begin{pgfscope}%
\pgfpathrectangle{\pgfqpoint{0.847495in}{0.540740in}}{\pgfqpoint{2.465909in}{1.372727in}} %
\pgfusepath{clip}%
\pgfsetrectcap%
\pgfsetroundjoin%
\pgfsetlinewidth{1.003750pt}%
\definecolor{currentstroke}{rgb}{0.627451,0.321569,0.176471}%
\pgfsetstrokecolor{currentstroke}%
\pgfsetdash{}{0pt}%
\pgfpathmoveto{\pgfqpoint{0.833606in}{1.227085in}}%
\pgfpathlineto{\pgfqpoint{1.184631in}{1.227992in}}%
\pgfpathlineto{\pgfqpoint{1.242425in}{1.233405in}}%
\pgfpathlineto{\pgfqpoint{1.300220in}{1.235786in}}%
\pgfpathlineto{\pgfqpoint{1.358015in}{1.246820in}}%
\pgfpathlineto{\pgfqpoint{1.415810in}{1.245835in}}%
\pgfpathlineto{\pgfqpoint{1.473604in}{1.260406in}}%
\pgfpathlineto{\pgfqpoint{1.531399in}{1.293195in}}%
\pgfpathlineto{\pgfqpoint{1.589194in}{1.442663in}}%
\pgfpathlineto{\pgfqpoint{1.646989in}{1.555605in}}%
\pgfpathlineto{\pgfqpoint{1.704783in}{1.546263in}}%
\pgfpathlineto{\pgfqpoint{1.762578in}{1.191453in}}%
\pgfpathlineto{\pgfqpoint{1.820373in}{0.811481in}}%
\pgfpathlineto{\pgfqpoint{1.878168in}{0.879321in}}%
\pgfpathlineto{\pgfqpoint{1.935962in}{1.049041in}}%
\pgfpathlineto{\pgfqpoint{1.993757in}{1.095057in}}%
\pgfpathlineto{\pgfqpoint{2.051552in}{1.295119in}}%
\pgfpathlineto{\pgfqpoint{2.109347in}{1.219213in}}%
\pgfpathlineto{\pgfqpoint{2.167141in}{1.178695in}}%
\pgfpathlineto{\pgfqpoint{2.224936in}{1.037796in}}%
\pgfpathlineto{\pgfqpoint{2.282731in}{0.838556in}}%
\pgfpathlineto{\pgfqpoint{2.340525in}{0.770717in}}%
\pgfpathlineto{\pgfqpoint{2.398320in}{1.230109in}}%
\pgfpathlineto{\pgfqpoint{2.456115in}{1.577890in}}%
\pgfpathlineto{\pgfqpoint{2.513910in}{1.669465in}}%
\pgfpathlineto{\pgfqpoint{2.571704in}{1.440554in}}%
\pgfpathlineto{\pgfqpoint{2.629499in}{1.326229in}}%
\pgfpathlineto{\pgfqpoint{2.687294in}{1.232292in}}%
\pgfpathlineto{\pgfqpoint{2.745089in}{1.243727in}}%
\pgfpathlineto{\pgfqpoint{2.802883in}{1.212381in}}%
\pgfpathlineto{\pgfqpoint{2.860678in}{1.225244in}}%
\pgfpathlineto{\pgfqpoint{2.918473in}{1.224269in}}%
\pgfpathlineto{\pgfqpoint{2.976268in}{1.231506in}}%
\pgfpathlineto{\pgfqpoint{3.034062in}{1.226198in}}%
\pgfpathlineto{\pgfqpoint{3.091857in}{1.228454in}}%
\pgfpathlineto{\pgfqpoint{3.207447in}{1.226912in}}%
\pgfpathlineto{\pgfqpoint{3.327293in}{1.227085in}}%
\pgfpathlineto{\pgfqpoint{3.327293in}{1.227085in}}%
\pgfusepath{stroke}%
\end{pgfscope}%
\begin{pgfscope}%
\pgfpathrectangle{\pgfqpoint{0.847495in}{0.540740in}}{\pgfqpoint{2.465909in}{1.372727in}} %
\pgfusepath{clip}%
\pgfsetbuttcap%
\pgfsetroundjoin%
\pgfsetlinewidth{0.501875pt}%
\definecolor{currentstroke}{rgb}{0.000000,0.000000,0.000000}%
\pgfsetstrokecolor{currentstroke}%
\pgfsetdash{{1.850000pt}{0.800000pt}}{0.000000pt}%
\pgfpathmoveto{\pgfqpoint{2.404569in}{0.526851in}}%
\pgfpathlineto{\pgfqpoint{2.404569in}{0.617003in}}%
\pgfpathlineto{\pgfqpoint{2.404569in}{0.769528in}}%
\pgfpathlineto{\pgfqpoint{2.404569in}{0.922053in}}%
\pgfpathlineto{\pgfqpoint{2.404569in}{1.074579in}}%
\pgfpathlineto{\pgfqpoint{2.404569in}{1.227104in}}%
\pgfusepath{stroke}%
\end{pgfscope}%
\begin{pgfscope}%
\pgfpathrectangle{\pgfqpoint{0.847495in}{0.540740in}}{\pgfqpoint{2.465909in}{1.372727in}} %
\pgfusepath{clip}%
\pgfsetbuttcap%
\pgfsetroundjoin%
\pgfsetlinewidth{0.501875pt}%
\definecolor{currentstroke}{rgb}{0.000000,0.000000,0.000000}%
\pgfsetstrokecolor{currentstroke}%
\pgfsetdash{{1.850000pt}{0.800000pt}}{0.000000pt}%
\pgfpathmoveto{\pgfqpoint{1.756329in}{0.526851in}}%
\pgfpathlineto{\pgfqpoint{1.756329in}{0.617003in}}%
\pgfpathlineto{\pgfqpoint{1.756329in}{0.769528in}}%
\pgfpathlineto{\pgfqpoint{1.756329in}{0.922053in}}%
\pgfpathlineto{\pgfqpoint{1.756329in}{1.074579in}}%
\pgfpathlineto{\pgfqpoint{1.756329in}{1.227104in}}%
\pgfusepath{stroke}%
\end{pgfscope}%
\begin{pgfscope}%
\pgfsetrectcap%
\pgfsetmiterjoin%
\pgfsetlinewidth{0.803000pt}%
\definecolor{currentstroke}{rgb}{0.000000,0.000000,0.000000}%
\pgfsetstrokecolor{currentstroke}%
\pgfsetdash{}{0pt}%
\pgfpathmoveto{\pgfqpoint{0.847495in}{0.540740in}}%
\pgfpathlineto{\pgfqpoint{0.847495in}{1.913467in}}%
\pgfusepath{stroke}%
\end{pgfscope}%
\begin{pgfscope}%
\pgfsetrectcap%
\pgfsetmiterjoin%
\pgfsetlinewidth{0.803000pt}%
\definecolor{currentstroke}{rgb}{0.000000,0.000000,0.000000}%
\pgfsetstrokecolor{currentstroke}%
\pgfsetdash{}{0pt}%
\pgfpathmoveto{\pgfqpoint{3.313404in}{0.540740in}}%
\pgfpathlineto{\pgfqpoint{3.313404in}{1.913467in}}%
\pgfusepath{stroke}%
\end{pgfscope}%
\begin{pgfscope}%
\pgfsetrectcap%
\pgfsetmiterjoin%
\pgfsetlinewidth{0.803000pt}%
\definecolor{currentstroke}{rgb}{0.000000,0.000000,0.000000}%
\pgfsetstrokecolor{currentstroke}%
\pgfsetdash{}{0pt}%
\pgfpathmoveto{\pgfqpoint{0.847495in}{0.540740in}}%
\pgfpathlineto{\pgfqpoint{3.313404in}{0.540740in}}%
\pgfusepath{stroke}%
\end{pgfscope}%
\begin{pgfscope}%
\pgfsetrectcap%
\pgfsetmiterjoin%
\pgfsetlinewidth{0.803000pt}%
\definecolor{currentstroke}{rgb}{0.000000,0.000000,0.000000}%
\pgfsetstrokecolor{currentstroke}%
\pgfsetdash{}{0pt}%
\pgfpathmoveto{\pgfqpoint{0.847495in}{1.913467in}}%
\pgfpathlineto{\pgfqpoint{3.313404in}{1.913467in}}%
\pgfusepath{stroke}%
\end{pgfscope}%
\begin{pgfscope}%
\pgftext[x=0.909142in,y=1.707558in,left,base]{\rmfamily\fontsize{10.000000}{12.000000}\selectfont (c)}%
\end{pgfscope}%
\begin{pgfscope}%
\pgfsetbuttcap%
\pgfsetmiterjoin%
\definecolor{currentfill}{rgb}{1.000000,1.000000,1.000000}%
\pgfsetfillcolor{currentfill}%
\pgfsetlinewidth{0.000000pt}%
\definecolor{currentstroke}{rgb}{0.000000,0.000000,0.000000}%
\pgfsetstrokecolor{currentstroke}%
\pgfsetstrokeopacity{0.000000}%
\pgfsetdash{}{0pt}%
\pgfpathmoveto{\pgfqpoint{3.806585in}{0.540740in}}%
\pgfpathlineto{\pgfqpoint{6.272495in}{0.540740in}}%
\pgfpathlineto{\pgfqpoint{6.272495in}{1.913467in}}%
\pgfpathlineto{\pgfqpoint{3.806585in}{1.913467in}}%
\pgfpathclose%
\pgfusepath{fill}%
\end{pgfscope}%
\begin{pgfscope}%
\pgfsetbuttcap%
\pgfsetroundjoin%
\definecolor{currentfill}{rgb}{0.000000,0.000000,0.000000}%
\pgfsetfillcolor{currentfill}%
\pgfsetlinewidth{0.803000pt}%
\definecolor{currentstroke}{rgb}{0.000000,0.000000,0.000000}%
\pgfsetstrokecolor{currentstroke}%
\pgfsetdash{}{0pt}%
\pgfsys@defobject{currentmarker}{\pgfqpoint{0.000000in}{-0.048611in}}{\pgfqpoint{0.000000in}{0.000000in}}{%
\pgfpathmoveto{\pgfqpoint{0.000000in}{0.000000in}}%
\pgfpathlineto{\pgfqpoint{0.000000in}{-0.048611in}}%
\pgfusepath{stroke,fill}%
}%
\begin{pgfscope}%
\pgfsys@transformshift{3.806585in}{0.540740in}%
\pgfsys@useobject{currentmarker}{}%
\end{pgfscope}%
\end{pgfscope}%
\begin{pgfscope}%
\pgftext[x=3.806585in,y=0.443518in,,top]{\rmfamily\fontsize{10.000000}{12.000000}\selectfont \(\displaystyle -1.0\)}%
\end{pgfscope}%
\begin{pgfscope}%
\pgfsetbuttcap%
\pgfsetroundjoin%
\definecolor{currentfill}{rgb}{0.000000,0.000000,0.000000}%
\pgfsetfillcolor{currentfill}%
\pgfsetlinewidth{0.803000pt}%
\definecolor{currentstroke}{rgb}{0.000000,0.000000,0.000000}%
\pgfsetstrokecolor{currentstroke}%
\pgfsetdash{}{0pt}%
\pgfsys@defobject{currentmarker}{\pgfqpoint{0.000000in}{-0.048611in}}{\pgfqpoint{0.000000in}{0.000000in}}{%
\pgfpathmoveto{\pgfqpoint{0.000000in}{0.000000in}}%
\pgfpathlineto{\pgfqpoint{0.000000in}{-0.048611in}}%
\pgfusepath{stroke,fill}%
}%
\begin{pgfscope}%
\pgfsys@transformshift{4.423063in}{0.540740in}%
\pgfsys@useobject{currentmarker}{}%
\end{pgfscope}%
\end{pgfscope}%
\begin{pgfscope}%
\pgftext[x=4.423063in,y=0.443518in,,top]{\rmfamily\fontsize{10.000000}{12.000000}\selectfont \(\displaystyle -0.5\)}%
\end{pgfscope}%
\begin{pgfscope}%
\pgfsetbuttcap%
\pgfsetroundjoin%
\definecolor{currentfill}{rgb}{0.000000,0.000000,0.000000}%
\pgfsetfillcolor{currentfill}%
\pgfsetlinewidth{0.803000pt}%
\definecolor{currentstroke}{rgb}{0.000000,0.000000,0.000000}%
\pgfsetstrokecolor{currentstroke}%
\pgfsetdash{}{0pt}%
\pgfsys@defobject{currentmarker}{\pgfqpoint{0.000000in}{-0.048611in}}{\pgfqpoint{0.000000in}{0.000000in}}{%
\pgfpathmoveto{\pgfqpoint{0.000000in}{0.000000in}}%
\pgfpathlineto{\pgfqpoint{0.000000in}{-0.048611in}}%
\pgfusepath{stroke,fill}%
}%
\begin{pgfscope}%
\pgfsys@transformshift{5.039540in}{0.540740in}%
\pgfsys@useobject{currentmarker}{}%
\end{pgfscope}%
\end{pgfscope}%
\begin{pgfscope}%
\pgftext[x=5.039540in,y=0.443518in,,top]{\rmfamily\fontsize{10.000000}{12.000000}\selectfont \(\displaystyle 0.0\)}%
\end{pgfscope}%
\begin{pgfscope}%
\pgfsetbuttcap%
\pgfsetroundjoin%
\definecolor{currentfill}{rgb}{0.000000,0.000000,0.000000}%
\pgfsetfillcolor{currentfill}%
\pgfsetlinewidth{0.803000pt}%
\definecolor{currentstroke}{rgb}{0.000000,0.000000,0.000000}%
\pgfsetstrokecolor{currentstroke}%
\pgfsetdash{}{0pt}%
\pgfsys@defobject{currentmarker}{\pgfqpoint{0.000000in}{-0.048611in}}{\pgfqpoint{0.000000in}{0.000000in}}{%
\pgfpathmoveto{\pgfqpoint{0.000000in}{0.000000in}}%
\pgfpathlineto{\pgfqpoint{0.000000in}{-0.048611in}}%
\pgfusepath{stroke,fill}%
}%
\begin{pgfscope}%
\pgfsys@transformshift{5.656017in}{0.540740in}%
\pgfsys@useobject{currentmarker}{}%
\end{pgfscope}%
\end{pgfscope}%
\begin{pgfscope}%
\pgftext[x=5.656017in,y=0.443518in,,top]{\rmfamily\fontsize{10.000000}{12.000000}\selectfont \(\displaystyle 0.5\)}%
\end{pgfscope}%
\begin{pgfscope}%
\pgfsetbuttcap%
\pgfsetroundjoin%
\definecolor{currentfill}{rgb}{0.000000,0.000000,0.000000}%
\pgfsetfillcolor{currentfill}%
\pgfsetlinewidth{0.803000pt}%
\definecolor{currentstroke}{rgb}{0.000000,0.000000,0.000000}%
\pgfsetstrokecolor{currentstroke}%
\pgfsetdash{}{0pt}%
\pgfsys@defobject{currentmarker}{\pgfqpoint{0.000000in}{-0.048611in}}{\pgfqpoint{0.000000in}{0.000000in}}{%
\pgfpathmoveto{\pgfqpoint{0.000000in}{0.000000in}}%
\pgfpathlineto{\pgfqpoint{0.000000in}{-0.048611in}}%
\pgfusepath{stroke,fill}%
}%
\begin{pgfscope}%
\pgfsys@transformshift{6.272495in}{0.540740in}%
\pgfsys@useobject{currentmarker}{}%
\end{pgfscope}%
\end{pgfscope}%
\begin{pgfscope}%
\pgftext[x=6.272495in,y=0.443518in,,top]{\rmfamily\fontsize{10.000000}{12.000000}\selectfont \(\displaystyle 1.0\)}%
\end{pgfscope}%
\begin{pgfscope}%
\pgftext[x=5.039540in,y=0.253550in,,top]{\rmfamily\fontsize{10.000000}{12.000000}\selectfont \(\displaystyle v_\parallel/c\)}%
\end{pgfscope}%
\begin{pgfscope}%
\pgfsetbuttcap%
\pgfsetroundjoin%
\definecolor{currentfill}{rgb}{0.000000,0.000000,0.000000}%
\pgfsetfillcolor{currentfill}%
\pgfsetlinewidth{0.803000pt}%
\definecolor{currentstroke}{rgb}{0.000000,0.000000,0.000000}%
\pgfsetstrokecolor{currentstroke}%
\pgfsetdash{}{0pt}%
\pgfsys@defobject{currentmarker}{\pgfqpoint{-0.048611in}{0.000000in}}{\pgfqpoint{0.000000in}{0.000000in}}{%
\pgfpathmoveto{\pgfqpoint{0.000000in}{0.000000in}}%
\pgfpathlineto{\pgfqpoint{-0.048611in}{0.000000in}}%
\pgfusepath{stroke,fill}%
}%
\begin{pgfscope}%
\pgfsys@transformshift{3.806585in}{0.540740in}%
\pgfsys@useobject{currentmarker}{}%
\end{pgfscope}%
\end{pgfscope}%
\begin{pgfscope}%
\pgfsetbuttcap%
\pgfsetroundjoin%
\definecolor{currentfill}{rgb}{0.000000,0.000000,0.000000}%
\pgfsetfillcolor{currentfill}%
\pgfsetlinewidth{0.803000pt}%
\definecolor{currentstroke}{rgb}{0.000000,0.000000,0.000000}%
\pgfsetstrokecolor{currentstroke}%
\pgfsetdash{}{0pt}%
\pgfsys@defobject{currentmarker}{\pgfqpoint{-0.048611in}{0.000000in}}{\pgfqpoint{0.000000in}{0.000000in}}{%
\pgfpathmoveto{\pgfqpoint{0.000000in}{0.000000in}}%
\pgfpathlineto{\pgfqpoint{-0.048611in}{0.000000in}}%
\pgfusepath{stroke,fill}%
}%
\begin{pgfscope}%
\pgfsys@transformshift{3.806585in}{0.883922in}%
\pgfsys@useobject{currentmarker}{}%
\end{pgfscope}%
\end{pgfscope}%
\begin{pgfscope}%
\pgfsetbuttcap%
\pgfsetroundjoin%
\definecolor{currentfill}{rgb}{0.000000,0.000000,0.000000}%
\pgfsetfillcolor{currentfill}%
\pgfsetlinewidth{0.803000pt}%
\definecolor{currentstroke}{rgb}{0.000000,0.000000,0.000000}%
\pgfsetstrokecolor{currentstroke}%
\pgfsetdash{}{0pt}%
\pgfsys@defobject{currentmarker}{\pgfqpoint{-0.048611in}{0.000000in}}{\pgfqpoint{0.000000in}{0.000000in}}{%
\pgfpathmoveto{\pgfqpoint{0.000000in}{0.000000in}}%
\pgfpathlineto{\pgfqpoint{-0.048611in}{0.000000in}}%
\pgfusepath{stroke,fill}%
}%
\begin{pgfscope}%
\pgfsys@transformshift{3.806585in}{1.227104in}%
\pgfsys@useobject{currentmarker}{}%
\end{pgfscope}%
\end{pgfscope}%
\begin{pgfscope}%
\pgfsetbuttcap%
\pgfsetroundjoin%
\definecolor{currentfill}{rgb}{0.000000,0.000000,0.000000}%
\pgfsetfillcolor{currentfill}%
\pgfsetlinewidth{0.803000pt}%
\definecolor{currentstroke}{rgb}{0.000000,0.000000,0.000000}%
\pgfsetstrokecolor{currentstroke}%
\pgfsetdash{}{0pt}%
\pgfsys@defobject{currentmarker}{\pgfqpoint{-0.048611in}{0.000000in}}{\pgfqpoint{0.000000in}{0.000000in}}{%
\pgfpathmoveto{\pgfqpoint{0.000000in}{0.000000in}}%
\pgfpathlineto{\pgfqpoint{-0.048611in}{0.000000in}}%
\pgfusepath{stroke,fill}%
}%
\begin{pgfscope}%
\pgfsys@transformshift{3.806585in}{1.570286in}%
\pgfsys@useobject{currentmarker}{}%
\end{pgfscope}%
\end{pgfscope}%
\begin{pgfscope}%
\pgfsetbuttcap%
\pgfsetroundjoin%
\definecolor{currentfill}{rgb}{0.000000,0.000000,0.000000}%
\pgfsetfillcolor{currentfill}%
\pgfsetlinewidth{0.803000pt}%
\definecolor{currentstroke}{rgb}{0.000000,0.000000,0.000000}%
\pgfsetstrokecolor{currentstroke}%
\pgfsetdash{}{0pt}%
\pgfsys@defobject{currentmarker}{\pgfqpoint{-0.048611in}{0.000000in}}{\pgfqpoint{0.000000in}{0.000000in}}{%
\pgfpathmoveto{\pgfqpoint{0.000000in}{0.000000in}}%
\pgfpathlineto{\pgfqpoint{-0.048611in}{0.000000in}}%
\pgfusepath{stroke,fill}%
}%
\begin{pgfscope}%
\pgfsys@transformshift{3.806585in}{1.913467in}%
\pgfsys@useobject{currentmarker}{}%
\end{pgfscope}%
\end{pgfscope}%
\begin{pgfscope}%
\pgfpathrectangle{\pgfqpoint{3.806585in}{0.540740in}}{\pgfqpoint{2.465909in}{1.372727in}} %
\pgfusepath{clip}%
\pgfsetbuttcap%
\pgfsetroundjoin%
\pgfsetlinewidth{0.501875pt}%
\definecolor{currentstroke}{rgb}{0.000000,0.000000,0.000000}%
\pgfsetstrokecolor{currentstroke}%
\pgfsetdash{{1.850000pt}{0.800000pt}}{0.000000pt}%
\pgfpathmoveto{\pgfqpoint{3.806585in}{1.227104in}}%
\pgfpathlineto{\pgfqpoint{4.080575in}{1.227104in}}%
\pgfpathlineto{\pgfqpoint{4.354565in}{1.227104in}}%
\pgfpathlineto{\pgfqpoint{4.628555in}{1.227104in}}%
\pgfpathlineto{\pgfqpoint{4.902545in}{1.227104in}}%
\pgfpathlineto{\pgfqpoint{5.176535in}{1.227104in}}%
\pgfpathlineto{\pgfqpoint{5.450525in}{1.227104in}}%
\pgfpathlineto{\pgfqpoint{5.724515in}{1.227104in}}%
\pgfpathlineto{\pgfqpoint{5.998505in}{1.227104in}}%
\pgfpathlineto{\pgfqpoint{6.272495in}{1.227104in}}%
\pgfusepath{stroke}%
\end{pgfscope}%
\begin{pgfscope}%
\pgfpathrectangle{\pgfqpoint{3.806585in}{0.540740in}}{\pgfqpoint{2.465909in}{1.372727in}} %
\pgfusepath{clip}%
\pgfsetrectcap%
\pgfsetroundjoin%
\pgfsetlinewidth{1.003750pt}%
\definecolor{currentstroke}{rgb}{0.627451,0.321569,0.176471}%
\pgfsetstrokecolor{currentstroke}%
\pgfsetdash{}{0pt}%
\pgfpathmoveto{\pgfqpoint{3.792697in}{1.227085in}}%
\pgfpathlineto{\pgfqpoint{4.028132in}{1.226346in}}%
\pgfpathlineto{\pgfqpoint{4.085927in}{1.229712in}}%
\pgfpathlineto{\pgfqpoint{4.143722in}{1.227289in}}%
\pgfpathlineto{\pgfqpoint{4.201516in}{1.230594in}}%
\pgfpathlineto{\pgfqpoint{4.259311in}{1.222432in}}%
\pgfpathlineto{\pgfqpoint{4.317106in}{1.218004in}}%
\pgfpathlineto{\pgfqpoint{4.374900in}{1.253566in}}%
\pgfpathlineto{\pgfqpoint{4.432695in}{1.247052in}}%
\pgfpathlineto{\pgfqpoint{4.490490in}{1.267190in}}%
\pgfpathlineto{\pgfqpoint{4.548285in}{1.408927in}}%
\pgfpathlineto{\pgfqpoint{4.606079in}{1.637837in}}%
\pgfpathlineto{\pgfqpoint{4.663874in}{1.638334in}}%
\pgfpathlineto{\pgfqpoint{4.721669in}{1.197075in}}%
\pgfpathlineto{\pgfqpoint{4.779464in}{0.803750in}}%
\pgfpathlineto{\pgfqpoint{4.837258in}{0.879321in}}%
\pgfpathlineto{\pgfqpoint{4.895053in}{1.085589in}}%
\pgfpathlineto{\pgfqpoint{4.952848in}{1.163935in}}%
\pgfpathlineto{\pgfqpoint{5.010643in}{1.158769in}}%
\pgfpathlineto{\pgfqpoint{5.068437in}{1.250138in}}%
\pgfpathlineto{\pgfqpoint{5.126232in}{1.117548in}}%
\pgfpathlineto{\pgfqpoint{5.184027in}{1.016711in}}%
\pgfpathlineto{\pgfqpoint{5.241822in}{0.804117in}}%
\pgfpathlineto{\pgfqpoint{5.299616in}{0.916204in}}%
\pgfpathlineto{\pgfqpoint{5.357411in}{1.206212in}}%
\pgfpathlineto{\pgfqpoint{5.415206in}{1.532909in}}%
\pgfpathlineto{\pgfqpoint{5.473001in}{1.674385in}}%
\pgfpathlineto{\pgfqpoint{5.530795in}{1.447583in}}%
\pgfpathlineto{\pgfqpoint{5.588590in}{1.291087in}}%
\pgfpathlineto{\pgfqpoint{5.646385in}{1.230184in}}%
\pgfpathlineto{\pgfqpoint{5.704180in}{1.242321in}}%
\pgfpathlineto{\pgfqpoint{5.761974in}{1.218707in}}%
\pgfpathlineto{\pgfqpoint{5.819769in}{1.222432in}}%
\pgfpathlineto{\pgfqpoint{5.877564in}{1.227783in}}%
\pgfpathlineto{\pgfqpoint{5.935359in}{1.229397in}}%
\pgfpathlineto{\pgfqpoint{5.993153in}{1.225495in}}%
\pgfpathlineto{\pgfqpoint{6.050948in}{1.227751in}}%
\pgfpathlineto{\pgfqpoint{6.166538in}{1.226912in}}%
\pgfpathlineto{\pgfqpoint{6.286383in}{1.227085in}}%
\pgfpathlineto{\pgfqpoint{6.286383in}{1.227085in}}%
\pgfusepath{stroke}%
\end{pgfscope}%
\begin{pgfscope}%
\pgfpathrectangle{\pgfqpoint{3.806585in}{0.540740in}}{\pgfqpoint{2.465909in}{1.372727in}} %
\pgfusepath{clip}%
\pgfsetbuttcap%
\pgfsetroundjoin%
\pgfsetlinewidth{0.501875pt}%
\definecolor{currentstroke}{rgb}{0.000000,0.000000,0.000000}%
\pgfsetstrokecolor{currentstroke}%
\pgfsetdash{{1.850000pt}{0.800000pt}}{0.000000pt}%
\pgfpathmoveto{\pgfqpoint{5.363660in}{0.526851in}}%
\pgfpathlineto{\pgfqpoint{5.363660in}{0.617003in}}%
\pgfpathlineto{\pgfqpoint{5.363660in}{0.769528in}}%
\pgfpathlineto{\pgfqpoint{5.363660in}{0.922053in}}%
\pgfpathlineto{\pgfqpoint{5.363660in}{1.074579in}}%
\pgfpathlineto{\pgfqpoint{5.363660in}{1.227104in}}%
\pgfusepath{stroke}%
\end{pgfscope}%
\begin{pgfscope}%
\pgfpathrectangle{\pgfqpoint{3.806585in}{0.540740in}}{\pgfqpoint{2.465909in}{1.372727in}} %
\pgfusepath{clip}%
\pgfsetbuttcap%
\pgfsetroundjoin%
\pgfsetlinewidth{0.501875pt}%
\definecolor{currentstroke}{rgb}{0.000000,0.000000,0.000000}%
\pgfsetstrokecolor{currentstroke}%
\pgfsetdash{{1.850000pt}{0.800000pt}}{0.000000pt}%
\pgfpathmoveto{\pgfqpoint{4.715420in}{0.526851in}}%
\pgfpathlineto{\pgfqpoint{4.715420in}{0.617003in}}%
\pgfpathlineto{\pgfqpoint{4.715420in}{0.769528in}}%
\pgfpathlineto{\pgfqpoint{4.715420in}{0.922053in}}%
\pgfpathlineto{\pgfqpoint{4.715420in}{1.074579in}}%
\pgfpathlineto{\pgfqpoint{4.715420in}{1.227104in}}%
\pgfusepath{stroke}%
\end{pgfscope}%
\begin{pgfscope}%
\pgfsetrectcap%
\pgfsetmiterjoin%
\pgfsetlinewidth{0.803000pt}%
\definecolor{currentstroke}{rgb}{0.000000,0.000000,0.000000}%
\pgfsetstrokecolor{currentstroke}%
\pgfsetdash{}{0pt}%
\pgfpathmoveto{\pgfqpoint{3.806585in}{0.540740in}}%
\pgfpathlineto{\pgfqpoint{3.806585in}{1.913467in}}%
\pgfusepath{stroke}%
\end{pgfscope}%
\begin{pgfscope}%
\pgfsetrectcap%
\pgfsetmiterjoin%
\pgfsetlinewidth{0.803000pt}%
\definecolor{currentstroke}{rgb}{0.000000,0.000000,0.000000}%
\pgfsetstrokecolor{currentstroke}%
\pgfsetdash{}{0pt}%
\pgfpathmoveto{\pgfqpoint{6.272495in}{0.540740in}}%
\pgfpathlineto{\pgfqpoint{6.272495in}{1.913467in}}%
\pgfusepath{stroke}%
\end{pgfscope}%
\begin{pgfscope}%
\pgfsetrectcap%
\pgfsetmiterjoin%
\pgfsetlinewidth{0.803000pt}%
\definecolor{currentstroke}{rgb}{0.000000,0.000000,0.000000}%
\pgfsetstrokecolor{currentstroke}%
\pgfsetdash{}{0pt}%
\pgfpathmoveto{\pgfqpoint{3.806585in}{0.540740in}}%
\pgfpathlineto{\pgfqpoint{6.272495in}{0.540740in}}%
\pgfusepath{stroke}%
\end{pgfscope}%
\begin{pgfscope}%
\pgfsetrectcap%
\pgfsetmiterjoin%
\pgfsetlinewidth{0.803000pt}%
\definecolor{currentstroke}{rgb}{0.000000,0.000000,0.000000}%
\pgfsetstrokecolor{currentstroke}%
\pgfsetdash{}{0pt}%
\pgfpathmoveto{\pgfqpoint{3.806585in}{1.913467in}}%
\pgfpathlineto{\pgfqpoint{6.272495in}{1.913467in}}%
\pgfusepath{stroke}%
\end{pgfscope}%
\begin{pgfscope}%
\pgftext[x=3.868233in,y=1.707558in,left,base]{\rmfamily\fontsize{10.000000}{12.000000}\selectfont (d)}%
\end{pgfscope}%
\end{pgfpicture}%
\makeatother%
\endgroup%

\caption{Run 1 with parameters listed in Tab. \ref{tab_parameters}: (a) Initial ($t=t_0=0$) and final ($t=t_\mr{f}=200\,|\Omega_\mr{ce}|$) distribution function in parallel direction obtained with standard finite element PIC methods from section \ref{sec_standard} . (b) Same as (a) for structure-preserving finite element PIC methods from section \ref{sec_geometric} with the Strang splitting scheme (\ref{eq_Strang}). (c) Difference between the initial and final distribution corresponding to (a). (d) Difference between the initial and final distribution corresponding to (b).\label{fig_distribution_functions}}
\end{figure}

In addition to the time evolution of the energies, we plot in Fig. \ref{fig_distribution_functions} the distribution functions $f_{\mr{h}\parallel}=2\pi\int f_\mr{h}v_\perp\,\mr{d}v_\perp$ for the parallel velocity at the beginning ($t=t_0=0$) and at the end ($t=t_\mr{f}=200\,|\Omega_\mr{ce}|$) of the simulations. In both cases, we observe a flattening of the distribution functions around the resonant velocities, which are expected to be at $v_\mr{R}=(\omega_\mr{r}+|\Omega_\mr{ce}|)/k\approx\pm0.26\,c$ for the wavenumber $k=2|\Omega_\mr{ce}|/c$. This means that energetic electrons initially close to the resonant velocities gain energy in parallel direction which can more clearly be seen in the plots below where we show the difference in the initial and final distributions. In contrast to that, energetic electrons lose energy in perpendicular direction (not shown). A quantitative analysis yields that the energetic electrons lose more energy in perpendicular direction than what they gain in parallel direction, which is of course expected because the wave energies grow due to energy transfer from the energetic electrons to the wave. Qualitatively, the two algorithms do not result in visible differences regarding the distribution functions.  

Finally, we check the conservation of the total energy $\mathcal{E}$ in the system and show in Fig. \ref{fig_comparison} the evolution of its relative error $|\mathcal{E}(t)-\mathcal{E}(0)|/\mathcal{E}(0)$ with respect to time for three cases: For the first case (purple), which is standard PIC, we find an oscillation of the error on a nearly bounded level until $t\approx40\,|\Omega_\mr{ce}|$. This is followed by a sudden increase of the error of about three orders of magnitude until a saturation phase is reached. Second, we plot the evolution of the relative error for geometric PIC with the first-order Lie-Trotter splitting (\ref{eq_LieTrotter}) (brown) and we observe that the error is again oscillating, however, uniformly bounded during the whole simulation. This is expected for a symplectic integrator \citep{Haireretal2006}. Third, we observe for geometric PIC with the Strang-splitting (\ref{eq_Strang}) (orange), which is second order in time, that the error is reduced by about three orders of magnitude and that it shows the same behavior as the Lie-Trotter splitting up to $t\approx110|\Omega_\mr{ce}|$, i.e. the error is oscillating and uniformly bounded. However, this is followed by a slow, linear increase of the error, where the oscillation vanishes.
\begin{figure}[!t]
\centering
\includegraphics[scale=1]{01_Figures/comparison_energies_1e5.pdf}
%%% Creator: Matplotlib, PGF backend
%%
%% To include the figure in your LaTeX document, write
%%   \input{<filename>.pgf}
%%
%% Make sure the required packages are loaded in your preamble
%%   \usepackage{pgf}
%%
%% Figures using additional raster images can only be included by \input if
%% they are in the same directory as the main LaTeX file. For loading figures
%% from other directories you can use the `import` package
%%   \usepackage{import}
%% and then include the figures with
%%   \import{<path to file>}{<filename>.pgf}
%%
%% Matplotlib used the following preamble
%%   \usepackage{fontspec}
%%   \setmainfont{DejaVu Serif}
%%   \setsansfont{DejaVu Sans}
%%   \setmonofont{DejaVu Sans Mono}
%%
\begingroup%
\makeatletter%
\begin{pgfpicture}%
\pgfpathrectangle{\pgfpointorigin}{\pgfqpoint{5.274168in}{2.339841in}}%
\pgfusepath{use as bounding box, clip}%
\begin{pgfscope}%
\pgfsetbuttcap%
\pgfsetmiterjoin%
\definecolor{currentfill}{rgb}{1.000000,1.000000,1.000000}%
\pgfsetfillcolor{currentfill}%
\pgfsetlinewidth{0.000000pt}%
\definecolor{currentstroke}{rgb}{1.000000,1.000000,1.000000}%
\pgfsetstrokecolor{currentstroke}%
\pgfsetdash{}{0pt}%
\pgfpathmoveto{\pgfqpoint{0.000000in}{0.000000in}}%
\pgfpathlineto{\pgfqpoint{5.274168in}{0.000000in}}%
\pgfpathlineto{\pgfqpoint{5.274168in}{2.339841in}}%
\pgfpathlineto{\pgfqpoint{0.000000in}{2.339841in}}%
\pgfpathclose%
\pgfusepath{fill}%
\end{pgfscope}%
\begin{pgfscope}%
\pgfsetbuttcap%
\pgfsetmiterjoin%
\definecolor{currentfill}{rgb}{1.000000,1.000000,1.000000}%
\pgfsetfillcolor{currentfill}%
\pgfsetlinewidth{0.000000pt}%
\definecolor{currentstroke}{rgb}{0.000000,0.000000,0.000000}%
\pgfsetstrokecolor{currentstroke}%
\pgfsetstrokeopacity{0.000000}%
\pgfsetdash{}{0pt}%
\pgfpathmoveto{\pgfqpoint{0.735032in}{0.526079in}}%
\pgfpathlineto{\pgfqpoint{2.905032in}{0.526079in}}%
\pgfpathlineto{\pgfqpoint{2.905032in}{2.187079in}}%
\pgfpathlineto{\pgfqpoint{0.735032in}{2.187079in}}%
\pgfpathclose%
\pgfusepath{fill}%
\end{pgfscope}%
\begin{pgfscope}%
\pgfsetbuttcap%
\pgfsetroundjoin%
\definecolor{currentfill}{rgb}{0.000000,0.000000,0.000000}%
\pgfsetfillcolor{currentfill}%
\pgfsetlinewidth{0.803000pt}%
\definecolor{currentstroke}{rgb}{0.000000,0.000000,0.000000}%
\pgfsetstrokecolor{currentstroke}%
\pgfsetdash{}{0pt}%
\pgfsys@defobject{currentmarker}{\pgfqpoint{0.000000in}{-0.048611in}}{\pgfqpoint{0.000000in}{0.000000in}}{%
\pgfpathmoveto{\pgfqpoint{0.000000in}{0.000000in}}%
\pgfpathlineto{\pgfqpoint{0.000000in}{-0.048611in}}%
\pgfusepath{stroke,fill}%
}%
\begin{pgfscope}%
\pgfsys@transformshift{0.735032in}{0.526079in}%
\pgfsys@useobject{currentmarker}{}%
\end{pgfscope}%
\end{pgfscope}%
\begin{pgfscope}%
\pgftext[x=0.735032in,y=0.428857in,,top]{\rmfamily\fontsize{10.000000}{12.000000}\selectfont \(\displaystyle 0\)}%
\end{pgfscope}%
\begin{pgfscope}%
\pgfsetbuttcap%
\pgfsetroundjoin%
\definecolor{currentfill}{rgb}{0.000000,0.000000,0.000000}%
\pgfsetfillcolor{currentfill}%
\pgfsetlinewidth{0.803000pt}%
\definecolor{currentstroke}{rgb}{0.000000,0.000000,0.000000}%
\pgfsetstrokecolor{currentstroke}%
\pgfsetdash{}{0pt}%
\pgfsys@defobject{currentmarker}{\pgfqpoint{0.000000in}{-0.048611in}}{\pgfqpoint{0.000000in}{0.000000in}}{%
\pgfpathmoveto{\pgfqpoint{0.000000in}{0.000000in}}%
\pgfpathlineto{\pgfqpoint{0.000000in}{-0.048611in}}%
\pgfusepath{stroke,fill}%
}%
\begin{pgfscope}%
\pgfsys@transformshift{1.277532in}{0.526079in}%
\pgfsys@useobject{currentmarker}{}%
\end{pgfscope}%
\end{pgfscope}%
\begin{pgfscope}%
\pgftext[x=1.277532in,y=0.428857in,,top]{\rmfamily\fontsize{10.000000}{12.000000}\selectfont \(\displaystyle 50\)}%
\end{pgfscope}%
\begin{pgfscope}%
\pgfsetbuttcap%
\pgfsetroundjoin%
\definecolor{currentfill}{rgb}{0.000000,0.000000,0.000000}%
\pgfsetfillcolor{currentfill}%
\pgfsetlinewidth{0.803000pt}%
\definecolor{currentstroke}{rgb}{0.000000,0.000000,0.000000}%
\pgfsetstrokecolor{currentstroke}%
\pgfsetdash{}{0pt}%
\pgfsys@defobject{currentmarker}{\pgfqpoint{0.000000in}{-0.048611in}}{\pgfqpoint{0.000000in}{0.000000in}}{%
\pgfpathmoveto{\pgfqpoint{0.000000in}{0.000000in}}%
\pgfpathlineto{\pgfqpoint{0.000000in}{-0.048611in}}%
\pgfusepath{stroke,fill}%
}%
\begin{pgfscope}%
\pgfsys@transformshift{1.820032in}{0.526079in}%
\pgfsys@useobject{currentmarker}{}%
\end{pgfscope}%
\end{pgfscope}%
\begin{pgfscope}%
\pgftext[x=1.820032in,y=0.428857in,,top]{\rmfamily\fontsize{10.000000}{12.000000}\selectfont \(\displaystyle 100\)}%
\end{pgfscope}%
\begin{pgfscope}%
\pgfsetbuttcap%
\pgfsetroundjoin%
\definecolor{currentfill}{rgb}{0.000000,0.000000,0.000000}%
\pgfsetfillcolor{currentfill}%
\pgfsetlinewidth{0.803000pt}%
\definecolor{currentstroke}{rgb}{0.000000,0.000000,0.000000}%
\pgfsetstrokecolor{currentstroke}%
\pgfsetdash{}{0pt}%
\pgfsys@defobject{currentmarker}{\pgfqpoint{0.000000in}{-0.048611in}}{\pgfqpoint{0.000000in}{0.000000in}}{%
\pgfpathmoveto{\pgfqpoint{0.000000in}{0.000000in}}%
\pgfpathlineto{\pgfqpoint{0.000000in}{-0.048611in}}%
\pgfusepath{stroke,fill}%
}%
\begin{pgfscope}%
\pgfsys@transformshift{2.362532in}{0.526079in}%
\pgfsys@useobject{currentmarker}{}%
\end{pgfscope}%
\end{pgfscope}%
\begin{pgfscope}%
\pgftext[x=2.362532in,y=0.428857in,,top]{\rmfamily\fontsize{10.000000}{12.000000}\selectfont \(\displaystyle 150\)}%
\end{pgfscope}%
\begin{pgfscope}%
\pgfsetbuttcap%
\pgfsetroundjoin%
\definecolor{currentfill}{rgb}{0.000000,0.000000,0.000000}%
\pgfsetfillcolor{currentfill}%
\pgfsetlinewidth{0.803000pt}%
\definecolor{currentstroke}{rgb}{0.000000,0.000000,0.000000}%
\pgfsetstrokecolor{currentstroke}%
\pgfsetdash{}{0pt}%
\pgfsys@defobject{currentmarker}{\pgfqpoint{0.000000in}{-0.048611in}}{\pgfqpoint{0.000000in}{0.000000in}}{%
\pgfpathmoveto{\pgfqpoint{0.000000in}{0.000000in}}%
\pgfpathlineto{\pgfqpoint{0.000000in}{-0.048611in}}%
\pgfusepath{stroke,fill}%
}%
\begin{pgfscope}%
\pgfsys@transformshift{2.905032in}{0.526079in}%
\pgfsys@useobject{currentmarker}{}%
\end{pgfscope}%
\end{pgfscope}%
\begin{pgfscope}%
\pgftext[x=2.905032in,y=0.428857in,,top]{\rmfamily\fontsize{10.000000}{12.000000}\selectfont \(\displaystyle 200\)}%
\end{pgfscope}%
\begin{pgfscope}%
\pgftext[x=1.820032in,y=0.238889in,,top]{\rmfamily\fontsize{10.000000}{12.000000}\selectfont \(\displaystyle t|\Omega_\mathrm{ce}|\)}%
\end{pgfscope}%
\begin{pgfscope}%
\pgfsetbuttcap%
\pgfsetroundjoin%
\definecolor{currentfill}{rgb}{0.000000,0.000000,0.000000}%
\pgfsetfillcolor{currentfill}%
\pgfsetlinewidth{0.803000pt}%
\definecolor{currentstroke}{rgb}{0.000000,0.000000,0.000000}%
\pgfsetstrokecolor{currentstroke}%
\pgfsetdash{}{0pt}%
\pgfsys@defobject{currentmarker}{\pgfqpoint{-0.048611in}{0.000000in}}{\pgfqpoint{0.000000in}{0.000000in}}{%
\pgfpathmoveto{\pgfqpoint{0.000000in}{0.000000in}}%
\pgfpathlineto{\pgfqpoint{-0.048611in}{0.000000in}}%
\pgfusepath{stroke,fill}%
}%
\begin{pgfscope}%
\pgfsys@transformshift{0.735032in}{0.526079in}%
\pgfsys@useobject{currentmarker}{}%
\end{pgfscope}%
\end{pgfscope}%
\begin{pgfscope}%
\pgftext[x=0.294444in,y=0.473318in,left,base]{\rmfamily\fontsize{10.000000}{12.000000}\selectfont \(\displaystyle 10^{-12}\)}%
\end{pgfscope}%
\begin{pgfscope}%
\pgfsetbuttcap%
\pgfsetroundjoin%
\definecolor{currentfill}{rgb}{0.000000,0.000000,0.000000}%
\pgfsetfillcolor{currentfill}%
\pgfsetlinewidth{0.803000pt}%
\definecolor{currentstroke}{rgb}{0.000000,0.000000,0.000000}%
\pgfsetstrokecolor{currentstroke}%
\pgfsetdash{}{0pt}%
\pgfsys@defobject{currentmarker}{\pgfqpoint{-0.048611in}{0.000000in}}{\pgfqpoint{0.000000in}{0.000000in}}{%
\pgfpathmoveto{\pgfqpoint{0.000000in}{0.000000in}}%
\pgfpathlineto{\pgfqpoint{-0.048611in}{0.000000in}}%
\pgfusepath{stroke,fill}%
}%
\begin{pgfscope}%
\pgfsys@transformshift{0.735032in}{0.858279in}%
\pgfsys@useobject{currentmarker}{}%
\end{pgfscope}%
\end{pgfscope}%
\begin{pgfscope}%
\pgftext[x=0.294444in,y=0.805518in,left,base]{\rmfamily\fontsize{10.000000}{12.000000}\selectfont \(\displaystyle 10^{-10}\)}%
\end{pgfscope}%
\begin{pgfscope}%
\pgfsetbuttcap%
\pgfsetroundjoin%
\definecolor{currentfill}{rgb}{0.000000,0.000000,0.000000}%
\pgfsetfillcolor{currentfill}%
\pgfsetlinewidth{0.803000pt}%
\definecolor{currentstroke}{rgb}{0.000000,0.000000,0.000000}%
\pgfsetstrokecolor{currentstroke}%
\pgfsetdash{}{0pt}%
\pgfsys@defobject{currentmarker}{\pgfqpoint{-0.048611in}{0.000000in}}{\pgfqpoint{0.000000in}{0.000000in}}{%
\pgfpathmoveto{\pgfqpoint{0.000000in}{0.000000in}}%
\pgfpathlineto{\pgfqpoint{-0.048611in}{0.000000in}}%
\pgfusepath{stroke,fill}%
}%
\begin{pgfscope}%
\pgfsys@transformshift{0.735032in}{1.190479in}%
\pgfsys@useobject{currentmarker}{}%
\end{pgfscope}%
\end{pgfscope}%
\begin{pgfscope}%
\pgftext[x=0.349807in,y=1.137718in,left,base]{\rmfamily\fontsize{10.000000}{12.000000}\selectfont \(\displaystyle 10^{-8}\)}%
\end{pgfscope}%
\begin{pgfscope}%
\pgfsetbuttcap%
\pgfsetroundjoin%
\definecolor{currentfill}{rgb}{0.000000,0.000000,0.000000}%
\pgfsetfillcolor{currentfill}%
\pgfsetlinewidth{0.803000pt}%
\definecolor{currentstroke}{rgb}{0.000000,0.000000,0.000000}%
\pgfsetstrokecolor{currentstroke}%
\pgfsetdash{}{0pt}%
\pgfsys@defobject{currentmarker}{\pgfqpoint{-0.048611in}{0.000000in}}{\pgfqpoint{0.000000in}{0.000000in}}{%
\pgfpathmoveto{\pgfqpoint{0.000000in}{0.000000in}}%
\pgfpathlineto{\pgfqpoint{-0.048611in}{0.000000in}}%
\pgfusepath{stroke,fill}%
}%
\begin{pgfscope}%
\pgfsys@transformshift{0.735032in}{1.522679in}%
\pgfsys@useobject{currentmarker}{}%
\end{pgfscope}%
\end{pgfscope}%
\begin{pgfscope}%
\pgftext[x=0.349807in,y=1.469918in,left,base]{\rmfamily\fontsize{10.000000}{12.000000}\selectfont \(\displaystyle 10^{-6}\)}%
\end{pgfscope}%
\begin{pgfscope}%
\pgfsetbuttcap%
\pgfsetroundjoin%
\definecolor{currentfill}{rgb}{0.000000,0.000000,0.000000}%
\pgfsetfillcolor{currentfill}%
\pgfsetlinewidth{0.803000pt}%
\definecolor{currentstroke}{rgb}{0.000000,0.000000,0.000000}%
\pgfsetstrokecolor{currentstroke}%
\pgfsetdash{}{0pt}%
\pgfsys@defobject{currentmarker}{\pgfqpoint{-0.048611in}{0.000000in}}{\pgfqpoint{0.000000in}{0.000000in}}{%
\pgfpathmoveto{\pgfqpoint{0.000000in}{0.000000in}}%
\pgfpathlineto{\pgfqpoint{-0.048611in}{0.000000in}}%
\pgfusepath{stroke,fill}%
}%
\begin{pgfscope}%
\pgfsys@transformshift{0.735032in}{1.854879in}%
\pgfsys@useobject{currentmarker}{}%
\end{pgfscope}%
\end{pgfscope}%
\begin{pgfscope}%
\pgftext[x=0.349807in,y=1.802118in,left,base]{\rmfamily\fontsize{10.000000}{12.000000}\selectfont \(\displaystyle 10^{-4}\)}%
\end{pgfscope}%
\begin{pgfscope}%
\pgfsetbuttcap%
\pgfsetroundjoin%
\definecolor{currentfill}{rgb}{0.000000,0.000000,0.000000}%
\pgfsetfillcolor{currentfill}%
\pgfsetlinewidth{0.803000pt}%
\definecolor{currentstroke}{rgb}{0.000000,0.000000,0.000000}%
\pgfsetstrokecolor{currentstroke}%
\pgfsetdash{}{0pt}%
\pgfsys@defobject{currentmarker}{\pgfqpoint{-0.048611in}{0.000000in}}{\pgfqpoint{0.000000in}{0.000000in}}{%
\pgfpathmoveto{\pgfqpoint{0.000000in}{0.000000in}}%
\pgfpathlineto{\pgfqpoint{-0.048611in}{0.000000in}}%
\pgfusepath{stroke,fill}%
}%
\begin{pgfscope}%
\pgfsys@transformshift{0.735032in}{2.187079in}%
\pgfsys@useobject{currentmarker}{}%
\end{pgfscope}%
\end{pgfscope}%
\begin{pgfscope}%
\pgftext[x=0.349807in,y=2.134318in,left,base]{\rmfamily\fontsize{10.000000}{12.000000}\selectfont \(\displaystyle 10^{-2}\)}%
\end{pgfscope}%
\begin{pgfscope}%
\pgftext[x=0.238889in,y=1.356579in,,bottom,rotate=90.000000]{\rmfamily\fontsize{10.000000}{12.000000}\selectfont \(\displaystyle |\mathcal{E}(t) - \mathcal{E}(0)|/\mathcal{E}(0)\)}%
\end{pgfscope}%
\begin{pgfscope}%
\pgfpathrectangle{\pgfqpoint{0.735032in}{0.526079in}}{\pgfqpoint{2.170000in}{1.661000in}} %
\pgfusepath{clip}%
\pgfsetrectcap%
\pgfsetroundjoin%
\pgfsetlinewidth{1.003750pt}%
\definecolor{currentstroke}{rgb}{0.627451,0.321569,0.176471}%
\pgfsetstrokecolor{currentstroke}%
\pgfsetdash{}{0pt}%
\pgfpathmoveto{\pgfqpoint{0.735167in}{0.512191in}}%
\pgfpathlineto{\pgfqpoint{0.736659in}{1.617419in}}%
\pgfpathlineto{\pgfqpoint{0.737473in}{1.626479in}}%
\pgfpathlineto{\pgfqpoint{0.738151in}{1.621223in}}%
\pgfpathlineto{\pgfqpoint{0.739236in}{1.576537in}}%
\pgfpathlineto{\pgfqpoint{0.739914in}{1.390060in}}%
\pgfpathlineto{\pgfqpoint{0.740593in}{1.585118in}}%
\pgfpathlineto{\pgfqpoint{0.743305in}{1.729277in}}%
\pgfpathlineto{\pgfqpoint{0.747509in}{1.803388in}}%
\pgfpathlineto{\pgfqpoint{0.751849in}{1.833694in}}%
\pgfpathlineto{\pgfqpoint{0.754698in}{1.838401in}}%
\pgfpathlineto{\pgfqpoint{0.756325in}{1.836357in}}%
\pgfpathlineto{\pgfqpoint{0.758766in}{1.826581in}}%
\pgfpathlineto{\pgfqpoint{0.762021in}{1.798351in}}%
\pgfpathlineto{\pgfqpoint{0.765412in}{1.739004in}}%
\pgfpathlineto{\pgfqpoint{0.767989in}{1.632988in}}%
\pgfpathlineto{\pgfqpoint{0.768938in}{1.488196in}}%
\pgfpathlineto{\pgfqpoint{0.769074in}{1.381082in}}%
\pgfpathlineto{\pgfqpoint{0.770294in}{1.604272in}}%
\pgfpathlineto{\pgfqpoint{0.771515in}{1.627150in}}%
\pgfpathlineto{\pgfqpoint{0.772193in}{1.622665in}}%
\pgfpathlineto{\pgfqpoint{0.773278in}{1.579693in}}%
\pgfpathlineto{\pgfqpoint{0.774092in}{1.393318in}}%
\pgfpathlineto{\pgfqpoint{0.774634in}{1.579539in}}%
\pgfpathlineto{\pgfqpoint{0.777347in}{1.727958in}}%
\pgfpathlineto{\pgfqpoint{0.781551in}{1.802957in}}%
\pgfpathlineto{\pgfqpoint{0.785756in}{1.832998in}}%
\pgfpathlineto{\pgfqpoint{0.788739in}{1.838428in}}%
\pgfpathlineto{\pgfqpoint{0.790231in}{1.836764in}}%
\pgfpathlineto{\pgfqpoint{0.792537in}{1.828269in}}%
\pgfpathlineto{\pgfqpoint{0.795656in}{1.803532in}}%
\pgfpathlineto{\pgfqpoint{0.799047in}{1.749818in}}%
\pgfpathlineto{\pgfqpoint{0.801759in}{1.653795in}}%
\pgfpathlineto{\pgfqpoint{0.802980in}{1.508086in}}%
\pgfpathlineto{\pgfqpoint{0.803251in}{1.373695in}}%
\pgfpathlineto{\pgfqpoint{0.804201in}{1.594103in}}%
\pgfpathlineto{\pgfqpoint{0.805693in}{1.626760in}}%
\pgfpathlineto{\pgfqpoint{0.806235in}{1.622836in}}%
\pgfpathlineto{\pgfqpoint{0.807320in}{1.583562in}}%
\pgfpathlineto{\pgfqpoint{0.807998in}{1.469036in}}%
\pgfpathlineto{\pgfqpoint{0.808134in}{1.249346in}}%
\pgfpathlineto{\pgfqpoint{0.809354in}{1.639520in}}%
\pgfpathlineto{\pgfqpoint{0.812609in}{1.756560in}}%
\pgfpathlineto{\pgfqpoint{0.816949in}{1.815674in}}%
\pgfpathlineto{\pgfqpoint{0.821018in}{1.836546in}}%
\pgfpathlineto{\pgfqpoint{0.823053in}{1.838295in}}%
\pgfpathlineto{\pgfqpoint{0.823324in}{1.838116in}}%
\pgfpathlineto{\pgfqpoint{0.825087in}{1.834711in}}%
\pgfpathlineto{\pgfqpoint{0.827799in}{1.820906in}}%
\pgfpathlineto{\pgfqpoint{0.831190in}{1.785065in}}%
\pgfpathlineto{\pgfqpoint{0.834309in}{1.718949in}}%
\pgfpathlineto{\pgfqpoint{0.836479in}{1.606300in}}%
\pgfpathlineto{\pgfqpoint{0.837158in}{1.470557in}}%
\pgfpathlineto{\pgfqpoint{0.837293in}{1.186336in}}%
\pgfpathlineto{\pgfqpoint{0.838514in}{1.605364in}}%
\pgfpathlineto{\pgfqpoint{0.839734in}{1.627115in}}%
\pgfpathlineto{\pgfqpoint{0.840413in}{1.621377in}}%
\pgfpathlineto{\pgfqpoint{0.841498in}{1.575856in}}%
\pgfpathlineto{\pgfqpoint{0.842176in}{1.372869in}}%
\pgfpathlineto{\pgfqpoint{0.842854in}{1.585796in}}%
\pgfpathlineto{\pgfqpoint{0.845566in}{1.729379in}}%
\pgfpathlineto{\pgfqpoint{0.849771in}{1.803548in}}%
\pgfpathlineto{\pgfqpoint{0.854111in}{1.833702in}}%
\pgfpathlineto{\pgfqpoint{0.856959in}{1.838391in}}%
\pgfpathlineto{\pgfqpoint{0.858586in}{1.836317in}}%
\pgfpathlineto{\pgfqpoint{0.861028in}{1.826421in}}%
\pgfpathlineto{\pgfqpoint{0.864283in}{1.798261in}}%
\pgfpathlineto{\pgfqpoint{0.867673in}{1.738811in}}%
\pgfpathlineto{\pgfqpoint{0.870114in}{1.641801in}}%
\pgfpathlineto{\pgfqpoint{0.871199in}{1.484697in}}%
\pgfpathlineto{\pgfqpoint{0.871335in}{1.358704in}}%
\pgfpathlineto{\pgfqpoint{0.872556in}{1.605187in}}%
\pgfpathlineto{\pgfqpoint{0.873912in}{1.627800in}}%
\pgfpathlineto{\pgfqpoint{0.874454in}{1.623836in}}%
\pgfpathlineto{\pgfqpoint{0.875539in}{1.582530in}}%
\pgfpathlineto{\pgfqpoint{0.876218in}{1.446337in}}%
\pgfpathlineto{\pgfqpoint{0.876353in}{1.360708in}}%
\pgfpathlineto{\pgfqpoint{0.877438in}{1.632617in}}%
\pgfpathlineto{\pgfqpoint{0.880558in}{1.751948in}}%
\pgfpathlineto{\pgfqpoint{0.885033in}{1.815044in}}%
\pgfpathlineto{\pgfqpoint{0.889102in}{1.836417in}}%
\pgfpathlineto{\pgfqpoint{0.891136in}{1.838381in}}%
\pgfpathlineto{\pgfqpoint{0.891543in}{1.838134in}}%
\pgfpathlineto{\pgfqpoint{0.893171in}{1.835026in}}%
\pgfpathlineto{\pgfqpoint{0.895883in}{1.821558in}}%
\pgfpathlineto{\pgfqpoint{0.899274in}{1.786217in}}%
\pgfpathlineto{\pgfqpoint{0.902529in}{1.716888in}}%
\pgfpathlineto{\pgfqpoint{0.904699in}{1.597298in}}%
\pgfpathlineto{\pgfqpoint{0.905377in}{1.392298in}}%
\pgfpathlineto{\pgfqpoint{0.906191in}{1.580821in}}%
\pgfpathlineto{\pgfqpoint{0.907954in}{1.629080in}}%
\pgfpathlineto{\pgfqpoint{0.908361in}{1.626322in}}%
\pgfpathlineto{\pgfqpoint{0.909310in}{1.600239in}}%
\pgfpathlineto{\pgfqpoint{0.910124in}{1.513967in}}%
\pgfpathlineto{\pgfqpoint{0.910395in}{1.179305in}}%
\pgfpathlineto{\pgfqpoint{0.911344in}{1.618667in}}%
\pgfpathlineto{\pgfqpoint{0.914328in}{1.744856in}}%
\pgfpathlineto{\pgfqpoint{0.918668in}{1.810983in}}%
\pgfpathlineto{\pgfqpoint{0.922737in}{1.835249in}}%
\pgfpathlineto{\pgfqpoint{0.925314in}{1.838427in}}%
\pgfpathlineto{\pgfqpoint{0.925449in}{1.838357in}}%
\pgfpathlineto{\pgfqpoint{0.927213in}{1.835170in}}%
\pgfpathlineto{\pgfqpoint{0.929789in}{1.822807in}}%
\pgfpathlineto{\pgfqpoint{0.933180in}{1.788684in}}%
\pgfpathlineto{\pgfqpoint{0.936435in}{1.721985in}}%
\pgfpathlineto{\pgfqpoint{0.938605in}{1.614628in}}%
\pgfpathlineto{\pgfqpoint{0.939419in}{1.417176in}}%
\pgfpathlineto{\pgfqpoint{0.940368in}{1.589263in}}%
\pgfpathlineto{\pgfqpoint{0.942131in}{1.630198in}}%
\pgfpathlineto{\pgfqpoint{0.942538in}{1.626485in}}%
\pgfpathlineto{\pgfqpoint{0.943759in}{1.579478in}}%
\pgfpathlineto{\pgfqpoint{0.944437in}{1.413304in}}%
\pgfpathlineto{\pgfqpoint{0.945115in}{1.585747in}}%
\pgfpathlineto{\pgfqpoint{0.947828in}{1.729888in}}%
\pgfpathlineto{\pgfqpoint{0.952032in}{1.803784in}}%
\pgfpathlineto{\pgfqpoint{0.956372in}{1.833817in}}%
\pgfpathlineto{\pgfqpoint{0.959220in}{1.838507in}}%
\pgfpathlineto{\pgfqpoint{0.960983in}{1.836099in}}%
\pgfpathlineto{\pgfqpoint{0.963560in}{1.824886in}}%
\pgfpathlineto{\pgfqpoint{0.966815in}{1.794690in}}%
\pgfpathlineto{\pgfqpoint{0.970206in}{1.730904in}}%
\pgfpathlineto{\pgfqpoint{0.972511in}{1.627825in}}%
\pgfpathlineto{\pgfqpoint{0.973596in}{1.399320in}}%
\pgfpathlineto{\pgfqpoint{0.974410in}{1.589488in}}%
\pgfpathlineto{\pgfqpoint{0.976038in}{1.631406in}}%
\pgfpathlineto{\pgfqpoint{0.976444in}{1.629704in}}%
\pgfpathlineto{\pgfqpoint{0.977394in}{1.608422in}}%
\pgfpathlineto{\pgfqpoint{0.978343in}{1.512645in}}%
\pgfpathlineto{\pgfqpoint{0.978614in}{1.191516in}}%
\pgfpathlineto{\pgfqpoint{0.979564in}{1.620655in}}%
\pgfpathlineto{\pgfqpoint{0.982683in}{1.749011in}}%
\pgfpathlineto{\pgfqpoint{0.987023in}{1.812766in}}%
\pgfpathlineto{\pgfqpoint{0.991092in}{1.835855in}}%
\pgfpathlineto{\pgfqpoint{0.993533in}{1.838450in}}%
\pgfpathlineto{\pgfqpoint{0.993669in}{1.838367in}}%
\pgfpathlineto{\pgfqpoint{0.995432in}{1.835049in}}%
\pgfpathlineto{\pgfqpoint{0.998009in}{1.822392in}}%
\pgfpathlineto{\pgfqpoint{1.001264in}{1.789696in}}%
\pgfpathlineto{\pgfqpoint{1.004519in}{1.723868in}}%
\pgfpathlineto{\pgfqpoint{1.006824in}{1.607283in}}%
\pgfpathlineto{\pgfqpoint{1.007503in}{1.473693in}}%
\pgfpathlineto{\pgfqpoint{1.007638in}{1.276708in}}%
\pgfpathlineto{\pgfqpoint{1.008859in}{1.607936in}}%
\pgfpathlineto{\pgfqpoint{1.010215in}{1.630560in}}%
\pgfpathlineto{\pgfqpoint{1.010758in}{1.626457in}}%
\pgfpathlineto{\pgfqpoint{1.011843in}{1.587345in}}%
\pgfpathlineto{\pgfqpoint{1.012521in}{1.477360in}}%
\pgfpathlineto{\pgfqpoint{1.012656in}{1.343484in}}%
\pgfpathlineto{\pgfqpoint{1.013877in}{1.639283in}}%
\pgfpathlineto{\pgfqpoint{1.017132in}{1.757069in}}%
\pgfpathlineto{\pgfqpoint{1.021472in}{1.816029in}}%
\pgfpathlineto{\pgfqpoint{1.025405in}{1.836443in}}%
\pgfpathlineto{\pgfqpoint{1.027711in}{1.838428in}}%
\pgfpathlineto{\pgfqpoint{1.027846in}{1.838344in}}%
\pgfpathlineto{\pgfqpoint{1.029609in}{1.834802in}}%
\pgfpathlineto{\pgfqpoint{1.032186in}{1.821785in}}%
\pgfpathlineto{\pgfqpoint{1.035441in}{1.788381in}}%
\pgfpathlineto{\pgfqpoint{1.038696in}{1.721080in}}%
\pgfpathlineto{\pgfqpoint{1.040866in}{1.610663in}}%
\pgfpathlineto{\pgfqpoint{1.041544in}{1.482092in}}%
\pgfpathlineto{\pgfqpoint{1.041680in}{1.339848in}}%
\pgfpathlineto{\pgfqpoint{1.042901in}{1.607367in}}%
\pgfpathlineto{\pgfqpoint{1.044393in}{1.631589in}}%
\pgfpathlineto{\pgfqpoint{1.044935in}{1.626438in}}%
\pgfpathlineto{\pgfqpoint{1.046156in}{1.572493in}}%
\pgfpathlineto{\pgfqpoint{1.046834in}{1.373628in}}%
\pgfpathlineto{\pgfqpoint{1.047376in}{1.580205in}}%
\pgfpathlineto{\pgfqpoint{1.050089in}{1.729666in}}%
\pgfpathlineto{\pgfqpoint{1.054293in}{1.803832in}}%
\pgfpathlineto{\pgfqpoint{1.058633in}{1.833992in}}%
\pgfpathlineto{\pgfqpoint{1.061481in}{1.838567in}}%
\pgfpathlineto{\pgfqpoint{1.063109in}{1.836476in}}%
\pgfpathlineto{\pgfqpoint{1.065550in}{1.826418in}}%
\pgfpathlineto{\pgfqpoint{1.068805in}{1.797745in}}%
\pgfpathlineto{\pgfqpoint{1.072196in}{1.737144in}}%
\pgfpathlineto{\pgfqpoint{1.074637in}{1.635761in}}%
\pgfpathlineto{\pgfqpoint{1.075586in}{1.487739in}}%
\pgfpathlineto{\pgfqpoint{1.075722in}{1.370085in}}%
\pgfpathlineto{\pgfqpoint{1.076943in}{1.608899in}}%
\pgfpathlineto{\pgfqpoint{1.078299in}{1.633937in}}%
\pgfpathlineto{\pgfqpoint{1.078977in}{1.629042in}}%
\pgfpathlineto{\pgfqpoint{1.080198in}{1.579030in}}%
\pgfpathlineto{\pgfqpoint{1.080876in}{1.375907in}}%
\pgfpathlineto{\pgfqpoint{1.081554in}{1.591394in}}%
\pgfpathlineto{\pgfqpoint{1.084402in}{1.735661in}}%
\pgfpathlineto{\pgfqpoint{1.088606in}{1.806426in}}%
\pgfpathlineto{\pgfqpoint{1.092811in}{1.834399in}}%
\pgfpathlineto{\pgfqpoint{1.095523in}{1.838698in}}%
\pgfpathlineto{\pgfqpoint{1.095659in}{1.838652in}}%
\pgfpathlineto{\pgfqpoint{1.097286in}{1.836299in}}%
\pgfpathlineto{\pgfqpoint{1.099728in}{1.825970in}}%
\pgfpathlineto{\pgfqpoint{1.102983in}{1.796602in}}%
\pgfpathlineto{\pgfqpoint{1.106373in}{1.734366in}}%
\pgfpathlineto{\pgfqpoint{1.108679in}{1.636517in}}%
\pgfpathlineto{\pgfqpoint{1.109628in}{1.494687in}}%
\pgfpathlineto{\pgfqpoint{1.109764in}{1.398508in}}%
\pgfpathlineto{\pgfqpoint{1.110984in}{1.608545in}}%
\pgfpathlineto{\pgfqpoint{1.112341in}{1.634556in}}%
\pgfpathlineto{\pgfqpoint{1.113019in}{1.630374in}}%
\pgfpathlineto{\pgfqpoint{1.113968in}{1.601759in}}%
\pgfpathlineto{\pgfqpoint{1.114782in}{1.500090in}}%
\pgfpathlineto{\pgfqpoint{1.115053in}{1.404098in}}%
\pgfpathlineto{\pgfqpoint{1.116003in}{1.626869in}}%
\pgfpathlineto{\pgfqpoint{1.119122in}{1.751328in}}%
\pgfpathlineto{\pgfqpoint{1.123462in}{1.813855in}}%
\pgfpathlineto{\pgfqpoint{1.127531in}{1.836144in}}%
\pgfpathlineto{\pgfqpoint{1.129836in}{1.838405in}}%
\pgfpathlineto{\pgfqpoint{1.129972in}{1.838348in}}%
\pgfpathlineto{\pgfqpoint{1.131599in}{1.835436in}}%
\pgfpathlineto{\pgfqpoint{1.134176in}{1.823421in}}%
\pgfpathlineto{\pgfqpoint{1.137567in}{1.789925in}}%
\pgfpathlineto{\pgfqpoint{1.140822in}{1.724334in}}%
\pgfpathlineto{\pgfqpoint{1.142992in}{1.619273in}}%
\pgfpathlineto{\pgfqpoint{1.143806in}{1.454319in}}%
\pgfpathlineto{\pgfqpoint{1.143941in}{1.350705in}}%
\pgfpathlineto{\pgfqpoint{1.145026in}{1.606134in}}%
\pgfpathlineto{\pgfqpoint{1.146518in}{1.633992in}}%
\pgfpathlineto{\pgfqpoint{1.147061in}{1.630558in}}%
\pgfpathlineto{\pgfqpoint{1.148146in}{1.594860in}}%
\pgfpathlineto{\pgfqpoint{1.148959in}{1.457090in}}%
\pgfpathlineto{\pgfqpoint{1.149095in}{1.356049in}}%
\pgfpathlineto{\pgfqpoint{1.150180in}{1.634544in}}%
\pgfpathlineto{\pgfqpoint{1.153299in}{1.753454in}}%
\pgfpathlineto{\pgfqpoint{1.157639in}{1.814761in}}%
\pgfpathlineto{\pgfqpoint{1.161708in}{1.836467in}}%
\pgfpathlineto{\pgfqpoint{1.163878in}{1.838553in}}%
\pgfpathlineto{\pgfqpoint{1.164149in}{1.838397in}}%
\pgfpathlineto{\pgfqpoint{1.165777in}{1.835322in}}%
\pgfpathlineto{\pgfqpoint{1.168489in}{1.821893in}}%
\pgfpathlineto{\pgfqpoint{1.171744in}{1.788689in}}%
\pgfpathlineto{\pgfqpoint{1.174999in}{1.721080in}}%
\pgfpathlineto{\pgfqpoint{1.177169in}{1.609386in}}%
\pgfpathlineto{\pgfqpoint{1.177848in}{1.478983in}}%
\pgfpathlineto{\pgfqpoint{1.177983in}{1.218992in}}%
\pgfpathlineto{\pgfqpoint{1.179204in}{1.610409in}}%
\pgfpathlineto{\pgfqpoint{1.180424in}{1.633773in}}%
\pgfpathlineto{\pgfqpoint{1.181238in}{1.629007in}}%
\pgfpathlineto{\pgfqpoint{1.182323in}{1.588204in}}%
\pgfpathlineto{\pgfqpoint{1.183001in}{1.479945in}}%
\pgfpathlineto{\pgfqpoint{1.183137in}{1.317615in}}%
\pgfpathlineto{\pgfqpoint{1.184358in}{1.641831in}}%
\pgfpathlineto{\pgfqpoint{1.187613in}{1.758249in}}%
\pgfpathlineto{\pgfqpoint{1.192088in}{1.817609in}}%
\pgfpathlineto{\pgfqpoint{1.196021in}{1.836826in}}%
\pgfpathlineto{\pgfqpoint{1.198056in}{1.838327in}}%
\pgfpathlineto{\pgfqpoint{1.198327in}{1.838066in}}%
\pgfpathlineto{\pgfqpoint{1.200361in}{1.833453in}}%
\pgfpathlineto{\pgfqpoint{1.203074in}{1.818023in}}%
\pgfpathlineto{\pgfqpoint{1.206464in}{1.778868in}}%
\pgfpathlineto{\pgfqpoint{1.209448in}{1.708900in}}%
\pgfpathlineto{\pgfqpoint{1.211347in}{1.596688in}}%
\pgfpathlineto{\pgfqpoint{1.211889in}{1.473713in}}%
\pgfpathlineto{\pgfqpoint{1.212025in}{1.210278in}}%
\pgfpathlineto{\pgfqpoint{1.213246in}{1.611395in}}%
\pgfpathlineto{\pgfqpoint{1.214738in}{1.637387in}}%
\pgfpathlineto{\pgfqpoint{1.215280in}{1.632091in}}%
\pgfpathlineto{\pgfqpoint{1.216365in}{1.595809in}}%
\pgfpathlineto{\pgfqpoint{1.217179in}{1.455342in}}%
\pgfpathlineto{\pgfqpoint{1.217314in}{1.368861in}}%
\pgfpathlineto{\pgfqpoint{1.218399in}{1.635558in}}%
\pgfpathlineto{\pgfqpoint{1.221519in}{1.753872in}}%
\pgfpathlineto{\pgfqpoint{1.225994in}{1.816142in}}%
\pgfpathlineto{\pgfqpoint{1.229928in}{1.836553in}}%
\pgfpathlineto{\pgfqpoint{1.231962in}{1.838650in}}%
\pgfpathlineto{\pgfqpoint{1.232369in}{1.838358in}}%
\pgfpathlineto{\pgfqpoint{1.233996in}{1.835264in}}%
\pgfpathlineto{\pgfqpoint{1.236573in}{1.822728in}}%
\pgfpathlineto{\pgfqpoint{1.239964in}{1.788350in}}%
\pgfpathlineto{\pgfqpoint{1.243219in}{1.720144in}}%
\pgfpathlineto{\pgfqpoint{1.245253in}{1.617859in}}%
\pgfpathlineto{\pgfqpoint{1.246203in}{1.397768in}}%
\pgfpathlineto{\pgfqpoint{1.247016in}{1.592747in}}%
\pgfpathlineto{\pgfqpoint{1.248779in}{1.636531in}}%
\pgfpathlineto{\pgfqpoint{1.249186in}{1.634430in}}%
\pgfpathlineto{\pgfqpoint{1.250271in}{1.607952in}}%
\pgfpathlineto{\pgfqpoint{1.251221in}{1.491415in}}%
\pgfpathlineto{\pgfqpoint{1.251356in}{1.400312in}}%
\pgfpathlineto{\pgfqpoint{1.252441in}{1.629531in}}%
\pgfpathlineto{\pgfqpoint{1.255561in}{1.752018in}}%
\pgfpathlineto{\pgfqpoint{1.260036in}{1.815399in}}%
\pgfpathlineto{\pgfqpoint{1.264105in}{1.836664in}}%
\pgfpathlineto{\pgfqpoint{1.266139in}{1.838542in}}%
\pgfpathlineto{\pgfqpoint{1.266546in}{1.838268in}}%
\pgfpathlineto{\pgfqpoint{1.268309in}{1.834581in}}%
\pgfpathlineto{\pgfqpoint{1.271022in}{1.820156in}}%
\pgfpathlineto{\pgfqpoint{1.274413in}{1.782584in}}%
\pgfpathlineto{\pgfqpoint{1.277532in}{1.710749in}}%
\pgfpathlineto{\pgfqpoint{1.279431in}{1.595353in}}%
\pgfpathlineto{\pgfqpoint{1.280109in}{1.367546in}}%
\pgfpathlineto{\pgfqpoint{1.280923in}{1.597648in}}%
\pgfpathlineto{\pgfqpoint{1.282821in}{1.645561in}}%
\pgfpathlineto{\pgfqpoint{1.283228in}{1.644071in}}%
\pgfpathlineto{\pgfqpoint{1.284042in}{1.631203in}}%
\pgfpathlineto{\pgfqpoint{1.285263in}{1.546709in}}%
\pgfpathlineto{\pgfqpoint{1.285669in}{1.357625in}}%
\pgfpathlineto{\pgfqpoint{1.286483in}{1.616009in}}%
\pgfpathlineto{\pgfqpoint{1.289467in}{1.747086in}}%
\pgfpathlineto{\pgfqpoint{1.293807in}{1.812690in}}%
\pgfpathlineto{\pgfqpoint{1.297876in}{1.836275in}}%
\pgfpathlineto{\pgfqpoint{1.300181in}{1.839007in}}%
\pgfpathlineto{\pgfqpoint{1.300453in}{1.838864in}}%
\pgfpathlineto{\pgfqpoint{1.302216in}{1.835615in}}%
\pgfpathlineto{\pgfqpoint{1.304793in}{1.822960in}}%
\pgfpathlineto{\pgfqpoint{1.308048in}{1.790094in}}%
\pgfpathlineto{\pgfqpoint{1.311303in}{1.722882in}}%
\pgfpathlineto{\pgfqpoint{1.313473in}{1.608382in}}%
\pgfpathlineto{\pgfqpoint{1.314286in}{1.415036in}}%
\pgfpathlineto{\pgfqpoint{1.314964in}{1.590001in}}%
\pgfpathlineto{\pgfqpoint{1.316863in}{1.646319in}}%
\pgfpathlineto{\pgfqpoint{1.317270in}{1.644723in}}%
\pgfpathlineto{\pgfqpoint{1.318219in}{1.629526in}}%
\pgfpathlineto{\pgfqpoint{1.319304in}{1.558483in}}%
\pgfpathlineto{\pgfqpoint{1.319847in}{1.412590in}}%
\pgfpathlineto{\pgfqpoint{1.320525in}{1.606201in}}%
\pgfpathlineto{\pgfqpoint{1.323373in}{1.741351in}}%
\pgfpathlineto{\pgfqpoint{1.327578in}{1.809216in}}%
\pgfpathlineto{\pgfqpoint{1.331782in}{1.835539in}}%
\pgfpathlineto{\pgfqpoint{1.334359in}{1.838939in}}%
\pgfpathlineto{\pgfqpoint{1.334494in}{1.838882in}}%
\pgfpathlineto{\pgfqpoint{1.336258in}{1.835663in}}%
\pgfpathlineto{\pgfqpoint{1.338699in}{1.824103in}}%
\pgfpathlineto{\pgfqpoint{1.342089in}{1.790109in}}%
\pgfpathlineto{\pgfqpoint{1.345209in}{1.725595in}}%
\pgfpathlineto{\pgfqpoint{1.347243in}{1.624197in}}%
\pgfpathlineto{\pgfqpoint{1.348057in}{1.408063in}}%
\pgfpathlineto{\pgfqpoint{1.348871in}{1.596862in}}%
\pgfpathlineto{\pgfqpoint{1.351041in}{1.655402in}}%
\pgfpathlineto{\pgfqpoint{1.351176in}{1.655392in}}%
\pgfpathlineto{\pgfqpoint{1.351990in}{1.648356in}}%
\pgfpathlineto{\pgfqpoint{1.353211in}{1.602886in}}%
\pgfpathlineto{\pgfqpoint{1.353889in}{1.486023in}}%
\pgfpathlineto{\pgfqpoint{1.354024in}{1.335758in}}%
\pgfpathlineto{\pgfqpoint{1.355245in}{1.649637in}}%
\pgfpathlineto{\pgfqpoint{1.358500in}{1.763789in}}%
\pgfpathlineto{\pgfqpoint{1.362840in}{1.820027in}}%
\pgfpathlineto{\pgfqpoint{1.366773in}{1.838314in}}%
\pgfpathlineto{\pgfqpoint{1.368808in}{1.839185in}}%
\pgfpathlineto{\pgfqpoint{1.368943in}{1.839082in}}%
\pgfpathlineto{\pgfqpoint{1.370706in}{1.835060in}}%
\pgfpathlineto{\pgfqpoint{1.373148in}{1.822103in}}%
\pgfpathlineto{\pgfqpoint{1.376538in}{1.785557in}}%
\pgfpathlineto{\pgfqpoint{1.379658in}{1.715092in}}%
\pgfpathlineto{\pgfqpoint{1.381556in}{1.601694in}}%
\pgfpathlineto{\pgfqpoint{1.382099in}{1.475226in}}%
\pgfpathlineto{\pgfqpoint{1.382234in}{1.237956in}}%
\pgfpathlineto{\pgfqpoint{1.383455in}{1.627427in}}%
\pgfpathlineto{\pgfqpoint{1.385083in}{1.656867in}}%
\pgfpathlineto{\pgfqpoint{1.385625in}{1.655786in}}%
\pgfpathlineto{\pgfqpoint{1.386710in}{1.635339in}}%
\pgfpathlineto{\pgfqpoint{1.387795in}{1.548468in}}%
\pgfpathlineto{\pgfqpoint{1.388202in}{1.410827in}}%
\pgfpathlineto{\pgfqpoint{1.389016in}{1.623751in}}%
\pgfpathlineto{\pgfqpoint{1.391864in}{1.748231in}}%
\pgfpathlineto{\pgfqpoint{1.396068in}{1.812948in}}%
\pgfpathlineto{\pgfqpoint{1.400137in}{1.836764in}}%
\pgfpathlineto{\pgfqpoint{1.402443in}{1.839604in}}%
\pgfpathlineto{\pgfqpoint{1.402578in}{1.839551in}}%
\pgfpathlineto{\pgfqpoint{1.404341in}{1.836516in}}%
\pgfpathlineto{\pgfqpoint{1.406647in}{1.825827in}}%
\pgfpathlineto{\pgfqpoint{1.410038in}{1.793011in}}%
\pgfpathlineto{\pgfqpoint{1.413157in}{1.730971in}}%
\pgfpathlineto{\pgfqpoint{1.415327in}{1.623992in}}%
\pgfpathlineto{\pgfqpoint{1.416005in}{1.494931in}}%
\pgfpathlineto{\pgfqpoint{1.416141in}{1.356514in}}%
\pgfpathlineto{\pgfqpoint{1.417361in}{1.627493in}}%
\pgfpathlineto{\pgfqpoint{1.419260in}{1.661908in}}%
\pgfpathlineto{\pgfqpoint{1.419667in}{1.661065in}}%
\pgfpathlineto{\pgfqpoint{1.420481in}{1.650075in}}%
\pgfpathlineto{\pgfqpoint{1.421837in}{1.578954in}}%
\pgfpathlineto{\pgfqpoint{1.422244in}{1.481167in}}%
\pgfpathlineto{\pgfqpoint{1.422379in}{1.278027in}}%
\pgfpathlineto{\pgfqpoint{1.423600in}{1.654552in}}%
\pgfpathlineto{\pgfqpoint{1.426855in}{1.766181in}}%
\pgfpathlineto{\pgfqpoint{1.431331in}{1.821742in}}%
\pgfpathlineto{\pgfqpoint{1.434993in}{1.838046in}}%
\pgfpathlineto{\pgfqpoint{1.436756in}{1.839138in}}%
\pgfpathlineto{\pgfqpoint{1.437163in}{1.838769in}}%
\pgfpathlineto{\pgfqpoint{1.438926in}{1.834737in}}%
\pgfpathlineto{\pgfqpoint{1.441774in}{1.818519in}}%
\pgfpathlineto{\pgfqpoint{1.445164in}{1.778039in}}%
\pgfpathlineto{\pgfqpoint{1.448148in}{1.703059in}}%
\pgfpathlineto{\pgfqpoint{1.449911in}{1.576087in}}%
\pgfpathlineto{\pgfqpoint{1.450318in}{1.415964in}}%
\pgfpathlineto{\pgfqpoint{1.451268in}{1.607073in}}%
\pgfpathlineto{\pgfqpoint{1.453438in}{1.656422in}}%
\pgfpathlineto{\pgfqpoint{1.453980in}{1.651741in}}%
\pgfpathlineto{\pgfqpoint{1.455201in}{1.617617in}}%
\pgfpathlineto{\pgfqpoint{1.456014in}{1.517844in}}%
\pgfpathlineto{\pgfqpoint{1.456286in}{1.332217in}}%
\pgfpathlineto{\pgfqpoint{1.457235in}{1.633338in}}%
\pgfpathlineto{\pgfqpoint{1.460354in}{1.756613in}}%
\pgfpathlineto{\pgfqpoint{1.464694in}{1.817199in}}%
\pgfpathlineto{\pgfqpoint{1.468628in}{1.837365in}}%
\pgfpathlineto{\pgfqpoint{1.470526in}{1.839152in}}%
\pgfpathlineto{\pgfqpoint{1.470933in}{1.838840in}}%
\pgfpathlineto{\pgfqpoint{1.472696in}{1.835118in}}%
\pgfpathlineto{\pgfqpoint{1.475409in}{1.820550in}}%
\pgfpathlineto{\pgfqpoint{1.478935in}{1.780126in}}%
\pgfpathlineto{\pgfqpoint{1.481783in}{1.712386in}}%
\pgfpathlineto{\pgfqpoint{1.483682in}{1.594199in}}%
\pgfpathlineto{\pgfqpoint{1.484224in}{1.418286in}}%
\pgfpathlineto{\pgfqpoint{1.485038in}{1.596713in}}%
\pgfpathlineto{\pgfqpoint{1.487208in}{1.656297in}}%
\pgfpathlineto{\pgfqpoint{1.487344in}{1.656277in}}%
\pgfpathlineto{\pgfqpoint{1.487886in}{1.652844in}}%
\pgfpathlineto{\pgfqpoint{1.489107in}{1.621439in}}%
\pgfpathlineto{\pgfqpoint{1.490056in}{1.505328in}}%
\pgfpathlineto{\pgfqpoint{1.490192in}{1.420168in}}%
\pgfpathlineto{\pgfqpoint{1.491277in}{1.637776in}}%
\pgfpathlineto{\pgfqpoint{1.494396in}{1.757379in}}%
\pgfpathlineto{\pgfqpoint{1.498601in}{1.815466in}}%
\pgfpathlineto{\pgfqpoint{1.502534in}{1.836076in}}%
\pgfpathlineto{\pgfqpoint{1.504568in}{1.837985in}}%
\pgfpathlineto{\pgfqpoint{1.504839in}{1.837764in}}%
\pgfpathlineto{\pgfqpoint{1.506738in}{1.833935in}}%
\pgfpathlineto{\pgfqpoint{1.509586in}{1.818281in}}%
\pgfpathlineto{\pgfqpoint{1.512977in}{1.779092in}}%
\pgfpathlineto{\pgfqpoint{1.516096in}{1.703306in}}%
\pgfpathlineto{\pgfqpoint{1.517859in}{1.587687in}}%
\pgfpathlineto{\pgfqpoint{1.518402in}{1.359338in}}%
\pgfpathlineto{\pgfqpoint{1.519216in}{1.591831in}}%
\pgfpathlineto{\pgfqpoint{1.521250in}{1.645517in}}%
\pgfpathlineto{\pgfqpoint{1.521386in}{1.645204in}}%
\pgfpathlineto{\pgfqpoint{1.522471in}{1.627972in}}%
\pgfpathlineto{\pgfqpoint{1.523556in}{1.548609in}}%
\pgfpathlineto{\pgfqpoint{1.523963in}{1.358077in}}%
\pgfpathlineto{\pgfqpoint{1.524776in}{1.615153in}}%
\pgfpathlineto{\pgfqpoint{1.527760in}{1.746646in}}%
\pgfpathlineto{\pgfqpoint{1.532100in}{1.812108in}}%
\pgfpathlineto{\pgfqpoint{1.536169in}{1.835482in}}%
\pgfpathlineto{\pgfqpoint{1.538610in}{1.838141in}}%
\pgfpathlineto{\pgfqpoint{1.538746in}{1.838075in}}%
\pgfpathlineto{\pgfqpoint{1.540238in}{1.835528in}}%
\pgfpathlineto{\pgfqpoint{1.542814in}{1.823759in}}%
\pgfpathlineto{\pgfqpoint{1.545934in}{1.794224in}}%
\pgfpathlineto{\pgfqpoint{1.549053in}{1.735886in}}%
\pgfpathlineto{\pgfqpoint{1.551359in}{1.632768in}}%
\pgfpathlineto{\pgfqpoint{1.552308in}{1.406247in}}%
\pgfpathlineto{\pgfqpoint{1.553122in}{1.594769in}}%
\pgfpathlineto{\pgfqpoint{1.555292in}{1.654069in}}%
\pgfpathlineto{\pgfqpoint{1.555699in}{1.653013in}}%
\pgfpathlineto{\pgfqpoint{1.556784in}{1.633938in}}%
\pgfpathlineto{\pgfqpoint{1.557869in}{1.557549in}}%
\pgfpathlineto{\pgfqpoint{1.558276in}{1.285194in}}%
\pgfpathlineto{\pgfqpoint{1.559089in}{1.616771in}}%
\pgfpathlineto{\pgfqpoint{1.561938in}{1.744954in}}%
\pgfpathlineto{\pgfqpoint{1.566142in}{1.810528in}}%
\pgfpathlineto{\pgfqpoint{1.570346in}{1.835267in}}%
\pgfpathlineto{\pgfqpoint{1.572516in}{1.837904in}}%
\pgfpathlineto{\pgfqpoint{1.572923in}{1.837593in}}%
\pgfpathlineto{\pgfqpoint{1.574686in}{1.834242in}}%
\pgfpathlineto{\pgfqpoint{1.577534in}{1.819201in}}%
\pgfpathlineto{\pgfqpoint{1.580789in}{1.783087in}}%
\pgfpathlineto{\pgfqpoint{1.584044in}{1.708145in}}%
\pgfpathlineto{\pgfqpoint{1.585943in}{1.591311in}}%
\pgfpathlineto{\pgfqpoint{1.586621in}{1.393114in}}%
\pgfpathlineto{\pgfqpoint{1.587435in}{1.595349in}}%
\pgfpathlineto{\pgfqpoint{1.589334in}{1.641670in}}%
\pgfpathlineto{\pgfqpoint{1.589741in}{1.639023in}}%
\pgfpathlineto{\pgfqpoint{1.590554in}{1.622374in}}%
\pgfpathlineto{\pgfqpoint{1.591775in}{1.507910in}}%
\pgfpathlineto{\pgfqpoint{1.592046in}{1.397828in}}%
\pgfpathlineto{\pgfqpoint{1.592996in}{1.627982in}}%
\pgfpathlineto{\pgfqpoint{1.596115in}{1.751944in}}%
\pgfpathlineto{\pgfqpoint{1.600455in}{1.814037in}}%
\pgfpathlineto{\pgfqpoint{1.604388in}{1.835544in}}%
\pgfpathlineto{\pgfqpoint{1.606423in}{1.838038in}}%
\pgfpathlineto{\pgfqpoint{1.606965in}{1.837775in}}%
\pgfpathlineto{\pgfqpoint{1.608593in}{1.834654in}}%
\pgfpathlineto{\pgfqpoint{1.611305in}{1.821197in}}%
\pgfpathlineto{\pgfqpoint{1.614831in}{1.783434in}}%
\pgfpathlineto{\pgfqpoint{1.617951in}{1.713675in}}%
\pgfpathlineto{\pgfqpoint{1.619985in}{1.589880in}}%
\pgfpathlineto{\pgfqpoint{1.620528in}{1.367945in}}%
\pgfpathlineto{\pgfqpoint{1.621341in}{1.593114in}}%
\pgfpathlineto{\pgfqpoint{1.623511in}{1.649744in}}%
\pgfpathlineto{\pgfqpoint{1.623918in}{1.648282in}}%
\pgfpathlineto{\pgfqpoint{1.625003in}{1.625498in}}%
\pgfpathlineto{\pgfqpoint{1.626088in}{1.519385in}}%
\pgfpathlineto{\pgfqpoint{1.626359in}{1.339818in}}%
\pgfpathlineto{\pgfqpoint{1.627309in}{1.630144in}}%
\pgfpathlineto{\pgfqpoint{1.630293in}{1.750576in}}%
\pgfpathlineto{\pgfqpoint{1.634633in}{1.813572in}}%
\pgfpathlineto{\pgfqpoint{1.638701in}{1.835399in}}%
\pgfpathlineto{\pgfqpoint{1.640600in}{1.837436in}}%
\pgfpathlineto{\pgfqpoint{1.641007in}{1.837168in}}%
\pgfpathlineto{\pgfqpoint{1.642634in}{1.834241in}}%
\pgfpathlineto{\pgfqpoint{1.645347in}{1.820650in}}%
\pgfpathlineto{\pgfqpoint{1.648738in}{1.784408in}}%
\pgfpathlineto{\pgfqpoint{1.651993in}{1.711174in}}%
\pgfpathlineto{\pgfqpoint{1.653891in}{1.598399in}}%
\pgfpathlineto{\pgfqpoint{1.654434in}{1.475268in}}%
\pgfpathlineto{\pgfqpoint{1.654569in}{1.226634in}}%
\pgfpathlineto{\pgfqpoint{1.655790in}{1.616240in}}%
\pgfpathlineto{\pgfqpoint{1.657418in}{1.645529in}}%
\pgfpathlineto{\pgfqpoint{1.657824in}{1.642972in}}%
\pgfpathlineto{\pgfqpoint{1.658909in}{1.617829in}}%
\pgfpathlineto{\pgfqpoint{1.659859in}{1.522390in}}%
\pgfpathlineto{\pgfqpoint{1.660130in}{1.306215in}}%
\pgfpathlineto{\pgfqpoint{1.661079in}{1.625088in}}%
\pgfpathlineto{\pgfqpoint{1.664063in}{1.749221in}}%
\pgfpathlineto{\pgfqpoint{1.668539in}{1.814746in}}%
\pgfpathlineto{\pgfqpoint{1.672472in}{1.836361in}}%
\pgfpathlineto{\pgfqpoint{1.674913in}{1.838743in}}%
\pgfpathlineto{\pgfqpoint{1.675049in}{1.838682in}}%
\pgfpathlineto{\pgfqpoint{1.676676in}{1.835591in}}%
\pgfpathlineto{\pgfqpoint{1.679118in}{1.824039in}}%
\pgfpathlineto{\pgfqpoint{1.682508in}{1.790023in}}%
\pgfpathlineto{\pgfqpoint{1.685492in}{1.729323in}}%
\pgfpathlineto{\pgfqpoint{1.687526in}{1.634717in}}%
\pgfpathlineto{\pgfqpoint{1.688340in}{1.499942in}}%
\pgfpathlineto{\pgfqpoint{1.688476in}{1.395336in}}%
\pgfpathlineto{\pgfqpoint{1.689696in}{1.624421in}}%
\pgfpathlineto{\pgfqpoint{1.691595in}{1.658901in}}%
\pgfpathlineto{\pgfqpoint{1.692002in}{1.656790in}}%
\pgfpathlineto{\pgfqpoint{1.693223in}{1.632505in}}%
\pgfpathlineto{\pgfqpoint{1.694308in}{1.527656in}}%
\pgfpathlineto{\pgfqpoint{1.694579in}{1.316813in}}%
\pgfpathlineto{\pgfqpoint{1.695528in}{1.633107in}}%
\pgfpathlineto{\pgfqpoint{1.698512in}{1.753353in}}%
\pgfpathlineto{\pgfqpoint{1.702716in}{1.814423in}}%
\pgfpathlineto{\pgfqpoint{1.706785in}{1.836587in}}%
\pgfpathlineto{\pgfqpoint{1.708955in}{1.838712in}}%
\pgfpathlineto{\pgfqpoint{1.709226in}{1.838536in}}%
\pgfpathlineto{\pgfqpoint{1.711125in}{1.834530in}}%
\pgfpathlineto{\pgfqpoint{1.714109in}{1.817488in}}%
\pgfpathlineto{\pgfqpoint{1.717499in}{1.776593in}}%
\pgfpathlineto{\pgfqpoint{1.720619in}{1.694029in}}%
\pgfpathlineto{\pgfqpoint{1.722111in}{1.582770in}}%
\pgfpathlineto{\pgfqpoint{1.722653in}{1.388709in}}%
\pgfpathlineto{\pgfqpoint{1.723467in}{1.607658in}}%
\pgfpathlineto{\pgfqpoint{1.725637in}{1.658869in}}%
\pgfpathlineto{\pgfqpoint{1.725773in}{1.658759in}}%
\pgfpathlineto{\pgfqpoint{1.726586in}{1.651129in}}%
\pgfpathlineto{\pgfqpoint{1.728078in}{1.585891in}}%
\pgfpathlineto{\pgfqpoint{1.728621in}{1.401408in}}%
\pgfpathlineto{\pgfqpoint{1.729299in}{1.599584in}}%
\pgfpathlineto{\pgfqpoint{1.732011in}{1.739328in}}%
\pgfpathlineto{\pgfqpoint{1.736080in}{1.808548in}}%
\pgfpathlineto{\pgfqpoint{1.740284in}{1.836777in}}%
\pgfpathlineto{\pgfqpoint{1.742997in}{1.840713in}}%
\pgfpathlineto{\pgfqpoint{1.744489in}{1.838814in}}%
\pgfpathlineto{\pgfqpoint{1.747066in}{1.828077in}}%
\pgfpathlineto{\pgfqpoint{1.750321in}{1.798033in}}%
\pgfpathlineto{\pgfqpoint{1.753576in}{1.735829in}}%
\pgfpathlineto{\pgfqpoint{1.755746in}{1.634531in}}%
\pgfpathlineto{\pgfqpoint{1.756559in}{1.466526in}}%
\pgfpathlineto{\pgfqpoint{1.756695in}{1.381240in}}%
\pgfpathlineto{\pgfqpoint{1.757780in}{1.626438in}}%
\pgfpathlineto{\pgfqpoint{1.759950in}{1.666986in}}%
\pgfpathlineto{\pgfqpoint{1.760086in}{1.666863in}}%
\pgfpathlineto{\pgfqpoint{1.761171in}{1.651775in}}%
\pgfpathlineto{\pgfqpoint{1.762391in}{1.583865in}}%
\pgfpathlineto{\pgfqpoint{1.762934in}{1.340663in}}%
\pgfpathlineto{\pgfqpoint{1.763612in}{1.609507in}}%
\pgfpathlineto{\pgfqpoint{1.766324in}{1.742941in}}%
\pgfpathlineto{\pgfqpoint{1.770529in}{1.811511in}}%
\pgfpathlineto{\pgfqpoint{1.774598in}{1.836470in}}%
\pgfpathlineto{\pgfqpoint{1.777039in}{1.839733in}}%
\pgfpathlineto{\pgfqpoint{1.777174in}{1.839654in}}%
\pgfpathlineto{\pgfqpoint{1.779209in}{1.835714in}}%
\pgfpathlineto{\pgfqpoint{1.782328in}{1.817971in}}%
\pgfpathlineto{\pgfqpoint{1.785854in}{1.773877in}}%
\pgfpathlineto{\pgfqpoint{1.788838in}{1.691657in}}%
\pgfpathlineto{\pgfqpoint{1.790330in}{1.568082in}}%
\pgfpathlineto{\pgfqpoint{1.790737in}{1.276117in}}%
\pgfpathlineto{\pgfqpoint{1.791551in}{1.606860in}}%
\pgfpathlineto{\pgfqpoint{1.793721in}{1.661337in}}%
\pgfpathlineto{\pgfqpoint{1.793992in}{1.660973in}}%
\pgfpathlineto{\pgfqpoint{1.794806in}{1.654146in}}%
\pgfpathlineto{\pgfqpoint{1.796026in}{1.607328in}}%
\pgfpathlineto{\pgfqpoint{1.796704in}{1.506672in}}%
\pgfpathlineto{\pgfqpoint{1.796840in}{1.417321in}}%
\pgfpathlineto{\pgfqpoint{1.797925in}{1.641191in}}%
\pgfpathlineto{\pgfqpoint{1.801044in}{1.760680in}}%
\pgfpathlineto{\pgfqpoint{1.805384in}{1.820119in}}%
\pgfpathlineto{\pgfqpoint{1.809453in}{1.840163in}}%
\pgfpathlineto{\pgfqpoint{1.811352in}{1.841463in}}%
\pgfpathlineto{\pgfqpoint{1.811894in}{1.840926in}}%
\pgfpathlineto{\pgfqpoint{1.813793in}{1.836066in}}%
\pgfpathlineto{\pgfqpoint{1.816777in}{1.817152in}}%
\pgfpathlineto{\pgfqpoint{1.819896in}{1.776863in}}%
\pgfpathlineto{\pgfqpoint{1.822744in}{1.699889in}}%
\pgfpathlineto{\pgfqpoint{1.824372in}{1.565608in}}%
\pgfpathlineto{\pgfqpoint{1.824779in}{1.412207in}}%
\pgfpathlineto{\pgfqpoint{1.825593in}{1.612389in}}%
\pgfpathlineto{\pgfqpoint{1.827898in}{1.671393in}}%
\pgfpathlineto{\pgfqpoint{1.828034in}{1.671518in}}%
\pgfpathlineto{\pgfqpoint{1.828305in}{1.670338in}}%
\pgfpathlineto{\pgfqpoint{1.829661in}{1.647620in}}%
\pgfpathlineto{\pgfqpoint{1.830882in}{1.547379in}}%
\pgfpathlineto{\pgfqpoint{1.831153in}{1.421167in}}%
\pgfpathlineto{\pgfqpoint{1.832103in}{1.635019in}}%
\pgfpathlineto{\pgfqpoint{1.835086in}{1.757189in}}%
\pgfpathlineto{\pgfqpoint{1.839291in}{1.817469in}}%
\pgfpathlineto{\pgfqpoint{1.843224in}{1.838637in}}%
\pgfpathlineto{\pgfqpoint{1.845394in}{1.840673in}}%
\pgfpathlineto{\pgfqpoint{1.845665in}{1.840481in}}%
\pgfpathlineto{\pgfqpoint{1.847428in}{1.836744in}}%
\pgfpathlineto{\pgfqpoint{1.850276in}{1.820893in}}%
\pgfpathlineto{\pgfqpoint{1.853531in}{1.782793in}}%
\pgfpathlineto{\pgfqpoint{1.856379in}{1.713042in}}%
\pgfpathlineto{\pgfqpoint{1.858143in}{1.596778in}}%
\pgfpathlineto{\pgfqpoint{1.858549in}{1.486850in}}%
\pgfpathlineto{\pgfqpoint{1.858685in}{1.202810in}}%
\pgfpathlineto{\pgfqpoint{1.859906in}{1.637565in}}%
\pgfpathlineto{\pgfqpoint{1.861804in}{1.674563in}}%
\pgfpathlineto{\pgfqpoint{1.862347in}{1.673819in}}%
\pgfpathlineto{\pgfqpoint{1.863161in}{1.665516in}}%
\pgfpathlineto{\pgfqpoint{1.864653in}{1.599760in}}%
\pgfpathlineto{\pgfqpoint{1.865331in}{1.384823in}}%
\pgfpathlineto{\pgfqpoint{1.866009in}{1.614920in}}%
\pgfpathlineto{\pgfqpoint{1.868721in}{1.747849in}}%
\pgfpathlineto{\pgfqpoint{1.872926in}{1.814750in}}%
\pgfpathlineto{\pgfqpoint{1.876859in}{1.838964in}}%
\pgfpathlineto{\pgfqpoint{1.879436in}{1.842178in}}%
\pgfpathlineto{\pgfqpoint{1.880928in}{1.839933in}}%
\pgfpathlineto{\pgfqpoint{1.883369in}{1.829486in}}%
\pgfpathlineto{\pgfqpoint{1.886488in}{1.800988in}}%
\pgfpathlineto{\pgfqpoint{1.889743in}{1.740373in}}%
\pgfpathlineto{\pgfqpoint{1.892049in}{1.631815in}}%
\pgfpathlineto{\pgfqpoint{1.892863in}{1.435821in}}%
\pgfpathlineto{\pgfqpoint{1.893676in}{1.602162in}}%
\pgfpathlineto{\pgfqpoint{1.895982in}{1.670416in}}%
\pgfpathlineto{\pgfqpoint{1.896253in}{1.670851in}}%
\pgfpathlineto{\pgfqpoint{1.896660in}{1.668389in}}%
\pgfpathlineto{\pgfqpoint{1.897881in}{1.645443in}}%
\pgfpathlineto{\pgfqpoint{1.898966in}{1.562033in}}%
\pgfpathlineto{\pgfqpoint{1.899373in}{1.334560in}}%
\pgfpathlineto{\pgfqpoint{1.900186in}{1.627464in}}%
\pgfpathlineto{\pgfqpoint{1.903034in}{1.751894in}}%
\pgfpathlineto{\pgfqpoint{1.907103in}{1.814307in}}%
\pgfpathlineto{\pgfqpoint{1.911308in}{1.838499in}}%
\pgfpathlineto{\pgfqpoint{1.913478in}{1.840880in}}%
\pgfpathlineto{\pgfqpoint{1.913749in}{1.840743in}}%
\pgfpathlineto{\pgfqpoint{1.915512in}{1.837191in}}%
\pgfpathlineto{\pgfqpoint{1.918224in}{1.822628in}}%
\pgfpathlineto{\pgfqpoint{1.921479in}{1.786180in}}%
\pgfpathlineto{\pgfqpoint{1.924192in}{1.726005in}}%
\pgfpathlineto{\pgfqpoint{1.926226in}{1.607935in}}%
\pgfpathlineto{\pgfqpoint{1.926904in}{1.328682in}}%
\pgfpathlineto{\pgfqpoint{1.927583in}{1.604984in}}%
\pgfpathlineto{\pgfqpoint{1.929888in}{1.673883in}}%
\pgfpathlineto{\pgfqpoint{1.930159in}{1.674534in}}%
\pgfpathlineto{\pgfqpoint{1.930702in}{1.671780in}}%
\pgfpathlineto{\pgfqpoint{1.931651in}{1.657046in}}%
\pgfpathlineto{\pgfqpoint{1.933008in}{1.573696in}}%
\pgfpathlineto{\pgfqpoint{1.933414in}{1.414800in}}%
\pgfpathlineto{\pgfqpoint{1.934228in}{1.623679in}}%
\pgfpathlineto{\pgfqpoint{1.937212in}{1.756022in}}%
\pgfpathlineto{\pgfqpoint{1.941688in}{1.819573in}}%
\pgfpathlineto{\pgfqpoint{1.945485in}{1.839284in}}%
\pgfpathlineto{\pgfqpoint{1.947248in}{1.841238in}}%
\pgfpathlineto{\pgfqpoint{1.947926in}{1.840778in}}%
\pgfpathlineto{\pgfqpoint{1.949418in}{1.837691in}}%
\pgfpathlineto{\pgfqpoint{1.952402in}{1.822056in}}%
\pgfpathlineto{\pgfqpoint{1.956199in}{1.776917in}}%
\pgfpathlineto{\pgfqpoint{1.958912in}{1.710218in}}%
\pgfpathlineto{\pgfqpoint{1.960811in}{1.588416in}}%
\pgfpathlineto{\pgfqpoint{1.961218in}{1.479859in}}%
\pgfpathlineto{\pgfqpoint{1.961353in}{1.198777in}}%
\pgfpathlineto{\pgfqpoint{1.962574in}{1.621402in}}%
\pgfpathlineto{\pgfqpoint{1.964201in}{1.650629in}}%
\pgfpathlineto{\pgfqpoint{1.964608in}{1.649531in}}%
\pgfpathlineto{\pgfqpoint{1.965422in}{1.634538in}}%
\pgfpathlineto{\pgfqpoint{1.966371in}{1.588002in}}%
\pgfpathlineto{\pgfqpoint{1.967049in}{1.375637in}}%
\pgfpathlineto{\pgfqpoint{1.967728in}{1.596097in}}%
\pgfpathlineto{\pgfqpoint{1.970711in}{1.744486in}}%
\pgfpathlineto{\pgfqpoint{1.975594in}{1.817010in}}%
\pgfpathlineto{\pgfqpoint{1.979798in}{1.837806in}}%
\pgfpathlineto{\pgfqpoint{1.981561in}{1.839251in}}%
\pgfpathlineto{\pgfqpoint{1.981968in}{1.838870in}}%
\pgfpathlineto{\pgfqpoint{1.983867in}{1.834673in}}%
\pgfpathlineto{\pgfqpoint{1.987393in}{1.812212in}}%
\pgfpathlineto{\pgfqpoint{1.990648in}{1.768312in}}%
\pgfpathlineto{\pgfqpoint{1.993496in}{1.687056in}}%
\pgfpathlineto{\pgfqpoint{1.994988in}{1.558311in}}%
\pgfpathlineto{\pgfqpoint{1.995395in}{1.161537in}}%
\pgfpathlineto{\pgfqpoint{1.996209in}{1.603017in}}%
\pgfpathlineto{\pgfqpoint{1.998108in}{1.653081in}}%
\pgfpathlineto{\pgfqpoint{1.998514in}{1.649509in}}%
\pgfpathlineto{\pgfqpoint{2.000278in}{1.599435in}}%
\pgfpathlineto{\pgfqpoint{2.000956in}{1.481216in}}%
\pgfpathlineto{\pgfqpoint{2.001091in}{1.361554in}}%
\pgfpathlineto{\pgfqpoint{2.002176in}{1.639957in}}%
\pgfpathlineto{\pgfqpoint{2.005431in}{1.760374in}}%
\pgfpathlineto{\pgfqpoint{2.010043in}{1.820543in}}%
\pgfpathlineto{\pgfqpoint{2.013840in}{1.838116in}}%
\pgfpathlineto{\pgfqpoint{2.015874in}{1.839238in}}%
\pgfpathlineto{\pgfqpoint{2.016010in}{1.839135in}}%
\pgfpathlineto{\pgfqpoint{2.018316in}{1.833325in}}%
\pgfpathlineto{\pgfqpoint{2.021571in}{1.811760in}}%
\pgfpathlineto{\pgfqpoint{2.024961in}{1.765999in}}%
\pgfpathlineto{\pgfqpoint{2.027809in}{1.685483in}}%
\pgfpathlineto{\pgfqpoint{2.029437in}{1.557943in}}%
\pgfpathlineto{\pgfqpoint{2.029844in}{1.371233in}}%
\pgfpathlineto{\pgfqpoint{2.030793in}{1.590736in}}%
\pgfpathlineto{\pgfqpoint{2.032692in}{1.624333in}}%
\pgfpathlineto{\pgfqpoint{2.032963in}{1.619604in}}%
\pgfpathlineto{\pgfqpoint{2.034455in}{1.496980in}}%
\pgfpathlineto{\pgfqpoint{2.034726in}{1.288864in}}%
\pgfpathlineto{\pgfqpoint{2.035540in}{1.611865in}}%
\pgfpathlineto{\pgfqpoint{2.039202in}{1.757167in}}%
\pgfpathlineto{\pgfqpoint{2.043678in}{1.816904in}}%
\pgfpathlineto{\pgfqpoint{2.047611in}{1.836510in}}%
\pgfpathlineto{\pgfqpoint{2.049781in}{1.838081in}}%
\pgfpathlineto{\pgfqpoint{2.050459in}{1.837397in}}%
\pgfpathlineto{\pgfqpoint{2.052493in}{1.831384in}}%
\pgfpathlineto{\pgfqpoint{2.055477in}{1.811348in}}%
\pgfpathlineto{\pgfqpoint{2.058868in}{1.765492in}}%
\pgfpathlineto{\pgfqpoint{2.061851in}{1.679099in}}%
\pgfpathlineto{\pgfqpoint{2.063479in}{1.522503in}}%
\pgfpathlineto{\pgfqpoint{2.063750in}{1.299477in}}%
\pgfpathlineto{\pgfqpoint{2.064699in}{1.598697in}}%
\pgfpathlineto{\pgfqpoint{2.066191in}{1.637587in}}%
\pgfpathlineto{\pgfqpoint{2.066734in}{1.631320in}}%
\pgfpathlineto{\pgfqpoint{2.067954in}{1.598067in}}%
\pgfpathlineto{\pgfqpoint{2.068633in}{1.510885in}}%
\pgfpathlineto{\pgfqpoint{2.068904in}{1.351182in}}%
\pgfpathlineto{\pgfqpoint{2.069853in}{1.620282in}}%
\pgfpathlineto{\pgfqpoint{2.072701in}{1.742241in}}%
\pgfpathlineto{\pgfqpoint{2.077041in}{1.810114in}}%
\pgfpathlineto{\pgfqpoint{2.081381in}{1.835312in}}%
\pgfpathlineto{\pgfqpoint{2.083823in}{1.838284in}}%
\pgfpathlineto{\pgfqpoint{2.083958in}{1.838230in}}%
\pgfpathlineto{\pgfqpoint{2.086264in}{1.833133in}}%
\pgfpathlineto{\pgfqpoint{2.088841in}{1.818853in}}%
\pgfpathlineto{\pgfqpoint{2.092503in}{1.776569in}}%
\pgfpathlineto{\pgfqpoint{2.095486in}{1.705750in}}%
\pgfpathlineto{\pgfqpoint{2.097385in}{1.591729in}}%
\pgfpathlineto{\pgfqpoint{2.098063in}{1.385249in}}%
\pgfpathlineto{\pgfqpoint{2.098877in}{1.581669in}}%
\pgfpathlineto{\pgfqpoint{2.100776in}{1.623896in}}%
\pgfpathlineto{\pgfqpoint{2.101047in}{1.621654in}}%
\pgfpathlineto{\pgfqpoint{2.101996in}{1.587749in}}%
\pgfpathlineto{\pgfqpoint{2.102810in}{1.463214in}}%
\pgfpathlineto{\pgfqpoint{2.102946in}{1.235339in}}%
\pgfpathlineto{\pgfqpoint{2.104166in}{1.640796in}}%
\pgfpathlineto{\pgfqpoint{2.107693in}{1.762825in}}%
\pgfpathlineto{\pgfqpoint{2.112033in}{1.818966in}}%
\pgfpathlineto{\pgfqpoint{2.115694in}{1.837178in}}%
\pgfpathlineto{\pgfqpoint{2.117729in}{1.839152in}}%
\pgfpathlineto{\pgfqpoint{2.118271in}{1.838684in}}%
\pgfpathlineto{\pgfqpoint{2.119763in}{1.835581in}}%
\pgfpathlineto{\pgfqpoint{2.121933in}{1.825665in}}%
\pgfpathlineto{\pgfqpoint{2.125324in}{1.793228in}}%
\pgfpathlineto{\pgfqpoint{2.128579in}{1.728713in}}%
\pgfpathlineto{\pgfqpoint{2.130884in}{1.606549in}}%
\pgfpathlineto{\pgfqpoint{2.131563in}{1.410801in}}%
\pgfpathlineto{\pgfqpoint{2.132376in}{1.594072in}}%
\pgfpathlineto{\pgfqpoint{2.134411in}{1.652092in}}%
\pgfpathlineto{\pgfqpoint{2.134546in}{1.651571in}}%
\pgfpathlineto{\pgfqpoint{2.135631in}{1.639554in}}%
\pgfpathlineto{\pgfqpoint{2.136716in}{1.587157in}}%
\pgfpathlineto{\pgfqpoint{2.137394in}{1.409150in}}%
\pgfpathlineto{\pgfqpoint{2.138073in}{1.603102in}}%
\pgfpathlineto{\pgfqpoint{2.140649in}{1.735174in}}%
\pgfpathlineto{\pgfqpoint{2.145396in}{1.812468in}}%
\pgfpathlineto{\pgfqpoint{2.149329in}{1.835846in}}%
\pgfpathlineto{\pgfqpoint{2.151906in}{1.839384in}}%
\pgfpathlineto{\pgfqpoint{2.152178in}{1.839239in}}%
\pgfpathlineto{\pgfqpoint{2.153941in}{1.835858in}}%
\pgfpathlineto{\pgfqpoint{2.157060in}{1.819090in}}%
\pgfpathlineto{\pgfqpoint{2.160858in}{1.773557in}}%
\pgfpathlineto{\pgfqpoint{2.164113in}{1.684668in}}%
\pgfpathlineto{\pgfqpoint{2.165740in}{1.543104in}}%
\pgfpathlineto{\pgfqpoint{2.166011in}{1.409990in}}%
\pgfpathlineto{\pgfqpoint{2.167096in}{1.599378in}}%
\pgfpathlineto{\pgfqpoint{2.168181in}{1.634163in}}%
\pgfpathlineto{\pgfqpoint{2.169266in}{1.632881in}}%
\pgfpathlineto{\pgfqpoint{2.170351in}{1.593896in}}%
\pgfpathlineto{\pgfqpoint{2.171165in}{1.454874in}}%
\pgfpathlineto{\pgfqpoint{2.171301in}{1.344519in}}%
\pgfpathlineto{\pgfqpoint{2.172386in}{1.635968in}}%
\pgfpathlineto{\pgfqpoint{2.175505in}{1.754639in}}%
\pgfpathlineto{\pgfqpoint{2.179845in}{1.815550in}}%
\pgfpathlineto{\pgfqpoint{2.183643in}{1.836561in}}%
\pgfpathlineto{\pgfqpoint{2.185948in}{1.839317in}}%
\pgfpathlineto{\pgfqpoint{2.186084in}{1.839202in}}%
\pgfpathlineto{\pgfqpoint{2.188661in}{1.833397in}}%
\pgfpathlineto{\pgfqpoint{2.191102in}{1.819031in}}%
\pgfpathlineto{\pgfqpoint{2.194628in}{1.777407in}}%
\pgfpathlineto{\pgfqpoint{2.197612in}{1.703378in}}%
\pgfpathlineto{\pgfqpoint{2.199511in}{1.556902in}}%
\pgfpathlineto{\pgfqpoint{2.199918in}{1.317519in}}%
\pgfpathlineto{\pgfqpoint{2.200731in}{1.601342in}}%
\pgfpathlineto{\pgfqpoint{2.202766in}{1.654323in}}%
\pgfpathlineto{\pgfqpoint{2.203037in}{1.652151in}}%
\pgfpathlineto{\pgfqpoint{2.204936in}{1.598582in}}%
\pgfpathlineto{\pgfqpoint{2.205749in}{1.336261in}}%
\pgfpathlineto{\pgfqpoint{2.206292in}{1.592981in}}%
\pgfpathlineto{\pgfqpoint{2.209004in}{1.735453in}}%
\pgfpathlineto{\pgfqpoint{2.213073in}{1.806059in}}%
\pgfpathlineto{\pgfqpoint{2.217413in}{1.835366in}}%
\pgfpathlineto{\pgfqpoint{2.220397in}{1.839278in}}%
\pgfpathlineto{\pgfqpoint{2.220533in}{1.839214in}}%
\pgfpathlineto{\pgfqpoint{2.222567in}{1.834691in}}%
\pgfpathlineto{\pgfqpoint{2.225822in}{1.814578in}}%
\pgfpathlineto{\pgfqpoint{2.229619in}{1.761562in}}%
\pgfpathlineto{\pgfqpoint{2.232332in}{1.678238in}}%
\pgfpathlineto{\pgfqpoint{2.233688in}{1.553600in}}%
\pgfpathlineto{\pgfqpoint{2.233959in}{1.421030in}}%
\pgfpathlineto{\pgfqpoint{2.235044in}{1.609853in}}%
\pgfpathlineto{\pgfqpoint{2.237079in}{1.648208in}}%
\pgfpathlineto{\pgfqpoint{2.237350in}{1.646858in}}%
\pgfpathlineto{\pgfqpoint{2.238706in}{1.602723in}}%
\pgfpathlineto{\pgfqpoint{2.239520in}{1.492173in}}%
\pgfpathlineto{\pgfqpoint{2.239656in}{1.314188in}}%
\pgfpathlineto{\pgfqpoint{2.240876in}{1.650034in}}%
\pgfpathlineto{\pgfqpoint{2.244403in}{1.768448in}}%
\pgfpathlineto{\pgfqpoint{2.248878in}{1.823035in}}%
\pgfpathlineto{\pgfqpoint{2.253083in}{1.839508in}}%
\pgfpathlineto{\pgfqpoint{2.255117in}{1.839149in}}%
\pgfpathlineto{\pgfqpoint{2.257829in}{1.829405in}}%
\pgfpathlineto{\pgfqpoint{2.260949in}{1.803702in}}%
\pgfpathlineto{\pgfqpoint{2.264339in}{1.747043in}}%
\pgfpathlineto{\pgfqpoint{2.266916in}{1.647217in}}%
\pgfpathlineto{\pgfqpoint{2.267866in}{1.480058in}}%
\pgfpathlineto{\pgfqpoint{2.268001in}{1.364844in}}%
\pgfpathlineto{\pgfqpoint{2.269222in}{1.622454in}}%
\pgfpathlineto{\pgfqpoint{2.270849in}{1.658115in}}%
\pgfpathlineto{\pgfqpoint{2.271392in}{1.656384in}}%
\pgfpathlineto{\pgfqpoint{2.271663in}{1.655500in}}%
\pgfpathlineto{\pgfqpoint{2.273019in}{1.615853in}}%
\pgfpathlineto{\pgfqpoint{2.273969in}{1.432545in}}%
\pgfpathlineto{\pgfqpoint{2.274647in}{1.595588in}}%
\pgfpathlineto{\pgfqpoint{2.277359in}{1.738295in}}%
\pgfpathlineto{\pgfqpoint{2.281971in}{1.812887in}}%
\pgfpathlineto{\pgfqpoint{2.285904in}{1.835871in}}%
\pgfpathlineto{\pgfqpoint{2.288481in}{1.839097in}}%
\pgfpathlineto{\pgfqpoint{2.289837in}{1.836907in}}%
\pgfpathlineto{\pgfqpoint{2.292143in}{1.827624in}}%
\pgfpathlineto{\pgfqpoint{2.295533in}{1.797757in}}%
\pgfpathlineto{\pgfqpoint{2.299195in}{1.726695in}}%
\pgfpathlineto{\pgfqpoint{2.301229in}{1.627865in}}%
\pgfpathlineto{\pgfqpoint{2.302043in}{1.494809in}}%
\pgfpathlineto{\pgfqpoint{2.302179in}{1.385909in}}%
\pgfpathlineto{\pgfqpoint{2.303399in}{1.614804in}}%
\pgfpathlineto{\pgfqpoint{2.305163in}{1.646820in}}%
\pgfpathlineto{\pgfqpoint{2.305705in}{1.640881in}}%
\pgfpathlineto{\pgfqpoint{2.306926in}{1.594854in}}%
\pgfpathlineto{\pgfqpoint{2.307604in}{1.493861in}}%
\pgfpathlineto{\pgfqpoint{2.307739in}{1.377576in}}%
\pgfpathlineto{\pgfqpoint{2.308960in}{1.648008in}}%
\pgfpathlineto{\pgfqpoint{2.312215in}{1.762401in}}%
\pgfpathlineto{\pgfqpoint{2.316555in}{1.819382in}}%
\pgfpathlineto{\pgfqpoint{2.320217in}{1.837560in}}%
\pgfpathlineto{\pgfqpoint{2.322116in}{1.839618in}}%
\pgfpathlineto{\pgfqpoint{2.322794in}{1.839101in}}%
\pgfpathlineto{\pgfqpoint{2.324693in}{1.834604in}}%
\pgfpathlineto{\pgfqpoint{2.327676in}{1.816767in}}%
\pgfpathlineto{\pgfqpoint{2.331203in}{1.771598in}}%
\pgfpathlineto{\pgfqpoint{2.334051in}{1.693822in}}%
\pgfpathlineto{\pgfqpoint{2.335678in}{1.549382in}}%
\pgfpathlineto{\pgfqpoint{2.336085in}{1.418922in}}%
\pgfpathlineto{\pgfqpoint{2.336899in}{1.608556in}}%
\pgfpathlineto{\pgfqpoint{2.339340in}{1.659570in}}%
\pgfpathlineto{\pgfqpoint{2.339611in}{1.658261in}}%
\pgfpathlineto{\pgfqpoint{2.341239in}{1.605657in}}%
\pgfpathlineto{\pgfqpoint{2.341917in}{1.483173in}}%
\pgfpathlineto{\pgfqpoint{2.342053in}{1.292436in}}%
\pgfpathlineto{\pgfqpoint{2.343273in}{1.650161in}}%
\pgfpathlineto{\pgfqpoint{2.346528in}{1.764014in}}%
\pgfpathlineto{\pgfqpoint{2.350461in}{1.816736in}}%
\pgfpathlineto{\pgfqpoint{2.354666in}{1.837266in}}%
\pgfpathlineto{\pgfqpoint{2.356564in}{1.838336in}}%
\pgfpathlineto{\pgfqpoint{2.356971in}{1.837932in}}%
\pgfpathlineto{\pgfqpoint{2.359006in}{1.832486in}}%
\pgfpathlineto{\pgfqpoint{2.361989in}{1.813145in}}%
\pgfpathlineto{\pgfqpoint{2.364702in}{1.778961in}}%
\pgfpathlineto{\pgfqpoint{2.367686in}{1.705147in}}%
\pgfpathlineto{\pgfqpoint{2.369584in}{1.564836in}}%
\pgfpathlineto{\pgfqpoint{2.369991in}{1.325227in}}%
\pgfpathlineto{\pgfqpoint{2.370941in}{1.611168in}}%
\pgfpathlineto{\pgfqpoint{2.372975in}{1.657566in}}%
\pgfpathlineto{\pgfqpoint{2.373111in}{1.657428in}}%
\pgfpathlineto{\pgfqpoint{2.373924in}{1.646692in}}%
\pgfpathlineto{\pgfqpoint{2.375145in}{1.594590in}}%
\pgfpathlineto{\pgfqpoint{2.375823in}{1.481292in}}%
\pgfpathlineto{\pgfqpoint{2.375959in}{1.252489in}}%
\pgfpathlineto{\pgfqpoint{2.377179in}{1.654791in}}%
\pgfpathlineto{\pgfqpoint{2.380299in}{1.762938in}}%
\pgfpathlineto{\pgfqpoint{2.384503in}{1.818678in}}%
\pgfpathlineto{\pgfqpoint{2.388301in}{1.838056in}}%
\pgfpathlineto{\pgfqpoint{2.390471in}{1.840027in}}%
\pgfpathlineto{\pgfqpoint{2.390606in}{1.839964in}}%
\pgfpathlineto{\pgfqpoint{2.392776in}{1.835047in}}%
\pgfpathlineto{\pgfqpoint{2.396167in}{1.812843in}}%
\pgfpathlineto{\pgfqpoint{2.399286in}{1.770379in}}%
\pgfpathlineto{\pgfqpoint{2.402134in}{1.687385in}}%
\pgfpathlineto{\pgfqpoint{2.403626in}{1.532912in}}%
\pgfpathlineto{\pgfqpoint{2.403898in}{1.368102in}}%
\pgfpathlineto{\pgfqpoint{2.404847in}{1.620728in}}%
\pgfpathlineto{\pgfqpoint{2.407288in}{1.668707in}}%
\pgfpathlineto{\pgfqpoint{2.408102in}{1.661806in}}%
\pgfpathlineto{\pgfqpoint{2.409187in}{1.627569in}}%
\pgfpathlineto{\pgfqpoint{2.410001in}{1.538795in}}%
\pgfpathlineto{\pgfqpoint{2.410272in}{1.384384in}}%
\pgfpathlineto{\pgfqpoint{2.411086in}{1.623022in}}%
\pgfpathlineto{\pgfqpoint{2.414612in}{1.765461in}}%
\pgfpathlineto{\pgfqpoint{2.418545in}{1.818107in}}%
\pgfpathlineto{\pgfqpoint{2.422478in}{1.838129in}}%
\pgfpathlineto{\pgfqpoint{2.424784in}{1.839310in}}%
\pgfpathlineto{\pgfqpoint{2.426140in}{1.836601in}}%
\pgfpathlineto{\pgfqpoint{2.429124in}{1.821295in}}%
\pgfpathlineto{\pgfqpoint{2.432650in}{1.780315in}}%
\pgfpathlineto{\pgfqpoint{2.435769in}{1.700729in}}%
\pgfpathlineto{\pgfqpoint{2.437397in}{1.576827in}}%
\pgfpathlineto{\pgfqpoint{2.437804in}{1.406497in}}%
\pgfpathlineto{\pgfqpoint{2.438753in}{1.612644in}}%
\pgfpathlineto{\pgfqpoint{2.440788in}{1.667130in}}%
\pgfpathlineto{\pgfqpoint{2.441194in}{1.665663in}}%
\pgfpathlineto{\pgfqpoint{2.441737in}{1.663936in}}%
\pgfpathlineto{\pgfqpoint{2.443229in}{1.614658in}}%
\pgfpathlineto{\pgfqpoint{2.444043in}{1.490623in}}%
\pgfpathlineto{\pgfqpoint{2.444178in}{1.318482in}}%
\pgfpathlineto{\pgfqpoint{2.445399in}{1.657120in}}%
\pgfpathlineto{\pgfqpoint{2.448925in}{1.771771in}}%
\pgfpathlineto{\pgfqpoint{2.453672in}{1.826509in}}%
\pgfpathlineto{\pgfqpoint{2.457469in}{1.840248in}}%
\pgfpathlineto{\pgfqpoint{2.459639in}{1.839116in}}%
\pgfpathlineto{\pgfqpoint{2.462623in}{1.827149in}}%
\pgfpathlineto{\pgfqpoint{2.466692in}{1.784811in}}%
\pgfpathlineto{\pgfqpoint{2.469811in}{1.710433in}}%
\pgfpathlineto{\pgfqpoint{2.471574in}{1.600591in}}%
\pgfpathlineto{\pgfqpoint{2.472253in}{1.405720in}}%
\pgfpathlineto{\pgfqpoint{2.472931in}{1.603003in}}%
\pgfpathlineto{\pgfqpoint{2.475236in}{1.663464in}}%
\pgfpathlineto{\pgfqpoint{2.476593in}{1.647537in}}%
\pgfpathlineto{\pgfqpoint{2.478084in}{1.544285in}}%
\pgfpathlineto{\pgfqpoint{2.478356in}{1.440697in}}%
\pgfpathlineto{\pgfqpoint{2.479169in}{1.615780in}}%
\pgfpathlineto{\pgfqpoint{2.482153in}{1.751257in}}%
\pgfpathlineto{\pgfqpoint{2.486493in}{1.815317in}}%
\pgfpathlineto{\pgfqpoint{2.490155in}{1.836252in}}%
\pgfpathlineto{\pgfqpoint{2.492732in}{1.839381in}}%
\pgfpathlineto{\pgfqpoint{2.492868in}{1.839322in}}%
\pgfpathlineto{\pgfqpoint{2.494766in}{1.835268in}}%
\pgfpathlineto{\pgfqpoint{2.496801in}{1.825547in}}%
\pgfpathlineto{\pgfqpoint{2.499378in}{1.802352in}}%
\pgfpathlineto{\pgfqpoint{2.502904in}{1.739880in}}%
\pgfpathlineto{\pgfqpoint{2.505209in}{1.636383in}}%
\pgfpathlineto{\pgfqpoint{2.506023in}{1.500069in}}%
\pgfpathlineto{\pgfqpoint{2.506159in}{1.194366in}}%
\pgfpathlineto{\pgfqpoint{2.507379in}{1.632480in}}%
\pgfpathlineto{\pgfqpoint{2.509549in}{1.670272in}}%
\pgfpathlineto{\pgfqpoint{2.510634in}{1.657353in}}%
\pgfpathlineto{\pgfqpoint{2.511855in}{1.601405in}}%
\pgfpathlineto{\pgfqpoint{2.512669in}{1.409155in}}%
\pgfpathlineto{\pgfqpoint{2.513211in}{1.606573in}}%
\pgfpathlineto{\pgfqpoint{2.515924in}{1.742683in}}%
\pgfpathlineto{\pgfqpoint{2.520264in}{1.813110in}}%
\pgfpathlineto{\pgfqpoint{2.524604in}{1.838259in}}%
\pgfpathlineto{\pgfqpoint{2.526503in}{1.840872in}}%
\pgfpathlineto{\pgfqpoint{2.527181in}{1.840403in}}%
\pgfpathlineto{\pgfqpoint{2.528266in}{1.838571in}}%
\pgfpathlineto{\pgfqpoint{2.530029in}{1.831862in}}%
\pgfpathlineto{\pgfqpoint{2.532470in}{1.814807in}}%
\pgfpathlineto{\pgfqpoint{2.536403in}{1.757683in}}%
\pgfpathlineto{\pgfqpoint{2.539116in}{1.660855in}}%
\pgfpathlineto{\pgfqpoint{2.540201in}{1.526667in}}%
\pgfpathlineto{\pgfqpoint{2.540472in}{1.250794in}}%
\pgfpathlineto{\pgfqpoint{2.541421in}{1.616479in}}%
\pgfpathlineto{\pgfqpoint{2.543184in}{1.663737in}}%
\pgfpathlineto{\pgfqpoint{2.543727in}{1.659237in}}%
\pgfpathlineto{\pgfqpoint{2.544134in}{1.659573in}}%
\pgfpathlineto{\pgfqpoint{2.544541in}{1.653922in}}%
\pgfpathlineto{\pgfqpoint{2.546033in}{1.584869in}}%
\pgfpathlineto{\pgfqpoint{2.546439in}{1.492103in}}%
\pgfpathlineto{\pgfqpoint{2.546575in}{1.291562in}}%
\pgfpathlineto{\pgfqpoint{2.547796in}{1.652965in}}%
\pgfpathlineto{\pgfqpoint{2.551051in}{1.765595in}}%
\pgfpathlineto{\pgfqpoint{2.554984in}{1.817298in}}%
\pgfpathlineto{\pgfqpoint{2.558917in}{1.837444in}}%
\pgfpathlineto{\pgfqpoint{2.560816in}{1.839024in}}%
\pgfpathlineto{\pgfqpoint{2.561494in}{1.838399in}}%
\pgfpathlineto{\pgfqpoint{2.562714in}{1.835778in}}%
\pgfpathlineto{\pgfqpoint{2.565020in}{1.825088in}}%
\pgfpathlineto{\pgfqpoint{2.568139in}{1.795400in}}%
\pgfpathlineto{\pgfqpoint{2.571123in}{1.740547in}}%
\pgfpathlineto{\pgfqpoint{2.573700in}{1.625039in}}%
\pgfpathlineto{\pgfqpoint{2.574378in}{1.473242in}}%
\pgfpathlineto{\pgfqpoint{2.574514in}{1.337788in}}%
\pgfpathlineto{\pgfqpoint{2.575734in}{1.626004in}}%
\pgfpathlineto{\pgfqpoint{2.577633in}{1.658361in}}%
\pgfpathlineto{\pgfqpoint{2.578040in}{1.655757in}}%
\pgfpathlineto{\pgfqpoint{2.579125in}{1.631871in}}%
\pgfpathlineto{\pgfqpoint{2.580210in}{1.545143in}}%
\pgfpathlineto{\pgfqpoint{2.580481in}{1.388872in}}%
\pgfpathlineto{\pgfqpoint{2.581431in}{1.625874in}}%
\pgfpathlineto{\pgfqpoint{2.584143in}{1.745933in}}%
\pgfpathlineto{\pgfqpoint{2.589297in}{1.820240in}}%
\pgfpathlineto{\pgfqpoint{2.593366in}{1.838523in}}%
\pgfpathlineto{\pgfqpoint{2.594722in}{1.839316in}}%
\pgfpathlineto{\pgfqpoint{2.595264in}{1.838962in}}%
\pgfpathlineto{\pgfqpoint{2.596892in}{1.835690in}}%
\pgfpathlineto{\pgfqpoint{2.599198in}{1.824201in}}%
\pgfpathlineto{\pgfqpoint{2.602859in}{1.786313in}}%
\pgfpathlineto{\pgfqpoint{2.605572in}{1.732062in}}%
\pgfpathlineto{\pgfqpoint{2.608013in}{1.616204in}}%
\pgfpathlineto{\pgfqpoint{2.608827in}{1.479404in}}%
\pgfpathlineto{\pgfqpoint{2.608963in}{1.264366in}}%
\pgfpathlineto{\pgfqpoint{2.610183in}{1.606360in}}%
\pgfpathlineto{\pgfqpoint{2.611268in}{1.632428in}}%
\pgfpathlineto{\pgfqpoint{2.612082in}{1.624599in}}%
\pgfpathlineto{\pgfqpoint{2.612624in}{1.614059in}}%
\pgfpathlineto{\pgfqpoint{2.613709in}{1.508596in}}%
\pgfpathlineto{\pgfqpoint{2.613981in}{1.364108in}}%
\pgfpathlineto{\pgfqpoint{2.614930in}{1.619977in}}%
\pgfpathlineto{\pgfqpoint{2.617778in}{1.742218in}}%
\pgfpathlineto{\pgfqpoint{2.622389in}{1.812183in}}%
\pgfpathlineto{\pgfqpoint{2.626594in}{1.835288in}}%
\pgfpathlineto{\pgfqpoint{2.628899in}{1.837470in}}%
\pgfpathlineto{\pgfqpoint{2.630527in}{1.834880in}}%
\pgfpathlineto{\pgfqpoint{2.633782in}{1.818577in}}%
\pgfpathlineto{\pgfqpoint{2.638258in}{1.762181in}}%
\pgfpathlineto{\pgfqpoint{2.641106in}{1.680841in}}%
\pgfpathlineto{\pgfqpoint{2.642733in}{1.543085in}}%
\pgfpathlineto{\pgfqpoint{2.643140in}{1.262893in}}%
\pgfpathlineto{\pgfqpoint{2.644089in}{1.583143in}}%
\pgfpathlineto{\pgfqpoint{2.645988in}{1.619226in}}%
\pgfpathlineto{\pgfqpoint{2.646124in}{1.617629in}}%
\pgfpathlineto{\pgfqpoint{2.647344in}{1.554237in}}%
\pgfpathlineto{\pgfqpoint{2.647751in}{1.399328in}}%
\pgfpathlineto{\pgfqpoint{2.648565in}{1.596086in}}%
\pgfpathlineto{\pgfqpoint{2.651278in}{1.727931in}}%
\pgfpathlineto{\pgfqpoint{2.655482in}{1.802247in}}%
\pgfpathlineto{\pgfqpoint{2.659686in}{1.831591in}}%
\pgfpathlineto{\pgfqpoint{2.662806in}{1.836629in}}%
\pgfpathlineto{\pgfqpoint{2.664298in}{1.834610in}}%
\pgfpathlineto{\pgfqpoint{2.667553in}{1.819679in}}%
\pgfpathlineto{\pgfqpoint{2.671079in}{1.783721in}}%
\pgfpathlineto{\pgfqpoint{2.674198in}{1.721532in}}%
\pgfpathlineto{\pgfqpoint{2.677046in}{1.596191in}}%
\pgfpathlineto{\pgfqpoint{2.677996in}{1.369020in}}%
\pgfpathlineto{\pgfqpoint{2.679081in}{1.525210in}}%
\pgfpathlineto{\pgfqpoint{2.679488in}{1.543490in}}%
\pgfpathlineto{\pgfqpoint{2.680166in}{1.509506in}}%
\pgfpathlineto{\pgfqpoint{2.680844in}{1.322119in}}%
\pgfpathlineto{\pgfqpoint{2.681386in}{1.538937in}}%
\pgfpathlineto{\pgfqpoint{2.684099in}{1.704012in}}%
\pgfpathlineto{\pgfqpoint{2.687761in}{1.782248in}}%
\pgfpathlineto{\pgfqpoint{2.692236in}{1.824825in}}%
\pgfpathlineto{\pgfqpoint{2.696305in}{1.835829in}}%
\pgfpathlineto{\pgfqpoint{2.697661in}{1.834712in}}%
\pgfpathlineto{\pgfqpoint{2.700374in}{1.825583in}}%
\pgfpathlineto{\pgfqpoint{2.703086in}{1.804876in}}%
\pgfpathlineto{\pgfqpoint{2.705934in}{1.766463in}}%
\pgfpathlineto{\pgfqpoint{2.708918in}{1.688378in}}%
\pgfpathlineto{\pgfqpoint{2.710681in}{1.574781in}}%
\pgfpathlineto{\pgfqpoint{2.711359in}{1.348256in}}%
\pgfpathlineto{\pgfqpoint{2.712173in}{1.565178in}}%
\pgfpathlineto{\pgfqpoint{2.713123in}{1.597685in}}%
\pgfpathlineto{\pgfqpoint{2.714072in}{1.590163in}}%
\pgfpathlineto{\pgfqpoint{2.714343in}{1.585644in}}%
\pgfpathlineto{\pgfqpoint{2.715293in}{1.459441in}}%
\pgfpathlineto{\pgfqpoint{2.715428in}{1.290914in}}%
\pgfpathlineto{\pgfqpoint{2.716649in}{1.625628in}}%
\pgfpathlineto{\pgfqpoint{2.719497in}{1.737633in}}%
\pgfpathlineto{\pgfqpoint{2.723430in}{1.802289in}}%
\pgfpathlineto{\pgfqpoint{2.727634in}{1.831128in}}%
\pgfpathlineto{\pgfqpoint{2.730483in}{1.836638in}}%
\pgfpathlineto{\pgfqpoint{2.730889in}{1.836363in}}%
\pgfpathlineto{\pgfqpoint{2.732788in}{1.833283in}}%
\pgfpathlineto{\pgfqpoint{2.734687in}{1.825468in}}%
\pgfpathlineto{\pgfqpoint{2.737128in}{1.807589in}}%
\pgfpathlineto{\pgfqpoint{2.740926in}{1.753737in}}%
\pgfpathlineto{\pgfqpoint{2.744452in}{1.628594in}}%
\pgfpathlineto{\pgfqpoint{2.745537in}{1.511839in}}%
\pgfpathlineto{\pgfqpoint{2.745808in}{1.401050in}}%
\pgfpathlineto{\pgfqpoint{2.746758in}{1.557337in}}%
\pgfpathlineto{\pgfqpoint{2.747571in}{1.583407in}}%
\pgfpathlineto{\pgfqpoint{2.748249in}{1.566196in}}%
\pgfpathlineto{\pgfqpoint{2.749199in}{1.498827in}}%
\pgfpathlineto{\pgfqpoint{2.749470in}{1.344940in}}%
\pgfpathlineto{\pgfqpoint{2.750284in}{1.590032in}}%
\pgfpathlineto{\pgfqpoint{2.753403in}{1.732129in}}%
\pgfpathlineto{\pgfqpoint{2.757608in}{1.802115in}}%
\pgfpathlineto{\pgfqpoint{2.761541in}{1.830844in}}%
\pgfpathlineto{\pgfqpoint{2.764524in}{1.837283in}}%
\pgfpathlineto{\pgfqpoint{2.765881in}{1.836387in}}%
\pgfpathlineto{\pgfqpoint{2.769000in}{1.825082in}}%
\pgfpathlineto{\pgfqpoint{2.772662in}{1.791060in}}%
\pgfpathlineto{\pgfqpoint{2.775781in}{1.730794in}}%
\pgfpathlineto{\pgfqpoint{2.778087in}{1.632749in}}%
\pgfpathlineto{\pgfqpoint{2.779036in}{1.482452in}}%
\pgfpathlineto{\pgfqpoint{2.779172in}{1.246365in}}%
\pgfpathlineto{\pgfqpoint{2.780393in}{1.604148in}}%
\pgfpathlineto{\pgfqpoint{2.781342in}{1.630303in}}%
\pgfpathlineto{\pgfqpoint{2.782291in}{1.627381in}}%
\pgfpathlineto{\pgfqpoint{2.782698in}{1.619480in}}%
\pgfpathlineto{\pgfqpoint{2.783919in}{1.522180in}}%
\pgfpathlineto{\pgfqpoint{2.784326in}{1.372334in}}%
\pgfpathlineto{\pgfqpoint{2.785004in}{1.602841in}}%
\pgfpathlineto{\pgfqpoint{2.788123in}{1.743227in}}%
\pgfpathlineto{\pgfqpoint{2.791921in}{1.804551in}}%
\pgfpathlineto{\pgfqpoint{2.796396in}{1.834052in}}%
\pgfpathlineto{\pgfqpoint{2.799109in}{1.837683in}}%
\pgfpathlineto{\pgfqpoint{2.799244in}{1.837610in}}%
\pgfpathlineto{\pgfqpoint{2.801550in}{1.832657in}}%
\pgfpathlineto{\pgfqpoint{2.804127in}{1.818267in}}%
\pgfpathlineto{\pgfqpoint{2.808060in}{1.772309in}}%
\pgfpathlineto{\pgfqpoint{2.811451in}{1.684013in}}%
\pgfpathlineto{\pgfqpoint{2.813078in}{1.565645in}}%
\pgfpathlineto{\pgfqpoint{2.813621in}{1.380389in}}%
\pgfpathlineto{\pgfqpoint{2.814434in}{1.572238in}}%
\pgfpathlineto{\pgfqpoint{2.815926in}{1.616269in}}%
\pgfpathlineto{\pgfqpoint{2.816333in}{1.602212in}}%
\pgfpathlineto{\pgfqpoint{2.817825in}{1.499105in}}%
\pgfpathlineto{\pgfqpoint{2.818096in}{1.425272in}}%
\pgfpathlineto{\pgfqpoint{2.819046in}{1.618990in}}%
\pgfpathlineto{\pgfqpoint{2.822301in}{1.748614in}}%
\pgfpathlineto{\pgfqpoint{2.827319in}{1.817567in}}%
\pgfpathlineto{\pgfqpoint{2.831930in}{1.837724in}}%
\pgfpathlineto{\pgfqpoint{2.834371in}{1.836300in}}%
\pgfpathlineto{\pgfqpoint{2.835863in}{1.831860in}}%
\pgfpathlineto{\pgfqpoint{2.838983in}{1.811356in}}%
\pgfpathlineto{\pgfqpoint{2.842373in}{1.765852in}}%
\pgfpathlineto{\pgfqpoint{2.845764in}{1.657708in}}%
\pgfpathlineto{\pgfqpoint{2.846984in}{1.527303in}}%
\pgfpathlineto{\pgfqpoint{2.847256in}{1.422937in}}%
\pgfpathlineto{\pgfqpoint{2.848341in}{1.609674in}}%
\pgfpathlineto{\pgfqpoint{2.849561in}{1.637851in}}%
\pgfpathlineto{\pgfqpoint{2.850375in}{1.637195in}}%
\pgfpathlineto{\pgfqpoint{2.851460in}{1.605639in}}%
\pgfpathlineto{\pgfqpoint{2.852545in}{1.420684in}}%
\pgfpathlineto{\pgfqpoint{2.853088in}{1.585165in}}%
\pgfpathlineto{\pgfqpoint{2.855664in}{1.725588in}}%
\pgfpathlineto{\pgfqpoint{2.860683in}{1.810467in}}%
\pgfpathlineto{\pgfqpoint{2.864887in}{1.835452in}}%
\pgfpathlineto{\pgfqpoint{2.867057in}{1.838231in}}%
\pgfpathlineto{\pgfqpoint{2.867193in}{1.838145in}}%
\pgfpathlineto{\pgfqpoint{2.870176in}{1.831315in}}%
\pgfpathlineto{\pgfqpoint{2.873296in}{1.809714in}}%
\pgfpathlineto{\pgfqpoint{2.876415in}{1.767788in}}%
\pgfpathlineto{\pgfqpoint{2.878856in}{1.704781in}}%
\pgfpathlineto{\pgfqpoint{2.880755in}{1.583040in}}%
\pgfpathlineto{\pgfqpoint{2.881433in}{1.415630in}}%
\pgfpathlineto{\pgfqpoint{2.882247in}{1.582433in}}%
\pgfpathlineto{\pgfqpoint{2.884010in}{1.631044in}}%
\pgfpathlineto{\pgfqpoint{2.884417in}{1.623286in}}%
\pgfpathlineto{\pgfqpoint{2.885231in}{1.601103in}}%
\pgfpathlineto{\pgfqpoint{2.886180in}{1.471705in}}%
\pgfpathlineto{\pgfqpoint{2.886316in}{1.257779in}}%
\pgfpathlineto{\pgfqpoint{2.887536in}{1.636611in}}%
\pgfpathlineto{\pgfqpoint{2.890791in}{1.756974in}}%
\pgfpathlineto{\pgfqpoint{2.894589in}{1.811727in}}%
\pgfpathlineto{\pgfqpoint{2.898386in}{1.835027in}}%
\pgfpathlineto{\pgfqpoint{2.900692in}{1.839089in}}%
\pgfpathlineto{\pgfqpoint{2.903133in}{1.835643in}}%
\pgfpathlineto{\pgfqpoint{2.905168in}{1.826672in}}%
\pgfpathlineto{\pgfqpoint{2.905168in}{1.826672in}}%
\pgfusepath{stroke}%
\end{pgfscope}%
\begin{pgfscope}%
\pgfpathrectangle{\pgfqpoint{0.735032in}{0.526079in}}{\pgfqpoint{2.170000in}{1.661000in}} %
\pgfusepath{clip}%
\pgfsetrectcap%
\pgfsetroundjoin%
\pgfsetlinewidth{1.003750pt}%
\definecolor{currentstroke}{rgb}{1.000000,0.549020,0.000000}%
\pgfsetstrokecolor{currentstroke}%
\pgfsetdash{}{0pt}%
\pgfpathmoveto{\pgfqpoint{0.735167in}{0.512191in}}%
\pgfpathlineto{\pgfqpoint{0.736659in}{1.133587in}}%
\pgfpathlineto{\pgfqpoint{0.737202in}{1.157148in}}%
\pgfpathlineto{\pgfqpoint{0.738423in}{1.146895in}}%
\pgfpathlineto{\pgfqpoint{0.738558in}{1.147225in}}%
\pgfpathlineto{\pgfqpoint{0.740050in}{1.068909in}}%
\pgfpathlineto{\pgfqpoint{0.740321in}{0.985787in}}%
\pgfpathlineto{\pgfqpoint{0.741406in}{1.166061in}}%
\pgfpathlineto{\pgfqpoint{0.744254in}{1.270268in}}%
\pgfpathlineto{\pgfqpoint{0.747645in}{1.325470in}}%
\pgfpathlineto{\pgfqpoint{0.752799in}{1.357626in}}%
\pgfpathlineto{\pgfqpoint{0.754562in}{1.359131in}}%
\pgfpathlineto{\pgfqpoint{0.754833in}{1.358891in}}%
\pgfpathlineto{\pgfqpoint{0.757817in}{1.351448in}}%
\pgfpathlineto{\pgfqpoint{0.759851in}{1.339589in}}%
\pgfpathlineto{\pgfqpoint{0.762971in}{1.306243in}}%
\pgfpathlineto{\pgfqpoint{0.766361in}{1.224452in}}%
\pgfpathlineto{\pgfqpoint{0.768124in}{1.078576in}}%
\pgfpathlineto{\pgfqpoint{0.768396in}{0.859604in}}%
\pgfpathlineto{\pgfqpoint{0.769481in}{1.148850in}}%
\pgfpathlineto{\pgfqpoint{0.771651in}{1.199660in}}%
\pgfpathlineto{\pgfqpoint{0.774228in}{1.095481in}}%
\pgfpathlineto{\pgfqpoint{0.774770in}{0.926698in}}%
\pgfpathlineto{\pgfqpoint{0.775448in}{1.126801in}}%
\pgfpathlineto{\pgfqpoint{0.778296in}{1.261598in}}%
\pgfpathlineto{\pgfqpoint{0.782501in}{1.329059in}}%
\pgfpathlineto{\pgfqpoint{0.787383in}{1.359149in}}%
\pgfpathlineto{\pgfqpoint{0.788739in}{1.359618in}}%
\pgfpathlineto{\pgfqpoint{0.788875in}{1.360008in}}%
\pgfpathlineto{\pgfqpoint{0.789282in}{1.360630in}}%
\pgfpathlineto{\pgfqpoint{0.790096in}{1.359101in}}%
\pgfpathlineto{\pgfqpoint{0.790231in}{1.359388in}}%
\pgfpathlineto{\pgfqpoint{0.790367in}{1.359451in}}%
\pgfpathlineto{\pgfqpoint{0.790774in}{1.358372in}}%
\pgfpathlineto{\pgfqpoint{0.793215in}{1.346989in}}%
\pgfpathlineto{\pgfqpoint{0.797419in}{1.302597in}}%
\pgfpathlineto{\pgfqpoint{0.801759in}{1.173666in}}%
\pgfpathlineto{\pgfqpoint{0.803251in}{0.889587in}}%
\pgfpathlineto{\pgfqpoint{0.803794in}{1.116850in}}%
\pgfpathlineto{\pgfqpoint{0.805828in}{1.169727in}}%
\pgfpathlineto{\pgfqpoint{0.806235in}{1.179748in}}%
\pgfpathlineto{\pgfqpoint{0.807320in}{1.178634in}}%
\pgfpathlineto{\pgfqpoint{0.808948in}{1.034985in}}%
\pgfpathlineto{\pgfqpoint{0.809219in}{0.907784in}}%
\pgfpathlineto{\pgfqpoint{0.810033in}{1.146836in}}%
\pgfpathlineto{\pgfqpoint{0.812609in}{1.265742in}}%
\pgfpathlineto{\pgfqpoint{0.815729in}{1.318017in}}%
\pgfpathlineto{\pgfqpoint{0.821154in}{1.353196in}}%
\pgfpathlineto{\pgfqpoint{0.822374in}{1.354839in}}%
\pgfpathlineto{\pgfqpoint{0.823053in}{1.353978in}}%
\pgfpathlineto{\pgfqpoint{0.825087in}{1.351252in}}%
\pgfpathlineto{\pgfqpoint{0.827799in}{1.337148in}}%
\pgfpathlineto{\pgfqpoint{0.832953in}{1.260334in}}%
\pgfpathlineto{\pgfqpoint{0.835666in}{1.099308in}}%
\pgfpathlineto{\pgfqpoint{0.835937in}{0.968856in}}%
\pgfpathlineto{\pgfqpoint{0.836886in}{1.143772in}}%
\pgfpathlineto{\pgfqpoint{0.838378in}{1.211880in}}%
\pgfpathlineto{\pgfqpoint{0.838785in}{1.210896in}}%
\pgfpathlineto{\pgfqpoint{0.839056in}{1.209929in}}%
\pgfpathlineto{\pgfqpoint{0.839463in}{1.215349in}}%
\pgfpathlineto{\pgfqpoint{0.839734in}{1.211579in}}%
\pgfpathlineto{\pgfqpoint{0.840277in}{1.219452in}}%
\pgfpathlineto{\pgfqpoint{0.841091in}{1.207575in}}%
\pgfpathlineto{\pgfqpoint{0.843125in}{1.136523in}}%
\pgfpathlineto{\pgfqpoint{0.843668in}{0.954614in}}%
\pgfpathlineto{\pgfqpoint{0.843803in}{0.839914in}}%
\pgfpathlineto{\pgfqpoint{0.844888in}{1.161652in}}%
\pgfpathlineto{\pgfqpoint{0.847736in}{1.277944in}}%
\pgfpathlineto{\pgfqpoint{0.851398in}{1.332994in}}%
\pgfpathlineto{\pgfqpoint{0.855060in}{1.355336in}}%
\pgfpathlineto{\pgfqpoint{0.857501in}{1.357275in}}%
\pgfpathlineto{\pgfqpoint{0.858315in}{1.355824in}}%
\pgfpathlineto{\pgfqpoint{0.865096in}{1.299344in}}%
\pgfpathlineto{\pgfqpoint{0.868351in}{1.198297in}}%
\pgfpathlineto{\pgfqpoint{0.869572in}{1.054324in}}%
\pgfpathlineto{\pgfqpoint{0.869708in}{0.670099in}}%
\pgfpathlineto{\pgfqpoint{0.871064in}{1.192326in}}%
\pgfpathlineto{\pgfqpoint{0.873098in}{1.228074in}}%
\pgfpathlineto{\pgfqpoint{0.874861in}{1.239566in}}%
\pgfpathlineto{\pgfqpoint{0.875268in}{1.235237in}}%
\pgfpathlineto{\pgfqpoint{0.876896in}{1.210051in}}%
\pgfpathlineto{\pgfqpoint{0.878523in}{1.072085in}}%
\pgfpathlineto{\pgfqpoint{0.879066in}{0.988031in}}%
\pgfpathlineto{\pgfqpoint{0.879473in}{1.157145in}}%
\pgfpathlineto{\pgfqpoint{0.882456in}{1.282364in}}%
\pgfpathlineto{\pgfqpoint{0.887339in}{1.337720in}}%
\pgfpathlineto{\pgfqpoint{0.890323in}{1.350001in}}%
\pgfpathlineto{\pgfqpoint{0.891543in}{1.348485in}}%
\pgfpathlineto{\pgfqpoint{0.893171in}{1.345050in}}%
\pgfpathlineto{\pgfqpoint{0.898324in}{1.295317in}}%
\pgfpathlineto{\pgfqpoint{0.902529in}{1.136709in}}%
\pgfpathlineto{\pgfqpoint{0.902800in}{1.078278in}}%
\pgfpathlineto{\pgfqpoint{0.902936in}{0.883440in}}%
\pgfpathlineto{\pgfqpoint{0.904156in}{1.177335in}}%
\pgfpathlineto{\pgfqpoint{0.909039in}{1.259735in}}%
\pgfpathlineto{\pgfqpoint{0.912701in}{1.160085in}}%
\pgfpathlineto{\pgfqpoint{0.913243in}{1.072166in}}%
\pgfpathlineto{\pgfqpoint{0.913379in}{0.932522in}}%
\pgfpathlineto{\pgfqpoint{0.914599in}{1.186218in}}%
\pgfpathlineto{\pgfqpoint{0.920024in}{1.328397in}}%
\pgfpathlineto{\pgfqpoint{0.922873in}{1.347710in}}%
\pgfpathlineto{\pgfqpoint{0.926128in}{1.352685in}}%
\pgfpathlineto{\pgfqpoint{0.931553in}{1.316594in}}%
\pgfpathlineto{\pgfqpoint{0.935757in}{1.231171in}}%
\pgfpathlineto{\pgfqpoint{0.937656in}{0.999681in}}%
\pgfpathlineto{\pgfqpoint{0.938063in}{1.145396in}}%
\pgfpathlineto{\pgfqpoint{0.940097in}{1.233033in}}%
\pgfpathlineto{\pgfqpoint{0.942131in}{1.249160in}}%
\pgfpathlineto{\pgfqpoint{0.942267in}{1.249813in}}%
\pgfpathlineto{\pgfqpoint{0.942538in}{1.244244in}}%
\pgfpathlineto{\pgfqpoint{0.943081in}{1.247866in}}%
\pgfpathlineto{\pgfqpoint{0.946064in}{1.181160in}}%
\pgfpathlineto{\pgfqpoint{0.947149in}{0.996092in}}%
\pgfpathlineto{\pgfqpoint{0.947828in}{1.126361in}}%
\pgfpathlineto{\pgfqpoint{0.950404in}{1.273217in}}%
\pgfpathlineto{\pgfqpoint{0.957999in}{1.350275in}}%
\pgfpathlineto{\pgfqpoint{0.958135in}{1.350062in}}%
\pgfpathlineto{\pgfqpoint{0.958678in}{1.348559in}}%
\pgfpathlineto{\pgfqpoint{0.958949in}{1.350744in}}%
\pgfpathlineto{\pgfqpoint{0.959084in}{1.351707in}}%
\pgfpathlineto{\pgfqpoint{0.959763in}{1.348363in}}%
\pgfpathlineto{\pgfqpoint{0.960169in}{1.349008in}}%
\pgfpathlineto{\pgfqpoint{0.962339in}{1.343237in}}%
\pgfpathlineto{\pgfqpoint{0.963153in}{1.338904in}}%
\pgfpathlineto{\pgfqpoint{0.968985in}{1.258659in}}%
\pgfpathlineto{\pgfqpoint{0.971019in}{1.177868in}}%
\pgfpathlineto{\pgfqpoint{0.971833in}{1.060387in}}%
\pgfpathlineto{\pgfqpoint{0.971969in}{0.930049in}}%
\pgfpathlineto{\pgfqpoint{0.973054in}{1.158004in}}%
\pgfpathlineto{\pgfqpoint{0.973189in}{1.157425in}}%
\pgfpathlineto{\pgfqpoint{0.973325in}{1.147597in}}%
\pgfpathlineto{\pgfqpoint{0.974003in}{1.209202in}}%
\pgfpathlineto{\pgfqpoint{0.974410in}{1.198897in}}%
\pgfpathlineto{\pgfqpoint{0.974546in}{1.194767in}}%
\pgfpathlineto{\pgfqpoint{0.974953in}{1.225279in}}%
\pgfpathlineto{\pgfqpoint{0.976038in}{1.238799in}}%
\pgfpathlineto{\pgfqpoint{0.975631in}{1.222419in}}%
\pgfpathlineto{\pgfqpoint{0.976444in}{1.225983in}}%
\pgfpathlineto{\pgfqpoint{0.976580in}{1.219988in}}%
\pgfpathlineto{\pgfqpoint{0.977123in}{1.232267in}}%
\pgfpathlineto{\pgfqpoint{0.977936in}{1.226753in}}%
\pgfpathlineto{\pgfqpoint{0.978343in}{1.217118in}}%
\pgfpathlineto{\pgfqpoint{0.980784in}{1.066597in}}%
\pgfpathlineto{\pgfqpoint{0.981056in}{0.897745in}}%
\pgfpathlineto{\pgfqpoint{0.981734in}{1.158872in}}%
\pgfpathlineto{\pgfqpoint{0.982141in}{1.180614in}}%
\pgfpathlineto{\pgfqpoint{0.984446in}{1.277537in}}%
\pgfpathlineto{\pgfqpoint{0.991092in}{1.347068in}}%
\pgfpathlineto{\pgfqpoint{0.993940in}{1.352651in}}%
\pgfpathlineto{\pgfqpoint{0.995025in}{1.351643in}}%
\pgfpathlineto{\pgfqpoint{1.001806in}{1.285765in}}%
\pgfpathlineto{\pgfqpoint{1.004112in}{1.162677in}}%
\pgfpathlineto{\pgfqpoint{1.004926in}{0.975809in}}%
\pgfpathlineto{\pgfqpoint{1.005604in}{1.159270in}}%
\pgfpathlineto{\pgfqpoint{1.008045in}{1.251165in}}%
\pgfpathlineto{\pgfqpoint{1.010486in}{1.270505in}}%
\pgfpathlineto{\pgfqpoint{1.008588in}{1.250334in}}%
\pgfpathlineto{\pgfqpoint{1.010758in}{1.265192in}}%
\pgfpathlineto{\pgfqpoint{1.011571in}{1.267353in}}%
\pgfpathlineto{\pgfqpoint{1.014148in}{1.221912in}}%
\pgfpathlineto{\pgfqpoint{1.015911in}{1.065486in}}%
\pgfpathlineto{\pgfqpoint{1.016047in}{0.997036in}}%
\pgfpathlineto{\pgfqpoint{1.017132in}{1.196730in}}%
\pgfpathlineto{\pgfqpoint{1.019709in}{1.273117in}}%
\pgfpathlineto{\pgfqpoint{1.024184in}{1.333994in}}%
\pgfpathlineto{\pgfqpoint{1.024320in}{1.333652in}}%
\pgfpathlineto{\pgfqpoint{1.024591in}{1.332512in}}%
\pgfpathlineto{\pgfqpoint{1.024863in}{1.335346in}}%
\pgfpathlineto{\pgfqpoint{1.027033in}{1.344238in}}%
\pgfpathlineto{\pgfqpoint{1.027575in}{1.343081in}}%
\pgfpathlineto{\pgfqpoint{1.027846in}{1.345332in}}%
\pgfpathlineto{\pgfqpoint{1.028118in}{1.346818in}}%
\pgfpathlineto{\pgfqpoint{1.028931in}{1.343983in}}%
\pgfpathlineto{\pgfqpoint{1.029067in}{1.344065in}}%
\pgfpathlineto{\pgfqpoint{1.033407in}{1.314187in}}%
\pgfpathlineto{\pgfqpoint{1.035170in}{1.279453in}}%
\pgfpathlineto{\pgfqpoint{1.038018in}{1.165627in}}%
\pgfpathlineto{\pgfqpoint{1.038696in}{1.058077in}}%
\pgfpathlineto{\pgfqpoint{1.038832in}{0.962720in}}%
\pgfpathlineto{\pgfqpoint{1.040053in}{1.183773in}}%
\pgfpathlineto{\pgfqpoint{1.040324in}{1.180015in}}%
\pgfpathlineto{\pgfqpoint{1.040459in}{1.183876in}}%
\pgfpathlineto{\pgfqpoint{1.042765in}{1.256422in}}%
\pgfpathlineto{\pgfqpoint{1.043986in}{1.265092in}}%
\pgfpathlineto{\pgfqpoint{1.043308in}{1.255695in}}%
\pgfpathlineto{\pgfqpoint{1.044528in}{1.263244in}}%
\pgfpathlineto{\pgfqpoint{1.044799in}{1.267090in}}%
\pgfpathlineto{\pgfqpoint{1.045206in}{1.252630in}}%
\pgfpathlineto{\pgfqpoint{1.046427in}{1.246018in}}%
\pgfpathlineto{\pgfqpoint{1.045884in}{1.269546in}}%
\pgfpathlineto{\pgfqpoint{1.046698in}{1.253698in}}%
\pgfpathlineto{\pgfqpoint{1.046969in}{1.256673in}}%
\pgfpathlineto{\pgfqpoint{1.047241in}{1.244321in}}%
\pgfpathlineto{\pgfqpoint{1.049818in}{1.080985in}}%
\pgfpathlineto{\pgfqpoint{1.049953in}{0.903691in}}%
\pgfpathlineto{\pgfqpoint{1.051174in}{1.206803in}}%
\pgfpathlineto{\pgfqpoint{1.053479in}{1.282868in}}%
\pgfpathlineto{\pgfqpoint{1.053751in}{1.280686in}}%
\pgfpathlineto{\pgfqpoint{1.057684in}{1.331916in}}%
\pgfpathlineto{\pgfqpoint{1.060125in}{1.343127in}}%
\pgfpathlineto{\pgfqpoint{1.061753in}{1.347750in}}%
\pgfpathlineto{\pgfqpoint{1.062159in}{1.344831in}}%
\pgfpathlineto{\pgfqpoint{1.062838in}{1.345572in}}%
\pgfpathlineto{\pgfqpoint{1.064329in}{1.336907in}}%
\pgfpathlineto{\pgfqpoint{1.064736in}{1.339177in}}%
\pgfpathlineto{\pgfqpoint{1.065414in}{1.334130in}}%
\pgfpathlineto{\pgfqpoint{1.067584in}{1.312499in}}%
\pgfpathlineto{\pgfqpoint{1.071518in}{1.200277in}}%
\pgfpathlineto{\pgfqpoint{1.072467in}{1.014353in}}%
\pgfpathlineto{\pgfqpoint{1.072603in}{0.944265in}}%
\pgfpathlineto{\pgfqpoint{1.073688in}{1.179276in}}%
\pgfpathlineto{\pgfqpoint{1.075586in}{1.252277in}}%
\pgfpathlineto{\pgfqpoint{1.077078in}{1.273610in}}%
\pgfpathlineto{\pgfqpoint{1.077485in}{1.270636in}}%
\pgfpathlineto{\pgfqpoint{1.078977in}{1.278188in}}%
\pgfpathlineto{\pgfqpoint{1.078434in}{1.268632in}}%
\pgfpathlineto{\pgfqpoint{1.079113in}{1.275490in}}%
\pgfpathlineto{\pgfqpoint{1.080469in}{1.267548in}}%
\pgfpathlineto{\pgfqpoint{1.079791in}{1.276182in}}%
\pgfpathlineto{\pgfqpoint{1.080876in}{1.268673in}}%
\pgfpathlineto{\pgfqpoint{1.081011in}{1.270553in}}%
\pgfpathlineto{\pgfqpoint{1.081554in}{1.259818in}}%
\pgfpathlineto{\pgfqpoint{1.083046in}{1.222757in}}%
\pgfpathlineto{\pgfqpoint{1.084402in}{1.042856in}}%
\pgfpathlineto{\pgfqpoint{1.084673in}{0.924492in}}%
\pgfpathlineto{\pgfqpoint{1.085487in}{1.191029in}}%
\pgfpathlineto{\pgfqpoint{1.089963in}{1.314600in}}%
\pgfpathlineto{\pgfqpoint{1.093760in}{1.336399in}}%
\pgfpathlineto{\pgfqpoint{1.094167in}{1.334312in}}%
\pgfpathlineto{\pgfqpoint{1.094303in}{1.334130in}}%
\pgfpathlineto{\pgfqpoint{1.094438in}{1.335053in}}%
\pgfpathlineto{\pgfqpoint{1.095930in}{1.341217in}}%
\pgfpathlineto{\pgfqpoint{1.096201in}{1.338403in}}%
\pgfpathlineto{\pgfqpoint{1.096879in}{1.339397in}}%
\pgfpathlineto{\pgfqpoint{1.098371in}{1.331988in}}%
\pgfpathlineto{\pgfqpoint{1.098778in}{1.332878in}}%
\pgfpathlineto{\pgfqpoint{1.099049in}{1.328407in}}%
\pgfpathlineto{\pgfqpoint{1.101355in}{1.309735in}}%
\pgfpathlineto{\pgfqpoint{1.101491in}{1.309948in}}%
\pgfpathlineto{\pgfqpoint{1.101898in}{1.307102in}}%
\pgfpathlineto{\pgfqpoint{1.105695in}{1.161739in}}%
\pgfpathlineto{\pgfqpoint{1.106102in}{0.938351in}}%
\pgfpathlineto{\pgfqpoint{1.107051in}{1.171072in}}%
\pgfpathlineto{\pgfqpoint{1.108950in}{1.247555in}}%
\pgfpathlineto{\pgfqpoint{1.109357in}{1.254106in}}%
\pgfpathlineto{\pgfqpoint{1.110849in}{1.284025in}}%
\pgfpathlineto{\pgfqpoint{1.111256in}{1.276947in}}%
\pgfpathlineto{\pgfqpoint{1.111527in}{1.280869in}}%
\pgfpathlineto{\pgfqpoint{1.112748in}{1.293620in}}%
\pgfpathlineto{\pgfqpoint{1.113154in}{1.287503in}}%
\pgfpathlineto{\pgfqpoint{1.115189in}{1.272815in}}%
\pgfpathlineto{\pgfqpoint{1.113833in}{1.295620in}}%
\pgfpathlineto{\pgfqpoint{1.115460in}{1.278839in}}%
\pgfpathlineto{\pgfqpoint{1.115596in}{1.281561in}}%
\pgfpathlineto{\pgfqpoint{1.116138in}{1.263353in}}%
\pgfpathlineto{\pgfqpoint{1.116681in}{1.271321in}}%
\pgfpathlineto{\pgfqpoint{1.118037in}{1.230353in}}%
\pgfpathlineto{\pgfqpoint{1.119936in}{0.749460in}}%
\pgfpathlineto{\pgfqpoint{1.120343in}{1.122875in}}%
\pgfpathlineto{\pgfqpoint{1.120478in}{1.107661in}}%
\pgfpathlineto{\pgfqpoint{1.121021in}{1.211607in}}%
\pgfpathlineto{\pgfqpoint{1.121156in}{1.213245in}}%
\pgfpathlineto{\pgfqpoint{1.121428in}{1.202408in}}%
\pgfpathlineto{\pgfqpoint{1.121563in}{1.198362in}}%
\pgfpathlineto{\pgfqpoint{1.121834in}{1.228128in}}%
\pgfpathlineto{\pgfqpoint{1.123191in}{1.283415in}}%
\pgfpathlineto{\pgfqpoint{1.123733in}{1.281130in}}%
\pgfpathlineto{\pgfqpoint{1.125903in}{1.316728in}}%
\pgfpathlineto{\pgfqpoint{1.126039in}{1.316464in}}%
\pgfpathlineto{\pgfqpoint{1.126174in}{1.316057in}}%
\pgfpathlineto{\pgfqpoint{1.126446in}{1.319204in}}%
\pgfpathlineto{\pgfqpoint{1.127938in}{1.332949in}}%
\pgfpathlineto{\pgfqpoint{1.128344in}{1.328012in}}%
\pgfpathlineto{\pgfqpoint{1.128887in}{1.337603in}}%
\pgfpathlineto{\pgfqpoint{1.130108in}{1.334674in}}%
\pgfpathlineto{\pgfqpoint{1.132142in}{1.326659in}}%
\pgfpathlineto{\pgfqpoint{1.132549in}{1.328367in}}%
\pgfpathlineto{\pgfqpoint{1.132820in}{1.324908in}}%
\pgfpathlineto{\pgfqpoint{1.135261in}{1.298633in}}%
\pgfpathlineto{\pgfqpoint{1.135533in}{1.299674in}}%
\pgfpathlineto{\pgfqpoint{1.135668in}{1.297147in}}%
\pgfpathlineto{\pgfqpoint{1.138652in}{1.208370in}}%
\pgfpathlineto{\pgfqpoint{1.139194in}{0.966712in}}%
\pgfpathlineto{\pgfqpoint{1.140279in}{1.168402in}}%
\pgfpathlineto{\pgfqpoint{1.140415in}{1.167440in}}%
\pgfpathlineto{\pgfqpoint{1.140551in}{1.171901in}}%
\pgfpathlineto{\pgfqpoint{1.142721in}{1.270750in}}%
\pgfpathlineto{\pgfqpoint{1.142856in}{1.270660in}}%
\pgfpathlineto{\pgfqpoint{1.143263in}{1.260328in}}%
\pgfpathlineto{\pgfqpoint{1.143806in}{1.282649in}}%
\pgfpathlineto{\pgfqpoint{1.144213in}{1.276802in}}%
\pgfpathlineto{\pgfqpoint{1.144484in}{1.282554in}}%
\pgfpathlineto{\pgfqpoint{1.145840in}{1.296411in}}%
\pgfpathlineto{\pgfqpoint{1.146247in}{1.292052in}}%
\pgfpathlineto{\pgfqpoint{1.147739in}{1.299408in}}%
\pgfpathlineto{\pgfqpoint{1.147874in}{1.296432in}}%
\pgfpathlineto{\pgfqpoint{1.149231in}{1.280617in}}%
\pgfpathlineto{\pgfqpoint{1.149773in}{1.284514in}}%
\pgfpathlineto{\pgfqpoint{1.153435in}{1.180708in}}%
\pgfpathlineto{\pgfqpoint{1.153978in}{0.887546in}}%
\pgfpathlineto{\pgfqpoint{1.154791in}{1.172622in}}%
\pgfpathlineto{\pgfqpoint{1.156826in}{1.274868in}}%
\pgfpathlineto{\pgfqpoint{1.160894in}{1.329692in}}%
\pgfpathlineto{\pgfqpoint{1.161301in}{1.327325in}}%
\pgfpathlineto{\pgfqpoint{1.162658in}{1.336372in}}%
\pgfpathlineto{\pgfqpoint{1.163336in}{1.334606in}}%
\pgfpathlineto{\pgfqpoint{1.163607in}{1.334970in}}%
\pgfpathlineto{\pgfqpoint{1.163743in}{1.334310in}}%
\pgfpathlineto{\pgfqpoint{1.165099in}{1.330236in}}%
\pgfpathlineto{\pgfqpoint{1.164692in}{1.337517in}}%
\pgfpathlineto{\pgfqpoint{1.165370in}{1.331891in}}%
\pgfpathlineto{\pgfqpoint{1.165641in}{1.334025in}}%
\pgfpathlineto{\pgfqpoint{1.166184in}{1.324026in}}%
\pgfpathlineto{\pgfqpoint{1.166726in}{1.329478in}}%
\pgfpathlineto{\pgfqpoint{1.172016in}{1.229470in}}%
\pgfpathlineto{\pgfqpoint{1.173643in}{1.000314in}}%
\pgfpathlineto{\pgfqpoint{1.174321in}{1.142218in}}%
\pgfpathlineto{\pgfqpoint{1.175813in}{1.247619in}}%
\pgfpathlineto{\pgfqpoint{1.176356in}{1.239392in}}%
\pgfpathlineto{\pgfqpoint{1.178933in}{1.294685in}}%
\pgfpathlineto{\pgfqpoint{1.180018in}{1.287748in}}%
\pgfpathlineto{\pgfqpoint{1.180424in}{1.293561in}}%
\pgfpathlineto{\pgfqpoint{1.181509in}{1.296084in}}%
\pgfpathlineto{\pgfqpoint{1.180967in}{1.288960in}}%
\pgfpathlineto{\pgfqpoint{1.181781in}{1.291690in}}%
\pgfpathlineto{\pgfqpoint{1.183137in}{1.279768in}}%
\pgfpathlineto{\pgfqpoint{1.182594in}{1.294828in}}%
\pgfpathlineto{\pgfqpoint{1.183408in}{1.286645in}}%
\pgfpathlineto{\pgfqpoint{1.183544in}{1.289139in}}%
\pgfpathlineto{\pgfqpoint{1.184086in}{1.276967in}}%
\pgfpathlineto{\pgfqpoint{1.184629in}{1.282164in}}%
\pgfpathlineto{\pgfqpoint{1.187070in}{1.160460in}}%
\pgfpathlineto{\pgfqpoint{1.187613in}{1.176183in}}%
\pgfpathlineto{\pgfqpoint{1.188562in}{1.039510in}}%
\pgfpathlineto{\pgfqpoint{1.188969in}{1.201620in}}%
\pgfpathlineto{\pgfqpoint{1.191817in}{1.300476in}}%
\pgfpathlineto{\pgfqpoint{1.191953in}{1.300434in}}%
\pgfpathlineto{\pgfqpoint{1.192359in}{1.294764in}}%
\pgfpathlineto{\pgfqpoint{1.192902in}{1.310612in}}%
\pgfpathlineto{\pgfqpoint{1.193173in}{1.307090in}}%
\pgfpathlineto{\pgfqpoint{1.193444in}{1.309203in}}%
\pgfpathlineto{\pgfqpoint{1.194801in}{1.325349in}}%
\pgfpathlineto{\pgfqpoint{1.195208in}{1.321945in}}%
\pgfpathlineto{\pgfqpoint{1.195343in}{1.320666in}}%
\pgfpathlineto{\pgfqpoint{1.195750in}{1.329324in}}%
\pgfpathlineto{\pgfqpoint{1.195886in}{1.328969in}}%
\pgfpathlineto{\pgfqpoint{1.196428in}{1.329414in}}%
\pgfpathlineto{\pgfqpoint{1.196699in}{1.333594in}}%
\pgfpathlineto{\pgfqpoint{1.197378in}{1.327897in}}%
\pgfpathlineto{\pgfqpoint{1.197920in}{1.330051in}}%
\pgfpathlineto{\pgfqpoint{1.199276in}{1.322085in}}%
\pgfpathlineto{\pgfqpoint{1.199683in}{1.325886in}}%
\pgfpathlineto{\pgfqpoint{1.203752in}{1.292442in}}%
\pgfpathlineto{\pgfqpoint{1.206058in}{1.231893in}}%
\pgfpathlineto{\pgfqpoint{1.206329in}{1.244187in}}%
\pgfpathlineto{\pgfqpoint{1.206871in}{1.191689in}}%
\pgfpathlineto{\pgfqpoint{1.207685in}{1.109604in}}%
\pgfpathlineto{\pgfqpoint{1.208499in}{0.839476in}}%
\pgfpathlineto{\pgfqpoint{1.208906in}{1.164243in}}%
\pgfpathlineto{\pgfqpoint{1.209177in}{1.140113in}}%
\pgfpathlineto{\pgfqpoint{1.209313in}{1.138998in}}%
\pgfpathlineto{\pgfqpoint{1.211483in}{1.257220in}}%
\pgfpathlineto{\pgfqpoint{1.213517in}{1.286919in}}%
\pgfpathlineto{\pgfqpoint{1.213653in}{1.292233in}}%
\pgfpathlineto{\pgfqpoint{1.214195in}{1.283896in}}%
\pgfpathlineto{\pgfqpoint{1.214873in}{1.284326in}}%
\pgfpathlineto{\pgfqpoint{1.216229in}{1.278326in}}%
\pgfpathlineto{\pgfqpoint{1.215687in}{1.292495in}}%
\pgfpathlineto{\pgfqpoint{1.216501in}{1.279522in}}%
\pgfpathlineto{\pgfqpoint{1.217314in}{1.280475in}}%
\pgfpathlineto{\pgfqpoint{1.217721in}{1.273385in}}%
\pgfpathlineto{\pgfqpoint{1.221654in}{1.130017in}}%
\pgfpathlineto{\pgfqpoint{1.221926in}{0.972082in}}%
\pgfpathlineto{\pgfqpoint{1.223011in}{1.187545in}}%
\pgfpathlineto{\pgfqpoint{1.223418in}{1.175198in}}%
\pgfpathlineto{\pgfqpoint{1.223824in}{1.216206in}}%
\pgfpathlineto{\pgfqpoint{1.229792in}{1.327148in}}%
\pgfpathlineto{\pgfqpoint{1.230199in}{1.324261in}}%
\pgfpathlineto{\pgfqpoint{1.230741in}{1.330190in}}%
\pgfpathlineto{\pgfqpoint{1.231284in}{1.328883in}}%
\pgfpathlineto{\pgfqpoint{1.231419in}{1.331400in}}%
\pgfpathlineto{\pgfqpoint{1.231691in}{1.335110in}}%
\pgfpathlineto{\pgfqpoint{1.232233in}{1.331065in}}%
\pgfpathlineto{\pgfqpoint{1.232911in}{1.331156in}}%
\pgfpathlineto{\pgfqpoint{1.235217in}{1.318402in}}%
\pgfpathlineto{\pgfqpoint{1.235759in}{1.315718in}}%
\pgfpathlineto{\pgfqpoint{1.239964in}{1.243779in}}%
\pgfpathlineto{\pgfqpoint{1.241863in}{1.069122in}}%
\pgfpathlineto{\pgfqpoint{1.241998in}{0.949507in}}%
\pgfpathlineto{\pgfqpoint{1.243083in}{1.208903in}}%
\pgfpathlineto{\pgfqpoint{1.243219in}{1.202004in}}%
\pgfpathlineto{\pgfqpoint{1.243354in}{1.199865in}}%
\pgfpathlineto{\pgfqpoint{1.243626in}{1.219379in}}%
\pgfpathlineto{\pgfqpoint{1.245660in}{1.272637in}}%
\pgfpathlineto{\pgfqpoint{1.247694in}{1.288844in}}%
\pgfpathlineto{\pgfqpoint{1.248644in}{1.296102in}}%
\pgfpathlineto{\pgfqpoint{1.247966in}{1.288468in}}%
\pgfpathlineto{\pgfqpoint{1.249593in}{1.293068in}}%
\pgfpathlineto{\pgfqpoint{1.249864in}{1.291652in}}%
\pgfpathlineto{\pgfqpoint{1.252170in}{1.271060in}}%
\pgfpathlineto{\pgfqpoint{1.250543in}{1.293992in}}%
\pgfpathlineto{\pgfqpoint{1.252441in}{1.274016in}}%
\pgfpathlineto{\pgfqpoint{1.252577in}{1.274505in}}%
\pgfpathlineto{\pgfqpoint{1.255968in}{1.071562in}}%
\pgfpathlineto{\pgfqpoint{1.256239in}{0.770819in}}%
\pgfpathlineto{\pgfqpoint{1.257053in}{1.192371in}}%
\pgfpathlineto{\pgfqpoint{1.257324in}{1.171826in}}%
\pgfpathlineto{\pgfqpoint{1.261799in}{1.320260in}}%
\pgfpathlineto{\pgfqpoint{1.261935in}{1.319116in}}%
\pgfpathlineto{\pgfqpoint{1.262206in}{1.315649in}}%
\pgfpathlineto{\pgfqpoint{1.262884in}{1.327462in}}%
\pgfpathlineto{\pgfqpoint{1.263291in}{1.323986in}}%
\pgfpathlineto{\pgfqpoint{1.263834in}{1.330093in}}%
\pgfpathlineto{\pgfqpoint{1.264376in}{1.325748in}}%
\pgfpathlineto{\pgfqpoint{1.264783in}{1.333278in}}%
\pgfpathlineto{\pgfqpoint{1.266004in}{1.328033in}}%
\pgfpathlineto{\pgfqpoint{1.266546in}{1.331832in}}%
\pgfpathlineto{\pgfqpoint{1.266953in}{1.326516in}}%
\pgfpathlineto{\pgfqpoint{1.267089in}{1.324531in}}%
\pgfpathlineto{\pgfqpoint{1.267631in}{1.329886in}}%
\pgfpathlineto{\pgfqpoint{1.268309in}{1.328176in}}%
\pgfpathlineto{\pgfqpoint{1.268445in}{1.329332in}}%
\pgfpathlineto{\pgfqpoint{1.268852in}{1.322102in}}%
\pgfpathlineto{\pgfqpoint{1.274277in}{1.220633in}}%
\pgfpathlineto{\pgfqpoint{1.275498in}{1.029932in}}%
\pgfpathlineto{\pgfqpoint{1.276176in}{1.157878in}}%
\pgfpathlineto{\pgfqpoint{1.277803in}{1.266447in}}%
\pgfpathlineto{\pgfqpoint{1.278210in}{1.264609in}}%
\pgfpathlineto{\pgfqpoint{1.280651in}{1.307007in}}%
\pgfpathlineto{\pgfqpoint{1.281194in}{1.304409in}}%
\pgfpathlineto{\pgfqpoint{1.281736in}{1.311934in}}%
\pgfpathlineto{\pgfqpoint{1.281872in}{1.310033in}}%
\pgfpathlineto{\pgfqpoint{1.282550in}{1.317324in}}%
\pgfpathlineto{\pgfqpoint{1.282957in}{1.308855in}}%
\pgfpathlineto{\pgfqpoint{1.283093in}{1.305976in}}%
\pgfpathlineto{\pgfqpoint{1.283771in}{1.315501in}}%
\pgfpathlineto{\pgfqpoint{1.284178in}{1.313349in}}%
\pgfpathlineto{\pgfqpoint{1.284449in}{1.317398in}}%
\pgfpathlineto{\pgfqpoint{1.284991in}{1.310499in}}%
\pgfpathlineto{\pgfqpoint{1.285669in}{1.311871in}}%
\pgfpathlineto{\pgfqpoint{1.290959in}{1.193471in}}%
\pgfpathlineto{\pgfqpoint{1.292315in}{0.932693in}}%
\pgfpathlineto{\pgfqpoint{1.292722in}{1.136371in}}%
\pgfpathlineto{\pgfqpoint{1.294756in}{1.271433in}}%
\pgfpathlineto{\pgfqpoint{1.299774in}{1.319495in}}%
\pgfpathlineto{\pgfqpoint{1.295028in}{1.269990in}}%
\pgfpathlineto{\pgfqpoint{1.300046in}{1.318590in}}%
\pgfpathlineto{\pgfqpoint{1.300181in}{1.318262in}}%
\pgfpathlineto{\pgfqpoint{1.300588in}{1.320214in}}%
\pgfpathlineto{\pgfqpoint{1.301538in}{1.327086in}}%
\pgfpathlineto{\pgfqpoint{1.301944in}{1.320630in}}%
\pgfpathlineto{\pgfqpoint{1.305064in}{1.276748in}}%
\pgfpathlineto{\pgfqpoint{1.305199in}{1.281088in}}%
\pgfpathlineto{\pgfqpoint{1.305335in}{1.283687in}}%
\pgfpathlineto{\pgfqpoint{1.305742in}{1.268009in}}%
\pgfpathlineto{\pgfqpoint{1.306420in}{1.271016in}}%
\pgfpathlineto{\pgfqpoint{1.309133in}{0.993591in}}%
\pgfpathlineto{\pgfqpoint{1.309268in}{0.925202in}}%
\pgfpathlineto{\pgfqpoint{1.310218in}{1.195258in}}%
\pgfpathlineto{\pgfqpoint{1.310353in}{1.193711in}}%
\pgfpathlineto{\pgfqpoint{1.310489in}{1.208916in}}%
\pgfpathlineto{\pgfqpoint{1.312523in}{1.274359in}}%
\pgfpathlineto{\pgfqpoint{1.314829in}{1.297373in}}%
\pgfpathlineto{\pgfqpoint{1.315100in}{1.295379in}}%
\pgfpathlineto{\pgfqpoint{1.315507in}{1.306647in}}%
\pgfpathlineto{\pgfqpoint{1.316049in}{1.299805in}}%
\pgfpathlineto{\pgfqpoint{1.317677in}{1.315341in}}%
\pgfpathlineto{\pgfqpoint{1.318084in}{1.303407in}}%
\pgfpathlineto{\pgfqpoint{1.318491in}{1.308423in}}%
\pgfpathlineto{\pgfqpoint{1.320796in}{1.283006in}}%
\pgfpathlineto{\pgfqpoint{1.325136in}{1.071798in}}%
\pgfpathlineto{\pgfqpoint{1.325408in}{1.147405in}}%
\pgfpathlineto{\pgfqpoint{1.326764in}{1.246056in}}%
\pgfpathlineto{\pgfqpoint{1.327306in}{1.235307in}}%
\pgfpathlineto{\pgfqpoint{1.328934in}{1.290557in}}%
\pgfpathlineto{\pgfqpoint{1.329069in}{1.285861in}}%
\pgfpathlineto{\pgfqpoint{1.329205in}{1.283027in}}%
\pgfpathlineto{\pgfqpoint{1.329612in}{1.300215in}}%
\pgfpathlineto{\pgfqpoint{1.331646in}{1.320209in}}%
\pgfpathlineto{\pgfqpoint{1.331782in}{1.318537in}}%
\pgfpathlineto{\pgfqpoint{1.332189in}{1.309201in}}%
\pgfpathlineto{\pgfqpoint{1.333274in}{1.314727in}}%
\pgfpathlineto{\pgfqpoint{1.333545in}{1.318581in}}%
\pgfpathlineto{\pgfqpoint{1.334088in}{1.310368in}}%
\pgfpathlineto{\pgfqpoint{1.334630in}{1.313376in}}%
\pgfpathlineto{\pgfqpoint{1.334901in}{1.307995in}}%
\pgfpathlineto{\pgfqpoint{1.337071in}{1.289815in}}%
\pgfpathlineto{\pgfqpoint{1.335444in}{1.309606in}}%
\pgfpathlineto{\pgfqpoint{1.337207in}{1.292983in}}%
\pgfpathlineto{\pgfqpoint{1.337343in}{1.296272in}}%
\pgfpathlineto{\pgfqpoint{1.338021in}{1.276402in}}%
\pgfpathlineto{\pgfqpoint{1.338428in}{1.286451in}}%
\pgfpathlineto{\pgfqpoint{1.341411in}{1.176949in}}%
\pgfpathlineto{\pgfqpoint{1.341954in}{0.917875in}}%
\pgfpathlineto{\pgfqpoint{1.342768in}{1.207474in}}%
\pgfpathlineto{\pgfqpoint{1.343174in}{1.184302in}}%
\pgfpathlineto{\pgfqpoint{1.343717in}{1.240682in}}%
\pgfpathlineto{\pgfqpoint{1.347379in}{1.313001in}}%
\pgfpathlineto{\pgfqpoint{1.349684in}{1.331659in}}%
\pgfpathlineto{\pgfqpoint{1.350905in}{1.322517in}}%
\pgfpathlineto{\pgfqpoint{1.351312in}{1.325135in}}%
\pgfpathlineto{\pgfqpoint{1.351448in}{1.328666in}}%
\pgfpathlineto{\pgfqpoint{1.351990in}{1.320562in}}%
\pgfpathlineto{\pgfqpoint{1.352668in}{1.320809in}}%
\pgfpathlineto{\pgfqpoint{1.352804in}{1.320798in}}%
\pgfpathlineto{\pgfqpoint{1.352939in}{1.318003in}}%
\pgfpathlineto{\pgfqpoint{1.353346in}{1.323799in}}%
\pgfpathlineto{\pgfqpoint{1.354160in}{1.321718in}}%
\pgfpathlineto{\pgfqpoint{1.354296in}{1.326059in}}%
\pgfpathlineto{\pgfqpoint{1.354974in}{1.308450in}}%
\pgfpathlineto{\pgfqpoint{1.355516in}{1.319799in}}%
\pgfpathlineto{\pgfqpoint{1.358364in}{1.275158in}}%
\pgfpathlineto{\pgfqpoint{1.360534in}{1.205850in}}%
\pgfpathlineto{\pgfqpoint{1.361484in}{1.108702in}}%
\pgfpathlineto{\pgfqpoint{1.361619in}{0.956131in}}%
\pgfpathlineto{\pgfqpoint{1.362840in}{1.216951in}}%
\pgfpathlineto{\pgfqpoint{1.364874in}{1.267965in}}%
\pgfpathlineto{\pgfqpoint{1.365010in}{1.266925in}}%
\pgfpathlineto{\pgfqpoint{1.365146in}{1.266822in}}%
\pgfpathlineto{\pgfqpoint{1.366773in}{1.291977in}}%
\pgfpathlineto{\pgfqpoint{1.367451in}{1.283509in}}%
\pgfpathlineto{\pgfqpoint{1.367994in}{1.292695in}}%
\pgfpathlineto{\pgfqpoint{1.368536in}{1.283177in}}%
\pgfpathlineto{\pgfqpoint{1.368943in}{1.272554in}}%
\pgfpathlineto{\pgfqpoint{1.370028in}{1.275537in}}%
\pgfpathlineto{\pgfqpoint{1.370299in}{1.282173in}}%
\pgfpathlineto{\pgfqpoint{1.370978in}{1.262156in}}%
\pgfpathlineto{\pgfqpoint{1.371249in}{1.265599in}}%
\pgfpathlineto{\pgfqpoint{1.371520in}{1.267362in}}%
\pgfpathlineto{\pgfqpoint{1.371791in}{1.261973in}}%
\pgfpathlineto{\pgfqpoint{1.372605in}{1.247436in}}%
\pgfpathlineto{\pgfqpoint{1.374368in}{1.123217in}}%
\pgfpathlineto{\pgfqpoint{1.374504in}{0.847701in}}%
\pgfpathlineto{\pgfqpoint{1.375860in}{1.216263in}}%
\pgfpathlineto{\pgfqpoint{1.376267in}{1.199014in}}%
\pgfpathlineto{\pgfqpoint{1.376674in}{1.244596in}}%
\pgfpathlineto{\pgfqpoint{1.379658in}{1.310521in}}%
\pgfpathlineto{\pgfqpoint{1.377352in}{1.236514in}}%
\pgfpathlineto{\pgfqpoint{1.379793in}{1.308535in}}%
\pgfpathlineto{\pgfqpoint{1.380200in}{1.306416in}}%
\pgfpathlineto{\pgfqpoint{1.380743in}{1.321275in}}%
\pgfpathlineto{\pgfqpoint{1.381828in}{1.320340in}}%
\pgfpathlineto{\pgfqpoint{1.381963in}{1.315709in}}%
\pgfpathlineto{\pgfqpoint{1.382641in}{1.330798in}}%
\pgfpathlineto{\pgfqpoint{1.383184in}{1.323005in}}%
\pgfpathlineto{\pgfqpoint{1.384676in}{1.335064in}}%
\pgfpathlineto{\pgfqpoint{1.384947in}{1.328029in}}%
\pgfpathlineto{\pgfqpoint{1.386439in}{1.337457in}}%
\pgfpathlineto{\pgfqpoint{1.386846in}{1.331533in}}%
\pgfpathlineto{\pgfqpoint{1.387931in}{1.325576in}}%
\pgfpathlineto{\pgfqpoint{1.387388in}{1.332449in}}%
\pgfpathlineto{\pgfqpoint{1.388338in}{1.327203in}}%
\pgfpathlineto{\pgfqpoint{1.388473in}{1.329294in}}%
\pgfpathlineto{\pgfqpoint{1.389287in}{1.320689in}}%
\pgfpathlineto{\pgfqpoint{1.391457in}{1.297145in}}%
\pgfpathlineto{\pgfqpoint{1.394441in}{1.125569in}}%
\pgfpathlineto{\pgfqpoint{1.394576in}{0.869381in}}%
\pgfpathlineto{\pgfqpoint{1.395661in}{1.202566in}}%
\pgfpathlineto{\pgfqpoint{1.395933in}{1.166128in}}%
\pgfpathlineto{\pgfqpoint{1.396068in}{1.133958in}}%
\pgfpathlineto{\pgfqpoint{1.396746in}{1.238044in}}%
\pgfpathlineto{\pgfqpoint{1.397153in}{1.210039in}}%
\pgfpathlineto{\pgfqpoint{1.399459in}{1.281860in}}%
\pgfpathlineto{\pgfqpoint{1.399594in}{1.281991in}}%
\pgfpathlineto{\pgfqpoint{1.399730in}{1.281337in}}%
\pgfpathlineto{\pgfqpoint{1.400001in}{1.270977in}}%
\pgfpathlineto{\pgfqpoint{1.400544in}{1.293849in}}%
\pgfpathlineto{\pgfqpoint{1.401222in}{1.280942in}}%
\pgfpathlineto{\pgfqpoint{1.402443in}{1.297860in}}%
\pgfpathlineto{\pgfqpoint{1.402036in}{1.273314in}}%
\pgfpathlineto{\pgfqpoint{1.402714in}{1.287073in}}%
\pgfpathlineto{\pgfqpoint{1.403934in}{1.277052in}}%
\pgfpathlineto{\pgfqpoint{1.403392in}{1.305826in}}%
\pgfpathlineto{\pgfqpoint{1.404206in}{1.289321in}}%
\pgfpathlineto{\pgfqpoint{1.404477in}{1.294104in}}%
\pgfpathlineto{\pgfqpoint{1.404884in}{1.267894in}}%
\pgfpathlineto{\pgfqpoint{1.405833in}{1.258403in}}%
\pgfpathlineto{\pgfqpoint{1.405426in}{1.279966in}}%
\pgfpathlineto{\pgfqpoint{1.406240in}{1.271451in}}%
\pgfpathlineto{\pgfqpoint{1.406376in}{1.277202in}}%
\pgfpathlineto{\pgfqpoint{1.406783in}{1.236229in}}%
\pgfpathlineto{\pgfqpoint{1.408681in}{1.050439in}}%
\pgfpathlineto{\pgfqpoint{1.409359in}{1.169093in}}%
\pgfpathlineto{\pgfqpoint{1.411258in}{1.256588in}}%
\pgfpathlineto{\pgfqpoint{1.413428in}{1.304986in}}%
\pgfpathlineto{\pgfqpoint{1.413971in}{1.300881in}}%
\pgfpathlineto{\pgfqpoint{1.414242in}{1.308960in}}%
\pgfpathlineto{\pgfqpoint{1.416412in}{1.330168in}}%
\pgfpathlineto{\pgfqpoint{1.416819in}{1.328032in}}%
\pgfpathlineto{\pgfqpoint{1.417090in}{1.333224in}}%
\pgfpathlineto{\pgfqpoint{1.418446in}{1.342295in}}%
\pgfpathlineto{\pgfqpoint{1.417904in}{1.331536in}}%
\pgfpathlineto{\pgfqpoint{1.418718in}{1.335652in}}%
\pgfpathlineto{\pgfqpoint{1.418853in}{1.334416in}}%
\pgfpathlineto{\pgfqpoint{1.419260in}{1.344443in}}%
\pgfpathlineto{\pgfqpoint{1.419396in}{1.345434in}}%
\pgfpathlineto{\pgfqpoint{1.419803in}{1.337860in}}%
\pgfpathlineto{\pgfqpoint{1.419938in}{1.336948in}}%
\pgfpathlineto{\pgfqpoint{1.420074in}{1.341061in}}%
\pgfpathlineto{\pgfqpoint{1.420481in}{1.351371in}}%
\pgfpathlineto{\pgfqpoint{1.421566in}{1.342381in}}%
\pgfpathlineto{\pgfqpoint{1.421973in}{1.334337in}}%
\pgfpathlineto{\pgfqpoint{1.422379in}{1.345987in}}%
\pgfpathlineto{\pgfqpoint{1.423193in}{1.336669in}}%
\pgfpathlineto{\pgfqpoint{1.429296in}{1.202430in}}%
\pgfpathlineto{\pgfqpoint{1.430381in}{0.852707in}}%
\pgfpathlineto{\pgfqpoint{1.431195in}{1.098839in}}%
\pgfpathlineto{\pgfqpoint{1.432823in}{1.225685in}}%
\pgfpathlineto{\pgfqpoint{1.432958in}{1.224039in}}%
\pgfpathlineto{\pgfqpoint{1.433094in}{1.223508in}}%
\pgfpathlineto{\pgfqpoint{1.433229in}{1.224187in}}%
\pgfpathlineto{\pgfqpoint{1.433501in}{1.240853in}}%
\pgfpathlineto{\pgfqpoint{1.434179in}{1.199124in}}%
\pgfpathlineto{\pgfqpoint{1.434857in}{1.232212in}}%
\pgfpathlineto{\pgfqpoint{1.436484in}{1.260150in}}%
\pgfpathlineto{\pgfqpoint{1.436620in}{1.259843in}}%
\pgfpathlineto{\pgfqpoint{1.437841in}{1.245159in}}%
\pgfpathlineto{\pgfqpoint{1.437163in}{1.272627in}}%
\pgfpathlineto{\pgfqpoint{1.438112in}{1.255412in}}%
\pgfpathlineto{\pgfqpoint{1.439333in}{1.270494in}}%
\pgfpathlineto{\pgfqpoint{1.438654in}{1.249468in}}%
\pgfpathlineto{\pgfqpoint{1.439604in}{1.255130in}}%
\pgfpathlineto{\pgfqpoint{1.441774in}{1.115220in}}%
\pgfpathlineto{\pgfqpoint{1.440146in}{1.255713in}}%
\pgfpathlineto{\pgfqpoint{1.442045in}{1.164133in}}%
\pgfpathlineto{\pgfqpoint{1.442181in}{1.167934in}}%
\pgfpathlineto{\pgfqpoint{1.442316in}{1.145334in}}%
\pgfpathlineto{\pgfqpoint{1.442452in}{1.017551in}}%
\pgfpathlineto{\pgfqpoint{1.443673in}{1.243073in}}%
\pgfpathlineto{\pgfqpoint{1.443808in}{1.243846in}}%
\pgfpathlineto{\pgfqpoint{1.444079in}{1.224553in}}%
\pgfpathlineto{\pgfqpoint{1.444758in}{1.261871in}}%
\pgfpathlineto{\pgfqpoint{1.445164in}{1.250201in}}%
\pgfpathlineto{\pgfqpoint{1.447470in}{1.314130in}}%
\pgfpathlineto{\pgfqpoint{1.449640in}{1.336844in}}%
\pgfpathlineto{\pgfqpoint{1.448148in}{1.305114in}}%
\pgfpathlineto{\pgfqpoint{1.449911in}{1.329493in}}%
\pgfpathlineto{\pgfqpoint{1.450047in}{1.324574in}}%
\pgfpathlineto{\pgfqpoint{1.450589in}{1.337984in}}%
\pgfpathlineto{\pgfqpoint{1.451132in}{1.333035in}}%
\pgfpathlineto{\pgfqpoint{1.451539in}{1.348944in}}%
\pgfpathlineto{\pgfqpoint{1.452759in}{1.342475in}}%
\pgfpathlineto{\pgfqpoint{1.453031in}{1.335712in}}%
\pgfpathlineto{\pgfqpoint{1.453573in}{1.349384in}}%
\pgfpathlineto{\pgfqpoint{1.454116in}{1.345076in}}%
\pgfpathlineto{\pgfqpoint{1.455472in}{1.351728in}}%
\pgfpathlineto{\pgfqpoint{1.454929in}{1.343781in}}%
\pgfpathlineto{\pgfqpoint{1.455608in}{1.347481in}}%
\pgfpathlineto{\pgfqpoint{1.455879in}{1.340996in}}%
\pgfpathlineto{\pgfqpoint{1.457235in}{1.344442in}}%
\pgfpathlineto{\pgfqpoint{1.457371in}{1.344543in}}%
\pgfpathlineto{\pgfqpoint{1.459812in}{1.316904in}}%
\pgfpathlineto{\pgfqpoint{1.460083in}{1.321712in}}%
\pgfpathlineto{\pgfqpoint{1.460219in}{1.324279in}}%
\pgfpathlineto{\pgfqpoint{1.460626in}{1.310549in}}%
\pgfpathlineto{\pgfqpoint{1.461033in}{1.314764in}}%
\pgfpathlineto{\pgfqpoint{1.464152in}{1.213630in}}%
\pgfpathlineto{\pgfqpoint{1.464694in}{1.037408in}}%
\pgfpathlineto{\pgfqpoint{1.465915in}{1.143998in}}%
\pgfpathlineto{\pgfqpoint{1.467543in}{1.274134in}}%
\pgfpathlineto{\pgfqpoint{1.467949in}{1.270440in}}%
\pgfpathlineto{\pgfqpoint{1.469306in}{1.310670in}}%
\pgfpathlineto{\pgfqpoint{1.469713in}{1.297808in}}%
\pgfpathlineto{\pgfqpoint{1.469848in}{1.295317in}}%
\pgfpathlineto{\pgfqpoint{1.470119in}{1.309399in}}%
\pgfpathlineto{\pgfqpoint{1.470391in}{1.320061in}}%
\pgfpathlineto{\pgfqpoint{1.470933in}{1.295003in}}%
\pgfpathlineto{\pgfqpoint{1.471476in}{1.311991in}}%
\pgfpathlineto{\pgfqpoint{1.472696in}{1.288677in}}%
\pgfpathlineto{\pgfqpoint{1.473103in}{1.306888in}}%
\pgfpathlineto{\pgfqpoint{1.473374in}{1.315516in}}%
\pgfpathlineto{\pgfqpoint{1.473781in}{1.280182in}}%
\pgfpathlineto{\pgfqpoint{1.474324in}{1.296322in}}%
\pgfpathlineto{\pgfqpoint{1.476629in}{1.191753in}}%
\pgfpathlineto{\pgfqpoint{1.476901in}{1.213554in}}%
\pgfpathlineto{\pgfqpoint{1.477036in}{1.227389in}}%
\pgfpathlineto{\pgfqpoint{1.477443in}{1.126712in}}%
\pgfpathlineto{\pgfqpoint{1.477579in}{0.920489in}}%
\pgfpathlineto{\pgfqpoint{1.478664in}{1.225128in}}%
\pgfpathlineto{\pgfqpoint{1.478799in}{1.214602in}}%
\pgfpathlineto{\pgfqpoint{1.479071in}{1.202615in}}%
\pgfpathlineto{\pgfqpoint{1.479206in}{1.219667in}}%
\pgfpathlineto{\pgfqpoint{1.480563in}{1.283699in}}%
\pgfpathlineto{\pgfqpoint{1.480969in}{1.273194in}}%
\pgfpathlineto{\pgfqpoint{1.484224in}{1.321128in}}%
\pgfpathlineto{\pgfqpoint{1.484360in}{1.320345in}}%
\pgfpathlineto{\pgfqpoint{1.484767in}{1.302820in}}%
\pgfpathlineto{\pgfqpoint{1.485309in}{1.327694in}}%
\pgfpathlineto{\pgfqpoint{1.485988in}{1.313736in}}%
\pgfpathlineto{\pgfqpoint{1.486259in}{1.322633in}}%
\pgfpathlineto{\pgfqpoint{1.486801in}{1.303533in}}%
\pgfpathlineto{\pgfqpoint{1.487479in}{1.316016in}}%
\pgfpathlineto{\pgfqpoint{1.488700in}{1.293409in}}%
\pgfpathlineto{\pgfqpoint{1.488158in}{1.319328in}}%
\pgfpathlineto{\pgfqpoint{1.488971in}{1.311781in}}%
\pgfpathlineto{\pgfqpoint{1.489243in}{1.320587in}}%
\pgfpathlineto{\pgfqpoint{1.489649in}{1.301548in}}%
\pgfpathlineto{\pgfqpoint{1.490328in}{1.309817in}}%
\pgfpathlineto{\pgfqpoint{1.491684in}{1.279143in}}%
\pgfpathlineto{\pgfqpoint{1.491141in}{1.312189in}}%
\pgfpathlineto{\pgfqpoint{1.491955in}{1.295307in}}%
\pgfpathlineto{\pgfqpoint{1.492091in}{1.295032in}}%
\pgfpathlineto{\pgfqpoint{1.495210in}{1.184082in}}%
\pgfpathlineto{\pgfqpoint{1.496159in}{0.949860in}}%
\pgfpathlineto{\pgfqpoint{1.496702in}{1.232899in}}%
\pgfpathlineto{\pgfqpoint{1.496973in}{1.195430in}}%
\pgfpathlineto{\pgfqpoint{1.497651in}{1.257427in}}%
\pgfpathlineto{\pgfqpoint{1.498058in}{1.247076in}}%
\pgfpathlineto{\pgfqpoint{1.500364in}{1.313872in}}%
\pgfpathlineto{\pgfqpoint{1.502398in}{1.330007in}}%
\pgfpathlineto{\pgfqpoint{1.502534in}{1.328852in}}%
\pgfpathlineto{\pgfqpoint{1.502941in}{1.324813in}}%
\pgfpathlineto{\pgfqpoint{1.504026in}{1.329024in}}%
\pgfpathlineto{\pgfqpoint{1.504161in}{1.328512in}}%
\pgfpathlineto{\pgfqpoint{1.504297in}{1.331731in}}%
\pgfpathlineto{\pgfqpoint{1.504839in}{1.330194in}}%
\pgfpathlineto{\pgfqpoint{1.505246in}{1.334287in}}%
\pgfpathlineto{\pgfqpoint{1.505924in}{1.328738in}}%
\pgfpathlineto{\pgfqpoint{1.506331in}{1.330637in}}%
\pgfpathlineto{\pgfqpoint{1.509586in}{1.284107in}}%
\pgfpathlineto{\pgfqpoint{1.509858in}{1.290498in}}%
\pgfpathlineto{\pgfqpoint{1.510264in}{1.275568in}}%
\pgfpathlineto{\pgfqpoint{1.513384in}{1.109505in}}%
\pgfpathlineto{\pgfqpoint{1.513519in}{0.976796in}}%
\pgfpathlineto{\pgfqpoint{1.514469in}{1.232411in}}%
\pgfpathlineto{\pgfqpoint{1.514740in}{1.218321in}}%
\pgfpathlineto{\pgfqpoint{1.515011in}{1.209739in}}%
\pgfpathlineto{\pgfqpoint{1.515147in}{1.222351in}}%
\pgfpathlineto{\pgfqpoint{1.516503in}{1.298918in}}%
\pgfpathlineto{\pgfqpoint{1.516910in}{1.285367in}}%
\pgfpathlineto{\pgfqpoint{1.517046in}{1.285937in}}%
\pgfpathlineto{\pgfqpoint{1.518402in}{1.322927in}}%
\pgfpathlineto{\pgfqpoint{1.518673in}{1.312118in}}%
\pgfpathlineto{\pgfqpoint{1.518944in}{1.293362in}}%
\pgfpathlineto{\pgfqpoint{1.519487in}{1.316039in}}%
\pgfpathlineto{\pgfqpoint{1.520165in}{1.311025in}}%
\pgfpathlineto{\pgfqpoint{1.520436in}{1.322502in}}%
\pgfpathlineto{\pgfqpoint{1.520843in}{1.303296in}}%
\pgfpathlineto{\pgfqpoint{1.521521in}{1.312807in}}%
\pgfpathlineto{\pgfqpoint{1.521928in}{1.294625in}}%
\pgfpathlineto{\pgfqpoint{1.522335in}{1.315117in}}%
\pgfpathlineto{\pgfqpoint{1.523013in}{1.305752in}}%
\pgfpathlineto{\pgfqpoint{1.524369in}{1.325075in}}%
\pgfpathlineto{\pgfqpoint{1.523827in}{1.304965in}}%
\pgfpathlineto{\pgfqpoint{1.524505in}{1.318194in}}%
\pgfpathlineto{\pgfqpoint{1.525861in}{1.292698in}}%
\pgfpathlineto{\pgfqpoint{1.526133in}{1.302865in}}%
\pgfpathlineto{\pgfqpoint{1.526268in}{1.302931in}}%
\pgfpathlineto{\pgfqpoint{1.528709in}{1.178935in}}%
\pgfpathlineto{\pgfqpoint{1.528981in}{1.228485in}}%
\pgfpathlineto{\pgfqpoint{1.529252in}{1.239218in}}%
\pgfpathlineto{\pgfqpoint{1.529659in}{1.204137in}}%
\pgfpathlineto{\pgfqpoint{1.530066in}{1.212992in}}%
\pgfpathlineto{\pgfqpoint{1.530337in}{0.768065in}}%
\pgfpathlineto{\pgfqpoint{1.531422in}{1.231870in}}%
\pgfpathlineto{\pgfqpoint{1.533185in}{1.275337in}}%
\pgfpathlineto{\pgfqpoint{1.534406in}{1.291301in}}%
\pgfpathlineto{\pgfqpoint{1.533863in}{1.242932in}}%
\pgfpathlineto{\pgfqpoint{1.534813in}{1.280979in}}%
\pgfpathlineto{\pgfqpoint{1.534948in}{1.280162in}}%
\pgfpathlineto{\pgfqpoint{1.536304in}{1.306910in}}%
\pgfpathlineto{\pgfqpoint{1.535762in}{1.270672in}}%
\pgfpathlineto{\pgfqpoint{1.536576in}{1.295991in}}%
\pgfpathlineto{\pgfqpoint{1.537661in}{1.277946in}}%
\pgfpathlineto{\pgfqpoint{1.537254in}{1.303413in}}%
\pgfpathlineto{\pgfqpoint{1.538068in}{1.294308in}}%
\pgfpathlineto{\pgfqpoint{1.539288in}{1.311390in}}%
\pgfpathlineto{\pgfqpoint{1.538746in}{1.269542in}}%
\pgfpathlineto{\pgfqpoint{1.539424in}{1.302996in}}%
\pgfpathlineto{\pgfqpoint{1.540644in}{1.270314in}}%
\pgfpathlineto{\pgfqpoint{1.540238in}{1.309140in}}%
\pgfpathlineto{\pgfqpoint{1.541051in}{1.291116in}}%
\pgfpathlineto{\pgfqpoint{1.541187in}{1.293907in}}%
\pgfpathlineto{\pgfqpoint{1.541323in}{1.286854in}}%
\pgfpathlineto{\pgfqpoint{1.541594in}{1.246560in}}%
\pgfpathlineto{\pgfqpoint{1.542136in}{1.299609in}}%
\pgfpathlineto{\pgfqpoint{1.542814in}{1.274394in}}%
\pgfpathlineto{\pgfqpoint{1.543086in}{1.299915in}}%
\pgfpathlineto{\pgfqpoint{1.543493in}{1.266547in}}%
\pgfpathlineto{\pgfqpoint{1.544306in}{1.278144in}}%
\pgfpathlineto{\pgfqpoint{1.545663in}{1.117783in}}%
\pgfpathlineto{\pgfqpoint{1.545934in}{1.224899in}}%
\pgfpathlineto{\pgfqpoint{1.546069in}{1.237863in}}%
\pgfpathlineto{\pgfqpoint{1.546341in}{1.148199in}}%
\pgfpathlineto{\pgfqpoint{1.547154in}{1.042182in}}%
\pgfpathlineto{\pgfqpoint{1.547561in}{1.219562in}}%
\pgfpathlineto{\pgfqpoint{1.548104in}{1.142811in}}%
\pgfpathlineto{\pgfqpoint{1.548511in}{1.244616in}}%
\pgfpathlineto{\pgfqpoint{1.548918in}{1.234471in}}%
\pgfpathlineto{\pgfqpoint{1.551223in}{1.313855in}}%
\pgfpathlineto{\pgfqpoint{1.551359in}{1.313737in}}%
\pgfpathlineto{\pgfqpoint{1.551494in}{1.314109in}}%
\pgfpathlineto{\pgfqpoint{1.551766in}{1.302132in}}%
\pgfpathlineto{\pgfqpoint{1.552308in}{1.321907in}}%
\pgfpathlineto{\pgfqpoint{1.552986in}{1.315915in}}%
\pgfpathlineto{\pgfqpoint{1.554207in}{1.333730in}}%
\pgfpathlineto{\pgfqpoint{1.553664in}{1.309882in}}%
\pgfpathlineto{\pgfqpoint{1.554614in}{1.326672in}}%
\pgfpathlineto{\pgfqpoint{1.555563in}{1.322882in}}%
\pgfpathlineto{\pgfqpoint{1.555156in}{1.334569in}}%
\pgfpathlineto{\pgfqpoint{1.555699in}{1.327181in}}%
\pgfpathlineto{\pgfqpoint{1.556106in}{1.340503in}}%
\pgfpathlineto{\pgfqpoint{1.556513in}{1.326225in}}%
\pgfpathlineto{\pgfqpoint{1.557326in}{1.331868in}}%
\pgfpathlineto{\pgfqpoint{1.558547in}{1.306685in}}%
\pgfpathlineto{\pgfqpoint{1.558954in}{1.320066in}}%
\pgfpathlineto{\pgfqpoint{1.559089in}{1.319937in}}%
\pgfpathlineto{\pgfqpoint{1.561531in}{1.204937in}}%
\pgfpathlineto{\pgfqpoint{1.561802in}{1.227034in}}%
\pgfpathlineto{\pgfqpoint{1.562073in}{1.241455in}}%
\pgfpathlineto{\pgfqpoint{1.562344in}{1.146582in}}%
\pgfpathlineto{\pgfqpoint{1.562480in}{1.065656in}}%
\pgfpathlineto{\pgfqpoint{1.563565in}{1.205662in}}%
\pgfpathlineto{\pgfqpoint{1.563836in}{1.167939in}}%
\pgfpathlineto{\pgfqpoint{1.563972in}{1.165264in}}%
\pgfpathlineto{\pgfqpoint{1.565464in}{1.280049in}}%
\pgfpathlineto{\pgfqpoint{1.565871in}{1.256794in}}%
\pgfpathlineto{\pgfqpoint{1.566006in}{1.255848in}}%
\pgfpathlineto{\pgfqpoint{1.566142in}{1.264729in}}%
\pgfpathlineto{\pgfqpoint{1.567634in}{1.308069in}}%
\pgfpathlineto{\pgfqpoint{1.567905in}{1.305166in}}%
\pgfpathlineto{\pgfqpoint{1.569804in}{1.330778in}}%
\pgfpathlineto{\pgfqpoint{1.570075in}{1.330142in}}%
\pgfpathlineto{\pgfqpoint{1.570753in}{1.337250in}}%
\pgfpathlineto{\pgfqpoint{1.571296in}{1.346673in}}%
\pgfpathlineto{\pgfqpoint{1.572516in}{1.342211in}}%
\pgfpathlineto{\pgfqpoint{1.574008in}{1.327917in}}%
\pgfpathlineto{\pgfqpoint{1.573059in}{1.343980in}}%
\pgfpathlineto{\pgfqpoint{1.574415in}{1.335809in}}%
\pgfpathlineto{\pgfqpoint{1.574551in}{1.339801in}}%
\pgfpathlineto{\pgfqpoint{1.575636in}{1.333447in}}%
\pgfpathlineto{\pgfqpoint{1.577670in}{1.303646in}}%
\pgfpathlineto{\pgfqpoint{1.577806in}{1.304380in}}%
\pgfpathlineto{\pgfqpoint{1.578077in}{1.299484in}}%
\pgfpathlineto{\pgfqpoint{1.579433in}{1.281979in}}%
\pgfpathlineto{\pgfqpoint{1.579704in}{1.288486in}}%
\pgfpathlineto{\pgfqpoint{1.579840in}{1.297916in}}%
\pgfpathlineto{\pgfqpoint{1.580654in}{1.264822in}}%
\pgfpathlineto{\pgfqpoint{1.581061in}{1.285161in}}%
\pgfpathlineto{\pgfqpoint{1.583502in}{1.011984in}}%
\pgfpathlineto{\pgfqpoint{1.584316in}{1.126589in}}%
\pgfpathlineto{\pgfqpoint{1.585401in}{1.235797in}}%
\pgfpathlineto{\pgfqpoint{1.584858in}{0.963882in}}%
\pgfpathlineto{\pgfqpoint{1.585808in}{1.204262in}}%
\pgfpathlineto{\pgfqpoint{1.585943in}{1.162579in}}%
\pgfpathlineto{\pgfqpoint{1.587164in}{1.261453in}}%
\pgfpathlineto{\pgfqpoint{1.587299in}{1.264752in}}%
\pgfpathlineto{\pgfqpoint{1.587571in}{1.243743in}}%
\pgfpathlineto{\pgfqpoint{1.588791in}{1.202295in}}%
\pgfpathlineto{\pgfqpoint{1.588249in}{1.260197in}}%
\pgfpathlineto{\pgfqpoint{1.589063in}{1.250046in}}%
\pgfpathlineto{\pgfqpoint{1.590283in}{1.286839in}}%
\pgfpathlineto{\pgfqpoint{1.589741in}{1.243102in}}%
\pgfpathlineto{\pgfqpoint{1.590554in}{1.269215in}}%
\pgfpathlineto{\pgfqpoint{1.591775in}{1.199520in}}%
\pgfpathlineto{\pgfqpoint{1.591233in}{1.289678in}}%
\pgfpathlineto{\pgfqpoint{1.592046in}{1.248366in}}%
\pgfpathlineto{\pgfqpoint{1.592182in}{1.265884in}}%
\pgfpathlineto{\pgfqpoint{1.592724in}{1.171267in}}%
\pgfpathlineto{\pgfqpoint{1.593267in}{1.222536in}}%
\pgfpathlineto{\pgfqpoint{1.593809in}{1.049101in}}%
\pgfpathlineto{\pgfqpoint{1.595166in}{1.115191in}}%
\pgfpathlineto{\pgfqpoint{1.595301in}{1.096000in}}%
\pgfpathlineto{\pgfqpoint{1.595708in}{1.222583in}}%
\pgfpathlineto{\pgfqpoint{1.596115in}{1.184818in}}%
\pgfpathlineto{\pgfqpoint{1.598421in}{1.316387in}}%
\pgfpathlineto{\pgfqpoint{1.598556in}{1.316880in}}%
\pgfpathlineto{\pgfqpoint{1.598963in}{1.299175in}}%
\pgfpathlineto{\pgfqpoint{1.599370in}{1.328440in}}%
\pgfpathlineto{\pgfqpoint{1.600048in}{1.314735in}}%
\pgfpathlineto{\pgfqpoint{1.601404in}{1.342801in}}%
\pgfpathlineto{\pgfqpoint{1.601811in}{1.332849in}}%
\pgfpathlineto{\pgfqpoint{1.603439in}{1.358185in}}%
\pgfpathlineto{\pgfqpoint{1.603710in}{1.349521in}}%
\pgfpathlineto{\pgfqpoint{1.606287in}{1.371360in}}%
\pgfpathlineto{\pgfqpoint{1.606423in}{1.367574in}}%
\pgfpathlineto{\pgfqpoint{1.607643in}{1.358054in}}%
\pgfpathlineto{\pgfqpoint{1.607101in}{1.374716in}}%
\pgfpathlineto{\pgfqpoint{1.607914in}{1.366767in}}%
\pgfpathlineto{\pgfqpoint{1.608050in}{1.370343in}}%
\pgfpathlineto{\pgfqpoint{1.608593in}{1.358220in}}%
\pgfpathlineto{\pgfqpoint{1.609271in}{1.366974in}}%
\pgfpathlineto{\pgfqpoint{1.610491in}{1.352748in}}%
\pgfpathlineto{\pgfqpoint{1.609949in}{1.368290in}}%
\pgfpathlineto{\pgfqpoint{1.610898in}{1.363296in}}%
\pgfpathlineto{\pgfqpoint{1.611034in}{1.366701in}}%
\pgfpathlineto{\pgfqpoint{1.611441in}{1.355766in}}%
\pgfpathlineto{\pgfqpoint{1.612119in}{1.359035in}}%
\pgfpathlineto{\pgfqpoint{1.613475in}{1.344641in}}%
\pgfpathlineto{\pgfqpoint{1.612933in}{1.359588in}}%
\pgfpathlineto{\pgfqpoint{1.613746in}{1.351373in}}%
\pgfpathlineto{\pgfqpoint{1.613882in}{1.355976in}}%
\pgfpathlineto{\pgfqpoint{1.614560in}{1.333451in}}%
\pgfpathlineto{\pgfqpoint{1.614831in}{1.341125in}}%
\pgfpathlineto{\pgfqpoint{1.616323in}{1.316099in}}%
\pgfpathlineto{\pgfqpoint{1.616730in}{1.323880in}}%
\pgfpathlineto{\pgfqpoint{1.616866in}{1.325479in}}%
\pgfpathlineto{\pgfqpoint{1.617137in}{1.314362in}}%
\pgfpathlineto{\pgfqpoint{1.617815in}{1.317390in}}%
\pgfpathlineto{\pgfqpoint{1.620799in}{1.120065in}}%
\pgfpathlineto{\pgfqpoint{1.622155in}{1.151572in}}%
\pgfpathlineto{\pgfqpoint{1.622562in}{0.789535in}}%
\pgfpathlineto{\pgfqpoint{1.623918in}{1.189400in}}%
\pgfpathlineto{\pgfqpoint{1.624189in}{1.026196in}}%
\pgfpathlineto{\pgfqpoint{1.625546in}{1.219312in}}%
\pgfpathlineto{\pgfqpoint{1.625817in}{1.161546in}}%
\pgfpathlineto{\pgfqpoint{1.626495in}{1.252524in}}%
\pgfpathlineto{\pgfqpoint{1.627580in}{1.204251in}}%
\pgfpathlineto{\pgfqpoint{1.627851in}{1.149291in}}%
\pgfpathlineto{\pgfqpoint{1.628394in}{1.262038in}}%
\pgfpathlineto{\pgfqpoint{1.628936in}{1.214679in}}%
\pgfpathlineto{\pgfqpoint{1.630564in}{1.294672in}}%
\pgfpathlineto{\pgfqpoint{1.630699in}{1.287567in}}%
\pgfpathlineto{\pgfqpoint{1.634361in}{1.344361in}}%
\pgfpathlineto{\pgfqpoint{1.634633in}{1.341281in}}%
\pgfpathlineto{\pgfqpoint{1.634904in}{1.339180in}}%
\pgfpathlineto{\pgfqpoint{1.635039in}{1.343601in}}%
\pgfpathlineto{\pgfqpoint{1.635446in}{1.347942in}}%
\pgfpathlineto{\pgfqpoint{1.635718in}{1.339623in}}%
\pgfpathlineto{\pgfqpoint{1.636531in}{1.342729in}}%
\pgfpathlineto{\pgfqpoint{1.636938in}{1.333095in}}%
\pgfpathlineto{\pgfqpoint{1.637752in}{1.349707in}}%
\pgfpathlineto{\pgfqpoint{1.638159in}{1.355901in}}%
\pgfpathlineto{\pgfqpoint{1.639379in}{1.358423in}}%
\pgfpathlineto{\pgfqpoint{1.638837in}{1.352025in}}%
\pgfpathlineto{\pgfqpoint{1.639651in}{1.355898in}}%
\pgfpathlineto{\pgfqpoint{1.639786in}{1.353827in}}%
\pgfpathlineto{\pgfqpoint{1.640464in}{1.366997in}}%
\pgfpathlineto{\pgfqpoint{1.641956in}{1.359796in}}%
\pgfpathlineto{\pgfqpoint{1.641278in}{1.368296in}}%
\pgfpathlineto{\pgfqpoint{1.642228in}{1.363166in}}%
\pgfpathlineto{\pgfqpoint{1.642363in}{1.364057in}}%
\pgfpathlineto{\pgfqpoint{1.643313in}{1.361390in}}%
\pgfpathlineto{\pgfqpoint{1.648466in}{1.326530in}}%
\pgfpathlineto{\pgfqpoint{1.648738in}{1.331524in}}%
\pgfpathlineto{\pgfqpoint{1.648873in}{1.335273in}}%
\pgfpathlineto{\pgfqpoint{1.649958in}{1.324376in}}%
\pgfpathlineto{\pgfqpoint{1.654163in}{1.190275in}}%
\pgfpathlineto{\pgfqpoint{1.654298in}{1.079280in}}%
\pgfpathlineto{\pgfqpoint{1.654705in}{1.256769in}}%
\pgfpathlineto{\pgfqpoint{1.655519in}{1.221285in}}%
\pgfpathlineto{\pgfqpoint{1.655654in}{1.251554in}}%
\pgfpathlineto{\pgfqpoint{1.656197in}{1.093440in}}%
\pgfpathlineto{\pgfqpoint{1.656468in}{1.107221in}}%
\pgfpathlineto{\pgfqpoint{1.657282in}{0.914316in}}%
\pgfpathlineto{\pgfqpoint{1.656875in}{1.207146in}}%
\pgfpathlineto{\pgfqpoint{1.657553in}{1.206662in}}%
\pgfpathlineto{\pgfqpoint{1.657689in}{1.239521in}}%
\pgfpathlineto{\pgfqpoint{1.658096in}{1.136333in}}%
\pgfpathlineto{\pgfqpoint{1.658231in}{0.940895in}}%
\pgfpathlineto{\pgfqpoint{1.658638in}{1.213908in}}%
\pgfpathlineto{\pgfqpoint{1.659452in}{1.181230in}}%
\pgfpathlineto{\pgfqpoint{1.660673in}{1.259895in}}%
\pgfpathlineto{\pgfqpoint{1.660130in}{1.080442in}}%
\pgfpathlineto{\pgfqpoint{1.660944in}{1.209298in}}%
\pgfpathlineto{\pgfqpoint{1.661215in}{1.202123in}}%
\pgfpathlineto{\pgfqpoint{1.661351in}{1.233599in}}%
\pgfpathlineto{\pgfqpoint{1.661486in}{1.252312in}}%
\pgfpathlineto{\pgfqpoint{1.662029in}{1.169412in}}%
\pgfpathlineto{\pgfqpoint{1.662707in}{1.230878in}}%
\pgfpathlineto{\pgfqpoint{1.662978in}{1.207735in}}%
\pgfpathlineto{\pgfqpoint{1.663656in}{1.279703in}}%
\pgfpathlineto{\pgfqpoint{1.663928in}{1.259656in}}%
\pgfpathlineto{\pgfqpoint{1.665419in}{1.310055in}}%
\pgfpathlineto{\pgfqpoint{1.666776in}{1.305224in}}%
\pgfpathlineto{\pgfqpoint{1.666911in}{1.291066in}}%
\pgfpathlineto{\pgfqpoint{1.668132in}{1.315315in}}%
\pgfpathlineto{\pgfqpoint{1.668810in}{1.312667in}}%
\pgfpathlineto{\pgfqpoint{1.670166in}{1.344615in}}%
\pgfpathlineto{\pgfqpoint{1.671116in}{1.354099in}}%
\pgfpathlineto{\pgfqpoint{1.670709in}{1.339265in}}%
\pgfpathlineto{\pgfqpoint{1.671523in}{1.344138in}}%
\pgfpathlineto{\pgfqpoint{1.671658in}{1.341722in}}%
\pgfpathlineto{\pgfqpoint{1.672201in}{1.355288in}}%
\pgfpathlineto{\pgfqpoint{1.672743in}{1.345324in}}%
\pgfpathlineto{\pgfqpoint{1.674099in}{1.365075in}}%
\pgfpathlineto{\pgfqpoint{1.674371in}{1.355095in}}%
\pgfpathlineto{\pgfqpoint{1.675591in}{1.345948in}}%
\pgfpathlineto{\pgfqpoint{1.675049in}{1.360792in}}%
\pgfpathlineto{\pgfqpoint{1.675863in}{1.355426in}}%
\pgfpathlineto{\pgfqpoint{1.675998in}{1.359008in}}%
\pgfpathlineto{\pgfqpoint{1.676541in}{1.345148in}}%
\pgfpathlineto{\pgfqpoint{1.676948in}{1.348912in}}%
\pgfpathlineto{\pgfqpoint{1.679389in}{1.303474in}}%
\pgfpathlineto{\pgfqpoint{1.679524in}{1.308544in}}%
\pgfpathlineto{\pgfqpoint{1.679796in}{1.317375in}}%
\pgfpathlineto{\pgfqpoint{1.680338in}{1.286925in}}%
\pgfpathlineto{\pgfqpoint{1.681423in}{1.264527in}}%
\pgfpathlineto{\pgfqpoint{1.680745in}{1.298216in}}%
\pgfpathlineto{\pgfqpoint{1.681559in}{1.276495in}}%
\pgfpathlineto{\pgfqpoint{1.681830in}{1.305749in}}%
\pgfpathlineto{\pgfqpoint{1.682373in}{1.256808in}}%
\pgfpathlineto{\pgfqpoint{1.683051in}{1.286619in}}%
\pgfpathlineto{\pgfqpoint{1.684407in}{1.077050in}}%
\pgfpathlineto{\pgfqpoint{1.685628in}{1.185342in}}%
\pgfpathlineto{\pgfqpoint{1.685763in}{1.206452in}}%
\pgfpathlineto{\pgfqpoint{1.686170in}{1.117817in}}%
\pgfpathlineto{\pgfqpoint{1.686306in}{0.977452in}}%
\pgfpathlineto{\pgfqpoint{1.687391in}{1.178417in}}%
\pgfpathlineto{\pgfqpoint{1.687662in}{1.116284in}}%
\pgfpathlineto{\pgfqpoint{1.687798in}{1.103733in}}%
\pgfpathlineto{\pgfqpoint{1.688069in}{1.202590in}}%
\pgfpathlineto{\pgfqpoint{1.688340in}{1.183010in}}%
\pgfpathlineto{\pgfqpoint{1.689154in}{1.214050in}}%
\pgfpathlineto{\pgfqpoint{1.688611in}{1.155325in}}%
\pgfpathlineto{\pgfqpoint{1.689289in}{1.189566in}}%
\pgfpathlineto{\pgfqpoint{1.689425in}{1.114371in}}%
\pgfpathlineto{\pgfqpoint{1.689968in}{1.208353in}}%
\pgfpathlineto{\pgfqpoint{1.690781in}{1.206919in}}%
\pgfpathlineto{\pgfqpoint{1.692002in}{1.240996in}}%
\pgfpathlineto{\pgfqpoint{1.692273in}{1.212325in}}%
\pgfpathlineto{\pgfqpoint{1.692409in}{1.181969in}}%
\pgfpathlineto{\pgfqpoint{1.692951in}{1.242473in}}%
\pgfpathlineto{\pgfqpoint{1.693765in}{1.209497in}}%
\pgfpathlineto{\pgfqpoint{1.694172in}{1.234668in}}%
\pgfpathlineto{\pgfqpoint{1.694714in}{1.178459in}}%
\pgfpathlineto{\pgfqpoint{1.695257in}{1.004076in}}%
\pgfpathlineto{\pgfqpoint{1.696071in}{1.198213in}}%
\pgfpathlineto{\pgfqpoint{1.696749in}{1.156061in}}%
\pgfpathlineto{\pgfqpoint{1.697563in}{1.190140in}}%
\pgfpathlineto{\pgfqpoint{1.702852in}{1.356549in}}%
\pgfpathlineto{\pgfqpoint{1.705429in}{1.386616in}}%
\pgfpathlineto{\pgfqpoint{1.705971in}{1.382193in}}%
\pgfpathlineto{\pgfqpoint{1.706107in}{1.381959in}}%
\pgfpathlineto{\pgfqpoint{1.707463in}{1.395785in}}%
\pgfpathlineto{\pgfqpoint{1.707870in}{1.389931in}}%
\pgfpathlineto{\pgfqpoint{1.709226in}{1.397580in}}%
\pgfpathlineto{\pgfqpoint{1.709633in}{1.390809in}}%
\pgfpathlineto{\pgfqpoint{1.711803in}{1.378622in}}%
\pgfpathlineto{\pgfqpoint{1.710311in}{1.392154in}}%
\pgfpathlineto{\pgfqpoint{1.712074in}{1.382471in}}%
\pgfpathlineto{\pgfqpoint{1.713159in}{1.387944in}}%
\pgfpathlineto{\pgfqpoint{1.712481in}{1.378728in}}%
\pgfpathlineto{\pgfqpoint{1.713566in}{1.382371in}}%
\pgfpathlineto{\pgfqpoint{1.716686in}{1.348347in}}%
\pgfpathlineto{\pgfqpoint{1.714109in}{1.382788in}}%
\pgfpathlineto{\pgfqpoint{1.716821in}{1.349358in}}%
\pgfpathlineto{\pgfqpoint{1.716957in}{1.350502in}}%
\pgfpathlineto{\pgfqpoint{1.717093in}{1.345959in}}%
\pgfpathlineto{\pgfqpoint{1.719263in}{1.304042in}}%
\pgfpathlineto{\pgfqpoint{1.719398in}{1.304805in}}%
\pgfpathlineto{\pgfqpoint{1.720754in}{1.319612in}}%
\pgfpathlineto{\pgfqpoint{1.720348in}{1.290772in}}%
\pgfpathlineto{\pgfqpoint{1.721026in}{1.310067in}}%
\pgfpathlineto{\pgfqpoint{1.721161in}{1.299960in}}%
\pgfpathlineto{\pgfqpoint{1.721839in}{1.320447in}}%
\pgfpathlineto{\pgfqpoint{1.722518in}{1.306893in}}%
\pgfpathlineto{\pgfqpoint{1.722653in}{1.315269in}}%
\pgfpathlineto{\pgfqpoint{1.723196in}{1.275865in}}%
\pgfpathlineto{\pgfqpoint{1.723874in}{1.311669in}}%
\pgfpathlineto{\pgfqpoint{1.724145in}{1.290952in}}%
\pgfpathlineto{\pgfqpoint{1.724688in}{1.311913in}}%
\pgfpathlineto{\pgfqpoint{1.725366in}{1.304055in}}%
\pgfpathlineto{\pgfqpoint{1.725637in}{1.321558in}}%
\pgfpathlineto{\pgfqpoint{1.726179in}{1.296800in}}%
\pgfpathlineto{\pgfqpoint{1.726858in}{1.303460in}}%
\pgfpathlineto{\pgfqpoint{1.727536in}{1.307030in}}%
\pgfpathlineto{\pgfqpoint{1.729028in}{1.229299in}}%
\pgfpathlineto{\pgfqpoint{1.730519in}{1.311141in}}%
\pgfpathlineto{\pgfqpoint{1.730791in}{1.289042in}}%
\pgfpathlineto{\pgfqpoint{1.730926in}{1.275139in}}%
\pgfpathlineto{\pgfqpoint{1.731469in}{1.318708in}}%
\pgfpathlineto{\pgfqpoint{1.732147in}{1.292550in}}%
\pgfpathlineto{\pgfqpoint{1.733503in}{1.331539in}}%
\pgfpathlineto{\pgfqpoint{1.733774in}{1.323960in}}%
\pgfpathlineto{\pgfqpoint{1.734046in}{1.315978in}}%
\pgfpathlineto{\pgfqpoint{1.734453in}{1.336039in}}%
\pgfpathlineto{\pgfqpoint{1.734995in}{1.331235in}}%
\pgfpathlineto{\pgfqpoint{1.736758in}{1.353544in}}%
\pgfpathlineto{\pgfqpoint{1.736894in}{1.351822in}}%
\pgfpathlineto{\pgfqpoint{1.737301in}{1.364420in}}%
\pgfpathlineto{\pgfqpoint{1.737843in}{1.357069in}}%
\pgfpathlineto{\pgfqpoint{1.739199in}{1.368748in}}%
\pgfpathlineto{\pgfqpoint{1.739471in}{1.364173in}}%
\pgfpathlineto{\pgfqpoint{1.739878in}{1.370932in}}%
\pgfpathlineto{\pgfqpoint{1.740284in}{1.366818in}}%
\pgfpathlineto{\pgfqpoint{1.741641in}{1.351673in}}%
\pgfpathlineto{\pgfqpoint{1.741912in}{1.357063in}}%
\pgfpathlineto{\pgfqpoint{1.742997in}{1.360408in}}%
\pgfpathlineto{\pgfqpoint{1.742590in}{1.348103in}}%
\pgfpathlineto{\pgfqpoint{1.743268in}{1.356973in}}%
\pgfpathlineto{\pgfqpoint{1.744353in}{1.350058in}}%
\pgfpathlineto{\pgfqpoint{1.744624in}{1.354995in}}%
\pgfpathlineto{\pgfqpoint{1.744896in}{1.363375in}}%
\pgfpathlineto{\pgfqpoint{1.745303in}{1.347771in}}%
\pgfpathlineto{\pgfqpoint{1.746116in}{1.354637in}}%
\pgfpathlineto{\pgfqpoint{1.747337in}{1.329119in}}%
\pgfpathlineto{\pgfqpoint{1.747744in}{1.343161in}}%
\pgfpathlineto{\pgfqpoint{1.748015in}{1.332450in}}%
\pgfpathlineto{\pgfqpoint{1.749236in}{1.298825in}}%
\pgfpathlineto{\pgfqpoint{1.749778in}{1.308734in}}%
\pgfpathlineto{\pgfqpoint{1.749914in}{1.307639in}}%
\pgfpathlineto{\pgfqpoint{1.752084in}{1.208738in}}%
\pgfpathlineto{\pgfqpoint{1.752219in}{1.216212in}}%
\pgfpathlineto{\pgfqpoint{1.752626in}{1.246426in}}%
\pgfpathlineto{\pgfqpoint{1.753033in}{1.184698in}}%
\pgfpathlineto{\pgfqpoint{1.753304in}{0.954034in}}%
\pgfpathlineto{\pgfqpoint{1.754254in}{1.191866in}}%
\pgfpathlineto{\pgfqpoint{1.754661in}{1.021194in}}%
\pgfpathlineto{\pgfqpoint{1.756966in}{1.290107in}}%
\pgfpathlineto{\pgfqpoint{1.757238in}{1.287721in}}%
\pgfpathlineto{\pgfqpoint{1.758187in}{1.288072in}}%
\pgfpathlineto{\pgfqpoint{1.759543in}{1.265223in}}%
\pgfpathlineto{\pgfqpoint{1.761035in}{1.294494in}}%
\pgfpathlineto{\pgfqpoint{1.761306in}{1.289659in}}%
\pgfpathlineto{\pgfqpoint{1.761713in}{1.302107in}}%
\pgfpathlineto{\pgfqpoint{1.762527in}{1.283250in}}%
\pgfpathlineto{\pgfqpoint{1.762663in}{1.283649in}}%
\pgfpathlineto{\pgfqpoint{1.762934in}{1.279041in}}%
\pgfpathlineto{\pgfqpoint{1.763069in}{1.282359in}}%
\pgfpathlineto{\pgfqpoint{1.763205in}{1.274816in}}%
\pgfpathlineto{\pgfqpoint{1.763748in}{1.303381in}}%
\pgfpathlineto{\pgfqpoint{1.764561in}{1.280519in}}%
\pgfpathlineto{\pgfqpoint{1.764833in}{1.276689in}}%
\pgfpathlineto{\pgfqpoint{1.765918in}{1.032894in}}%
\pgfpathlineto{\pgfqpoint{1.766460in}{1.239474in}}%
\pgfpathlineto{\pgfqpoint{1.768088in}{1.304909in}}%
\pgfpathlineto{\pgfqpoint{1.768494in}{1.300883in}}%
\pgfpathlineto{\pgfqpoint{1.768766in}{1.291646in}}%
\pgfpathlineto{\pgfqpoint{1.769308in}{1.327779in}}%
\pgfpathlineto{\pgfqpoint{1.769579in}{1.323155in}}%
\pgfpathlineto{\pgfqpoint{1.773106in}{1.363220in}}%
\pgfpathlineto{\pgfqpoint{1.773377in}{1.358121in}}%
\pgfpathlineto{\pgfqpoint{1.774191in}{1.372072in}}%
\pgfpathlineto{\pgfqpoint{1.774598in}{1.365366in}}%
\pgfpathlineto{\pgfqpoint{1.775683in}{1.369996in}}%
\pgfpathlineto{\pgfqpoint{1.776089in}{1.379180in}}%
\pgfpathlineto{\pgfqpoint{1.777039in}{1.367974in}}%
\pgfpathlineto{\pgfqpoint{1.777174in}{1.368622in}}%
\pgfpathlineto{\pgfqpoint{1.777581in}{1.361226in}}%
\pgfpathlineto{\pgfqpoint{1.778124in}{1.373297in}}%
\pgfpathlineto{\pgfqpoint{1.778802in}{1.366071in}}%
\pgfpathlineto{\pgfqpoint{1.779887in}{1.371541in}}%
\pgfpathlineto{\pgfqpoint{1.779616in}{1.365153in}}%
\pgfpathlineto{\pgfqpoint{1.780158in}{1.367205in}}%
\pgfpathlineto{\pgfqpoint{1.782735in}{1.339010in}}%
\pgfpathlineto{\pgfqpoint{1.782871in}{1.341737in}}%
\pgfpathlineto{\pgfqpoint{1.783413in}{1.327004in}}%
\pgfpathlineto{\pgfqpoint{1.783820in}{1.327147in}}%
\pgfpathlineto{\pgfqpoint{1.787075in}{1.151119in}}%
\pgfpathlineto{\pgfqpoint{1.787211in}{1.002730in}}%
\pgfpathlineto{\pgfqpoint{1.788431in}{1.238818in}}%
\pgfpathlineto{\pgfqpoint{1.788703in}{1.206414in}}%
\pgfpathlineto{\pgfqpoint{1.789245in}{1.256397in}}%
\pgfpathlineto{\pgfqpoint{1.789788in}{1.253138in}}%
\pgfpathlineto{\pgfqpoint{1.790194in}{1.288845in}}%
\pgfpathlineto{\pgfqpoint{1.791415in}{1.262510in}}%
\pgfpathlineto{\pgfqpoint{1.791551in}{1.262445in}}%
\pgfpathlineto{\pgfqpoint{1.793992in}{1.324616in}}%
\pgfpathlineto{\pgfqpoint{1.794399in}{1.279685in}}%
\pgfpathlineto{\pgfqpoint{1.795755in}{1.288399in}}%
\pgfpathlineto{\pgfqpoint{1.795891in}{1.304692in}}%
\pgfpathlineto{\pgfqpoint{1.796569in}{1.264454in}}%
\pgfpathlineto{\pgfqpoint{1.797111in}{1.277596in}}%
\pgfpathlineto{\pgfqpoint{1.798332in}{1.169265in}}%
\pgfpathlineto{\pgfqpoint{1.799010in}{1.187000in}}%
\pgfpathlineto{\pgfqpoint{1.799553in}{0.948288in}}%
\pgfpathlineto{\pgfqpoint{1.799824in}{1.192112in}}%
\pgfpathlineto{\pgfqpoint{1.800502in}{1.186504in}}%
\pgfpathlineto{\pgfqpoint{1.800773in}{1.166287in}}%
\pgfpathlineto{\pgfqpoint{1.802536in}{1.279532in}}%
\pgfpathlineto{\pgfqpoint{1.802672in}{1.278876in}}%
\pgfpathlineto{\pgfqpoint{1.802808in}{1.282769in}}%
\pgfpathlineto{\pgfqpoint{1.804164in}{1.331791in}}%
\pgfpathlineto{\pgfqpoint{1.804571in}{1.327619in}}%
\pgfpathlineto{\pgfqpoint{1.808097in}{1.369162in}}%
\pgfpathlineto{\pgfqpoint{1.809046in}{1.366547in}}%
\pgfpathlineto{\pgfqpoint{1.810403in}{1.353592in}}%
\pgfpathlineto{\pgfqpoint{1.810674in}{1.361905in}}%
\pgfpathlineto{\pgfqpoint{1.811759in}{1.366440in}}%
\pgfpathlineto{\pgfqpoint{1.811216in}{1.356165in}}%
\pgfpathlineto{\pgfqpoint{1.812030in}{1.358355in}}%
\pgfpathlineto{\pgfqpoint{1.814471in}{1.332223in}}%
\pgfpathlineto{\pgfqpoint{1.812844in}{1.360395in}}%
\pgfpathlineto{\pgfqpoint{1.814607in}{1.333997in}}%
\pgfpathlineto{\pgfqpoint{1.814743in}{1.335193in}}%
\pgfpathlineto{\pgfqpoint{1.815014in}{1.328172in}}%
\pgfpathlineto{\pgfqpoint{1.817184in}{1.235453in}}%
\pgfpathlineto{\pgfqpoint{1.817591in}{1.266615in}}%
\pgfpathlineto{\pgfqpoint{1.817726in}{1.270192in}}%
\pgfpathlineto{\pgfqpoint{1.817998in}{1.254289in}}%
\pgfpathlineto{\pgfqpoint{1.819083in}{1.157129in}}%
\pgfpathlineto{\pgfqpoint{1.819489in}{1.243014in}}%
\pgfpathlineto{\pgfqpoint{1.819625in}{1.254409in}}%
\pgfpathlineto{\pgfqpoint{1.820032in}{1.193651in}}%
\pgfpathlineto{\pgfqpoint{1.821795in}{0.938169in}}%
\pgfpathlineto{\pgfqpoint{1.821524in}{1.204058in}}%
\pgfpathlineto{\pgfqpoint{1.821931in}{1.075450in}}%
\pgfpathlineto{\pgfqpoint{1.823016in}{1.202726in}}%
\pgfpathlineto{\pgfqpoint{1.822609in}{1.005768in}}%
\pgfpathlineto{\pgfqpoint{1.823558in}{1.177879in}}%
\pgfpathlineto{\pgfqpoint{1.825186in}{1.264405in}}%
\pgfpathlineto{\pgfqpoint{1.825321in}{1.258565in}}%
\pgfpathlineto{\pgfqpoint{1.827763in}{0.982070in}}%
\pgfpathlineto{\pgfqpoint{1.828576in}{1.015298in}}%
\pgfpathlineto{\pgfqpoint{1.829390in}{1.172951in}}%
\pgfpathlineto{\pgfqpoint{1.829797in}{0.878167in}}%
\pgfpathlineto{\pgfqpoint{1.830068in}{1.142929in}}%
\pgfpathlineto{\pgfqpoint{1.830339in}{0.932098in}}%
\pgfpathlineto{\pgfqpoint{1.831289in}{1.180391in}}%
\pgfpathlineto{\pgfqpoint{1.833052in}{1.291761in}}%
\pgfpathlineto{\pgfqpoint{1.833188in}{1.282837in}}%
\pgfpathlineto{\pgfqpoint{1.833730in}{1.275135in}}%
\pgfpathlineto{\pgfqpoint{1.834001in}{1.300947in}}%
\pgfpathlineto{\pgfqpoint{1.834137in}{1.299056in}}%
\pgfpathlineto{\pgfqpoint{1.834544in}{1.314276in}}%
\pgfpathlineto{\pgfqpoint{1.834679in}{1.313402in}}%
\pgfpathlineto{\pgfqpoint{1.834951in}{1.320233in}}%
\pgfpathlineto{\pgfqpoint{1.837256in}{1.347216in}}%
\pgfpathlineto{\pgfqpoint{1.837528in}{1.345086in}}%
\pgfpathlineto{\pgfqpoint{1.837934in}{1.350693in}}%
\pgfpathlineto{\pgfqpoint{1.838341in}{1.349436in}}%
\pgfpathlineto{\pgfqpoint{1.841189in}{1.367587in}}%
\pgfpathlineto{\pgfqpoint{1.841325in}{1.366294in}}%
\pgfpathlineto{\pgfqpoint{1.841732in}{1.358874in}}%
\pgfpathlineto{\pgfqpoint{1.842139in}{1.366903in}}%
\pgfpathlineto{\pgfqpoint{1.842817in}{1.365707in}}%
\pgfpathlineto{\pgfqpoint{1.842953in}{1.367077in}}%
\pgfpathlineto{\pgfqpoint{1.843224in}{1.360789in}}%
\pgfpathlineto{\pgfqpoint{1.843495in}{1.351587in}}%
\pgfpathlineto{\pgfqpoint{1.844309in}{1.368409in}}%
\pgfpathlineto{\pgfqpoint{1.844716in}{1.362285in}}%
\pgfpathlineto{\pgfqpoint{1.845123in}{1.370380in}}%
\pgfpathlineto{\pgfqpoint{1.846208in}{1.365546in}}%
\pgfpathlineto{\pgfqpoint{1.848649in}{1.343185in}}%
\pgfpathlineto{\pgfqpoint{1.848920in}{1.349075in}}%
\pgfpathlineto{\pgfqpoint{1.849598in}{1.333028in}}%
\pgfpathlineto{\pgfqpoint{1.849869in}{1.340753in}}%
\pgfpathlineto{\pgfqpoint{1.851768in}{1.273651in}}%
\pgfpathlineto{\pgfqpoint{1.853803in}{0.966243in}}%
\pgfpathlineto{\pgfqpoint{1.854074in}{1.090805in}}%
\pgfpathlineto{\pgfqpoint{1.855023in}{1.068560in}}%
\pgfpathlineto{\pgfqpoint{1.856108in}{1.242977in}}%
\pgfpathlineto{\pgfqpoint{1.857058in}{1.282547in}}%
\pgfpathlineto{\pgfqpoint{1.857871in}{1.275344in}}%
\pgfpathlineto{\pgfqpoint{1.859228in}{1.299320in}}%
\pgfpathlineto{\pgfqpoint{1.858414in}{1.245507in}}%
\pgfpathlineto{\pgfqpoint{1.859363in}{1.280734in}}%
\pgfpathlineto{\pgfqpoint{1.859499in}{1.279113in}}%
\pgfpathlineto{\pgfqpoint{1.859634in}{1.286636in}}%
\pgfpathlineto{\pgfqpoint{1.861126in}{1.320921in}}%
\pgfpathlineto{\pgfqpoint{1.861398in}{1.310274in}}%
\pgfpathlineto{\pgfqpoint{1.862618in}{1.295631in}}%
\pgfpathlineto{\pgfqpoint{1.861940in}{1.314131in}}%
\pgfpathlineto{\pgfqpoint{1.862754in}{1.310875in}}%
\pgfpathlineto{\pgfqpoint{1.863839in}{1.320724in}}%
\pgfpathlineto{\pgfqpoint{1.863296in}{1.294565in}}%
\pgfpathlineto{\pgfqpoint{1.864110in}{1.305363in}}%
\pgfpathlineto{\pgfqpoint{1.864517in}{1.292234in}}%
\pgfpathlineto{\pgfqpoint{1.864924in}{1.323523in}}%
\pgfpathlineto{\pgfqpoint{1.865602in}{1.303111in}}%
\pgfpathlineto{\pgfqpoint{1.866823in}{1.316379in}}%
\pgfpathlineto{\pgfqpoint{1.866280in}{1.290768in}}%
\pgfpathlineto{\pgfqpoint{1.866958in}{1.312002in}}%
\pgfpathlineto{\pgfqpoint{1.867365in}{1.269200in}}%
\pgfpathlineto{\pgfqpoint{1.868450in}{1.293134in}}%
\pgfpathlineto{\pgfqpoint{1.868721in}{1.312615in}}%
\pgfpathlineto{\pgfqpoint{1.869264in}{1.272158in}}%
\pgfpathlineto{\pgfqpoint{1.869942in}{1.294791in}}%
\pgfpathlineto{\pgfqpoint{1.871298in}{1.241153in}}%
\pgfpathlineto{\pgfqpoint{1.870620in}{1.296353in}}%
\pgfpathlineto{\pgfqpoint{1.871434in}{1.269389in}}%
\pgfpathlineto{\pgfqpoint{1.871705in}{1.290927in}}%
\pgfpathlineto{\pgfqpoint{1.872112in}{1.243149in}}%
\pgfpathlineto{\pgfqpoint{1.872926in}{1.270880in}}%
\pgfpathlineto{\pgfqpoint{1.873604in}{1.279157in}}%
\pgfpathlineto{\pgfqpoint{1.875231in}{1.170083in}}%
\pgfpathlineto{\pgfqpoint{1.875503in}{1.225181in}}%
\pgfpathlineto{\pgfqpoint{1.876045in}{0.931819in}}%
\pgfpathlineto{\pgfqpoint{1.876723in}{1.189877in}}%
\pgfpathlineto{\pgfqpoint{1.876994in}{1.125332in}}%
\pgfpathlineto{\pgfqpoint{1.877401in}{1.231408in}}%
\pgfpathlineto{\pgfqpoint{1.878215in}{1.168603in}}%
\pgfpathlineto{\pgfqpoint{1.878351in}{1.214563in}}%
\pgfpathlineto{\pgfqpoint{1.878758in}{1.082542in}}%
\pgfpathlineto{\pgfqpoint{1.878893in}{0.694405in}}%
\pgfpathlineto{\pgfqpoint{1.880114in}{1.266226in}}%
\pgfpathlineto{\pgfqpoint{1.880385in}{1.283348in}}%
\pgfpathlineto{\pgfqpoint{1.881063in}{1.244154in}}%
\pgfpathlineto{\pgfqpoint{1.881470in}{1.263502in}}%
\pgfpathlineto{\pgfqpoint{1.881606in}{1.260580in}}%
\pgfpathlineto{\pgfqpoint{1.881877in}{1.275485in}}%
\pgfpathlineto{\pgfqpoint{1.882013in}{1.267343in}}%
\pgfpathlineto{\pgfqpoint{1.883776in}{1.292690in}}%
\pgfpathlineto{\pgfqpoint{1.885132in}{1.337508in}}%
\pgfpathlineto{\pgfqpoint{1.885539in}{1.317046in}}%
\pgfpathlineto{\pgfqpoint{1.888658in}{1.366857in}}%
\pgfpathlineto{\pgfqpoint{1.891913in}{1.392923in}}%
\pgfpathlineto{\pgfqpoint{1.892456in}{1.385134in}}%
\pgfpathlineto{\pgfqpoint{1.892591in}{1.382119in}}%
\pgfpathlineto{\pgfqpoint{1.893134in}{1.388396in}}%
\pgfpathlineto{\pgfqpoint{1.893812in}{1.384869in}}%
\pgfpathlineto{\pgfqpoint{1.893948in}{1.389255in}}%
\pgfpathlineto{\pgfqpoint{1.895304in}{1.386901in}}%
\pgfpathlineto{\pgfqpoint{1.895846in}{1.388134in}}%
\pgfpathlineto{\pgfqpoint{1.896931in}{1.392082in}}%
\pgfpathlineto{\pgfqpoint{1.896389in}{1.386161in}}%
\pgfpathlineto{\pgfqpoint{1.897203in}{1.387093in}}%
\pgfpathlineto{\pgfqpoint{1.897609in}{1.383588in}}%
\pgfpathlineto{\pgfqpoint{1.898016in}{1.395252in}}%
\pgfpathlineto{\pgfqpoint{1.898559in}{1.387705in}}%
\pgfpathlineto{\pgfqpoint{1.898694in}{1.387986in}}%
\pgfpathlineto{\pgfqpoint{1.898830in}{1.387179in}}%
\pgfpathlineto{\pgfqpoint{1.898966in}{1.387547in}}%
\pgfpathlineto{\pgfqpoint{1.901271in}{1.373162in}}%
\pgfpathlineto{\pgfqpoint{1.901407in}{1.373147in}}%
\pgfpathlineto{\pgfqpoint{1.902221in}{1.363492in}}%
\pgfpathlineto{\pgfqpoint{1.901814in}{1.374376in}}%
\pgfpathlineto{\pgfqpoint{1.902763in}{1.373213in}}%
\pgfpathlineto{\pgfqpoint{1.902899in}{1.376236in}}%
\pgfpathlineto{\pgfqpoint{1.903306in}{1.370260in}}%
\pgfpathlineto{\pgfqpoint{1.904119in}{1.370453in}}%
\pgfpathlineto{\pgfqpoint{1.904798in}{1.375932in}}%
\pgfpathlineto{\pgfqpoint{1.904526in}{1.368159in}}%
\pgfpathlineto{\pgfqpoint{1.905611in}{1.370373in}}%
\pgfpathlineto{\pgfqpoint{1.906154in}{1.361106in}}%
\pgfpathlineto{\pgfqpoint{1.908188in}{1.341039in}}%
\pgfpathlineto{\pgfqpoint{1.906561in}{1.362203in}}%
\pgfpathlineto{\pgfqpoint{1.908324in}{1.343637in}}%
\pgfpathlineto{\pgfqpoint{1.908459in}{1.348159in}}%
\pgfpathlineto{\pgfqpoint{1.909138in}{1.330164in}}%
\pgfpathlineto{\pgfqpoint{1.909680in}{1.340036in}}%
\pgfpathlineto{\pgfqpoint{1.911986in}{1.294935in}}%
\pgfpathlineto{\pgfqpoint{1.912528in}{1.331199in}}%
\pgfpathlineto{\pgfqpoint{1.913749in}{1.317988in}}%
\pgfpathlineto{\pgfqpoint{1.914834in}{1.298737in}}%
\pgfpathlineto{\pgfqpoint{1.914427in}{1.318566in}}%
\pgfpathlineto{\pgfqpoint{1.915105in}{1.310718in}}%
\pgfpathlineto{\pgfqpoint{1.915376in}{1.329670in}}%
\pgfpathlineto{\pgfqpoint{1.916054in}{1.293529in}}%
\pgfpathlineto{\pgfqpoint{1.916461in}{1.308015in}}%
\pgfpathlineto{\pgfqpoint{1.916868in}{1.291078in}}%
\pgfpathlineto{\pgfqpoint{1.917139in}{1.321203in}}%
\pgfpathlineto{\pgfqpoint{1.918360in}{1.338020in}}%
\pgfpathlineto{\pgfqpoint{1.917682in}{1.293679in}}%
\pgfpathlineto{\pgfqpoint{1.918496in}{1.324446in}}%
\pgfpathlineto{\pgfqpoint{1.919581in}{1.304910in}}%
\pgfpathlineto{\pgfqpoint{1.919174in}{1.330411in}}%
\pgfpathlineto{\pgfqpoint{1.919988in}{1.325707in}}%
\pgfpathlineto{\pgfqpoint{1.920801in}{1.320330in}}%
\pgfpathlineto{\pgfqpoint{1.922022in}{1.359469in}}%
\pgfpathlineto{\pgfqpoint{1.923107in}{1.373788in}}%
\pgfpathlineto{\pgfqpoint{1.922700in}{1.358913in}}%
\pgfpathlineto{\pgfqpoint{1.923514in}{1.361918in}}%
\pgfpathlineto{\pgfqpoint{1.923649in}{1.358080in}}%
\pgfpathlineto{\pgfqpoint{1.924056in}{1.374938in}}%
\pgfpathlineto{\pgfqpoint{1.924734in}{1.369991in}}%
\pgfpathlineto{\pgfqpoint{1.926091in}{1.388603in}}%
\pgfpathlineto{\pgfqpoint{1.926362in}{1.384469in}}%
\pgfpathlineto{\pgfqpoint{1.926498in}{1.381100in}}%
\pgfpathlineto{\pgfqpoint{1.927040in}{1.393252in}}%
\pgfpathlineto{\pgfqpoint{1.927718in}{1.387593in}}%
\pgfpathlineto{\pgfqpoint{1.928939in}{1.401870in}}%
\pgfpathlineto{\pgfqpoint{1.929210in}{1.394186in}}%
\pgfpathlineto{\pgfqpoint{1.929346in}{1.388974in}}%
\pgfpathlineto{\pgfqpoint{1.930024in}{1.403760in}}%
\pgfpathlineto{\pgfqpoint{1.930566in}{1.396886in}}%
\pgfpathlineto{\pgfqpoint{1.931923in}{1.405724in}}%
\pgfpathlineto{\pgfqpoint{1.931244in}{1.396420in}}%
\pgfpathlineto{\pgfqpoint{1.932194in}{1.401407in}}%
\pgfpathlineto{\pgfqpoint{1.933279in}{1.397509in}}%
\pgfpathlineto{\pgfqpoint{1.932872in}{1.403952in}}%
\pgfpathlineto{\pgfqpoint{1.933686in}{1.401178in}}%
\pgfpathlineto{\pgfqpoint{1.933821in}{1.401741in}}%
\pgfpathlineto{\pgfqpoint{1.934093in}{1.396571in}}%
\pgfpathlineto{\pgfqpoint{1.934228in}{1.395694in}}%
\pgfpathlineto{\pgfqpoint{1.934635in}{1.403049in}}%
\pgfpathlineto{\pgfqpoint{1.935720in}{1.405759in}}%
\pgfpathlineto{\pgfqpoint{1.935313in}{1.397431in}}%
\pgfpathlineto{\pgfqpoint{1.935856in}{1.402836in}}%
\pgfpathlineto{\pgfqpoint{1.936263in}{1.393505in}}%
\pgfpathlineto{\pgfqpoint{1.936805in}{1.403889in}}%
\pgfpathlineto{\pgfqpoint{1.937483in}{1.396586in}}%
\pgfpathlineto{\pgfqpoint{1.938704in}{1.398520in}}%
\pgfpathlineto{\pgfqpoint{1.938297in}{1.391130in}}%
\pgfpathlineto{\pgfqpoint{1.938839in}{1.394505in}}%
\pgfpathlineto{\pgfqpoint{1.940060in}{1.378383in}}%
\pgfpathlineto{\pgfqpoint{1.940467in}{1.384106in}}%
\pgfpathlineto{\pgfqpoint{1.940603in}{1.385500in}}%
\pgfpathlineto{\pgfqpoint{1.940874in}{1.376585in}}%
\pgfpathlineto{\pgfqpoint{1.941145in}{1.372559in}}%
\pgfpathlineto{\pgfqpoint{1.941688in}{1.385687in}}%
\pgfpathlineto{\pgfqpoint{1.942366in}{1.377875in}}%
\pgfpathlineto{\pgfqpoint{1.942501in}{1.380730in}}%
\pgfpathlineto{\pgfqpoint{1.943044in}{1.366617in}}%
\pgfpathlineto{\pgfqpoint{1.943451in}{1.370515in}}%
\pgfpathlineto{\pgfqpoint{1.945892in}{1.344269in}}%
\pgfpathlineto{\pgfqpoint{1.946028in}{1.340308in}}%
\pgfpathlineto{\pgfqpoint{1.946570in}{1.360245in}}%
\pgfpathlineto{\pgfqpoint{1.946977in}{1.354525in}}%
\pgfpathlineto{\pgfqpoint{1.947248in}{1.359179in}}%
\pgfpathlineto{\pgfqpoint{1.947926in}{1.354396in}}%
\pgfpathlineto{\pgfqpoint{1.949147in}{1.375536in}}%
\pgfpathlineto{\pgfqpoint{1.950368in}{1.390061in}}%
\pgfpathlineto{\pgfqpoint{1.951046in}{1.389946in}}%
\pgfpathlineto{\pgfqpoint{1.951317in}{1.390550in}}%
\pgfpathlineto{\pgfqpoint{1.951453in}{1.388493in}}%
\pgfpathlineto{\pgfqpoint{1.952809in}{1.380016in}}%
\pgfpathlineto{\pgfqpoint{1.952131in}{1.389886in}}%
\pgfpathlineto{\pgfqpoint{1.953216in}{1.382362in}}%
\pgfpathlineto{\pgfqpoint{1.953623in}{1.384692in}}%
\pgfpathlineto{\pgfqpoint{1.953894in}{1.381485in}}%
\pgfpathlineto{\pgfqpoint{1.954301in}{1.382262in}}%
\pgfpathlineto{\pgfqpoint{1.954572in}{1.374897in}}%
\pgfpathlineto{\pgfqpoint{1.955114in}{1.388892in}}%
\pgfpathlineto{\pgfqpoint{1.955657in}{1.382985in}}%
\pgfpathlineto{\pgfqpoint{1.957013in}{1.400129in}}%
\pgfpathlineto{\pgfqpoint{1.957420in}{1.392539in}}%
\pgfpathlineto{\pgfqpoint{1.957691in}{1.391001in}}%
\pgfpathlineto{\pgfqpoint{1.957963in}{1.398323in}}%
\pgfpathlineto{\pgfqpoint{1.959048in}{1.403213in}}%
\pgfpathlineto{\pgfqpoint{1.958505in}{1.393086in}}%
\pgfpathlineto{\pgfqpoint{1.959183in}{1.401093in}}%
\pgfpathlineto{\pgfqpoint{1.960539in}{1.394170in}}%
\pgfpathlineto{\pgfqpoint{1.960675in}{1.395256in}}%
\pgfpathlineto{\pgfqpoint{1.962845in}{1.412101in}}%
\pgfpathlineto{\pgfqpoint{1.963523in}{1.410007in}}%
\pgfpathlineto{\pgfqpoint{1.964201in}{1.414469in}}%
\pgfpathlineto{\pgfqpoint{1.965558in}{1.410437in}}%
\pgfpathlineto{\pgfqpoint{1.965015in}{1.416545in}}%
\pgfpathlineto{\pgfqpoint{1.965693in}{1.412170in}}%
\pgfpathlineto{\pgfqpoint{1.966778in}{1.416279in}}%
\pgfpathlineto{\pgfqpoint{1.966371in}{1.411284in}}%
\pgfpathlineto{\pgfqpoint{1.967185in}{1.412692in}}%
\pgfpathlineto{\pgfqpoint{1.968406in}{1.408913in}}%
\pgfpathlineto{\pgfqpoint{1.967863in}{1.416125in}}%
\pgfpathlineto{\pgfqpoint{1.968541in}{1.410925in}}%
\pgfpathlineto{\pgfqpoint{1.968813in}{1.415976in}}%
\pgfpathlineto{\pgfqpoint{1.969355in}{1.403112in}}%
\pgfpathlineto{\pgfqpoint{1.969898in}{1.408754in}}%
\pgfpathlineto{\pgfqpoint{1.970033in}{1.409028in}}%
\pgfpathlineto{\pgfqpoint{1.971254in}{1.397074in}}%
\pgfpathlineto{\pgfqpoint{1.971661in}{1.401556in}}%
\pgfpathlineto{\pgfqpoint{1.971796in}{1.402738in}}%
\pgfpathlineto{\pgfqpoint{1.972474in}{1.397315in}}%
\pgfpathlineto{\pgfqpoint{1.972746in}{1.398599in}}%
\pgfpathlineto{\pgfqpoint{1.974238in}{1.386568in}}%
\pgfpathlineto{\pgfqpoint{1.973559in}{1.402345in}}%
\pgfpathlineto{\pgfqpoint{1.974509in}{1.393496in}}%
\pgfpathlineto{\pgfqpoint{1.974644in}{1.395332in}}%
\pgfpathlineto{\pgfqpoint{1.975187in}{1.384413in}}%
\pgfpathlineto{\pgfqpoint{1.975594in}{1.388969in}}%
\pgfpathlineto{\pgfqpoint{1.975729in}{1.388617in}}%
\pgfpathlineto{\pgfqpoint{1.977221in}{1.375403in}}%
\pgfpathlineto{\pgfqpoint{1.977357in}{1.380684in}}%
\pgfpathlineto{\pgfqpoint{1.977493in}{1.385420in}}%
\pgfpathlineto{\pgfqpoint{1.978713in}{1.376111in}}%
\pgfpathlineto{\pgfqpoint{1.979934in}{1.364316in}}%
\pgfpathlineto{\pgfqpoint{1.979527in}{1.377249in}}%
\pgfpathlineto{\pgfqpoint{1.980341in}{1.372630in}}%
\pgfpathlineto{\pgfqpoint{1.980476in}{1.373897in}}%
\pgfpathlineto{\pgfqpoint{1.980612in}{1.368625in}}%
\pgfpathlineto{\pgfqpoint{1.981019in}{1.350931in}}%
\pgfpathlineto{\pgfqpoint{1.982104in}{1.372654in}}%
\pgfpathlineto{\pgfqpoint{1.983324in}{1.381998in}}%
\pgfpathlineto{\pgfqpoint{1.982782in}{1.365946in}}%
\pgfpathlineto{\pgfqpoint{1.983596in}{1.376078in}}%
\pgfpathlineto{\pgfqpoint{1.983731in}{1.371716in}}%
\pgfpathlineto{\pgfqpoint{1.984274in}{1.383006in}}%
\pgfpathlineto{\pgfqpoint{1.984952in}{1.382088in}}%
\pgfpathlineto{\pgfqpoint{1.986308in}{1.393808in}}%
\pgfpathlineto{\pgfqpoint{1.985766in}{1.380097in}}%
\pgfpathlineto{\pgfqpoint{1.986579in}{1.388423in}}%
\pgfpathlineto{\pgfqpoint{1.986715in}{1.386332in}}%
\pgfpathlineto{\pgfqpoint{1.987258in}{1.401080in}}%
\pgfpathlineto{\pgfqpoint{1.987800in}{1.386828in}}%
\pgfpathlineto{\pgfqpoint{1.989292in}{1.399394in}}%
\pgfpathlineto{\pgfqpoint{1.989428in}{1.397105in}}%
\pgfpathlineto{\pgfqpoint{1.990648in}{1.392140in}}%
\pgfpathlineto{\pgfqpoint{1.990106in}{1.403044in}}%
\pgfpathlineto{\pgfqpoint{1.990784in}{1.394501in}}%
\pgfpathlineto{\pgfqpoint{1.992140in}{1.403577in}}%
\pgfpathlineto{\pgfqpoint{1.992276in}{1.398835in}}%
\pgfpathlineto{\pgfqpoint{1.992683in}{1.393379in}}%
\pgfpathlineto{\pgfqpoint{1.993496in}{1.404276in}}%
\pgfpathlineto{\pgfqpoint{1.993903in}{1.408901in}}%
\pgfpathlineto{\pgfqpoint{1.994446in}{1.399630in}}%
\pgfpathlineto{\pgfqpoint{1.994581in}{1.399019in}}%
\pgfpathlineto{\pgfqpoint{1.994717in}{1.403633in}}%
\pgfpathlineto{\pgfqpoint{1.994988in}{1.410278in}}%
\pgfpathlineto{\pgfqpoint{1.996209in}{1.402069in}}%
\pgfpathlineto{\pgfqpoint{1.996480in}{1.400948in}}%
\pgfpathlineto{\pgfqpoint{1.996751in}{1.408326in}}%
\pgfpathlineto{\pgfqpoint{1.996887in}{1.409695in}}%
\pgfpathlineto{\pgfqpoint{1.997429in}{1.399682in}}%
\pgfpathlineto{\pgfqpoint{1.997972in}{1.406347in}}%
\pgfpathlineto{\pgfqpoint{1.999328in}{1.395934in}}%
\pgfpathlineto{\pgfqpoint{1.998786in}{1.407155in}}%
\pgfpathlineto{\pgfqpoint{1.999871in}{1.403696in}}%
\pgfpathlineto{\pgfqpoint{2.000142in}{1.401778in}}%
\pgfpathlineto{\pgfqpoint{2.000549in}{1.404870in}}%
\pgfpathlineto{\pgfqpoint{2.000684in}{1.406314in}}%
\pgfpathlineto{\pgfqpoint{2.001091in}{1.397668in}}%
\pgfpathlineto{\pgfqpoint{2.001634in}{1.399752in}}%
\pgfpathlineto{\pgfqpoint{2.003126in}{1.384586in}}%
\pgfpathlineto{\pgfqpoint{2.003533in}{1.392207in}}%
\pgfpathlineto{\pgfqpoint{2.004075in}{1.381186in}}%
\pgfpathlineto{\pgfqpoint{2.004753in}{1.387001in}}%
\pgfpathlineto{\pgfqpoint{2.008008in}{1.346105in}}%
\pgfpathlineto{\pgfqpoint{2.008279in}{1.355976in}}%
\pgfpathlineto{\pgfqpoint{2.008551in}{1.368311in}}%
\pgfpathlineto{\pgfqpoint{2.009093in}{1.352846in}}%
\pgfpathlineto{\pgfqpoint{2.009636in}{1.353432in}}%
\pgfpathlineto{\pgfqpoint{2.013026in}{1.311586in}}%
\pgfpathlineto{\pgfqpoint{2.013162in}{1.323200in}}%
\pgfpathlineto{\pgfqpoint{2.014518in}{1.340772in}}%
\pgfpathlineto{\pgfqpoint{2.013840in}{1.320725in}}%
\pgfpathlineto{\pgfqpoint{2.014789in}{1.333817in}}%
\pgfpathlineto{\pgfqpoint{2.014925in}{1.332840in}}%
\pgfpathlineto{\pgfqpoint{2.015061in}{1.337420in}}%
\pgfpathlineto{\pgfqpoint{2.015468in}{1.355370in}}%
\pgfpathlineto{\pgfqpoint{2.016688in}{1.344573in}}%
\pgfpathlineto{\pgfqpoint{2.024012in}{1.397680in}}%
\pgfpathlineto{\pgfqpoint{2.017773in}{1.344156in}}%
\pgfpathlineto{\pgfqpoint{2.024283in}{1.393184in}}%
\pgfpathlineto{\pgfqpoint{2.024419in}{1.391583in}}%
\pgfpathlineto{\pgfqpoint{2.024961in}{1.401112in}}%
\pgfpathlineto{\pgfqpoint{2.025639in}{1.399657in}}%
\pgfpathlineto{\pgfqpoint{2.027131in}{1.411422in}}%
\pgfpathlineto{\pgfqpoint{2.028488in}{1.404814in}}%
\pgfpathlineto{\pgfqpoint{2.028759in}{1.409630in}}%
\pgfpathlineto{\pgfqpoint{2.028894in}{1.412171in}}%
\pgfpathlineto{\pgfqpoint{2.029437in}{1.407072in}}%
\pgfpathlineto{\pgfqpoint{2.030115in}{1.408918in}}%
\pgfpathlineto{\pgfqpoint{2.030522in}{1.404761in}}%
\pgfpathlineto{\pgfqpoint{2.031336in}{1.409924in}}%
\pgfpathlineto{\pgfqpoint{2.033641in}{1.425985in}}%
\pgfpathlineto{\pgfqpoint{2.033913in}{1.424699in}}%
\pgfpathlineto{\pgfqpoint{2.034319in}{1.420617in}}%
\pgfpathlineto{\pgfqpoint{2.034862in}{1.427745in}}%
\pgfpathlineto{\pgfqpoint{2.035404in}{1.423578in}}%
\pgfpathlineto{\pgfqpoint{2.035811in}{1.427378in}}%
\pgfpathlineto{\pgfqpoint{2.036354in}{1.418070in}}%
\pgfpathlineto{\pgfqpoint{2.036761in}{1.421955in}}%
\pgfpathlineto{\pgfqpoint{2.037032in}{1.418212in}}%
\pgfpathlineto{\pgfqpoint{2.039066in}{1.401625in}}%
\pgfpathlineto{\pgfqpoint{2.039202in}{1.401215in}}%
\pgfpathlineto{\pgfqpoint{2.039473in}{1.404654in}}%
\pgfpathlineto{\pgfqpoint{2.039744in}{1.406503in}}%
\pgfpathlineto{\pgfqpoint{2.040151in}{1.399891in}}%
\pgfpathlineto{\pgfqpoint{2.040694in}{1.404948in}}%
\pgfpathlineto{\pgfqpoint{2.042186in}{1.392272in}}%
\pgfpathlineto{\pgfqpoint{2.042728in}{1.394694in}}%
\pgfpathlineto{\pgfqpoint{2.043135in}{1.391547in}}%
\pgfpathlineto{\pgfqpoint{2.043406in}{1.395602in}}%
\pgfpathlineto{\pgfqpoint{2.043542in}{1.396769in}}%
\pgfpathlineto{\pgfqpoint{2.043949in}{1.387526in}}%
\pgfpathlineto{\pgfqpoint{2.044491in}{1.391618in}}%
\pgfpathlineto{\pgfqpoint{2.045983in}{1.376338in}}%
\pgfpathlineto{\pgfqpoint{2.046390in}{1.385963in}}%
\pgfpathlineto{\pgfqpoint{2.047068in}{1.376244in}}%
\pgfpathlineto{\pgfqpoint{2.047475in}{1.377092in}}%
\pgfpathlineto{\pgfqpoint{2.047746in}{1.372269in}}%
\pgfpathlineto{\pgfqpoint{2.048424in}{1.386672in}}%
\pgfpathlineto{\pgfqpoint{2.048831in}{1.378217in}}%
\pgfpathlineto{\pgfqpoint{2.049103in}{1.383133in}}%
\pgfpathlineto{\pgfqpoint{2.049645in}{1.372676in}}%
\pgfpathlineto{\pgfqpoint{2.050323in}{1.378776in}}%
\pgfpathlineto{\pgfqpoint{2.050866in}{1.365583in}}%
\pgfpathlineto{\pgfqpoint{2.051273in}{1.379089in}}%
\pgfpathlineto{\pgfqpoint{2.051815in}{1.373659in}}%
\pgfpathlineto{\pgfqpoint{2.052086in}{1.382777in}}%
\pgfpathlineto{\pgfqpoint{2.052764in}{1.369393in}}%
\pgfpathlineto{\pgfqpoint{2.053443in}{1.379195in}}%
\pgfpathlineto{\pgfqpoint{2.053578in}{1.375739in}}%
\pgfpathlineto{\pgfqpoint{2.054121in}{1.385542in}}%
\pgfpathlineto{\pgfqpoint{2.054663in}{1.378742in}}%
\pgfpathlineto{\pgfqpoint{2.056019in}{1.397062in}}%
\pgfpathlineto{\pgfqpoint{2.056426in}{1.393172in}}%
\pgfpathlineto{\pgfqpoint{2.056562in}{1.392429in}}%
\pgfpathlineto{\pgfqpoint{2.056833in}{1.399029in}}%
\pgfpathlineto{\pgfqpoint{2.058325in}{1.411866in}}%
\pgfpathlineto{\pgfqpoint{2.058596in}{1.411149in}}%
\pgfpathlineto{\pgfqpoint{2.061038in}{1.423382in}}%
\pgfpathlineto{\pgfqpoint{2.061444in}{1.417244in}}%
\pgfpathlineto{\pgfqpoint{2.061987in}{1.428132in}}%
\pgfpathlineto{\pgfqpoint{2.062665in}{1.421408in}}%
\pgfpathlineto{\pgfqpoint{2.063886in}{1.432527in}}%
\pgfpathlineto{\pgfqpoint{2.064428in}{1.427423in}}%
\pgfpathlineto{\pgfqpoint{2.065649in}{1.432452in}}%
\pgfpathlineto{\pgfqpoint{2.066056in}{1.429752in}}%
\pgfpathlineto{\pgfqpoint{2.067276in}{1.424724in}}%
\pgfpathlineto{\pgfqpoint{2.066869in}{1.433570in}}%
\pgfpathlineto{\pgfqpoint{2.067683in}{1.427394in}}%
\pgfpathlineto{\pgfqpoint{2.068633in}{1.431011in}}%
\pgfpathlineto{\pgfqpoint{2.068226in}{1.427266in}}%
\pgfpathlineto{\pgfqpoint{2.069039in}{1.427453in}}%
\pgfpathlineto{\pgfqpoint{2.071074in}{1.420505in}}%
\pgfpathlineto{\pgfqpoint{2.070260in}{1.428317in}}%
\pgfpathlineto{\pgfqpoint{2.071209in}{1.420790in}}%
\pgfpathlineto{\pgfqpoint{2.071616in}{1.424660in}}%
\pgfpathlineto{\pgfqpoint{2.072159in}{1.419986in}}%
\pgfpathlineto{\pgfqpoint{2.072566in}{1.421703in}}%
\pgfpathlineto{\pgfqpoint{2.074193in}{1.411702in}}%
\pgfpathlineto{\pgfqpoint{2.074871in}{1.411994in}}%
\pgfpathlineto{\pgfqpoint{2.075143in}{1.412647in}}%
\pgfpathlineto{\pgfqpoint{2.075414in}{1.409843in}}%
\pgfpathlineto{\pgfqpoint{2.076906in}{1.401874in}}%
\pgfpathlineto{\pgfqpoint{2.076499in}{1.409985in}}%
\pgfpathlineto{\pgfqpoint{2.077177in}{1.404984in}}%
\pgfpathlineto{\pgfqpoint{2.077313in}{1.407365in}}%
\pgfpathlineto{\pgfqpoint{2.077855in}{1.396905in}}%
\pgfpathlineto{\pgfqpoint{2.078398in}{1.402680in}}%
\pgfpathlineto{\pgfqpoint{2.080974in}{1.376350in}}%
\pgfpathlineto{\pgfqpoint{2.081924in}{1.372763in}}%
\pgfpathlineto{\pgfqpoint{2.081246in}{1.376544in}}%
\pgfpathlineto{\pgfqpoint{2.082059in}{1.376538in}}%
\pgfpathlineto{\pgfqpoint{2.082331in}{1.379645in}}%
\pgfpathlineto{\pgfqpoint{2.082602in}{1.363915in}}%
\pgfpathlineto{\pgfqpoint{2.083687in}{1.358626in}}%
\pgfpathlineto{\pgfqpoint{2.083280in}{1.375087in}}%
\pgfpathlineto{\pgfqpoint{2.084094in}{1.364605in}}%
\pgfpathlineto{\pgfqpoint{2.085314in}{1.375378in}}%
\pgfpathlineto{\pgfqpoint{2.084772in}{1.358413in}}%
\pgfpathlineto{\pgfqpoint{2.085586in}{1.369546in}}%
\pgfpathlineto{\pgfqpoint{2.085857in}{1.361656in}}%
\pgfpathlineto{\pgfqpoint{2.086806in}{1.377988in}}%
\pgfpathlineto{\pgfqpoint{2.088298in}{1.388680in}}%
\pgfpathlineto{\pgfqpoint{2.087620in}{1.374028in}}%
\pgfpathlineto{\pgfqpoint{2.088434in}{1.386565in}}%
\pgfpathlineto{\pgfqpoint{2.089519in}{1.382692in}}%
\pgfpathlineto{\pgfqpoint{2.089112in}{1.390183in}}%
\pgfpathlineto{\pgfqpoint{2.089790in}{1.388776in}}%
\pgfpathlineto{\pgfqpoint{2.092096in}{1.411980in}}%
\pgfpathlineto{\pgfqpoint{2.092638in}{1.407715in}}%
\pgfpathlineto{\pgfqpoint{2.092909in}{1.413849in}}%
\pgfpathlineto{\pgfqpoint{2.093316in}{1.413338in}}%
\pgfpathlineto{\pgfqpoint{2.093723in}{1.418525in}}%
\pgfpathlineto{\pgfqpoint{2.095215in}{1.426293in}}%
\pgfpathlineto{\pgfqpoint{2.095486in}{1.425337in}}%
\pgfpathlineto{\pgfqpoint{2.095622in}{1.424245in}}%
\pgfpathlineto{\pgfqpoint{2.096300in}{1.429764in}}%
\pgfpathlineto{\pgfqpoint{2.097385in}{1.433975in}}%
\pgfpathlineto{\pgfqpoint{2.098063in}{1.432337in}}%
\pgfpathlineto{\pgfqpoint{2.098199in}{1.432217in}}%
\pgfpathlineto{\pgfqpoint{2.098334in}{1.432947in}}%
\pgfpathlineto{\pgfqpoint{2.099826in}{1.438177in}}%
\pgfpathlineto{\pgfqpoint{2.099148in}{1.431194in}}%
\pgfpathlineto{\pgfqpoint{2.100233in}{1.436860in}}%
\pgfpathlineto{\pgfqpoint{2.100369in}{1.436951in}}%
\pgfpathlineto{\pgfqpoint{2.100776in}{1.440368in}}%
\pgfpathlineto{\pgfqpoint{2.101318in}{1.435201in}}%
\pgfpathlineto{\pgfqpoint{2.101861in}{1.438008in}}%
\pgfpathlineto{\pgfqpoint{2.102403in}{1.432906in}}%
\pgfpathlineto{\pgfqpoint{2.103353in}{1.437109in}}%
\pgfpathlineto{\pgfqpoint{2.103624in}{1.438547in}}%
\pgfpathlineto{\pgfqpoint{2.104031in}{1.433337in}}%
\pgfpathlineto{\pgfqpoint{2.104166in}{1.433239in}}%
\pgfpathlineto{\pgfqpoint{2.104302in}{1.434438in}}%
\pgfpathlineto{\pgfqpoint{2.105523in}{1.439829in}}%
\pgfpathlineto{\pgfqpoint{2.105116in}{1.434004in}}%
\pgfpathlineto{\pgfqpoint{2.105794in}{1.436538in}}%
\pgfpathlineto{\pgfqpoint{2.107150in}{1.427085in}}%
\pgfpathlineto{\pgfqpoint{2.107557in}{1.428687in}}%
\pgfpathlineto{\pgfqpoint{2.109049in}{1.420763in}}%
\pgfpathlineto{\pgfqpoint{2.109184in}{1.422528in}}%
\pgfpathlineto{\pgfqpoint{2.110541in}{1.426379in}}%
\pgfpathlineto{\pgfqpoint{2.109998in}{1.419597in}}%
\pgfpathlineto{\pgfqpoint{2.110676in}{1.422395in}}%
\pgfpathlineto{\pgfqpoint{2.112168in}{1.408180in}}%
\pgfpathlineto{\pgfqpoint{2.112439in}{1.410482in}}%
\pgfpathlineto{\pgfqpoint{2.112711in}{1.408095in}}%
\pgfpathlineto{\pgfqpoint{2.113524in}{1.408606in}}%
\pgfpathlineto{\pgfqpoint{2.114745in}{1.395445in}}%
\pgfpathlineto{\pgfqpoint{2.115288in}{1.397440in}}%
\pgfpathlineto{\pgfqpoint{2.116779in}{1.385909in}}%
\pgfpathlineto{\pgfqpoint{2.117186in}{1.391661in}}%
\pgfpathlineto{\pgfqpoint{2.117729in}{1.376620in}}%
\pgfpathlineto{\pgfqpoint{2.118271in}{1.386657in}}%
\pgfpathlineto{\pgfqpoint{2.118678in}{1.379584in}}%
\pgfpathlineto{\pgfqpoint{2.119763in}{1.383966in}}%
\pgfpathlineto{\pgfqpoint{2.120034in}{1.392452in}}%
\pgfpathlineto{\pgfqpoint{2.120577in}{1.380615in}}%
\pgfpathlineto{\pgfqpoint{2.121255in}{1.388318in}}%
\pgfpathlineto{\pgfqpoint{2.121526in}{1.380820in}}%
\pgfpathlineto{\pgfqpoint{2.122069in}{1.392188in}}%
\pgfpathlineto{\pgfqpoint{2.122747in}{1.384141in}}%
\pgfpathlineto{\pgfqpoint{2.124917in}{1.404023in}}%
\pgfpathlineto{\pgfqpoint{2.125053in}{1.403757in}}%
\pgfpathlineto{\pgfqpoint{2.125324in}{1.401500in}}%
\pgfpathlineto{\pgfqpoint{2.125595in}{1.405295in}}%
\pgfpathlineto{\pgfqpoint{2.126951in}{1.419513in}}%
\pgfpathlineto{\pgfqpoint{2.126409in}{1.404503in}}%
\pgfpathlineto{\pgfqpoint{2.127223in}{1.414417in}}%
\pgfpathlineto{\pgfqpoint{2.127358in}{1.410767in}}%
\pgfpathlineto{\pgfqpoint{2.128036in}{1.418943in}}%
\pgfpathlineto{\pgfqpoint{2.128579in}{1.418029in}}%
\pgfpathlineto{\pgfqpoint{2.130071in}{1.426692in}}%
\pgfpathlineto{\pgfqpoint{2.130342in}{1.424760in}}%
\pgfpathlineto{\pgfqpoint{2.132512in}{1.415636in}}%
\pgfpathlineto{\pgfqpoint{2.130749in}{1.425696in}}%
\pgfpathlineto{\pgfqpoint{2.133326in}{1.417965in}}%
\pgfpathlineto{\pgfqpoint{2.134818in}{1.423135in}}%
\pgfpathlineto{\pgfqpoint{2.135089in}{1.421385in}}%
\pgfpathlineto{\pgfqpoint{2.136038in}{1.416113in}}%
\pgfpathlineto{\pgfqpoint{2.136445in}{1.423196in}}%
\pgfpathlineto{\pgfqpoint{2.136581in}{1.425192in}}%
\pgfpathlineto{\pgfqpoint{2.137530in}{1.416683in}}%
\pgfpathlineto{\pgfqpoint{2.139293in}{1.408924in}}%
\pgfpathlineto{\pgfqpoint{2.138615in}{1.419958in}}%
\pgfpathlineto{\pgfqpoint{2.139429in}{1.410661in}}%
\pgfpathlineto{\pgfqpoint{2.139700in}{1.413109in}}%
\pgfpathlineto{\pgfqpoint{2.140785in}{1.408757in}}%
\pgfpathlineto{\pgfqpoint{2.141599in}{1.406138in}}%
\pgfpathlineto{\pgfqpoint{2.141328in}{1.410857in}}%
\pgfpathlineto{\pgfqpoint{2.142006in}{1.409307in}}%
\pgfpathlineto{\pgfqpoint{2.142141in}{1.411753in}}%
\pgfpathlineto{\pgfqpoint{2.142955in}{1.404654in}}%
\pgfpathlineto{\pgfqpoint{2.143498in}{1.411445in}}%
\pgfpathlineto{\pgfqpoint{2.145939in}{1.393764in}}%
\pgfpathlineto{\pgfqpoint{2.146210in}{1.396494in}}%
\pgfpathlineto{\pgfqpoint{2.147431in}{1.385499in}}%
\pgfpathlineto{\pgfqpoint{2.147702in}{1.378399in}}%
\pgfpathlineto{\pgfqpoint{2.148244in}{1.391762in}}%
\pgfpathlineto{\pgfqpoint{2.148923in}{1.382054in}}%
\pgfpathlineto{\pgfqpoint{2.150279in}{1.393375in}}%
\pgfpathlineto{\pgfqpoint{2.149736in}{1.379623in}}%
\pgfpathlineto{\pgfqpoint{2.150550in}{1.388194in}}%
\pgfpathlineto{\pgfqpoint{2.150957in}{1.379305in}}%
\pgfpathlineto{\pgfqpoint{2.152042in}{1.387003in}}%
\pgfpathlineto{\pgfqpoint{2.154619in}{1.375541in}}%
\pgfpathlineto{\pgfqpoint{2.154754in}{1.378615in}}%
\pgfpathlineto{\pgfqpoint{2.156246in}{1.390702in}}%
\pgfpathlineto{\pgfqpoint{2.156518in}{1.390174in}}%
\pgfpathlineto{\pgfqpoint{2.160179in}{1.410471in}}%
\pgfpathlineto{\pgfqpoint{2.161264in}{1.409502in}}%
\pgfpathlineto{\pgfqpoint{2.161400in}{1.406813in}}%
\pgfpathlineto{\pgfqpoint{2.161943in}{1.415593in}}%
\pgfpathlineto{\pgfqpoint{2.162621in}{1.410501in}}%
\pgfpathlineto{\pgfqpoint{2.164926in}{1.422838in}}%
\pgfpathlineto{\pgfqpoint{2.165198in}{1.419658in}}%
\pgfpathlineto{\pgfqpoint{2.165469in}{1.419896in}}%
\pgfpathlineto{\pgfqpoint{2.166961in}{1.425886in}}%
\pgfpathlineto{\pgfqpoint{2.166147in}{1.416720in}}%
\pgfpathlineto{\pgfqpoint{2.167096in}{1.424135in}}%
\pgfpathlineto{\pgfqpoint{2.167368in}{1.421302in}}%
\pgfpathlineto{\pgfqpoint{2.167910in}{1.426003in}}%
\pgfpathlineto{\pgfqpoint{2.168453in}{1.425389in}}%
\pgfpathlineto{\pgfqpoint{2.168724in}{1.426327in}}%
\pgfpathlineto{\pgfqpoint{2.168995in}{1.422522in}}%
\pgfpathlineto{\pgfqpoint{2.170080in}{1.420023in}}%
\pgfpathlineto{\pgfqpoint{2.170351in}{1.422871in}}%
\pgfpathlineto{\pgfqpoint{2.171708in}{1.428634in}}%
\pgfpathlineto{\pgfqpoint{2.171165in}{1.418449in}}%
\pgfpathlineto{\pgfqpoint{2.171979in}{1.424714in}}%
\pgfpathlineto{\pgfqpoint{2.172928in}{1.420427in}}%
\pgfpathlineto{\pgfqpoint{2.173335in}{1.424842in}}%
\pgfpathlineto{\pgfqpoint{2.173471in}{1.426888in}}%
\pgfpathlineto{\pgfqpoint{2.174013in}{1.421886in}}%
\pgfpathlineto{\pgfqpoint{2.174691in}{1.422318in}}%
\pgfpathlineto{\pgfqpoint{2.175912in}{1.416267in}}%
\pgfpathlineto{\pgfqpoint{2.175369in}{1.426912in}}%
\pgfpathlineto{\pgfqpoint{2.176183in}{1.419884in}}%
\pgfpathlineto{\pgfqpoint{2.177404in}{1.426938in}}%
\pgfpathlineto{\pgfqpoint{2.177675in}{1.422827in}}%
\pgfpathlineto{\pgfqpoint{2.177946in}{1.416393in}}%
\pgfpathlineto{\pgfqpoint{2.178489in}{1.423681in}}%
\pgfpathlineto{\pgfqpoint{2.179303in}{1.419372in}}%
\pgfpathlineto{\pgfqpoint{2.180794in}{1.413633in}}%
\pgfpathlineto{\pgfqpoint{2.180116in}{1.421379in}}%
\pgfpathlineto{\pgfqpoint{2.180930in}{1.415404in}}%
\pgfpathlineto{\pgfqpoint{2.182151in}{1.420524in}}%
\pgfpathlineto{\pgfqpoint{2.181744in}{1.414076in}}%
\pgfpathlineto{\pgfqpoint{2.182422in}{1.419288in}}%
\pgfpathlineto{\pgfqpoint{2.182693in}{1.413258in}}%
\pgfpathlineto{\pgfqpoint{2.183236in}{1.423104in}}%
\pgfpathlineto{\pgfqpoint{2.183914in}{1.418046in}}%
\pgfpathlineto{\pgfqpoint{2.184999in}{1.422253in}}%
\pgfpathlineto{\pgfqpoint{2.184592in}{1.416560in}}%
\pgfpathlineto{\pgfqpoint{2.185406in}{1.420091in}}%
\pgfpathlineto{\pgfqpoint{2.186491in}{1.417026in}}%
\pgfpathlineto{\pgfqpoint{2.186219in}{1.422220in}}%
\pgfpathlineto{\pgfqpoint{2.186762in}{1.420450in}}%
\pgfpathlineto{\pgfqpoint{2.188254in}{1.429782in}}%
\pgfpathlineto{\pgfqpoint{2.187576in}{1.417731in}}%
\pgfpathlineto{\pgfqpoint{2.188525in}{1.427640in}}%
\pgfpathlineto{\pgfqpoint{2.188661in}{1.427519in}}%
\pgfpathlineto{\pgfqpoint{2.188796in}{1.428641in}}%
\pgfpathlineto{\pgfqpoint{2.190831in}{1.437269in}}%
\pgfpathlineto{\pgfqpoint{2.189339in}{1.427549in}}%
\pgfpathlineto{\pgfqpoint{2.190966in}{1.435438in}}%
\pgfpathlineto{\pgfqpoint{2.191373in}{1.431265in}}%
\pgfpathlineto{\pgfqpoint{2.191780in}{1.438151in}}%
\pgfpathlineto{\pgfqpoint{2.192458in}{1.434380in}}%
\pgfpathlineto{\pgfqpoint{2.193814in}{1.440950in}}%
\pgfpathlineto{\pgfqpoint{2.194086in}{1.438747in}}%
\pgfpathlineto{\pgfqpoint{2.194357in}{1.435688in}}%
\pgfpathlineto{\pgfqpoint{2.194899in}{1.443223in}}%
\pgfpathlineto{\pgfqpoint{2.195442in}{1.440646in}}%
\pgfpathlineto{\pgfqpoint{2.196527in}{1.445048in}}%
\pgfpathlineto{\pgfqpoint{2.197748in}{1.447366in}}%
\pgfpathlineto{\pgfqpoint{2.197205in}{1.441236in}}%
\pgfpathlineto{\pgfqpoint{2.198019in}{1.444904in}}%
\pgfpathlineto{\pgfqpoint{2.198154in}{1.444059in}}%
\pgfpathlineto{\pgfqpoint{2.198426in}{1.447754in}}%
\pgfpathlineto{\pgfqpoint{2.199782in}{1.452938in}}%
\pgfpathlineto{\pgfqpoint{2.199239in}{1.447190in}}%
\pgfpathlineto{\pgfqpoint{2.200053in}{1.452534in}}%
\pgfpathlineto{\pgfqpoint{2.201138in}{1.452201in}}%
\pgfpathlineto{\pgfqpoint{2.200460in}{1.453826in}}%
\pgfpathlineto{\pgfqpoint{2.201274in}{1.453226in}}%
\pgfpathlineto{\pgfqpoint{2.201545in}{1.455848in}}%
\pgfpathlineto{\pgfqpoint{2.202630in}{1.454138in}}%
\pgfpathlineto{\pgfqpoint{2.204122in}{1.448026in}}%
\pgfpathlineto{\pgfqpoint{2.203579in}{1.455662in}}%
\pgfpathlineto{\pgfqpoint{2.204258in}{1.449214in}}%
\pgfpathlineto{\pgfqpoint{2.204529in}{1.451992in}}%
\pgfpathlineto{\pgfqpoint{2.205207in}{1.448976in}}%
\pgfpathlineto{\pgfqpoint{2.205749in}{1.450306in}}%
\pgfpathlineto{\pgfqpoint{2.206428in}{1.450638in}}%
\pgfpathlineto{\pgfqpoint{2.207784in}{1.442744in}}%
\pgfpathlineto{\pgfqpoint{2.208326in}{1.447252in}}%
\pgfpathlineto{\pgfqpoint{2.208869in}{1.441858in}}%
\pgfpathlineto{\pgfqpoint{2.209276in}{1.443363in}}%
\pgfpathlineto{\pgfqpoint{2.214836in}{1.416908in}}%
\pgfpathlineto{\pgfqpoint{2.215108in}{1.422726in}}%
\pgfpathlineto{\pgfqpoint{2.215243in}{1.424516in}}%
\pgfpathlineto{\pgfqpoint{2.215921in}{1.414184in}}%
\pgfpathlineto{\pgfqpoint{2.216193in}{1.417194in}}%
\pgfpathlineto{\pgfqpoint{2.216599in}{1.418340in}}%
\pgfpathlineto{\pgfqpoint{2.217278in}{1.416555in}}%
\pgfpathlineto{\pgfqpoint{2.217549in}{1.413261in}}%
\pgfpathlineto{\pgfqpoint{2.218091in}{1.417405in}}%
\pgfpathlineto{\pgfqpoint{2.218634in}{1.416619in}}%
\pgfpathlineto{\pgfqpoint{2.219990in}{1.420831in}}%
\pgfpathlineto{\pgfqpoint{2.219448in}{1.411203in}}%
\pgfpathlineto{\pgfqpoint{2.220126in}{1.419943in}}%
\pgfpathlineto{\pgfqpoint{2.220533in}{1.409665in}}%
\pgfpathlineto{\pgfqpoint{2.221753in}{1.416826in}}%
\pgfpathlineto{\pgfqpoint{2.222024in}{1.421478in}}%
\pgfpathlineto{\pgfqpoint{2.222567in}{1.414672in}}%
\pgfpathlineto{\pgfqpoint{2.223381in}{1.418465in}}%
\pgfpathlineto{\pgfqpoint{2.224330in}{1.415375in}}%
\pgfpathlineto{\pgfqpoint{2.223788in}{1.420877in}}%
\pgfpathlineto{\pgfqpoint{2.224601in}{1.418399in}}%
\pgfpathlineto{\pgfqpoint{2.225958in}{1.429511in}}%
\pgfpathlineto{\pgfqpoint{2.226500in}{1.427159in}}%
\pgfpathlineto{\pgfqpoint{2.228670in}{1.437592in}}%
\pgfpathlineto{\pgfqpoint{2.229213in}{1.431926in}}%
\pgfpathlineto{\pgfqpoint{2.229755in}{1.439216in}}%
\pgfpathlineto{\pgfqpoint{2.230162in}{1.437176in}}%
\pgfpathlineto{\pgfqpoint{2.231789in}{1.444251in}}%
\pgfpathlineto{\pgfqpoint{2.231247in}{1.435108in}}%
\pgfpathlineto{\pgfqpoint{2.232196in}{1.440239in}}%
\pgfpathlineto{\pgfqpoint{2.232332in}{1.438754in}}%
\pgfpathlineto{\pgfqpoint{2.232874in}{1.445737in}}%
\pgfpathlineto{\pgfqpoint{2.233281in}{1.445020in}}%
\pgfpathlineto{\pgfqpoint{2.235316in}{1.450270in}}%
\pgfpathlineto{\pgfqpoint{2.235587in}{1.451350in}}%
\pgfpathlineto{\pgfqpoint{2.235723in}{1.449873in}}%
\pgfpathlineto{\pgfqpoint{2.235994in}{1.444792in}}%
\pgfpathlineto{\pgfqpoint{2.237350in}{1.446835in}}%
\pgfpathlineto{\pgfqpoint{2.238571in}{1.448510in}}%
\pgfpathlineto{\pgfqpoint{2.238164in}{1.444926in}}%
\pgfpathlineto{\pgfqpoint{2.238706in}{1.447910in}}%
\pgfpathlineto{\pgfqpoint{2.239791in}{1.441506in}}%
\pgfpathlineto{\pgfqpoint{2.240334in}{1.444751in}}%
\pgfpathlineto{\pgfqpoint{2.240876in}{1.435398in}}%
\pgfpathlineto{\pgfqpoint{2.242233in}{1.439502in}}%
\pgfpathlineto{\pgfqpoint{2.242368in}{1.440373in}}%
\pgfpathlineto{\pgfqpoint{2.242775in}{1.433904in}}%
\pgfpathlineto{\pgfqpoint{2.242911in}{1.433573in}}%
\pgfpathlineto{\pgfqpoint{2.243182in}{1.436584in}}%
\pgfpathlineto{\pgfqpoint{2.243453in}{1.441569in}}%
\pgfpathlineto{\pgfqpoint{2.243996in}{1.434304in}}%
\pgfpathlineto{\pgfqpoint{2.244674in}{1.436174in}}%
\pgfpathlineto{\pgfqpoint{2.244809in}{1.436296in}}%
\pgfpathlineto{\pgfqpoint{2.245352in}{1.435069in}}%
\pgfpathlineto{\pgfqpoint{2.249828in}{1.408709in}}%
\pgfpathlineto{\pgfqpoint{2.249963in}{1.409463in}}%
\pgfpathlineto{\pgfqpoint{2.250234in}{1.413887in}}%
\pgfpathlineto{\pgfqpoint{2.251184in}{1.405678in}}%
\pgfpathlineto{\pgfqpoint{2.251319in}{1.405843in}}%
\pgfpathlineto{\pgfqpoint{2.252676in}{1.399408in}}%
\pgfpathlineto{\pgfqpoint{2.252133in}{1.411691in}}%
\pgfpathlineto{\pgfqpoint{2.252811in}{1.402795in}}%
\pgfpathlineto{\pgfqpoint{2.253083in}{1.411246in}}%
\pgfpathlineto{\pgfqpoint{2.253489in}{1.399850in}}%
\pgfpathlineto{\pgfqpoint{2.254303in}{1.402224in}}%
\pgfpathlineto{\pgfqpoint{2.255659in}{1.397490in}}%
\pgfpathlineto{\pgfqpoint{2.255117in}{1.407814in}}%
\pgfpathlineto{\pgfqpoint{2.255795in}{1.399737in}}%
\pgfpathlineto{\pgfqpoint{2.257829in}{1.416510in}}%
\pgfpathlineto{\pgfqpoint{2.258101in}{1.412639in}}%
\pgfpathlineto{\pgfqpoint{2.258372in}{1.410741in}}%
\pgfpathlineto{\pgfqpoint{2.258779in}{1.419284in}}%
\pgfpathlineto{\pgfqpoint{2.258914in}{1.419701in}}%
\pgfpathlineto{\pgfqpoint{2.259050in}{1.417763in}}%
\pgfpathlineto{\pgfqpoint{2.259321in}{1.409279in}}%
\pgfpathlineto{\pgfqpoint{2.259864in}{1.419405in}}%
\pgfpathlineto{\pgfqpoint{2.260542in}{1.413770in}}%
\pgfpathlineto{\pgfqpoint{2.262712in}{1.427083in}}%
\pgfpathlineto{\pgfqpoint{2.263119in}{1.422885in}}%
\pgfpathlineto{\pgfqpoint{2.263526in}{1.431135in}}%
\pgfpathlineto{\pgfqpoint{2.264746in}{1.434654in}}%
\pgfpathlineto{\pgfqpoint{2.264204in}{1.427427in}}%
\pgfpathlineto{\pgfqpoint{2.265018in}{1.433888in}}%
\pgfpathlineto{\pgfqpoint{2.265289in}{1.429800in}}%
\pgfpathlineto{\pgfqpoint{2.266374in}{1.434133in}}%
\pgfpathlineto{\pgfqpoint{2.267459in}{1.435696in}}%
\pgfpathlineto{\pgfqpoint{2.267052in}{1.432350in}}%
\pgfpathlineto{\pgfqpoint{2.267730in}{1.433690in}}%
\pgfpathlineto{\pgfqpoint{2.267866in}{1.431602in}}%
\pgfpathlineto{\pgfqpoint{2.268544in}{1.441894in}}%
\pgfpathlineto{\pgfqpoint{2.269086in}{1.432879in}}%
\pgfpathlineto{\pgfqpoint{2.269493in}{1.441488in}}%
\pgfpathlineto{\pgfqpoint{2.270714in}{1.438672in}}%
\pgfpathlineto{\pgfqpoint{2.272070in}{1.431569in}}%
\pgfpathlineto{\pgfqpoint{2.272477in}{1.433253in}}%
\pgfpathlineto{\pgfqpoint{2.273833in}{1.430383in}}%
\pgfpathlineto{\pgfqpoint{2.273426in}{1.434534in}}%
\pgfpathlineto{\pgfqpoint{2.274104in}{1.432804in}}%
\pgfpathlineto{\pgfqpoint{2.274240in}{1.433409in}}%
\pgfpathlineto{\pgfqpoint{2.274647in}{1.430646in}}%
\pgfpathlineto{\pgfqpoint{2.276003in}{1.418544in}}%
\pgfpathlineto{\pgfqpoint{2.275325in}{1.431944in}}%
\pgfpathlineto{\pgfqpoint{2.276681in}{1.420953in}}%
\pgfpathlineto{\pgfqpoint{2.279665in}{1.398771in}}%
\pgfpathlineto{\pgfqpoint{2.280072in}{1.410386in}}%
\pgfpathlineto{\pgfqpoint{2.281428in}{1.405451in}}%
\pgfpathlineto{\pgfqpoint{2.282920in}{1.389077in}}%
\pgfpathlineto{\pgfqpoint{2.283463in}{1.391868in}}%
\pgfpathlineto{\pgfqpoint{2.285497in}{1.373480in}}%
\pgfpathlineto{\pgfqpoint{2.286853in}{1.381530in}}%
\pgfpathlineto{\pgfqpoint{2.286989in}{1.383550in}}%
\pgfpathlineto{\pgfqpoint{2.287531in}{1.371058in}}%
\pgfpathlineto{\pgfqpoint{2.288209in}{1.379994in}}%
\pgfpathlineto{\pgfqpoint{2.289294in}{1.363856in}}%
\pgfpathlineto{\pgfqpoint{2.289701in}{1.380281in}}%
\pgfpathlineto{\pgfqpoint{2.289837in}{1.382747in}}%
\pgfpathlineto{\pgfqpoint{2.290379in}{1.371934in}}%
\pgfpathlineto{\pgfqpoint{2.290922in}{1.380650in}}%
\pgfpathlineto{\pgfqpoint{2.291329in}{1.367859in}}%
\pgfpathlineto{\pgfqpoint{2.291871in}{1.383977in}}%
\pgfpathlineto{\pgfqpoint{2.292549in}{1.376571in}}%
\pgfpathlineto{\pgfqpoint{2.293499in}{1.388382in}}%
\pgfpathlineto{\pgfqpoint{2.294313in}{1.385529in}}%
\pgfpathlineto{\pgfqpoint{2.295398in}{1.390995in}}%
\pgfpathlineto{\pgfqpoint{2.296754in}{1.403876in}}%
\pgfpathlineto{\pgfqpoint{2.296076in}{1.387373in}}%
\pgfpathlineto{\pgfqpoint{2.297025in}{1.394547in}}%
\pgfpathlineto{\pgfqpoint{2.297161in}{1.392847in}}%
\pgfpathlineto{\pgfqpoint{2.297703in}{1.404730in}}%
\pgfpathlineto{\pgfqpoint{2.298110in}{1.400320in}}%
\pgfpathlineto{\pgfqpoint{2.298517in}{1.405344in}}%
\pgfpathlineto{\pgfqpoint{2.300144in}{1.416830in}}%
\pgfpathlineto{\pgfqpoint{2.300280in}{1.415851in}}%
\pgfpathlineto{\pgfqpoint{2.300551in}{1.416305in}}%
\pgfpathlineto{\pgfqpoint{2.300823in}{1.414564in}}%
\pgfpathlineto{\pgfqpoint{2.300958in}{1.413630in}}%
\pgfpathlineto{\pgfqpoint{2.301501in}{1.419733in}}%
\pgfpathlineto{\pgfqpoint{2.301772in}{1.420818in}}%
\pgfpathlineto{\pgfqpoint{2.302043in}{1.415697in}}%
\pgfpathlineto{\pgfqpoint{2.302179in}{1.415582in}}%
\pgfpathlineto{\pgfqpoint{2.302314in}{1.416364in}}%
\pgfpathlineto{\pgfqpoint{2.303535in}{1.419883in}}%
\pgfpathlineto{\pgfqpoint{2.303128in}{1.413827in}}%
\pgfpathlineto{\pgfqpoint{2.303806in}{1.416538in}}%
\pgfpathlineto{\pgfqpoint{2.304078in}{1.414355in}}%
\pgfpathlineto{\pgfqpoint{2.304756in}{1.420735in}}%
\pgfpathlineto{\pgfqpoint{2.304891in}{1.420595in}}%
\pgfpathlineto{\pgfqpoint{2.305027in}{1.421168in}}%
\pgfpathlineto{\pgfqpoint{2.306383in}{1.423402in}}%
\pgfpathlineto{\pgfqpoint{2.305841in}{1.418528in}}%
\pgfpathlineto{\pgfqpoint{2.306519in}{1.422907in}}%
\pgfpathlineto{\pgfqpoint{2.308011in}{1.411128in}}%
\pgfpathlineto{\pgfqpoint{2.307333in}{1.424171in}}%
\pgfpathlineto{\pgfqpoint{2.308146in}{1.415632in}}%
\pgfpathlineto{\pgfqpoint{2.308418in}{1.422486in}}%
\pgfpathlineto{\pgfqpoint{2.308960in}{1.409442in}}%
\pgfpathlineto{\pgfqpoint{2.309503in}{1.412275in}}%
\pgfpathlineto{\pgfqpoint{2.311673in}{1.405160in}}%
\pgfpathlineto{\pgfqpoint{2.311944in}{1.407503in}}%
\pgfpathlineto{\pgfqpoint{2.313300in}{1.411941in}}%
\pgfpathlineto{\pgfqpoint{2.312622in}{1.399652in}}%
\pgfpathlineto{\pgfqpoint{2.313436in}{1.407645in}}%
\pgfpathlineto{\pgfqpoint{2.314656in}{1.396029in}}%
\pgfpathlineto{\pgfqpoint{2.314928in}{1.404406in}}%
\pgfpathlineto{\pgfqpoint{2.315199in}{1.407945in}}%
\pgfpathlineto{\pgfqpoint{2.315741in}{1.394135in}}%
\pgfpathlineto{\pgfqpoint{2.316013in}{1.397205in}}%
\pgfpathlineto{\pgfqpoint{2.316148in}{1.397320in}}%
\pgfpathlineto{\pgfqpoint{2.317640in}{1.378986in}}%
\pgfpathlineto{\pgfqpoint{2.316962in}{1.397500in}}%
\pgfpathlineto{\pgfqpoint{2.317911in}{1.387258in}}%
\pgfpathlineto{\pgfqpoint{2.318047in}{1.391412in}}%
\pgfpathlineto{\pgfqpoint{2.318589in}{1.371755in}}%
\pgfpathlineto{\pgfqpoint{2.319268in}{1.384556in}}%
\pgfpathlineto{\pgfqpoint{2.319674in}{1.362693in}}%
\pgfpathlineto{\pgfqpoint{2.320895in}{1.366891in}}%
\pgfpathlineto{\pgfqpoint{2.321166in}{1.370809in}}%
\pgfpathlineto{\pgfqpoint{2.322251in}{1.365395in}}%
\pgfpathlineto{\pgfqpoint{2.322387in}{1.366929in}}%
\pgfpathlineto{\pgfqpoint{2.322523in}{1.366448in}}%
\pgfpathlineto{\pgfqpoint{2.323879in}{1.385995in}}%
\pgfpathlineto{\pgfqpoint{2.324150in}{1.374362in}}%
\pgfpathlineto{\pgfqpoint{2.324421in}{1.366255in}}%
\pgfpathlineto{\pgfqpoint{2.324828in}{1.390716in}}%
\pgfpathlineto{\pgfqpoint{2.325506in}{1.375123in}}%
\pgfpathlineto{\pgfqpoint{2.326863in}{1.389946in}}%
\pgfpathlineto{\pgfqpoint{2.327134in}{1.385299in}}%
\pgfpathlineto{\pgfqpoint{2.327405in}{1.380836in}}%
\pgfpathlineto{\pgfqpoint{2.328354in}{1.392778in}}%
\pgfpathlineto{\pgfqpoint{2.329846in}{1.400295in}}%
\pgfpathlineto{\pgfqpoint{2.329168in}{1.385628in}}%
\pgfpathlineto{\pgfqpoint{2.329982in}{1.397842in}}%
\pgfpathlineto{\pgfqpoint{2.330524in}{1.399135in}}%
\pgfpathlineto{\pgfqpoint{2.330660in}{1.398410in}}%
\pgfpathlineto{\pgfqpoint{2.333508in}{1.413765in}}%
\pgfpathlineto{\pgfqpoint{2.331203in}{1.394752in}}%
\pgfpathlineto{\pgfqpoint{2.333644in}{1.410835in}}%
\pgfpathlineto{\pgfqpoint{2.333915in}{1.405151in}}%
\pgfpathlineto{\pgfqpoint{2.334458in}{1.417536in}}%
\pgfpathlineto{\pgfqpoint{2.335000in}{1.409143in}}%
\pgfpathlineto{\pgfqpoint{2.336628in}{1.427171in}}%
\pgfpathlineto{\pgfqpoint{2.336899in}{1.422499in}}%
\pgfpathlineto{\pgfqpoint{2.337034in}{1.420540in}}%
\pgfpathlineto{\pgfqpoint{2.338119in}{1.427045in}}%
\pgfpathlineto{\pgfqpoint{2.338391in}{1.429256in}}%
\pgfpathlineto{\pgfqpoint{2.338798in}{1.421597in}}%
\pgfpathlineto{\pgfqpoint{2.338933in}{1.420281in}}%
\pgfpathlineto{\pgfqpoint{2.339340in}{1.427628in}}%
\pgfpathlineto{\pgfqpoint{2.339883in}{1.423376in}}%
\pgfpathlineto{\pgfqpoint{2.340289in}{1.429184in}}%
\pgfpathlineto{\pgfqpoint{2.340832in}{1.418393in}}%
\pgfpathlineto{\pgfqpoint{2.341510in}{1.428109in}}%
\pgfpathlineto{\pgfqpoint{2.343816in}{1.413742in}}%
\pgfpathlineto{\pgfqpoint{2.344223in}{1.418277in}}%
\pgfpathlineto{\pgfqpoint{2.344765in}{1.410741in}}%
\pgfpathlineto{\pgfqpoint{2.345308in}{1.415553in}}%
\pgfpathlineto{\pgfqpoint{2.345714in}{1.405081in}}%
\pgfpathlineto{\pgfqpoint{2.346257in}{1.416895in}}%
\pgfpathlineto{\pgfqpoint{2.347071in}{1.409736in}}%
\pgfpathlineto{\pgfqpoint{2.347206in}{1.411008in}}%
\pgfpathlineto{\pgfqpoint{2.347613in}{1.402914in}}%
\pgfpathlineto{\pgfqpoint{2.348969in}{1.392191in}}%
\pgfpathlineto{\pgfqpoint{2.349376in}{1.392262in}}%
\pgfpathlineto{\pgfqpoint{2.349919in}{1.400221in}}%
\pgfpathlineto{\pgfqpoint{2.350461in}{1.386482in}}%
\pgfpathlineto{\pgfqpoint{2.352631in}{1.370802in}}%
\pgfpathlineto{\pgfqpoint{2.350868in}{1.386910in}}%
\pgfpathlineto{\pgfqpoint{2.352767in}{1.377826in}}%
\pgfpathlineto{\pgfqpoint{2.353174in}{1.387683in}}%
\pgfpathlineto{\pgfqpoint{2.354123in}{1.374980in}}%
\pgfpathlineto{\pgfqpoint{2.354259in}{1.374264in}}%
\pgfpathlineto{\pgfqpoint{2.354530in}{1.380467in}}%
\pgfpathlineto{\pgfqpoint{2.354937in}{1.383271in}}%
\pgfpathlineto{\pgfqpoint{2.355751in}{1.379480in}}%
\pgfpathlineto{\pgfqpoint{2.355886in}{1.379525in}}%
\pgfpathlineto{\pgfqpoint{2.357378in}{1.367097in}}%
\pgfpathlineto{\pgfqpoint{2.356836in}{1.381079in}}%
\pgfpathlineto{\pgfqpoint{2.357649in}{1.373111in}}%
\pgfpathlineto{\pgfqpoint{2.359006in}{1.384589in}}%
\pgfpathlineto{\pgfqpoint{2.359277in}{1.383836in}}%
\pgfpathlineto{\pgfqpoint{2.360633in}{1.393915in}}%
\pgfpathlineto{\pgfqpoint{2.360904in}{1.389897in}}%
\pgfpathlineto{\pgfqpoint{2.361176in}{1.385558in}}%
\pgfpathlineto{\pgfqpoint{2.361718in}{1.401088in}}%
\pgfpathlineto{\pgfqpoint{2.362261in}{1.388387in}}%
\pgfpathlineto{\pgfqpoint{2.363753in}{1.409986in}}%
\pgfpathlineto{\pgfqpoint{2.364024in}{1.401337in}}%
\pgfpathlineto{\pgfqpoint{2.364159in}{1.396559in}}%
\pgfpathlineto{\pgfqpoint{2.364838in}{1.411015in}}%
\pgfpathlineto{\pgfqpoint{2.365380in}{1.403483in}}%
\pgfpathlineto{\pgfqpoint{2.367821in}{1.422278in}}%
\pgfpathlineto{\pgfqpoint{2.369178in}{1.417619in}}%
\pgfpathlineto{\pgfqpoint{2.368635in}{1.427037in}}%
\pgfpathlineto{\pgfqpoint{2.369313in}{1.419249in}}%
\pgfpathlineto{\pgfqpoint{2.371483in}{1.428037in}}%
\pgfpathlineto{\pgfqpoint{2.371619in}{1.427744in}}%
\pgfpathlineto{\pgfqpoint{2.372839in}{1.423814in}}%
\pgfpathlineto{\pgfqpoint{2.372433in}{1.429804in}}%
\pgfpathlineto{\pgfqpoint{2.373111in}{1.427195in}}%
\pgfpathlineto{\pgfqpoint{2.373382in}{1.434187in}}%
\pgfpathlineto{\pgfqpoint{2.374060in}{1.421481in}}%
\pgfpathlineto{\pgfqpoint{2.374603in}{1.428620in}}%
\pgfpathlineto{\pgfqpoint{2.375009in}{1.422196in}}%
\pgfpathlineto{\pgfqpoint{2.376094in}{1.426511in}}%
\pgfpathlineto{\pgfqpoint{2.376501in}{1.428460in}}%
\pgfpathlineto{\pgfqpoint{2.377179in}{1.421608in}}%
\pgfpathlineto{\pgfqpoint{2.377451in}{1.420415in}}%
\pgfpathlineto{\pgfqpoint{2.377858in}{1.424303in}}%
\pgfpathlineto{\pgfqpoint{2.378129in}{1.426992in}}%
\pgfpathlineto{\pgfqpoint{2.378671in}{1.419088in}}%
\pgfpathlineto{\pgfqpoint{2.379214in}{1.422643in}}%
\pgfpathlineto{\pgfqpoint{2.380706in}{1.413658in}}%
\pgfpathlineto{\pgfqpoint{2.380163in}{1.422999in}}%
\pgfpathlineto{\pgfqpoint{2.380977in}{1.417070in}}%
\pgfpathlineto{\pgfqpoint{2.381926in}{1.419365in}}%
\pgfpathlineto{\pgfqpoint{2.381655in}{1.415793in}}%
\pgfpathlineto{\pgfqpoint{2.382198in}{1.417079in}}%
\pgfpathlineto{\pgfqpoint{2.383554in}{1.401840in}}%
\pgfpathlineto{\pgfqpoint{2.383825in}{1.407361in}}%
\pgfpathlineto{\pgfqpoint{2.384096in}{1.409424in}}%
\pgfpathlineto{\pgfqpoint{2.384639in}{1.399411in}}%
\pgfpathlineto{\pgfqpoint{2.385181in}{1.408157in}}%
\pgfpathlineto{\pgfqpoint{2.387487in}{1.390083in}}%
\pgfpathlineto{\pgfqpoint{2.388843in}{1.398488in}}%
\pgfpathlineto{\pgfqpoint{2.388979in}{1.395386in}}%
\pgfpathlineto{\pgfqpoint{2.390471in}{1.382746in}}%
\pgfpathlineto{\pgfqpoint{2.389793in}{1.397078in}}%
\pgfpathlineto{\pgfqpoint{2.390606in}{1.388690in}}%
\pgfpathlineto{\pgfqpoint{2.391827in}{1.400300in}}%
\pgfpathlineto{\pgfqpoint{2.392098in}{1.394366in}}%
\pgfpathlineto{\pgfqpoint{2.392369in}{1.385589in}}%
\pgfpathlineto{\pgfqpoint{2.393048in}{1.394975in}}%
\pgfpathlineto{\pgfqpoint{2.393454in}{1.393394in}}%
\pgfpathlineto{\pgfqpoint{2.394811in}{1.405547in}}%
\pgfpathlineto{\pgfqpoint{2.395218in}{1.401173in}}%
\pgfpathlineto{\pgfqpoint{2.395353in}{1.400819in}}%
\pgfpathlineto{\pgfqpoint{2.395489in}{1.402984in}}%
\pgfpathlineto{\pgfqpoint{2.396981in}{1.413184in}}%
\pgfpathlineto{\pgfqpoint{2.396303in}{1.402105in}}%
\pgfpathlineto{\pgfqpoint{2.397252in}{1.410915in}}%
\pgfpathlineto{\pgfqpoint{2.397523in}{1.409639in}}%
\pgfpathlineto{\pgfqpoint{2.398066in}{1.417352in}}%
\pgfpathlineto{\pgfqpoint{2.400507in}{1.430872in}}%
\pgfpathlineto{\pgfqpoint{2.400778in}{1.427721in}}%
\pgfpathlineto{\pgfqpoint{2.401049in}{1.425049in}}%
\pgfpathlineto{\pgfqpoint{2.401592in}{1.433085in}}%
\pgfpathlineto{\pgfqpoint{2.402270in}{1.427815in}}%
\pgfpathlineto{\pgfqpoint{2.404033in}{1.432858in}}%
\pgfpathlineto{\pgfqpoint{2.404169in}{1.432863in}}%
\pgfpathlineto{\pgfqpoint{2.405254in}{1.440921in}}%
\pgfpathlineto{\pgfqpoint{2.405661in}{1.434827in}}%
\pgfpathlineto{\pgfqpoint{2.405796in}{1.433453in}}%
\pgfpathlineto{\pgfqpoint{2.406474in}{1.439451in}}%
\pgfpathlineto{\pgfqpoint{2.407017in}{1.436789in}}%
\pgfpathlineto{\pgfqpoint{2.408373in}{1.441098in}}%
\pgfpathlineto{\pgfqpoint{2.407831in}{1.434361in}}%
\pgfpathlineto{\pgfqpoint{2.408644in}{1.438285in}}%
\pgfpathlineto{\pgfqpoint{2.409729in}{1.434195in}}%
\pgfpathlineto{\pgfqpoint{2.409187in}{1.439483in}}%
\pgfpathlineto{\pgfqpoint{2.410001in}{1.436916in}}%
\pgfpathlineto{\pgfqpoint{2.410136in}{1.438927in}}%
\pgfpathlineto{\pgfqpoint{2.410814in}{1.428065in}}%
\pgfpathlineto{\pgfqpoint{2.411357in}{1.434109in}}%
\pgfpathlineto{\pgfqpoint{2.412849in}{1.422974in}}%
\pgfpathlineto{\pgfqpoint{2.412984in}{1.426051in}}%
\pgfpathlineto{\pgfqpoint{2.413256in}{1.430435in}}%
\pgfpathlineto{\pgfqpoint{2.413934in}{1.419878in}}%
\pgfpathlineto{\pgfqpoint{2.414341in}{1.423238in}}%
\pgfpathlineto{\pgfqpoint{2.417596in}{1.404380in}}%
\pgfpathlineto{\pgfqpoint{2.417731in}{1.407869in}}%
\pgfpathlineto{\pgfqpoint{2.418138in}{1.411230in}}%
\pgfpathlineto{\pgfqpoint{2.418816in}{1.402854in}}%
\pgfpathlineto{\pgfqpoint{2.418952in}{1.403301in}}%
\pgfpathlineto{\pgfqpoint{2.419223in}{1.400910in}}%
\pgfpathlineto{\pgfqpoint{2.420037in}{1.404790in}}%
\pgfpathlineto{\pgfqpoint{2.421258in}{1.387176in}}%
\pgfpathlineto{\pgfqpoint{2.422478in}{1.378615in}}%
\pgfpathlineto{\pgfqpoint{2.421800in}{1.394066in}}%
\pgfpathlineto{\pgfqpoint{2.422749in}{1.384031in}}%
\pgfpathlineto{\pgfqpoint{2.423156in}{1.389182in}}%
\pgfpathlineto{\pgfqpoint{2.423563in}{1.378616in}}%
\pgfpathlineto{\pgfqpoint{2.424513in}{1.389011in}}%
\pgfpathlineto{\pgfqpoint{2.425733in}{1.383046in}}%
\pgfpathlineto{\pgfqpoint{2.425191in}{1.389416in}}%
\pgfpathlineto{\pgfqpoint{2.426140in}{1.387184in}}%
\pgfpathlineto{\pgfqpoint{2.426547in}{1.396742in}}%
\pgfpathlineto{\pgfqpoint{2.427089in}{1.383849in}}%
\pgfpathlineto{\pgfqpoint{2.427768in}{1.395370in}}%
\pgfpathlineto{\pgfqpoint{2.428174in}{1.386205in}}%
\pgfpathlineto{\pgfqpoint{2.428717in}{1.398467in}}%
\pgfpathlineto{\pgfqpoint{2.429395in}{1.391317in}}%
\pgfpathlineto{\pgfqpoint{2.429802in}{1.403922in}}%
\pgfpathlineto{\pgfqpoint{2.431023in}{1.397774in}}%
\pgfpathlineto{\pgfqpoint{2.431158in}{1.397066in}}%
\pgfpathlineto{\pgfqpoint{2.431429in}{1.403530in}}%
\pgfpathlineto{\pgfqpoint{2.433599in}{1.424332in}}%
\pgfpathlineto{\pgfqpoint{2.433735in}{1.422958in}}%
\pgfpathlineto{\pgfqpoint{2.434142in}{1.415709in}}%
\pgfpathlineto{\pgfqpoint{2.435227in}{1.420612in}}%
\pgfpathlineto{\pgfqpoint{2.436583in}{1.425622in}}%
\pgfpathlineto{\pgfqpoint{2.436041in}{1.420006in}}%
\pgfpathlineto{\pgfqpoint{2.436854in}{1.422549in}}%
\pgfpathlineto{\pgfqpoint{2.438211in}{1.432119in}}%
\pgfpathlineto{\pgfqpoint{2.439431in}{1.436979in}}%
\pgfpathlineto{\pgfqpoint{2.438889in}{1.426098in}}%
\pgfpathlineto{\pgfqpoint{2.439703in}{1.433366in}}%
\pgfpathlineto{\pgfqpoint{2.440923in}{1.428959in}}%
\pgfpathlineto{\pgfqpoint{2.440516in}{1.435862in}}%
\pgfpathlineto{\pgfqpoint{2.441194in}{1.432642in}}%
\pgfpathlineto{\pgfqpoint{2.441737in}{1.438680in}}%
\pgfpathlineto{\pgfqpoint{2.442822in}{1.435949in}}%
\pgfpathlineto{\pgfqpoint{2.443907in}{1.434308in}}%
\pgfpathlineto{\pgfqpoint{2.443364in}{1.439159in}}%
\pgfpathlineto{\pgfqpoint{2.444721in}{1.434763in}}%
\pgfpathlineto{\pgfqpoint{2.445128in}{1.440200in}}%
\pgfpathlineto{\pgfqpoint{2.445806in}{1.431756in}}%
\pgfpathlineto{\pgfqpoint{2.446213in}{1.433679in}}%
\pgfpathlineto{\pgfqpoint{2.447026in}{1.433902in}}%
\pgfpathlineto{\pgfqpoint{2.446619in}{1.431619in}}%
\pgfpathlineto{\pgfqpoint{2.447162in}{1.433775in}}%
\pgfpathlineto{\pgfqpoint{2.448654in}{1.425527in}}%
\pgfpathlineto{\pgfqpoint{2.448925in}{1.429601in}}%
\pgfpathlineto{\pgfqpoint{2.450010in}{1.431639in}}%
\pgfpathlineto{\pgfqpoint{2.449603in}{1.424550in}}%
\pgfpathlineto{\pgfqpoint{2.450146in}{1.430143in}}%
\pgfpathlineto{\pgfqpoint{2.450688in}{1.416635in}}%
\pgfpathlineto{\pgfqpoint{2.451909in}{1.419242in}}%
\pgfpathlineto{\pgfqpoint{2.452180in}{1.418892in}}%
\pgfpathlineto{\pgfqpoint{2.454350in}{1.401003in}}%
\pgfpathlineto{\pgfqpoint{2.454757in}{1.413432in}}%
\pgfpathlineto{\pgfqpoint{2.455299in}{1.399929in}}%
\pgfpathlineto{\pgfqpoint{2.456113in}{1.406989in}}%
\pgfpathlineto{\pgfqpoint{2.457334in}{1.395380in}}%
\pgfpathlineto{\pgfqpoint{2.456791in}{1.412179in}}%
\pgfpathlineto{\pgfqpoint{2.457741in}{1.404241in}}%
\pgfpathlineto{\pgfqpoint{2.458012in}{1.405764in}}%
\pgfpathlineto{\pgfqpoint{2.458283in}{1.401919in}}%
\pgfpathlineto{\pgfqpoint{2.458826in}{1.405202in}}%
\pgfpathlineto{\pgfqpoint{2.459233in}{1.391923in}}%
\pgfpathlineto{\pgfqpoint{2.460453in}{1.395986in}}%
\pgfpathlineto{\pgfqpoint{2.461809in}{1.407918in}}%
\pgfpathlineto{\pgfqpoint{2.462081in}{1.403367in}}%
\pgfpathlineto{\pgfqpoint{2.462216in}{1.400086in}}%
\pgfpathlineto{\pgfqpoint{2.463301in}{1.411177in}}%
\pgfpathlineto{\pgfqpoint{2.465471in}{1.425551in}}%
\pgfpathlineto{\pgfqpoint{2.465743in}{1.422869in}}%
\pgfpathlineto{\pgfqpoint{2.466014in}{1.417772in}}%
\pgfpathlineto{\pgfqpoint{2.466556in}{1.431098in}}%
\pgfpathlineto{\pgfqpoint{2.467099in}{1.422153in}}%
\pgfpathlineto{\pgfqpoint{2.468591in}{1.433850in}}%
\pgfpathlineto{\pgfqpoint{2.469133in}{1.431637in}}%
\pgfpathlineto{\pgfqpoint{2.473202in}{1.447722in}}%
\pgfpathlineto{\pgfqpoint{2.474423in}{1.446763in}}%
\pgfpathlineto{\pgfqpoint{2.475779in}{1.444406in}}%
\pgfpathlineto{\pgfqpoint{2.475236in}{1.450556in}}%
\pgfpathlineto{\pgfqpoint{2.475914in}{1.446891in}}%
\pgfpathlineto{\pgfqpoint{2.476186in}{1.449992in}}%
\pgfpathlineto{\pgfqpoint{2.476728in}{1.443775in}}%
\pgfpathlineto{\pgfqpoint{2.477406in}{1.446750in}}%
\pgfpathlineto{\pgfqpoint{2.478898in}{1.442101in}}%
\pgfpathlineto{\pgfqpoint{2.478220in}{1.449811in}}%
\pgfpathlineto{\pgfqpoint{2.479034in}{1.443298in}}%
\pgfpathlineto{\pgfqpoint{2.479305in}{1.447344in}}%
\pgfpathlineto{\pgfqpoint{2.480390in}{1.441581in}}%
\pgfpathlineto{\pgfqpoint{2.481339in}{1.437404in}}%
\pgfpathlineto{\pgfqpoint{2.481611in}{1.441204in}}%
\pgfpathlineto{\pgfqpoint{2.481746in}{1.442934in}}%
\pgfpathlineto{\pgfqpoint{2.482560in}{1.434835in}}%
\pgfpathlineto{\pgfqpoint{2.483103in}{1.440620in}}%
\pgfpathlineto{\pgfqpoint{2.484594in}{1.431962in}}%
\pgfpathlineto{\pgfqpoint{2.484866in}{1.436145in}}%
\pgfpathlineto{\pgfqpoint{2.485137in}{1.437180in}}%
\pgfpathlineto{\pgfqpoint{2.485544in}{1.431118in}}%
\pgfpathlineto{\pgfqpoint{2.485951in}{1.433524in}}%
\pgfpathlineto{\pgfqpoint{2.486493in}{1.430815in}}%
\pgfpathlineto{\pgfqpoint{2.486764in}{1.434930in}}%
\pgfpathlineto{\pgfqpoint{2.486900in}{1.435810in}}%
\pgfpathlineto{\pgfqpoint{2.487307in}{1.429776in}}%
\pgfpathlineto{\pgfqpoint{2.488528in}{1.428674in}}%
\pgfpathlineto{\pgfqpoint{2.487985in}{1.436778in}}%
\pgfpathlineto{\pgfqpoint{2.488663in}{1.430211in}}%
\pgfpathlineto{\pgfqpoint{2.489070in}{1.432964in}}%
\pgfpathlineto{\pgfqpoint{2.489477in}{1.427860in}}%
\pgfpathlineto{\pgfqpoint{2.490019in}{1.429343in}}%
\pgfpathlineto{\pgfqpoint{2.492054in}{1.419617in}}%
\pgfpathlineto{\pgfqpoint{2.492189in}{1.420969in}}%
\pgfpathlineto{\pgfqpoint{2.492596in}{1.426288in}}%
\pgfpathlineto{\pgfqpoint{2.493139in}{1.417649in}}%
\pgfpathlineto{\pgfqpoint{2.493681in}{1.423183in}}%
\pgfpathlineto{\pgfqpoint{2.494224in}{1.409922in}}%
\pgfpathlineto{\pgfqpoint{2.494902in}{1.424523in}}%
\pgfpathlineto{\pgfqpoint{2.495444in}{1.418103in}}%
\pgfpathlineto{\pgfqpoint{2.496665in}{1.428998in}}%
\pgfpathlineto{\pgfqpoint{2.497343in}{1.426960in}}%
\pgfpathlineto{\pgfqpoint{2.497479in}{1.426818in}}%
\pgfpathlineto{\pgfqpoint{2.497750in}{1.428233in}}%
\pgfpathlineto{\pgfqpoint{2.501276in}{1.442967in}}%
\pgfpathlineto{\pgfqpoint{2.501819in}{1.441040in}}%
\pgfpathlineto{\pgfqpoint{2.502226in}{1.444104in}}%
\pgfpathlineto{\pgfqpoint{2.503446in}{1.451069in}}%
\pgfpathlineto{\pgfqpoint{2.503718in}{1.444663in}}%
\pgfpathlineto{\pgfqpoint{2.503853in}{1.443422in}}%
\pgfpathlineto{\pgfqpoint{2.504260in}{1.452094in}}%
\pgfpathlineto{\pgfqpoint{2.505345in}{1.455524in}}%
\pgfpathlineto{\pgfqpoint{2.504938in}{1.448337in}}%
\pgfpathlineto{\pgfqpoint{2.505616in}{1.453409in}}%
\pgfpathlineto{\pgfqpoint{2.505888in}{1.448283in}}%
\pgfpathlineto{\pgfqpoint{2.506430in}{1.455716in}}%
\pgfpathlineto{\pgfqpoint{2.507244in}{1.449481in}}%
\pgfpathlineto{\pgfqpoint{2.508600in}{1.452751in}}%
\pgfpathlineto{\pgfqpoint{2.509007in}{1.451177in}}%
\pgfpathlineto{\pgfqpoint{2.509549in}{1.449281in}}%
\pgfpathlineto{\pgfqpoint{2.509821in}{1.451171in}}%
\pgfpathlineto{\pgfqpoint{2.510092in}{1.453707in}}%
\pgfpathlineto{\pgfqpoint{2.510634in}{1.446456in}}%
\pgfpathlineto{\pgfqpoint{2.511313in}{1.451297in}}%
\pgfpathlineto{\pgfqpoint{2.513483in}{1.445085in}}%
\pgfpathlineto{\pgfqpoint{2.513618in}{1.445147in}}%
\pgfpathlineto{\pgfqpoint{2.514025in}{1.446345in}}%
\pgfpathlineto{\pgfqpoint{2.514296in}{1.442853in}}%
\pgfpathlineto{\pgfqpoint{2.515653in}{1.433916in}}%
\pgfpathlineto{\pgfqpoint{2.514974in}{1.443720in}}%
\pgfpathlineto{\pgfqpoint{2.516059in}{1.439556in}}%
\pgfpathlineto{\pgfqpoint{2.516195in}{1.439681in}}%
\pgfpathlineto{\pgfqpoint{2.516738in}{1.433425in}}%
\pgfpathlineto{\pgfqpoint{2.517958in}{1.436022in}}%
\pgfpathlineto{\pgfqpoint{2.518365in}{1.438816in}}%
\pgfpathlineto{\pgfqpoint{2.518772in}{1.434562in}}%
\pgfpathlineto{\pgfqpoint{2.520264in}{1.426276in}}%
\pgfpathlineto{\pgfqpoint{2.520535in}{1.430560in}}%
\pgfpathlineto{\pgfqpoint{2.520806in}{1.432683in}}%
\pgfpathlineto{\pgfqpoint{2.521349in}{1.423695in}}%
\pgfpathlineto{\pgfqpoint{2.521891in}{1.429048in}}%
\pgfpathlineto{\pgfqpoint{2.522434in}{1.420531in}}%
\pgfpathlineto{\pgfqpoint{2.523519in}{1.424729in}}%
\pgfpathlineto{\pgfqpoint{2.524875in}{1.429887in}}%
\pgfpathlineto{\pgfqpoint{2.524197in}{1.421784in}}%
\pgfpathlineto{\pgfqpoint{2.525011in}{1.426735in}}%
\pgfpathlineto{\pgfqpoint{2.526367in}{1.416006in}}%
\pgfpathlineto{\pgfqpoint{2.526638in}{1.418831in}}%
\pgfpathlineto{\pgfqpoint{2.527723in}{1.420582in}}%
\pgfpathlineto{\pgfqpoint{2.527181in}{1.415541in}}%
\pgfpathlineto{\pgfqpoint{2.528130in}{1.418898in}}%
\pgfpathlineto{\pgfqpoint{2.528266in}{1.417995in}}%
\pgfpathlineto{\pgfqpoint{2.528944in}{1.421623in}}%
\pgfpathlineto{\pgfqpoint{2.529079in}{1.421110in}}%
\pgfpathlineto{\pgfqpoint{2.529486in}{1.427011in}}%
\pgfpathlineto{\pgfqpoint{2.530571in}{1.425138in}}%
\pgfpathlineto{\pgfqpoint{2.530978in}{1.415375in}}%
\pgfpathlineto{\pgfqpoint{2.531521in}{1.430389in}}%
\pgfpathlineto{\pgfqpoint{2.532063in}{1.421088in}}%
\pgfpathlineto{\pgfqpoint{2.533555in}{1.435687in}}%
\pgfpathlineto{\pgfqpoint{2.534098in}{1.430710in}}%
\pgfpathlineto{\pgfqpoint{2.536539in}{1.445682in}}%
\pgfpathlineto{\pgfqpoint{2.536810in}{1.443778in}}%
\pgfpathlineto{\pgfqpoint{2.538031in}{1.447940in}}%
\pgfpathlineto{\pgfqpoint{2.538438in}{1.449539in}}%
\pgfpathlineto{\pgfqpoint{2.538980in}{1.445344in}}%
\pgfpathlineto{\pgfqpoint{2.539523in}{1.448224in}}%
\pgfpathlineto{\pgfqpoint{2.539658in}{1.447761in}}%
\pgfpathlineto{\pgfqpoint{2.540065in}{1.451103in}}%
\pgfpathlineto{\pgfqpoint{2.540743in}{1.448752in}}%
\pgfpathlineto{\pgfqpoint{2.542099in}{1.457276in}}%
\pgfpathlineto{\pgfqpoint{2.542506in}{1.454009in}}%
\pgfpathlineto{\pgfqpoint{2.542642in}{1.453786in}}%
\pgfpathlineto{\pgfqpoint{2.542778in}{1.454129in}}%
\pgfpathlineto{\pgfqpoint{2.543184in}{1.463670in}}%
\pgfpathlineto{\pgfqpoint{2.544541in}{1.461287in}}%
\pgfpathlineto{\pgfqpoint{2.547389in}{1.455976in}}%
\pgfpathlineto{\pgfqpoint{2.545219in}{1.463333in}}%
\pgfpathlineto{\pgfqpoint{2.547660in}{1.457209in}}%
\pgfpathlineto{\pgfqpoint{2.547931in}{1.459374in}}%
\pgfpathlineto{\pgfqpoint{2.548474in}{1.455089in}}%
\pgfpathlineto{\pgfqpoint{2.549016in}{1.455950in}}%
\pgfpathlineto{\pgfqpoint{2.551458in}{1.448259in}}%
\pgfpathlineto{\pgfqpoint{2.551729in}{1.449638in}}%
\pgfpathlineto{\pgfqpoint{2.552271in}{1.443060in}}%
\pgfpathlineto{\pgfqpoint{2.553492in}{1.441688in}}%
\pgfpathlineto{\pgfqpoint{2.552949in}{1.446145in}}%
\pgfpathlineto{\pgfqpoint{2.553628in}{1.443080in}}%
\pgfpathlineto{\pgfqpoint{2.553899in}{1.446256in}}%
\pgfpathlineto{\pgfqpoint{2.554441in}{1.438260in}}%
\pgfpathlineto{\pgfqpoint{2.555119in}{1.443651in}}%
\pgfpathlineto{\pgfqpoint{2.556883in}{1.433222in}}%
\pgfpathlineto{\pgfqpoint{2.557018in}{1.433915in}}%
\pgfpathlineto{\pgfqpoint{2.557154in}{1.434000in}}%
\pgfpathlineto{\pgfqpoint{2.557561in}{1.432670in}}%
\pgfpathlineto{\pgfqpoint{2.557696in}{1.432823in}}%
\pgfpathlineto{\pgfqpoint{2.559188in}{1.421351in}}%
\pgfpathlineto{\pgfqpoint{2.558510in}{1.433973in}}%
\pgfpathlineto{\pgfqpoint{2.559459in}{1.425946in}}%
\pgfpathlineto{\pgfqpoint{2.559731in}{1.430592in}}%
\pgfpathlineto{\pgfqpoint{2.560273in}{1.421607in}}%
\pgfpathlineto{\pgfqpoint{2.560816in}{1.425833in}}%
\pgfpathlineto{\pgfqpoint{2.562172in}{1.420413in}}%
\pgfpathlineto{\pgfqpoint{2.561765in}{1.427820in}}%
\pgfpathlineto{\pgfqpoint{2.562443in}{1.423694in}}%
\pgfpathlineto{\pgfqpoint{2.564613in}{1.435541in}}%
\pgfpathlineto{\pgfqpoint{2.564884in}{1.434411in}}%
\pgfpathlineto{\pgfqpoint{2.565291in}{1.436507in}}%
\pgfpathlineto{\pgfqpoint{2.565427in}{1.438889in}}%
\pgfpathlineto{\pgfqpoint{2.566105in}{1.432300in}}%
\pgfpathlineto{\pgfqpoint{2.566783in}{1.438070in}}%
\pgfpathlineto{\pgfqpoint{2.567054in}{1.436355in}}%
\pgfpathlineto{\pgfqpoint{2.567868in}{1.439396in}}%
\pgfpathlineto{\pgfqpoint{2.570309in}{1.450689in}}%
\pgfpathlineto{\pgfqpoint{2.568818in}{1.438153in}}%
\pgfpathlineto{\pgfqpoint{2.570581in}{1.447050in}}%
\pgfpathlineto{\pgfqpoint{2.570852in}{1.440036in}}%
\pgfpathlineto{\pgfqpoint{2.571394in}{1.453423in}}%
\pgfpathlineto{\pgfqpoint{2.572073in}{1.446733in}}%
\pgfpathlineto{\pgfqpoint{2.573429in}{1.456494in}}%
\pgfpathlineto{\pgfqpoint{2.573700in}{1.453072in}}%
\pgfpathlineto{\pgfqpoint{2.573971in}{1.449839in}}%
\pgfpathlineto{\pgfqpoint{2.574378in}{1.454438in}}%
\pgfpathlineto{\pgfqpoint{2.575192in}{1.453011in}}%
\pgfpathlineto{\pgfqpoint{2.578176in}{1.460494in}}%
\pgfpathlineto{\pgfqpoint{2.578447in}{1.458925in}}%
\pgfpathlineto{\pgfqpoint{2.579532in}{1.453993in}}%
\pgfpathlineto{\pgfqpoint{2.578989in}{1.459689in}}%
\pgfpathlineto{\pgfqpoint{2.579939in}{1.458447in}}%
\pgfpathlineto{\pgfqpoint{2.580074in}{1.458665in}}%
\pgfpathlineto{\pgfqpoint{2.580210in}{1.457425in}}%
\pgfpathlineto{\pgfqpoint{2.581566in}{1.451926in}}%
\pgfpathlineto{\pgfqpoint{2.581838in}{1.455606in}}%
\pgfpathlineto{\pgfqpoint{2.582109in}{1.459074in}}%
\pgfpathlineto{\pgfqpoint{2.582651in}{1.451729in}}%
\pgfpathlineto{\pgfqpoint{2.583329in}{1.458616in}}%
\pgfpathlineto{\pgfqpoint{2.585228in}{1.449652in}}%
\pgfpathlineto{\pgfqpoint{2.585771in}{1.451909in}}%
\pgfpathlineto{\pgfqpoint{2.587263in}{1.441673in}}%
\pgfpathlineto{\pgfqpoint{2.588348in}{1.437861in}}%
\pgfpathlineto{\pgfqpoint{2.587805in}{1.446629in}}%
\pgfpathlineto{\pgfqpoint{2.588619in}{1.440625in}}%
\pgfpathlineto{\pgfqpoint{2.588754in}{1.443034in}}%
\pgfpathlineto{\pgfqpoint{2.589297in}{1.435322in}}%
\pgfpathlineto{\pgfqpoint{2.589975in}{1.440007in}}%
\pgfpathlineto{\pgfqpoint{2.591196in}{1.433991in}}%
\pgfpathlineto{\pgfqpoint{2.590789in}{1.440754in}}%
\pgfpathlineto{\pgfqpoint{2.591467in}{1.437503in}}%
\pgfpathlineto{\pgfqpoint{2.591738in}{1.442478in}}%
\pgfpathlineto{\pgfqpoint{2.592281in}{1.430833in}}%
\pgfpathlineto{\pgfqpoint{2.592823in}{1.437322in}}%
\pgfpathlineto{\pgfqpoint{2.593773in}{1.437480in}}%
\pgfpathlineto{\pgfqpoint{2.595264in}{1.430272in}}%
\pgfpathlineto{\pgfqpoint{2.595400in}{1.430661in}}%
\pgfpathlineto{\pgfqpoint{2.595671in}{1.427841in}}%
\pgfpathlineto{\pgfqpoint{2.595943in}{1.424552in}}%
\pgfpathlineto{\pgfqpoint{2.596485in}{1.437288in}}%
\pgfpathlineto{\pgfqpoint{2.597163in}{1.427406in}}%
\pgfpathlineto{\pgfqpoint{2.598519in}{1.438963in}}%
\pgfpathlineto{\pgfqpoint{2.598791in}{1.433795in}}%
\pgfpathlineto{\pgfqpoint{2.599062in}{1.430071in}}%
\pgfpathlineto{\pgfqpoint{2.599604in}{1.442723in}}%
\pgfpathlineto{\pgfqpoint{2.600147in}{1.436811in}}%
\pgfpathlineto{\pgfqpoint{2.601503in}{1.445352in}}%
\pgfpathlineto{\pgfqpoint{2.601910in}{1.442102in}}%
\pgfpathlineto{\pgfqpoint{2.602046in}{1.441853in}}%
\pgfpathlineto{\pgfqpoint{2.602181in}{1.443392in}}%
\pgfpathlineto{\pgfqpoint{2.604351in}{1.455702in}}%
\pgfpathlineto{\pgfqpoint{2.607064in}{1.464647in}}%
\pgfpathlineto{\pgfqpoint{2.607606in}{1.460267in}}%
\pgfpathlineto{\pgfqpoint{2.608013in}{1.465853in}}%
\pgfpathlineto{\pgfqpoint{2.608149in}{1.468122in}}%
\pgfpathlineto{\pgfqpoint{2.608691in}{1.459575in}}%
\pgfpathlineto{\pgfqpoint{2.609505in}{1.466827in}}%
\pgfpathlineto{\pgfqpoint{2.610861in}{1.464058in}}%
\pgfpathlineto{\pgfqpoint{2.610183in}{1.468066in}}%
\pgfpathlineto{\pgfqpoint{2.610997in}{1.465556in}}%
\pgfpathlineto{\pgfqpoint{2.611268in}{1.467606in}}%
\pgfpathlineto{\pgfqpoint{2.612082in}{1.465203in}}%
\pgfpathlineto{\pgfqpoint{2.612489in}{1.465464in}}%
\pgfpathlineto{\pgfqpoint{2.612760in}{1.465708in}}%
\pgfpathlineto{\pgfqpoint{2.613031in}{1.464450in}}%
\pgfpathlineto{\pgfqpoint{2.614523in}{1.457719in}}%
\pgfpathlineto{\pgfqpoint{2.614659in}{1.459781in}}%
\pgfpathlineto{\pgfqpoint{2.614930in}{1.461783in}}%
\pgfpathlineto{\pgfqpoint{2.615608in}{1.454196in}}%
\pgfpathlineto{\pgfqpoint{2.616015in}{1.457619in}}%
\pgfpathlineto{\pgfqpoint{2.617100in}{1.452585in}}%
\pgfpathlineto{\pgfqpoint{2.618456in}{1.443323in}}%
\pgfpathlineto{\pgfqpoint{2.618863in}{1.447169in}}%
\pgfpathlineto{\pgfqpoint{2.620491in}{1.433903in}}%
\pgfpathlineto{\pgfqpoint{2.620898in}{1.440053in}}%
\pgfpathlineto{\pgfqpoint{2.621169in}{1.442771in}}%
\pgfpathlineto{\pgfqpoint{2.622118in}{1.434460in}}%
\pgfpathlineto{\pgfqpoint{2.624017in}{1.425601in}}%
\pgfpathlineto{\pgfqpoint{2.623339in}{1.435256in}}%
\pgfpathlineto{\pgfqpoint{2.624153in}{1.426461in}}%
\pgfpathlineto{\pgfqpoint{2.624695in}{1.433799in}}%
\pgfpathlineto{\pgfqpoint{2.625238in}{1.420914in}}%
\pgfpathlineto{\pgfqpoint{2.625644in}{1.427088in}}%
\pgfpathlineto{\pgfqpoint{2.626187in}{1.419385in}}%
\pgfpathlineto{\pgfqpoint{2.626729in}{1.428477in}}%
\pgfpathlineto{\pgfqpoint{2.627408in}{1.422834in}}%
\pgfpathlineto{\pgfqpoint{2.627814in}{1.429553in}}%
\pgfpathlineto{\pgfqpoint{2.628899in}{1.424060in}}%
\pgfpathlineto{\pgfqpoint{2.629171in}{1.421870in}}%
\pgfpathlineto{\pgfqpoint{2.629713in}{1.429945in}}%
\pgfpathlineto{\pgfqpoint{2.630256in}{1.425232in}}%
\pgfpathlineto{\pgfqpoint{2.631748in}{1.428223in}}%
\pgfpathlineto{\pgfqpoint{2.631205in}{1.421192in}}%
\pgfpathlineto{\pgfqpoint{2.631883in}{1.426774in}}%
\pgfpathlineto{\pgfqpoint{2.632154in}{1.425432in}}%
\pgfpathlineto{\pgfqpoint{2.632697in}{1.429341in}}%
\pgfpathlineto{\pgfqpoint{2.634324in}{1.438443in}}%
\pgfpathlineto{\pgfqpoint{2.633782in}{1.428711in}}%
\pgfpathlineto{\pgfqpoint{2.634596in}{1.437603in}}%
\pgfpathlineto{\pgfqpoint{2.635681in}{1.432695in}}%
\pgfpathlineto{\pgfqpoint{2.635274in}{1.440693in}}%
\pgfpathlineto{\pgfqpoint{2.636088in}{1.438361in}}%
\pgfpathlineto{\pgfqpoint{2.637444in}{1.445328in}}%
\pgfpathlineto{\pgfqpoint{2.636901in}{1.435653in}}%
\pgfpathlineto{\pgfqpoint{2.637715in}{1.443885in}}%
\pgfpathlineto{\pgfqpoint{2.637851in}{1.441098in}}%
\pgfpathlineto{\pgfqpoint{2.638529in}{1.449088in}}%
\pgfpathlineto{\pgfqpoint{2.639071in}{1.445288in}}%
\pgfpathlineto{\pgfqpoint{2.639749in}{1.448875in}}%
\pgfpathlineto{\pgfqpoint{2.640563in}{1.446024in}}%
\pgfpathlineto{\pgfqpoint{2.640699in}{1.445960in}}%
\pgfpathlineto{\pgfqpoint{2.640834in}{1.446537in}}%
\pgfpathlineto{\pgfqpoint{2.642326in}{1.453430in}}%
\pgfpathlineto{\pgfqpoint{2.642598in}{1.450502in}}%
\pgfpathlineto{\pgfqpoint{2.645039in}{1.461064in}}%
\pgfpathlineto{\pgfqpoint{2.645310in}{1.456471in}}%
\pgfpathlineto{\pgfqpoint{2.645581in}{1.453133in}}%
\pgfpathlineto{\pgfqpoint{2.646124in}{1.461205in}}%
\pgfpathlineto{\pgfqpoint{2.646802in}{1.456460in}}%
\pgfpathlineto{\pgfqpoint{2.648158in}{1.462052in}}%
\pgfpathlineto{\pgfqpoint{2.647616in}{1.452269in}}%
\pgfpathlineto{\pgfqpoint{2.648429in}{1.459503in}}%
\pgfpathlineto{\pgfqpoint{2.648836in}{1.454344in}}%
\pgfpathlineto{\pgfqpoint{2.649379in}{1.459754in}}%
\pgfpathlineto{\pgfqpoint{2.650057in}{1.457421in}}%
\pgfpathlineto{\pgfqpoint{2.651684in}{1.459924in}}%
\pgfpathlineto{\pgfqpoint{2.650464in}{1.457087in}}%
\pgfpathlineto{\pgfqpoint{2.651820in}{1.459046in}}%
\pgfpathlineto{\pgfqpoint{2.653312in}{1.450571in}}%
\pgfpathlineto{\pgfqpoint{2.653719in}{1.453652in}}%
\pgfpathlineto{\pgfqpoint{2.654126in}{1.451598in}}%
\pgfpathlineto{\pgfqpoint{2.656160in}{1.446184in}}%
\pgfpathlineto{\pgfqpoint{2.654804in}{1.455090in}}%
\pgfpathlineto{\pgfqpoint{2.656296in}{1.446756in}}%
\pgfpathlineto{\pgfqpoint{2.656703in}{1.450956in}}%
\pgfpathlineto{\pgfqpoint{2.657381in}{1.441763in}}%
\pgfpathlineto{\pgfqpoint{2.657788in}{1.446511in}}%
\pgfpathlineto{\pgfqpoint{2.658059in}{1.445059in}}%
\pgfpathlineto{\pgfqpoint{2.659279in}{1.438467in}}%
\pgfpathlineto{\pgfqpoint{2.659822in}{1.440495in}}%
\pgfpathlineto{\pgfqpoint{2.659958in}{1.441017in}}%
\pgfpathlineto{\pgfqpoint{2.660229in}{1.437549in}}%
\pgfpathlineto{\pgfqpoint{2.660636in}{1.435200in}}%
\pgfpathlineto{\pgfqpoint{2.661043in}{1.441617in}}%
\pgfpathlineto{\pgfqpoint{2.661314in}{1.443943in}}%
\pgfpathlineto{\pgfqpoint{2.661856in}{1.435678in}}%
\pgfpathlineto{\pgfqpoint{2.662399in}{1.440726in}}%
\pgfpathlineto{\pgfqpoint{2.664026in}{1.431490in}}%
\pgfpathlineto{\pgfqpoint{2.663484in}{1.442167in}}%
\pgfpathlineto{\pgfqpoint{2.664298in}{1.437078in}}%
\pgfpathlineto{\pgfqpoint{2.664569in}{1.443947in}}%
\pgfpathlineto{\pgfqpoint{2.665111in}{1.435047in}}%
\pgfpathlineto{\pgfqpoint{2.665925in}{1.441723in}}%
\pgfpathlineto{\pgfqpoint{2.666061in}{1.440990in}}%
\pgfpathlineto{\pgfqpoint{2.666468in}{1.444269in}}%
\pgfpathlineto{\pgfqpoint{2.667010in}{1.443455in}}%
\pgfpathlineto{\pgfqpoint{2.668095in}{1.449455in}}%
\pgfpathlineto{\pgfqpoint{2.669723in}{1.457422in}}%
\pgfpathlineto{\pgfqpoint{2.669858in}{1.457100in}}%
\pgfpathlineto{\pgfqpoint{2.670265in}{1.455457in}}%
\pgfpathlineto{\pgfqpoint{2.670672in}{1.456668in}}%
\pgfpathlineto{\pgfqpoint{2.672028in}{1.461661in}}%
\pgfpathlineto{\pgfqpoint{2.672435in}{1.460058in}}%
\pgfpathlineto{\pgfqpoint{2.672706in}{1.461980in}}%
\pgfpathlineto{\pgfqpoint{2.674198in}{1.467382in}}%
\pgfpathlineto{\pgfqpoint{2.673656in}{1.461364in}}%
\pgfpathlineto{\pgfqpoint{2.674334in}{1.467163in}}%
\pgfpathlineto{\pgfqpoint{2.675826in}{1.461592in}}%
\pgfpathlineto{\pgfqpoint{2.675148in}{1.468655in}}%
\pgfpathlineto{\pgfqpoint{2.676097in}{1.465398in}}%
\pgfpathlineto{\pgfqpoint{2.676368in}{1.468838in}}%
\pgfpathlineto{\pgfqpoint{2.676911in}{1.464295in}}%
\pgfpathlineto{\pgfqpoint{2.677724in}{1.467499in}}%
\pgfpathlineto{\pgfqpoint{2.678131in}{1.465521in}}%
\pgfpathlineto{\pgfqpoint{2.679352in}{1.466585in}}%
\pgfpathlineto{\pgfqpoint{2.679894in}{1.471191in}}%
\pgfpathlineto{\pgfqpoint{2.681115in}{1.470223in}}%
\pgfpathlineto{\pgfqpoint{2.681793in}{1.470744in}}%
\pgfpathlineto{\pgfqpoint{2.681929in}{1.469816in}}%
\pgfpathlineto{\pgfqpoint{2.682200in}{1.466501in}}%
\pgfpathlineto{\pgfqpoint{2.682878in}{1.470493in}}%
\pgfpathlineto{\pgfqpoint{2.683421in}{1.469401in}}%
\pgfpathlineto{\pgfqpoint{2.683828in}{1.471695in}}%
\pgfpathlineto{\pgfqpoint{2.684506in}{1.467979in}}%
\pgfpathlineto{\pgfqpoint{2.684913in}{1.469356in}}%
\pgfpathlineto{\pgfqpoint{2.685455in}{1.464844in}}%
\pgfpathlineto{\pgfqpoint{2.685998in}{1.469801in}}%
\pgfpathlineto{\pgfqpoint{2.686676in}{1.466507in}}%
\pgfpathlineto{\pgfqpoint{2.688032in}{1.468372in}}%
\pgfpathlineto{\pgfqpoint{2.687625in}{1.465411in}}%
\pgfpathlineto{\pgfqpoint{2.688303in}{1.467133in}}%
\pgfpathlineto{\pgfqpoint{2.690066in}{1.461168in}}%
\pgfpathlineto{\pgfqpoint{2.690202in}{1.461918in}}%
\pgfpathlineto{\pgfqpoint{2.690609in}{1.464243in}}%
\pgfpathlineto{\pgfqpoint{2.691016in}{1.459380in}}%
\pgfpathlineto{\pgfqpoint{2.692372in}{1.457058in}}%
\pgfpathlineto{\pgfqpoint{2.691694in}{1.463323in}}%
\pgfpathlineto{\pgfqpoint{2.692508in}{1.458173in}}%
\pgfpathlineto{\pgfqpoint{2.692779in}{1.459647in}}%
\pgfpathlineto{\pgfqpoint{2.693186in}{1.457179in}}%
\pgfpathlineto{\pgfqpoint{2.693728in}{1.458562in}}%
\pgfpathlineto{\pgfqpoint{2.695356in}{1.451331in}}%
\pgfpathlineto{\pgfqpoint{2.695627in}{1.452068in}}%
\pgfpathlineto{\pgfqpoint{2.696712in}{1.455167in}}%
\pgfpathlineto{\pgfqpoint{2.696034in}{1.451144in}}%
\pgfpathlineto{\pgfqpoint{2.697254in}{1.452536in}}%
\pgfpathlineto{\pgfqpoint{2.697797in}{1.454550in}}%
\pgfpathlineto{\pgfqpoint{2.698068in}{1.452195in}}%
\pgfpathlineto{\pgfqpoint{2.698611in}{1.449169in}}%
\pgfpathlineto{\pgfqpoint{2.699424in}{1.453023in}}%
\pgfpathlineto{\pgfqpoint{2.699560in}{1.452217in}}%
\pgfpathlineto{\pgfqpoint{2.700645in}{1.447840in}}%
\pgfpathlineto{\pgfqpoint{2.700103in}{1.453605in}}%
\pgfpathlineto{\pgfqpoint{2.700916in}{1.451136in}}%
\pgfpathlineto{\pgfqpoint{2.702408in}{1.457929in}}%
\pgfpathlineto{\pgfqpoint{2.701866in}{1.450368in}}%
\pgfpathlineto{\pgfqpoint{2.702679in}{1.456312in}}%
\pgfpathlineto{\pgfqpoint{2.702815in}{1.454735in}}%
\pgfpathlineto{\pgfqpoint{2.703493in}{1.460136in}}%
\pgfpathlineto{\pgfqpoint{2.704036in}{1.456936in}}%
\pgfpathlineto{\pgfqpoint{2.706341in}{1.464330in}}%
\pgfpathlineto{\pgfqpoint{2.707019in}{1.461993in}}%
\pgfpathlineto{\pgfqpoint{2.707291in}{1.460751in}}%
\pgfpathlineto{\pgfqpoint{2.707969in}{1.465916in}}%
\pgfpathlineto{\pgfqpoint{2.709868in}{1.469783in}}%
\pgfpathlineto{\pgfqpoint{2.708511in}{1.464588in}}%
\pgfpathlineto{\pgfqpoint{2.710410in}{1.466947in}}%
\pgfpathlineto{\pgfqpoint{2.710546in}{1.466594in}}%
\pgfpathlineto{\pgfqpoint{2.710817in}{1.468977in}}%
\pgfpathlineto{\pgfqpoint{2.712038in}{1.472612in}}%
\pgfpathlineto{\pgfqpoint{2.711495in}{1.468523in}}%
\pgfpathlineto{\pgfqpoint{2.712309in}{1.469910in}}%
\pgfpathlineto{\pgfqpoint{2.712444in}{1.468903in}}%
\pgfpathlineto{\pgfqpoint{2.712987in}{1.475466in}}%
\pgfpathlineto{\pgfqpoint{2.713123in}{1.475908in}}%
\pgfpathlineto{\pgfqpoint{2.713394in}{1.472043in}}%
\pgfpathlineto{\pgfqpoint{2.713665in}{1.467861in}}%
\pgfpathlineto{\pgfqpoint{2.714208in}{1.473181in}}%
\pgfpathlineto{\pgfqpoint{2.714886in}{1.469618in}}%
\pgfpathlineto{\pgfqpoint{2.716649in}{1.474404in}}%
\pgfpathlineto{\pgfqpoint{2.717327in}{1.469578in}}%
\pgfpathlineto{\pgfqpoint{2.718412in}{1.470140in}}%
\pgfpathlineto{\pgfqpoint{2.719768in}{1.472834in}}%
\pgfpathlineto{\pgfqpoint{2.720039in}{1.470299in}}%
\pgfpathlineto{\pgfqpoint{2.720446in}{1.466790in}}%
\pgfpathlineto{\pgfqpoint{2.721803in}{1.467868in}}%
\pgfpathlineto{\pgfqpoint{2.724379in}{1.460846in}}%
\pgfpathlineto{\pgfqpoint{2.724515in}{1.461610in}}%
\pgfpathlineto{\pgfqpoint{2.724786in}{1.464674in}}%
\pgfpathlineto{\pgfqpoint{2.725464in}{1.458825in}}%
\pgfpathlineto{\pgfqpoint{2.726143in}{1.464484in}}%
\pgfpathlineto{\pgfqpoint{2.727906in}{1.456398in}}%
\pgfpathlineto{\pgfqpoint{2.728313in}{1.461917in}}%
\pgfpathlineto{\pgfqpoint{2.728448in}{1.462007in}}%
\pgfpathlineto{\pgfqpoint{2.728584in}{1.460802in}}%
\pgfpathlineto{\pgfqpoint{2.728991in}{1.454635in}}%
\pgfpathlineto{\pgfqpoint{2.730211in}{1.456393in}}%
\pgfpathlineto{\pgfqpoint{2.730618in}{1.459064in}}%
\pgfpathlineto{\pgfqpoint{2.731025in}{1.454277in}}%
\pgfpathlineto{\pgfqpoint{2.731703in}{1.457716in}}%
\pgfpathlineto{\pgfqpoint{2.732246in}{1.453531in}}%
\pgfpathlineto{\pgfqpoint{2.733059in}{1.459779in}}%
\pgfpathlineto{\pgfqpoint{2.734009in}{1.457503in}}%
\pgfpathlineto{\pgfqpoint{2.734958in}{1.461518in}}%
\pgfpathlineto{\pgfqpoint{2.735365in}{1.460306in}}%
\pgfpathlineto{\pgfqpoint{2.736179in}{1.462330in}}%
\pgfpathlineto{\pgfqpoint{2.737942in}{1.468407in}}%
\pgfpathlineto{\pgfqpoint{2.738349in}{1.465819in}}%
\pgfpathlineto{\pgfqpoint{2.738620in}{1.464160in}}%
\pgfpathlineto{\pgfqpoint{2.739163in}{1.469024in}}%
\pgfpathlineto{\pgfqpoint{2.739705in}{1.467213in}}%
\pgfpathlineto{\pgfqpoint{2.741333in}{1.476136in}}%
\pgfpathlineto{\pgfqpoint{2.741739in}{1.473281in}}%
\pgfpathlineto{\pgfqpoint{2.741875in}{1.472732in}}%
\pgfpathlineto{\pgfqpoint{2.742282in}{1.475063in}}%
\pgfpathlineto{\pgfqpoint{2.742824in}{1.474604in}}%
\pgfpathlineto{\pgfqpoint{2.744588in}{1.480033in}}%
\pgfpathlineto{\pgfqpoint{2.744859in}{1.478380in}}%
\pgfpathlineto{\pgfqpoint{2.745266in}{1.475560in}}%
\pgfpathlineto{\pgfqpoint{2.746215in}{1.478574in}}%
\pgfpathlineto{\pgfqpoint{2.746351in}{1.478100in}}%
\pgfpathlineto{\pgfqpoint{2.746758in}{1.477273in}}%
\pgfpathlineto{\pgfqpoint{2.747571in}{1.478435in}}%
\pgfpathlineto{\pgfqpoint{2.748249in}{1.482443in}}%
\pgfpathlineto{\pgfqpoint{2.749334in}{1.481215in}}%
\pgfpathlineto{\pgfqpoint{2.750284in}{1.478343in}}%
\pgfpathlineto{\pgfqpoint{2.749741in}{1.482119in}}%
\pgfpathlineto{\pgfqpoint{2.750691in}{1.480539in}}%
\pgfpathlineto{\pgfqpoint{2.750962in}{1.482298in}}%
\pgfpathlineto{\pgfqpoint{2.751504in}{1.477008in}}%
\pgfpathlineto{\pgfqpoint{2.752047in}{1.479598in}}%
\pgfpathlineto{\pgfqpoint{2.753810in}{1.472575in}}%
\pgfpathlineto{\pgfqpoint{2.754081in}{1.474601in}}%
\pgfpathlineto{\pgfqpoint{2.754353in}{1.475486in}}%
\pgfpathlineto{\pgfqpoint{2.754895in}{1.473264in}}%
\pgfpathlineto{\pgfqpoint{2.755438in}{1.474914in}}%
\pgfpathlineto{\pgfqpoint{2.757065in}{1.466140in}}%
\pgfpathlineto{\pgfqpoint{2.757608in}{1.470500in}}%
\pgfpathlineto{\pgfqpoint{2.757743in}{1.471443in}}%
\pgfpathlineto{\pgfqpoint{2.758421in}{1.466691in}}%
\pgfpathlineto{\pgfqpoint{2.758693in}{1.468291in}}%
\pgfpathlineto{\pgfqpoint{2.759235in}{1.462879in}}%
\pgfpathlineto{\pgfqpoint{2.759778in}{1.468479in}}%
\pgfpathlineto{\pgfqpoint{2.760456in}{1.464993in}}%
\pgfpathlineto{\pgfqpoint{2.761676in}{1.466968in}}%
\pgfpathlineto{\pgfqpoint{2.761134in}{1.464109in}}%
\pgfpathlineto{\pgfqpoint{2.761948in}{1.464428in}}%
\pgfpathlineto{\pgfqpoint{2.763168in}{1.460665in}}%
\pgfpathlineto{\pgfqpoint{2.762626in}{1.465558in}}%
\pgfpathlineto{\pgfqpoint{2.763575in}{1.461853in}}%
\pgfpathlineto{\pgfqpoint{2.764253in}{1.460053in}}%
\pgfpathlineto{\pgfqpoint{2.765067in}{1.463431in}}%
\pgfpathlineto{\pgfqpoint{2.765609in}{1.458013in}}%
\pgfpathlineto{\pgfqpoint{2.766966in}{1.458462in}}%
\pgfpathlineto{\pgfqpoint{2.768458in}{1.464327in}}%
\pgfpathlineto{\pgfqpoint{2.767779in}{1.457494in}}%
\pgfpathlineto{\pgfqpoint{2.768729in}{1.460201in}}%
\pgfpathlineto{\pgfqpoint{2.769949in}{1.458191in}}%
\pgfpathlineto{\pgfqpoint{2.769407in}{1.463437in}}%
\pgfpathlineto{\pgfqpoint{2.770221in}{1.460145in}}%
\pgfpathlineto{\pgfqpoint{2.770628in}{1.463490in}}%
\pgfpathlineto{\pgfqpoint{2.771713in}{1.462392in}}%
\pgfpathlineto{\pgfqpoint{2.772255in}{1.459428in}}%
\pgfpathlineto{\pgfqpoint{2.773069in}{1.463961in}}%
\pgfpathlineto{\pgfqpoint{2.773476in}{1.464768in}}%
\pgfpathlineto{\pgfqpoint{2.774018in}{1.462376in}}%
\pgfpathlineto{\pgfqpoint{2.774154in}{1.461801in}}%
\pgfpathlineto{\pgfqpoint{2.774696in}{1.466010in}}%
\pgfpathlineto{\pgfqpoint{2.775510in}{1.466568in}}%
\pgfpathlineto{\pgfqpoint{2.776866in}{1.469682in}}%
\pgfpathlineto{\pgfqpoint{2.777138in}{1.467567in}}%
\pgfpathlineto{\pgfqpoint{2.777409in}{1.465045in}}%
\pgfpathlineto{\pgfqpoint{2.778087in}{1.471763in}}%
\pgfpathlineto{\pgfqpoint{2.778629in}{1.467909in}}%
\pgfpathlineto{\pgfqpoint{2.781342in}{1.476604in}}%
\pgfpathlineto{\pgfqpoint{2.781478in}{1.476120in}}%
\pgfpathlineto{\pgfqpoint{2.782020in}{1.473879in}}%
\pgfpathlineto{\pgfqpoint{2.783241in}{1.476160in}}%
\pgfpathlineto{\pgfqpoint{2.784190in}{1.474910in}}%
\pgfpathlineto{\pgfqpoint{2.784868in}{1.479036in}}%
\pgfpathlineto{\pgfqpoint{2.785275in}{1.476315in}}%
\pgfpathlineto{\pgfqpoint{2.785953in}{1.479748in}}%
\pgfpathlineto{\pgfqpoint{2.786360in}{1.479028in}}%
\pgfpathlineto{\pgfqpoint{2.786631in}{1.479863in}}%
\pgfpathlineto{\pgfqpoint{2.787038in}{1.477587in}}%
\pgfpathlineto{\pgfqpoint{2.788259in}{1.475664in}}%
\pgfpathlineto{\pgfqpoint{2.787581in}{1.478188in}}%
\pgfpathlineto{\pgfqpoint{2.788530in}{1.478039in}}%
\pgfpathlineto{\pgfqpoint{2.788666in}{1.479375in}}%
\pgfpathlineto{\pgfqpoint{2.789208in}{1.474404in}}%
\pgfpathlineto{\pgfqpoint{2.790022in}{1.477912in}}%
\pgfpathlineto{\pgfqpoint{2.791514in}{1.472081in}}%
\pgfpathlineto{\pgfqpoint{2.791785in}{1.472827in}}%
\pgfpathlineto{\pgfqpoint{2.793819in}{1.467242in}}%
\pgfpathlineto{\pgfqpoint{2.793955in}{1.467702in}}%
\pgfpathlineto{\pgfqpoint{2.794362in}{1.470142in}}%
\pgfpathlineto{\pgfqpoint{2.794633in}{1.467354in}}%
\pgfpathlineto{\pgfqpoint{2.795989in}{1.461607in}}%
\pgfpathlineto{\pgfqpoint{2.796261in}{1.463530in}}%
\pgfpathlineto{\pgfqpoint{2.796532in}{1.466220in}}%
\pgfpathlineto{\pgfqpoint{2.797074in}{1.460781in}}%
\pgfpathlineto{\pgfqpoint{2.797888in}{1.464658in}}%
\pgfpathlineto{\pgfqpoint{2.800058in}{1.459805in}}%
\pgfpathlineto{\pgfqpoint{2.800329in}{1.461347in}}%
\pgfpathlineto{\pgfqpoint{2.801414in}{1.461862in}}%
\pgfpathlineto{\pgfqpoint{2.803042in}{1.466404in}}%
\pgfpathlineto{\pgfqpoint{2.803178in}{1.465565in}}%
\pgfpathlineto{\pgfqpoint{2.803449in}{1.463268in}}%
\pgfpathlineto{\pgfqpoint{2.804669in}{1.466188in}}%
\pgfpathlineto{\pgfqpoint{2.806297in}{1.474789in}}%
\pgfpathlineto{\pgfqpoint{2.806433in}{1.472644in}}%
\pgfpathlineto{\pgfqpoint{2.807789in}{1.471023in}}%
\pgfpathlineto{\pgfqpoint{2.807246in}{1.475262in}}%
\pgfpathlineto{\pgfqpoint{2.807924in}{1.472888in}}%
\pgfpathlineto{\pgfqpoint{2.808467in}{1.477598in}}%
\pgfpathlineto{\pgfqpoint{2.808874in}{1.472467in}}%
\pgfpathlineto{\pgfqpoint{2.809688in}{1.476926in}}%
\pgfpathlineto{\pgfqpoint{2.810366in}{1.474712in}}%
\pgfpathlineto{\pgfqpoint{2.810908in}{1.478267in}}%
\pgfpathlineto{\pgfqpoint{2.811179in}{1.478933in}}%
\pgfpathlineto{\pgfqpoint{2.811586in}{1.476895in}}%
\pgfpathlineto{\pgfqpoint{2.812264in}{1.477415in}}%
\pgfpathlineto{\pgfqpoint{2.813078in}{1.479780in}}%
\pgfpathlineto{\pgfqpoint{2.814028in}{1.477884in}}%
\pgfpathlineto{\pgfqpoint{2.814299in}{1.477440in}}%
\pgfpathlineto{\pgfqpoint{2.814706in}{1.479896in}}%
\pgfpathlineto{\pgfqpoint{2.816062in}{1.481609in}}%
\pgfpathlineto{\pgfqpoint{2.815384in}{1.478074in}}%
\pgfpathlineto{\pgfqpoint{2.816198in}{1.480785in}}%
\pgfpathlineto{\pgfqpoint{2.817418in}{1.478102in}}%
\pgfpathlineto{\pgfqpoint{2.816876in}{1.482804in}}%
\pgfpathlineto{\pgfqpoint{2.817689in}{1.480558in}}%
\pgfpathlineto{\pgfqpoint{2.817961in}{1.483058in}}%
\pgfpathlineto{\pgfqpoint{2.818639in}{1.476691in}}%
\pgfpathlineto{\pgfqpoint{2.819181in}{1.481059in}}%
\pgfpathlineto{\pgfqpoint{2.820809in}{1.476311in}}%
\pgfpathlineto{\pgfqpoint{2.820131in}{1.481476in}}%
\pgfpathlineto{\pgfqpoint{2.820944in}{1.476614in}}%
\pgfpathlineto{\pgfqpoint{2.821216in}{1.478767in}}%
\pgfpathlineto{\pgfqpoint{2.821894in}{1.473728in}}%
\pgfpathlineto{\pgfqpoint{2.822572in}{1.477362in}}%
\pgfpathlineto{\pgfqpoint{2.826098in}{1.467635in}}%
\pgfpathlineto{\pgfqpoint{2.826369in}{1.468548in}}%
\pgfpathlineto{\pgfqpoint{2.826505in}{1.469935in}}%
\pgfpathlineto{\pgfqpoint{2.827183in}{1.462776in}}%
\pgfpathlineto{\pgfqpoint{2.827590in}{1.465337in}}%
\pgfpathlineto{\pgfqpoint{2.828675in}{1.467171in}}%
\pgfpathlineto{\pgfqpoint{2.827997in}{1.463979in}}%
\pgfpathlineto{\pgfqpoint{2.828811in}{1.465363in}}%
\pgfpathlineto{\pgfqpoint{2.830303in}{1.460773in}}%
\pgfpathlineto{\pgfqpoint{2.830574in}{1.460969in}}%
\pgfpathlineto{\pgfqpoint{2.831116in}{1.465840in}}%
\pgfpathlineto{\pgfqpoint{2.832201in}{1.464986in}}%
\pgfpathlineto{\pgfqpoint{2.832608in}{1.458742in}}%
\pgfpathlineto{\pgfqpoint{2.833151in}{1.466270in}}%
\pgfpathlineto{\pgfqpoint{2.833964in}{1.461486in}}%
\pgfpathlineto{\pgfqpoint{2.835456in}{1.470503in}}%
\pgfpathlineto{\pgfqpoint{2.835863in}{1.467412in}}%
\pgfpathlineto{\pgfqpoint{2.835999in}{1.466531in}}%
\pgfpathlineto{\pgfqpoint{2.836541in}{1.469156in}}%
\pgfpathlineto{\pgfqpoint{2.837219in}{1.468128in}}%
\pgfpathlineto{\pgfqpoint{2.838304in}{1.468741in}}%
\pgfpathlineto{\pgfqpoint{2.838033in}{1.466639in}}%
\pgfpathlineto{\pgfqpoint{2.838440in}{1.468144in}}%
\pgfpathlineto{\pgfqpoint{2.838847in}{1.465674in}}%
\pgfpathlineto{\pgfqpoint{2.839525in}{1.471037in}}%
\pgfpathlineto{\pgfqpoint{2.840068in}{1.469948in}}%
\pgfpathlineto{\pgfqpoint{2.840339in}{1.471483in}}%
\pgfpathlineto{\pgfqpoint{2.840881in}{1.470993in}}%
\pgfpathlineto{\pgfqpoint{2.841966in}{1.472631in}}%
\pgfpathlineto{\pgfqpoint{2.843323in}{1.470216in}}%
\pgfpathlineto{\pgfqpoint{2.842916in}{1.473601in}}%
\pgfpathlineto{\pgfqpoint{2.843594in}{1.471559in}}%
\pgfpathlineto{\pgfqpoint{2.845086in}{1.479816in}}%
\pgfpathlineto{\pgfqpoint{2.845357in}{1.477834in}}%
\pgfpathlineto{\pgfqpoint{2.845628in}{1.474086in}}%
\pgfpathlineto{\pgfqpoint{2.846306in}{1.482015in}}%
\pgfpathlineto{\pgfqpoint{2.846849in}{1.476825in}}%
\pgfpathlineto{\pgfqpoint{2.847391in}{1.482501in}}%
\pgfpathlineto{\pgfqpoint{2.848883in}{1.481188in}}%
\pgfpathlineto{\pgfqpoint{2.851596in}{1.486247in}}%
\pgfpathlineto{\pgfqpoint{2.852003in}{1.484221in}}%
\pgfpathlineto{\pgfqpoint{2.852138in}{1.483601in}}%
\pgfpathlineto{\pgfqpoint{2.852816in}{1.486676in}}%
\pgfpathlineto{\pgfqpoint{2.853223in}{1.485761in}}%
\pgfpathlineto{\pgfqpoint{2.853494in}{1.486019in}}%
\pgfpathlineto{\pgfqpoint{2.853766in}{1.485306in}}%
\pgfpathlineto{\pgfqpoint{2.855258in}{1.481428in}}%
\pgfpathlineto{\pgfqpoint{2.854715in}{1.485719in}}%
\pgfpathlineto{\pgfqpoint{2.855529in}{1.483041in}}%
\pgfpathlineto{\pgfqpoint{2.855664in}{1.484183in}}%
\pgfpathlineto{\pgfqpoint{2.856207in}{1.479903in}}%
\pgfpathlineto{\pgfqpoint{2.856749in}{1.481134in}}%
\pgfpathlineto{\pgfqpoint{2.858648in}{1.475357in}}%
\pgfpathlineto{\pgfqpoint{2.859191in}{1.479179in}}%
\pgfpathlineto{\pgfqpoint{2.859326in}{1.480228in}}%
\pgfpathlineto{\pgfqpoint{2.860004in}{1.474355in}}%
\pgfpathlineto{\pgfqpoint{2.860411in}{1.477358in}}%
\pgfpathlineto{\pgfqpoint{2.860547in}{1.477369in}}%
\pgfpathlineto{\pgfqpoint{2.862039in}{1.468680in}}%
\pgfpathlineto{\pgfqpoint{2.862310in}{1.471665in}}%
\pgfpathlineto{\pgfqpoint{2.862581in}{1.473967in}}%
\pgfpathlineto{\pgfqpoint{2.863259in}{1.469006in}}%
\pgfpathlineto{\pgfqpoint{2.863531in}{1.470101in}}%
\pgfpathlineto{\pgfqpoint{2.863938in}{1.466472in}}%
\pgfpathlineto{\pgfqpoint{2.864480in}{1.470755in}}%
\pgfpathlineto{\pgfqpoint{2.864616in}{1.472308in}}%
\pgfpathlineto{\pgfqpoint{2.865701in}{1.467858in}}%
\pgfpathlineto{\pgfqpoint{2.867057in}{1.464238in}}%
\pgfpathlineto{\pgfqpoint{2.866514in}{1.469170in}}%
\pgfpathlineto{\pgfqpoint{2.867328in}{1.465849in}}%
\pgfpathlineto{\pgfqpoint{2.868820in}{1.468099in}}%
\pgfpathlineto{\pgfqpoint{2.868278in}{1.463706in}}%
\pgfpathlineto{\pgfqpoint{2.868956in}{1.467873in}}%
\pgfpathlineto{\pgfqpoint{2.870448in}{1.463310in}}%
\pgfpathlineto{\pgfqpoint{2.869905in}{1.470097in}}%
\pgfpathlineto{\pgfqpoint{2.870719in}{1.466424in}}%
\pgfpathlineto{\pgfqpoint{2.872211in}{1.474540in}}%
\pgfpathlineto{\pgfqpoint{2.872482in}{1.471440in}}%
\pgfpathlineto{\pgfqpoint{2.872753in}{1.469177in}}%
\pgfpathlineto{\pgfqpoint{2.873431in}{1.474799in}}%
\pgfpathlineto{\pgfqpoint{2.873838in}{1.472818in}}%
\pgfpathlineto{\pgfqpoint{2.874109in}{1.473940in}}%
\pgfpathlineto{\pgfqpoint{2.874923in}{1.473363in}}%
\pgfpathlineto{\pgfqpoint{2.875737in}{1.475526in}}%
\pgfpathlineto{\pgfqpoint{2.875873in}{1.474940in}}%
\pgfpathlineto{\pgfqpoint{2.876822in}{1.477427in}}%
\pgfpathlineto{\pgfqpoint{2.878043in}{1.480736in}}%
\pgfpathlineto{\pgfqpoint{2.877364in}{1.476660in}}%
\pgfpathlineto{\pgfqpoint{2.878992in}{1.479745in}}%
\pgfpathlineto{\pgfqpoint{2.879670in}{1.483413in}}%
\pgfpathlineto{\pgfqpoint{2.880755in}{1.484335in}}%
\pgfpathlineto{\pgfqpoint{2.881026in}{1.483055in}}%
\pgfpathlineto{\pgfqpoint{2.882383in}{1.479592in}}%
\pgfpathlineto{\pgfqpoint{2.881840in}{1.484842in}}%
\pgfpathlineto{\pgfqpoint{2.882518in}{1.481347in}}%
\pgfpathlineto{\pgfqpoint{2.882925in}{1.486730in}}%
\pgfpathlineto{\pgfqpoint{2.883468in}{1.478524in}}%
\pgfpathlineto{\pgfqpoint{2.884010in}{1.483758in}}%
\pgfpathlineto{\pgfqpoint{2.885773in}{1.477330in}}%
\pgfpathlineto{\pgfqpoint{2.886316in}{1.479873in}}%
\pgfpathlineto{\pgfqpoint{2.886451in}{1.480347in}}%
\pgfpathlineto{\pgfqpoint{2.887129in}{1.477558in}}%
\pgfpathlineto{\pgfqpoint{2.887536in}{1.479130in}}%
\pgfpathlineto{\pgfqpoint{2.889028in}{1.476559in}}%
\pgfpathlineto{\pgfqpoint{2.889299in}{1.476951in}}%
\pgfpathlineto{\pgfqpoint{2.889571in}{1.479934in}}%
\pgfpathlineto{\pgfqpoint{2.890249in}{1.476150in}}%
\pgfpathlineto{\pgfqpoint{2.890791in}{1.478625in}}%
\pgfpathlineto{\pgfqpoint{2.891063in}{1.477820in}}%
\pgfpathlineto{\pgfqpoint{2.891605in}{1.480942in}}%
\pgfpathlineto{\pgfqpoint{2.892283in}{1.478986in}}%
\pgfpathlineto{\pgfqpoint{2.892961in}{1.476843in}}%
\pgfpathlineto{\pgfqpoint{2.893639in}{1.479066in}}%
\pgfpathlineto{\pgfqpoint{2.895538in}{1.470612in}}%
\pgfpathlineto{\pgfqpoint{2.895809in}{1.473707in}}%
\pgfpathlineto{\pgfqpoint{2.895945in}{1.475178in}}%
\pgfpathlineto{\pgfqpoint{2.896488in}{1.468773in}}%
\pgfpathlineto{\pgfqpoint{2.897301in}{1.473393in}}%
\pgfpathlineto{\pgfqpoint{2.898658in}{1.466821in}}%
\pgfpathlineto{\pgfqpoint{2.898115in}{1.473992in}}%
\pgfpathlineto{\pgfqpoint{2.899064in}{1.469805in}}%
\pgfpathlineto{\pgfqpoint{2.900285in}{1.473478in}}%
\pgfpathlineto{\pgfqpoint{2.899878in}{1.468099in}}%
\pgfpathlineto{\pgfqpoint{2.900556in}{1.470845in}}%
\pgfpathlineto{\pgfqpoint{2.901913in}{1.467915in}}%
\pgfpathlineto{\pgfqpoint{2.901370in}{1.472411in}}%
\pgfpathlineto{\pgfqpoint{2.902184in}{1.469514in}}%
\pgfpathlineto{\pgfqpoint{2.903269in}{1.470913in}}%
\pgfpathlineto{\pgfqpoint{2.903540in}{1.468971in}}%
\pgfpathlineto{\pgfqpoint{2.903676in}{1.468685in}}%
\pgfpathlineto{\pgfqpoint{2.904218in}{1.471095in}}%
\pgfpathlineto{\pgfqpoint{2.904489in}{1.470212in}}%
\pgfpathlineto{\pgfqpoint{2.905032in}{1.470957in}}%
\pgfpathlineto{\pgfqpoint{2.905032in}{1.470957in}}%
\pgfusepath{stroke}%
\end{pgfscope}%
\begin{pgfscope}%
\pgfpathrectangle{\pgfqpoint{0.735032in}{0.526079in}}{\pgfqpoint{2.170000in}{1.661000in}} %
\pgfusepath{clip}%
\pgfsetrectcap%
\pgfsetroundjoin%
\pgfsetlinewidth{1.003750pt}%
\definecolor{currentstroke}{rgb}{0.501961,0.000000,0.501961}%
\pgfsetstrokecolor{currentstroke}%
\pgfsetdash{}{0pt}%
\pgfpathmoveto{\pgfqpoint{0.735167in}{0.512191in}}%
\pgfpathlineto{\pgfqpoint{0.736659in}{1.336605in}}%
\pgfpathlineto{\pgfqpoint{0.737202in}{1.330713in}}%
\pgfpathlineto{\pgfqpoint{0.741135in}{1.094340in}}%
\pgfpathlineto{\pgfqpoint{0.741406in}{0.813932in}}%
\pgfpathlineto{\pgfqpoint{0.742084in}{1.130783in}}%
\pgfpathlineto{\pgfqpoint{0.742491in}{1.106428in}}%
\pgfpathlineto{\pgfqpoint{0.744390in}{1.259229in}}%
\pgfpathlineto{\pgfqpoint{0.747781in}{1.172520in}}%
\pgfpathlineto{\pgfqpoint{0.748323in}{1.209595in}}%
\pgfpathlineto{\pgfqpoint{0.748594in}{1.215719in}}%
\pgfpathlineto{\pgfqpoint{0.749137in}{1.202303in}}%
\pgfpathlineto{\pgfqpoint{0.749815in}{0.991940in}}%
\pgfpathlineto{\pgfqpoint{0.750358in}{1.226757in}}%
\pgfpathlineto{\pgfqpoint{0.751578in}{1.272483in}}%
\pgfpathlineto{\pgfqpoint{0.751985in}{1.253568in}}%
\pgfpathlineto{\pgfqpoint{0.752392in}{1.163920in}}%
\pgfpathlineto{\pgfqpoint{0.752663in}{1.025941in}}%
\pgfpathlineto{\pgfqpoint{0.753748in}{1.262211in}}%
\pgfpathlineto{\pgfqpoint{0.754833in}{1.269047in}}%
\pgfpathlineto{\pgfqpoint{0.754291in}{1.257904in}}%
\pgfpathlineto{\pgfqpoint{0.755104in}{1.261111in}}%
\pgfpathlineto{\pgfqpoint{0.755783in}{1.188775in}}%
\pgfpathlineto{\pgfqpoint{0.757546in}{0.980256in}}%
\pgfpathlineto{\pgfqpoint{0.758902in}{1.201723in}}%
\pgfpathlineto{\pgfqpoint{0.759173in}{1.193949in}}%
\pgfpathlineto{\pgfqpoint{0.759987in}{1.120733in}}%
\pgfpathlineto{\pgfqpoint{0.759444in}{1.198558in}}%
\pgfpathlineto{\pgfqpoint{0.760801in}{1.184599in}}%
\pgfpathlineto{\pgfqpoint{0.760936in}{1.187544in}}%
\pgfpathlineto{\pgfqpoint{0.761072in}{1.176077in}}%
\pgfpathlineto{\pgfqpoint{0.761208in}{1.154558in}}%
\pgfpathlineto{\pgfqpoint{0.762293in}{1.196306in}}%
\pgfpathlineto{\pgfqpoint{0.762428in}{1.189737in}}%
\pgfpathlineto{\pgfqpoint{0.763649in}{1.245563in}}%
\pgfpathlineto{\pgfqpoint{0.762835in}{1.072812in}}%
\pgfpathlineto{\pgfqpoint{0.764056in}{1.214006in}}%
\pgfpathlineto{\pgfqpoint{0.765005in}{1.014686in}}%
\pgfpathlineto{\pgfqpoint{0.765548in}{1.222409in}}%
\pgfpathlineto{\pgfqpoint{0.765819in}{1.236371in}}%
\pgfpathlineto{\pgfqpoint{0.766497in}{1.292043in}}%
\pgfpathlineto{\pgfqpoint{0.767582in}{1.267414in}}%
\pgfpathlineto{\pgfqpoint{0.767989in}{1.141512in}}%
\pgfpathlineto{\pgfqpoint{0.768667in}{1.269775in}}%
\pgfpathlineto{\pgfqpoint{0.769074in}{1.258193in}}%
\pgfpathlineto{\pgfqpoint{0.770159in}{1.295810in}}%
\pgfpathlineto{\pgfqpoint{0.769481in}{1.248999in}}%
\pgfpathlineto{\pgfqpoint{0.770837in}{1.276930in}}%
\pgfpathlineto{\pgfqpoint{0.771922in}{1.189667in}}%
\pgfpathlineto{\pgfqpoint{0.772736in}{0.933063in}}%
\pgfpathlineto{\pgfqpoint{0.773414in}{1.200210in}}%
\pgfpathlineto{\pgfqpoint{0.773549in}{1.064167in}}%
\pgfpathlineto{\pgfqpoint{0.773685in}{1.001516in}}%
\pgfpathlineto{\pgfqpoint{0.774634in}{1.250002in}}%
\pgfpathlineto{\pgfqpoint{0.775991in}{1.307282in}}%
\pgfpathlineto{\pgfqpoint{0.776533in}{1.284407in}}%
\pgfpathlineto{\pgfqpoint{0.777618in}{0.906976in}}%
\pgfpathlineto{\pgfqpoint{0.778703in}{0.969238in}}%
\pgfpathlineto{\pgfqpoint{0.780059in}{1.214394in}}%
\pgfpathlineto{\pgfqpoint{0.780331in}{1.198518in}}%
\pgfpathlineto{\pgfqpoint{0.780738in}{1.117486in}}%
\pgfpathlineto{\pgfqpoint{0.782908in}{1.290178in}}%
\pgfpathlineto{\pgfqpoint{0.785213in}{0.969834in}}%
\pgfpathlineto{\pgfqpoint{0.785349in}{1.199741in}}%
\pgfpathlineto{\pgfqpoint{0.786434in}{1.105253in}}%
\pgfpathlineto{\pgfqpoint{0.786163in}{1.208261in}}%
\pgfpathlineto{\pgfqpoint{0.786841in}{1.173184in}}%
\pgfpathlineto{\pgfqpoint{0.786976in}{0.972447in}}%
\pgfpathlineto{\pgfqpoint{0.787926in}{1.173443in}}%
\pgfpathlineto{\pgfqpoint{0.788333in}{1.165519in}}%
\pgfpathlineto{\pgfqpoint{0.788468in}{1.170617in}}%
\pgfpathlineto{\pgfqpoint{0.788739in}{1.131133in}}%
\pgfpathlineto{\pgfqpoint{0.788875in}{1.098059in}}%
\pgfpathlineto{\pgfqpoint{0.789553in}{1.252042in}}%
\pgfpathlineto{\pgfqpoint{0.789689in}{1.215832in}}%
\pgfpathlineto{\pgfqpoint{0.790638in}{1.266232in}}%
\pgfpathlineto{\pgfqpoint{0.790909in}{1.191007in}}%
\pgfpathlineto{\pgfqpoint{0.791181in}{1.220907in}}%
\pgfpathlineto{\pgfqpoint{0.793351in}{1.029972in}}%
\pgfpathlineto{\pgfqpoint{0.791859in}{1.239215in}}%
\pgfpathlineto{\pgfqpoint{0.793622in}{1.145053in}}%
\pgfpathlineto{\pgfqpoint{0.794029in}{1.184726in}}%
\pgfpathlineto{\pgfqpoint{0.794164in}{1.119459in}}%
\pgfpathlineto{\pgfqpoint{0.794300in}{1.134394in}}%
\pgfpathlineto{\pgfqpoint{0.794436in}{0.964135in}}%
\pgfpathlineto{\pgfqpoint{0.795656in}{1.283739in}}%
\pgfpathlineto{\pgfqpoint{0.796877in}{1.341605in}}%
\pgfpathlineto{\pgfqpoint{0.796063in}{1.282430in}}%
\pgfpathlineto{\pgfqpoint{0.797691in}{1.329476in}}%
\pgfpathlineto{\pgfqpoint{0.797826in}{1.331670in}}%
\pgfpathlineto{\pgfqpoint{0.798369in}{1.318054in}}%
\pgfpathlineto{\pgfqpoint{0.799454in}{1.252304in}}%
\pgfpathlineto{\pgfqpoint{0.800268in}{1.289365in}}%
\pgfpathlineto{\pgfqpoint{0.800403in}{1.312174in}}%
\pgfpathlineto{\pgfqpoint{0.801081in}{1.216331in}}%
\pgfpathlineto{\pgfqpoint{0.801217in}{1.237982in}}%
\pgfpathlineto{\pgfqpoint{0.801624in}{1.261952in}}%
\pgfpathlineto{\pgfqpoint{0.802844in}{1.097548in}}%
\pgfpathlineto{\pgfqpoint{0.804065in}{1.023323in}}%
\pgfpathlineto{\pgfqpoint{0.804608in}{1.256441in}}%
\pgfpathlineto{\pgfqpoint{0.805828in}{1.292175in}}%
\pgfpathlineto{\pgfqpoint{0.805964in}{1.272353in}}%
\pgfpathlineto{\pgfqpoint{0.806235in}{1.047972in}}%
\pgfpathlineto{\pgfqpoint{0.807320in}{1.303481in}}%
\pgfpathlineto{\pgfqpoint{0.808269in}{1.318544in}}%
\pgfpathlineto{\pgfqpoint{0.808676in}{1.297617in}}%
\pgfpathlineto{\pgfqpoint{0.809083in}{1.285964in}}%
\pgfpathlineto{\pgfqpoint{0.809761in}{1.310536in}}%
\pgfpathlineto{\pgfqpoint{0.810439in}{1.334008in}}%
\pgfpathlineto{\pgfqpoint{0.811118in}{1.312102in}}%
\pgfpathlineto{\pgfqpoint{0.812067in}{1.044090in}}%
\pgfpathlineto{\pgfqpoint{0.812745in}{1.290369in}}%
\pgfpathlineto{\pgfqpoint{0.813016in}{1.315891in}}%
\pgfpathlineto{\pgfqpoint{0.813966in}{1.238131in}}%
\pgfpathlineto{\pgfqpoint{0.814373in}{1.155486in}}%
\pgfpathlineto{\pgfqpoint{0.814508in}{1.095203in}}%
\pgfpathlineto{\pgfqpoint{0.815458in}{1.297489in}}%
\pgfpathlineto{\pgfqpoint{0.816407in}{1.320335in}}%
\pgfpathlineto{\pgfqpoint{0.815729in}{1.294638in}}%
\pgfpathlineto{\pgfqpoint{0.817085in}{1.300235in}}%
\pgfpathlineto{\pgfqpoint{0.817221in}{1.301669in}}%
\pgfpathlineto{\pgfqpoint{0.818170in}{1.159412in}}%
\pgfpathlineto{\pgfqpoint{0.818306in}{0.890686in}}%
\pgfpathlineto{\pgfqpoint{0.818848in}{1.229896in}}%
\pgfpathlineto{\pgfqpoint{0.819662in}{1.210696in}}%
\pgfpathlineto{\pgfqpoint{0.819933in}{0.943606in}}%
\pgfpathlineto{\pgfqpoint{0.820883in}{1.275208in}}%
\pgfpathlineto{\pgfqpoint{0.821289in}{1.290705in}}%
\pgfpathlineto{\pgfqpoint{0.821696in}{1.230147in}}%
\pgfpathlineto{\pgfqpoint{0.821832in}{1.252426in}}%
\pgfpathlineto{\pgfqpoint{0.822374in}{1.146050in}}%
\pgfpathlineto{\pgfqpoint{0.823053in}{1.266997in}}%
\pgfpathlineto{\pgfqpoint{0.823188in}{1.266378in}}%
\pgfpathlineto{\pgfqpoint{0.823324in}{1.283164in}}%
\pgfpathlineto{\pgfqpoint{0.824002in}{1.190046in}}%
\pgfpathlineto{\pgfqpoint{0.824138in}{1.211784in}}%
\pgfpathlineto{\pgfqpoint{0.824544in}{0.994842in}}%
\pgfpathlineto{\pgfqpoint{0.825494in}{1.220491in}}%
\pgfpathlineto{\pgfqpoint{0.826850in}{1.307193in}}%
\pgfpathlineto{\pgfqpoint{0.827393in}{1.268473in}}%
\pgfpathlineto{\pgfqpoint{0.827799in}{1.148290in}}%
\pgfpathlineto{\pgfqpoint{0.828613in}{1.316176in}}%
\pgfpathlineto{\pgfqpoint{0.832275in}{1.242290in}}%
\pgfpathlineto{\pgfqpoint{0.832682in}{1.248981in}}%
\pgfpathlineto{\pgfqpoint{0.833496in}{1.286957in}}%
\pgfpathlineto{\pgfqpoint{0.833089in}{1.246596in}}%
\pgfpathlineto{\pgfqpoint{0.834309in}{1.258148in}}%
\pgfpathlineto{\pgfqpoint{0.834852in}{1.168009in}}%
\pgfpathlineto{\pgfqpoint{0.834988in}{1.068729in}}%
\pgfpathlineto{\pgfqpoint{0.835666in}{1.232410in}}%
\pgfpathlineto{\pgfqpoint{0.836344in}{1.196964in}}%
\pgfpathlineto{\pgfqpoint{0.839192in}{1.331341in}}%
\pgfpathlineto{\pgfqpoint{0.839734in}{1.321040in}}%
\pgfpathlineto{\pgfqpoint{0.841226in}{1.253243in}}%
\pgfpathlineto{\pgfqpoint{0.841498in}{1.268469in}}%
\pgfpathlineto{\pgfqpoint{0.843532in}{1.334793in}}%
\pgfpathlineto{\pgfqpoint{0.844210in}{1.331638in}}%
\pgfpathlineto{\pgfqpoint{0.843939in}{1.341824in}}%
\pgfpathlineto{\pgfqpoint{0.844346in}{1.339060in}}%
\pgfpathlineto{\pgfqpoint{0.846380in}{1.380727in}}%
\pgfpathlineto{\pgfqpoint{0.846516in}{1.379912in}}%
\pgfpathlineto{\pgfqpoint{0.847058in}{1.377218in}}%
\pgfpathlineto{\pgfqpoint{0.847329in}{1.379423in}}%
\pgfpathlineto{\pgfqpoint{0.848279in}{1.385261in}}%
\pgfpathlineto{\pgfqpoint{0.848550in}{1.382593in}}%
\pgfpathlineto{\pgfqpoint{0.851669in}{1.085174in}}%
\pgfpathlineto{\pgfqpoint{0.851805in}{0.944125in}}%
\pgfpathlineto{\pgfqpoint{0.852890in}{1.303685in}}%
\pgfpathlineto{\pgfqpoint{0.853704in}{1.334254in}}%
\pgfpathlineto{\pgfqpoint{0.854382in}{1.313923in}}%
\pgfpathlineto{\pgfqpoint{0.854653in}{1.294160in}}%
\pgfpathlineto{\pgfqpoint{0.855603in}{1.323495in}}%
\pgfpathlineto{\pgfqpoint{0.855738in}{1.317487in}}%
\pgfpathlineto{\pgfqpoint{0.857094in}{1.360530in}}%
\pgfpathlineto{\pgfqpoint{0.857501in}{1.348191in}}%
\pgfpathlineto{\pgfqpoint{0.858993in}{1.181122in}}%
\pgfpathlineto{\pgfqpoint{0.859129in}{1.051300in}}%
\pgfpathlineto{\pgfqpoint{0.860078in}{1.287585in}}%
\pgfpathlineto{\pgfqpoint{0.860485in}{1.229281in}}%
\pgfpathlineto{\pgfqpoint{0.861434in}{1.292639in}}%
\pgfpathlineto{\pgfqpoint{0.860756in}{1.225349in}}%
\pgfpathlineto{\pgfqpoint{0.862248in}{1.268852in}}%
\pgfpathlineto{\pgfqpoint{0.862519in}{1.273647in}}%
\pgfpathlineto{\pgfqpoint{0.863740in}{1.150215in}}%
\pgfpathlineto{\pgfqpoint{0.864011in}{1.001425in}}%
\pgfpathlineto{\pgfqpoint{0.864961in}{1.295284in}}%
\pgfpathlineto{\pgfqpoint{0.865232in}{1.305481in}}%
\pgfpathlineto{\pgfqpoint{0.865910in}{1.253055in}}%
\pgfpathlineto{\pgfqpoint{0.866181in}{1.118040in}}%
\pgfpathlineto{\pgfqpoint{0.866859in}{1.294576in}}%
\pgfpathlineto{\pgfqpoint{0.867266in}{1.286670in}}%
\pgfpathlineto{\pgfqpoint{0.867944in}{1.199348in}}%
\pgfpathlineto{\pgfqpoint{0.868080in}{1.021800in}}%
\pgfpathlineto{\pgfqpoint{0.869165in}{1.262058in}}%
\pgfpathlineto{\pgfqpoint{0.869436in}{1.259597in}}%
\pgfpathlineto{\pgfqpoint{0.870657in}{1.073945in}}%
\pgfpathlineto{\pgfqpoint{0.871064in}{1.200102in}}%
\pgfpathlineto{\pgfqpoint{0.872284in}{1.343844in}}%
\pgfpathlineto{\pgfqpoint{0.872963in}{1.335981in}}%
\pgfpathlineto{\pgfqpoint{0.874183in}{1.240438in}}%
\pgfpathlineto{\pgfqpoint{0.874997in}{1.297864in}}%
\pgfpathlineto{\pgfqpoint{0.875811in}{1.346590in}}%
\pgfpathlineto{\pgfqpoint{0.876760in}{1.335092in}}%
\pgfpathlineto{\pgfqpoint{0.878930in}{1.236152in}}%
\pgfpathlineto{\pgfqpoint{0.879337in}{1.090010in}}%
\pgfpathlineto{\pgfqpoint{0.880286in}{1.259225in}}%
\pgfpathlineto{\pgfqpoint{0.880422in}{1.241808in}}%
\pgfpathlineto{\pgfqpoint{0.882321in}{1.329487in}}%
\pgfpathlineto{\pgfqpoint{0.882728in}{1.326476in}}%
\pgfpathlineto{\pgfqpoint{0.886389in}{1.178106in}}%
\pgfpathlineto{\pgfqpoint{0.886525in}{1.013461in}}%
\pgfpathlineto{\pgfqpoint{0.886932in}{1.220286in}}%
\pgfpathlineto{\pgfqpoint{0.887746in}{1.162866in}}%
\pgfpathlineto{\pgfqpoint{0.889509in}{1.350846in}}%
\pgfpathlineto{\pgfqpoint{0.889916in}{1.338621in}}%
\pgfpathlineto{\pgfqpoint{0.891001in}{1.275231in}}%
\pgfpathlineto{\pgfqpoint{0.891814in}{1.316964in}}%
\pgfpathlineto{\pgfqpoint{0.893171in}{1.338829in}}%
\pgfpathlineto{\pgfqpoint{0.893578in}{1.337750in}}%
\pgfpathlineto{\pgfqpoint{0.895612in}{1.166463in}}%
\pgfpathlineto{\pgfqpoint{0.895748in}{1.021226in}}%
\pgfpathlineto{\pgfqpoint{0.896154in}{1.233337in}}%
\pgfpathlineto{\pgfqpoint{0.897104in}{1.217496in}}%
\pgfpathlineto{\pgfqpoint{0.897239in}{1.195762in}}%
\pgfpathlineto{\pgfqpoint{0.898324in}{1.247507in}}%
\pgfpathlineto{\pgfqpoint{0.898460in}{1.244508in}}%
\pgfpathlineto{\pgfqpoint{0.899138in}{1.076628in}}%
\pgfpathlineto{\pgfqpoint{0.899545in}{1.282578in}}%
\pgfpathlineto{\pgfqpoint{0.900766in}{1.367840in}}%
\pgfpathlineto{\pgfqpoint{0.901851in}{1.356572in}}%
\pgfpathlineto{\pgfqpoint{0.903071in}{1.246842in}}%
\pgfpathlineto{\pgfqpoint{0.903343in}{1.123113in}}%
\pgfpathlineto{\pgfqpoint{0.904021in}{1.280065in}}%
\pgfpathlineto{\pgfqpoint{0.904563in}{1.260219in}}%
\pgfpathlineto{\pgfqpoint{0.905241in}{1.155943in}}%
\pgfpathlineto{\pgfqpoint{0.905513in}{0.961592in}}%
\pgfpathlineto{\pgfqpoint{0.906326in}{1.269703in}}%
\pgfpathlineto{\pgfqpoint{0.906598in}{1.263366in}}%
\pgfpathlineto{\pgfqpoint{0.907004in}{1.268703in}}%
\pgfpathlineto{\pgfqpoint{0.907140in}{1.254483in}}%
\pgfpathlineto{\pgfqpoint{0.907547in}{1.256333in}}%
\pgfpathlineto{\pgfqpoint{0.908225in}{1.155834in}}%
\pgfpathlineto{\pgfqpoint{0.908361in}{1.076331in}}%
\pgfpathlineto{\pgfqpoint{0.909310in}{1.346163in}}%
\pgfpathlineto{\pgfqpoint{0.909446in}{1.343865in}}%
\pgfpathlineto{\pgfqpoint{0.909581in}{1.346698in}}%
\pgfpathlineto{\pgfqpoint{0.909853in}{1.326971in}}%
\pgfpathlineto{\pgfqpoint{0.910124in}{1.330846in}}%
\pgfpathlineto{\pgfqpoint{0.911209in}{1.236301in}}%
\pgfpathlineto{\pgfqpoint{0.912158in}{1.296156in}}%
\pgfpathlineto{\pgfqpoint{0.912972in}{0.996479in}}%
\pgfpathlineto{\pgfqpoint{0.914599in}{1.372033in}}%
\pgfpathlineto{\pgfqpoint{0.914735in}{1.367536in}}%
\pgfpathlineto{\pgfqpoint{0.915006in}{1.355458in}}%
\pgfpathlineto{\pgfqpoint{0.916363in}{1.241551in}}%
\pgfpathlineto{\pgfqpoint{0.917448in}{1.295690in}}%
\pgfpathlineto{\pgfqpoint{0.920024in}{1.348796in}}%
\pgfpathlineto{\pgfqpoint{0.920296in}{1.342879in}}%
\pgfpathlineto{\pgfqpoint{0.920431in}{1.342030in}}%
\pgfpathlineto{\pgfqpoint{0.920567in}{1.350674in}}%
\pgfpathlineto{\pgfqpoint{0.920838in}{1.347296in}}%
\pgfpathlineto{\pgfqpoint{0.920974in}{1.349561in}}%
\pgfpathlineto{\pgfqpoint{0.921516in}{1.336609in}}%
\pgfpathlineto{\pgfqpoint{0.923551in}{1.261926in}}%
\pgfpathlineto{\pgfqpoint{0.924093in}{1.268596in}}%
\pgfpathlineto{\pgfqpoint{0.924364in}{1.275426in}}%
\pgfpathlineto{\pgfqpoint{0.925992in}{1.358083in}}%
\pgfpathlineto{\pgfqpoint{0.926263in}{1.348805in}}%
\pgfpathlineto{\pgfqpoint{0.928840in}{1.083708in}}%
\pgfpathlineto{\pgfqpoint{0.928976in}{1.194655in}}%
\pgfpathlineto{\pgfqpoint{0.929518in}{1.092504in}}%
\pgfpathlineto{\pgfqpoint{0.931146in}{1.333246in}}%
\pgfpathlineto{\pgfqpoint{0.931281in}{1.335801in}}%
\pgfpathlineto{\pgfqpoint{0.931553in}{1.317232in}}%
\pgfpathlineto{\pgfqpoint{0.932095in}{1.262190in}}%
\pgfpathlineto{\pgfqpoint{0.932366in}{0.987337in}}%
\pgfpathlineto{\pgfqpoint{0.933451in}{1.286876in}}%
\pgfpathlineto{\pgfqpoint{0.933994in}{1.312420in}}%
\pgfpathlineto{\pgfqpoint{0.934536in}{1.272189in}}%
\pgfpathlineto{\pgfqpoint{0.935079in}{1.237755in}}%
\pgfpathlineto{\pgfqpoint{0.935486in}{1.297539in}}%
\pgfpathlineto{\pgfqpoint{0.937927in}{1.392983in}}%
\pgfpathlineto{\pgfqpoint{0.938334in}{1.395980in}}%
\pgfpathlineto{\pgfqpoint{0.940097in}{1.193155in}}%
\pgfpathlineto{\pgfqpoint{0.940775in}{1.281388in}}%
\pgfpathlineto{\pgfqpoint{0.942267in}{1.368449in}}%
\pgfpathlineto{\pgfqpoint{0.942538in}{1.333010in}}%
\pgfpathlineto{\pgfqpoint{0.944030in}{0.920753in}}%
\pgfpathlineto{\pgfqpoint{0.944437in}{1.208145in}}%
\pgfpathlineto{\pgfqpoint{0.944573in}{1.240883in}}%
\pgfpathlineto{\pgfqpoint{0.944844in}{1.150540in}}%
\pgfpathlineto{\pgfqpoint{0.944979in}{0.958979in}}%
\pgfpathlineto{\pgfqpoint{0.946200in}{1.232651in}}%
\pgfpathlineto{\pgfqpoint{0.948099in}{1.317550in}}%
\pgfpathlineto{\pgfqpoint{0.948234in}{1.314366in}}%
\pgfpathlineto{\pgfqpoint{0.948913in}{1.204897in}}%
\pgfpathlineto{\pgfqpoint{0.949726in}{1.321328in}}%
\pgfpathlineto{\pgfqpoint{0.951761in}{1.385748in}}%
\pgfpathlineto{\pgfqpoint{0.953524in}{1.340399in}}%
\pgfpathlineto{\pgfqpoint{0.954202in}{1.171449in}}%
\pgfpathlineto{\pgfqpoint{0.955151in}{1.326106in}}%
\pgfpathlineto{\pgfqpoint{0.956914in}{1.280144in}}%
\pgfpathlineto{\pgfqpoint{0.957186in}{0.984342in}}%
\pgfpathlineto{\pgfqpoint{0.957999in}{1.338005in}}%
\pgfpathlineto{\pgfqpoint{0.958271in}{1.337724in}}%
\pgfpathlineto{\pgfqpoint{0.958542in}{1.343636in}}%
\pgfpathlineto{\pgfqpoint{0.959084in}{1.319362in}}%
\pgfpathlineto{\pgfqpoint{0.959491in}{1.162256in}}%
\pgfpathlineto{\pgfqpoint{0.960576in}{1.322938in}}%
\pgfpathlineto{\pgfqpoint{0.961526in}{1.379839in}}%
\pgfpathlineto{\pgfqpoint{0.962339in}{1.354942in}}%
\pgfpathlineto{\pgfqpoint{0.962882in}{1.314668in}}%
\pgfpathlineto{\pgfqpoint{0.963696in}{1.186144in}}%
\pgfpathlineto{\pgfqpoint{0.964645in}{1.281553in}}%
\pgfpathlineto{\pgfqpoint{0.966001in}{1.323874in}}%
\pgfpathlineto{\pgfqpoint{0.966137in}{1.289928in}}%
\pgfpathlineto{\pgfqpoint{0.966679in}{0.967497in}}%
\pgfpathlineto{\pgfqpoint{0.967629in}{1.251675in}}%
\pgfpathlineto{\pgfqpoint{0.967900in}{1.261470in}}%
\pgfpathlineto{\pgfqpoint{0.968443in}{1.242513in}}%
\pgfpathlineto{\pgfqpoint{0.968849in}{1.200830in}}%
\pgfpathlineto{\pgfqpoint{0.969528in}{1.302103in}}%
\pgfpathlineto{\pgfqpoint{0.969663in}{1.301961in}}%
\pgfpathlineto{\pgfqpoint{0.970477in}{1.343042in}}%
\pgfpathlineto{\pgfqpoint{0.971019in}{1.274216in}}%
\pgfpathlineto{\pgfqpoint{0.971155in}{1.283406in}}%
\pgfpathlineto{\pgfqpoint{0.971291in}{1.252594in}}%
\pgfpathlineto{\pgfqpoint{0.971969in}{1.142538in}}%
\pgfpathlineto{\pgfqpoint{0.972647in}{1.259890in}}%
\pgfpathlineto{\pgfqpoint{0.972783in}{1.252675in}}%
\pgfpathlineto{\pgfqpoint{0.973868in}{1.318116in}}%
\pgfpathlineto{\pgfqpoint{0.974139in}{1.279458in}}%
\pgfpathlineto{\pgfqpoint{0.975359in}{1.183017in}}%
\pgfpathlineto{\pgfqpoint{0.975631in}{1.247939in}}%
\pgfpathlineto{\pgfqpoint{0.977258in}{1.385261in}}%
\pgfpathlineto{\pgfqpoint{0.977394in}{1.381616in}}%
\pgfpathlineto{\pgfqpoint{0.977801in}{1.391267in}}%
\pgfpathlineto{\pgfqpoint{0.978343in}{1.363068in}}%
\pgfpathlineto{\pgfqpoint{0.979021in}{1.326784in}}%
\pgfpathlineto{\pgfqpoint{0.981598in}{1.069730in}}%
\pgfpathlineto{\pgfqpoint{0.981734in}{1.212503in}}%
\pgfpathlineto{\pgfqpoint{0.982819in}{1.040680in}}%
\pgfpathlineto{\pgfqpoint{0.983497in}{1.341221in}}%
\pgfpathlineto{\pgfqpoint{0.983768in}{1.343463in}}%
\pgfpathlineto{\pgfqpoint{0.984311in}{1.278689in}}%
\pgfpathlineto{\pgfqpoint{0.984718in}{1.140794in}}%
\pgfpathlineto{\pgfqpoint{0.985531in}{1.284288in}}%
\pgfpathlineto{\pgfqpoint{0.985938in}{1.250929in}}%
\pgfpathlineto{\pgfqpoint{0.986616in}{1.132155in}}%
\pgfpathlineto{\pgfqpoint{0.987023in}{1.271983in}}%
\pgfpathlineto{\pgfqpoint{0.988244in}{1.345447in}}%
\pgfpathlineto{\pgfqpoint{0.988786in}{1.322037in}}%
\pgfpathlineto{\pgfqpoint{0.989058in}{1.322622in}}%
\pgfpathlineto{\pgfqpoint{0.990549in}{1.247240in}}%
\pgfpathlineto{\pgfqpoint{0.990821in}{1.049463in}}%
\pgfpathlineto{\pgfqpoint{0.991906in}{1.309153in}}%
\pgfpathlineto{\pgfqpoint{0.992177in}{1.299840in}}%
\pgfpathlineto{\pgfqpoint{0.992313in}{1.203680in}}%
\pgfpathlineto{\pgfqpoint{0.992991in}{1.360528in}}%
\pgfpathlineto{\pgfqpoint{0.993533in}{1.347369in}}%
\pgfpathlineto{\pgfqpoint{0.993669in}{1.338533in}}%
\pgfpathlineto{\pgfqpoint{0.994754in}{1.349306in}}%
\pgfpathlineto{\pgfqpoint{0.995025in}{1.347228in}}%
\pgfpathlineto{\pgfqpoint{0.995839in}{1.368038in}}%
\pgfpathlineto{\pgfqpoint{0.996653in}{1.363753in}}%
\pgfpathlineto{\pgfqpoint{0.998144in}{1.223936in}}%
\pgfpathlineto{\pgfqpoint{0.999229in}{1.286302in}}%
\pgfpathlineto{\pgfqpoint{0.999501in}{1.305940in}}%
\pgfpathlineto{\pgfqpoint{0.999772in}{1.211252in}}%
\pgfpathlineto{\pgfqpoint{1.000314in}{1.085655in}}%
\pgfpathlineto{\pgfqpoint{1.001128in}{1.293063in}}%
\pgfpathlineto{\pgfqpoint{1.003163in}{1.388348in}}%
\pgfpathlineto{\pgfqpoint{1.004112in}{1.367525in}}%
\pgfpathlineto{\pgfqpoint{1.006282in}{1.168141in}}%
\pgfpathlineto{\pgfqpoint{1.006689in}{1.244075in}}%
\pgfpathlineto{\pgfqpoint{1.008316in}{1.383165in}}%
\pgfpathlineto{\pgfqpoint{1.008994in}{1.358818in}}%
\pgfpathlineto{\pgfqpoint{1.009944in}{1.308501in}}%
\pgfpathlineto{\pgfqpoint{1.010486in}{1.347371in}}%
\pgfpathlineto{\pgfqpoint{1.010758in}{1.362571in}}%
\pgfpathlineto{\pgfqpoint{1.011571in}{1.338827in}}%
\pgfpathlineto{\pgfqpoint{1.011978in}{1.353010in}}%
\pgfpathlineto{\pgfqpoint{1.012114in}{1.352492in}}%
\pgfpathlineto{\pgfqpoint{1.012656in}{1.356205in}}%
\pgfpathlineto{\pgfqpoint{1.012792in}{1.354118in}}%
\pgfpathlineto{\pgfqpoint{1.014148in}{1.387332in}}%
\pgfpathlineto{\pgfqpoint{1.014555in}{1.365659in}}%
\pgfpathlineto{\pgfqpoint{1.014691in}{1.365523in}}%
\pgfpathlineto{\pgfqpoint{1.014826in}{1.368548in}}%
\pgfpathlineto{\pgfqpoint{1.015233in}{1.352151in}}%
\pgfpathlineto{\pgfqpoint{1.016047in}{1.288012in}}%
\pgfpathlineto{\pgfqpoint{1.016861in}{1.327892in}}%
\pgfpathlineto{\pgfqpoint{1.017539in}{1.360304in}}%
\pgfpathlineto{\pgfqpoint{1.018217in}{1.337988in}}%
\pgfpathlineto{\pgfqpoint{1.020523in}{1.103915in}}%
\pgfpathlineto{\pgfqpoint{1.020658in}{0.937378in}}%
\pgfpathlineto{\pgfqpoint{1.021879in}{1.293485in}}%
\pgfpathlineto{\pgfqpoint{1.022286in}{1.333358in}}%
\pgfpathlineto{\pgfqpoint{1.022828in}{1.280748in}}%
\pgfpathlineto{\pgfqpoint{1.023371in}{1.302855in}}%
\pgfpathlineto{\pgfqpoint{1.023778in}{1.270698in}}%
\pgfpathlineto{\pgfqpoint{1.024863in}{1.156443in}}%
\pgfpathlineto{\pgfqpoint{1.024184in}{1.280915in}}%
\pgfpathlineto{\pgfqpoint{1.025269in}{1.265262in}}%
\pgfpathlineto{\pgfqpoint{1.026761in}{1.323749in}}%
\pgfpathlineto{\pgfqpoint{1.025676in}{1.224181in}}%
\pgfpathlineto{\pgfqpoint{1.027168in}{1.323395in}}%
\pgfpathlineto{\pgfqpoint{1.028253in}{1.124923in}}%
\pgfpathlineto{\pgfqpoint{1.029609in}{1.179087in}}%
\pgfpathlineto{\pgfqpoint{1.030694in}{1.315696in}}%
\pgfpathlineto{\pgfqpoint{1.031237in}{1.285472in}}%
\pgfpathlineto{\pgfqpoint{1.031644in}{1.123982in}}%
\pgfpathlineto{\pgfqpoint{1.032186in}{1.314864in}}%
\pgfpathlineto{\pgfqpoint{1.033678in}{1.362540in}}%
\pgfpathlineto{\pgfqpoint{1.033949in}{1.357276in}}%
\pgfpathlineto{\pgfqpoint{1.036662in}{1.271647in}}%
\pgfpathlineto{\pgfqpoint{1.036798in}{1.292700in}}%
\pgfpathlineto{\pgfqpoint{1.037340in}{1.355568in}}%
\pgfpathlineto{\pgfqpoint{1.038289in}{1.320823in}}%
\pgfpathlineto{\pgfqpoint{1.039103in}{1.122056in}}%
\pgfpathlineto{\pgfqpoint{1.039781in}{1.324581in}}%
\pgfpathlineto{\pgfqpoint{1.041138in}{1.387722in}}%
\pgfpathlineto{\pgfqpoint{1.041680in}{1.382258in}}%
\pgfpathlineto{\pgfqpoint{1.042223in}{1.373037in}}%
\pgfpathlineto{\pgfqpoint{1.042358in}{1.376553in}}%
\pgfpathlineto{\pgfqpoint{1.043579in}{1.227964in}}%
\pgfpathlineto{\pgfqpoint{1.043714in}{1.055667in}}%
\pgfpathlineto{\pgfqpoint{1.044257in}{1.254816in}}%
\pgfpathlineto{\pgfqpoint{1.045071in}{1.076042in}}%
\pgfpathlineto{\pgfqpoint{1.046291in}{1.237594in}}%
\pgfpathlineto{\pgfqpoint{1.046563in}{1.166502in}}%
\pgfpathlineto{\pgfqpoint{1.046698in}{1.059024in}}%
\pgfpathlineto{\pgfqpoint{1.047783in}{1.338537in}}%
\pgfpathlineto{\pgfqpoint{1.048054in}{1.340026in}}%
\pgfpathlineto{\pgfqpoint{1.048461in}{1.349553in}}%
\pgfpathlineto{\pgfqpoint{1.049139in}{1.320381in}}%
\pgfpathlineto{\pgfqpoint{1.049275in}{1.329779in}}%
\pgfpathlineto{\pgfqpoint{1.049411in}{1.326074in}}%
\pgfpathlineto{\pgfqpoint{1.049818in}{1.352175in}}%
\pgfpathlineto{\pgfqpoint{1.050224in}{1.367360in}}%
\pgfpathlineto{\pgfqpoint{1.051038in}{1.334380in}}%
\pgfpathlineto{\pgfqpoint{1.051716in}{1.115564in}}%
\pgfpathlineto{\pgfqpoint{1.052801in}{1.291878in}}%
\pgfpathlineto{\pgfqpoint{1.053615in}{1.312165in}}%
\pgfpathlineto{\pgfqpoint{1.053208in}{1.257718in}}%
\pgfpathlineto{\pgfqpoint{1.054022in}{1.261086in}}%
\pgfpathlineto{\pgfqpoint{1.054700in}{1.121253in}}%
\pgfpathlineto{\pgfqpoint{1.054429in}{1.275460in}}%
\pgfpathlineto{\pgfqpoint{1.055243in}{1.262432in}}%
\pgfpathlineto{\pgfqpoint{1.057141in}{1.348921in}}%
\pgfpathlineto{\pgfqpoint{1.057955in}{1.363772in}}%
\pgfpathlineto{\pgfqpoint{1.058498in}{1.346492in}}%
\pgfpathlineto{\pgfqpoint{1.059040in}{1.309378in}}%
\pgfpathlineto{\pgfqpoint{1.059854in}{1.363356in}}%
\pgfpathlineto{\pgfqpoint{1.060396in}{1.385017in}}%
\pgfpathlineto{\pgfqpoint{1.060939in}{1.352722in}}%
\pgfpathlineto{\pgfqpoint{1.061888in}{1.205065in}}%
\pgfpathlineto{\pgfqpoint{1.062024in}{0.957305in}}%
\pgfpathlineto{\pgfqpoint{1.063244in}{1.335862in}}%
\pgfpathlineto{\pgfqpoint{1.064601in}{1.378654in}}%
\pgfpathlineto{\pgfqpoint{1.065279in}{1.376460in}}%
\pgfpathlineto{\pgfqpoint{1.065550in}{1.379159in}}%
\pgfpathlineto{\pgfqpoint{1.066093in}{1.395602in}}%
\pgfpathlineto{\pgfqpoint{1.066635in}{1.378178in}}%
\pgfpathlineto{\pgfqpoint{1.067856in}{1.231511in}}%
\pgfpathlineto{\pgfqpoint{1.067991in}{1.149520in}}%
\pgfpathlineto{\pgfqpoint{1.068805in}{1.289004in}}%
\pgfpathlineto{\pgfqpoint{1.069348in}{1.258587in}}%
\pgfpathlineto{\pgfqpoint{1.069619in}{1.282936in}}%
\pgfpathlineto{\pgfqpoint{1.070026in}{1.158266in}}%
\pgfpathlineto{\pgfqpoint{1.070975in}{1.314602in}}%
\pgfpathlineto{\pgfqpoint{1.073009in}{1.391232in}}%
\pgfpathlineto{\pgfqpoint{1.073281in}{1.384545in}}%
\pgfpathlineto{\pgfqpoint{1.074501in}{1.272541in}}%
\pgfpathlineto{\pgfqpoint{1.074773in}{1.187884in}}%
\pgfpathlineto{\pgfqpoint{1.075315in}{1.316333in}}%
\pgfpathlineto{\pgfqpoint{1.075993in}{1.283968in}}%
\pgfpathlineto{\pgfqpoint{1.076264in}{1.127507in}}%
\pgfpathlineto{\pgfqpoint{1.077214in}{1.290999in}}%
\pgfpathlineto{\pgfqpoint{1.077621in}{1.270853in}}%
\pgfpathlineto{\pgfqpoint{1.080198in}{1.367636in}}%
\pgfpathlineto{\pgfqpoint{1.080740in}{1.372439in}}%
\pgfpathlineto{\pgfqpoint{1.081283in}{1.348237in}}%
\pgfpathlineto{\pgfqpoint{1.081825in}{1.300876in}}%
\pgfpathlineto{\pgfqpoint{1.081961in}{1.283657in}}%
\pgfpathlineto{\pgfqpoint{1.082639in}{1.355532in}}%
\pgfpathlineto{\pgfqpoint{1.082910in}{1.353984in}}%
\pgfpathlineto{\pgfqpoint{1.083181in}{1.367110in}}%
\pgfpathlineto{\pgfqpoint{1.083995in}{1.382861in}}%
\pgfpathlineto{\pgfqpoint{1.084673in}{1.369247in}}%
\pgfpathlineto{\pgfqpoint{1.085623in}{1.265372in}}%
\pgfpathlineto{\pgfqpoint{1.086436in}{0.974281in}}%
\pgfpathlineto{\pgfqpoint{1.086979in}{1.279555in}}%
\pgfpathlineto{\pgfqpoint{1.087114in}{1.287083in}}%
\pgfpathlineto{\pgfqpoint{1.087386in}{1.243200in}}%
\pgfpathlineto{\pgfqpoint{1.087793in}{1.137864in}}%
\pgfpathlineto{\pgfqpoint{1.088471in}{1.314484in}}%
\pgfpathlineto{\pgfqpoint{1.088606in}{1.311441in}}%
\pgfpathlineto{\pgfqpoint{1.089691in}{1.198278in}}%
\pgfpathlineto{\pgfqpoint{1.089827in}{0.901743in}}%
\pgfpathlineto{\pgfqpoint{1.090776in}{1.379118in}}%
\pgfpathlineto{\pgfqpoint{1.091048in}{1.378773in}}%
\pgfpathlineto{\pgfqpoint{1.091183in}{1.383065in}}%
\pgfpathlineto{\pgfqpoint{1.091861in}{1.360760in}}%
\pgfpathlineto{\pgfqpoint{1.091997in}{1.365002in}}%
\pgfpathlineto{\pgfqpoint{1.092268in}{1.356923in}}%
\pgfpathlineto{\pgfqpoint{1.093082in}{1.327165in}}%
\pgfpathlineto{\pgfqpoint{1.093760in}{1.352673in}}%
\pgfpathlineto{\pgfqpoint{1.093896in}{1.357410in}}%
\pgfpathlineto{\pgfqpoint{1.094303in}{1.330297in}}%
\pgfpathlineto{\pgfqpoint{1.096066in}{1.096854in}}%
\pgfpathlineto{\pgfqpoint{1.096744in}{1.221225in}}%
\pgfpathlineto{\pgfqpoint{1.097286in}{1.302365in}}%
\pgfpathlineto{\pgfqpoint{1.098236in}{1.233209in}}%
\pgfpathlineto{\pgfqpoint{1.098371in}{1.226370in}}%
\pgfpathlineto{\pgfqpoint{1.098507in}{1.078434in}}%
\pgfpathlineto{\pgfqpoint{1.099049in}{1.313008in}}%
\pgfpathlineto{\pgfqpoint{1.099728in}{1.305407in}}%
\pgfpathlineto{\pgfqpoint{1.100541in}{1.230804in}}%
\pgfpathlineto{\pgfqpoint{1.101084in}{0.989278in}}%
\pgfpathlineto{\pgfqpoint{1.101491in}{1.250489in}}%
\pgfpathlineto{\pgfqpoint{1.101762in}{1.248477in}}%
\pgfpathlineto{\pgfqpoint{1.102304in}{1.348681in}}%
\pgfpathlineto{\pgfqpoint{1.103254in}{1.254896in}}%
\pgfpathlineto{\pgfqpoint{1.103525in}{1.097480in}}%
\pgfpathlineto{\pgfqpoint{1.104203in}{1.365416in}}%
\pgfpathlineto{\pgfqpoint{1.104474in}{1.362489in}}%
\pgfpathlineto{\pgfqpoint{1.104610in}{1.365154in}}%
\pgfpathlineto{\pgfqpoint{1.105288in}{1.349841in}}%
\pgfpathlineto{\pgfqpoint{1.105831in}{1.270378in}}%
\pgfpathlineto{\pgfqpoint{1.106102in}{1.089197in}}%
\pgfpathlineto{\pgfqpoint{1.107051in}{1.340831in}}%
\pgfpathlineto{\pgfqpoint{1.107187in}{1.330538in}}%
\pgfpathlineto{\pgfqpoint{1.107729in}{1.284808in}}%
\pgfpathlineto{\pgfqpoint{1.108272in}{1.068771in}}%
\pgfpathlineto{\pgfqpoint{1.109086in}{1.347182in}}%
\pgfpathlineto{\pgfqpoint{1.110442in}{1.274806in}}%
\pgfpathlineto{\pgfqpoint{1.110849in}{1.306948in}}%
\pgfpathlineto{\pgfqpoint{1.111527in}{1.363936in}}%
\pgfpathlineto{\pgfqpoint{1.112341in}{1.305009in}}%
\pgfpathlineto{\pgfqpoint{1.113697in}{1.096922in}}%
\pgfpathlineto{\pgfqpoint{1.114104in}{1.251610in}}%
\pgfpathlineto{\pgfqpoint{1.115867in}{1.364690in}}%
\pgfpathlineto{\pgfqpoint{1.116545in}{1.380893in}}%
\pgfpathlineto{\pgfqpoint{1.117494in}{1.369460in}}%
\pgfpathlineto{\pgfqpoint{1.117630in}{1.369373in}}%
\pgfpathlineto{\pgfqpoint{1.118173in}{1.375670in}}%
\pgfpathlineto{\pgfqpoint{1.119258in}{1.301340in}}%
\pgfpathlineto{\pgfqpoint{1.120614in}{0.896864in}}%
\pgfpathlineto{\pgfqpoint{1.120885in}{1.251039in}}%
\pgfpathlineto{\pgfqpoint{1.122919in}{1.402278in}}%
\pgfpathlineto{\pgfqpoint{1.124683in}{1.437674in}}%
\pgfpathlineto{\pgfqpoint{1.125225in}{1.436252in}}%
\pgfpathlineto{\pgfqpoint{1.127124in}{1.401719in}}%
\pgfpathlineto{\pgfqpoint{1.128344in}{1.410596in}}%
\pgfpathlineto{\pgfqpoint{1.128751in}{1.407182in}}%
\pgfpathlineto{\pgfqpoint{1.129701in}{1.362034in}}%
\pgfpathlineto{\pgfqpoint{1.130379in}{1.396828in}}%
\pgfpathlineto{\pgfqpoint{1.131735in}{1.411336in}}%
\pgfpathlineto{\pgfqpoint{1.130921in}{1.389674in}}%
\pgfpathlineto{\pgfqpoint{1.131871in}{1.397941in}}%
\pgfpathlineto{\pgfqpoint{1.133091in}{1.220063in}}%
\pgfpathlineto{\pgfqpoint{1.133227in}{1.114247in}}%
\pgfpathlineto{\pgfqpoint{1.134448in}{1.339675in}}%
\pgfpathlineto{\pgfqpoint{1.134583in}{1.344443in}}%
\pgfpathlineto{\pgfqpoint{1.134990in}{1.313384in}}%
\pgfpathlineto{\pgfqpoint{1.135533in}{1.327693in}}%
\pgfpathlineto{\pgfqpoint{1.136211in}{1.290547in}}%
\pgfpathlineto{\pgfqpoint{1.136346in}{1.342469in}}%
\pgfpathlineto{\pgfqpoint{1.136482in}{1.339285in}}%
\pgfpathlineto{\pgfqpoint{1.137296in}{1.372742in}}%
\pgfpathlineto{\pgfqpoint{1.138109in}{1.355569in}}%
\pgfpathlineto{\pgfqpoint{1.138245in}{1.355304in}}%
\pgfpathlineto{\pgfqpoint{1.139194in}{1.223901in}}%
\pgfpathlineto{\pgfqpoint{1.140144in}{1.289001in}}%
\pgfpathlineto{\pgfqpoint{1.140822in}{0.885632in}}%
\pgfpathlineto{\pgfqpoint{1.141771in}{0.786780in}}%
\pgfpathlineto{\pgfqpoint{1.142449in}{1.338929in}}%
\pgfpathlineto{\pgfqpoint{1.142992in}{1.258359in}}%
\pgfpathlineto{\pgfqpoint{1.144348in}{0.599600in}}%
\pgfpathlineto{\pgfqpoint{1.143534in}{1.265089in}}%
\pgfpathlineto{\pgfqpoint{1.144484in}{1.166437in}}%
\pgfpathlineto{\pgfqpoint{1.144891in}{1.129167in}}%
\pgfpathlineto{\pgfqpoint{1.145298in}{1.238888in}}%
\pgfpathlineto{\pgfqpoint{1.146383in}{1.292101in}}%
\pgfpathlineto{\pgfqpoint{1.145569in}{1.227299in}}%
\pgfpathlineto{\pgfqpoint{1.146654in}{1.266168in}}%
\pgfpathlineto{\pgfqpoint{1.146789in}{1.125577in}}%
\pgfpathlineto{\pgfqpoint{1.148010in}{1.342578in}}%
\pgfpathlineto{\pgfqpoint{1.149638in}{1.274855in}}%
\pgfpathlineto{\pgfqpoint{1.150316in}{1.280257in}}%
\pgfpathlineto{\pgfqpoint{1.150858in}{1.249011in}}%
\pgfpathlineto{\pgfqpoint{1.151401in}{1.266021in}}%
\pgfpathlineto{\pgfqpoint{1.152079in}{1.323514in}}%
\pgfpathlineto{\pgfqpoint{1.152621in}{1.248464in}}%
\pgfpathlineto{\pgfqpoint{1.152757in}{1.096117in}}%
\pgfpathlineto{\pgfqpoint{1.153978in}{1.347380in}}%
\pgfpathlineto{\pgfqpoint{1.155741in}{1.413814in}}%
\pgfpathlineto{\pgfqpoint{1.156690in}{1.386969in}}%
\pgfpathlineto{\pgfqpoint{1.157504in}{1.246477in}}%
\pgfpathlineto{\pgfqpoint{1.157639in}{1.251455in}}%
\pgfpathlineto{\pgfqpoint{1.157775in}{1.106657in}}%
\pgfpathlineto{\pgfqpoint{1.158996in}{1.381621in}}%
\pgfpathlineto{\pgfqpoint{1.159267in}{1.381876in}}%
\pgfpathlineto{\pgfqpoint{1.159403in}{1.378156in}}%
\pgfpathlineto{\pgfqpoint{1.160352in}{1.310981in}}%
\pgfpathlineto{\pgfqpoint{1.160759in}{1.110707in}}%
\pgfpathlineto{\pgfqpoint{1.161708in}{1.325008in}}%
\pgfpathlineto{\pgfqpoint{1.161844in}{1.319621in}}%
\pgfpathlineto{\pgfqpoint{1.163064in}{1.378648in}}%
\pgfpathlineto{\pgfqpoint{1.164014in}{1.364370in}}%
\pgfpathlineto{\pgfqpoint{1.164421in}{1.332937in}}%
\pgfpathlineto{\pgfqpoint{1.165641in}{1.349039in}}%
\pgfpathlineto{\pgfqpoint{1.165777in}{1.349995in}}%
\pgfpathlineto{\pgfqpoint{1.167676in}{1.398170in}}%
\pgfpathlineto{\pgfqpoint{1.168625in}{1.411177in}}%
\pgfpathlineto{\pgfqpoint{1.169303in}{1.404975in}}%
\pgfpathlineto{\pgfqpoint{1.171744in}{1.362039in}}%
\pgfpathlineto{\pgfqpoint{1.172151in}{1.383941in}}%
\pgfpathlineto{\pgfqpoint{1.173236in}{1.413972in}}%
\pgfpathlineto{\pgfqpoint{1.173914in}{1.404463in}}%
\pgfpathlineto{\pgfqpoint{1.175813in}{1.338276in}}%
\pgfpathlineto{\pgfqpoint{1.176084in}{1.351163in}}%
\pgfpathlineto{\pgfqpoint{1.176898in}{1.387054in}}%
\pgfpathlineto{\pgfqpoint{1.177848in}{1.366470in}}%
\pgfpathlineto{\pgfqpoint{1.178254in}{1.359153in}}%
\pgfpathlineto{\pgfqpoint{1.178661in}{1.371100in}}%
\pgfpathlineto{\pgfqpoint{1.180289in}{1.427871in}}%
\pgfpathlineto{\pgfqpoint{1.180967in}{1.409689in}}%
\pgfpathlineto{\pgfqpoint{1.181509in}{1.374099in}}%
\pgfpathlineto{\pgfqpoint{1.182594in}{1.118982in}}%
\pgfpathlineto{\pgfqpoint{1.183815in}{1.222269in}}%
\pgfpathlineto{\pgfqpoint{1.185985in}{1.395217in}}%
\pgfpathlineto{\pgfqpoint{1.186121in}{1.389497in}}%
\pgfpathlineto{\pgfqpoint{1.186799in}{1.423229in}}%
\pgfpathlineto{\pgfqpoint{1.187070in}{1.432079in}}%
\pgfpathlineto{\pgfqpoint{1.187748in}{1.417572in}}%
\pgfpathlineto{\pgfqpoint{1.188291in}{1.422131in}}%
\pgfpathlineto{\pgfqpoint{1.190189in}{1.303444in}}%
\pgfpathlineto{\pgfqpoint{1.191817in}{1.066144in}}%
\pgfpathlineto{\pgfqpoint{1.190732in}{1.306218in}}%
\pgfpathlineto{\pgfqpoint{1.192224in}{1.230503in}}%
\pgfpathlineto{\pgfqpoint{1.193716in}{1.323300in}}%
\pgfpathlineto{\pgfqpoint{1.193987in}{1.322141in}}%
\pgfpathlineto{\pgfqpoint{1.194394in}{1.275982in}}%
\pgfpathlineto{\pgfqpoint{1.194665in}{0.966271in}}%
\pgfpathlineto{\pgfqpoint{1.195072in}{1.313340in}}%
\pgfpathlineto{\pgfqpoint{1.195886in}{1.243394in}}%
\pgfpathlineto{\pgfqpoint{1.196021in}{1.251877in}}%
\pgfpathlineto{\pgfqpoint{1.196157in}{1.059010in}}%
\pgfpathlineto{\pgfqpoint{1.197106in}{1.327364in}}%
\pgfpathlineto{\pgfqpoint{1.197513in}{1.311582in}}%
\pgfpathlineto{\pgfqpoint{1.198598in}{1.017509in}}%
\pgfpathlineto{\pgfqpoint{1.199954in}{1.161488in}}%
\pgfpathlineto{\pgfqpoint{1.201853in}{1.380891in}}%
\pgfpathlineto{\pgfqpoint{1.202260in}{1.382943in}}%
\pgfpathlineto{\pgfqpoint{1.202531in}{1.372099in}}%
\pgfpathlineto{\pgfqpoint{1.202938in}{1.367120in}}%
\pgfpathlineto{\pgfqpoint{1.204837in}{1.065001in}}%
\pgfpathlineto{\pgfqpoint{1.205108in}{1.274751in}}%
\pgfpathlineto{\pgfqpoint{1.205786in}{1.344995in}}%
\pgfpathlineto{\pgfqpoint{1.206600in}{1.284019in}}%
\pgfpathlineto{\pgfqpoint{1.206871in}{1.255465in}}%
\pgfpathlineto{\pgfqpoint{1.207549in}{1.329498in}}%
\pgfpathlineto{\pgfqpoint{1.208092in}{1.277804in}}%
\pgfpathlineto{\pgfqpoint{1.209584in}{1.380966in}}%
\pgfpathlineto{\pgfqpoint{1.210669in}{1.366876in}}%
\pgfpathlineto{\pgfqpoint{1.210940in}{1.361216in}}%
\pgfpathlineto{\pgfqpoint{1.211754in}{1.204083in}}%
\pgfpathlineto{\pgfqpoint{1.211889in}{1.076693in}}%
\pgfpathlineto{\pgfqpoint{1.212432in}{1.306976in}}%
\pgfpathlineto{\pgfqpoint{1.213246in}{1.146577in}}%
\pgfpathlineto{\pgfqpoint{1.214738in}{1.347657in}}%
\pgfpathlineto{\pgfqpoint{1.215687in}{1.345611in}}%
\pgfpathlineto{\pgfqpoint{1.216365in}{1.286245in}}%
\pgfpathlineto{\pgfqpoint{1.216908in}{1.340323in}}%
\pgfpathlineto{\pgfqpoint{1.219078in}{1.437273in}}%
\pgfpathlineto{\pgfqpoint{1.219756in}{1.439076in}}%
\pgfpathlineto{\pgfqpoint{1.223960in}{1.377245in}}%
\pgfpathlineto{\pgfqpoint{1.225045in}{1.333759in}}%
\pgfpathlineto{\pgfqpoint{1.225588in}{1.345951in}}%
\pgfpathlineto{\pgfqpoint{1.229792in}{1.432562in}}%
\pgfpathlineto{\pgfqpoint{1.229928in}{1.425873in}}%
\pgfpathlineto{\pgfqpoint{1.231962in}{1.375417in}}%
\pgfpathlineto{\pgfqpoint{1.232776in}{1.355167in}}%
\pgfpathlineto{\pgfqpoint{1.233725in}{1.357818in}}%
\pgfpathlineto{\pgfqpoint{1.235217in}{1.399231in}}%
\pgfpathlineto{\pgfqpoint{1.235759in}{1.383669in}}%
\pgfpathlineto{\pgfqpoint{1.235895in}{1.360688in}}%
\pgfpathlineto{\pgfqpoint{1.237116in}{1.384627in}}%
\pgfpathlineto{\pgfqpoint{1.237251in}{1.382753in}}%
\pgfpathlineto{\pgfqpoint{1.237794in}{1.399642in}}%
\pgfpathlineto{\pgfqpoint{1.239693in}{1.440440in}}%
\pgfpathlineto{\pgfqpoint{1.240913in}{1.438237in}}%
\pgfpathlineto{\pgfqpoint{1.240371in}{1.441806in}}%
\pgfpathlineto{\pgfqpoint{1.241184in}{1.439566in}}%
\pgfpathlineto{\pgfqpoint{1.242134in}{1.454221in}}%
\pgfpathlineto{\pgfqpoint{1.242812in}{1.443716in}}%
\pgfpathlineto{\pgfqpoint{1.243897in}{1.416978in}}%
\pgfpathlineto{\pgfqpoint{1.244575in}{1.407595in}}%
\pgfpathlineto{\pgfqpoint{1.245253in}{1.426841in}}%
\pgfpathlineto{\pgfqpoint{1.245389in}{1.423674in}}%
\pgfpathlineto{\pgfqpoint{1.245931in}{1.442205in}}%
\pgfpathlineto{\pgfqpoint{1.249051in}{1.483529in}}%
\pgfpathlineto{\pgfqpoint{1.249322in}{1.480800in}}%
\pgfpathlineto{\pgfqpoint{1.250814in}{1.465120in}}%
\pgfpathlineto{\pgfqpoint{1.251763in}{1.469213in}}%
\pgfpathlineto{\pgfqpoint{1.252848in}{1.472367in}}%
\pgfpathlineto{\pgfqpoint{1.253255in}{1.468703in}}%
\pgfpathlineto{\pgfqpoint{1.254611in}{1.447485in}}%
\pgfpathlineto{\pgfqpoint{1.255289in}{1.454962in}}%
\pgfpathlineto{\pgfqpoint{1.257731in}{1.481764in}}%
\pgfpathlineto{\pgfqpoint{1.258002in}{1.479322in}}%
\pgfpathlineto{\pgfqpoint{1.258680in}{1.473277in}}%
\pgfpathlineto{\pgfqpoint{1.259494in}{1.475675in}}%
\pgfpathlineto{\pgfqpoint{1.260308in}{1.484085in}}%
\pgfpathlineto{\pgfqpoint{1.261121in}{1.480707in}}%
\pgfpathlineto{\pgfqpoint{1.265054in}{1.457754in}}%
\pgfpathlineto{\pgfqpoint{1.265326in}{1.459577in}}%
\pgfpathlineto{\pgfqpoint{1.265597in}{1.462565in}}%
\pgfpathlineto{\pgfqpoint{1.268174in}{1.495341in}}%
\pgfpathlineto{\pgfqpoint{1.268445in}{1.494971in}}%
\pgfpathlineto{\pgfqpoint{1.268581in}{1.494518in}}%
\pgfpathlineto{\pgfqpoint{1.269123in}{1.497609in}}%
\pgfpathlineto{\pgfqpoint{1.269530in}{1.495443in}}%
\pgfpathlineto{\pgfqpoint{1.272378in}{1.521626in}}%
\pgfpathlineto{\pgfqpoint{1.272921in}{1.517135in}}%
\pgfpathlineto{\pgfqpoint{1.274819in}{1.491180in}}%
\pgfpathlineto{\pgfqpoint{1.275633in}{1.494361in}}%
\pgfpathlineto{\pgfqpoint{1.276989in}{1.504274in}}%
\pgfpathlineto{\pgfqpoint{1.277532in}{1.500292in}}%
\pgfpathlineto{\pgfqpoint{1.278346in}{1.489528in}}%
\pgfpathlineto{\pgfqpoint{1.279159in}{1.495789in}}%
\pgfpathlineto{\pgfqpoint{1.280380in}{1.499250in}}%
\pgfpathlineto{\pgfqpoint{1.280651in}{1.495714in}}%
\pgfpathlineto{\pgfqpoint{1.280787in}{1.494009in}}%
\pgfpathlineto{\pgfqpoint{1.281465in}{1.502541in}}%
\pgfpathlineto{\pgfqpoint{1.281601in}{1.501692in}}%
\pgfpathlineto{\pgfqpoint{1.283499in}{1.508859in}}%
\pgfpathlineto{\pgfqpoint{1.284042in}{1.507702in}}%
\pgfpathlineto{\pgfqpoint{1.286348in}{1.498793in}}%
\pgfpathlineto{\pgfqpoint{1.286619in}{1.497277in}}%
\pgfpathlineto{\pgfqpoint{1.287568in}{1.501278in}}%
\pgfpathlineto{\pgfqpoint{1.289467in}{1.507886in}}%
\pgfpathlineto{\pgfqpoint{1.289738in}{1.506997in}}%
\pgfpathlineto{\pgfqpoint{1.291501in}{1.500917in}}%
\pgfpathlineto{\pgfqpoint{1.290416in}{1.509088in}}%
\pgfpathlineto{\pgfqpoint{1.291773in}{1.503284in}}%
\pgfpathlineto{\pgfqpoint{1.293943in}{1.527852in}}%
\pgfpathlineto{\pgfqpoint{1.295163in}{1.518568in}}%
\pgfpathlineto{\pgfqpoint{1.296791in}{1.504358in}}%
\pgfpathlineto{\pgfqpoint{1.297469in}{1.509424in}}%
\pgfpathlineto{\pgfqpoint{1.300453in}{1.524280in}}%
\pgfpathlineto{\pgfqpoint{1.300724in}{1.523503in}}%
\pgfpathlineto{\pgfqpoint{1.301402in}{1.522319in}}%
\pgfpathlineto{\pgfqpoint{1.301538in}{1.520969in}}%
\pgfpathlineto{\pgfqpoint{1.302623in}{1.524366in}}%
\pgfpathlineto{\pgfqpoint{1.303029in}{1.524776in}}%
\pgfpathlineto{\pgfqpoint{1.303572in}{1.523242in}}%
\pgfpathlineto{\pgfqpoint{1.304114in}{1.522168in}}%
\pgfpathlineto{\pgfqpoint{1.304521in}{1.524017in}}%
\pgfpathlineto{\pgfqpoint{1.306149in}{1.533743in}}%
\pgfpathlineto{\pgfqpoint{1.306691in}{1.530756in}}%
\pgfpathlineto{\pgfqpoint{1.307641in}{1.529378in}}%
\pgfpathlineto{\pgfqpoint{1.307776in}{1.531377in}}%
\pgfpathlineto{\pgfqpoint{1.307912in}{1.533189in}}%
\pgfpathlineto{\pgfqpoint{1.309133in}{1.530293in}}%
\pgfpathlineto{\pgfqpoint{1.309539in}{1.528244in}}%
\pgfpathlineto{\pgfqpoint{1.310218in}{1.533098in}}%
\pgfpathlineto{\pgfqpoint{1.312794in}{1.546344in}}%
\pgfpathlineto{\pgfqpoint{1.312930in}{1.546131in}}%
\pgfpathlineto{\pgfqpoint{1.313744in}{1.546757in}}%
\pgfpathlineto{\pgfqpoint{1.313879in}{1.547210in}}%
\pgfpathlineto{\pgfqpoint{1.315643in}{1.548365in}}%
\pgfpathlineto{\pgfqpoint{1.315914in}{1.547792in}}%
\pgfpathlineto{\pgfqpoint{1.318084in}{1.541640in}}%
\pgfpathlineto{\pgfqpoint{1.318491in}{1.544254in}}%
\pgfpathlineto{\pgfqpoint{1.319169in}{1.546897in}}%
\pgfpathlineto{\pgfqpoint{1.320254in}{1.546664in}}%
\pgfpathlineto{\pgfqpoint{1.320525in}{1.546533in}}%
\pgfpathlineto{\pgfqpoint{1.323509in}{1.558316in}}%
\pgfpathlineto{\pgfqpoint{1.324458in}{1.561009in}}%
\pgfpathlineto{\pgfqpoint{1.325136in}{1.562487in}}%
\pgfpathlineto{\pgfqpoint{1.325679in}{1.560642in}}%
\pgfpathlineto{\pgfqpoint{1.327035in}{1.550720in}}%
\pgfpathlineto{\pgfqpoint{1.328391in}{1.551614in}}%
\pgfpathlineto{\pgfqpoint{1.330561in}{1.554904in}}%
\pgfpathlineto{\pgfqpoint{1.329341in}{1.550782in}}%
\pgfpathlineto{\pgfqpoint{1.330833in}{1.553234in}}%
\pgfpathlineto{\pgfqpoint{1.331511in}{1.549178in}}%
\pgfpathlineto{\pgfqpoint{1.332867in}{1.543475in}}%
\pgfpathlineto{\pgfqpoint{1.333409in}{1.544083in}}%
\pgfpathlineto{\pgfqpoint{1.335037in}{1.552015in}}%
\pgfpathlineto{\pgfqpoint{1.335986in}{1.551409in}}%
\pgfpathlineto{\pgfqpoint{1.336529in}{1.552553in}}%
\pgfpathlineto{\pgfqpoint{1.337207in}{1.550711in}}%
\pgfpathlineto{\pgfqpoint{1.338021in}{1.549474in}}%
\pgfpathlineto{\pgfqpoint{1.338292in}{1.551077in}}%
\pgfpathlineto{\pgfqpoint{1.338563in}{1.550403in}}%
\pgfpathlineto{\pgfqpoint{1.338834in}{1.551559in}}%
\pgfpathlineto{\pgfqpoint{1.339784in}{1.548874in}}%
\pgfpathlineto{\pgfqpoint{1.339919in}{1.549212in}}%
\pgfpathlineto{\pgfqpoint{1.340055in}{1.548837in}}%
\pgfpathlineto{\pgfqpoint{1.340733in}{1.550807in}}%
\pgfpathlineto{\pgfqpoint{1.340869in}{1.550605in}}%
\pgfpathlineto{\pgfqpoint{1.343717in}{1.564266in}}%
\pgfpathlineto{\pgfqpoint{1.343988in}{1.563800in}}%
\pgfpathlineto{\pgfqpoint{1.348057in}{1.557440in}}%
\pgfpathlineto{\pgfqpoint{1.348193in}{1.557754in}}%
\pgfpathlineto{\pgfqpoint{1.353618in}{1.576866in}}%
\pgfpathlineto{\pgfqpoint{1.354703in}{1.581024in}}%
\pgfpathlineto{\pgfqpoint{1.355516in}{1.578973in}}%
\pgfpathlineto{\pgfqpoint{1.357279in}{1.575053in}}%
\pgfpathlineto{\pgfqpoint{1.358229in}{1.575772in}}%
\pgfpathlineto{\pgfqpoint{1.359043in}{1.577528in}}%
\pgfpathlineto{\pgfqpoint{1.360806in}{1.582063in}}%
\pgfpathlineto{\pgfqpoint{1.361619in}{1.581109in}}%
\pgfpathlineto{\pgfqpoint{1.362704in}{1.576936in}}%
\pgfpathlineto{\pgfqpoint{1.363654in}{1.579490in}}%
\pgfpathlineto{\pgfqpoint{1.365824in}{1.582778in}}%
\pgfpathlineto{\pgfqpoint{1.369621in}{1.578900in}}%
\pgfpathlineto{\pgfqpoint{1.370028in}{1.579616in}}%
\pgfpathlineto{\pgfqpoint{1.374233in}{1.592803in}}%
\pgfpathlineto{\pgfqpoint{1.374639in}{1.592080in}}%
\pgfpathlineto{\pgfqpoint{1.376538in}{1.593675in}}%
\pgfpathlineto{\pgfqpoint{1.374911in}{1.591619in}}%
\pgfpathlineto{\pgfqpoint{1.377081in}{1.593282in}}%
\pgfpathlineto{\pgfqpoint{1.378166in}{1.591119in}}%
\pgfpathlineto{\pgfqpoint{1.379793in}{1.585131in}}%
\pgfpathlineto{\pgfqpoint{1.380200in}{1.587003in}}%
\pgfpathlineto{\pgfqpoint{1.382777in}{1.595990in}}%
\pgfpathlineto{\pgfqpoint{1.383184in}{1.595163in}}%
\pgfpathlineto{\pgfqpoint{1.383455in}{1.595506in}}%
\pgfpathlineto{\pgfqpoint{1.385218in}{1.602959in}}%
\pgfpathlineto{\pgfqpoint{1.385489in}{1.602863in}}%
\pgfpathlineto{\pgfqpoint{1.388066in}{1.601436in}}%
\pgfpathlineto{\pgfqpoint{1.389829in}{1.602542in}}%
\pgfpathlineto{\pgfqpoint{1.389965in}{1.601395in}}%
\pgfpathlineto{\pgfqpoint{1.391999in}{1.595079in}}%
\pgfpathlineto{\pgfqpoint{1.392135in}{1.595952in}}%
\pgfpathlineto{\pgfqpoint{1.392406in}{1.595697in}}%
\pgfpathlineto{\pgfqpoint{1.397424in}{1.606599in}}%
\pgfpathlineto{\pgfqpoint{1.399052in}{1.606812in}}%
\pgfpathlineto{\pgfqpoint{1.401358in}{1.605362in}}%
\pgfpathlineto{\pgfqpoint{1.401629in}{1.605823in}}%
\pgfpathlineto{\pgfqpoint{1.402985in}{1.609797in}}%
\pgfpathlineto{\pgfqpoint{1.403934in}{1.609307in}}%
\pgfpathlineto{\pgfqpoint{1.407461in}{1.609026in}}%
\pgfpathlineto{\pgfqpoint{1.408953in}{1.607727in}}%
\pgfpathlineto{\pgfqpoint{1.409359in}{1.608724in}}%
\pgfpathlineto{\pgfqpoint{1.409495in}{1.609479in}}%
\pgfpathlineto{\pgfqpoint{1.410309in}{1.608184in}}%
\pgfpathlineto{\pgfqpoint{1.410851in}{1.608864in}}%
\pgfpathlineto{\pgfqpoint{1.412343in}{1.605519in}}%
\pgfpathlineto{\pgfqpoint{1.412750in}{1.606816in}}%
\pgfpathlineto{\pgfqpoint{1.413971in}{1.608970in}}%
\pgfpathlineto{\pgfqpoint{1.416819in}{1.616654in}}%
\pgfpathlineto{\pgfqpoint{1.419938in}{1.613601in}}%
\pgfpathlineto{\pgfqpoint{1.421430in}{1.613009in}}%
\pgfpathlineto{\pgfqpoint{1.421701in}{1.613327in}}%
\pgfpathlineto{\pgfqpoint{1.424278in}{1.618365in}}%
\pgfpathlineto{\pgfqpoint{1.424414in}{1.617977in}}%
\pgfpathlineto{\pgfqpoint{1.425092in}{1.619593in}}%
\pgfpathlineto{\pgfqpoint{1.427126in}{1.623025in}}%
\pgfpathlineto{\pgfqpoint{1.428483in}{1.625022in}}%
\pgfpathlineto{\pgfqpoint{1.429432in}{1.624001in}}%
\pgfpathlineto{\pgfqpoint{1.432009in}{1.622662in}}%
\pgfpathlineto{\pgfqpoint{1.433908in}{1.625358in}}%
\pgfpathlineto{\pgfqpoint{1.436349in}{1.629095in}}%
\pgfpathlineto{\pgfqpoint{1.437434in}{1.627468in}}%
\pgfpathlineto{\pgfqpoint{1.438383in}{1.629297in}}%
\pgfpathlineto{\pgfqpoint{1.439333in}{1.631388in}}%
\pgfpathlineto{\pgfqpoint{1.440011in}{1.630007in}}%
\pgfpathlineto{\pgfqpoint{1.441231in}{1.627789in}}%
\pgfpathlineto{\pgfqpoint{1.441774in}{1.629378in}}%
\pgfpathlineto{\pgfqpoint{1.444215in}{1.634038in}}%
\pgfpathlineto{\pgfqpoint{1.445707in}{1.635560in}}%
\pgfpathlineto{\pgfqpoint{1.445978in}{1.636657in}}%
\pgfpathlineto{\pgfqpoint{1.447334in}{1.635899in}}%
\pgfpathlineto{\pgfqpoint{1.453573in}{1.638558in}}%
\pgfpathlineto{\pgfqpoint{1.453980in}{1.637594in}}%
\pgfpathlineto{\pgfqpoint{1.460490in}{1.638829in}}%
\pgfpathlineto{\pgfqpoint{1.462389in}{1.641031in}}%
\pgfpathlineto{\pgfqpoint{1.462796in}{1.640458in}}%
\pgfpathlineto{\pgfqpoint{1.464288in}{1.640440in}}%
\pgfpathlineto{\pgfqpoint{1.464423in}{1.640720in}}%
\pgfpathlineto{\pgfqpoint{1.465915in}{1.640997in}}%
\pgfpathlineto{\pgfqpoint{1.466051in}{1.641413in}}%
\pgfpathlineto{\pgfqpoint{1.467949in}{1.642754in}}%
\pgfpathlineto{\pgfqpoint{1.469984in}{1.643954in}}%
\pgfpathlineto{\pgfqpoint{1.470933in}{1.645504in}}%
\pgfpathlineto{\pgfqpoint{1.471747in}{1.645143in}}%
\pgfpathlineto{\pgfqpoint{1.473510in}{1.643925in}}%
\pgfpathlineto{\pgfqpoint{1.473781in}{1.644210in}}%
\pgfpathlineto{\pgfqpoint{1.475680in}{1.645870in}}%
\pgfpathlineto{\pgfqpoint{1.480291in}{1.649358in}}%
\pgfpathlineto{\pgfqpoint{1.483818in}{1.648478in}}%
\pgfpathlineto{\pgfqpoint{1.485445in}{1.650352in}}%
\pgfpathlineto{\pgfqpoint{1.485852in}{1.649483in}}%
\pgfpathlineto{\pgfqpoint{1.486937in}{1.648087in}}%
\pgfpathlineto{\pgfqpoint{1.487615in}{1.649030in}}%
\pgfpathlineto{\pgfqpoint{1.489514in}{1.649369in}}%
\pgfpathlineto{\pgfqpoint{1.490734in}{1.649685in}}%
\pgfpathlineto{\pgfqpoint{1.490870in}{1.650041in}}%
\pgfpathlineto{\pgfqpoint{1.492498in}{1.651089in}}%
\pgfpathlineto{\pgfqpoint{1.492633in}{1.650990in}}%
\pgfpathlineto{\pgfqpoint{1.493176in}{1.650572in}}%
\pgfpathlineto{\pgfqpoint{1.493854in}{1.651686in}}%
\pgfpathlineto{\pgfqpoint{1.494803in}{1.653082in}}%
\pgfpathlineto{\pgfqpoint{1.495617in}{1.652349in}}%
\pgfpathlineto{\pgfqpoint{1.495753in}{1.652082in}}%
\pgfpathlineto{\pgfqpoint{1.496702in}{1.653033in}}%
\pgfpathlineto{\pgfqpoint{1.498872in}{1.654958in}}%
\pgfpathlineto{\pgfqpoint{1.500771in}{1.653953in}}%
\pgfpathlineto{\pgfqpoint{1.501856in}{1.652382in}}%
\pgfpathlineto{\pgfqpoint{1.502398in}{1.653091in}}%
\pgfpathlineto{\pgfqpoint{1.503890in}{1.653989in}}%
\pgfpathlineto{\pgfqpoint{1.504161in}{1.653580in}}%
\pgfpathlineto{\pgfqpoint{1.505111in}{1.652957in}}%
\pgfpathlineto{\pgfqpoint{1.505653in}{1.653809in}}%
\pgfpathlineto{\pgfqpoint{1.509993in}{1.657785in}}%
\pgfpathlineto{\pgfqpoint{1.513113in}{1.656973in}}%
\pgfpathlineto{\pgfqpoint{1.516503in}{1.660979in}}%
\pgfpathlineto{\pgfqpoint{1.518266in}{1.660371in}}%
\pgfpathlineto{\pgfqpoint{1.518402in}{1.660605in}}%
\pgfpathlineto{\pgfqpoint{1.519758in}{1.662753in}}%
\pgfpathlineto{\pgfqpoint{1.520708in}{1.661744in}}%
\pgfpathlineto{\pgfqpoint{1.521928in}{1.660524in}}%
\pgfpathlineto{\pgfqpoint{1.522471in}{1.661168in}}%
\pgfpathlineto{\pgfqpoint{1.524505in}{1.662954in}}%
\pgfpathlineto{\pgfqpoint{1.524776in}{1.662433in}}%
\pgfpathlineto{\pgfqpoint{1.526133in}{1.661917in}}%
\pgfpathlineto{\pgfqpoint{1.526404in}{1.662538in}}%
\pgfpathlineto{\pgfqpoint{1.530473in}{1.665647in}}%
\pgfpathlineto{\pgfqpoint{1.530608in}{1.665469in}}%
\pgfpathlineto{\pgfqpoint{1.533728in}{1.665476in}}%
\pgfpathlineto{\pgfqpoint{1.536576in}{1.668390in}}%
\pgfpathlineto{\pgfqpoint{1.536983in}{1.667864in}}%
\pgfpathlineto{\pgfqpoint{1.537932in}{1.668860in}}%
\pgfpathlineto{\pgfqpoint{1.538203in}{1.668321in}}%
\pgfpathlineto{\pgfqpoint{1.540780in}{1.669970in}}%
\pgfpathlineto{\pgfqpoint{1.543628in}{1.671247in}}%
\pgfpathlineto{\pgfqpoint{1.544171in}{1.670341in}}%
\pgfpathlineto{\pgfqpoint{1.559089in}{1.679739in}}%
\pgfpathlineto{\pgfqpoint{1.564514in}{1.680497in}}%
\pgfpathlineto{\pgfqpoint{1.566142in}{1.679307in}}%
\pgfpathlineto{\pgfqpoint{1.566684in}{1.680496in}}%
\pgfpathlineto{\pgfqpoint{1.570618in}{1.684374in}}%
\pgfpathlineto{\pgfqpoint{1.573737in}{1.683915in}}%
\pgfpathlineto{\pgfqpoint{1.577534in}{1.687505in}}%
\pgfpathlineto{\pgfqpoint{1.578484in}{1.686335in}}%
\pgfpathlineto{\pgfqpoint{1.580383in}{1.687030in}}%
\pgfpathlineto{\pgfqpoint{1.583095in}{1.687534in}}%
\pgfpathlineto{\pgfqpoint{1.583231in}{1.687345in}}%
\pgfpathlineto{\pgfqpoint{1.585129in}{1.686047in}}%
\pgfpathlineto{\pgfqpoint{1.586079in}{1.686227in}}%
\pgfpathlineto{\pgfqpoint{1.586621in}{1.685316in}}%
\pgfpathlineto{\pgfqpoint{1.588249in}{1.685476in}}%
\pgfpathlineto{\pgfqpoint{1.588384in}{1.685656in}}%
\pgfpathlineto{\pgfqpoint{1.591775in}{1.688277in}}%
\pgfpathlineto{\pgfqpoint{1.592996in}{1.689239in}}%
\pgfpathlineto{\pgfqpoint{1.594623in}{1.689562in}}%
\pgfpathlineto{\pgfqpoint{1.594894in}{1.689281in}}%
\pgfpathlineto{\pgfqpoint{1.598149in}{1.688816in}}%
\pgfpathlineto{\pgfqpoint{1.600048in}{1.689891in}}%
\pgfpathlineto{\pgfqpoint{1.600184in}{1.689782in}}%
\pgfpathlineto{\pgfqpoint{1.601947in}{1.691168in}}%
\pgfpathlineto{\pgfqpoint{1.603846in}{1.691341in}}%
\pgfpathlineto{\pgfqpoint{1.603981in}{1.691162in}}%
\pgfpathlineto{\pgfqpoint{1.606151in}{1.689657in}}%
\pgfpathlineto{\pgfqpoint{1.606423in}{1.690257in}}%
\pgfpathlineto{\pgfqpoint{1.608593in}{1.693833in}}%
\pgfpathlineto{\pgfqpoint{1.609678in}{1.692925in}}%
\pgfpathlineto{\pgfqpoint{1.610356in}{1.692675in}}%
\pgfpathlineto{\pgfqpoint{1.610763in}{1.693453in}}%
\pgfpathlineto{\pgfqpoint{1.611169in}{1.693321in}}%
\pgfpathlineto{\pgfqpoint{1.612933in}{1.693193in}}%
\pgfpathlineto{\pgfqpoint{1.613068in}{1.693012in}}%
\pgfpathlineto{\pgfqpoint{1.614018in}{1.692636in}}%
\pgfpathlineto{\pgfqpoint{1.614696in}{1.693599in}}%
\pgfpathlineto{\pgfqpoint{1.616594in}{1.693175in}}%
\pgfpathlineto{\pgfqpoint{1.618900in}{1.694456in}}%
\pgfpathlineto{\pgfqpoint{1.621613in}{1.697638in}}%
\pgfpathlineto{\pgfqpoint{1.626088in}{1.698105in}}%
\pgfpathlineto{\pgfqpoint{1.629208in}{1.698065in}}%
\pgfpathlineto{\pgfqpoint{1.631920in}{1.700404in}}%
\pgfpathlineto{\pgfqpoint{1.636803in}{1.700109in}}%
\pgfpathlineto{\pgfqpoint{1.638973in}{1.700534in}}%
\pgfpathlineto{\pgfqpoint{1.641685in}{1.702702in}}%
\pgfpathlineto{\pgfqpoint{1.643313in}{1.703565in}}%
\pgfpathlineto{\pgfqpoint{1.643448in}{1.703471in}}%
\pgfpathlineto{\pgfqpoint{1.645347in}{1.702849in}}%
\pgfpathlineto{\pgfqpoint{1.645483in}{1.702964in}}%
\pgfpathlineto{\pgfqpoint{1.646974in}{1.702914in}}%
\pgfpathlineto{\pgfqpoint{1.647788in}{1.702290in}}%
\pgfpathlineto{\pgfqpoint{1.648602in}{1.702612in}}%
\pgfpathlineto{\pgfqpoint{1.649958in}{1.702832in}}%
\pgfpathlineto{\pgfqpoint{1.650229in}{1.702485in}}%
\pgfpathlineto{\pgfqpoint{1.651721in}{1.703046in}}%
\pgfpathlineto{\pgfqpoint{1.654434in}{1.703766in}}%
\pgfpathlineto{\pgfqpoint{1.656197in}{1.703899in}}%
\pgfpathlineto{\pgfqpoint{1.656739in}{1.704127in}}%
\pgfpathlineto{\pgfqpoint{1.657418in}{1.703273in}}%
\pgfpathlineto{\pgfqpoint{1.659181in}{1.702492in}}%
\pgfpathlineto{\pgfqpoint{1.659316in}{1.702784in}}%
\pgfpathlineto{\pgfqpoint{1.667725in}{1.705717in}}%
\pgfpathlineto{\pgfqpoint{1.669353in}{1.704810in}}%
\pgfpathlineto{\pgfqpoint{1.669624in}{1.705060in}}%
\pgfpathlineto{\pgfqpoint{1.672065in}{1.705982in}}%
\pgfpathlineto{\pgfqpoint{1.674642in}{1.706961in}}%
\pgfpathlineto{\pgfqpoint{1.680203in}{1.707442in}}%
\pgfpathlineto{\pgfqpoint{1.683864in}{1.708224in}}%
\pgfpathlineto{\pgfqpoint{1.686577in}{1.709297in}}%
\pgfpathlineto{\pgfqpoint{1.688340in}{1.709836in}}%
\pgfpathlineto{\pgfqpoint{1.688476in}{1.709641in}}%
\pgfpathlineto{\pgfqpoint{1.689561in}{1.709678in}}%
\pgfpathlineto{\pgfqpoint{1.689968in}{1.710394in}}%
\pgfpathlineto{\pgfqpoint{1.693087in}{1.711124in}}%
\pgfpathlineto{\pgfqpoint{1.696478in}{1.712359in}}%
\pgfpathlineto{\pgfqpoint{1.706921in}{1.713310in}}%
\pgfpathlineto{\pgfqpoint{1.708684in}{1.712512in}}%
\pgfpathlineto{\pgfqpoint{1.713295in}{1.712334in}}%
\pgfpathlineto{\pgfqpoint{1.717093in}{1.714529in}}%
\pgfpathlineto{\pgfqpoint{1.717228in}{1.714392in}}%
\pgfpathlineto{\pgfqpoint{1.722111in}{1.714059in}}%
\pgfpathlineto{\pgfqpoint{1.725637in}{1.714321in}}%
\pgfpathlineto{\pgfqpoint{1.727400in}{1.714220in}}%
\pgfpathlineto{\pgfqpoint{1.729841in}{1.713208in}}%
\pgfpathlineto{\pgfqpoint{1.731062in}{1.713269in}}%
\pgfpathlineto{\pgfqpoint{1.731469in}{1.713766in}}%
\pgfpathlineto{\pgfqpoint{1.738657in}{1.716594in}}%
\pgfpathlineto{\pgfqpoint{1.738793in}{1.716223in}}%
\pgfpathlineto{\pgfqpoint{1.740556in}{1.716721in}}%
\pgfpathlineto{\pgfqpoint{1.742183in}{1.716003in}}%
\pgfpathlineto{\pgfqpoint{1.743811in}{1.716903in}}%
\pgfpathlineto{\pgfqpoint{1.746523in}{1.718350in}}%
\pgfpathlineto{\pgfqpoint{1.749236in}{1.717112in}}%
\pgfpathlineto{\pgfqpoint{1.749643in}{1.717375in}}%
\pgfpathlineto{\pgfqpoint{1.750592in}{1.716665in}}%
\pgfpathlineto{\pgfqpoint{1.752219in}{1.716772in}}%
\pgfpathlineto{\pgfqpoint{1.752355in}{1.716905in}}%
\pgfpathlineto{\pgfqpoint{1.753847in}{1.716238in}}%
\pgfpathlineto{\pgfqpoint{1.755339in}{1.716945in}}%
\pgfpathlineto{\pgfqpoint{1.764561in}{1.717318in}}%
\pgfpathlineto{\pgfqpoint{1.767409in}{1.718053in}}%
\pgfpathlineto{\pgfqpoint{1.772563in}{1.719406in}}%
\pgfpathlineto{\pgfqpoint{1.773377in}{1.719392in}}%
\pgfpathlineto{\pgfqpoint{1.773919in}{1.719912in}}%
\pgfpathlineto{\pgfqpoint{1.776496in}{1.720271in}}%
\pgfpathlineto{\pgfqpoint{1.780701in}{1.719791in}}%
\pgfpathlineto{\pgfqpoint{1.783956in}{1.721238in}}%
\pgfpathlineto{\pgfqpoint{1.784769in}{1.721596in}}%
\pgfpathlineto{\pgfqpoint{1.784905in}{1.722067in}}%
\pgfpathlineto{\pgfqpoint{1.786804in}{1.723290in}}%
\pgfpathlineto{\pgfqpoint{1.788838in}{1.723011in}}%
\pgfpathlineto{\pgfqpoint{1.795484in}{1.724040in}}%
\pgfpathlineto{\pgfqpoint{1.797654in}{1.727044in}}%
\pgfpathlineto{\pgfqpoint{1.805249in}{1.727360in}}%
\pgfpathlineto{\pgfqpoint{1.807690in}{1.727513in}}%
\pgfpathlineto{\pgfqpoint{1.813793in}{1.727784in}}%
\pgfpathlineto{\pgfqpoint{1.821117in}{1.730123in}}%
\pgfpathlineto{\pgfqpoint{1.823829in}{1.729150in}}%
\pgfpathlineto{\pgfqpoint{1.847971in}{1.730752in}}%
\pgfpathlineto{\pgfqpoint{1.849734in}{1.730607in}}%
\pgfpathlineto{\pgfqpoint{1.867772in}{1.728983in}}%
\pgfpathlineto{\pgfqpoint{1.869942in}{1.728520in}}%
\pgfpathlineto{\pgfqpoint{1.872519in}{1.728790in}}%
\pgfpathlineto{\pgfqpoint{1.874282in}{1.728871in}}%
\pgfpathlineto{\pgfqpoint{1.874418in}{1.728757in}}%
\pgfpathlineto{\pgfqpoint{1.900051in}{1.728763in}}%
\pgfpathlineto{\pgfqpoint{1.902492in}{1.729924in}}%
\pgfpathlineto{\pgfqpoint{1.903984in}{1.728932in}}%
\pgfpathlineto{\pgfqpoint{1.904526in}{1.729586in}}%
\pgfpathlineto{\pgfqpoint{1.906154in}{1.729104in}}%
\pgfpathlineto{\pgfqpoint{1.909138in}{1.728606in}}%
\pgfpathlineto{\pgfqpoint{1.911579in}{1.730305in}}%
\pgfpathlineto{\pgfqpoint{1.911714in}{1.730190in}}%
\pgfpathlineto{\pgfqpoint{1.916597in}{1.728533in}}%
\pgfpathlineto{\pgfqpoint{1.916868in}{1.729038in}}%
\pgfpathlineto{\pgfqpoint{1.918903in}{1.730196in}}%
\pgfpathlineto{\pgfqpoint{1.919038in}{1.730100in}}%
\pgfpathlineto{\pgfqpoint{1.921479in}{1.729882in}}%
\pgfpathlineto{\pgfqpoint{1.924870in}{1.729660in}}%
\pgfpathlineto{\pgfqpoint{1.928532in}{1.728808in}}%
\pgfpathlineto{\pgfqpoint{1.930838in}{1.728310in}}%
\pgfpathlineto{\pgfqpoint{1.935720in}{1.728525in}}%
\pgfpathlineto{\pgfqpoint{1.937076in}{1.728441in}}%
\pgfpathlineto{\pgfqpoint{1.937483in}{1.728954in}}%
\pgfpathlineto{\pgfqpoint{1.940467in}{1.729351in}}%
\pgfpathlineto{\pgfqpoint{1.942773in}{1.730070in}}%
\pgfpathlineto{\pgfqpoint{1.945078in}{1.729988in}}%
\pgfpathlineto{\pgfqpoint{1.949283in}{1.728074in}}%
\pgfpathlineto{\pgfqpoint{1.957420in}{1.727626in}}%
\pgfpathlineto{\pgfqpoint{1.967185in}{1.727641in}}%
\pgfpathlineto{\pgfqpoint{1.969491in}{1.727431in}}%
\pgfpathlineto{\pgfqpoint{1.973695in}{1.727824in}}%
\pgfpathlineto{\pgfqpoint{1.977221in}{1.727016in}}%
\pgfpathlineto{\pgfqpoint{1.980883in}{1.727618in}}%
\pgfpathlineto{\pgfqpoint{1.984409in}{1.727812in}}%
\pgfpathlineto{\pgfqpoint{1.997565in}{1.727914in}}%
\pgfpathlineto{\pgfqpoint{2.000142in}{1.725724in}}%
\pgfpathlineto{\pgfqpoint{2.000278in}{1.725914in}}%
\pgfpathlineto{\pgfqpoint{2.003939in}{1.727061in}}%
\pgfpathlineto{\pgfqpoint{2.006245in}{1.726680in}}%
\pgfpathlineto{\pgfqpoint{2.010721in}{1.724550in}}%
\pgfpathlineto{\pgfqpoint{2.011941in}{1.725792in}}%
\pgfpathlineto{\pgfqpoint{2.012619in}{1.725123in}}%
\pgfpathlineto{\pgfqpoint{2.014518in}{1.724055in}}%
\pgfpathlineto{\pgfqpoint{2.014654in}{1.724159in}}%
\pgfpathlineto{\pgfqpoint{2.017231in}{1.724433in}}%
\pgfpathlineto{\pgfqpoint{2.020757in}{1.724623in}}%
\pgfpathlineto{\pgfqpoint{2.021706in}{1.724593in}}%
\pgfpathlineto{\pgfqpoint{2.022249in}{1.724169in}}%
\pgfpathlineto{\pgfqpoint{2.025368in}{1.724042in}}%
\pgfpathlineto{\pgfqpoint{2.027538in}{1.724667in}}%
\pgfpathlineto{\pgfqpoint{2.041643in}{1.721910in}}%
\pgfpathlineto{\pgfqpoint{2.044084in}{1.723765in}}%
\pgfpathlineto{\pgfqpoint{2.047204in}{1.723822in}}%
\pgfpathlineto{\pgfqpoint{2.049509in}{1.723469in}}%
\pgfpathlineto{\pgfqpoint{2.052086in}{1.722785in}}%
\pgfpathlineto{\pgfqpoint{2.053849in}{1.721713in}}%
\pgfpathlineto{\pgfqpoint{2.054121in}{1.721883in}}%
\pgfpathlineto{\pgfqpoint{2.058732in}{1.722061in}}%
\pgfpathlineto{\pgfqpoint{2.060359in}{1.721927in}}%
\pgfpathlineto{\pgfqpoint{2.060495in}{1.721644in}}%
\pgfpathlineto{\pgfqpoint{2.061716in}{1.722416in}}%
\pgfpathlineto{\pgfqpoint{2.064021in}{1.722511in}}%
\pgfpathlineto{\pgfqpoint{2.066056in}{1.721549in}}%
\pgfpathlineto{\pgfqpoint{2.066598in}{1.722698in}}%
\pgfpathlineto{\pgfqpoint{2.068904in}{1.724356in}}%
\pgfpathlineto{\pgfqpoint{2.079211in}{1.723223in}}%
\pgfpathlineto{\pgfqpoint{2.080161in}{1.723847in}}%
\pgfpathlineto{\pgfqpoint{2.080974in}{1.723268in}}%
\pgfpathlineto{\pgfqpoint{2.083009in}{1.722463in}}%
\pgfpathlineto{\pgfqpoint{2.089112in}{1.723862in}}%
\pgfpathlineto{\pgfqpoint{2.091960in}{1.723724in}}%
\pgfpathlineto{\pgfqpoint{2.093316in}{1.724231in}}%
\pgfpathlineto{\pgfqpoint{2.093588in}{1.723958in}}%
\pgfpathlineto{\pgfqpoint{2.100640in}{1.721298in}}%
\pgfpathlineto{\pgfqpoint{2.104031in}{1.720369in}}%
\pgfpathlineto{\pgfqpoint{2.104573in}{1.720270in}}%
\pgfpathlineto{\pgfqpoint{2.105251in}{1.721069in}}%
\pgfpathlineto{\pgfqpoint{2.106336in}{1.721877in}}%
\pgfpathlineto{\pgfqpoint{2.107150in}{1.721227in}}%
\pgfpathlineto{\pgfqpoint{2.109456in}{1.721193in}}%
\pgfpathlineto{\pgfqpoint{2.112846in}{1.720333in}}%
\pgfpathlineto{\pgfqpoint{2.114474in}{1.719370in}}%
\pgfpathlineto{\pgfqpoint{2.114745in}{1.719804in}}%
\pgfpathlineto{\pgfqpoint{2.115288in}{1.720437in}}%
\pgfpathlineto{\pgfqpoint{2.116237in}{1.719621in}}%
\pgfpathlineto{\pgfqpoint{2.118271in}{1.719185in}}%
\pgfpathlineto{\pgfqpoint{2.124917in}{1.717035in}}%
\pgfpathlineto{\pgfqpoint{2.126680in}{1.716620in}}%
\pgfpathlineto{\pgfqpoint{2.133326in}{1.715408in}}%
\pgfpathlineto{\pgfqpoint{2.134411in}{1.713148in}}%
\pgfpathlineto{\pgfqpoint{2.135224in}{1.714168in}}%
\pgfpathlineto{\pgfqpoint{2.136445in}{1.715058in}}%
\pgfpathlineto{\pgfqpoint{2.141599in}{1.715436in}}%
\pgfpathlineto{\pgfqpoint{2.147566in}{1.713737in}}%
\pgfpathlineto{\pgfqpoint{2.149736in}{1.714904in}}%
\pgfpathlineto{\pgfqpoint{2.149872in}{1.714760in}}%
\pgfpathlineto{\pgfqpoint{2.153127in}{1.714646in}}%
\pgfpathlineto{\pgfqpoint{2.153805in}{1.714948in}}%
\pgfpathlineto{\pgfqpoint{2.154483in}{1.714100in}}%
\pgfpathlineto{\pgfqpoint{2.155026in}{1.713563in}}%
\pgfpathlineto{\pgfqpoint{2.155297in}{1.712795in}}%
\pgfpathlineto{\pgfqpoint{2.156382in}{1.711300in}}%
\pgfpathlineto{\pgfqpoint{2.156924in}{1.712202in}}%
\pgfpathlineto{\pgfqpoint{2.159230in}{1.713456in}}%
\pgfpathlineto{\pgfqpoint{2.159501in}{1.713192in}}%
\pgfpathlineto{\pgfqpoint{2.161671in}{1.712700in}}%
\pgfpathlineto{\pgfqpoint{2.166011in}{1.712825in}}%
\pgfpathlineto{\pgfqpoint{2.168046in}{1.712743in}}%
\pgfpathlineto{\pgfqpoint{2.170894in}{1.713197in}}%
\pgfpathlineto{\pgfqpoint{2.177404in}{1.712505in}}%
\pgfpathlineto{\pgfqpoint{2.179167in}{1.712181in}}%
\pgfpathlineto{\pgfqpoint{2.180659in}{1.712866in}}%
\pgfpathlineto{\pgfqpoint{2.185406in}{1.712475in}}%
\pgfpathlineto{\pgfqpoint{2.185813in}{1.711599in}}%
\pgfpathlineto{\pgfqpoint{2.186898in}{1.712612in}}%
\pgfpathlineto{\pgfqpoint{2.190017in}{1.712957in}}%
\pgfpathlineto{\pgfqpoint{2.194764in}{1.712943in}}%
\pgfpathlineto{\pgfqpoint{2.195984in}{1.712355in}}%
\pgfpathlineto{\pgfqpoint{2.196391in}{1.713048in}}%
\pgfpathlineto{\pgfqpoint{2.198426in}{1.713944in}}%
\pgfpathlineto{\pgfqpoint{2.198561in}{1.713770in}}%
\pgfpathlineto{\pgfqpoint{2.200324in}{1.713093in}}%
\pgfpathlineto{\pgfqpoint{2.206699in}{1.713983in}}%
\pgfpathlineto{\pgfqpoint{2.209411in}{1.714183in}}%
\pgfpathlineto{\pgfqpoint{2.214158in}{1.713322in}}%
\pgfpathlineto{\pgfqpoint{2.215786in}{1.713814in}}%
\pgfpathlineto{\pgfqpoint{2.217549in}{1.714105in}}%
\pgfpathlineto{\pgfqpoint{2.219583in}{1.713789in}}%
\pgfpathlineto{\pgfqpoint{2.224466in}{1.714579in}}%
\pgfpathlineto{\pgfqpoint{2.226500in}{1.713455in}}%
\pgfpathlineto{\pgfqpoint{2.228806in}{1.713404in}}%
\pgfpathlineto{\pgfqpoint{2.235858in}{1.713910in}}%
\pgfpathlineto{\pgfqpoint{2.238299in}{1.713488in}}%
\pgfpathlineto{\pgfqpoint{2.242368in}{1.714391in}}%
\pgfpathlineto{\pgfqpoint{2.244674in}{1.714177in}}%
\pgfpathlineto{\pgfqpoint{2.246573in}{1.714095in}}%
\pgfpathlineto{\pgfqpoint{2.248200in}{1.715097in}}%
\pgfpathlineto{\pgfqpoint{2.254439in}{1.715735in}}%
\pgfpathlineto{\pgfqpoint{2.255524in}{1.715862in}}%
\pgfpathlineto{\pgfqpoint{2.256066in}{1.715274in}}%
\pgfpathlineto{\pgfqpoint{2.259864in}{1.715078in}}%
\pgfpathlineto{\pgfqpoint{2.291600in}{1.717535in}}%
\pgfpathlineto{\pgfqpoint{2.293228in}{1.717883in}}%
\pgfpathlineto{\pgfqpoint{2.293499in}{1.717437in}}%
\pgfpathlineto{\pgfqpoint{2.297296in}{1.715900in}}%
\pgfpathlineto{\pgfqpoint{2.300009in}{1.716757in}}%
\pgfpathlineto{\pgfqpoint{2.301229in}{1.716634in}}%
\pgfpathlineto{\pgfqpoint{2.301636in}{1.717140in}}%
\pgfpathlineto{\pgfqpoint{2.304078in}{1.717292in}}%
\pgfpathlineto{\pgfqpoint{2.305841in}{1.717179in}}%
\pgfpathlineto{\pgfqpoint{2.305976in}{1.717350in}}%
\pgfpathlineto{\pgfqpoint{2.307875in}{1.717425in}}%
\pgfpathlineto{\pgfqpoint{2.309638in}{1.717580in}}%
\pgfpathlineto{\pgfqpoint{2.314385in}{1.717761in}}%
\pgfpathlineto{\pgfqpoint{2.316691in}{1.718509in}}%
\pgfpathlineto{\pgfqpoint{2.316826in}{1.718307in}}%
\pgfpathlineto{\pgfqpoint{2.326863in}{1.719349in}}%
\pgfpathlineto{\pgfqpoint{2.329168in}{1.720028in}}%
\pgfpathlineto{\pgfqpoint{2.330118in}{1.719267in}}%
\pgfpathlineto{\pgfqpoint{2.330253in}{1.718988in}}%
\pgfpathlineto{\pgfqpoint{2.332423in}{1.719418in}}%
\pgfpathlineto{\pgfqpoint{2.335271in}{1.719634in}}%
\pgfpathlineto{\pgfqpoint{2.336899in}{1.720499in}}%
\pgfpathlineto{\pgfqpoint{2.339340in}{1.720382in}}%
\pgfpathlineto{\pgfqpoint{2.341103in}{1.721141in}}%
\pgfpathlineto{\pgfqpoint{2.342459in}{1.721218in}}%
\pgfpathlineto{\pgfqpoint{2.342731in}{1.721087in}}%
\pgfpathlineto{\pgfqpoint{2.349241in}{1.721083in}}%
\pgfpathlineto{\pgfqpoint{2.353988in}{1.718939in}}%
\pgfpathlineto{\pgfqpoint{2.356700in}{1.718305in}}%
\pgfpathlineto{\pgfqpoint{2.358463in}{1.718307in}}%
\pgfpathlineto{\pgfqpoint{2.364295in}{1.718239in}}%
\pgfpathlineto{\pgfqpoint{2.368635in}{1.717755in}}%
\pgfpathlineto{\pgfqpoint{2.370805in}{1.717606in}}%
\pgfpathlineto{\pgfqpoint{2.373653in}{1.717557in}}%
\pgfpathlineto{\pgfqpoint{2.375416in}{1.717025in}}%
\pgfpathlineto{\pgfqpoint{2.380028in}{1.717924in}}%
\pgfpathlineto{\pgfqpoint{2.381791in}{1.717454in}}%
\pgfpathlineto{\pgfqpoint{2.383961in}{1.717797in}}%
\pgfpathlineto{\pgfqpoint{2.384096in}{1.718130in}}%
\pgfpathlineto{\pgfqpoint{2.385453in}{1.718624in}}%
\pgfpathlineto{\pgfqpoint{2.385859in}{1.718268in}}%
\pgfpathlineto{\pgfqpoint{2.386944in}{1.717826in}}%
\pgfpathlineto{\pgfqpoint{2.387487in}{1.718204in}}%
\pgfpathlineto{\pgfqpoint{2.389386in}{1.719123in}}%
\pgfpathlineto{\pgfqpoint{2.389521in}{1.718947in}}%
\pgfpathlineto{\pgfqpoint{2.392505in}{1.717830in}}%
\pgfpathlineto{\pgfqpoint{2.398608in}{1.718673in}}%
\pgfpathlineto{\pgfqpoint{2.406339in}{1.718343in}}%
\pgfpathlineto{\pgfqpoint{2.408509in}{1.718243in}}%
\pgfpathlineto{\pgfqpoint{2.408644in}{1.718518in}}%
\pgfpathlineto{\pgfqpoint{2.410272in}{1.719199in}}%
\pgfpathlineto{\pgfqpoint{2.410679in}{1.718866in}}%
\pgfpathlineto{\pgfqpoint{2.415426in}{1.718158in}}%
\pgfpathlineto{\pgfqpoint{2.417189in}{1.718900in}}%
\pgfpathlineto{\pgfqpoint{2.420851in}{1.719244in}}%
\pgfpathlineto{\pgfqpoint{2.423021in}{1.719789in}}%
\pgfpathlineto{\pgfqpoint{2.428581in}{1.719177in}}%
\pgfpathlineto{\pgfqpoint{2.436176in}{1.719477in}}%
\pgfpathlineto{\pgfqpoint{2.439703in}{1.719205in}}%
\pgfpathlineto{\pgfqpoint{2.445670in}{1.718318in}}%
\pgfpathlineto{\pgfqpoint{2.447298in}{1.718525in}}%
\pgfpathlineto{\pgfqpoint{2.449603in}{1.717763in}}%
\pgfpathlineto{\pgfqpoint{2.449874in}{1.718143in}}%
\pgfpathlineto{\pgfqpoint{2.455978in}{1.718741in}}%
\pgfpathlineto{\pgfqpoint{2.458148in}{1.718543in}}%
\pgfpathlineto{\pgfqpoint{2.461538in}{1.720538in}}%
\pgfpathlineto{\pgfqpoint{2.469133in}{1.720588in}}%
\pgfpathlineto{\pgfqpoint{2.471032in}{1.720860in}}%
\pgfpathlineto{\pgfqpoint{2.472659in}{1.720058in}}%
\pgfpathlineto{\pgfqpoint{2.475236in}{1.720205in}}%
\pgfpathlineto{\pgfqpoint{2.478220in}{1.720536in}}%
\pgfpathlineto{\pgfqpoint{2.481068in}{1.719529in}}%
\pgfpathlineto{\pgfqpoint{2.485815in}{1.720325in}}%
\pgfpathlineto{\pgfqpoint{2.488799in}{1.720371in}}%
\pgfpathlineto{\pgfqpoint{2.495851in}{1.720751in}}%
\pgfpathlineto{\pgfqpoint{2.497614in}{1.718640in}}%
\pgfpathlineto{\pgfqpoint{2.498021in}{1.719042in}}%
\pgfpathlineto{\pgfqpoint{2.500869in}{1.721143in}}%
\pgfpathlineto{\pgfqpoint{2.501141in}{1.720935in}}%
\pgfpathlineto{\pgfqpoint{2.503175in}{1.720500in}}%
\pgfpathlineto{\pgfqpoint{2.503311in}{1.720635in}}%
\pgfpathlineto{\pgfqpoint{2.504531in}{1.720160in}}%
\pgfpathlineto{\pgfqpoint{2.504667in}{1.719917in}}%
\pgfpathlineto{\pgfqpoint{2.507651in}{1.720253in}}%
\pgfpathlineto{\pgfqpoint{2.509685in}{1.720999in}}%
\pgfpathlineto{\pgfqpoint{2.511448in}{1.720519in}}%
\pgfpathlineto{\pgfqpoint{2.514839in}{1.720764in}}%
\pgfpathlineto{\pgfqpoint{2.519450in}{1.721789in}}%
\pgfpathlineto{\pgfqpoint{2.521756in}{1.720390in}}%
\pgfpathlineto{\pgfqpoint{2.523790in}{1.720109in}}%
\pgfpathlineto{\pgfqpoint{2.566241in}{1.719312in}}%
\pgfpathlineto{\pgfqpoint{2.567190in}{1.719843in}}%
\pgfpathlineto{\pgfqpoint{2.568004in}{1.719332in}}%
\pgfpathlineto{\pgfqpoint{2.571666in}{1.719310in}}%
\pgfpathlineto{\pgfqpoint{2.576141in}{1.720242in}}%
\pgfpathlineto{\pgfqpoint{2.577633in}{1.720194in}}%
\pgfpathlineto{\pgfqpoint{2.577904in}{1.719877in}}%
\pgfpathlineto{\pgfqpoint{2.580210in}{1.718179in}}%
\pgfpathlineto{\pgfqpoint{2.582516in}{1.718496in}}%
\pgfpathlineto{\pgfqpoint{2.585906in}{1.719693in}}%
\pgfpathlineto{\pgfqpoint{2.588754in}{1.720141in}}%
\pgfpathlineto{\pgfqpoint{2.591467in}{1.718588in}}%
\pgfpathlineto{\pgfqpoint{2.594044in}{1.719281in}}%
\pgfpathlineto{\pgfqpoint{2.599876in}{1.719730in}}%
\pgfpathlineto{\pgfqpoint{2.601910in}{1.719626in}}%
\pgfpathlineto{\pgfqpoint{2.604758in}{1.718304in}}%
\pgfpathlineto{\pgfqpoint{2.608013in}{1.718025in}}%
\pgfpathlineto{\pgfqpoint{2.610590in}{1.718444in}}%
\pgfpathlineto{\pgfqpoint{2.613438in}{1.719107in}}%
\pgfpathlineto{\pgfqpoint{2.616829in}{1.718974in}}%
\pgfpathlineto{\pgfqpoint{2.618999in}{1.718783in}}%
\pgfpathlineto{\pgfqpoint{2.636088in}{1.717236in}}%
\pgfpathlineto{\pgfqpoint{2.636494in}{1.717982in}}%
\pgfpathlineto{\pgfqpoint{2.637579in}{1.717315in}}%
\pgfpathlineto{\pgfqpoint{2.640970in}{1.716117in}}%
\pgfpathlineto{\pgfqpoint{2.647751in}{1.715448in}}%
\pgfpathlineto{\pgfqpoint{2.649786in}{1.715047in}}%
\pgfpathlineto{\pgfqpoint{2.653312in}{1.715263in}}%
\pgfpathlineto{\pgfqpoint{2.653719in}{1.715975in}}%
\pgfpathlineto{\pgfqpoint{2.654939in}{1.715426in}}%
\pgfpathlineto{\pgfqpoint{2.656974in}{1.716169in}}%
\pgfpathlineto{\pgfqpoint{2.658330in}{1.716567in}}%
\pgfpathlineto{\pgfqpoint{2.658873in}{1.716035in}}%
\pgfpathlineto{\pgfqpoint{2.688168in}{1.720858in}}%
\pgfpathlineto{\pgfqpoint{2.689795in}{1.721466in}}%
\pgfpathlineto{\pgfqpoint{2.690202in}{1.720968in}}%
\pgfpathlineto{\pgfqpoint{2.692914in}{1.719853in}}%
\pgfpathlineto{\pgfqpoint{2.694542in}{1.720006in}}%
\pgfpathlineto{\pgfqpoint{2.694678in}{1.719766in}}%
\pgfpathlineto{\pgfqpoint{2.696441in}{1.720218in}}%
\pgfpathlineto{\pgfqpoint{2.696712in}{1.719792in}}%
\pgfpathlineto{\pgfqpoint{2.700103in}{1.719614in}}%
\pgfpathlineto{\pgfqpoint{2.701866in}{1.720534in}}%
\pgfpathlineto{\pgfqpoint{2.702273in}{1.720665in}}%
\pgfpathlineto{\pgfqpoint{2.702951in}{1.719668in}}%
\pgfpathlineto{\pgfqpoint{2.703086in}{1.719688in}}%
\pgfpathlineto{\pgfqpoint{2.704985in}{1.718283in}}%
\pgfpathlineto{\pgfqpoint{2.705121in}{1.718368in}}%
\pgfpathlineto{\pgfqpoint{2.713123in}{1.721018in}}%
\pgfpathlineto{\pgfqpoint{2.714886in}{1.720501in}}%
\pgfpathlineto{\pgfqpoint{2.715021in}{1.720704in}}%
\pgfpathlineto{\pgfqpoint{2.716649in}{1.722198in}}%
\pgfpathlineto{\pgfqpoint{2.726956in}{1.723935in}}%
\pgfpathlineto{\pgfqpoint{2.730754in}{1.725000in}}%
\pgfpathlineto{\pgfqpoint{2.733873in}{1.724278in}}%
\pgfpathlineto{\pgfqpoint{2.735636in}{1.724615in}}%
\pgfpathlineto{\pgfqpoint{2.735772in}{1.724364in}}%
\pgfpathlineto{\pgfqpoint{2.739027in}{1.724135in}}%
\pgfpathlineto{\pgfqpoint{2.747571in}{1.723816in}}%
\pgfpathlineto{\pgfqpoint{2.749741in}{1.723969in}}%
\pgfpathlineto{\pgfqpoint{2.752454in}{1.724585in}}%
\pgfpathlineto{\pgfqpoint{2.752589in}{1.724346in}}%
\pgfpathlineto{\pgfqpoint{2.755166in}{1.723120in}}%
\pgfpathlineto{\pgfqpoint{2.755980in}{1.723630in}}%
\pgfpathlineto{\pgfqpoint{2.756794in}{1.723077in}}%
\pgfpathlineto{\pgfqpoint{2.758286in}{1.723394in}}%
\pgfpathlineto{\pgfqpoint{2.761134in}{1.723638in}}%
\pgfpathlineto{\pgfqpoint{2.769271in}{1.723581in}}%
\pgfpathlineto{\pgfqpoint{2.772255in}{1.723392in}}%
\pgfpathlineto{\pgfqpoint{2.776053in}{1.723802in}}%
\pgfpathlineto{\pgfqpoint{2.779172in}{1.723839in}}%
\pgfpathlineto{\pgfqpoint{2.781206in}{1.724600in}}%
\pgfpathlineto{\pgfqpoint{2.787581in}{1.726190in}}%
\pgfpathlineto{\pgfqpoint{2.788801in}{1.727099in}}%
\pgfpathlineto{\pgfqpoint{2.789208in}{1.726713in}}%
\pgfpathlineto{\pgfqpoint{2.798838in}{1.725688in}}%
\pgfpathlineto{\pgfqpoint{2.802635in}{1.726799in}}%
\pgfpathlineto{\pgfqpoint{2.805076in}{1.725174in}}%
\pgfpathlineto{\pgfqpoint{2.805212in}{1.725349in}}%
\pgfpathlineto{\pgfqpoint{2.807789in}{1.725288in}}%
\pgfpathlineto{\pgfqpoint{2.809959in}{1.725822in}}%
\pgfpathlineto{\pgfqpoint{2.812671in}{1.725787in}}%
\pgfpathlineto{\pgfqpoint{2.815113in}{1.725402in}}%
\pgfpathlineto{\pgfqpoint{2.815926in}{1.725097in}}%
\pgfpathlineto{\pgfqpoint{2.816198in}{1.724538in}}%
\pgfpathlineto{\pgfqpoint{2.819588in}{1.722728in}}%
\pgfpathlineto{\pgfqpoint{2.819724in}{1.722868in}}%
\pgfpathlineto{\pgfqpoint{2.823386in}{1.724975in}}%
\pgfpathlineto{\pgfqpoint{2.826098in}{1.723345in}}%
\pgfpathlineto{\pgfqpoint{2.827590in}{1.723001in}}%
\pgfpathlineto{\pgfqpoint{2.827726in}{1.722780in}}%
\pgfpathlineto{\pgfqpoint{2.832744in}{1.724169in}}%
\pgfpathlineto{\pgfqpoint{2.841831in}{1.723519in}}%
\pgfpathlineto{\pgfqpoint{2.843594in}{1.723405in}}%
\pgfpathlineto{\pgfqpoint{2.847391in}{1.722595in}}%
\pgfpathlineto{\pgfqpoint{2.847527in}{1.722898in}}%
\pgfpathlineto{\pgfqpoint{2.858513in}{1.723984in}}%
\pgfpathlineto{\pgfqpoint{2.861361in}{1.723452in}}%
\pgfpathlineto{\pgfqpoint{2.861496in}{1.723709in}}%
\pgfpathlineto{\pgfqpoint{2.862988in}{1.724697in}}%
\pgfpathlineto{\pgfqpoint{2.863259in}{1.724503in}}%
\pgfpathlineto{\pgfqpoint{2.866921in}{1.723588in}}%
\pgfpathlineto{\pgfqpoint{2.872211in}{1.724810in}}%
\pgfpathlineto{\pgfqpoint{2.875194in}{1.723496in}}%
\pgfpathlineto{\pgfqpoint{2.877093in}{1.723719in}}%
\pgfpathlineto{\pgfqpoint{2.880213in}{1.722565in}}%
\pgfpathlineto{\pgfqpoint{2.881976in}{1.722122in}}%
\pgfpathlineto{\pgfqpoint{2.883196in}{1.721525in}}%
\pgfpathlineto{\pgfqpoint{2.883603in}{1.722067in}}%
\pgfpathlineto{\pgfqpoint{2.885095in}{1.722027in}}%
\pgfpathlineto{\pgfqpoint{2.885231in}{1.721889in}}%
\pgfpathlineto{\pgfqpoint{2.886723in}{1.720196in}}%
\pgfpathlineto{\pgfqpoint{2.887265in}{1.720739in}}%
\pgfpathlineto{\pgfqpoint{2.890113in}{1.720855in}}%
\pgfpathlineto{\pgfqpoint{2.892961in}{1.721489in}}%
\pgfpathlineto{\pgfqpoint{2.894318in}{1.721490in}}%
\pgfpathlineto{\pgfqpoint{2.894453in}{1.721207in}}%
\pgfpathlineto{\pgfqpoint{2.898251in}{1.719276in}}%
\pgfpathlineto{\pgfqpoint{2.900692in}{1.720301in}}%
\pgfpathlineto{\pgfqpoint{2.902726in}{1.719709in}}%
\pgfpathlineto{\pgfqpoint{2.902862in}{1.719852in}}%
\pgfpathlineto{\pgfqpoint{2.905032in}{1.721665in}}%
\pgfpathlineto{\pgfqpoint{2.905032in}{1.721665in}}%
\pgfusepath{stroke}%
\end{pgfscope}%
\begin{pgfscope}%
\pgfsetrectcap%
\pgfsetmiterjoin%
\pgfsetlinewidth{0.803000pt}%
\definecolor{currentstroke}{rgb}{0.000000,0.000000,0.000000}%
\pgfsetstrokecolor{currentstroke}%
\pgfsetdash{}{0pt}%
\pgfpathmoveto{\pgfqpoint{0.735032in}{0.526079in}}%
\pgfpathlineto{\pgfqpoint{0.735032in}{2.187079in}}%
\pgfusepath{stroke}%
\end{pgfscope}%
\begin{pgfscope}%
\pgfsetrectcap%
\pgfsetmiterjoin%
\pgfsetlinewidth{0.803000pt}%
\definecolor{currentstroke}{rgb}{0.000000,0.000000,0.000000}%
\pgfsetstrokecolor{currentstroke}%
\pgfsetdash{}{0pt}%
\pgfpathmoveto{\pgfqpoint{2.905032in}{0.526079in}}%
\pgfpathlineto{\pgfqpoint{2.905032in}{2.187079in}}%
\pgfusepath{stroke}%
\end{pgfscope}%
\begin{pgfscope}%
\pgfsetrectcap%
\pgfsetmiterjoin%
\pgfsetlinewidth{0.803000pt}%
\definecolor{currentstroke}{rgb}{0.000000,0.000000,0.000000}%
\pgfsetstrokecolor{currentstroke}%
\pgfsetdash{}{0pt}%
\pgfpathmoveto{\pgfqpoint{0.735032in}{0.526079in}}%
\pgfpathlineto{\pgfqpoint{2.905032in}{0.526079in}}%
\pgfusepath{stroke}%
\end{pgfscope}%
\begin{pgfscope}%
\pgfsetrectcap%
\pgfsetmiterjoin%
\pgfsetlinewidth{0.803000pt}%
\definecolor{currentstroke}{rgb}{0.000000,0.000000,0.000000}%
\pgfsetstrokecolor{currentstroke}%
\pgfsetdash{}{0pt}%
\pgfpathmoveto{\pgfqpoint{0.735032in}{2.187079in}}%
\pgfpathlineto{\pgfqpoint{2.905032in}{2.187079in}}%
\pgfusepath{stroke}%
\end{pgfscope}%
\begin{pgfscope}%
\pgfsetbuttcap%
\pgfsetmiterjoin%
\definecolor{currentfill}{rgb}{1.000000,1.000000,1.000000}%
\pgfsetfillcolor{currentfill}%
\pgfsetfillopacity{0.800000}%
\pgfsetlinewidth{1.003750pt}%
\definecolor{currentstroke}{rgb}{0.800000,0.800000,0.800000}%
\pgfsetstrokecolor{currentstroke}%
\pgfsetstrokeopacity{0.800000}%
\pgfsetdash{}{0pt}%
\pgfpathmoveto{\pgfqpoint{3.050674in}{1.462430in}}%
\pgfpathlineto{\pgfqpoint{5.146390in}{1.462430in}}%
\pgfpathquadraticcurveto{\pgfqpoint{5.174168in}{1.462430in}}{\pgfqpoint{5.174168in}{1.490208in}}%
\pgfpathlineto{\pgfqpoint{5.174168in}{2.089857in}}%
\pgfpathquadraticcurveto{\pgfqpoint{5.174168in}{2.117635in}}{\pgfqpoint{5.146390in}{2.117635in}}%
\pgfpathlineto{\pgfqpoint{3.050674in}{2.117635in}}%
\pgfpathquadraticcurveto{\pgfqpoint{3.022896in}{2.117635in}}{\pgfqpoint{3.022896in}{2.089857in}}%
\pgfpathlineto{\pgfqpoint{3.022896in}{1.490208in}}%
\pgfpathquadraticcurveto{\pgfqpoint{3.022896in}{1.462430in}}{\pgfqpoint{3.050674in}{1.462430in}}%
\pgfpathclose%
\pgfusepath{stroke,fill}%
\end{pgfscope}%
\begin{pgfscope}%
\pgfsetrectcap%
\pgfsetroundjoin%
\pgfsetlinewidth{1.003750pt}%
\definecolor{currentstroke}{rgb}{0.501961,0.000000,0.501961}%
\pgfsetstrokecolor{currentstroke}%
\pgfsetdash{}{0pt}%
\pgfpathmoveto{\pgfqpoint{3.078451in}{2.005167in}}%
\pgfpathlineto{\pgfqpoint{3.356229in}{2.005167in}}%
\pgfusepath{stroke}%
\end{pgfscope}%
\begin{pgfscope}%
\pgftext[x=3.467340in,y=1.956556in,left,base]{\rmfamily\fontsize{10.000000}{12.000000}\selectfont Standard}%
\end{pgfscope}%
\begin{pgfscope}%
\pgfsetrectcap%
\pgfsetroundjoin%
\pgfsetlinewidth{1.003750pt}%
\definecolor{currentstroke}{rgb}{0.627451,0.321569,0.176471}%
\pgfsetstrokecolor{currentstroke}%
\pgfsetdash{}{0pt}%
\pgfpathmoveto{\pgfqpoint{3.078451in}{1.801310in}}%
\pgfpathlineto{\pgfqpoint{3.356229in}{1.801310in}}%
\pgfusepath{stroke}%
\end{pgfscope}%
\begin{pgfscope}%
\pgftext[x=3.467340in,y=1.752699in,left,base]{\rmfamily\fontsize{10.000000}{12.000000}\selectfont Geometric (Lie-Trotter)}%
\end{pgfscope}%
\begin{pgfscope}%
\pgfsetrectcap%
\pgfsetroundjoin%
\pgfsetlinewidth{1.003750pt}%
\definecolor{currentstroke}{rgb}{1.000000,0.549020,0.000000}%
\pgfsetstrokecolor{currentstroke}%
\pgfsetdash{}{0pt}%
\pgfpathmoveto{\pgfqpoint{3.078451in}{1.597453in}}%
\pgfpathlineto{\pgfqpoint{3.356229in}{1.597453in}}%
\pgfusepath{stroke}%
\end{pgfscope}%
\begin{pgfscope}%
\pgftext[x=3.467340in,y=1.548842in,left,base]{\rmfamily\fontsize{10.000000}{12.000000}\selectfont Geometric (Strang)}%
\end{pgfscope}%
\end{pgfpicture}%
\makeatother%
\endgroup%

\caption{Run 1 with parameters listed in Tab. \ref{tab_parameters}: Time evolution of the relative error in the conservation of energy for three cases: Standard finite element PIC (purple), structure-preserving finite element PIC with Lie-Trotter splitting (brown) and Strang splitting (orange).\label{fig_comparison}}
\end{figure}

\begin{figure}[!b]
\centering
\includegraphics[scale=1]{01_Figures/Spectra_1e5.pdf}
%%% Creator: Matplotlib, PGF backend
%%
%% To include the figure in your LaTeX document, write
%%   \input{<filename>.pgf}
%%
%% Make sure the required packages are loaded in your preamble
%%   \usepackage{pgf}
%%
%% Figures using additional raster images can only be included by \input if
%% they are in the same directory as the main LaTeX file. For loading figures
%% from other directories you can use the `import` package
%%   \usepackage{import}
%% and then include the figures with
%%   \import{<path to file>}{<filename>.pgf}
%%
%% Matplotlib used the following preamble
%%   \usepackage{fontspec}
%%   \setmainfont{DejaVu Serif}
%%   \setsansfont{DejaVu Sans}
%%   \setmonofont{DejaVu Sans Mono}
%%
\begingroup%
\makeatletter%
\begin{pgfpicture}%
\pgfpathrectangle{\pgfpointorigin}{\pgfqpoint{6.405390in}{3.856040in}}%
\pgfusepath{use as bounding box, clip}%
\begin{pgfscope}%
\pgfsetbuttcap%
\pgfsetmiterjoin%
\definecolor{currentfill}{rgb}{1.000000,1.000000,1.000000}%
\pgfsetfillcolor{currentfill}%
\pgfsetlinewidth{0.000000pt}%
\definecolor{currentstroke}{rgb}{1.000000,1.000000,1.000000}%
\pgfsetstrokecolor{currentstroke}%
\pgfsetdash{}{0pt}%
\pgfpathmoveto{\pgfqpoint{0.000000in}{0.000000in}}%
\pgfpathlineto{\pgfqpoint{6.405390in}{0.000000in}}%
\pgfpathlineto{\pgfqpoint{6.405390in}{3.856040in}}%
\pgfpathlineto{\pgfqpoint{0.000000in}{3.856040in}}%
\pgfpathclose%
\pgfusepath{fill}%
\end{pgfscope}%
\begin{pgfscope}%
\pgfsys@transformshift{0.638889in}{2.244929in}%
\pgftext[left,bottom]{\pgfimage[interpolate=true,width=2.236111in,height=1.305556in]{Spectra_1e5-img0.png}}%
\end{pgfscope}%
\begin{pgfscope}%
\pgfsetroundcap%
\pgfsetroundjoin%
\pgfsetlinewidth{1.003750pt}%
\definecolor{currentstroke}{rgb}{0.000000,0.000000,0.000000}%
\pgfsetstrokecolor{currentstroke}%
\pgfsetdash{}{0pt}%
\pgfpathmoveto{\pgfqpoint{1.759449in}{3.086514in}}%
\pgfpathquadraticcurveto{\pgfqpoint{1.199015in}{2.462819in}}{\pgfqpoint{0.638581in}{1.839123in}}%
\pgfusepath{stroke}%
\end{pgfscope}%
\begin{pgfscope}%
\pgfsetroundcap%
\pgfsetroundjoin%
\pgfsetlinewidth{1.003750pt}%
\definecolor{currentstroke}{rgb}{0.000000,0.000000,0.000000}%
\pgfsetstrokecolor{currentstroke}%
\pgfsetdash{}{0pt}%
\pgfpathmoveto{\pgfqpoint{1.759449in}{2.889558in}}%
\pgfpathquadraticcurveto{\pgfqpoint{1.507253in}{2.364340in}}{\pgfqpoint{1.255058in}{1.839123in}}%
\pgfusepath{stroke}%
\end{pgfscope}%
\begin{pgfscope}%
\pgfsetroundcap%
\pgfsetroundjoin%
\pgfsetlinewidth{1.003750pt}%
\definecolor{currentstroke}{rgb}{0.000000,0.000000,0.000000}%
\pgfsetstrokecolor{currentstroke}%
\pgfsetdash{}{0pt}%
\pgfpathmoveto{\pgfqpoint{2.205428in}{3.086514in}}%
\pgfpathquadraticcurveto{\pgfqpoint{2.542872in}{2.462819in}}{\pgfqpoint{2.880317in}{1.839123in}}%
\pgfusepath{stroke}%
\end{pgfscope}%
\begin{pgfscope}%
\pgfsetroundcap%
\pgfsetroundjoin%
\pgfsetlinewidth{1.003750pt}%
\definecolor{currentstroke}{rgb}{0.000000,0.000000,0.000000}%
\pgfsetstrokecolor{currentstroke}%
\pgfsetdash{}{0pt}%
\pgfpathmoveto{\pgfqpoint{2.205428in}{2.889558in}}%
\pgfpathquadraticcurveto{\pgfqpoint{2.346720in}{2.364340in}}{\pgfqpoint{2.488013in}{1.839123in}}%
\pgfusepath{stroke}%
\end{pgfscope}%
\begin{pgfscope}%
\pgfsetbuttcap%
\pgfsetroundjoin%
\definecolor{currentfill}{rgb}{0.000000,0.000000,0.000000}%
\pgfsetfillcolor{currentfill}%
\pgfsetlinewidth{0.803000pt}%
\definecolor{currentstroke}{rgb}{0.000000,0.000000,0.000000}%
\pgfsetstrokecolor{currentstroke}%
\pgfsetdash{}{0pt}%
\pgfsys@defobject{currentmarker}{\pgfqpoint{0.000000in}{-0.048611in}}{\pgfqpoint{0.000000in}{0.000000in}}{%
\pgfpathmoveto{\pgfqpoint{0.000000in}{0.000000in}}%
\pgfpathlineto{\pgfqpoint{0.000000in}{-0.048611in}}%
\pgfusepath{stroke,fill}%
}%
\begin{pgfscope}%
\pgfsys@transformshift{0.644501in}{2.233036in}%
\pgfsys@useobject{currentmarker}{}%
\end{pgfscope}%
\end{pgfscope}%
\begin{pgfscope}%
\pgftext[x=0.644501in,y=2.135814in,,top]{\rmfamily\fontsize{10.000000}{12.000000}\selectfont \(\displaystyle -20\)}%
\end{pgfscope}%
\begin{pgfscope}%
\pgfsetbuttcap%
\pgfsetroundjoin%
\definecolor{currentfill}{rgb}{0.000000,0.000000,0.000000}%
\pgfsetfillcolor{currentfill}%
\pgfsetlinewidth{0.803000pt}%
\definecolor{currentstroke}{rgb}{0.000000,0.000000,0.000000}%
\pgfsetstrokecolor{currentstroke}%
\pgfsetdash{}{0pt}%
\pgfsys@defobject{currentmarker}{\pgfqpoint{0.000000in}{-0.048611in}}{\pgfqpoint{0.000000in}{0.000000in}}{%
\pgfpathmoveto{\pgfqpoint{0.000000in}{0.000000in}}%
\pgfpathlineto{\pgfqpoint{0.000000in}{-0.048611in}}%
\pgfusepath{stroke,fill}%
}%
\begin{pgfscope}%
\pgfsys@transformshift{1.201975in}{2.233036in}%
\pgfsys@useobject{currentmarker}{}%
\end{pgfscope}%
\end{pgfscope}%
\begin{pgfscope}%
\pgftext[x=1.201975in,y=2.135814in,,top]{\rmfamily\fontsize{10.000000}{12.000000}\selectfont \(\displaystyle -10\)}%
\end{pgfscope}%
\begin{pgfscope}%
\pgfsetbuttcap%
\pgfsetroundjoin%
\definecolor{currentfill}{rgb}{0.000000,0.000000,0.000000}%
\pgfsetfillcolor{currentfill}%
\pgfsetlinewidth{0.803000pt}%
\definecolor{currentstroke}{rgb}{0.000000,0.000000,0.000000}%
\pgfsetstrokecolor{currentstroke}%
\pgfsetdash{}{0pt}%
\pgfsys@defobject{currentmarker}{\pgfqpoint{0.000000in}{-0.048611in}}{\pgfqpoint{0.000000in}{0.000000in}}{%
\pgfpathmoveto{\pgfqpoint{0.000000in}{0.000000in}}%
\pgfpathlineto{\pgfqpoint{0.000000in}{-0.048611in}}%
\pgfusepath{stroke,fill}%
}%
\begin{pgfscope}%
\pgfsys@transformshift{1.759449in}{2.233036in}%
\pgfsys@useobject{currentmarker}{}%
\end{pgfscope}%
\end{pgfscope}%
\begin{pgfscope}%
\pgftext[x=1.759449in,y=2.135814in,,top]{\rmfamily\fontsize{10.000000}{12.000000}\selectfont \(\displaystyle 0\)}%
\end{pgfscope}%
\begin{pgfscope}%
\pgfsetbuttcap%
\pgfsetroundjoin%
\definecolor{currentfill}{rgb}{0.000000,0.000000,0.000000}%
\pgfsetfillcolor{currentfill}%
\pgfsetlinewidth{0.803000pt}%
\definecolor{currentstroke}{rgb}{0.000000,0.000000,0.000000}%
\pgfsetstrokecolor{currentstroke}%
\pgfsetdash{}{0pt}%
\pgfsys@defobject{currentmarker}{\pgfqpoint{0.000000in}{-0.048611in}}{\pgfqpoint{0.000000in}{0.000000in}}{%
\pgfpathmoveto{\pgfqpoint{0.000000in}{0.000000in}}%
\pgfpathlineto{\pgfqpoint{0.000000in}{-0.048611in}}%
\pgfusepath{stroke,fill}%
}%
\begin{pgfscope}%
\pgfsys@transformshift{2.316923in}{2.233036in}%
\pgfsys@useobject{currentmarker}{}%
\end{pgfscope}%
\end{pgfscope}%
\begin{pgfscope}%
\pgftext[x=2.316923in,y=2.135814in,,top]{\rmfamily\fontsize{10.000000}{12.000000}\selectfont \(\displaystyle 10\)}%
\end{pgfscope}%
\begin{pgfscope}%
\pgfsetbuttcap%
\pgfsetroundjoin%
\definecolor{currentfill}{rgb}{0.000000,0.000000,0.000000}%
\pgfsetfillcolor{currentfill}%
\pgfsetlinewidth{0.803000pt}%
\definecolor{currentstroke}{rgb}{0.000000,0.000000,0.000000}%
\pgfsetstrokecolor{currentstroke}%
\pgfsetdash{}{0pt}%
\pgfsys@defobject{currentmarker}{\pgfqpoint{0.000000in}{-0.048611in}}{\pgfqpoint{0.000000in}{0.000000in}}{%
\pgfpathmoveto{\pgfqpoint{0.000000in}{0.000000in}}%
\pgfpathlineto{\pgfqpoint{0.000000in}{-0.048611in}}%
\pgfusepath{stroke,fill}%
}%
\begin{pgfscope}%
\pgfsys@transformshift{2.874397in}{2.233036in}%
\pgfsys@useobject{currentmarker}{}%
\end{pgfscope}%
\end{pgfscope}%
\begin{pgfscope}%
\pgftext[x=2.874397in,y=2.135814in,,top]{\rmfamily\fontsize{10.000000}{12.000000}\selectfont \(\displaystyle 20\)}%
\end{pgfscope}%
\begin{pgfscope}%
\pgfsetbuttcap%
\pgfsetroundjoin%
\definecolor{currentfill}{rgb}{0.000000,0.000000,0.000000}%
\pgfsetfillcolor{currentfill}%
\pgfsetlinewidth{0.803000pt}%
\definecolor{currentstroke}{rgb}{0.000000,0.000000,0.000000}%
\pgfsetstrokecolor{currentstroke}%
\pgfsetdash{}{0pt}%
\pgfsys@defobject{currentmarker}{\pgfqpoint{-0.048611in}{0.000000in}}{\pgfqpoint{0.000000in}{0.000000in}}{%
\pgfpathmoveto{\pgfqpoint{0.000000in}{0.000000in}}%
\pgfpathlineto{\pgfqpoint{-0.048611in}{0.000000in}}%
\pgfusepath{stroke,fill}%
}%
\begin{pgfscope}%
\pgfsys@transformshift{0.638581in}{2.233036in}%
\pgfsys@useobject{currentmarker}{}%
\end{pgfscope}%
\end{pgfscope}%
\begin{pgfscope}%
\pgftext[x=0.294444in,y=2.180274in,left,base]{\rmfamily\fontsize{10.000000}{12.000000}\selectfont \(\displaystyle -20\)}%
\end{pgfscope}%
\begin{pgfscope}%
\pgfsetbuttcap%
\pgfsetroundjoin%
\definecolor{currentfill}{rgb}{0.000000,0.000000,0.000000}%
\pgfsetfillcolor{currentfill}%
\pgfsetlinewidth{0.803000pt}%
\definecolor{currentstroke}{rgb}{0.000000,0.000000,0.000000}%
\pgfsetstrokecolor{currentstroke}%
\pgfsetdash{}{0pt}%
\pgfsys@defobject{currentmarker}{\pgfqpoint{-0.048611in}{0.000000in}}{\pgfqpoint{0.000000in}{0.000000in}}{%
\pgfpathmoveto{\pgfqpoint{0.000000in}{0.000000in}}%
\pgfpathlineto{\pgfqpoint{-0.048611in}{0.000000in}}%
\pgfusepath{stroke,fill}%
}%
\begin{pgfscope}%
\pgfsys@transformshift{0.638581in}{2.561297in}%
\pgfsys@useobject{currentmarker}{}%
\end{pgfscope}%
\end{pgfscope}%
\begin{pgfscope}%
\pgftext[x=0.294444in,y=2.508535in,left,base]{\rmfamily\fontsize{10.000000}{12.000000}\selectfont \(\displaystyle -10\)}%
\end{pgfscope}%
\begin{pgfscope}%
\pgfsetbuttcap%
\pgfsetroundjoin%
\definecolor{currentfill}{rgb}{0.000000,0.000000,0.000000}%
\pgfsetfillcolor{currentfill}%
\pgfsetlinewidth{0.803000pt}%
\definecolor{currentstroke}{rgb}{0.000000,0.000000,0.000000}%
\pgfsetstrokecolor{currentstroke}%
\pgfsetdash{}{0pt}%
\pgfsys@defobject{currentmarker}{\pgfqpoint{-0.048611in}{0.000000in}}{\pgfqpoint{0.000000in}{0.000000in}}{%
\pgfpathmoveto{\pgfqpoint{0.000000in}{0.000000in}}%
\pgfpathlineto{\pgfqpoint{-0.048611in}{0.000000in}}%
\pgfusepath{stroke,fill}%
}%
\begin{pgfscope}%
\pgfsys@transformshift{0.638581in}{2.889558in}%
\pgfsys@useobject{currentmarker}{}%
\end{pgfscope}%
\end{pgfscope}%
\begin{pgfscope}%
\pgftext[x=0.471914in,y=2.836796in,left,base]{\rmfamily\fontsize{10.000000}{12.000000}\selectfont \(\displaystyle 0\)}%
\end{pgfscope}%
\begin{pgfscope}%
\pgfsetbuttcap%
\pgfsetroundjoin%
\definecolor{currentfill}{rgb}{0.000000,0.000000,0.000000}%
\pgfsetfillcolor{currentfill}%
\pgfsetlinewidth{0.803000pt}%
\definecolor{currentstroke}{rgb}{0.000000,0.000000,0.000000}%
\pgfsetstrokecolor{currentstroke}%
\pgfsetdash{}{0pt}%
\pgfsys@defobject{currentmarker}{\pgfqpoint{-0.048611in}{0.000000in}}{\pgfqpoint{0.000000in}{0.000000in}}{%
\pgfpathmoveto{\pgfqpoint{0.000000in}{0.000000in}}%
\pgfpathlineto{\pgfqpoint{-0.048611in}{0.000000in}}%
\pgfusepath{stroke,fill}%
}%
\begin{pgfscope}%
\pgfsys@transformshift{0.638581in}{3.217819in}%
\pgfsys@useobject{currentmarker}{}%
\end{pgfscope}%
\end{pgfscope}%
\begin{pgfscope}%
\pgftext[x=0.402469in,y=3.165057in,left,base]{\rmfamily\fontsize{10.000000}{12.000000}\selectfont \(\displaystyle 10\)}%
\end{pgfscope}%
\begin{pgfscope}%
\pgfsetbuttcap%
\pgfsetroundjoin%
\definecolor{currentfill}{rgb}{0.000000,0.000000,0.000000}%
\pgfsetfillcolor{currentfill}%
\pgfsetlinewidth{0.803000pt}%
\definecolor{currentstroke}{rgb}{0.000000,0.000000,0.000000}%
\pgfsetstrokecolor{currentstroke}%
\pgfsetdash{}{0pt}%
\pgfsys@defobject{currentmarker}{\pgfqpoint{-0.048611in}{0.000000in}}{\pgfqpoint{0.000000in}{0.000000in}}{%
\pgfpathmoveto{\pgfqpoint{0.000000in}{0.000000in}}%
\pgfpathlineto{\pgfqpoint{-0.048611in}{0.000000in}}%
\pgfusepath{stroke,fill}%
}%
\begin{pgfscope}%
\pgfsys@transformshift{0.638581in}{3.546079in}%
\pgfsys@useobject{currentmarker}{}%
\end{pgfscope}%
\end{pgfscope}%
\begin{pgfscope}%
\pgftext[x=0.402469in,y=3.493318in,left,base]{\rmfamily\fontsize{10.000000}{12.000000}\selectfont \(\displaystyle 20\)}%
\end{pgfscope}%
\begin{pgfscope}%
\pgftext[x=0.238889in,y=2.889558in,,bottom,rotate=90.000000]{\rmfamily\fontsize{10.000000}{12.000000}\selectfont \(\displaystyle \omega_\mathrm{r}/ |\Omega_\mathrm{ce}|\)}%
\end{pgfscope}%
\begin{pgfscope}%
\pgfpathrectangle{\pgfqpoint{0.638581in}{2.233036in}}{\pgfqpoint{2.241736in}{1.313043in}} %
\pgfusepath{clip}%
\pgfsetrectcap%
\pgfsetroundjoin%
\pgfsetlinewidth{1.003750pt}%
\definecolor{currentstroke}{rgb}{0.000000,0.000000,0.000000}%
\pgfsetstrokecolor{currentstroke}%
\pgfsetdash{}{0pt}%
\pgfpathmoveto{\pgfqpoint{1.759449in}{2.889558in}}%
\pgfpathlineto{\pgfqpoint{1.809002in}{2.889558in}}%
\pgfpathlineto{\pgfqpoint{1.858555in}{2.889558in}}%
\pgfpathlineto{\pgfqpoint{1.908108in}{2.889558in}}%
\pgfpathlineto{\pgfqpoint{1.957662in}{2.889558in}}%
\pgfpathlineto{\pgfqpoint{2.007215in}{2.889558in}}%
\pgfpathlineto{\pgfqpoint{2.056768in}{2.889558in}}%
\pgfpathlineto{\pgfqpoint{2.106321in}{2.889558in}}%
\pgfpathlineto{\pgfqpoint{2.155875in}{2.889558in}}%
\pgfpathlineto{\pgfqpoint{2.205428in}{2.889558in}}%
\pgfusepath{stroke}%
\end{pgfscope}%
\begin{pgfscope}%
\pgfpathrectangle{\pgfqpoint{0.638581in}{2.233036in}}{\pgfqpoint{2.241736in}{1.313043in}} %
\pgfusepath{clip}%
\pgfsetrectcap%
\pgfsetroundjoin%
\pgfsetlinewidth{1.003750pt}%
\definecolor{currentstroke}{rgb}{0.000000,0.000000,0.000000}%
\pgfsetstrokecolor{currentstroke}%
\pgfsetdash{}{0pt}%
\pgfpathmoveto{\pgfqpoint{1.759449in}{3.086514in}}%
\pgfpathlineto{\pgfqpoint{1.809002in}{3.086514in}}%
\pgfpathlineto{\pgfqpoint{1.858555in}{3.086514in}}%
\pgfpathlineto{\pgfqpoint{1.908108in}{3.086514in}}%
\pgfpathlineto{\pgfqpoint{1.957662in}{3.086514in}}%
\pgfpathlineto{\pgfqpoint{2.007215in}{3.086514in}}%
\pgfpathlineto{\pgfqpoint{2.056768in}{3.086514in}}%
\pgfpathlineto{\pgfqpoint{2.106321in}{3.086514in}}%
\pgfpathlineto{\pgfqpoint{2.155875in}{3.086514in}}%
\pgfpathlineto{\pgfqpoint{2.205428in}{3.086514in}}%
\pgfusepath{stroke}%
\end{pgfscope}%
\begin{pgfscope}%
\pgfpathrectangle{\pgfqpoint{0.638581in}{2.233036in}}{\pgfqpoint{2.241736in}{1.313043in}} %
\pgfusepath{clip}%
\pgfsetrectcap%
\pgfsetroundjoin%
\pgfsetlinewidth{1.003750pt}%
\definecolor{currentstroke}{rgb}{0.000000,0.000000,0.000000}%
\pgfsetstrokecolor{currentstroke}%
\pgfsetdash{}{0pt}%
\pgfpathmoveto{\pgfqpoint{1.759449in}{2.889558in}}%
\pgfpathlineto{\pgfqpoint{1.759449in}{2.911442in}}%
\pgfpathlineto{\pgfqpoint{1.759449in}{2.933326in}}%
\pgfpathlineto{\pgfqpoint{1.759449in}{2.955210in}}%
\pgfpathlineto{\pgfqpoint{1.759449in}{2.977094in}}%
\pgfpathlineto{\pgfqpoint{1.759449in}{2.998978in}}%
\pgfpathlineto{\pgfqpoint{1.759449in}{3.020862in}}%
\pgfpathlineto{\pgfqpoint{1.759449in}{3.042746in}}%
\pgfpathlineto{\pgfqpoint{1.759449in}{3.064630in}}%
\pgfpathlineto{\pgfqpoint{1.759449in}{3.086514in}}%
\pgfusepath{stroke}%
\end{pgfscope}%
\begin{pgfscope}%
\pgfpathrectangle{\pgfqpoint{0.638581in}{2.233036in}}{\pgfqpoint{2.241736in}{1.313043in}} %
\pgfusepath{clip}%
\pgfsetrectcap%
\pgfsetroundjoin%
\pgfsetlinewidth{1.003750pt}%
\definecolor{currentstroke}{rgb}{0.000000,0.000000,0.000000}%
\pgfsetstrokecolor{currentstroke}%
\pgfsetdash{}{0pt}%
\pgfpathmoveto{\pgfqpoint{2.205428in}{2.889558in}}%
\pgfpathlineto{\pgfqpoint{2.205428in}{2.911442in}}%
\pgfpathlineto{\pgfqpoint{2.205428in}{2.933326in}}%
\pgfpathlineto{\pgfqpoint{2.205428in}{2.955210in}}%
\pgfpathlineto{\pgfqpoint{2.205428in}{2.977094in}}%
\pgfpathlineto{\pgfqpoint{2.205428in}{2.998978in}}%
\pgfpathlineto{\pgfqpoint{2.205428in}{3.020862in}}%
\pgfpathlineto{\pgfqpoint{2.205428in}{3.042746in}}%
\pgfpathlineto{\pgfqpoint{2.205428in}{3.064630in}}%
\pgfpathlineto{\pgfqpoint{2.205428in}{3.086514in}}%
\pgfusepath{stroke}%
\end{pgfscope}%
\begin{pgfscope}%
\pgfsetrectcap%
\pgfsetmiterjoin%
\pgfsetlinewidth{0.803000pt}%
\definecolor{currentstroke}{rgb}{0.000000,0.000000,0.000000}%
\pgfsetstrokecolor{currentstroke}%
\pgfsetdash{}{0pt}%
\pgfpathmoveto{\pgfqpoint{0.638581in}{2.233036in}}%
\pgfpathlineto{\pgfqpoint{0.638581in}{3.546079in}}%
\pgfusepath{stroke}%
\end{pgfscope}%
\begin{pgfscope}%
\pgfsetrectcap%
\pgfsetmiterjoin%
\pgfsetlinewidth{0.803000pt}%
\definecolor{currentstroke}{rgb}{0.000000,0.000000,0.000000}%
\pgfsetstrokecolor{currentstroke}%
\pgfsetdash{}{0pt}%
\pgfpathmoveto{\pgfqpoint{2.880317in}{2.233036in}}%
\pgfpathlineto{\pgfqpoint{2.880317in}{3.546079in}}%
\pgfusepath{stroke}%
\end{pgfscope}%
\begin{pgfscope}%
\pgfsetrectcap%
\pgfsetmiterjoin%
\pgfsetlinewidth{0.803000pt}%
\definecolor{currentstroke}{rgb}{0.000000,0.000000,0.000000}%
\pgfsetstrokecolor{currentstroke}%
\pgfsetdash{}{0pt}%
\pgfpathmoveto{\pgfqpoint{0.638581in}{2.233036in}}%
\pgfpathlineto{\pgfqpoint{2.880317in}{2.233036in}}%
\pgfusepath{stroke}%
\end{pgfscope}%
\begin{pgfscope}%
\pgfsetrectcap%
\pgfsetmiterjoin%
\pgfsetlinewidth{0.803000pt}%
\definecolor{currentstroke}{rgb}{0.000000,0.000000,0.000000}%
\pgfsetstrokecolor{currentstroke}%
\pgfsetdash{}{0pt}%
\pgfpathmoveto{\pgfqpoint{0.638581in}{3.546079in}}%
\pgfpathlineto{\pgfqpoint{2.880317in}{3.546079in}}%
\pgfusepath{stroke}%
\end{pgfscope}%
\begin{pgfscope}%
\pgftext[x=0.700248in,y=3.365536in,left,base]{\rmfamily\fontsize{10.000000}{12.000000}\selectfont (a)}%
\end{pgfscope}%
\begin{pgfscope}%
\pgftext[x=1.759449in,y=3.629413in,,base]{\rmfamily\fontsize{12.000000}{14.400000}\selectfont Standard}%
\end{pgfscope}%
\begin{pgfscope}%
\pgfsys@transformshift{3.375000in}{2.244929in}%
\pgftext[left,bottom]{\pgfimage[interpolate=true,width=2.152778in,height=1.305556in]{Spectra_1e5-img1.png}}%
\end{pgfscope}%
\begin{pgfscope}%
\pgfsetroundcap%
\pgfsetroundjoin%
\pgfsetlinewidth{1.003750pt}%
\definecolor{currentstroke}{rgb}{0.000000,0.000000,0.000000}%
\pgfsetstrokecolor{currentstroke}%
\pgfsetdash{}{0pt}%
\pgfpathmoveto{\pgfqpoint{4.449531in}{3.086514in}}%
\pgfpathquadraticcurveto{\pgfqpoint{3.911515in}{2.462819in}}{\pgfqpoint{3.373498in}{1.839123in}}%
\pgfusepath{stroke}%
\end{pgfscope}%
\begin{pgfscope}%
\pgfsetroundcap%
\pgfsetroundjoin%
\pgfsetlinewidth{1.003750pt}%
\definecolor{currentstroke}{rgb}{0.000000,0.000000,0.000000}%
\pgfsetstrokecolor{currentstroke}%
\pgfsetdash{}{0pt}%
\pgfpathmoveto{\pgfqpoint{4.449531in}{2.889558in}}%
\pgfpathquadraticcurveto{\pgfqpoint{4.207424in}{2.364340in}}{\pgfqpoint{3.965317in}{1.839123in}}%
\pgfusepath{stroke}%
\end{pgfscope}%
\begin{pgfscope}%
\pgfsetroundcap%
\pgfsetroundjoin%
\pgfsetlinewidth{1.003750pt}%
\definecolor{currentstroke}{rgb}{0.000000,0.000000,0.000000}%
\pgfsetstrokecolor{currentstroke}%
\pgfsetdash{}{0pt}%
\pgfpathmoveto{\pgfqpoint{4.877671in}{3.086514in}}%
\pgfpathquadraticcurveto{\pgfqpoint{5.201618in}{2.462819in}}{\pgfqpoint{5.525564in}{1.839123in}}%
\pgfusepath{stroke}%
\end{pgfscope}%
\begin{pgfscope}%
\pgfsetroundcap%
\pgfsetroundjoin%
\pgfsetlinewidth{1.003750pt}%
\definecolor{currentstroke}{rgb}{0.000000,0.000000,0.000000}%
\pgfsetstrokecolor{currentstroke}%
\pgfsetdash{}{0pt}%
\pgfpathmoveto{\pgfqpoint{4.877671in}{2.889558in}}%
\pgfpathquadraticcurveto{\pgfqpoint{5.013312in}{2.364340in}}{\pgfqpoint{5.148953in}{1.839123in}}%
\pgfusepath{stroke}%
\end{pgfscope}%
\begin{pgfscope}%
\pgfsetbuttcap%
\pgfsetroundjoin%
\definecolor{currentfill}{rgb}{0.000000,0.000000,0.000000}%
\pgfsetfillcolor{currentfill}%
\pgfsetlinewidth{0.803000pt}%
\definecolor{currentstroke}{rgb}{0.000000,0.000000,0.000000}%
\pgfsetstrokecolor{currentstroke}%
\pgfsetdash{}{0pt}%
\pgfsys@defobject{currentmarker}{\pgfqpoint{0.000000in}{-0.048611in}}{\pgfqpoint{0.000000in}{0.000000in}}{%
\pgfpathmoveto{\pgfqpoint{0.000000in}{0.000000in}}%
\pgfpathlineto{\pgfqpoint{0.000000in}{-0.048611in}}%
\pgfusepath{stroke,fill}%
}%
\begin{pgfscope}%
\pgfsys@transformshift{3.379182in}{2.233036in}%
\pgfsys@useobject{currentmarker}{}%
\end{pgfscope}%
\end{pgfscope}%
\begin{pgfscope}%
\pgftext[x=3.379182in,y=2.135814in,,top]{\rmfamily\fontsize{10.000000}{12.000000}\selectfont \(\displaystyle -20\)}%
\end{pgfscope}%
\begin{pgfscope}%
\pgfsetbuttcap%
\pgfsetroundjoin%
\definecolor{currentfill}{rgb}{0.000000,0.000000,0.000000}%
\pgfsetfillcolor{currentfill}%
\pgfsetlinewidth{0.803000pt}%
\definecolor{currentstroke}{rgb}{0.000000,0.000000,0.000000}%
\pgfsetstrokecolor{currentstroke}%
\pgfsetdash{}{0pt}%
\pgfsys@defobject{currentmarker}{\pgfqpoint{0.000000in}{-0.048611in}}{\pgfqpoint{0.000000in}{0.000000in}}{%
\pgfpathmoveto{\pgfqpoint{0.000000in}{0.000000in}}%
\pgfpathlineto{\pgfqpoint{0.000000in}{-0.048611in}}%
\pgfusepath{stroke,fill}%
}%
\begin{pgfscope}%
\pgfsys@transformshift{3.914356in}{2.233036in}%
\pgfsys@useobject{currentmarker}{}%
\end{pgfscope}%
\end{pgfscope}%
\begin{pgfscope}%
\pgftext[x=3.914356in,y=2.135814in,,top]{\rmfamily\fontsize{10.000000}{12.000000}\selectfont \(\displaystyle -10\)}%
\end{pgfscope}%
\begin{pgfscope}%
\pgfsetbuttcap%
\pgfsetroundjoin%
\definecolor{currentfill}{rgb}{0.000000,0.000000,0.000000}%
\pgfsetfillcolor{currentfill}%
\pgfsetlinewidth{0.803000pt}%
\definecolor{currentstroke}{rgb}{0.000000,0.000000,0.000000}%
\pgfsetstrokecolor{currentstroke}%
\pgfsetdash{}{0pt}%
\pgfsys@defobject{currentmarker}{\pgfqpoint{0.000000in}{-0.048611in}}{\pgfqpoint{0.000000in}{0.000000in}}{%
\pgfpathmoveto{\pgfqpoint{0.000000in}{0.000000in}}%
\pgfpathlineto{\pgfqpoint{0.000000in}{-0.048611in}}%
\pgfusepath{stroke,fill}%
}%
\begin{pgfscope}%
\pgfsys@transformshift{4.449531in}{2.233036in}%
\pgfsys@useobject{currentmarker}{}%
\end{pgfscope}%
\end{pgfscope}%
\begin{pgfscope}%
\pgftext[x=4.449531in,y=2.135814in,,top]{\rmfamily\fontsize{10.000000}{12.000000}\selectfont \(\displaystyle 0\)}%
\end{pgfscope}%
\begin{pgfscope}%
\pgfsetbuttcap%
\pgfsetroundjoin%
\definecolor{currentfill}{rgb}{0.000000,0.000000,0.000000}%
\pgfsetfillcolor{currentfill}%
\pgfsetlinewidth{0.803000pt}%
\definecolor{currentstroke}{rgb}{0.000000,0.000000,0.000000}%
\pgfsetstrokecolor{currentstroke}%
\pgfsetdash{}{0pt}%
\pgfsys@defobject{currentmarker}{\pgfqpoint{0.000000in}{-0.048611in}}{\pgfqpoint{0.000000in}{0.000000in}}{%
\pgfpathmoveto{\pgfqpoint{0.000000in}{0.000000in}}%
\pgfpathlineto{\pgfqpoint{0.000000in}{-0.048611in}}%
\pgfusepath{stroke,fill}%
}%
\begin{pgfscope}%
\pgfsys@transformshift{4.984706in}{2.233036in}%
\pgfsys@useobject{currentmarker}{}%
\end{pgfscope}%
\end{pgfscope}%
\begin{pgfscope}%
\pgftext[x=4.984706in,y=2.135814in,,top]{\rmfamily\fontsize{10.000000}{12.000000}\selectfont \(\displaystyle 10\)}%
\end{pgfscope}%
\begin{pgfscope}%
\pgfsetbuttcap%
\pgfsetroundjoin%
\definecolor{currentfill}{rgb}{0.000000,0.000000,0.000000}%
\pgfsetfillcolor{currentfill}%
\pgfsetlinewidth{0.803000pt}%
\definecolor{currentstroke}{rgb}{0.000000,0.000000,0.000000}%
\pgfsetstrokecolor{currentstroke}%
\pgfsetdash{}{0pt}%
\pgfsys@defobject{currentmarker}{\pgfqpoint{0.000000in}{-0.048611in}}{\pgfqpoint{0.000000in}{0.000000in}}{%
\pgfpathmoveto{\pgfqpoint{0.000000in}{0.000000in}}%
\pgfpathlineto{\pgfqpoint{0.000000in}{-0.048611in}}%
\pgfusepath{stroke,fill}%
}%
\begin{pgfscope}%
\pgfsys@transformshift{5.519881in}{2.233036in}%
\pgfsys@useobject{currentmarker}{}%
\end{pgfscope}%
\end{pgfscope}%
\begin{pgfscope}%
\pgftext[x=5.519881in,y=2.135814in,,top]{\rmfamily\fontsize{10.000000}{12.000000}\selectfont \(\displaystyle 20\)}%
\end{pgfscope}%
\begin{pgfscope}%
\pgfsetbuttcap%
\pgfsetroundjoin%
\definecolor{currentfill}{rgb}{0.000000,0.000000,0.000000}%
\pgfsetfillcolor{currentfill}%
\pgfsetlinewidth{0.803000pt}%
\definecolor{currentstroke}{rgb}{0.000000,0.000000,0.000000}%
\pgfsetstrokecolor{currentstroke}%
\pgfsetdash{}{0pt}%
\pgfsys@defobject{currentmarker}{\pgfqpoint{-0.048611in}{0.000000in}}{\pgfqpoint{0.000000in}{0.000000in}}{%
\pgfpathmoveto{\pgfqpoint{0.000000in}{0.000000in}}%
\pgfpathlineto{\pgfqpoint{-0.048611in}{0.000000in}}%
\pgfusepath{stroke,fill}%
}%
\begin{pgfscope}%
\pgfsys@transformshift{3.373498in}{2.233036in}%
\pgfsys@useobject{currentmarker}{}%
\end{pgfscope}%
\end{pgfscope}%
\begin{pgfscope}%
\pgfsetbuttcap%
\pgfsetroundjoin%
\definecolor{currentfill}{rgb}{0.000000,0.000000,0.000000}%
\pgfsetfillcolor{currentfill}%
\pgfsetlinewidth{0.803000pt}%
\definecolor{currentstroke}{rgb}{0.000000,0.000000,0.000000}%
\pgfsetstrokecolor{currentstroke}%
\pgfsetdash{}{0pt}%
\pgfsys@defobject{currentmarker}{\pgfqpoint{-0.048611in}{0.000000in}}{\pgfqpoint{0.000000in}{0.000000in}}{%
\pgfpathmoveto{\pgfqpoint{0.000000in}{0.000000in}}%
\pgfpathlineto{\pgfqpoint{-0.048611in}{0.000000in}}%
\pgfusepath{stroke,fill}%
}%
\begin{pgfscope}%
\pgfsys@transformshift{3.373498in}{2.561297in}%
\pgfsys@useobject{currentmarker}{}%
\end{pgfscope}%
\end{pgfscope}%
\begin{pgfscope}%
\pgfsetbuttcap%
\pgfsetroundjoin%
\definecolor{currentfill}{rgb}{0.000000,0.000000,0.000000}%
\pgfsetfillcolor{currentfill}%
\pgfsetlinewidth{0.803000pt}%
\definecolor{currentstroke}{rgb}{0.000000,0.000000,0.000000}%
\pgfsetstrokecolor{currentstroke}%
\pgfsetdash{}{0pt}%
\pgfsys@defobject{currentmarker}{\pgfqpoint{-0.048611in}{0.000000in}}{\pgfqpoint{0.000000in}{0.000000in}}{%
\pgfpathmoveto{\pgfqpoint{0.000000in}{0.000000in}}%
\pgfpathlineto{\pgfqpoint{-0.048611in}{0.000000in}}%
\pgfusepath{stroke,fill}%
}%
\begin{pgfscope}%
\pgfsys@transformshift{3.373498in}{2.889558in}%
\pgfsys@useobject{currentmarker}{}%
\end{pgfscope}%
\end{pgfscope}%
\begin{pgfscope}%
\pgfsetbuttcap%
\pgfsetroundjoin%
\definecolor{currentfill}{rgb}{0.000000,0.000000,0.000000}%
\pgfsetfillcolor{currentfill}%
\pgfsetlinewidth{0.803000pt}%
\definecolor{currentstroke}{rgb}{0.000000,0.000000,0.000000}%
\pgfsetstrokecolor{currentstroke}%
\pgfsetdash{}{0pt}%
\pgfsys@defobject{currentmarker}{\pgfqpoint{-0.048611in}{0.000000in}}{\pgfqpoint{0.000000in}{0.000000in}}{%
\pgfpathmoveto{\pgfqpoint{0.000000in}{0.000000in}}%
\pgfpathlineto{\pgfqpoint{-0.048611in}{0.000000in}}%
\pgfusepath{stroke,fill}%
}%
\begin{pgfscope}%
\pgfsys@transformshift{3.373498in}{3.217819in}%
\pgfsys@useobject{currentmarker}{}%
\end{pgfscope}%
\end{pgfscope}%
\begin{pgfscope}%
\pgfsetbuttcap%
\pgfsetroundjoin%
\definecolor{currentfill}{rgb}{0.000000,0.000000,0.000000}%
\pgfsetfillcolor{currentfill}%
\pgfsetlinewidth{0.803000pt}%
\definecolor{currentstroke}{rgb}{0.000000,0.000000,0.000000}%
\pgfsetstrokecolor{currentstroke}%
\pgfsetdash{}{0pt}%
\pgfsys@defobject{currentmarker}{\pgfqpoint{-0.048611in}{0.000000in}}{\pgfqpoint{0.000000in}{0.000000in}}{%
\pgfpathmoveto{\pgfqpoint{0.000000in}{0.000000in}}%
\pgfpathlineto{\pgfqpoint{-0.048611in}{0.000000in}}%
\pgfusepath{stroke,fill}%
}%
\begin{pgfscope}%
\pgfsys@transformshift{3.373498in}{3.546079in}%
\pgfsys@useobject{currentmarker}{}%
\end{pgfscope}%
\end{pgfscope}%
\begin{pgfscope}%
\pgfpathrectangle{\pgfqpoint{3.373498in}{2.233036in}}{\pgfqpoint{2.152066in}{1.313043in}} %
\pgfusepath{clip}%
\pgfsetrectcap%
\pgfsetroundjoin%
\pgfsetlinewidth{1.003750pt}%
\definecolor{currentstroke}{rgb}{0.000000,0.000000,0.000000}%
\pgfsetstrokecolor{currentstroke}%
\pgfsetdash{}{0pt}%
\pgfpathmoveto{\pgfqpoint{4.449531in}{2.889558in}}%
\pgfpathlineto{\pgfqpoint{4.497102in}{2.889558in}}%
\pgfpathlineto{\pgfqpoint{4.544674in}{2.889558in}}%
\pgfpathlineto{\pgfqpoint{4.592245in}{2.889558in}}%
\pgfpathlineto{\pgfqpoint{4.639816in}{2.889558in}}%
\pgfpathlineto{\pgfqpoint{4.687387in}{2.889558in}}%
\pgfpathlineto{\pgfqpoint{4.734958in}{2.889558in}}%
\pgfpathlineto{\pgfqpoint{4.782529in}{2.889558in}}%
\pgfpathlineto{\pgfqpoint{4.830100in}{2.889558in}}%
\pgfpathlineto{\pgfqpoint{4.877671in}{2.889558in}}%
\pgfusepath{stroke}%
\end{pgfscope}%
\begin{pgfscope}%
\pgfpathrectangle{\pgfqpoint{3.373498in}{2.233036in}}{\pgfqpoint{2.152066in}{1.313043in}} %
\pgfusepath{clip}%
\pgfsetrectcap%
\pgfsetroundjoin%
\pgfsetlinewidth{1.003750pt}%
\definecolor{currentstroke}{rgb}{0.000000,0.000000,0.000000}%
\pgfsetstrokecolor{currentstroke}%
\pgfsetdash{}{0pt}%
\pgfpathmoveto{\pgfqpoint{4.449531in}{3.086514in}}%
\pgfpathlineto{\pgfqpoint{4.497102in}{3.086514in}}%
\pgfpathlineto{\pgfqpoint{4.544674in}{3.086514in}}%
\pgfpathlineto{\pgfqpoint{4.592245in}{3.086514in}}%
\pgfpathlineto{\pgfqpoint{4.639816in}{3.086514in}}%
\pgfpathlineto{\pgfqpoint{4.687387in}{3.086514in}}%
\pgfpathlineto{\pgfqpoint{4.734958in}{3.086514in}}%
\pgfpathlineto{\pgfqpoint{4.782529in}{3.086514in}}%
\pgfpathlineto{\pgfqpoint{4.830100in}{3.086514in}}%
\pgfpathlineto{\pgfqpoint{4.877671in}{3.086514in}}%
\pgfusepath{stroke}%
\end{pgfscope}%
\begin{pgfscope}%
\pgfpathrectangle{\pgfqpoint{3.373498in}{2.233036in}}{\pgfqpoint{2.152066in}{1.313043in}} %
\pgfusepath{clip}%
\pgfsetrectcap%
\pgfsetroundjoin%
\pgfsetlinewidth{1.003750pt}%
\definecolor{currentstroke}{rgb}{0.000000,0.000000,0.000000}%
\pgfsetstrokecolor{currentstroke}%
\pgfsetdash{}{0pt}%
\pgfpathmoveto{\pgfqpoint{4.449531in}{2.889558in}}%
\pgfpathlineto{\pgfqpoint{4.449531in}{2.911442in}}%
\pgfpathlineto{\pgfqpoint{4.449531in}{2.933326in}}%
\pgfpathlineto{\pgfqpoint{4.449531in}{2.955210in}}%
\pgfpathlineto{\pgfqpoint{4.449531in}{2.977094in}}%
\pgfpathlineto{\pgfqpoint{4.449531in}{2.998978in}}%
\pgfpathlineto{\pgfqpoint{4.449531in}{3.020862in}}%
\pgfpathlineto{\pgfqpoint{4.449531in}{3.042746in}}%
\pgfpathlineto{\pgfqpoint{4.449531in}{3.064630in}}%
\pgfpathlineto{\pgfqpoint{4.449531in}{3.086514in}}%
\pgfusepath{stroke}%
\end{pgfscope}%
\begin{pgfscope}%
\pgfpathrectangle{\pgfqpoint{3.373498in}{2.233036in}}{\pgfqpoint{2.152066in}{1.313043in}} %
\pgfusepath{clip}%
\pgfsetrectcap%
\pgfsetroundjoin%
\pgfsetlinewidth{1.003750pt}%
\definecolor{currentstroke}{rgb}{0.000000,0.000000,0.000000}%
\pgfsetstrokecolor{currentstroke}%
\pgfsetdash{}{0pt}%
\pgfpathmoveto{\pgfqpoint{4.877671in}{2.889558in}}%
\pgfpathlineto{\pgfqpoint{4.877671in}{2.911442in}}%
\pgfpathlineto{\pgfqpoint{4.877671in}{2.933326in}}%
\pgfpathlineto{\pgfqpoint{4.877671in}{2.955210in}}%
\pgfpathlineto{\pgfqpoint{4.877671in}{2.977094in}}%
\pgfpathlineto{\pgfqpoint{4.877671in}{2.998978in}}%
\pgfpathlineto{\pgfqpoint{4.877671in}{3.020862in}}%
\pgfpathlineto{\pgfqpoint{4.877671in}{3.042746in}}%
\pgfpathlineto{\pgfqpoint{4.877671in}{3.064630in}}%
\pgfpathlineto{\pgfqpoint{4.877671in}{3.086514in}}%
\pgfusepath{stroke}%
\end{pgfscope}%
\begin{pgfscope}%
\pgfsetrectcap%
\pgfsetmiterjoin%
\pgfsetlinewidth{0.803000pt}%
\definecolor{currentstroke}{rgb}{0.000000,0.000000,0.000000}%
\pgfsetstrokecolor{currentstroke}%
\pgfsetdash{}{0pt}%
\pgfpathmoveto{\pgfqpoint{3.373498in}{2.233036in}}%
\pgfpathlineto{\pgfqpoint{3.373498in}{3.546079in}}%
\pgfusepath{stroke}%
\end{pgfscope}%
\begin{pgfscope}%
\pgfsetrectcap%
\pgfsetmiterjoin%
\pgfsetlinewidth{0.803000pt}%
\definecolor{currentstroke}{rgb}{0.000000,0.000000,0.000000}%
\pgfsetstrokecolor{currentstroke}%
\pgfsetdash{}{0pt}%
\pgfpathmoveto{\pgfqpoint{5.525564in}{2.233036in}}%
\pgfpathlineto{\pgfqpoint{5.525564in}{3.546079in}}%
\pgfusepath{stroke}%
\end{pgfscope}%
\begin{pgfscope}%
\pgfsetrectcap%
\pgfsetmiterjoin%
\pgfsetlinewidth{0.803000pt}%
\definecolor{currentstroke}{rgb}{0.000000,0.000000,0.000000}%
\pgfsetstrokecolor{currentstroke}%
\pgfsetdash{}{0pt}%
\pgfpathmoveto{\pgfqpoint{3.373498in}{2.233036in}}%
\pgfpathlineto{\pgfqpoint{5.525564in}{2.233036in}}%
\pgfusepath{stroke}%
\end{pgfscope}%
\begin{pgfscope}%
\pgfsetrectcap%
\pgfsetmiterjoin%
\pgfsetlinewidth{0.803000pt}%
\definecolor{currentstroke}{rgb}{0.000000,0.000000,0.000000}%
\pgfsetstrokecolor{currentstroke}%
\pgfsetdash{}{0pt}%
\pgfpathmoveto{\pgfqpoint{3.373498in}{3.546079in}}%
\pgfpathlineto{\pgfqpoint{5.525564in}{3.546079in}}%
\pgfusepath{stroke}%
\end{pgfscope}%
\begin{pgfscope}%
\pgftext[x=3.432699in,y=3.365536in,left,base]{\rmfamily\fontsize{10.000000}{12.000000}\selectfont (b)}%
\end{pgfscope}%
\begin{pgfscope}%
\pgftext[x=4.449531in,y=3.629413in,,base]{\rmfamily\fontsize{12.000000}{14.400000}\selectfont Geometric (Strang)}%
\end{pgfscope}%
\begin{pgfscope}%
\pgfsys@transformshift{0.638889in}{0.536596in}%
\pgftext[left,bottom]{\pgfimage[interpolate=true,width=2.236111in,height=1.305556in]{Spectra_1e5-img2.png}}%
\end{pgfscope}%
\begin{pgfscope}%
\pgfsetbuttcap%
\pgfsetroundjoin%
\definecolor{currentfill}{rgb}{0.000000,0.000000,0.000000}%
\pgfsetfillcolor{currentfill}%
\pgfsetlinewidth{0.803000pt}%
\definecolor{currentstroke}{rgb}{0.000000,0.000000,0.000000}%
\pgfsetstrokecolor{currentstroke}%
\pgfsetdash{}{0pt}%
\pgfsys@defobject{currentmarker}{\pgfqpoint{0.000000in}{-0.048611in}}{\pgfqpoint{0.000000in}{0.000000in}}{%
\pgfpathmoveto{\pgfqpoint{0.000000in}{0.000000in}}%
\pgfpathlineto{\pgfqpoint{0.000000in}{-0.048611in}}%
\pgfusepath{stroke,fill}%
}%
\begin{pgfscope}%
\pgfsys@transformshift{0.638581in}{0.526079in}%
\pgfsys@useobject{currentmarker}{}%
\end{pgfscope}%
\end{pgfscope}%
\begin{pgfscope}%
\pgftext[x=0.638581in,y=0.428857in,,top]{\rmfamily\fontsize{10.000000}{12.000000}\selectfont \(\displaystyle 0\)}%
\end{pgfscope}%
\begin{pgfscope}%
\pgfsetbuttcap%
\pgfsetroundjoin%
\definecolor{currentfill}{rgb}{0.000000,0.000000,0.000000}%
\pgfsetfillcolor{currentfill}%
\pgfsetlinewidth{0.803000pt}%
\definecolor{currentstroke}{rgb}{0.000000,0.000000,0.000000}%
\pgfsetstrokecolor{currentstroke}%
\pgfsetdash{}{0pt}%
\pgfsys@defobject{currentmarker}{\pgfqpoint{0.000000in}{-0.048611in}}{\pgfqpoint{0.000000in}{0.000000in}}{%
\pgfpathmoveto{\pgfqpoint{0.000000in}{0.000000in}}%
\pgfpathlineto{\pgfqpoint{0.000000in}{-0.048611in}}%
\pgfusepath{stroke,fill}%
}%
\begin{pgfscope}%
\pgfsys@transformshift{1.199015in}{0.526079in}%
\pgfsys@useobject{currentmarker}{}%
\end{pgfscope}%
\end{pgfscope}%
\begin{pgfscope}%
\pgftext[x=1.199015in,y=0.428857in,,top]{\rmfamily\fontsize{10.000000}{12.000000}\selectfont \(\displaystyle 2\)}%
\end{pgfscope}%
\begin{pgfscope}%
\pgfsetbuttcap%
\pgfsetroundjoin%
\definecolor{currentfill}{rgb}{0.000000,0.000000,0.000000}%
\pgfsetfillcolor{currentfill}%
\pgfsetlinewidth{0.803000pt}%
\definecolor{currentstroke}{rgb}{0.000000,0.000000,0.000000}%
\pgfsetstrokecolor{currentstroke}%
\pgfsetdash{}{0pt}%
\pgfsys@defobject{currentmarker}{\pgfqpoint{0.000000in}{-0.048611in}}{\pgfqpoint{0.000000in}{0.000000in}}{%
\pgfpathmoveto{\pgfqpoint{0.000000in}{0.000000in}}%
\pgfpathlineto{\pgfqpoint{0.000000in}{-0.048611in}}%
\pgfusepath{stroke,fill}%
}%
\begin{pgfscope}%
\pgfsys@transformshift{1.759449in}{0.526079in}%
\pgfsys@useobject{currentmarker}{}%
\end{pgfscope}%
\end{pgfscope}%
\begin{pgfscope}%
\pgftext[x=1.759449in,y=0.428857in,,top]{\rmfamily\fontsize{10.000000}{12.000000}\selectfont \(\displaystyle 4\)}%
\end{pgfscope}%
\begin{pgfscope}%
\pgfsetbuttcap%
\pgfsetroundjoin%
\definecolor{currentfill}{rgb}{0.000000,0.000000,0.000000}%
\pgfsetfillcolor{currentfill}%
\pgfsetlinewidth{0.803000pt}%
\definecolor{currentstroke}{rgb}{0.000000,0.000000,0.000000}%
\pgfsetstrokecolor{currentstroke}%
\pgfsetdash{}{0pt}%
\pgfsys@defobject{currentmarker}{\pgfqpoint{0.000000in}{-0.048611in}}{\pgfqpoint{0.000000in}{0.000000in}}{%
\pgfpathmoveto{\pgfqpoint{0.000000in}{0.000000in}}%
\pgfpathlineto{\pgfqpoint{0.000000in}{-0.048611in}}%
\pgfusepath{stroke,fill}%
}%
\begin{pgfscope}%
\pgfsys@transformshift{2.319883in}{0.526079in}%
\pgfsys@useobject{currentmarker}{}%
\end{pgfscope}%
\end{pgfscope}%
\begin{pgfscope}%
\pgftext[x=2.319883in,y=0.428857in,,top]{\rmfamily\fontsize{10.000000}{12.000000}\selectfont \(\displaystyle 6\)}%
\end{pgfscope}%
\begin{pgfscope}%
\pgfsetbuttcap%
\pgfsetroundjoin%
\definecolor{currentfill}{rgb}{0.000000,0.000000,0.000000}%
\pgfsetfillcolor{currentfill}%
\pgfsetlinewidth{0.803000pt}%
\definecolor{currentstroke}{rgb}{0.000000,0.000000,0.000000}%
\pgfsetstrokecolor{currentstroke}%
\pgfsetdash{}{0pt}%
\pgfsys@defobject{currentmarker}{\pgfqpoint{0.000000in}{-0.048611in}}{\pgfqpoint{0.000000in}{0.000000in}}{%
\pgfpathmoveto{\pgfqpoint{0.000000in}{0.000000in}}%
\pgfpathlineto{\pgfqpoint{0.000000in}{-0.048611in}}%
\pgfusepath{stroke,fill}%
}%
\begin{pgfscope}%
\pgfsys@transformshift{2.880317in}{0.526079in}%
\pgfsys@useobject{currentmarker}{}%
\end{pgfscope}%
\end{pgfscope}%
\begin{pgfscope}%
\pgftext[x=2.880317in,y=0.428857in,,top]{\rmfamily\fontsize{10.000000}{12.000000}\selectfont \(\displaystyle 8\)}%
\end{pgfscope}%
\begin{pgfscope}%
\pgftext[x=1.759449in,y=0.238889in,,top]{\rmfamily\fontsize{10.000000}{12.000000}\selectfont \(\displaystyle kc/ |\Omega_\mathrm{ce}|\)}%
\end{pgfscope}%
\begin{pgfscope}%
\pgfsetbuttcap%
\pgfsetroundjoin%
\definecolor{currentfill}{rgb}{0.000000,0.000000,0.000000}%
\pgfsetfillcolor{currentfill}%
\pgfsetlinewidth{0.803000pt}%
\definecolor{currentstroke}{rgb}{0.000000,0.000000,0.000000}%
\pgfsetstrokecolor{currentstroke}%
\pgfsetdash{}{0pt}%
\pgfsys@defobject{currentmarker}{\pgfqpoint{-0.048611in}{0.000000in}}{\pgfqpoint{0.000000in}{0.000000in}}{%
\pgfpathmoveto{\pgfqpoint{0.000000in}{0.000000in}}%
\pgfpathlineto{\pgfqpoint{-0.048611in}{0.000000in}}%
\pgfusepath{stroke,fill}%
}%
\begin{pgfscope}%
\pgfsys@transformshift{0.638581in}{0.526079in}%
\pgfsys@useobject{currentmarker}{}%
\end{pgfscope}%
\end{pgfscope}%
\begin{pgfscope}%
\pgftext[x=0.471914in,y=0.473318in,left,base]{\rmfamily\fontsize{10.000000}{12.000000}\selectfont \(\displaystyle 0\)}%
\end{pgfscope}%
\begin{pgfscope}%
\pgfsetbuttcap%
\pgfsetroundjoin%
\definecolor{currentfill}{rgb}{0.000000,0.000000,0.000000}%
\pgfsetfillcolor{currentfill}%
\pgfsetlinewidth{0.803000pt}%
\definecolor{currentstroke}{rgb}{0.000000,0.000000,0.000000}%
\pgfsetstrokecolor{currentstroke}%
\pgfsetdash{}{0pt}%
\pgfsys@defobject{currentmarker}{\pgfqpoint{-0.048611in}{0.000000in}}{\pgfqpoint{0.000000in}{0.000000in}}{%
\pgfpathmoveto{\pgfqpoint{0.000000in}{0.000000in}}%
\pgfpathlineto{\pgfqpoint{-0.048611in}{0.000000in}}%
\pgfusepath{stroke,fill}%
}%
\begin{pgfscope}%
\pgfsys@transformshift{0.638581in}{0.963761in}%
\pgfsys@useobject{currentmarker}{}%
\end{pgfscope}%
\end{pgfscope}%
\begin{pgfscope}%
\pgftext[x=0.471914in,y=0.910999in,left,base]{\rmfamily\fontsize{10.000000}{12.000000}\selectfont \(\displaystyle 2\)}%
\end{pgfscope}%
\begin{pgfscope}%
\pgfsetbuttcap%
\pgfsetroundjoin%
\definecolor{currentfill}{rgb}{0.000000,0.000000,0.000000}%
\pgfsetfillcolor{currentfill}%
\pgfsetlinewidth{0.803000pt}%
\definecolor{currentstroke}{rgb}{0.000000,0.000000,0.000000}%
\pgfsetstrokecolor{currentstroke}%
\pgfsetdash{}{0pt}%
\pgfsys@defobject{currentmarker}{\pgfqpoint{-0.048611in}{0.000000in}}{\pgfqpoint{0.000000in}{0.000000in}}{%
\pgfpathmoveto{\pgfqpoint{0.000000in}{0.000000in}}%
\pgfpathlineto{\pgfqpoint{-0.048611in}{0.000000in}}%
\pgfusepath{stroke,fill}%
}%
\begin{pgfscope}%
\pgfsys@transformshift{0.638581in}{1.401442in}%
\pgfsys@useobject{currentmarker}{}%
\end{pgfscope}%
\end{pgfscope}%
\begin{pgfscope}%
\pgftext[x=0.471914in,y=1.348680in,left,base]{\rmfamily\fontsize{10.000000}{12.000000}\selectfont \(\displaystyle 4\)}%
\end{pgfscope}%
\begin{pgfscope}%
\pgfsetbuttcap%
\pgfsetroundjoin%
\definecolor{currentfill}{rgb}{0.000000,0.000000,0.000000}%
\pgfsetfillcolor{currentfill}%
\pgfsetlinewidth{0.803000pt}%
\definecolor{currentstroke}{rgb}{0.000000,0.000000,0.000000}%
\pgfsetstrokecolor{currentstroke}%
\pgfsetdash{}{0pt}%
\pgfsys@defobject{currentmarker}{\pgfqpoint{-0.048611in}{0.000000in}}{\pgfqpoint{0.000000in}{0.000000in}}{%
\pgfpathmoveto{\pgfqpoint{0.000000in}{0.000000in}}%
\pgfpathlineto{\pgfqpoint{-0.048611in}{0.000000in}}%
\pgfusepath{stroke,fill}%
}%
\begin{pgfscope}%
\pgfsys@transformshift{0.638581in}{1.839123in}%
\pgfsys@useobject{currentmarker}{}%
\end{pgfscope}%
\end{pgfscope}%
\begin{pgfscope}%
\pgftext[x=0.471914in,y=1.786361in,left,base]{\rmfamily\fontsize{10.000000}{12.000000}\selectfont \(\displaystyle 6\)}%
\end{pgfscope}%
\begin{pgfscope}%
\pgftext[x=0.416358in,y=1.182601in,,bottom,rotate=90.000000]{\rmfamily\fontsize{10.000000}{12.000000}\selectfont \(\displaystyle \omega_\mathrm{r}/ |\Omega_\mathrm{ce}|\)}%
\end{pgfscope}%
\begin{pgfscope}%
\pgfpathrectangle{\pgfqpoint{0.638581in}{0.526079in}}{\pgfqpoint{2.241736in}{1.313043in}} %
\pgfusepath{clip}%
\pgfsetbuttcap%
\pgfsetroundjoin%
\pgfsetlinewidth{1.003750pt}%
\definecolor{currentstroke}{rgb}{0.000000,0.000000,0.000000}%
\pgfsetstrokecolor{currentstroke}%
\pgfsetdash{{3.700000pt}{1.600000pt}}{0.000000pt}%
\pgfpathmoveto{\pgfqpoint{0.666603in}{0.526625in}}%
\pgfpathlineto{\pgfqpoint{0.723365in}{0.530950in}}%
\pgfpathlineto{\pgfqpoint{0.780126in}{0.539032in}}%
\pgfpathlineto{\pgfqpoint{0.836888in}{0.549917in}}%
\pgfpathlineto{\pgfqpoint{0.893650in}{0.562581in}}%
\pgfpathlineto{\pgfqpoint{0.950412in}{0.576129in}}%
\pgfpathlineto{\pgfqpoint{1.007174in}{0.589860in}}%
\pgfpathlineto{\pgfqpoint{1.063936in}{0.603270in}}%
\pgfpathlineto{\pgfqpoint{1.120698in}{0.616024in}}%
\pgfpathlineto{\pgfqpoint{1.177460in}{0.627919in}}%
\pgfpathlineto{\pgfqpoint{1.234222in}{0.638851in}}%
\pgfpathlineto{\pgfqpoint{1.290983in}{0.648791in}}%
\pgfpathlineto{\pgfqpoint{1.347745in}{0.657756in}}%
\pgfpathlineto{\pgfqpoint{1.404507in}{0.665798in}}%
\pgfpathlineto{\pgfqpoint{1.461269in}{0.672985in}}%
\pgfpathlineto{\pgfqpoint{1.518031in}{0.679395in}}%
\pgfpathlineto{\pgfqpoint{1.574793in}{0.685105in}}%
\pgfpathlineto{\pgfqpoint{1.631555in}{0.690192in}}%
\pgfpathlineto{\pgfqpoint{1.688317in}{0.694727in}}%
\pgfpathlineto{\pgfqpoint{1.745079in}{0.698775in}}%
\pgfpathlineto{\pgfqpoint{1.801841in}{0.702394in}}%
\pgfpathlineto{\pgfqpoint{1.858602in}{0.705636in}}%
\pgfpathlineto{\pgfqpoint{1.915364in}{0.708546in}}%
\pgfpathlineto{\pgfqpoint{1.972126in}{0.711165in}}%
\pgfpathlineto{\pgfqpoint{2.028888in}{0.713526in}}%
\pgfpathlineto{\pgfqpoint{2.085650in}{0.715661in}}%
\pgfpathlineto{\pgfqpoint{2.142412in}{0.717595in}}%
\pgfpathlineto{\pgfqpoint{2.199174in}{0.719352in}}%
\pgfpathlineto{\pgfqpoint{2.255936in}{0.720951in}}%
\pgfpathlineto{\pgfqpoint{2.312698in}{0.722410in}}%
\pgfpathlineto{\pgfqpoint{2.369459in}{0.723745in}}%
\pgfpathlineto{\pgfqpoint{2.426221in}{0.724967in}}%
\pgfpathlineto{\pgfqpoint{2.482983in}{0.726090in}}%
\pgfpathlineto{\pgfqpoint{2.539745in}{0.727123in}}%
\pgfpathlineto{\pgfqpoint{2.596507in}{0.728076in}}%
\pgfpathlineto{\pgfqpoint{2.653269in}{0.728956in}}%
\pgfpathlineto{\pgfqpoint{2.710031in}{0.729770in}}%
\pgfpathlineto{\pgfqpoint{2.766793in}{0.730525in}}%
\pgfpathlineto{\pgfqpoint{2.823555in}{0.731226in}}%
\pgfpathlineto{\pgfqpoint{2.880317in}{0.731877in}}%
\pgfusepath{stroke}%
\end{pgfscope}%
\begin{pgfscope}%
\pgfpathrectangle{\pgfqpoint{0.638581in}{0.526079in}}{\pgfqpoint{2.241736in}{1.313043in}} %
\pgfusepath{clip}%
\pgfsetbuttcap%
\pgfsetroundjoin%
\pgfsetlinewidth{1.003750pt}%
\definecolor{currentstroke}{rgb}{0.000000,0.000000,0.000000}%
\pgfsetstrokecolor{currentstroke}%
\pgfsetdash{{3.700000pt}{1.600000pt}}{0.000000pt}%
\pgfpathmoveto{\pgfqpoint{0.666603in}{0.868680in}}%
\pgfpathlineto{\pgfqpoint{0.723365in}{0.875643in}}%
\pgfpathlineto{\pgfqpoint{0.780126in}{0.889021in}}%
\pgfpathlineto{\pgfqpoint{0.836888in}{0.907853in}}%
\pgfpathlineto{\pgfqpoint{0.893650in}{0.931109in}}%
\pgfpathlineto{\pgfqpoint{0.950412in}{0.957864in}}%
\pgfpathlineto{\pgfqpoint{1.007174in}{0.987371in}}%
\pgfpathlineto{\pgfqpoint{1.063936in}{1.019049in}}%
\pgfpathlineto{\pgfqpoint{1.120698in}{1.052459in}}%
\pgfpathlineto{\pgfqpoint{1.177460in}{1.087268in}}%
\pgfpathlineto{\pgfqpoint{1.234222in}{1.123221in}}%
\pgfpathlineto{\pgfqpoint{1.290983in}{1.160120in}}%
\pgfpathlineto{\pgfqpoint{1.347745in}{1.197812in}}%
\pgfpathlineto{\pgfqpoint{1.404507in}{1.236175in}}%
\pgfpathlineto{\pgfqpoint{1.461269in}{1.275111in}}%
\pgfpathlineto{\pgfqpoint{1.518031in}{1.314541in}}%
\pgfpathlineto{\pgfqpoint{1.574793in}{1.354399in}}%
\pgfpathlineto{\pgfqpoint{1.631555in}{1.394632in}}%
\pgfpathlineto{\pgfqpoint{1.688317in}{1.435193in}}%
\pgfpathlineto{\pgfqpoint{1.745079in}{1.476045in}}%
\pgfpathlineto{\pgfqpoint{1.801841in}{1.517156in}}%
\pgfpathlineto{\pgfqpoint{1.858602in}{1.558497in}}%
\pgfpathlineto{\pgfqpoint{1.915364in}{1.600044in}}%
\pgfpathlineto{\pgfqpoint{1.972126in}{1.641778in}}%
\pgfpathlineto{\pgfqpoint{2.028888in}{1.683680in}}%
\pgfpathlineto{\pgfqpoint{2.085650in}{1.725734in}}%
\pgfpathlineto{\pgfqpoint{2.142412in}{1.767927in}}%
\pgfpathlineto{\pgfqpoint{2.199174in}{1.810245in}}%
\pgfpathlineto{\pgfqpoint{2.255936in}{1.852679in}}%
\pgfpathlineto{\pgfqpoint{2.256379in}{1.853012in}}%
\pgfusepath{stroke}%
\end{pgfscope}%
\begin{pgfscope}%
\pgfpathrectangle{\pgfqpoint{0.638581in}{0.526079in}}{\pgfqpoint{2.241736in}{1.313043in}} %
\pgfusepath{clip}%
\pgfsetbuttcap%
\pgfsetroundjoin%
\pgfsetlinewidth{1.003750pt}%
\definecolor{currentstroke}{rgb}{0.000000,0.000000,0.000000}%
\pgfsetstrokecolor{currentstroke}%
\pgfsetdash{{3.700000pt}{1.600000pt}}{0.000000pt}%
\pgfpathmoveto{\pgfqpoint{0.666603in}{1.086975in}}%
\pgfpathlineto{\pgfqpoint{0.723365in}{1.089613in}}%
\pgfpathlineto{\pgfqpoint{0.780126in}{1.094908in}}%
\pgfpathlineto{\pgfqpoint{0.836888in}{1.102857in}}%
\pgfpathlineto{\pgfqpoint{0.893650in}{1.113448in}}%
\pgfpathlineto{\pgfqpoint{0.950412in}{1.126656in}}%
\pgfpathlineto{\pgfqpoint{1.007174in}{1.142431in}}%
\pgfpathlineto{\pgfqpoint{1.063936in}{1.160699in}}%
\pgfpathlineto{\pgfqpoint{1.120698in}{1.181355in}}%
\pgfpathlineto{\pgfqpoint{1.177460in}{1.204269in}}%
\pgfpathlineto{\pgfqpoint{1.234222in}{1.229289in}}%
\pgfpathlineto{\pgfqpoint{1.290983in}{1.256249in}}%
\pgfpathlineto{\pgfqpoint{1.347745in}{1.284976in}}%
\pgfpathlineto{\pgfqpoint{1.404507in}{1.315297in}}%
\pgfpathlineto{\pgfqpoint{1.461269in}{1.347046in}}%
\pgfpathlineto{\pgfqpoint{1.518031in}{1.380067in}}%
\pgfpathlineto{\pgfqpoint{1.574793in}{1.414215in}}%
\pgfpathlineto{\pgfqpoint{1.631555in}{1.449360in}}%
\pgfpathlineto{\pgfqpoint{1.688317in}{1.485386in}}%
\pgfpathlineto{\pgfqpoint{1.745079in}{1.522191in}}%
\pgfpathlineto{\pgfqpoint{1.801841in}{1.559682in}}%
\pgfpathlineto{\pgfqpoint{1.858602in}{1.597781in}}%
\pgfpathlineto{\pgfqpoint{1.915364in}{1.636418in}}%
\pgfpathlineto{\pgfqpoint{1.972126in}{1.675534in}}%
\pgfpathlineto{\pgfqpoint{2.028888in}{1.715074in}}%
\pgfpathlineto{\pgfqpoint{2.085650in}{1.754993in}}%
\pgfpathlineto{\pgfqpoint{2.142412in}{1.795252in}}%
\pgfpathlineto{\pgfqpoint{2.199174in}{1.835813in}}%
\pgfpathlineto{\pgfqpoint{2.223080in}{1.853012in}}%
\pgfusepath{stroke}%
\end{pgfscope}%
\begin{pgfscope}%
\pgfsetrectcap%
\pgfsetmiterjoin%
\pgfsetlinewidth{0.803000pt}%
\definecolor{currentstroke}{rgb}{0.000000,0.000000,0.000000}%
\pgfsetstrokecolor{currentstroke}%
\pgfsetdash{}{0pt}%
\pgfpathmoveto{\pgfqpoint{0.638581in}{0.526079in}}%
\pgfpathlineto{\pgfqpoint{0.638581in}{1.839123in}}%
\pgfusepath{stroke}%
\end{pgfscope}%
\begin{pgfscope}%
\pgfsetrectcap%
\pgfsetmiterjoin%
\pgfsetlinewidth{0.803000pt}%
\definecolor{currentstroke}{rgb}{0.000000,0.000000,0.000000}%
\pgfsetstrokecolor{currentstroke}%
\pgfsetdash{}{0pt}%
\pgfpathmoveto{\pgfqpoint{2.880317in}{0.526079in}}%
\pgfpathlineto{\pgfqpoint{2.880317in}{1.839123in}}%
\pgfusepath{stroke}%
\end{pgfscope}%
\begin{pgfscope}%
\pgfsetrectcap%
\pgfsetmiterjoin%
\pgfsetlinewidth{0.803000pt}%
\definecolor{currentstroke}{rgb}{0.000000,0.000000,0.000000}%
\pgfsetstrokecolor{currentstroke}%
\pgfsetdash{}{0pt}%
\pgfpathmoveto{\pgfqpoint{0.638581in}{0.526079in}}%
\pgfpathlineto{\pgfqpoint{2.880317in}{0.526079in}}%
\pgfusepath{stroke}%
\end{pgfscope}%
\begin{pgfscope}%
\pgfsetrectcap%
\pgfsetmiterjoin%
\pgfsetlinewidth{0.803000pt}%
\definecolor{currentstroke}{rgb}{0.000000,0.000000,0.000000}%
\pgfsetstrokecolor{currentstroke}%
\pgfsetdash{}{0pt}%
\pgfpathmoveto{\pgfqpoint{0.638581in}{1.839123in}}%
\pgfpathlineto{\pgfqpoint{2.880317in}{1.839123in}}%
\pgfusepath{stroke}%
\end{pgfscope}%
\begin{pgfscope}%
\pgftext[x=0.722646in,y=1.664050in,left,base]{\rmfamily\fontsize{10.000000}{12.000000}\selectfont (c)}%
\end{pgfscope}%
\begin{pgfscope}%
\pgfsys@transformshift{3.375000in}{0.536596in}%
\pgftext[left,bottom]{\pgfimage[interpolate=true,width=2.152778in,height=1.305556in]{Spectra_1e5-img3.png}}%
\end{pgfscope}%
\begin{pgfscope}%
\pgfsetbuttcap%
\pgfsetroundjoin%
\definecolor{currentfill}{rgb}{0.000000,0.000000,0.000000}%
\pgfsetfillcolor{currentfill}%
\pgfsetlinewidth{0.803000pt}%
\definecolor{currentstroke}{rgb}{0.000000,0.000000,0.000000}%
\pgfsetstrokecolor{currentstroke}%
\pgfsetdash{}{0pt}%
\pgfsys@defobject{currentmarker}{\pgfqpoint{0.000000in}{-0.048611in}}{\pgfqpoint{0.000000in}{0.000000in}}{%
\pgfpathmoveto{\pgfqpoint{0.000000in}{0.000000in}}%
\pgfpathlineto{\pgfqpoint{0.000000in}{-0.048611in}}%
\pgfusepath{stroke,fill}%
}%
\begin{pgfscope}%
\pgfsys@transformshift{3.373498in}{0.526079in}%
\pgfsys@useobject{currentmarker}{}%
\end{pgfscope}%
\end{pgfscope}%
\begin{pgfscope}%
\pgftext[x=3.373498in,y=0.428857in,,top]{\rmfamily\fontsize{10.000000}{12.000000}\selectfont \(\displaystyle 0\)}%
\end{pgfscope}%
\begin{pgfscope}%
\pgfsetbuttcap%
\pgfsetroundjoin%
\definecolor{currentfill}{rgb}{0.000000,0.000000,0.000000}%
\pgfsetfillcolor{currentfill}%
\pgfsetlinewidth{0.803000pt}%
\definecolor{currentstroke}{rgb}{0.000000,0.000000,0.000000}%
\pgfsetstrokecolor{currentstroke}%
\pgfsetdash{}{0pt}%
\pgfsys@defobject{currentmarker}{\pgfqpoint{0.000000in}{-0.048611in}}{\pgfqpoint{0.000000in}{0.000000in}}{%
\pgfpathmoveto{\pgfqpoint{0.000000in}{0.000000in}}%
\pgfpathlineto{\pgfqpoint{0.000000in}{-0.048611in}}%
\pgfusepath{stroke,fill}%
}%
\begin{pgfscope}%
\pgfsys@transformshift{3.911515in}{0.526079in}%
\pgfsys@useobject{currentmarker}{}%
\end{pgfscope}%
\end{pgfscope}%
\begin{pgfscope}%
\pgftext[x=3.911515in,y=0.428857in,,top]{\rmfamily\fontsize{10.000000}{12.000000}\selectfont \(\displaystyle 2\)}%
\end{pgfscope}%
\begin{pgfscope}%
\pgfsetbuttcap%
\pgfsetroundjoin%
\definecolor{currentfill}{rgb}{0.000000,0.000000,0.000000}%
\pgfsetfillcolor{currentfill}%
\pgfsetlinewidth{0.803000pt}%
\definecolor{currentstroke}{rgb}{0.000000,0.000000,0.000000}%
\pgfsetstrokecolor{currentstroke}%
\pgfsetdash{}{0pt}%
\pgfsys@defobject{currentmarker}{\pgfqpoint{0.000000in}{-0.048611in}}{\pgfqpoint{0.000000in}{0.000000in}}{%
\pgfpathmoveto{\pgfqpoint{0.000000in}{0.000000in}}%
\pgfpathlineto{\pgfqpoint{0.000000in}{-0.048611in}}%
\pgfusepath{stroke,fill}%
}%
\begin{pgfscope}%
\pgfsys@transformshift{4.449531in}{0.526079in}%
\pgfsys@useobject{currentmarker}{}%
\end{pgfscope}%
\end{pgfscope}%
\begin{pgfscope}%
\pgftext[x=4.449531in,y=0.428857in,,top]{\rmfamily\fontsize{10.000000}{12.000000}\selectfont \(\displaystyle 4\)}%
\end{pgfscope}%
\begin{pgfscope}%
\pgfsetbuttcap%
\pgfsetroundjoin%
\definecolor{currentfill}{rgb}{0.000000,0.000000,0.000000}%
\pgfsetfillcolor{currentfill}%
\pgfsetlinewidth{0.803000pt}%
\definecolor{currentstroke}{rgb}{0.000000,0.000000,0.000000}%
\pgfsetstrokecolor{currentstroke}%
\pgfsetdash{}{0pt}%
\pgfsys@defobject{currentmarker}{\pgfqpoint{0.000000in}{-0.048611in}}{\pgfqpoint{0.000000in}{0.000000in}}{%
\pgfpathmoveto{\pgfqpoint{0.000000in}{0.000000in}}%
\pgfpathlineto{\pgfqpoint{0.000000in}{-0.048611in}}%
\pgfusepath{stroke,fill}%
}%
\begin{pgfscope}%
\pgfsys@transformshift{4.987548in}{0.526079in}%
\pgfsys@useobject{currentmarker}{}%
\end{pgfscope}%
\end{pgfscope}%
\begin{pgfscope}%
\pgftext[x=4.987548in,y=0.428857in,,top]{\rmfamily\fontsize{10.000000}{12.000000}\selectfont \(\displaystyle 6\)}%
\end{pgfscope}%
\begin{pgfscope}%
\pgfsetbuttcap%
\pgfsetroundjoin%
\definecolor{currentfill}{rgb}{0.000000,0.000000,0.000000}%
\pgfsetfillcolor{currentfill}%
\pgfsetlinewidth{0.803000pt}%
\definecolor{currentstroke}{rgb}{0.000000,0.000000,0.000000}%
\pgfsetstrokecolor{currentstroke}%
\pgfsetdash{}{0pt}%
\pgfsys@defobject{currentmarker}{\pgfqpoint{0.000000in}{-0.048611in}}{\pgfqpoint{0.000000in}{0.000000in}}{%
\pgfpathmoveto{\pgfqpoint{0.000000in}{0.000000in}}%
\pgfpathlineto{\pgfqpoint{0.000000in}{-0.048611in}}%
\pgfusepath{stroke,fill}%
}%
\begin{pgfscope}%
\pgfsys@transformshift{5.525564in}{0.526079in}%
\pgfsys@useobject{currentmarker}{}%
\end{pgfscope}%
\end{pgfscope}%
\begin{pgfscope}%
\pgftext[x=5.525564in,y=0.428857in,,top]{\rmfamily\fontsize{10.000000}{12.000000}\selectfont \(\displaystyle 8\)}%
\end{pgfscope}%
\begin{pgfscope}%
\pgftext[x=4.449531in,y=0.238889in,,top]{\rmfamily\fontsize{10.000000}{12.000000}\selectfont \(\displaystyle kc/ |\Omega_\mathrm{ce}|\)}%
\end{pgfscope}%
\begin{pgfscope}%
\pgfsetbuttcap%
\pgfsetroundjoin%
\definecolor{currentfill}{rgb}{0.000000,0.000000,0.000000}%
\pgfsetfillcolor{currentfill}%
\pgfsetlinewidth{0.803000pt}%
\definecolor{currentstroke}{rgb}{0.000000,0.000000,0.000000}%
\pgfsetstrokecolor{currentstroke}%
\pgfsetdash{}{0pt}%
\pgfsys@defobject{currentmarker}{\pgfqpoint{-0.048611in}{0.000000in}}{\pgfqpoint{0.000000in}{0.000000in}}{%
\pgfpathmoveto{\pgfqpoint{0.000000in}{0.000000in}}%
\pgfpathlineto{\pgfqpoint{-0.048611in}{0.000000in}}%
\pgfusepath{stroke,fill}%
}%
\begin{pgfscope}%
\pgfsys@transformshift{3.373498in}{0.526079in}%
\pgfsys@useobject{currentmarker}{}%
\end{pgfscope}%
\end{pgfscope}%
\begin{pgfscope}%
\pgfsetbuttcap%
\pgfsetroundjoin%
\definecolor{currentfill}{rgb}{0.000000,0.000000,0.000000}%
\pgfsetfillcolor{currentfill}%
\pgfsetlinewidth{0.803000pt}%
\definecolor{currentstroke}{rgb}{0.000000,0.000000,0.000000}%
\pgfsetstrokecolor{currentstroke}%
\pgfsetdash{}{0pt}%
\pgfsys@defobject{currentmarker}{\pgfqpoint{-0.048611in}{0.000000in}}{\pgfqpoint{0.000000in}{0.000000in}}{%
\pgfpathmoveto{\pgfqpoint{0.000000in}{0.000000in}}%
\pgfpathlineto{\pgfqpoint{-0.048611in}{0.000000in}}%
\pgfusepath{stroke,fill}%
}%
\begin{pgfscope}%
\pgfsys@transformshift{3.373498in}{0.963761in}%
\pgfsys@useobject{currentmarker}{}%
\end{pgfscope}%
\end{pgfscope}%
\begin{pgfscope}%
\pgfsetbuttcap%
\pgfsetroundjoin%
\definecolor{currentfill}{rgb}{0.000000,0.000000,0.000000}%
\pgfsetfillcolor{currentfill}%
\pgfsetlinewidth{0.803000pt}%
\definecolor{currentstroke}{rgb}{0.000000,0.000000,0.000000}%
\pgfsetstrokecolor{currentstroke}%
\pgfsetdash{}{0pt}%
\pgfsys@defobject{currentmarker}{\pgfqpoint{-0.048611in}{0.000000in}}{\pgfqpoint{0.000000in}{0.000000in}}{%
\pgfpathmoveto{\pgfqpoint{0.000000in}{0.000000in}}%
\pgfpathlineto{\pgfqpoint{-0.048611in}{0.000000in}}%
\pgfusepath{stroke,fill}%
}%
\begin{pgfscope}%
\pgfsys@transformshift{3.373498in}{1.401442in}%
\pgfsys@useobject{currentmarker}{}%
\end{pgfscope}%
\end{pgfscope}%
\begin{pgfscope}%
\pgfsetbuttcap%
\pgfsetroundjoin%
\definecolor{currentfill}{rgb}{0.000000,0.000000,0.000000}%
\pgfsetfillcolor{currentfill}%
\pgfsetlinewidth{0.803000pt}%
\definecolor{currentstroke}{rgb}{0.000000,0.000000,0.000000}%
\pgfsetstrokecolor{currentstroke}%
\pgfsetdash{}{0pt}%
\pgfsys@defobject{currentmarker}{\pgfqpoint{-0.048611in}{0.000000in}}{\pgfqpoint{0.000000in}{0.000000in}}{%
\pgfpathmoveto{\pgfqpoint{0.000000in}{0.000000in}}%
\pgfpathlineto{\pgfqpoint{-0.048611in}{0.000000in}}%
\pgfusepath{stroke,fill}%
}%
\begin{pgfscope}%
\pgfsys@transformshift{3.373498in}{1.839123in}%
\pgfsys@useobject{currentmarker}{}%
\end{pgfscope}%
\end{pgfscope}%
\begin{pgfscope}%
\pgfpathrectangle{\pgfqpoint{3.373498in}{0.526079in}}{\pgfqpoint{2.152066in}{1.313043in}} %
\pgfusepath{clip}%
\pgfsetbuttcap%
\pgfsetroundjoin%
\pgfsetlinewidth{1.003750pt}%
\definecolor{currentstroke}{rgb}{0.000000,0.000000,0.000000}%
\pgfsetstrokecolor{currentstroke}%
\pgfsetdash{{3.700000pt}{1.600000pt}}{0.000000pt}%
\pgfpathmoveto{\pgfqpoint{3.400399in}{0.526625in}}%
\pgfpathlineto{\pgfqpoint{3.454891in}{0.530950in}}%
\pgfpathlineto{\pgfqpoint{3.509382in}{0.539032in}}%
\pgfpathlineto{\pgfqpoint{3.563873in}{0.549917in}}%
\pgfpathlineto{\pgfqpoint{3.618365in}{0.562581in}}%
\pgfpathlineto{\pgfqpoint{3.672856in}{0.576129in}}%
\pgfpathlineto{\pgfqpoint{3.727348in}{0.589860in}}%
\pgfpathlineto{\pgfqpoint{3.781839in}{0.603270in}}%
\pgfpathlineto{\pgfqpoint{3.836330in}{0.616024in}}%
\pgfpathlineto{\pgfqpoint{3.890822in}{0.627919in}}%
\pgfpathlineto{\pgfqpoint{3.945313in}{0.638851in}}%
\pgfpathlineto{\pgfqpoint{3.999805in}{0.648791in}}%
\pgfpathlineto{\pgfqpoint{4.054296in}{0.657756in}}%
\pgfpathlineto{\pgfqpoint{4.108788in}{0.665798in}}%
\pgfpathlineto{\pgfqpoint{4.163279in}{0.672985in}}%
\pgfpathlineto{\pgfqpoint{4.217770in}{0.679395in}}%
\pgfpathlineto{\pgfqpoint{4.272262in}{0.685105in}}%
\pgfpathlineto{\pgfqpoint{4.326753in}{0.690192in}}%
\pgfpathlineto{\pgfqpoint{4.381245in}{0.694727in}}%
\pgfpathlineto{\pgfqpoint{4.435736in}{0.698775in}}%
\pgfpathlineto{\pgfqpoint{4.490228in}{0.702394in}}%
\pgfpathlineto{\pgfqpoint{4.544719in}{0.705636in}}%
\pgfpathlineto{\pgfqpoint{4.599210in}{0.708546in}}%
\pgfpathlineto{\pgfqpoint{4.653702in}{0.711165in}}%
\pgfpathlineto{\pgfqpoint{4.708193in}{0.713526in}}%
\pgfpathlineto{\pgfqpoint{4.762685in}{0.715661in}}%
\pgfpathlineto{\pgfqpoint{4.817176in}{0.717595in}}%
\pgfpathlineto{\pgfqpoint{4.871667in}{0.719352in}}%
\pgfpathlineto{\pgfqpoint{4.926159in}{0.720951in}}%
\pgfpathlineto{\pgfqpoint{4.980650in}{0.722410in}}%
\pgfpathlineto{\pgfqpoint{5.035142in}{0.723745in}}%
\pgfpathlineto{\pgfqpoint{5.089633in}{0.724967in}}%
\pgfpathlineto{\pgfqpoint{5.144125in}{0.726090in}}%
\pgfpathlineto{\pgfqpoint{5.198616in}{0.727123in}}%
\pgfpathlineto{\pgfqpoint{5.253107in}{0.728076in}}%
\pgfpathlineto{\pgfqpoint{5.307599in}{0.728956in}}%
\pgfpathlineto{\pgfqpoint{5.362090in}{0.729770in}}%
\pgfpathlineto{\pgfqpoint{5.416582in}{0.730525in}}%
\pgfpathlineto{\pgfqpoint{5.471073in}{0.731226in}}%
\pgfpathlineto{\pgfqpoint{5.525564in}{0.731877in}}%
\pgfusepath{stroke}%
\end{pgfscope}%
\begin{pgfscope}%
\pgfpathrectangle{\pgfqpoint{3.373498in}{0.526079in}}{\pgfqpoint{2.152066in}{1.313043in}} %
\pgfusepath{clip}%
\pgfsetbuttcap%
\pgfsetroundjoin%
\pgfsetlinewidth{1.003750pt}%
\definecolor{currentstroke}{rgb}{0.000000,0.000000,0.000000}%
\pgfsetstrokecolor{currentstroke}%
\pgfsetdash{{3.700000pt}{1.600000pt}}{0.000000pt}%
\pgfpathmoveto{\pgfqpoint{3.400399in}{0.868680in}}%
\pgfpathlineto{\pgfqpoint{3.454891in}{0.875643in}}%
\pgfpathlineto{\pgfqpoint{3.509382in}{0.889021in}}%
\pgfpathlineto{\pgfqpoint{3.563873in}{0.907853in}}%
\pgfpathlineto{\pgfqpoint{3.618365in}{0.931109in}}%
\pgfpathlineto{\pgfqpoint{3.672856in}{0.957864in}}%
\pgfpathlineto{\pgfqpoint{3.727348in}{0.987371in}}%
\pgfpathlineto{\pgfqpoint{3.781839in}{1.019049in}}%
\pgfpathlineto{\pgfqpoint{3.836330in}{1.052459in}}%
\pgfpathlineto{\pgfqpoint{3.890822in}{1.087268in}}%
\pgfpathlineto{\pgfqpoint{3.945313in}{1.123221in}}%
\pgfpathlineto{\pgfqpoint{3.999805in}{1.160120in}}%
\pgfpathlineto{\pgfqpoint{4.054296in}{1.197812in}}%
\pgfpathlineto{\pgfqpoint{4.108788in}{1.236175in}}%
\pgfpathlineto{\pgfqpoint{4.163279in}{1.275111in}}%
\pgfpathlineto{\pgfqpoint{4.217770in}{1.314541in}}%
\pgfpathlineto{\pgfqpoint{4.272262in}{1.354399in}}%
\pgfpathlineto{\pgfqpoint{4.326753in}{1.394632in}}%
\pgfpathlineto{\pgfqpoint{4.381245in}{1.435193in}}%
\pgfpathlineto{\pgfqpoint{4.435736in}{1.476045in}}%
\pgfpathlineto{\pgfqpoint{4.490228in}{1.517156in}}%
\pgfpathlineto{\pgfqpoint{4.544719in}{1.558497in}}%
\pgfpathlineto{\pgfqpoint{4.599210in}{1.600044in}}%
\pgfpathlineto{\pgfqpoint{4.653702in}{1.641778in}}%
\pgfpathlineto{\pgfqpoint{4.708193in}{1.683680in}}%
\pgfpathlineto{\pgfqpoint{4.762685in}{1.725734in}}%
\pgfpathlineto{\pgfqpoint{4.817176in}{1.767927in}}%
\pgfpathlineto{\pgfqpoint{4.871667in}{1.810245in}}%
\pgfpathlineto{\pgfqpoint{4.926159in}{1.852679in}}%
\pgfpathlineto{\pgfqpoint{4.926585in}{1.853012in}}%
\pgfusepath{stroke}%
\end{pgfscope}%
\begin{pgfscope}%
\pgfpathrectangle{\pgfqpoint{3.373498in}{0.526079in}}{\pgfqpoint{2.152066in}{1.313043in}} %
\pgfusepath{clip}%
\pgfsetbuttcap%
\pgfsetroundjoin%
\pgfsetlinewidth{1.003750pt}%
\definecolor{currentstroke}{rgb}{0.000000,0.000000,0.000000}%
\pgfsetstrokecolor{currentstroke}%
\pgfsetdash{{3.700000pt}{1.600000pt}}{0.000000pt}%
\pgfpathmoveto{\pgfqpoint{3.400399in}{1.086975in}}%
\pgfpathlineto{\pgfqpoint{3.454891in}{1.089613in}}%
\pgfpathlineto{\pgfqpoint{3.509382in}{1.094908in}}%
\pgfpathlineto{\pgfqpoint{3.563873in}{1.102857in}}%
\pgfpathlineto{\pgfqpoint{3.618365in}{1.113448in}}%
\pgfpathlineto{\pgfqpoint{3.672856in}{1.126656in}}%
\pgfpathlineto{\pgfqpoint{3.727348in}{1.142431in}}%
\pgfpathlineto{\pgfqpoint{3.781839in}{1.160699in}}%
\pgfpathlineto{\pgfqpoint{3.836330in}{1.181355in}}%
\pgfpathlineto{\pgfqpoint{3.890822in}{1.204269in}}%
\pgfpathlineto{\pgfqpoint{3.945313in}{1.229289in}}%
\pgfpathlineto{\pgfqpoint{3.999805in}{1.256249in}}%
\pgfpathlineto{\pgfqpoint{4.054296in}{1.284976in}}%
\pgfpathlineto{\pgfqpoint{4.108788in}{1.315297in}}%
\pgfpathlineto{\pgfqpoint{4.163279in}{1.347046in}}%
\pgfpathlineto{\pgfqpoint{4.217770in}{1.380067in}}%
\pgfpathlineto{\pgfqpoint{4.272262in}{1.414215in}}%
\pgfpathlineto{\pgfqpoint{4.326753in}{1.449360in}}%
\pgfpathlineto{\pgfqpoint{4.381245in}{1.485386in}}%
\pgfpathlineto{\pgfqpoint{4.435736in}{1.522191in}}%
\pgfpathlineto{\pgfqpoint{4.490228in}{1.559682in}}%
\pgfpathlineto{\pgfqpoint{4.544719in}{1.597781in}}%
\pgfpathlineto{\pgfqpoint{4.599210in}{1.636418in}}%
\pgfpathlineto{\pgfqpoint{4.653702in}{1.675534in}}%
\pgfpathlineto{\pgfqpoint{4.708193in}{1.715074in}}%
\pgfpathlineto{\pgfqpoint{4.762685in}{1.754993in}}%
\pgfpathlineto{\pgfqpoint{4.817176in}{1.795252in}}%
\pgfpathlineto{\pgfqpoint{4.871667in}{1.835813in}}%
\pgfpathlineto{\pgfqpoint{4.894617in}{1.853012in}}%
\pgfusepath{stroke}%
\end{pgfscope}%
\begin{pgfscope}%
\pgfsetrectcap%
\pgfsetmiterjoin%
\pgfsetlinewidth{0.803000pt}%
\definecolor{currentstroke}{rgb}{0.000000,0.000000,0.000000}%
\pgfsetstrokecolor{currentstroke}%
\pgfsetdash{}{0pt}%
\pgfpathmoveto{\pgfqpoint{3.373498in}{0.526079in}}%
\pgfpathlineto{\pgfqpoint{3.373498in}{1.839123in}}%
\pgfusepath{stroke}%
\end{pgfscope}%
\begin{pgfscope}%
\pgfsetrectcap%
\pgfsetmiterjoin%
\pgfsetlinewidth{0.803000pt}%
\definecolor{currentstroke}{rgb}{0.000000,0.000000,0.000000}%
\pgfsetstrokecolor{currentstroke}%
\pgfsetdash{}{0pt}%
\pgfpathmoveto{\pgfqpoint{5.525564in}{0.526079in}}%
\pgfpathlineto{\pgfqpoint{5.525564in}{1.839123in}}%
\pgfusepath{stroke}%
\end{pgfscope}%
\begin{pgfscope}%
\pgfsetrectcap%
\pgfsetmiterjoin%
\pgfsetlinewidth{0.803000pt}%
\definecolor{currentstroke}{rgb}{0.000000,0.000000,0.000000}%
\pgfsetstrokecolor{currentstroke}%
\pgfsetdash{}{0pt}%
\pgfpathmoveto{\pgfqpoint{3.373498in}{0.526079in}}%
\pgfpathlineto{\pgfqpoint{5.525564in}{0.526079in}}%
\pgfusepath{stroke}%
\end{pgfscope}%
\begin{pgfscope}%
\pgfsetrectcap%
\pgfsetmiterjoin%
\pgfsetlinewidth{0.803000pt}%
\definecolor{currentstroke}{rgb}{0.000000,0.000000,0.000000}%
\pgfsetstrokecolor{currentstroke}%
\pgfsetdash{}{0pt}%
\pgfpathmoveto{\pgfqpoint{3.373498in}{1.839123in}}%
\pgfpathlineto{\pgfqpoint{5.525564in}{1.839123in}}%
\pgfusepath{stroke}%
\end{pgfscope}%
\begin{pgfscope}%
\pgftext[x=3.454201in,y=1.664050in,left,base]{\rmfamily\fontsize{10.000000}{12.000000}\selectfont (d)}%
\end{pgfscope}%
\begin{pgfscope}%
\pgfpathrectangle{\pgfqpoint{5.660069in}{2.233036in}}{\pgfqpoint{0.065652in}{1.313043in}} %
\pgfusepath{clip}%
\pgfsetbuttcap%
\pgfsetmiterjoin%
\definecolor{currentfill}{rgb}{1.000000,1.000000,1.000000}%
\pgfsetfillcolor{currentfill}%
\pgfsetlinewidth{0.010037pt}%
\definecolor{currentstroke}{rgb}{1.000000,1.000000,1.000000}%
\pgfsetstrokecolor{currentstroke}%
\pgfsetdash{}{0pt}%
\pgfpathmoveto{\pgfqpoint{5.660069in}{2.233036in}}%
\pgfpathlineto{\pgfqpoint{5.660069in}{2.255291in}}%
\pgfpathlineto{\pgfqpoint{5.660069in}{3.523824in}}%
\pgfpathlineto{\pgfqpoint{5.660069in}{3.546079in}}%
\pgfpathlineto{\pgfqpoint{5.725721in}{3.546079in}}%
\pgfpathlineto{\pgfqpoint{5.725721in}{3.523824in}}%
\pgfpathlineto{\pgfqpoint{5.725721in}{2.255291in}}%
\pgfpathlineto{\pgfqpoint{5.725721in}{2.233036in}}%
\pgfpathclose%
\pgfusepath{stroke,fill}%
\end{pgfscope}%
\begin{pgfscope}%
\pgfsys@transformshift{5.666667in}{2.244929in}%
\pgftext[left,bottom]{\pgfimage[interpolate=true,width=0.055556in,height=1.305556in]{Spectra_1e5-img4.png}}%
\end{pgfscope}%
\begin{pgfscope}%
\pgfsetbuttcap%
\pgfsetroundjoin%
\definecolor{currentfill}{rgb}{0.000000,0.000000,0.000000}%
\pgfsetfillcolor{currentfill}%
\pgfsetlinewidth{0.803000pt}%
\definecolor{currentstroke}{rgb}{0.000000,0.000000,0.000000}%
\pgfsetstrokecolor{currentstroke}%
\pgfsetdash{}{0pt}%
\pgfsys@defobject{currentmarker}{\pgfqpoint{0.000000in}{0.000000in}}{\pgfqpoint{0.048611in}{0.000000in}}{%
\pgfpathmoveto{\pgfqpoint{0.000000in}{0.000000in}}%
\pgfpathlineto{\pgfqpoint{0.048611in}{0.000000in}}%
\pgfusepath{stroke,fill}%
}%
\begin{pgfscope}%
\pgfsys@transformshift{5.725721in}{2.233036in}%
\pgfsys@useobject{currentmarker}{}%
\end{pgfscope}%
\end{pgfscope}%
\begin{pgfscope}%
\pgftext[x=5.822943in,y=2.180274in,left,base]{\rmfamily\fontsize{10.000000}{12.000000}\selectfont \(\displaystyle 10^{-8}\)}%
\end{pgfscope}%
\begin{pgfscope}%
\pgfsetbuttcap%
\pgfsetroundjoin%
\definecolor{currentfill}{rgb}{0.000000,0.000000,0.000000}%
\pgfsetfillcolor{currentfill}%
\pgfsetlinewidth{0.803000pt}%
\definecolor{currentstroke}{rgb}{0.000000,0.000000,0.000000}%
\pgfsetstrokecolor{currentstroke}%
\pgfsetdash{}{0pt}%
\pgfsys@defobject{currentmarker}{\pgfqpoint{0.000000in}{0.000000in}}{\pgfqpoint{0.048611in}{0.000000in}}{%
\pgfpathmoveto{\pgfqpoint{0.000000in}{0.000000in}}%
\pgfpathlineto{\pgfqpoint{0.048611in}{0.000000in}}%
\pgfusepath{stroke,fill}%
}%
\begin{pgfscope}%
\pgfsys@transformshift{5.725721in}{2.561297in}%
\pgfsys@useobject{currentmarker}{}%
\end{pgfscope}%
\end{pgfscope}%
\begin{pgfscope}%
\pgftext[x=5.822943in,y=2.508535in,left,base]{\rmfamily\fontsize{10.000000}{12.000000}\selectfont \(\displaystyle 10^{-6}\)}%
\end{pgfscope}%
\begin{pgfscope}%
\pgfsetbuttcap%
\pgfsetroundjoin%
\definecolor{currentfill}{rgb}{0.000000,0.000000,0.000000}%
\pgfsetfillcolor{currentfill}%
\pgfsetlinewidth{0.803000pt}%
\definecolor{currentstroke}{rgb}{0.000000,0.000000,0.000000}%
\pgfsetstrokecolor{currentstroke}%
\pgfsetdash{}{0pt}%
\pgfsys@defobject{currentmarker}{\pgfqpoint{0.000000in}{0.000000in}}{\pgfqpoint{0.048611in}{0.000000in}}{%
\pgfpathmoveto{\pgfqpoint{0.000000in}{0.000000in}}%
\pgfpathlineto{\pgfqpoint{0.048611in}{0.000000in}}%
\pgfusepath{stroke,fill}%
}%
\begin{pgfscope}%
\pgfsys@transformshift{5.725721in}{2.889558in}%
\pgfsys@useobject{currentmarker}{}%
\end{pgfscope}%
\end{pgfscope}%
\begin{pgfscope}%
\pgftext[x=5.822943in,y=2.836796in,left,base]{\rmfamily\fontsize{10.000000}{12.000000}\selectfont \(\displaystyle 10^{-4}\)}%
\end{pgfscope}%
\begin{pgfscope}%
\pgfsetbuttcap%
\pgfsetroundjoin%
\definecolor{currentfill}{rgb}{0.000000,0.000000,0.000000}%
\pgfsetfillcolor{currentfill}%
\pgfsetlinewidth{0.803000pt}%
\definecolor{currentstroke}{rgb}{0.000000,0.000000,0.000000}%
\pgfsetstrokecolor{currentstroke}%
\pgfsetdash{}{0pt}%
\pgfsys@defobject{currentmarker}{\pgfqpoint{0.000000in}{0.000000in}}{\pgfqpoint{0.048611in}{0.000000in}}{%
\pgfpathmoveto{\pgfqpoint{0.000000in}{0.000000in}}%
\pgfpathlineto{\pgfqpoint{0.048611in}{0.000000in}}%
\pgfusepath{stroke,fill}%
}%
\begin{pgfscope}%
\pgfsys@transformshift{5.725721in}{3.217819in}%
\pgfsys@useobject{currentmarker}{}%
\end{pgfscope}%
\end{pgfscope}%
\begin{pgfscope}%
\pgftext[x=5.822943in,y=3.165057in,left,base]{\rmfamily\fontsize{10.000000}{12.000000}\selectfont \(\displaystyle 10^{-2}\)}%
\end{pgfscope}%
\begin{pgfscope}%
\pgfsetbuttcap%
\pgfsetroundjoin%
\definecolor{currentfill}{rgb}{0.000000,0.000000,0.000000}%
\pgfsetfillcolor{currentfill}%
\pgfsetlinewidth{0.803000pt}%
\definecolor{currentstroke}{rgb}{0.000000,0.000000,0.000000}%
\pgfsetstrokecolor{currentstroke}%
\pgfsetdash{}{0pt}%
\pgfsys@defobject{currentmarker}{\pgfqpoint{0.000000in}{0.000000in}}{\pgfqpoint{0.048611in}{0.000000in}}{%
\pgfpathmoveto{\pgfqpoint{0.000000in}{0.000000in}}%
\pgfpathlineto{\pgfqpoint{0.048611in}{0.000000in}}%
\pgfusepath{stroke,fill}%
}%
\begin{pgfscope}%
\pgfsys@transformshift{5.725721in}{3.546079in}%
\pgfsys@useobject{currentmarker}{}%
\end{pgfscope}%
\end{pgfscope}%
\begin{pgfscope}%
\pgftext[x=5.822943in,y=3.493318in,left,base]{\rmfamily\fontsize{10.000000}{12.000000}\selectfont \(\displaystyle 10^{0}\)}%
\end{pgfscope}%
\begin{pgfscope}%
\pgftext[x=6.166501in,y=2.889558in,,top,rotate=90.000000]{\rmfamily\fontsize{10.000000}{12.000000}\selectfont \(\displaystyle |\mathrm{DFT}(B_x)|/|\mathrm{DFT}(B_x)|_\mathrm{max}\)}%
\end{pgfscope}%
\begin{pgfscope}%
\pgfsetbuttcap%
\pgfsetmiterjoin%
\pgfsetlinewidth{0.803000pt}%
\definecolor{currentstroke}{rgb}{0.000000,0.000000,0.000000}%
\pgfsetstrokecolor{currentstroke}%
\pgfsetdash{}{0pt}%
\pgfpathmoveto{\pgfqpoint{5.660069in}{2.233036in}}%
\pgfpathlineto{\pgfqpoint{5.660069in}{2.255291in}}%
\pgfpathlineto{\pgfqpoint{5.660069in}{3.523824in}}%
\pgfpathlineto{\pgfqpoint{5.660069in}{3.546079in}}%
\pgfpathlineto{\pgfqpoint{5.725721in}{3.546079in}}%
\pgfpathlineto{\pgfqpoint{5.725721in}{3.523824in}}%
\pgfpathlineto{\pgfqpoint{5.725721in}{2.255291in}}%
\pgfpathlineto{\pgfqpoint{5.725721in}{2.233036in}}%
\pgfpathclose%
\pgfusepath{stroke}%
\end{pgfscope}%
\begin{pgfscope}%
\pgfpathrectangle{\pgfqpoint{5.660069in}{0.526079in}}{\pgfqpoint{0.065652in}{1.313043in}} %
\pgfusepath{clip}%
\pgfsetbuttcap%
\pgfsetmiterjoin%
\definecolor{currentfill}{rgb}{1.000000,1.000000,1.000000}%
\pgfsetfillcolor{currentfill}%
\pgfsetlinewidth{0.010037pt}%
\definecolor{currentstroke}{rgb}{1.000000,1.000000,1.000000}%
\pgfsetstrokecolor{currentstroke}%
\pgfsetdash{}{0pt}%
\pgfpathmoveto{\pgfqpoint{5.660069in}{0.526079in}}%
\pgfpathlineto{\pgfqpoint{5.660069in}{0.548334in}}%
\pgfpathlineto{\pgfqpoint{5.660069in}{1.816868in}}%
\pgfpathlineto{\pgfqpoint{5.660069in}{1.839123in}}%
\pgfpathlineto{\pgfqpoint{5.725721in}{1.839123in}}%
\pgfpathlineto{\pgfqpoint{5.725721in}{1.816868in}}%
\pgfpathlineto{\pgfqpoint{5.725721in}{0.548334in}}%
\pgfpathlineto{\pgfqpoint{5.725721in}{0.526079in}}%
\pgfpathclose%
\pgfusepath{stroke,fill}%
\end{pgfscope}%
\begin{pgfscope}%
\pgfsys@transformshift{5.666667in}{0.536596in}%
\pgftext[left,bottom]{\pgfimage[interpolate=true,width=0.055556in,height=1.305556in]{Spectra_1e5-img5.png}}%
\end{pgfscope}%
\begin{pgfscope}%
\pgfsetbuttcap%
\pgfsetroundjoin%
\definecolor{currentfill}{rgb}{0.000000,0.000000,0.000000}%
\pgfsetfillcolor{currentfill}%
\pgfsetlinewidth{0.803000pt}%
\definecolor{currentstroke}{rgb}{0.000000,0.000000,0.000000}%
\pgfsetstrokecolor{currentstroke}%
\pgfsetdash{}{0pt}%
\pgfsys@defobject{currentmarker}{\pgfqpoint{0.000000in}{0.000000in}}{\pgfqpoint{0.048611in}{0.000000in}}{%
\pgfpathmoveto{\pgfqpoint{0.000000in}{0.000000in}}%
\pgfpathlineto{\pgfqpoint{0.048611in}{0.000000in}}%
\pgfusepath{stroke,fill}%
}%
\begin{pgfscope}%
\pgfsys@transformshift{5.725721in}{0.526079in}%
\pgfsys@useobject{currentmarker}{}%
\end{pgfscope}%
\end{pgfscope}%
\begin{pgfscope}%
\pgftext[x=5.822943in,y=0.473318in,left,base]{\rmfamily\fontsize{10.000000}{12.000000}\selectfont \(\displaystyle 10^{-8}\)}%
\end{pgfscope}%
\begin{pgfscope}%
\pgfsetbuttcap%
\pgfsetroundjoin%
\definecolor{currentfill}{rgb}{0.000000,0.000000,0.000000}%
\pgfsetfillcolor{currentfill}%
\pgfsetlinewidth{0.803000pt}%
\definecolor{currentstroke}{rgb}{0.000000,0.000000,0.000000}%
\pgfsetstrokecolor{currentstroke}%
\pgfsetdash{}{0pt}%
\pgfsys@defobject{currentmarker}{\pgfqpoint{0.000000in}{0.000000in}}{\pgfqpoint{0.048611in}{0.000000in}}{%
\pgfpathmoveto{\pgfqpoint{0.000000in}{0.000000in}}%
\pgfpathlineto{\pgfqpoint{0.048611in}{0.000000in}}%
\pgfusepath{stroke,fill}%
}%
\begin{pgfscope}%
\pgfsys@transformshift{5.725721in}{0.854340in}%
\pgfsys@useobject{currentmarker}{}%
\end{pgfscope}%
\end{pgfscope}%
\begin{pgfscope}%
\pgftext[x=5.822943in,y=0.801579in,left,base]{\rmfamily\fontsize{10.000000}{12.000000}\selectfont \(\displaystyle 10^{-6}\)}%
\end{pgfscope}%
\begin{pgfscope}%
\pgfsetbuttcap%
\pgfsetroundjoin%
\definecolor{currentfill}{rgb}{0.000000,0.000000,0.000000}%
\pgfsetfillcolor{currentfill}%
\pgfsetlinewidth{0.803000pt}%
\definecolor{currentstroke}{rgb}{0.000000,0.000000,0.000000}%
\pgfsetstrokecolor{currentstroke}%
\pgfsetdash{}{0pt}%
\pgfsys@defobject{currentmarker}{\pgfqpoint{0.000000in}{0.000000in}}{\pgfqpoint{0.048611in}{0.000000in}}{%
\pgfpathmoveto{\pgfqpoint{0.000000in}{0.000000in}}%
\pgfpathlineto{\pgfqpoint{0.048611in}{0.000000in}}%
\pgfusepath{stroke,fill}%
}%
\begin{pgfscope}%
\pgfsys@transformshift{5.725721in}{1.182601in}%
\pgfsys@useobject{currentmarker}{}%
\end{pgfscope}%
\end{pgfscope}%
\begin{pgfscope}%
\pgftext[x=5.822943in,y=1.129840in,left,base]{\rmfamily\fontsize{10.000000}{12.000000}\selectfont \(\displaystyle 10^{-4}\)}%
\end{pgfscope}%
\begin{pgfscope}%
\pgfsetbuttcap%
\pgfsetroundjoin%
\definecolor{currentfill}{rgb}{0.000000,0.000000,0.000000}%
\pgfsetfillcolor{currentfill}%
\pgfsetlinewidth{0.803000pt}%
\definecolor{currentstroke}{rgb}{0.000000,0.000000,0.000000}%
\pgfsetstrokecolor{currentstroke}%
\pgfsetdash{}{0pt}%
\pgfsys@defobject{currentmarker}{\pgfqpoint{0.000000in}{0.000000in}}{\pgfqpoint{0.048611in}{0.000000in}}{%
\pgfpathmoveto{\pgfqpoint{0.000000in}{0.000000in}}%
\pgfpathlineto{\pgfqpoint{0.048611in}{0.000000in}}%
\pgfusepath{stroke,fill}%
}%
\begin{pgfscope}%
\pgfsys@transformshift{5.725721in}{1.510862in}%
\pgfsys@useobject{currentmarker}{}%
\end{pgfscope}%
\end{pgfscope}%
\begin{pgfscope}%
\pgftext[x=5.822943in,y=1.458101in,left,base]{\rmfamily\fontsize{10.000000}{12.000000}\selectfont \(\displaystyle 10^{-2}\)}%
\end{pgfscope}%
\begin{pgfscope}%
\pgfsetbuttcap%
\pgfsetroundjoin%
\definecolor{currentfill}{rgb}{0.000000,0.000000,0.000000}%
\pgfsetfillcolor{currentfill}%
\pgfsetlinewidth{0.803000pt}%
\definecolor{currentstroke}{rgb}{0.000000,0.000000,0.000000}%
\pgfsetstrokecolor{currentstroke}%
\pgfsetdash{}{0pt}%
\pgfsys@defobject{currentmarker}{\pgfqpoint{0.000000in}{0.000000in}}{\pgfqpoint{0.048611in}{0.000000in}}{%
\pgfpathmoveto{\pgfqpoint{0.000000in}{0.000000in}}%
\pgfpathlineto{\pgfqpoint{0.048611in}{0.000000in}}%
\pgfusepath{stroke,fill}%
}%
\begin{pgfscope}%
\pgfsys@transformshift{5.725721in}{1.839123in}%
\pgfsys@useobject{currentmarker}{}%
\end{pgfscope}%
\end{pgfscope}%
\begin{pgfscope}%
\pgftext[x=5.822943in,y=1.786361in,left,base]{\rmfamily\fontsize{10.000000}{12.000000}\selectfont \(\displaystyle 10^{0}\)}%
\end{pgfscope}%
\begin{pgfscope}%
\pgftext[x=6.166501in,y=1.182601in,,top,rotate=90.000000]{\rmfamily\fontsize{10.000000}{12.000000}\selectfont \(\displaystyle |\mathrm{DFT}(B_x)|/|\mathrm{DFT}(B_x)|_\mathrm{max}\)}%
\end{pgfscope}%
\begin{pgfscope}%
\pgfsetbuttcap%
\pgfsetmiterjoin%
\pgfsetlinewidth{0.803000pt}%
\definecolor{currentstroke}{rgb}{0.000000,0.000000,0.000000}%
\pgfsetstrokecolor{currentstroke}%
\pgfsetdash{}{0pt}%
\pgfpathmoveto{\pgfqpoint{5.660069in}{0.526079in}}%
\pgfpathlineto{\pgfqpoint{5.660069in}{0.548334in}}%
\pgfpathlineto{\pgfqpoint{5.660069in}{1.816868in}}%
\pgfpathlineto{\pgfqpoint{5.660069in}{1.839123in}}%
\pgfpathlineto{\pgfqpoint{5.725721in}{1.839123in}}%
\pgfpathlineto{\pgfqpoint{5.725721in}{1.816868in}}%
\pgfpathlineto{\pgfqpoint{5.725721in}{0.548334in}}%
\pgfpathlineto{\pgfqpoint{5.725721in}{0.526079in}}%
\pgfpathclose%
\pgfusepath{stroke}%
\end{pgfscope}%
\end{pgfpicture}%
\makeatother%
\endgroup%

\caption{Run 2 with parameters $\vpar=\vperp=0.1\,c$, $\nu_\mr{h}=0.002$, $\Omega_\mr{pe}=2|\Omega_\mr{ce}|$, $L=80c/|\Omega_\mr{ce}|$, $N_\mr{el}=512$, $p=1$, $N_\mr{p}=1\cdot 10^5$ and $\Delta t=0.05|\Omega_\mr{ce}|^{-1}$. The simulation was run until $t_\mr{f}=300|\Omega_\mr{ce}|$: (a) Normalized 2d Discrete Fourier Transform of the $x$-component of the magnetic field for standard finite element PIC. (b) Same as (a) for structure preserving finite element PIC. (c) Comparison of the spectrum (a) with the real part of the analytical dispersion relation (\ref{eq_dispersion_relation}). (d) Same as (c) for the spectrum (b).}
\end{figure}

\subsection{Run 2: Multiple $k$-modes}
So far we have initialized the code with a small perturbation of the $x$-component of the magnetic field for a single wavenumber $k$. Next, we want to excite multiple $k$-modes of the system at the same time. This can be achieved by directly using the fact that the random initialization of the particles in phase space induces a low-level noise in the system. In Fig. \ref{fig_comparison}, one can see the normalized two-dimensional Discrete Fourier transform (DFT) of a run
that has been initialized with a low density ($\nu_\mr{h}=0.002$), isotropic Maxwellian ($\vpar=\vperp=0.1\,c$) for the energetic electrons and no electromagnetic fields and cold current density. With this choice of parameters, there is no wave growth expected, however, by taking a look at the spectrum in the $k$-$\omega$-plane in Fig. \ref{fig_comparison}, we see that the particle noise leads to an excitation of all three characteristic waves (see Sec. \ref{sec_dispersion}) with a continuous spectrum in each quadrant. For both numerical methods, we obtain similar results for small wavenumbers and frequencies which we can compare to the real part of the dispersion relation (\ref{eq_dispersion_relation}) and for which we find a very good agreement. However, there is an obvious different behavior when it comes to higher wavenumbers. In case of standard PIC, the two branches corresponding to vacuum light waves ''bend down'' which is not the case for structure-preserving PIC. Although it also differs from the expected straight line representing the speed of light, there are no unphysical modes as for standard PIC. This is also true for the Whistler branch below the electron cyclotron frequency $\Omega_\mr{ce}$. Whereas there are unphysical modes with a rather large intensity (red) for the highest wavenumbers $k\approx 20|\Omega_\mr{ce}|/c$ for standard PIC, this is not the case for structure-preserving PIC. The reason for this qualitative behavior is not obvious and needs to be analyzed further. 


\section{Summary}
\label{sec_summary}
In this article, we have developed two different finite element particle-in-cell algorithms for a four-dimensional hybrid plasma model and compared the results for two test runs. The considered hybrid plasma model is a combined kinetic/fluid description for a magnetized plasma, which consists of cold (fluid) electrons and energetic (kinetic) electrons that move in a stationary, neutralizing background of ions. The model's key physics content for wave propagation parallel to a uniform background magnetic field is that it predicts the existence of growing/damped modes due to energy exchange between the energetic electrons and waves which propagate in the cold plasma.

For this case, first, a combination of one-dimensional B-spline finite elements for Maxwell's equations and the momentum balance equation for the cold electrons and the standard particle-in-cell method with a Boris particle pusher for the Vlasov equation (one dimension in real space and three dimensions in velocity space) has been applied in an intuitive way without taking into account the geometric structure of the equations. Second, geometric finite element particle-in-cell methods \citep{Krausetal2017} which use tools from \textit{finite element exterior calculus} have been applied on the same model. By choosing finite elements spaces and projectors on these spaces satisfying a commuting diagram with the continuous spaces, a semi-discrete system (discrete in space and continuous in time) for the time evolution of all finite element coefficients and particle configurations has been derived. By proofing the skew-symmetry and the Jacobi identity of the Poisson matrix, it has been shown that the semi-discrete system exhibits a noncanonical Hamiltonian structure. The subsequent construction of Poisson time integrators by splitting the Hamiltonian and analytically solving the resulting subsystems has led to a uniformly bounded error in the conservation of energy for the first presented numerical experiment in the linear and nonlinear stage which is was not the case for standard PIC. Finally, the second numerical experiment revealed that standard PIC leads to spurious modes for large wavenumbers (compared to the inverse of the element size) which is not the case for structure-preserving geometric PIC.
 


\newpage
\appendix

\section{Poisson matrix}
\label{sec_appendix1}
\noindent The matrix $\mathbb{J}$ in (\ref{eq_Hamiltonian_structure}) reads
\renewcommand{\arraystretch}{2.0}
\begin{align}
\hspace{3cm}
\rotatebox{90}{$
\left(\begin{array}{c|c|c|c|c|c|c|c|c|c}
 & & &\frac{1}{\epsilon_0}\mathbb{M}_0^{-1}\mathbb{G}^\top & -\Omega_\mathrm{pe}^2\mathbb{M}_0^{-1} & & &-\frac{q_\text{e}}{\epsilon_0m_\text{e}}\mathbb{M}_0^{-1}\mathbb{Q}^0  & &
\\ 
\hline & & -\frac{1}{\epsilon_0}\mathbb{M}_0^{-1}\mathbb{G}^\top & & & -\Omega_\mathrm{pe}^2\mathbb{M}_0^{-1} & & &-\frac{q_\text{e}}{\epsilon_0m_\text{e}}\mathbb{M}_0^{-1}\mathbb{Q}^0 &
 \\
\hline & \frac{1}{\epsilon_0}\mathbb{G}\mathbb{M}_0^{-1} & & & & & & & &
 \\
\hline -\frac{1}{\epsilon_0}\mathbb{G}\mathbb{M}_0^{-1} & & & & & & & & &
 \\
\hline  \Omega_\mathrm{pe}^2\mathbb{M}_0^{-1} & & & & & \epsilon_0\Omega_\mathrm{pe}^2\Omega_\mathrm{ce}\mathbb{M}_0^{-1} & & & &
 \\
\hline & \Omega_\mathrm{pe}^2\mathbb{M}_0^{-1} & & & -\epsilon_0\Omega_\mathrm{pe}^2\Omega_\mathrm{ce}\mathbb{M}_0^{-1} & & & & &
 \\
\hline & & & & & & & & &\mathbb{W}^{-1}
 \\
\hline \frac{q_\text{e}}{\epsilon_0m_\text{e}}(\mathbb{Q}^0)^\top\mathbb{M}_0^{-1} & & & & & & & &\Omega_\mathrm{ce}\mathbb{W}^{-1} &-\frac{q_\text{e}}{m_\text{e}}\mathbb{B}_y\mathbb{W}^{-1}
 \\
\hline &\frac{q_\text{e}}{\epsilon_0m_\text{e}}(\mathbb{Q}^0)^\top\mathbb{M}_0^{-1}  & & & & & & -\Omega_\mathrm{ce}\mathbb{W}^{-1} & &\frac{q_\text{e}}{m_\text{e}}\mathbb{B}_x\mathbb{W}^{-1}
 \\
\hline & & & & & &-\mathbb{W}^{-1} &\frac{q_\text{e}}{m_\text{e}}\mathbb{B}_y\mathbb{W}^{-1} &-\frac{q_\text{e}}{m_\text{e}}\mathbb{B}_x\mathbb{W}^{-1} &
 \end{array}\right)$
 }\label{eq_Poisson_matrix}
\end{align}



\section{Jacobi identity}
\label{sec_appendix2}
\begin{table}[ht!]
\centering
\caption{Block index triples for which the Jacobi identity needs to be proven.\label{tab_Jacobi2}}
\begin{tabular}{|c|c|c|c|}
\hline
(i,j,k) &terms &block matrix term & explicit expression \\[1mm]
\hline
(1,8,10) &\uproman{5}+\uproman{7} & &\\[1mm]
(8,10,1) &\uproman{4}+\uproman{9} &\large$\block{8}{10}{\mb{b}_y}{4}{1}+\block{1}{8}{\mb{Z}}{7}{10}$ &\large$\frac{q_\mr{e}}{m_\mr{e}\epsilon_0}\left(\frac{\pa(\mathbb{B}_y\mathbb{W}^{-1})}{\pa\mb{b}_y}\mathbb{G}\mathbb{M}_0^{-1}-\frac{\pa(\mathbb{M}_0^{-1}\mathbb{Q}_0)}{\pa \mb{Z}}\mathbb{W}^{-1}\right)$ \\[1mm]
(10,1,8) &\uproman{6}+\uproman{8} & &\\
\hline

(1,10,8) &\uproman{5}+\uproman{9} & &\\[1mm]
(10,8,1) &\uproman{4}+\uproman{8} &\large$\block{10}{8}{\mb{b}_y}{4}{1}+\block{8}{1}{\mb{Z}}{7}{10}$ &\large$-\frac{q_\mr{e}}{m_\mr{e}\epsilon_0}\left(\frac{\pa(\mathbb{B}_y\mathbb{W}^{-1})}{\pa\mb{b}_y}\mathbb{G}\mathbb{M}_0^{-1}-\frac{\pa(\mathbb{M}_0^{-1}\mathbb{Q}_0)}{\pa \mb{Z}}\mathbb{W}^{-1}\right)$ \\[1mm]
(8,1,10) &\uproman{6}+\uproman{7} & &\\
\hline


(2,9,10) &\uproman{2}+\uproman{7} & & \\[1mm]
(9,10,2) &\uproman{1}+\uproman{9} &\large$\block{9}{10}{\mb{b}_x}{3}{2}+\block{2}{9}{\mb{Z}}{7}{10}$ &\large$\frac{q_\mr{e}}{m_\mr{e}\epsilon_0}\left(\frac{\pa(\mathbb{B}_x\mathbb{W}^{-1})}{\pa\mb{b}_x}\mathbb{G}\mathbb{M}_0^{-1}-\frac{\pa(\mathbb{M}_0^{-1}\mathbb{Q}_0)}{\pa \mb{Z}}\mathbb{W}^{-1}\right)$ \\[1mm]
(10,2,9) &\uproman{3}+\uproman{8} & &\\
\hline


(2,10,9) &\uproman{2}+\uproman{9} & &\\[1mm]
(10,9,2) &\uproman{1}+\uproman{8} &\large$\block{10}{9}{\mb{b}_x}{3}{2}+\block{9}{2}{\mb{Z}}{7}{10}$ &\large$-\frac{q_\mr{e}}{m_\mr{e}\epsilon_0}\left(\frac{\pa(\mathbb{B}_x\mathbb{W}^{-1})}{\pa\mb{b}_x}\mathbb{G}\mathbb{M}_0^{-1}-\frac{\pa(\mathbb{M}_0^{-1}\mathbb{Q}_0)}{\pa \mb{Z}}\mathbb{W}^{-1}\right)$ \\[1mm]
(9,2,10) &\uproman{3}+\uproman{7} & &\\
\hline


(8,10,10) &\uproman{7}+\uproman{9} & &\\[1mm]
(10,10,8) &\uproman{8}+\uproman{9} &\large$\block{8}{10}{\mb{Z}}{7}{10}+\block{10}{8}{\mb{Z}}{7}{10}$ &\large$-\frac{q_\mr{e}}{m_\mr{e}}\frac{\pa(\mathbb{B}_y\mathbb{W}^{-1})}{\pa\mb{Z}}\mathbb{W}^{-1}+\frac{q_\mr{e}}{m_\mr{e}}\frac{\pa(\mathbb{B}_y\mathbb{W}^{-1})}{\pa\mb{Z}}\mathbb{W}^{-1}=0$ \\[1mm]
(10,8,10) &\uproman{7}+\uproman{8} & &\\
\hline



(9,10,10) &\uproman{7}+\uproman{9} & &\\[1mm]
(10,10,9) &\uproman{8}+\uproman{9} &\large$\block{9}{10}{\mb{Z}}{7}{10}+\block{10}{9}{\mb{Z}}{7}{10}$ &\large$\frac{q_\mr{e}}{m_\mr{e}}\frac{\pa(\mathbb{B}_x\mathbb{W}^{-1})}{\pa\mb{Z}}\mathbb{W}^{-1}-\frac{q_\mr{e}}{m_\mr{e}}\frac{\pa(\mathbb{B}_x\mathbb{W}^{-1})}{\pa\mb{Z}}\mathbb{W}^{-1}=0$ \\[1mm]
(10,9,10) &\uproman{7}+\uproman{8} & &\\

\hline
\end{tabular}
\end{table}

\newpage
\section{Time integrators for Hamiltonian splitting}
\label{sec_appendix3}
\noindent \textbf{Problem 1}. For $t\in[0,\Delta t]$ and $\textbf{u}(t=0)=\textbf{u}^0$ we have
\begin{align}
\frac{\mathrm{d} \textbf{u}}{\mathrm{d}t}=\mathbb{J}(\textbf{u})\nabla_{\textbf{u}} H_E(\textbf{u})=\mathbb{J}(\mb{u})\nabla_{\textbf{u}}\left[\frac{\epsilon_0}{2}(\textbf{e}_x^\top\mathbb{M}_0\textbf{e}_x+\textbf{e}_y^\top\mathbb{M}_0\textbf{e}_y)\right].
\end{align}
This can be solved analytically as
\begin{subequations}
\begin{alignat}{3}
    &\frac{\mathrm{d} \textbf{e}_x}{\mathrm{d}t}=0 &&\Longrightarrow\quad \textbf{e}_x(\Delta t) = \textbf{e}_x^0,\\
    &\frac{\mathrm{d} \textbf{e}_y}{\mathrm{d}t}=0 &&\Longrightarrow\quad \textbf{e}_y(\Delta t) = \textbf{e}_y^0,\\
    &\frac{\mathrm{d} \textbf{b}_x}{\mathrm{d}t}=\frac{1}{\epsilon_0}\mathbb{G}\mathbb{M}_0^{-1}\epsilon_0\mathbb{M}_0\textbf{e}_y &&\Longrightarrow\quad \textbf{b}_x(\Delta t) = \textbf{b}_x^0 + \Delta t\mathbb{G}\textbf{e}_y^0,\\
    &\frac{\mathrm{d} \textbf{b}_y}{\mathrm{d}t}=-\frac{1}{\epsilon_0}\mathbb{G}\mathbb{M}_0^{-1}\epsilon_0\mathbb{M}_0\textbf{e}_x &&\Longrightarrow\quad \textbf{b}_y(\Delta t) = \textbf{b}_y^0 - \Delta t\mathbb{G}\textbf{e}_x^0,\\
     &\frac{\mathrm{d} \textbf{y}_x}{\mathrm{d}t}=\Omega_\mathrm{pe}^2\mathbb{M}_0^{-1}\epsilon_0\mathbb{M}_0\textbf{e}_x &&\Longrightarrow\quad \textbf{y}_x(\Delta t) = \textbf{y}_x^0 + \Delta t\epsilon_0\Omega_\mathrm{pe}^2\textbf{e}_x^0,\\
     &\frac{\mathrm{d} \textbf{y}_y}{\mathrm{d}t}=\Omega_\mathrm{pe}^2\mathbb{M}_0^{-1}\epsilon_0\mathbb{M}_0\textbf{e}_y &&\Longrightarrow\quad \textbf{y}_y(\Delta t) = \textbf{y}_y^0 + \Delta t\epsilon_0\Omega_\mathrm{pe}^2\textbf{e}_y^0,\\
     &\frac{\mathrm{d} \textbf{Z}}{\mathrm{d}t}=0 &&\Longrightarrow\quad \textbf{Z}(\Delta t) = \textbf{Z}^0,\\
     &\frac{\mathrm{d} \textbf{V}_x}{\mathrm{d}t}=\frac{q_\text{e}}{\epsilon_0m_\text{e}}(\mathbb{Q}^0)^\top\mathbb{M}_0^{-1}\epsilon_0\mathbb{M}_0\textbf{e}_x\quad &&\Longrightarrow\quad \textbf{V}_x(\Delta t) = \textbf{V}_x^0 + \Delta t\frac{q_\text{e}}{m_\text{e}}(\mathbb{Q}^0)^\top(\textbf{Z}^0)\textbf{e}_x^0,\\ 
     &\frac{\mathrm{d} \textbf{V}_y}{\mathrm{d}t}=\frac{q_\text{e}}{\epsilon_0m_\text{e}}(\mathbb{Q}^0)^\top\mathbb{M}_0^{-1}\epsilon_0\mathbb{M}_0\textbf{e}_y &&\Longrightarrow\quad \textbf{V}_y(\Delta t) = \textbf{V}_x^0 + \Delta t\frac{q_\text{e}}{m_\text{e}}(\mathbb{Q}^0)^\top(\textbf{Z}^0)\textbf{e}_y^0,\\ 
     &\frac{\mathrm{d} \textbf{V}_z}{\mathrm{d}t}=0 &&\Longrightarrow\quad \textbf{V}_z(\Delta t) = \textbf{V}_z^0.
\end{alignat}
\end{subequations}
The corresponding integrator is denoted by $\textbf{u}(\Delta t)=\Phi_{\Delta t}^E(\textbf{u}^0)$.\\ \\
\textbf{Problem 2}. For $t\in[0,\Delta t]$ and $\textbf{u}(t=0)=\textbf{u}^0$ we have
\begin{align}
\frac{\mathrm{d} \textbf{u}}{\mathrm{d}t}=\mathbb{J}(\textbf{u})\nabla_{\textbf{u}} H_B(\textbf{u})=\mathbb{J}(\mb{u})\nabla_{\textbf{u}}\left[\frac{1}{2\mu_0}(\textbf{b}_x^\top\mathbb{M}_1\textbf{b}_x+\textbf{b}_y^\top\mathbb{M}_1\textbf{b}_y)\right].
\end{align}
This can be solved analytically as
\begin{subequations}
\begin{alignat}{3}
    &\frac{\mathrm{d} \textbf{e}_x}{\mathrm{d}t}=\frac{1}{\epsilon_0}\mathbb{M}_0^{-1}\mathbb{G}^\top\frac{1}{\mu_0}\mathbb{M}_1\textbf{b}_y &&\Longrightarrow\quad \textbf{e}_x(\Delta t) = \textbf{e}_x^0 + \Delta tc^2\mathbb{M}_0^{-1}\mathbb{G}^\top\mathbb{M}_1\textbf{b}_y^0,\\
    &\frac{\mathrm{d} \textbf{e}_y}{\mathrm{d}t}=-\frac{1}{\epsilon_0}\mathbb{M}_0^{-1}\mathbb{G}^\top\frac{1}{\mu_0}\mathbb{M}_1\textbf{b}_x \quad &&\Longrightarrow\quad \textbf{e}_y(\Delta t) = \textbf{e}_y^0 - \Delta tc^2\mathbb{M}_0^{-1}\mathbb{G}^\top\mathbb{M}_1\textbf{b}_x^0,\\
    &\frac{\mathrm{d} \textbf{b}_x}{\mathrm{d}t}=0 &&\Longrightarrow\quad \textbf{b}_x(\Delta t) = \textbf{b}_x^0,\\
    &\frac{\mathrm{d} \textbf{b}_y}{\mathrm{d}t}=0 &&\Longrightarrow\quad \textbf{b}_y(\Delta t) = \textbf{b}_y^0,\\
     &\frac{\mathrm{d} \textbf{y}_x}{\mathrm{d}t}=0 &&\Longrightarrow\quad \textbf{y}_x(\Delta t) = \textbf{y}_x^0,\\
     &\frac{\mathrm{d} \textbf{y}_y}{\mathrm{d}t}=0 &&\Longrightarrow\quad \textbf{y}_y(\Delta t) = \textbf{y}_y^0,
\end{alignat}
\begin{alignat}{3}
     &\frac{\mathrm{d} \textbf{Z}}{\mathrm{d}t}=0\hspace{3cm} &&\Longrightarrow\quad \textbf{Z}(\Delta t) = \textbf{Z}^0,\\
     &\frac{\mathrm{d} \textbf{V}_x}{\mathrm{d}t}=0 &&\Longrightarrow\quad \textbf{V}_x(\Delta t) = \textbf{V}_x^0,\\
     &\frac{\mathrm{d} \textbf{V}_y}{\mathrm{d}t}=0 &&\Longrightarrow\quad \textbf{V}_y(\Delta t) = \textbf{V}_y^0,\\
     &\frac{\mathrm{d} \textbf{V}_z}{\mathrm{d}t}=0 &&\Longrightarrow\quad \textbf{V}_z(\Delta t) = \textbf{V}_z^0.
\end{alignat}
\end{subequations}
The corresponding integrator is denoted by $\textbf{u}(\Delta t)=\Phi_{\Delta t}^B(\textbf{u}^0)$.\\ \\
\textbf{Problem 3}. For $t\in[0,\Delta t]$ and $\textbf{u}(t=0)=\textbf{u}^0$, we have
\begin{align}
\frac{\mathrm{d} \textbf{u}}{\mathrm{d}t}=\mathbb{J}(\textbf{u})\nabla_{\textbf{u}} H_Y(\textbf{u})=\mathbb{J}(\mb{u})\nabla_{\textbf{u}}\left[\frac{1}{2\epsilon_0\Omega_\mr{pe}^2}(\textbf{y}_x^\top\mathbb{M}_0\textbf{y}_x+\textbf{y}_y^\top\mathbb{M}_0\textbf{y}_y)\right].
\end{align}
This can be solved analytically as
\begin{subequations}
\begin{alignat}{3}
\begin{split}
    &\frac{\mathrm{d} \textbf{e}_x}{\mathrm{d}t}=-\Omega_\mathrm{pe}^2\mathbb{M}_0^{-1}\frac{1}{\epsilon_0\Omega_\mathrm{pe}^2}\mathbb{M}_0\textbf{y}_x \\
    &\Longrightarrow \textbf{e}_x(\Delta t)=\textbf{e}_x^0 - \frac{1}{\epsilon_0}\int_{0}^{\Delta t} y_x(t^\prime)\mathrm{d}t^\prime =\textbf{e}_x^0 - \frac{1}{\epsilon_0\Omega_\mathrm{ce}}[\textbf{y}_x^0\sin(\Omega_\mathrm{ce}t)-\textbf{y}_y^0\cos(\Omega_\mathrm{ce}t)+\textbf{y}_y^0],
\end{split}\\
\begin{split}
    &\frac{\mathrm{d} \textbf{e}_y}{\mathrm{d}t}=-\Omega_\mathrm{pe}^2\mathbb{M}_0^{-1}\frac{1}{\epsilon_0\Omega_\mathrm{pe}^2}\mathbb{M}_0\textbf{y}_y \\
    &\Longrightarrow \textbf{e}_y(\Delta t)= \textbf{e}_y^0 - \frac{1}{\epsilon_0}\int_{0}^{\Delta t} y_y(t^\prime)\mathrm{d}t^\prime =\textbf{e}_y^0 - \frac{1}{\epsilon_0\Omega_\mathrm{ce}}[\textbf{y}_y^0\sin(\Omega_\mathrm{ce}t)+\textbf{y}_x^0\cos(\Omega_\mathrm{ce}t)-\textbf{y}_x^0],
\end{split}
\end{alignat}
\begin{alignat}{3}
    &\frac{\mathrm{d} \textbf{b}_x}{\mathrm{d}t}=0 &&\Longrightarrow\quad \textbf{b}_x(\Delta t) = \textbf{b}_x^0,\\
    &\frac{\mathrm{d} \textbf{b}_y}{\mathrm{d}t}=0 &&\Longrightarrow\quad \textbf{b}_y(\Delta t) = \textbf{b}_y^0,\\
     &\frac{\mathrm{d} \textbf{y}_x}{\mathrm{d}t}=\epsilon_0\Omega_\mathrm{pe}^2\Omega_\mathrm{ce}\mathbb{M}_0^{-1}\frac{1}{\epsilon_0\Omega_\mathrm{pe}^2}\mathbb{M}_0\textbf{y}_y &&\Longrightarrow\quad \textbf{y}_x(\Delta t) = \textbf{y}_x^0\cos(\Omega_\mathrm{ce}\Delta t)+\textbf{y}_y^0\sin(\Omega_\mathrm{ce}\Delta t),\\
     &\frac{\mathrm{d} \textbf{y}_y}{\mathrm{d}t}=-\epsilon_0\Omega_\mathrm{pe}^2\Omega_\mathrm{ce}\mathbb{M}_0^{-1}\frac{1}{\epsilon_0\Omega_\mathrm{pe}^2}\mathbb{M}_0\textbf{y}_x \quad &&\Longrightarrow\quad\textbf{y}_y(\Delta t) = \textbf{y}_y^0\cos(\Omega_\mathrm{ce}\Delta t)-\textbf{y}_x^0\sin(\Omega_\mathrm{ce}\Delta t),\\
     &\frac{\mathrm{d} \textbf{Z}}{\mathrm{d}t}=0 &&\Longrightarrow\quad \textbf{Z}(\Delta t) = \textbf{Z}^0,\\
     &\frac{\mathrm{d} \textbf{V}_x}{\mathrm{d}t}=0 &&\Longrightarrow\quad\textbf{V}_x(\Delta t) = \textbf{V}_x^0,\\
     &\frac{\mathrm{d} \textbf{V}_y}{\mathrm{d}t}=0 &&\Longrightarrow\quad \textbf{V}_y(\Delta t) = \textbf{V}_y^0,\\
     &\frac{\mathrm{d} \textbf{V}_z}{\mathrm{d}t}=0 &&\Longrightarrow\quad\textbf{V}_z(\Delta t) = \textbf{V}_z^0.
\end{alignat}
\end{subequations}
The corresponding integrator is denoted by $\textbf{u}(\Delta t)=\Phi_{\Delta t}^Y(\textbf{u}^0)$.\\ \\
\textbf{Problem 4.} For $t\in[0,\Delta t]$ and $\textbf{u}(t=0)=\textbf{u}^0$, we have
\begin{align}
\frac{\mathrm{d} \textbf{u}}{\mathrm{d}t}=\mathbb{J}(\textbf{u})\nabla_{\textbf{u}} H_x(\textbf{u})=\mathbb{J}(\mb{u})\nabla_{\textbf{u}}\left(\frac{m_\mr{e}}{2}\textbf{V}_x^\top\mathbb{W}\textbf{V}_x\right).
\end{align}
This can be solved analytically as
\begin{subequations}
\begin{alignat}{3}
    &\frac{\mathrm{d} \textbf{e}_x}{\mathrm{d}t}=-\frac{q_\text{e}}{\epsilon_0m_\text{e}}\mathbb{M}_0^{-1}\mathbb{Q}^0m_\text{e}\mathbb{W}\textbf{V}_x\quad &&\Longrightarrow\quad\textbf{e}_x(\Delta t) =\textbf{e}_y^0 - \Delta t\frac{q_\text{e}}{\epsilon_0}\mathbb{M}_0^{-1}\mathbb{Q}^0(\textbf{Z}^0)\mathbb{W}\textbf{V}_x^0,\\
    &\frac{\mathrm{d} \textbf{e}_y}{\mathrm{d}t}=0 &&\Longrightarrow\quad \textbf{e}_y(\Delta t) = \textbf{e}_y^0,\\
    &\frac{\mathrm{d} \textbf{b}_x}{\mathrm{d}t}=0 &&\Longrightarrow\quad \textbf{b}_x(\Delta t) = \textbf{b}_x^0,\\
    &\frac{\mathrm{d} \textbf{b}_y}{\mathrm{d}t}=0 &&\Longrightarrow\quad \textbf{b}_y(\Delta t) = \textbf{b}_y^0,\\
     &\frac{\mathrm{d} \textbf{y}_x}{\mathrm{d}t}=0 &&\Longrightarrow\quad \textbf{y}_x(\Delta t) = \textbf{y}_x^0,\\
     &\frac{\mathrm{d} \textbf{y}_y}{\mathrm{d}t}=0 &&\Longrightarrow\quad \textbf{y}_y(\Delta t) = \textbf{y}_y^0,\\
     &\frac{\mathrm{d} \textbf{Z}}{\mathrm{d}t}=0 &&\Longrightarrow\quad \textbf{Z}(\Delta t) = \textbf{Z}^0,\\
     &\frac{\mathrm{d} \textbf{V}_x}{\mathrm{d}t}=0 &&\Longrightarrow\quad \textbf{V}_x(\Delta t) = \textbf{V}_x^0,\\
     &\frac{\mathrm{d} \textbf{V}_y}{\mathrm{d}t}=-\frac{\Omega_\mathrm{ce}}{m}\mathbb{W}^{-1}m_\text{e}\mathbb{W}\textbf{V}_x &&\Longrightarrow\quad \textbf{V}_y(\Delta t) = \textbf{V}_y^0-\Delta t\Omega_\text{ce}\textbf{V}_x^0,\\
     &\frac{\mathrm{d} \textbf{V}_z}{\mathrm{d}t}=\frac{q_\text{e}}{m_\text{e}^2}\mathbb{B}_y\mathbb{W}^{-1}m_\text{e}\mathbb{W}\textbf{V}_x \quad &&\Longrightarrow\quad \textbf{V}_z(\Delta t) = \textbf{V}_z^0+\Delta t\frac{q_\text{e}}{m_\text{e}}\mathbb{B}_y(\textbf{Z}^0,\textbf{b}_y^0)\textbf{V}_x^0.
\end{alignat}
\end{subequations}
The corresponding integrator is denoted by $\textbf{u}(\Delta t)=\Phi_{\Delta t}^y(\textbf{u}^0)$.\\ \\
\textbf{Problem 5.} For $t\in[0,\Delta t]$ and $\textbf{u}(t=0)=\textbf{u}^0$, we have
\begin{align}
\frac{\mathrm{d}\textbf{u}}{\mathrm{d}t}=\mathbb{J}(\textbf{u})\nabla_{\textbf{u}} H_y(\textbf{u})=\mathbb{J}(\mb{u})\nabla_{\textbf{u}}\left(\frac{m_\mr{e}}{2}\textbf{V}_y^\top\mathbb{W}\textbf{V}_y\right).
\end{align}
This can be solved analytically as
\begin{subequations}
\begin{alignat}{3}
    &\frac{\mathrm{d} \textbf{e}_x}{\mathrm{d}t}=0 &&\Longrightarrow\quad \textbf{e}_x(\Delta t) = \textbf{e}_x^0,\\
    &\frac{\mathrm{d} \textbf{e}_y}{\mathrm{d}t}=-\frac{q_\text{e}}{\epsilon_0m_\text{e}}\mathbb{M}_0^{-1}\mathbb{Q}^0m_\text{e}\mathbb{W}\textbf{V}_y\quad &&\Longrightarrow\quad \textbf{e}_y(\Delta t) = \textbf{e}_y^0 - \Delta t\frac{q_\text{e}}{\epsilon_0}\mathbb{M}_0^{-1}\mathbb{Q}^0(\textbf{Z}^0)\mathbb{W}\textbf{V}_y^0,\\
    &\frac{\mathrm{d} \textbf{b}_x}{\mathrm{d}t}=0 &&\Longrightarrow\quad \textbf{b}_x(\Delta t) = \textbf{b}_x^0,\\
    &\frac{\mathrm{d} \textbf{b}_y}{\mathrm{d}t}=0 &&\Longrightarrow\quad \textbf{b}_y(\Delta t) = \textbf{b}_y^0,\\
     &\frac{\mathrm{d} \textbf{y}_x}{\mathrm{d}t}=0 &&\Longrightarrow\quad \textbf{y}_x(\Delta t) = \textbf{y}_x^0,\\
     &\frac{\mathrm{d} \textbf{y}_y}{\mathrm{d}t}=0 &&\Longrightarrow\quad \textbf{y}_y(\Delta t) = \textbf{y}_y^0,\\
     &\frac{\mathrm{d} \textbf{Z}}{\mathrm{d}t}=0 &&\Longrightarrow\quad \textbf{Z}(\Delta t) = \textbf{Z}^0,\\
     &\frac{\mathrm{d} \textbf{V}_x}{\mathrm{d}t}=\frac{\Omega_\mathrm{ce}}{m}\mathbb{W}^{-1}m_\text{e}\mathbb{W}\textbf{V}_y &&\Longrightarrow\quad \textbf{V}_x(\Delta t) = \textbf{V}_x^0+\Delta t\Omega_\mathrm{ce}\textbf{V}_y^0,\\
     &\frac{\mathrm{d} \textbf{V}_y}{\mathrm{d}t}=0 &&\Longrightarrow\quad \textbf{V}_y(\Delta t) = \textbf{V}_y^0,\\
     &\frac{\mathrm{d} \textbf{V}_z}{\mathrm{d}t}=-\frac{q_\text{e}}{m_\text{e}^2}\mathbb{B}_x\mathbb{W}^{-1}m_\text{e}\mathbb{W}\textbf{V}_y &&\Longrightarrow\quad \textbf{V}_z(\Delta t) = \textbf{V}_z^0-\Delta t\frac{q_\text{e}}{m_\text{e}}\mathbb{B}_x(\textbf{Z}^0,\textbf{b}_x^0)\textbf{V}_y^0.
\end{alignat}
\end{subequations}
The corresponding integrator is denoted by $\textbf{u}(\Delta t)=\Phi_{\Delta t}^y(\textbf{u}^0)$.\\ \\
\textbf{Problem 6.} For $t\in[0,\Delta t]$ and $\textbf{u}(t=0)=\textbf{u}^0$, we have
\begin{align}
\frac{\mathrm{d} \textbf{u}}{\mathrm{d}t}=\mathbb{J}(\textbf{u})\nabla_{\textbf{u}} H_z(\textbf{u})=\mathbb{J}(\mb{u})\nabla_{\textbf{u}}\left(\frac{m_\mr{e}}{2}\textbf{V}_z^\top\mathbb{W}\textbf{V}_z\right).
\end{align}
This can be solved analytically as
\begin{subequations}
\begin{alignat}{3}
    &\frac{\mathrm{d} \textbf{e}_x}{\mathrm{d}t}=0\quad &&\Longrightarrow\quad \textbf{e}_x(\Delta t) = \textbf{e}_x^0,\\
    &\frac{\mathrm{d} \textbf{e}_y}{\mathrm{d}t}=0 &&\Longrightarrow\quad \textbf{e}_y(\Delta t) = \textbf{e}_y^0,\\
    &\frac{\mathrm{d} \textbf{b}_x}{\mathrm{d}t}=0 &&\Longrightarrow\quad \textbf{b}_x(\Delta t) = \textbf{b}_x^0,\\
    &\frac{\mathrm{d} \textbf{b}_y}{\mathrm{d}t}=0 &&\Longrightarrow\quad \textbf{b}_y(\Delta t) = \textbf{b}_y^0,\\
     &\frac{\mathrm{d} \textbf{y}_x}{\mathrm{d}t}=0 &&\Longrightarrow\quad \textbf{y}_x(\Delta t) = \textbf{y}_x^0,\\
     &\frac{\mathrm{d} \textbf{y}_y}{\mathrm{d}t}=0 &&\Longrightarrow\quad \textbf{y}_y(\Delta t) = \textbf{y}_y^0,\\
     &\frac{\mathrm{d} \textbf{Z}}{\mathrm{d}t}=\frac{1}{m}\mathbb{W}^{-1}m_\text{e}\mathbb{W}\textbf{V}_z &&\Longrightarrow\quad \textbf{Z}(\Delta t) = \textbf{Z}^0+\Delta t\textbf{V}_z^0,\\
     &\frac{\mathrm{d} \textbf{V}_x}{\mathrm{d}t}=-\frac{q_\text{e}}{m_\text{e}^2}\mathbb{B}_y\mathbb{W}^{-1}m_\text{e}\mathbb{W}\textbf{V}_z \quad &&\Longrightarrow\quad \textbf{V}_x(\Delta t) = \textbf{V}_x^0-\frac{q_\text{e}}{m_\text{e}}\int_{0}^{\Delta t}\mathbb{B}_y(\textbf{Z}(s),\textbf{b}_y^0)\mathrm{d}s\textbf{V}_z^0\label{eq_lineIntegral_1}\\
     &\frac{\mathrm{d} \textbf{V}_y}{\mathrm{d}t}=\frac{q_\text{e}}{m_\text{e}^2}\mathbb{B}_x\mathbb{W}^{-1}m_\text{e}\mathbb{W}\textbf{V}_z &&\Longrightarrow\quad \textbf{V}_y(\Delta t) = \textbf{V}_y^0+\frac{q_\text{e}}{m_\text{e}}\int_{0}^{\Delta t}\mathbb{B}_x(\textbf{Z}(s),\textbf{b}_x^0)\mathrm{d}s\textbf{V}_z^0\label{eq_lineIntegral_2}\\
     &\frac{\mathrm{d} \textbf{V}_z}{\mathrm{d}t}=0 &&\Longrightarrow\quad \textbf{V}_z(\Delta t) = \textbf{V}_z^0.
\end{alignat}
\end{subequations}
The corresponding integrator is denoted by $\textbf{u}(\Delta t)=\Phi_{\Delta t}^z(\textbf{u}^0)$. Note that the integrals can be computed exactly along each particle trajectories as the basis functions are piecewise polynomials.

\section*{Acknowledgments}
To end this article, we would like to thank C. Tronci for stimulating discussions and collaborations.

%%Vancouver style references.
\bibliographystyle{model1-num-names}
\bibliography{refs}
\end{document}

%%
